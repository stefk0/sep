
\subsection{Езикът \texttt{REG++}}

Нека тук с $\alpha,\beta,\gamma$ да означаваме произволни регулярни изрази.
Да въведем означението $\alpha \leq \beta$, ако $\mathcal{L}(\alpha) \subseteq \mathcal{L}(\beta)$.

\begin{problem}
  Докажете, че ако $\alpha \cdot \gamma + \beta \leq \gamma$, то $\alpha^\star \beta \leq \gamma$.
\end{problem}
\begin{hint}
  Докажете, че $\alpha^n \cdot \beta \leq \gamma$ за всяко естествено число $n$.
\end{hint}

Да разгледаме следното разширение на езика \REG, който да наречем $\REG\texttt{++}$.
Термовете на езика \REGPP се формират спрямо следната абстрактна граматика:
\[\tau ::= \alpha\ |\ \alpha \cdot X\ |\ \tau_1 + \tau_2,\]
където с $\alpha$ означаваме произволен регулярен израз, т.е.
\[\alpha ::= \emptyset\ |\ \varepsilon\ |\ a\ |\ \alpha_1 + \alpha_2\ |\ \alpha_1 \cdot \alpha_2\ |\ \alpha^\star_1\]
\marginpar{Новото е, че тук позволяваме да имаме термове от вида $\alpha \cdot X$, за произволен регулярен израз $\alpha$, докато в езика \REG\ позволявахме само $a \cdot X$, за произволна буква $a$.}

\begin{problem}[Правило на Ардън \cite{arden}]\label{prob:reg:arden}
  \index{правило на Ардън}
  Да разгледаме програмата на езика \REGPP:
  \[X = \alpha \cdot X + \beta.\]
  Докажете, че най-малкото решение на системата за тази програма е езикът описан с регулярния израз $\alpha^\star \cdot \beta$. Докажете, че ако $\varepsilon \not \in \mathcal{L}(\alpha)$, то това решение е и едиствено.
\end{problem}

\begin{example}
  Да разгледаме следния краен детерминиран автомат:
  \marginpar{Означаваме с $\abs{\omega}_a$ броя на срещанията на $a$ в думата $\omega$.}
  \begin{figure}[H]
    \centering
    \begin{tikzpicture}[->,>=stealth,thick,node distance=65pt]
      \tikzstyle{every state}=[circle,minimum size=20pt,auto]
      
      \node[initial, state, accepting]         (0) {$q_1$};
      \node[state]                             (1) [right of=0]{$q_2$};
      \node[state]                             (2) [right of=1]{$q_3$};
      
      \path 
      (0) edge [loop above]   node   [above] {$b$}    (0)
      (0) edge [bend left=15] node   [above] {$a$}    (1)
      (1) edge [loop above]   node   [above] {$b$}    (1)
      (1) edge [bend left=15] node   [above] {$a$}    (2)
      (2) edge [loop above]   node   [above] {$b$}    (2)
      (2) edge [bend left=45] node   [below] {$a$}    (0);
    \end{tikzpicture}
    \caption{Автомат разпознаващ $\{\omega \in \{a,b\}^\star \mid \abs{\omega}_{a} \equiv 0 \bmod\ 3\}$}
  \end{figure}
  На този автомат съответства следната програма $R$, която е програма и на езика \REGPP:
  \marginpar{Тук $\Sigma = \{a,b\}$.}
  \begin{align*}
    & X_1 = a \cdot X_2 + b \cdot X_1 + \varepsilon \\
    & X_2 = a \cdot X_3 + b \cdot X_2\\
    & X_3 = a \cdot X_1 + b \cdot X_3.
  \end{align*}

  Да видим как можем да намерим най-малкото решение на тази система като използваме само \hyperref[prob:reg:arden]{правилото на Ардън}.
  Нека първо да разгледаме само последния ред на системата като си мислим, че имаме азбука $\Sigma_3 = \{a,b,X_1,X_2\}$. Тогава получаваме следното:
  \[X_3 = \underbrace{b}_{\alpha} \cdot X_3 + \underbrace{a \cdot X_1}_{\beta}.\]
  Според \hyperref[prob:reg:arden]{правилото на Ардън}, най-малкото решение на тази система е $b^\star a X_1$.
  Заместваме $X_3$ с този израз в горните две уравнения и получаваме следното:
  \begin{align*}
    & X_1 = b \cdot X_1 + a \cdot X_2 + \varepsilon \\
    & X_2 = b \cdot X_2 + a b^\star a \cdot X_1.
  \end{align*}
  Ако позволим да имаме регулярни изрази по ребрата на автомат, то новият автомат би бил този:

  \begin{figure}[H]
    \centering
    \begin{tikzpicture}[->,>=stealth,thick,node distance=65pt]
      \tikzstyle{every state}=[circle,minimum size=20pt,auto]
      
      \node[initial, state, accepting]         (0) {$q_1$};
      \node[state]                             (1) [right of=0]{$q_2$};
      % \node[state]                             (2) [right of=1]{$q_3$};
      
      \path 
      (0) edge [loop above]   node   [above] {$b$}    (0)
      (0) edge [bend left=35] node   [above] {$a$}    (1)
      (1) edge [loop above]   node   [above] {$b$}    (1)
      (1) edge [bend left=35] node   [below] {$ab^\star a$}    (0);
      % (2) edge [loop above]   node   [above] {$b$}    (2)
      % (2) edge [bend left=45] node   [below] {$a$}    (0);
    \end{tikzpicture}
    \caption{Автомат разпознаващ $\{\omega \in \{a,b\}^\star \mid \abs{\omega}_{a} \equiv 0 \bmod\ 3\}$}
  \end{figure}
  
  Сега разглеждаме втория ред като си мислим, че имаме азбуката $\Sigma_2 = \{a,b,X_1,X_3\}$. Тогава получаваме следното:
  \[X_2 = \underbrace{b}_{\alpha} \cdot X_2 + \underbrace{a b^\star a X_1}_{\beta}.\]
  Според \hyperref[prob:reg:arden]{правилото на Ардън}, най-малкото решение на тази система е $b^\star a b^\star a X_1$.
  Заместваме $X_2$ с израза $b^\star a b^\star a X_1$ в първия ред и получаваме:
  \[X_1 = b \cdot X_1 + ab^\star a b^\star a \cdot X_1  + \varepsilon.\]
  Оттук имаме, че:
  \[X_1 = \underbrace{(b + ab^\star a b^\star a)}_{\alpha} \cdot X_1 + \underbrace{\varepsilon}_{\beta}.\]
  Така получаваме, че най-малкото решение на първия ред е $(b + ab^\star a b^\star a)^\star$.
  Заключаваме, че най-малкото решение на системата е тройката от езици:
  \marginpar{Тук един от проблемите е, че изобщо не е ясно дали тези регулярни изрази са ,,оптимални'' в някакъв смисъл.}
  \begin{align*}
    \hat{L}_1 & = (b + ab^\star a b^\star a)^\star\\
    \hat{L}_2 & = b^\star a b^\star a (b + ab^\star a b^\star a)^\star\\
    \hat{L}_3 & = b^\star a (b + ab^\star a b^\star a)^\star.
  \end{align*}
  
  Да напомним, че вече намерихме следното представяне на най-малкото решение на системата:
  \begin{align*}
    \hat{L}_1 & = \{\omega \in \{a,b\}^\star \mid \abs{\omega}_{a} \equiv 0 \bmod\ 3\}\\
    \hat{L}_2 & = \{\omega \in \{a,b\}^\star \mid \abs{\omega}_{a} \equiv 2 \bmod\ 3\}\\
    \hat{L}_3 & = \{\omega \in \{a,b\}^\star \mid \abs{\omega}_{a} \equiv 1 \bmod\ 3\}.
  \end{align*}
\end{example}
