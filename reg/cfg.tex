\subsection{Езикът \CFG}

Термовете на езика \CFG се описват със следната абстрактна граматика:
\[\tau ::= \varepsilon\ |\ a\ |\ X\ |\ \tau_1 \cdot \tau_2\ |\ \tau_1 + \tau_2.\]
Нека оттук нататък да приемем, че сме фиксирали една програма на езика \CFG:
\begin{align*}
  & X_1 = \tau_1[X_1,\dots,X_n]\\
  & \ \vdots\\
  & X_n = \tau_n[X_1,\dots,X_n].
\end{align*}


\subsubsection*{Операционна семантика}

С индукция по $\ell$, дефинираме релацията $\tau \Downarrow^\ell \alpha$,
където $\tau$ е терм на езика \CFG със свободни променливи измежду $X_1,\dots, X_n$.

\begin{framed}
  \begin{figure}[H]
    \begin{subfigure}[b]{0.5\textwidth}
      \begin{prooftree}
        \AxiomC{}
        \LeftLabel{\scriptsize{(eps)}}
        \UnaryInfC{$\varepsilon \Downarrow^0 \varepsilon$}
      \end{prooftree}
    \end{subfigure}
    ~
    \begin{subfigure}[b]{0.5\textwidth}
      \begin{prooftree}
        \AxiomC{}
        \RightLabel{\scriptsize{(letter)}}
        \UnaryInfC{$a \Downarrow^1 a$}
      \end{prooftree}
    \end{subfigure}
    \begin{subfigure}[b]{0.5\textwidth}
      \begin{prooftree}
        \AxiomC{$\tau_1 \Downarrow^\ell \alpha$}
        \LeftLabel{\scriptsize{(left-or)}}
        \UnaryInfC{$\tau_1 + \tau_2 \Downarrow^{\ell+1} \alpha$}
      \end{prooftree}
    \end{subfigure}
    ~
    \begin{subfigure}[b]{0.5\textwidth}
      \begin{prooftree}
        \AxiomC{$\tau_2 \Downarrow^\ell \alpha$}
        \RightLabel{\scriptsize{(right-or)}}
        \UnaryInfC{$\tau_1 + \tau_2 \Downarrow^{\ell+1} \alpha$}
      \end{prooftree}
    \end{subfigure}

    \begin{subfigure}[b]{0.5\textwidth}
      \begin{prooftree}
        \AxiomC{$\tau_i \Downarrow^\ell \alpha$}
        \LeftLabel{\scriptsize{(rec)}}
        \UnaryInfC{$X_i \Downarrow^{\ell+1} \alpha$}
      \end{prooftree}
    \end{subfigure}
    ~
    \begin{subfigure}[b]{0.5\textwidth}
      \begin{prooftree}
        \AxiomC{$\tau_1 \Downarrow^\ell_1 \alpha_1$}
        \AxiomC{$\tau_2 \Downarrow^\ell_2 \alpha_2$}
        \RightLabel{\scriptsize{(concat)}}
        \BinaryInfC{$\tau_1 \cdot \tau_2 \Downarrow^{\ell_1+\ell_2+1} \alpha_1\cdot\alpha_2$}
      \end{prooftree}
    \end{subfigure}
    \caption{Правила на операционната семантика на езика \CFG.}
  \end{figure}
\end{framed}
Ще пишем $\tau \Downarrow \alpha$ точно тогава, когато съществува $\ell$, за което $\tau \Downarrow^\ell \alpha$.


\subsubsection*{Денотационна семантика}

За всеки терм $\tau[X_1,\dots,X_n]$ на езика \CFG дефинираме изображението
% \marginpar{Тук за променливите използваме главни букви за да се сещаме, че те приемат стойности множества от думи.}
\[\val{\tau}: \mathcal{P}(\Sigma^\star)^n \to \mathcal{P}(\Sigma^\star)\]
 по следния начин:
\begin{itemize}
\item 
  $\val{\varepsilon}(L_1,\dots,L_n) = \{\varepsilon\}$.
\item 
  $\val{a}(L_1,\dots,L_n) = \{a\}$.
\item 
  $\val{X_j}(L_1,\dots,L_n) = L_j$.
\item
  $\val{\tau_1 + \tau_2}(L_1,\dots,L_n) = \val{\tau_1}(L_1,\dots,L_n) \cup \val{\tau_2}(L_1,\dots,L_n)$.
\item
  $\val{\tau_1 \cdot \tau_2}(L_1,\dots,L_n) = \val{\tau_1}(L_1,\dots,L_n) \cdot \val{\tau_2}(L_1,\dots,L_n)$.
\end{itemize}

\begin{problem}
  Докажете, че за всеки терм $\tau$ на езика \REG имаме, че $\val{\tau}$ е непрекъснато изображение в областта на Скот
  $\mathcal{S} = ( \mathcal{P}(\Sigma^\star),\subseteq, \emptyset)$.
\end{problem}

\marginpar{Тук \hyperref[prob:reg:arden]{правилото на Ардън} не работи! Защо?}

\begin{example}
  Да разгледаме следната програма на езика \CFG.
  \begin{align*}
    X & = \overbrace{a\cdot X \cdot c + Y}^{\tau_1[X,Y]}\\
    Y & = \underbrace{b \cdot Y \cdot c + \varepsilon}_{\tau_2[X,Y]}.
  \end{align*}
  На тази програма съответства системата от непрекъснати изображения:
  \begin{align*}
    L & = \val{\tau_1}(L,M) = \{a\}\cdot L_1 \cdot \{c\} \cup L_2\\
    M & = \val{\tau_2}(L_1,L_2) = \{b\} \cdot L_2 \cdot \{c\} \cup \{\varepsilon\}.
  \end{align*}
  Да намерим най-малкото решение $(\hat{L},\hat{M})$ на тази система.
  \begin{align*}
    L_0 & = \emptyset\\
    M_0 & = \emptyset\\
    L_1 & = \val{\tau_1}(L_0,M_0)\\
        & = \{a\}\cdot \emptyset \cdot \{c\} \cup \emptyset = \emptyset\\
    M_1 & = \val{\tau_2}(L_0,M_0)\\
        & = \{b\} \cdot \emptyset \cdot \{c\} \cup \{\varepsilon\} = \{\varepsilon\}\\
    L_2 & = \{a\}\cdot L_1 \cdot \{c\} \cup M_1 = \{\varepsilon\}\\
    M_2 & = \{b\} \cdot M_1 \cdot \{c\} \cup M_1\\
        & = \{b\} \cdot \{\varepsilon\} \cdot \{c\} \cup \{\varepsilon\} = \{bc,\varepsilon\}\\
    L_3 & = \{a\}\cdot L_2 \cdot \{c\} \cup M_2 \\
        & = \{ac,bc,\varepsilon\}\\
        & = \{a^mb^nc^k \mid 2 > m+n = k\}\\
    M_3 & = \{b\} \cdot M_2 \cdot \{c\} \cup M_2\\
        & = \{b\} \cdot \{bc,\varepsilon\} \cdot \{c\} \cup \{bc,\varepsilon\}\\
        & = \{b^nc^n \mid 3 > n\}.
  \end{align*}

  Сега вече сме готови да формулираме нашето индукционно предположение:
  \begin{align*}
    L_t & = \{a^mb^nc^k \mid t-1 > m+n = k\}\\
    M_t & = \{b^nc^n \mid t > n \}.
  \end{align*}

  Докажете, че $\hat{L} = \{a^mb^nc^k \mid m+n = k\}$ и $\hat{M} = \{b^nc^n \mid n \in \Nat\}$.
  
\end{example}


\begin{problem}[Теорема за характеризация]
  Да разгледаме системата от непрекъснати оператори
  \begin{align*}
    & \val{\tau_1}(L_1,\dots,L_n) = L_1\\
    & \ \ \vdots\\
    & \val{\tau_n}(L_1,\dots,L_n) = L_n.
  \end{align*}
  Знаем, че тя притежава най-малко решение $(\hat{L}_1,\dots,\hat{L}_n)$.
  Докажете, че всеки от езиците $\hat{L}_i$ е безконтекстен.

  Докажете, че всеки безконтекстен език е елемент от най-малкото решение 
  на някоя система от непрекъснати изображения от горния вид.
\end{problem}


\subsubsection*{Еквивалентност}

\begin{problem}
  Докажете, че за всеки индекс $r$ и всеки терм $\tau$ на езика \CFG с променливи измежду $X_1,\dots,X_n$ е изпълнено, че:
  \[\val{\tau}(L^r_1,\dots,L^r_n) \subseteq \{\alpha \in \Sigma^\star \mid \tau \Downarrow \alpha\}.\]
  Заключете, че ако $L_1,\dots,L_n$ е най-малкото решение на непрекъснатото изображение $\Gamma \df \val{\tau_1}\times\cdots\times\val{\tau_n}$, то
  за произволен терм $\tau$ с променливи измежду $X_1,\dots,X_n$ е изпълнено, че:
  \[\val{\tau}(L_1,\dots,L_n) \subseteq \{\alpha \in \Sigma^\star \mid \tau \Downarrow \alpha\}.\]
\end{problem}

\begin{problem}
  Нека $L_1,\dots,L_n$ е неподвижна точка на $\Gamma \df \val{\tau_1}\times \cdots \times \val{\tau_n}$.
  Докажете, че за всеки терм $\tau$ на езика \CFG с променливи измежду $X_1,\dots,X_n$ е  изпълнено, че:
  \[\{\alpha \in \Sigma^\star \mid \tau \Downarrow \alpha\} \subseteq \val{\tau}(L_1,\dots,L_n).\]

  Заключете, че ако $L_1,\dots,L_n$ е най-малката неподвижна точка на $\Gamma$, то за всеки индекс $i$ измежду $1,\dots,n$ е изпълнено, че:
  \[L_i = \{\alpha \in \Sigma^\star \mid \tau_i \Downarrow \alpha\}.\]  
\end{problem}

%%% Local Variables:
%%% mode: latex
%%% TeX-master: "../sep"
%%% End:
