\section{Алгебрични области на Скот}

\marginpar{\cite{abramsky94}, \cite{barendregt-handbook}}

\begin{itemize}
\item 
  \index{компактен елемент}
  Нека $\A$ е област на Скот.
  Казваме, че елементът $c$ е {\bf компактен}, ако 
  за всяка верига $\chain{a}{i}$, за която $c \sqsubseteq \bigsqcup_i a_i$,
  съществува индекс $i_0$, за който $c \sqsubseteq a_{i_0}$.
  Компактните елементи на $\A$ ще означаваме с $K(\A)$.
\item
  \index{област на Скот!алгебрична}
  Ще казваме, че областта на Скот $\A$ е {\bf алгебрична}, ако за всеки елемент $a \in \A$,
  съществува верига от {\em компактни} елементи $\chain{c}{i}$ в $\A$, за която $a = \bigsqcup_i c_i$.
\end{itemize}

\begin{example}
  Нека да разгледаме $\A = \pair{A,\sqsubseteq, \bot}$,
  където 
  \[A = \{a_0,a_1,\dots,\} \cup \{a_\omega, b\},\]
  релацията $\sqsubseteq$ е представена на \Fig{noncompact-element}.
  Лесно се съобразява, че $\A$ е област на Скот.
  Всеки от елементите на $A$ е компактен, с изключение на $a_\omega$ и $b$.
  Също така е ясно, че $\A$ {\em не е} алгебрична област на Скот, защото 
  няма верига от крайни елементи, чиято точна горна граница да е елементът $b$.
  \begin{framed}
    \begin{figure}[H]
    \centering
    \begin{tikzpicture}[shorten >=1pt,->]
      \tikzstyle{vertex}=[circle,minimum size=17pt,inner sep=0pt]
      
      \node[vertex] (omega) at (0,5) {$a_\omega$};
      \node[vertex] (2) at (0,3) {$a_3$};
      \node[vertex] (1) at (0,2) {$a_2$};
      \node[vertex] (0) at (0,1) {$a_1$};
      \node[vertex] (bot) at (0,0) {$a_0$};
      
      \node[vertex] (a) at (3,3) {$b$};

      \draw (bot) -- (a);
      \draw (a)   -- (omega);
      \draw (bot) -- (0);
      \draw (0)   -- (1);
      \draw (1)   -- (2);
      \draw[dashed] (2) -- (omega);
    \end{tikzpicture}    
    \caption{Графично представяне на $\sqsubseteq$ върху $\A$}
    \label{fig:noncompact-element}
  \end{figure}
\end{framed}
\end{example}

\begin{example}
  Областта на Скот $\F_n$ е алгебрична.
  Компактните елементи са тези функции, за които $|Dom(f)| < \infty$, т.е.
  крайните функции.   
\end{example}


\begin{framed}
  \begin{thm}
    Нека $\A$ и $\B$ са области на Скот, където $\A$ е {\em алгебрична}.
    Тогава $f \in \Cont{\A}{\B}$ точно тогава, когато за произволен елемент $a \in A$,
    \[f(a) = \bigsqcup\{f(c) \mid c \sqsubseteq a\ \&\ c \in K(\A)\}.\]
  \end{thm}
\end{framed}
\begin{proof}
  \marginpar{\cite[стр. 17]{barendregt-handbook}}
  \begin{enumerate}[(1)]
  \item
    Нека $f \in \Cont{\A}{\B}$ и да разгледаме произволен елемент $a \in A$.
    Нека $c$ е комапктен елемент, за който $c \sqsubseteq a$.
    \marginpar{Всяко непрекъснато изображение е монотонно}
    Тогава $f(c) \sqsubseteq f(a)$, защото $f$ е монотонно изображение.
    Това означава, че $f(a)$ е горна граница на множеството
    $\{f(c) \mid c \sqsubseteq a\ \&\ c\in K(\A)\}$.
    
    Нека сега $b$ е друга горна граница на $\{f(c) \mid c \sqsubseteq a\ \&\ c\in K(\A)\}$.
    Ще докажем, че $f(a) \sqsubseteq b$.

    Понеже $\A$ е алгебрична област на Скот, то $a = \bigsqcup_i c_i$, за някоя вергига $\chain{c}{i}$ от компактни елементи.
    Знаем, че $f(c_i) \sqsubseteq b$ за всеки компактен елемент $c_i \sqsubseteq a$.
    \marginpar{$\chain{f(c_i)}{i}$ образуват верига и следователно притежава точна горна граница}
    Тогава $\bigsqcup_i f(c_i) \sqsubseteq b$ и следователно
    $f(\bigsqcup_i c_i) \sqsubseteq b$, защото $f$ е непрекъснато изображение.
    Понеже $a = \bigsqcup_i c_i$, то получаваме, че $f(a) \sqsubseteq b$.

    От всичко това следва, че
    \[f(a) = \bigsqcup \{f(c) \mid c \sqsubseteq a\ \&\ c\in K(\A)\}.\]
  \item
    Сега да разгледаме обратната посока, т.е. нека $f$ е изображение, за което
    за произволен елемент $a \in A$ е изпълнено, че
    \[f(a) = \bigsqcup\{f(c) \mid c \sqsubseteq a\ \&\ c \in K(\A)\}.\]

    Нека първо да проверим, че $f$ е монотонно изображение.
    За целта, нека разгледаме произволни елементи $a,b\in A$, за които $a \sqsubseteq b$.
    Ясно е, че
    \[\{f(c) \mid c \sqsubseteq a\ \&\ c \in K(\A) \}\subseteq \{f(c) \mid c \sqsubseteq b\ \&\ c \in K(\A) \}.\]
    Оттук директно получаваме, че
    \begin{align*}
      f(a) & = \bigsqcup \{f(c) \mid c \sqsubseteq a\ \&\ c \in K(\A) \}\\
           & \sqsubseteq \bigsqcup\{f(c) \mid c \sqsubseteq b\ \&\ c \in K(\A) \}\\
           & = f(b).
    \end{align*}
    Така, щом $f$ е монотонно изображение, то можем да заключим, че
    за произволна верига $(a_i)^\infty_{i=0}$ е изпълнено
    \[\bigsqcup_i f(a_i) \sqsubseteq f(\bigsqcup_i a_i).\]

    За другата посока, да разгледаме произволна верига $(a_i)^\infty_{i=0}$.
    Тогава ако $c$ е компактен елемент и $c \sqsubseteq \bigsqcup_i a_i$,
    то съществува индекс $i_0$, за който $c \sqsubseteq a_{i_0}$.
    Понеже $f$ е монотонно изображение, то
    \[f(c) \sqsubseteq f(a_{i_0}) \sqsubseteq \bigsqcup_i f(a_i).\]
    Това означава, че елементът $\bigsqcup_i f(a_i)$
    е горна граница на множеството
    \[\{f(c) \mid c \sqsubseteq \bigsqcup_i a_i\ \&\ c \in K(\A)\}.\]
    Оттук заключаваме, че
    \[f(\bigsqcup_i a_i) = \bigsqcup\{f(c) \mid c \sqsubseteq \bigsqcup_i a_i\ \&\ c \in K(\A)\} \sqsubseteq \bigsqcup_i f(a_i).\]
  \end{enumerate}
\end{proof}


Използвайки факта, че $\F_n$ е алгебрична област на Скот, то имаме следната полезна харектеризация.
\begin{framed}
  \begin{cor}
    Следните условия са еквивалентни:
    \begin{enumerate}[(1)]
    \item
      $\Gamma \in \Cont{\F_n}{\F_m}$;
    \item
      $\Gamma(f)(\ov{x}) \simeq y \iff (\exists \theta \subseteq f)[\ \theta\text{ е крайна функция}\ \&\ \Gamma(\theta)(\ov{x}) \simeq y\ ]$.
    \end{enumerate}
  \end{cor}
\end{framed}
Понякога се оказва, че за проверката дали даден оператор $\Gamma$ е непрекъснат е по-лесно да се провери условието (2).

\Stefan{Да се даде поне един пример!}

%%% Local Variables:
%%% mode: latex
%%% TeX-master: "../sep"
%%% End:
