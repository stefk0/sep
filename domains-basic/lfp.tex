\section{Най-малки неподвижни точки}
\index{най-малка неподвижна точка}

\begin{itemize}
\item 
  \index{неподвижна точка}
  Да фиксираме произволна област на Скот $\A = (A, \sqsubseteq, \bot)$ и да разгледаме едно изображение $f:\A\to\A$.
  Казваме, че $a \in \A$ е {\bf неподвижна точка} на $f$, ако $f(a) = a$.
\item
  \index{най-малка неподвижна точка}
  Казваме, че $a \in \A$ е {\bf най-малката неподвижна точка} на $f$, ако:
  \begin{itemize}
  \item 
    $a$ е неподвижна точка, т.е. $f(a) = a$;
  \item
    за всяко $b \in \A$ със свойството, че $f(b) = b$ имаме $a \sqsubseteq b$.
  \item
    \marginpar{least fixed point}
    Ще означаваме най-малката неподвижна точка на $f$ като $\lfp(f)$.
  \end{itemize}
\end{itemize}

\begin{framed}
\begin{thm}[Клини]
  \label{th:knaster-tarski}
  \index{Клини}
  Нека $\A$ е област на Скот.
  Всяко $f \in \Cont{\A}{\A}$ притежава най-малка неподвижна точка.
\end{thm}
\end{framed}
\begin{proof}
  \marginpar{В \cite{ditchev-soskov} се нарича теорема на Кнастер-Тарски. Според \href{https://en.wikipedia.org/wiki/Kleene_fixed-point_theorem}{уикипедия} е теорема на Клини}
  Определяме монотонно растяща редица от елементи на $\A$ по следния начин:
  \begin{align*}
    & a_0 \dff \bot & \comment = f^0(\bot)\\
    & a_{n+1} \dff f(a_n) & \comment = f^{n+1}(\bot).
  \end{align*}

  Първо ще докажем с индукция по $n$, че $\chain{a}{n}$ е верига.
  Ясно е, че $a_0 \sqsubseteq a_1$.
  Да приемем, че $a_n \sqsubseteq a_{n+1}$. Тогава, понеже всяко непрекъснато
  изображение е монотонно, то имаме, че
  \[\underbrace{f(a_n)}_{a_{n+1}} \sqsubseteq \underbrace{f(a_{n+1})}_{a_{n+2}}.\]

  Нека $a \dff \bigsqcup_i a_i$. Тогава 
  \begin{align*}
    f(a) & = f(\bigsqcup_i a_i) & \comment a \dff \bigsqcup_i a_i\\
         & = \bigsqcup_i f(a_i) & \comment f \text{ е непрекъсната}\\
         & = \bigsqcup_i a_{i+1} & \comment a_{i+1} = f(a_i)\\
         & = \bigsqcup_i a_i & \comment \text{защото }\chain{a}{i}\text{ е верига}\\
         & = a.
  \end{align*}
  Така доказахме, че $a$ е {\em неподвижна точка} на $f$.
  Остана да видим, че е най-малката неподвижна точка на $f$.

  Нека $b = f(b)$. С индукция по $n$ ще докажем, $(\forall n)[a_n \sqsubseteq b]$.
  \begin{itemize}
  \item 
    За $n = 0$ е очевидно.
  \item
    Да приемем, че $a_n \sqsubseteq b$.
    Тогава $a_{n+1} \dff f(a_n) \sqsubseteq f(b) = b$, защото $f$ е монотонно изображение.    
  \end{itemize}
  Така доказахме, че $b$ е горна граница на веригата $\chain{a}{n}$.
  Заключаваме, че $a \dff \bigsqcup_n a_n \sqsubseteq b$.
  Следователно, $a$ е {\em най-малката неподвижна точка} на $f$,
  т.е. $a = \lfp(f)$.
\end{proof}

\begin{prop}
  \label{pr:prefix-point}
  За всяко $f \in \Cont{\A}{\A}$ е изпълнено, че 
  \[(\forall a \in \Pref(f))[\lfp(f) \sqsubseteq a],\]
  където
  \index{преднеподвижна точка}
  \[\Pref(f) \dff \{a \in \A \mid f(a) \sqsubseteq a\}\]
  е множеството от всички преднеподвижни точки на $f$.
  Това означава, че $\lfp(f)$ е най-малката преднеподвижна точка на $f$.
\end{prop}
\begin{proof}
  Знаем от Теоремата на Клини, че $\lfp(f) = \bigsqcup_n f^n(\bot)$.
  Нека за улеснение $b_n \dff f^n(\bot)$.
  От доказателството на \hyperref[th:knaster-tarski]{Теоремата на Клини} знаем, че $\chain{b}{n}$ е верига.
  Ясно е също, че $\texttt{Pref}(f) \neq \emptyset$, защото $\lfp(f) \in \texttt{Pref}(f)$.
  Да фиксираме прозиволен елемент $a\in \texttt{Pref}(f)$.
  С индукция по $n$ ще докажем, че $b_n \sqsubseteq a$ за всяко $n$.
  \begin{itemize}
  \item 
    За $n = 0$ е очевидно, защото тогава $b_0 \dff \bot \sqsubseteq a$.
  \item
    Да приемем, че $b_n \sqsubseteq a$.
    Ще докажем, че $b_{n+1} \sqsubseteq a$.
    Но това е лесно.
    \begin{align*}
      b_{n+1} & = f(b_n) & \comment \text{от деф. на }b_{n+1}\\
      & \sqsubseteq f(a) & \comment b_n \sqsubseteq a\ \&\ f\text{ е мон.}\\
      & \sqsubseteq a & \comment a \in \texttt{Pref}(f).
    \end{align*}
  \end{itemize}
  Така доказахме, че за всяко $n$, $b_n \sqsubseteq a$,
  откъдето следва, че $a$ е горна граница за веригата $\chain{b}{n}$, откъдето директно получаваме, че
  \[\lfp(f) = \bigsqcup_n b_n \sqsubseteq a.\]
\end{proof}


%%% Local Variables:
%%% mode: latex
%%% TeX-master: "../sep-notes"
%%% End:
