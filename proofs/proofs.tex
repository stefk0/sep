\chapter{Доказване на свойства на програми}

След като вече имаме точна дефиниция на семантиката на една рекурсивна програма, можем да видим как можем да доказваме
формално свойства на рекурсивни програми.

\section{Свойства}
\begin{itemize}
\item 
  Да фиксираме една област на Скот $\A$. Подмножествата $P \subseteq \A$ ще наричаме {\bf свойства}.
\item
  \index{непрекъснато свойство}
  \marginpar{В \cite[стр. 166]{winskel} се наричат {\em inclusive subsets}. В \cite{bird-haskell} се наричат {\em chain-complete assertions}}
  Казваме, че $P$ е {\bf непрекъснато (или допустимо, индуктивно) свойство} над областта на Скот $\A$, ако за всяка верига $(a_i)^{\infty}_{i=0}$ от елементи на $\A$, е изпълнено:
  \begin{prooftree}
    \AxiomC{$P(a_0)$}
    \AxiomC{$P(a_1)$}
    \AxiomC{$P(a_2)$}
    \AxiomC{$P(a_3)$}
    \AxiomC{$\ldots$}
    \QuinaryInfC{$P(\bigsqcup_i a_i)$}
  \end{prooftree}
  % за която $P(a_i)$, то е изпълнено и $P(\bigsqcup_i a_i)$.
  \marginpar{Понеже $\A$ е област на Скот, ние знаем, че $\bigsqcup_i a_i$ съществува}
\end{itemize}

Нека да видим, че има свойства, който не са непрекъснати.

\begin{example}
  \label{ex:complement-not-inclusive}
  % Нека да разгледаме областта на Скот от точните изображения $\Strict{\Nat_\bot}{\Nat_\bot}$.
  Да разгледаме свойството $P \subseteq \Cont{\Nat_\bot}{\Nat_\bot}$, което е дефинирано по следния начин:
  \[P(f) \dfff (\exists x \in \Nat)[f(x) = \bot].\]
  Да разгледаме изображенията $f_i$, дефинирани по следния начин:
  \begin{align*}
    f_i(x) & =
    \begin{cases}
      42, & x \in \Nat\ \&\ x \leq i\\
      \bot, & x \in \Nat\ \&\ x > i\\
      \bot, & x = \bot.
    \end{cases}
  \end{align*}
  Лесно се съобразява, че $(f_i)^{\infty}_{i=0}$ е верига в $\Strict{\Nat_\bot}{\Nat_\bot}$ и че 
  за всяко $i$, $P(f_i)$. Да разгледаме точната горна граница $g$ на тази верига, за която знаем, че
  за всяко $y \in \Nat$,
  \[g(x) = y \iff (\exists i)[f_i(x) = y].\]
  Според конструкцията на $g$, $g(x) = 42$ за всяко $x \in \Nat$.
  Оттук директно получаваме, че $\neg P(g)$.
  Така видяхме, че $P$ {\em не е непрекъснато свойство}.
\end{example}

% \begin{example}
%   Нека да разгледаме областта на Скот $\S_1$, т.е. точните функции на един аргумент.
%   Да разгледаме свойството $P \subseteq \S_1$, което е дефинирано по следния начин:
%   \[P(f) \dfff (\exists x \in \Nat)[f(x) = \bot].\]
%   Да разгледаме функциите $f_i$, дефинирани по следния начин:
%   \begin{align*}
%     f_i(x) & =
%     \begin{cases}
%       42, & x \leq i\\
%       \bot, & x > i\\
%       \bot, & x = \bot.
%     \end{cases}
%   \end{align*}
%   Ясно е, че $P(f_i)$ е изпълнено за всяко $i$.
%   Да разгледаме функцията $g = \bigsqcup_i f_i$, която е дефинирана като:
%   \[g(x) = \bigsqcup_i\{f_i(x)\}.\]
%   Лесно се съобразява, че за всяко $x \in \Nat$, $g(x) \neq \bot$. Следователно, $\neg P(g)$.
%   Така видяхме, че $P$ не е непрекъснато свойство.
% \end{example}

\begin{problem}
  \label{prob:inclusive-property}
  Докажете, че свойството $Q$ над $\Cont{\Nat_\bot}{\Nat_\bot}$, където
  \[Q(f) \iff (\forall x \in \Nat)[f(x) \neq \bot],\]
  е непрекъснато.
\end{problem}

\subsection{Основни свойства}

\begin{proposition}
  \label{pr:continuous-property}
  Нека $\A$ и $\B$ са области на Скот и $f_1,f_2 \in \Cont{\A}{\B}$.
  Тогава следните свойства над $\A$ са непрекъснати:
  \begin{itemize}
  \item 
    $P(a) \dfff f_1(a) \sqsubseteq f_2(a)$;
  \item
    $P(a) \dfff f_1(a) = f_2(a)$;
  \end{itemize}
\end{proposition}

\begin{proposition}
  \label{pr:fixed-element-property}
  Нека $\A$ е област на Скот.
  Да фиксираме произволен елемент $a_0 \in \A$.
  Тогава следните свойства над $\A$ са непрекъснати:
  \begin{itemize}
  \item 
    $P(a) \dfff a \sqsubseteq a_0$;
  \item
    $P(a) \dfff a = a_0$;
  \end{itemize}
\end{proposition}

\subsection{Сечение}
\begin{proposition}
  Нека $P_1$ и $P_2$ са непрекъснати свойства над областта на Скот $\A$.
  Тогава $P_1 \cap P_2$ също е непрекъснато свойство.
\end{proposition}

\subsection{Обединение}

\begin{proposition}
  Нека $P_1$ и $P_2$ са непрекъснати свойства над областта на Скот $\A$.
  Тогава $P_1 \cup P_2$ също е непрекъснато свойство.
\end{proposition}

\subsection{Допълнение}

\begin{proposition}
  Съществува непрекъснато свойство $P$ над областта на Скот $\A$,
  за което $\A \setminus P$ {\bf не} е непрекъснато свойство.
\end{proposition}
\begin{proof}
  Да вземем $\A = \Cont{\Nat_\bot}{\Nat_\bot}$.
  Свойството $Q$ от \Problem{inclusive-property} е непрекъснато, 
  докато $\Cont{\Nat_\bot}{\Nat_\bot} \setminus Q = P$, което е точно свойството от \Ex{complement-not-inclusive}, а 
  за него знаем, че не е непрекъснато.
\end{proof}

% \subsection{Образ}

% \begin{problem}
%   Нека $P$ е непрекъснато свойство в областта на Скот $\B$.
%   Нека $f \in \Cont{\A}{\B}$.
%   Да разгледаме свойството 
%   \[f[P] \dff \{f(a) \mid f(a) \in P\}.\]
%   Не винаги $f[P]$ е непрекъснато свойство в $\A$.
% \end{problem}
% \ifhints
% \begin{hint}
%   Нека $(b_n)^\infty_{n=0}$ е верига в $\B$,
%   като $\bigsqcup_n b_n$ не е елемент на веригата.
%   Нека дефинираме изображението $f$ по следния начин:
%   $f(\bot) \dff \bot$ и $f(n) \dff b_n$.
%   Лесно се съобразява, че $f \in \Cont{\Nat_\bot}{\B}$.
%   Нека $P = \Nat$, което очевидно е непрекъснато свойство, защото елементите на $\Nat$ 
%   не са сравними относно плоската наредба.
%   Тогава $f[P] = \{b_n \mid n \in \Nat\}$.
%   Лесно се проверява, че $f[P]$ не е непрекъснато свойство.
% \end{hint}
% \fi

% \subsection{Първообраз}

% \begin{problem}
%   Нека $P$ е непрекъснато свойство в областта на Скот $\B$.
%   Нека $f \in \Cont{\A}{\B}$.
%   Да разгледаме свойството 
%   \[f^{-1}[P] \dff \{a \in \A \mid f(a) \in P\}.\]
%   Докажете, че $f^{-1}[P]$ е непрекъснато свойство в $\A$.
% \end{problem}

% \subsection{Композиция}

% \begin{problem}
%   Нека $P$ е непрекъснато свойство в областта на Скот $\A \times \B$,
%   а $Q$ е непрекъснато свойство в областта на Скот $\B \times \C$.
%   Композицията 
%   \[Q \circ P = \{\pair{a,c} \in \A\times \C \mid (\exists b\in\B)[\pair{a,b} \in P\ \&\ \pair{b,c} \in Q]\}.\]
%   Докажете, че {\bf не винаги} $Q \circ P$ е непрекъснато свойство.
% \end{problem}
% \ifhints
% \begin{hint}
%   Нека $(a_n)^\infty_{n=0}$ е верига в $\A$, а $(c_n)^{\infty}_{n=0}$ е верига в $\C$,
%   като и двете вериги са такива, че $\bigsqcup_n a_n$ не е елемент на $(a_n)^{\infty}_{n=0}$
%   и $\bigsqcup_n c_n$ не е елемент на $(c_n)^\infty_{n=0}$.
%   Нека $\B = \Nat_\bot$.  
%   Тогава дефинираме свойствата по следния начин:
%   \begin{align*}
%     & P \dff \{\pair{a_n,n} \mid n \in \Nat\};\\
%     & Q \dff \{\pair{n,c_n} \mid n \in \Nat\}.
%   \end{align*}
%   Лесно се проверява, че тези свойства са непрекъснати.
%   Тогава,
%   \[Q \circ P = \{\pair{a_n,c_n} \mid n \in \Nat\},\]
%   което очевидно не е непрекъснато свойство.
% \end{hint}
% \fi

% \subsection{Проекции}

% \begin{problem}
%   Нека $P$ е непрекъснато свойство в областта на Скот $\A \times \B$.
%   Нека за произволно $a$ да дефинираме свойството 
%   \[P_a \dff \{b \in \B \mid \pair{a,b} \in P\}.\]
%   Тогава $P_a$ е непрекъснато свойство.
%   Наричаме $P_a$ проекция на $P$ по първата компонента.
% \end{problem}

% Ние знаем, че едно изображение $f:\A\times \B \to \C$ е непрекъснато точно тогава, когато
% $f$ е непрекъснато по всеки от аргументите си.
% Ако $P$ е непрекъснато свойство в $\A\times\B$, то е ясно, че $P$ е непрекъснато по всяка от проекциите си.

% \begin{problem}
%   Да разгледаме свойството $P$ в $\A \times \B$, за което 
%   имаме, че $P$ е непрекъснато свойство по всяка от проекциите.
%   Вярно ли е, че тогава $P$ е непрекъснатото?
% \end{problem}
% \ifhints
% \begin{hint}
%   Нека $\A$ и $\B$ са такива области на Скот, в които има вериги съответно $(a_n)^\infty_{n=0}$ и $(b_n)^\infty_{n=0}$,
%   за които $\bigsqcup_n a_n$ и $\bigsqcup_n a_n$ не са елементи на съответните вериги.
%   Нека $P = \{\pair{a_n,b_n} \mid n \in \Nat\}$.
%   Тогава за всяко $n$, $P_{a_n}$ и $P_{b_n}$ са непрекъснати, защото
%   $P_{a_n} = \{b_n\}$ и $P_{b_n} = \{a_n\}$,
%   но е очевидно, че $P$ не е непрекъснато свойство.
% \end{hint}
% \fi

% \subsection{Универсално затваряне}

% \begin{problem}
%   Нека $P$ е непрекъснато свойство в областта на Скот $\A \times \B$.
%   Нека за произволно $a$ да дефинираме свойството 
%   \[Q \dff \{b \in \B \mid (\forall a \in \A)[\pair{a,b} \in P]\}.\]
%   Вярно ли е, че $Q$ е непрекъснато свойство?
%   Обосновете отговора си!
% \end{problem}
% \ifhints
% \begin{hint}
%   Вярно е.
% \end{hint}
% \fi
% \ifhints
% \begin{hint}
%   Нека $\A$ и $\B$ са такива области на Скот, в които има вериги съответно $(a_n)^\infty_{n=0}$ и $(b_n)^\infty_{n=0}$,
%   за които $\bigsqcup_n a_n$ и $\bigsqcup_n a_n$ не са елементи на съответните вериги.
%   Нека 
%   \[P \dff \{\pair{a_n,b_k} \mid n,k \in \Nat\} \cup \{\pair{\bigsqcup_n a_n, b_k} \mid k \in \Nat\} \cup \{\pair{\bigsqcup_n a_n, \bigsqcup_k b_k}\}.\]
%   Тогава $Q = \{a_n \mid n \in \Nat\}$.
% \end{hint}
% \fi

% \subsection{Екзистенциално затваряне}

% \begin{problem}
%   Нека $P$ е непрекъснато свойство в областта на Скот $\A \times \B$.
%   Нека за произволно $a$ да дефинираме свойството 
%   \[Q \dff \{b \in \B \mid (\exists a \in \A)[\pair{a,b} \in P]\}.\]
%   Вярно ли е, че $Q$ е непрекъснато свойство?
% \end{problem}
% \ifhints
% \begin{hint}
%   Разгледайте $\A = \B = \Int \cup\{-\infty,+\infty\}$.
%   Нека 
%   \[P = \{\pair{-n,n} \mid n \in \Nat\}.\]
%   Тогава $P$ е непрекъснато свойство.
%   Но тогава $Q = \{n \mid n \in \Nat\}$ не е непрекъснато свойство.
% \end{hint}
% \fi

% \subsection{Непрекъснати изображения}

% Знаем, че ако $\A$ и $\B$ са области на Скот, то съвкупността от всички непрекъснати изображения $\Cont{\A}{\B}$ образува
% област на Скот. Сега ще разгледаме аналог на тази теорема за непрекъснати свойства.

% \begin{proposition}
%   Нека $P$ и $Q$ са свойства съответно в $\A$ и $\B$.
%   Да разгледаме свойството $R$ в $\Cont{\A}{\B}$ дефинирано като:
%   \[R \dff \{f \in \Cont{\A}{\B} \mid (\forall a \in \A)[P(a) \implies Q(f(a))]\}.\]
%   Докажете, че ако $Q$ е непрекъснато свойство, то $R$ е непрекъснато свойство.
% \end{proposition}

\subsection{Частична коректност}

\begin{itemize}
\item
  Да разгледаме едно свойство $I$ в областта на Скот $\A$, което наричаме условие за входа, и
  свойство $O$ в областта на Скот $\A \times \B$, което наричаме условие за изхода.
\item
  \index{частична коректност}
  {\bf Свойство от тип частична коректност} относно $I$ и $O$ представлява 
  свойство $P \subseteq \Mapping{\A}{\B}$ със следната дефиниция
  \[P(f) \dfff (\forall a \in \A)[\ I(a)\ \&\ f(a) \neq \bot \implies O(a,f(a))\ ].\]
\end{itemize}

\begin{proposition}
  Нека $I \subseteq \A$, а $O$ е непрекъснато свойство в $\A \times \B$.
  \[P(f) \dfff (\forall a \in \A)[\ I(a)\ \&\ f(a) \neq \bot \implies O(a,f(a))\ ].\]
  Тогава свойството $P$ е непрекъснато в областта на Скот $\Mapping{\A}{\B}$.
\end{proposition}

% \begin{example}
%   Нека $\A = \B = \Nat_\bot$ и $I(x) \dfff x > 1$, а $O(x,y) \dfff x,y\in\Nat\ \&\ x = y^2$.
%   Ясно е, че $O$ е непрекъснато свойство в $\Nat^2_\bot$. Да разгледаме свойството
%   \[P(f) \dfff (\forall x \in \Nat_\bot)[x > 1\ \&\ f(x) \neq \bot \implies O(x,f(x))].\]
%   Знаем, че $P$ е от тип частична коректно и следователно е непрекъснато в областта на Скот 
%   $\Mapping{\Nat^2_\bot}{\Nat_\bot}$.
%   Понеже $\Sigma^\star:\Cont{\Partial{\Nat^2}{\Nat}}{\Strict{\Nat^2_\bot}{\Nat_\bot}}$, то
%   $Q \dfff (\Sigma^\star)^{-1}[P]$ е непрекъснато свойство в областта на Скот $\Partial{\Nat^2}{\Nat}$.
%   Ясно е, че
%   \[Q(f) \equiv (\forall x \in \Nat)[x > 1\ \&\ !f(x) \implies O(x,f(x))].\]
% \end{example}


% \subsection{Тотална коректност}

% \index{тотална коректност}
% {\bf Свойство от тип тотална коректност} относно $I$ и $O$ представлява 
% свойство $P \subseteq \Mapping{\A}{\B}$ със следната дефиниция
% \[P(f) \dfff (\forall a \in \A)[I(a) \implies (f(a) \neq \bot\ \&\  O(a,f(a)))].\]

% \begin{proposition}
%   Нека $I$ е свойство в $\A$, а $O$ е непрекъснато свойство в $\A \times \B$.
%   Тогава свойството
%   \[P(f) \dfff (\forall a \in \A)[I(a) \implies (f(a) \neq \bot\ \&\ O(a,f(a)))]\]
%   е непрекъснато в $\Mapping{\A}{\B}$.
% \end{proposition}

%%% Local Variables:
%%% mode: latex
%%% TeX-master: "../sep"
%%% End:


% \newpage
% \setcounter{problem}{0}

% \begin{problem}
%   Нека е даден следния оператор $\Gamma:\F_1\to\F_1$:
%   \begin{align*}
%     \Gamma(f)(x) \simeq 
%     \begin{cases}
%       0, & f\text{ е крайна функция}\\
%       1, & \text{ иначе }.
%     \end{cases}
%   \end{align*}
%   Проверете дали:
%   \begin{enumerate}[a)]
%   \item 
%     $\Gamma$ е монотонен оператор;
%   \item
%     $\Gamma$ е компактен оператор.
%   \end{enumerate}
% \end{problem}
% \begin{proof}
%   \begin{enumerate}[a)]
%   \item 
%     Трябва да проверим дали:
%     \[(\forall f,g\in\F_1)[f \subseteq g\ \Rightarrow \Gamma(f) \subseteq \Gamma(g)].\]
%     Да изберем $f \subseteq g$ да бъдат такива функции, че $f$ е крайна функция, но $g$ не е крайна функция.
%     Тогава $(\forall x \in \Nat)[\Gamma(f)(x) \simeq 0\ \&\ \Gamma(g)(x) \simeq 1]$.
%     Очевидно е, че за така избраните $f$ и $g$, $\Gamma(f) \not\subseteq \Gamma(g)$.
%   \item
%     Знаем, че всеки компактен оператор е монотонен.
%     Щом $\Gamma$ не е монотонен, то със сигурност $\Gamma$ не е компактен.
%   \end{enumerate}
% \end{proof}

% \begin{problem}
%   Нека е даден следния оператор $\Gamma:\F_1\to\F_1$:
%   \begin{align*}
%     \Gamma(f)(x) \simeq 
%     \begin{cases}
%       \neg !, & f\text{ е крайна функция}\\
%       1, & \text{ иначе }.
%     \end{cases}
%   \end{align*}
%   Проверете дали:
%   \begin{enumerate}[a)]
%   \item 
%     $\Gamma$ е монотонен оператор;
%   \item
%     $\Gamma$ е компактен оператор.
%   \end{enumerate}
% \end{problem}
% \begin{proof}
%   \begin{enumerate}[a)]
%   \item 
%     Трябва да проверим дали:
%     \[(\forall f,g\in\F_1)[f \subseteq g\ \Rightarrow \Gamma(f) \subseteq \Gamma(g)].\]
%     Нека $f \subseteq g$ са произволни функции от $\F_1$.
%     Ще разгледаме два случая.
%     \begin{itemize}
%     \item 
%       $f$ е крайна функция. Тогава $\Gamma(f) = \emptyset^{(1)}$ и е очевидно, че 
%       \[\Gamma(f) = \emptyset^{(1)} \subseteq \Gamma(g).\]
%     \item
%       $f$ не е крайна функция. Щом $f \subseteq g$, то $g$ също не е крайна функция.
%       Тогава 
%       \[(\forall x \in \Nat)[\Gamma(f)(x) \simeq 1 \simeq \Gamma(g)(x)],\]
%       от което следва, че 
%       \[\Gamma(f) \subseteq \Gamma(g).\]
%     \end{itemize}
%     Разгледахме всички възможни случаи за $f$ и във всеки от тях получихме, че $\Gamma(f) \subseteq \Gamma(g)$.
%     Следователно, $\Gamma$ е монотонен оператор.
%   \item
%     Сега ще проверим дали е изпълнено, че:
%     \begin{equation}
%       \label{eq:compact}
%       (\forall f \in \F_1)(\forall x,y\in\Nat)[\Gamma(f)(x)\simeq y\ \Rightarrow\ (\exists \theta \subseteq f)[\Gamma(\theta)(x) \simeq y]].
%     \end{equation}
%     Тук с $\theta$ означаваме крайна функция.
%     Нека $f$ е не е крайна функция.% , например, $f(x) = x^2$ за всяко $x \in \Nat$.
%     Тогава е ясно, че за всяко $x \in \Nat$, $\Gamma(f)(x) \simeq 1$.
%     От друга страна, понеже $\theta$ е крайна, $\neg ! \Gamma(\theta)(x)$ за всяко $x \in \Nat$.
    
%     Така видяхме, че ако $f$ не е крайна, то за произволно $x$, $\Gamma(f)(x) \simeq 1$,
%     но не съществува крайна $\theta \subseteq f$, за която $\Gamma(\theta)(x) \simeq 1$.
%     От това следва, че Формула (\ref{eq:compact}) не е изпълнена.
%     Следователно, $\Gamma$ не е компактен оператор.
%   \end{enumerate}
% \end{proof}


% \begin{problem}
%   Нека е даден следния оператор $\Gamma:\F_2\to\F_2$:
%   \begin{align*}
%     \Gamma(f)(x,y) \simeq &
%     \begin{cases}
%       y, & x = 0\\
%       f(x, f(x-1,y)), & x > 0.
%     \end{cases}
%   \end{align*}
%   \begin{enumerate}[a)]
%   \item 
%     Докажете, че $\Gamma$ е компактен оператор.
%   \item
%     Коя е най-малката неподвижна точка на $\Gamma$?
%   \item
%     Има ли $\Gamma$ други неподвижни точки ?
%   \end{enumerate}
% \end{problem}


% \begin{problem}
%   Нека е даден следния оператор $\Gamma:\F_1\to\F_1$:
%   \begin{align*}
%     \Gamma(f)(x) \simeq &
%     \begin{cases}
%       0, & x = 0\\
%       f(f(x-1)+1)), & x > 0.
%     \end{cases}
%   \end{align*}
%   \begin{enumerate}[a)]
%   \item 
%     Докажете, че $\Gamma$ е компактен оператор.
%   \item
%     Коя е най-малката неподвижна точка на $\Gamma$?
%   \item
%     Има ли $\Gamma$ други неподвижни точки ?
%   \end{enumerate}
% \end{problem}

% \section{Структурна индукция}

% % \Stefan{Да се махне оттук.}
% \begin{problem}
%   Да разгледаме програмите $\texttt{fib}$ и $\texttt{fib'}$:
  
%   \begin{haskellcode}
% fib(n) = f(n) where
%   f(n) = if n == 0 then 0
%            else if n == 1 then 1
%              else f(n-1) + f(n-2)

% fib'(n) = g(0,1,n) where
%   g(a,b,n) = if n == 0 then a
%                else g(b, a+b, n-1)
%   \end{haskellcode}
  
%   Докажете, че $\D_V\val{\texttt{fib}} = \D_V\val{\texttt{fib'}}$.
% \end{problem}
% \begin{hint}
%   Да разгледаме операторите:
%   \begin{align*}
%     \Gamma(f)(x) =
%     \begin{cases}
%       0, & x = 0\\
%       1, & x = 1\\
%       f(x-1) + f(x-2), & x \geq 2\\
%       \bot, & x = \bot.
%     \end{cases}
%   \end{align*}

%   \begin{align*}
%     \Delta(g)(x,y,z) =
%     \begin{cases}
%       x, & z = 0\\
%       g(y,x+y,z-1), & z \geq 1\\
%       \bot, & z = \bot.
%     \end{cases}
%   \end{align*}
%   \Stefan{Малко е объркващо, че от една страна работим с $\Nat_\bot$, където имаме плоска наредба, а 
%   от друга правим индукция по наредбата на естествените числа}

%   Очевидно е, че тези оператори са непрекъснати.
%   Нека $\gamma$ е най-малката неподвижна точка на $\Gamma$ и
%   $\delta$ е най-малката неподвижна точка на $\Delta$.

%   Докажете, че с индукция по $n \in \Nat$, че 
%   \[(\forall i \in \Nat)[\delta(\gamma(i), \gamma(i+1), n) = \gamma(n+i)].\]
% \end{hint}

\section{Правило на Скот}
\marginpar{\cite[стр. 166]{winskel}}
\index{правило на Скот}
\begin{itemize}
\item 
  Нека $\A$ е област на Скот и нека $f \in \Cont{\A}{\A}$.
\item
  Всяко $P \subseteq A$ ще наричаме свойство.
\item
  \marginpar{С $\texttt{lfp}(f)$ означаваме най-малката неподвижна точка на $f$ (от англ. least fixed point)}
  Тогава {\bf правилото на Скот} гласи следното:
  \begin{prooftree}
  \AxiomC{$P(\bot)$}
  \AxiomC{$(\forall a \in \A)[P(a) \implies P(f(a))]$}
  \BinaryInfC{$P(\texttt{lfp}(f))$}
\end{prooftree}
\end{itemize}
\begin{proof}
  ....
\end{proof}

\begin{problem}
  \marginpar{Сравнете с \Prop{prefix-point}}
  Нека $f \in \Cont{\A}{\A}$.
  Да означим множеството от преднеподвижни точки на $f$ като
  \[\texttt{Pref}(f) \df \{a \in \A \mid f(a) \sqsubseteq a\}.\]
  Тогава 
  \[(\forall a \in \A)[a \in \texttt{Pref}(f) \implies \lfp(f) \sqsubseteq a].\]
\end{problem}
\begin{proof}
  Да фиксираме елемент $a \in \texttt{Pref}(f)$.
  Да разгледаме непрекъснатото свойство
  \marginpar{Сами се убедете, че $P$ е непрекъснато свойство!}
  \[P(b) \df b \sqsubseteq a.\]
  Ясно е, че $P(\bot)$.
  Нека $b\in \A$, за който $P(b)$. Ще докажем, че $P(f(b))$.
  \begin{align*}
    b \sqsubseteq a & \implies f(b) \sqsubseteq f(a) & \comment{f \text{ е мон.}}\\
    & \implies f(b) \sqsubseteq f(a) \sqsubseteq a & \comment{a \in \texttt{Pref}(f)}\\
    & \implies f(b) \sqsubseteq a & \comment{\sqsubseteq \text{ е транз. рел.}}.
  \end{align*}
  От правилото на Скот, заключаваме, че $P(\lfp(f))$, т.е.
  $\lfp(f) \sqsubseteq a$.
\end{proof}
  
\begin{problem}
  Нека $f,g \in \Cont{\A}{\A}$ като имаме свойствата:
  \begin{itemize}
  \item
    $f(\bot) \sqsubseteq g(\bot)$;
  \item
    $f \circ g \sqsubseteq g \circ f$.
  \end{itemize}
  Докажете, че $\lfp(f) \sqsubseteq \lfp(g)$.
\end{problem}
\begin{proof}
  Разгледайте непрекъснатото свойството 
  \[P(a) \df f(a) \sqsubseteq g(a).\]
  От условието имаме, че $P(\bot)$.
  Нека $P(a)$. Ще докажем, че $P(g(a))$.
  \begin{align*}
    P(a) & \iff f(a) \sqsubseteq g(a)\\
         & \implies g(f(a)) \sqsubseteq g(g(a)) & \comment{g \text{ е мон.}}\\
         & \implies f(g(a)) \sqsubseteq g(g(a)) & \comment{ f\circ g \sqsubseteq g\circ f}\\
         & \iff P(g(a)).
  \end{align*}
  Тогава по правилото на Скот ще заключим, че $P(\lfp(g))$, откъдето
  \[f(\lfp(g)) \sqsubseteq g(\lfp(g)) = \lfp(g).\]
  Това означава, че $\lfp(g)$ е преднеподвижна точка на $f$, т.е.
  \[\lfp(g) \in \texttt{Pref}(f).\]
  Понеже $\lfp(f)$ е най-малката преднеподвижна точка на $f$,
  то заключаваме, че $\lfp(f) \sqsubseteq \lfp(g)$.
\end{proof}

\begin{problem}
  Нека $h \in \Cont{\A}{\B}$, $f \in \Cont{\A}{\A}$, $g \in \Cont{\B}{\B}$,
  като имаме свойствата:
  \begin{itemize}
  \item 
    $h$ е точна, т.е. $h(\bot^\A) = \bot^\B$;
  \item
    $g\circ h = h \circ f$.
  \end{itemize}
  Докажете, че $\lfp(g) = h(\lfp(f))$.
\end{problem}
\ifhints
\begin{hint}
  \begin{itemize}
  \item 
    Разгледайте непрекъснатото свойство 
    \[P_1(a) \df h(a) \sqsubseteq g(h(a)).\]
    Докажете с правилото на Скот, че $P_1(\lfp(f))$.
    Тогава
    \begin{align*}
      h(\lfp(f)) \sqsubseteq g(h(\lfp(f)) & \iff h(f(\lfp(f))) \sqsubseteq g(h(\lfp(f)\\
                                          & \iff h(f(\lfp(f))) \sqsubseteq h(f(\lfp(f)))\\
                                          & \iff g(h(\lfp(f))) \sqsubseteq h(\lfp(f)).
    \end{align*}
    Това означава, че $h(\lfp(f))$ е преднеподвижна точка на $g$, т.е.
    \[h(\lfp(f)) \in \texttt{Pref}(g).\]
    Заключаваме, че $\lfp(g) \sqsubseteq h(\lfp(f))$.
  \item
    Другата посока е по-лесна. Разгледайте непрекъснатото свойство
    \[P_2(a) \df h(a) \sqsubseteq \lfp(g).\]
    Докажете, че $P_2(\lfp(f))$.
  \end{itemize}
\end{hint}
\fi

\begin{problem}
  Нека $f,g \in \Cont{\A}{\A}$ като имаме свойствата:
  \begin{itemize}
  \item
    $f(\bot) = g(\bot)$;
  \item
    $f \circ g = g \circ f$.
  \end{itemize}
  Докажете, че $\lfp(f \circ g) = \lfp(g \circ f)$.
\end{problem}
\ifhints
\begin{hint}
  Разгледайте непрекъснатото свойство
  \marginpar{Лесно се вижда, че $P$ е непрекъснато свойство, защото $f$ и $g$ са непрекъснати изображения.}
  \[P(a) \df g(f(a)) \sqsubseteq a.\]
  Ясно е, че $P(\bot)$.
  Нека $P(a)$. Ще докажем, че $P(f(g(a))$.
  \begin{align*}
    g(f(a)) \sqsubseteq a & \implies g(f(g(f(a)))) \sqsubseteq g(f(a))\\
    & \implies g(f(f(g(a)))) \sqsubseteq f(g(a))\\
    & \implies P(f(g(a))).
  \end{align*}
  От правилото на Скот заключваме, че $P(\lfp(f\circ g))$.
  Това означава, че 
  \[g(f(\lfp(f\circ g))) \sqsubseteq \lfp(f\circ g),\] т.е.
  $\lfp(f\circ g) \in \texttt{Pref}(g \circ f)$.
  Следователно,
  \[\lfp(g \circ f) \sqsubseteq \lfp(f\circ g).\]

  За другата посока разсъждаваме аналогично.
\end{hint}
\fi


\begin{problem}
  Нека $p \in \Cont{\A}{\Nat_\bot}$ и $h \in \Cont{\A}{\A}$, като $h$ е точна, т.е. $h(\bot) = \bot$.
  Да разгледаме 
  \[\Gamma \in \Cont{\Cont{\A\times\A}{\A}}{\Cont{\A\times\A}{\A}},\]
  където
  \begin{align*}
    \Gamma(f)(x,y) =
    \begin{cases}
      y, & p(x) = 0\\
      h(f(h(x),y)), & p(x) \in \Nat^+\\
      \bot, & p(x) = \bot.
    \end{cases}
  \end{align*}
  Докажете, че ако $f_\Gamma \df \lfp(\Gamma)$, то
  \[(\forall a,b\in\A)[h(f_\Gamma(a,b)) = f_\Gamma(a,h(b))].\]
\end{problem}
\ifhints
\begin{hint}
  Разгледайте непрекъснатото свойство
  \[P(g) \df (\forall a,b\in\A)[h(g(a,b)) = g(a,h(b))].\]
\end{hint}
\fi

\begin{problem}
  Нека $p \in \Cont{\A}{\Nat_\bot}$ и $h \in \Cont{\A}{\A}$, като $p$ е точна, т.е. $p(\bot) = \bot$.
  Да разгледаме 
  \[\Gamma \in \Cont{\Cont{\A}{\A}}{\Cont{\A}{\A}},\] 
  където:
  \begin{align*}
    \Gamma(f)(x) =
    \begin{cases}
      x, & p(x) = 0\\
      f(f(h(x))), & p(x) \in \Nat^+\\
      \bot, & p(x) = \bot.
    \end{cases}
  \end{align*}
  Докажете, че ако $f_\Gamma \df \lfp(\Gamma)$, то
  \[(\forall a\in\A)[f_\Gamma(f_\Gamma(a)) = f_\Gamma(a)].\]
\end{problem}
\ifhints
\begin{hint}
  Разгледайте непрекъснатото свойство
  \[P(g) \df (\forall a \in \A)[f_\Gamma(g(a)) = g(a)].\]
\end{hint}
\fi

\begin{problem}
  Нека $p \in \Cont{\A}{\Nat_\bot}$ и $h,k \in \Cont{\A}{\A}$, като $h$ е точна, т.е. $h(\bot) = \bot$.
  Да разгледаме $\Gamma_{1,2} \in \Cont{\Cont{\A\times\A}{\A}}{\Cont{\A\times\A}{\A}}$, където:
  \begin{align*}
    & \Gamma_1(f)(x,y) =
    \begin{cases}
      y, & p(x) = 0\\
      h(f(k(x),y)), & p(x) \in \Nat^+\\
      \bot, & p(x) = \bot;\\
    \end{cases}\\
   & \Gamma_2(f)(x,y) =
    \begin{cases}
      y, & p(x) = 0\\
      f(k(x),h(y)), & p(x) \in \Nat^+\\
      \bot, & p(x) = \bot;
    \end{cases}
  \end{align*}
  Докажете, че ако $f_1 \df \lfp(\Gamma_1)$ и $f_2 = \lfp(\Gamma_2)$, то
  $f_1 = f_2$.
\end{problem}
\ifhints
\begin{hint}
  Разгледайте непрекъснатото изображение $\Delta$, където
  \[\Delta(f,g) = \pair{\Gamma_1(f),\Gamma_2(g)}.\]
  Разгледайте свойството:
  \[P(f,g) \dff f = g\ \&\ (\forall a,b \in \A)[h(f(a,b))) = f(a,h(b))].\]
  Първо трябва да се съобрази, че това свойство е непрекъснато, което не е трудно.
  Ясно е, че $P(\bot,\bot)$.
  Докажете, че $P(f,g) \implies P(\Delta(f,g))$.
\end{hint}
\fi


%%% Local Variables:
%%% mode: latex
%%% TeX-master: "../sep"
%%% End:



% \begin{problem}
%   Операторът $\Gamma:\mathcal{F}_2\rightarrow \mathcal{F}_2$ действа по правилото:
%   \begin{equation*}
%     \Gamma(f)(x,y)\simeq 
%     \begin{cases} 
%       1, & \text{ ако } x + y \text{ е просто},\\
%       f(x+y,y)+1, & \text{ иначе.}
%     \end{cases}
%   \end{equation*}
%   Да се докаже, че:
%   \begin{enumerate}[a)]
%   \item
%     операторът $\Gamma$ е компактен.
%   \item 
%     ако $f_{\Gamma}$ е най-малката неподвижна точка на $\Gamma$, то:
%     \begin{equation*}
%       (\forall x,y \in \Nat)[!f_{\Gamma}(x,y) \Rightarrow x + y.f_\Gamma(x,y) \text{ е просто}].
%     \end{equation*}
%   \end{enumerate}
% \end{problem}
% \begin{solution}
%   \begin{enumerate}[a)]
%   \item
%     Да се убедим, че $\Gamma$ е компактен.
%     \begin{itemize}
%     \item 
%       Първо да видим, че $\Gamma$ е монотонен.
%       Нека $f \subseteq g$. Ще докажем, че $\Gamma(f) \subseteq \Gamma(g)$, т.е.
%       \[(\forall x,y,z\in\Nat)[\Gamma(f)(x,y) \simeq z\ \rightarrow\ \Gamma(g)(x,y) \simeq z].\]
%       И така, нека $\Gamma(f)(x,y) \simeq z$. Гледайки дефиницията на $\Gamma$, трябва да разгледаме два случая:
%       \begin{itemize}
%       \item 
%         ако $x+y$ е просто число, то по дефиницията на $\Gamma$,
%         \[\Gamma(f)(x,y) \simeq 1 \simeq \Gamma(g)(x,y).\]
%       \item
%         ако $x+y$ не е просто число, то 
%         \[\Gamma(f)(x,y) \simeq f(x+y,y)+1 \simeq z.\]
%         Това означава, че съществува число $u$, такова че $f(x+y,y) \simeq u$ и $z = u+1$.
%         Понеже $f \subseteq g$, то $g(x+y,y) \simeq u$.
%         Тогава 
%         \begin{align*}
%           \Gamma(g)(x,y) & \simeq g(x+y,y) + 1 & (\text{от деф. на }\Gamma)\\
%           & \simeq u+1 & (\text{защото }f \subseteq g)\\
%           & \simeq z.
%         \end{align*}
%       \end{itemize}
%       За всички възможни двойки $x$, $y$ доказахме, че ако $!\Gamma(f)(x,y)\simeq z$, то
%       $\Gamma(g)(x,y) \simeq z$.
%       Следователно, $\Gamma$ е монотонен.
%     \item
%       За втората част, нека $\Gamma(f)(x,y) \simeq z$, за някои $f \in \F_2$ и $x,y,z \in \Nat$.
%       Ще докажем, че съществува крайна функция $\theta \subseteq f$, за която $\Gamma(\theta)(x,y) \simeq z$.
%       За целта ще разгледаме два случая за $x$ и $y$.
%       \begin{itemize}
%       \item 
%         $x+y$ е просто число. Тогава $\Gamma(f)(x,y) \simeq 1$. 
%         Да вземем крайната функция $\theta = \emptyset^{(2)}$.
%         Очевидно е, че $\Gamma(\emptyset^{(2)})(x,y) \simeq 1$.
%       \item
%         $x+y$ не е просто число. Тогава 
%         \[\Gamma(f)(x,y) \simeq f(x+y,y)+1 \simeq z.\]
%         Да положим $u \simeq f(x+y,y)$. Тогава $z = u+1$.
%         В този случай, избираме крайната функция $\theta \subseteq f$ да бъде такава, че
%         \begin{align*}
%           \theta(a,b) \simeq 
%           \begin{cases}
%             u, & a = x+y\ \&\ b = y\\
%             \neg!, & \text{ иначе}.
%           \end{cases}
%         \end{align*}
%         Тогава 
%         \begin{align*}
%           \Gamma(\theta)(x,y) & \simeq \theta(x+y,y) + 1 & (\text{от деф. на }\theta)\\
%           & \simeq u + 1 & (\text{от деф. на }\theta)\\
%           & \simeq z.
%         \end{align*}
%       \end{itemize}
%       Така видяхме, че във всички случаи за $x$ и $y$  съществува крайна $\theta \subseteq f$, 
%       за която $\Gamma(\theta)(x,y) \simeq z$.
%     \end{itemize}
%     От всичко по-горе следва, че операторът $\Gamma$ е компактен.
%   \item
%     Да дефинираме свойството $P$ като
%     \[P(f)\ \iff\ (\forall x,y \in \Nat)[!f(x,y) \Rightarrow x + y.f(x,y) \text{ е просто}].\]
%     Това е свойство от тип частична коректност, защото ако положим
%     \begin{align*}
%       I(x,y) & \equiv\ x,y\in\Nat,\\
%       O(x,y,r) & \equiv\ x + yr\text{ е просто число},
%     \end{align*}
%     то можем да представим $P$ като
%     \[P(f)\ \equiv\ (\forall x,y)[I(x,y)\ \&\ !f(x,y)\ \Rightarrow\ O(x,y,f(x,y))].\]
%     Щом $P$ е от тип частична коректност, то $P$ е непрекъснато свойство.
%     Понеже $\Gamma$ е компактен оператор, а $P$ е непрекъснато, можем да приложим индукционното
%     правило на Скот. Така ще докажем, че $P(f_\Gamma)$.
%     \begin{itemize}
%     \item 
%       \marginpar{няма $x,y$, за които $!\emptyset^{(2)}(x,y)$}
%       $P(\emptyset^{(2)})$ е изпълнено, защото лява страна на импликацията в дефиницията на $P$
%       винаги е неистина и следователно импликацията винаги е истина.
%     \item
%       Нека приемем, че за някое $f \in \F_2$ е изпълнено $P(f)$.
%       Ще докажем, че $P(\Gamma(f))$, т.е.
%       \[(\forall x,y \in \Nat)[!\Gamma(f)(x,y) \Rightarrow x + y.\Gamma(f)(x,y) \text{ е просто}]\]
%       Нека $!\Gamma(f)(x,y)$. Отново ще разгледаме два случая за $x$ и $y$.
%       \begin{itemize}
%       \item 
%         $x+y$ е просто число. Тогава от дефиницията на $\Gamma$ имаме, че $\Gamma(f)(x,y) \simeq 1$
%         и е ясно, че \[x + y.\Gamma(f)(x,y) \simeq x+y.1 = x+y\] е просто число.
%       \item
%         $x+y$ не е просто число. Тогава $\Gamma(f)(x,y) \simeq f(x+y,y)+1$.
%         Да положим $u \simeq f(x+y,y)$. Получаваме, че:
%         \begin{align*}
%           x + y.\Gamma(f)(x,y) & \simeq x+y.(f(x+y,y)+1) & (\text{от деф. на }\Gamma)\\
%           & \simeq x+y.(u+1) \\
%           & \simeq (x+y) + y.u
%         \end{align*}
       
%         Сега използваме, че $P(f)$ е изпълнено. Тогава:
%         \[!f(x+y,y) \simeq u\ \Rightarrow\ (x+y) + y.u+ \text{ е просто число}. \]
%         Заключаваме, че 
%         \[x + y.\Gamma(f)(x,y) \simeq (x+y) + y.u\]
%         е просто число.
%       \end{itemize}      
%     \end{itemize}
%     Разгледахме всички случаи за $x$ и $y$, и следователно, $P(\Gamma(f))$ е изпълнено.    
%     От индукционното правило на Скот получаваме, че $P(f_\Gamma)$, което ни дава точно това, което 
%     трябваше да докажем.
%   \end{enumerate}
% \end{solution}

% \begin{problem}
%   Да рагледаме оператора $\Gamma:\F_3 \to \F_3$, където:
%   \begin{align*}
%     \Gamma(f)(x,y,z) \simeq 
%     \begin{cases}
%       y, & x = 0\\
%       f(x-1, y+2z, y), & x > 0.
%     \end{cases}
%   \end{align*}
%   Докажете, че 
%   \[(\forall x,u\in\Nat)[u\geq 1\ \&\ !f_\Gamma(x,2^u,2^{u-1})\ \Rightarrow\ f_\Gamma(x,2^u,2^{u-1}) \simeq 2^{x+u}].\]
% \end{problem}
% \begin{hint}
%   Лесно се съобразява, че $\Gamma$ е компактен (непрекъснат) оператор.
%   Целта ни е да дефинираме непрекъснато свойство $P$, за което да докажем с индукционното
%   правило на Скот, че $P(f_\Gamma)$.
%   Разгледайте правилото:
%   \begin{align*}
%     P(f)\ \equiv\ (\forall x,y,z\in\Nat)[& (\exists u \geq 1)[y = 2^u\ \&\ z = 2^{u-1}]\ \&\ !f(x,y,z) \Rightarrow\ \\
%     & (\exists u \geq 1)[y = 2^u\ \&\ z = 2^{u-1}\ \&\ f(x,y,z) \simeq 2^{x+u}]].
%   \end{align*}
%   $P$ е свойство от тип частична коректност, защото използвайки предикатите:
%   \begin{align*}
%     I(x,y,z) & \equiv x,y,z\in\Nat\ \&\ (\exists u \geq 1)[y = 2^u\ \&\ z = 2^{u-1}],\\
%     O(x,y,z,r) & \equiv (\exists u \geq 1)[y = 2^u\ \&\ z = 2^{u-1}\ \&\ r = 2^{x+u}],
%   \end{align*}
%   можем да представим свойството $P$ по следния начин:
%   \[P(f) \equiv  (\forall x,y,z)[I(x,y,z)\ \&\ !f(x,y,z) \Rightarrow\ O(x,y,z,f(x,y,z))].\]
  
%   \marginpar{Очевидно е, че $P(\emptyset^{(3)})$}
%   Да приемем, че имаме $P(f)$. Ще докажем $P(\Gamma(f))$.
%   Нека $x,y,z\in\Nat$ са такива, че $!\Gamma(f)(x,y,z)$ и да фиксираме $u \geq 1$, за което $y = 2^u$, $z = 2^{u-1}$.
%   Ще докажем, че $\Gamma(f)(x,y,z) \simeq 2^{x+u}$.
%   Според дефиницията на $\Gamma$, трябва да разгледаме два случая.
%   \begin{itemize}
%   \item 
%     $x = 0$. Тогава $\Gamma(x,y,z) \simeq y = 2^{u+0}$.
%   \item
%     $x > 0$. Тогава $\Gamma(x,y,z) \simeq f(x-1,y+2z,y)$.
%     Понеже $y = 2^u$ и $z = 2^{u-1}$, $y+2z = 2^u+2.2^{u-1} = 2^{u+1}$.
%     Имаме, че $I(x-1,y+2z,y)$ е истина и $!f(x-1,y+2z,y)$.
%     От $P(f)$ следва, че $O(x-1,y+2z,y,f(x-1,y+2z,y)$, т.е. 
%     \[y+2z = 2^{u+1}\ \&\ y = 2^{(u+1)-1}\ \&\ f(x-1,y+2z,y) \simeq 2^{(x-1)+(u+1)}.\]
%     Като обединим всичко, което знаем до момента, получаваме:
%     \begin{align*}
%       \Gamma(f)(x,y,z) & \simeq f(x-1,y+2z,y) & (\text{от деф. на }\Gamma)\\
%       & \simeq f(x-1,2^{u+1},2^{u}) & (y+2z = 2^{u+1},y=2^{(u+1)-1})\\
%       & \simeq 2^{(x-1)+(u+1)} & (\text{от }P(f))\\
%       & = 2^{x+u}
%     \end{align*}
%     Заключаваме, че $O(x,y,z,\Gamma(f)(x,y,z))$, откъдето следва, че $P(\Gamma(f))$.
%   \end{itemize}
%   От правилото на Скот следва, че $P(f_\Gamma)$.
% \end{hint}




% \begin{problem}
%   Операторът $\Gamma:\F_1 \to \F_1$ е зададен с равенството:
%   \begin{align*}
%     \Gamma(f)(x) \simeq
%     \begin{cases}
%       \sqrt{x}, & \text{ ако } x \text{ е точен квадрат}\\
%       f(f(x(x+5))), & \text{ иначе}.
%     \end{cases}
%   \end{align*}
%   Докажете, че:
%   \begin{enumerate}[a)]
%   \item 
%     операторът $\Gamma$ е компактен;
%   \item
%     $(\forall x\in\Nat)[!f_\Gamma(x)\ \&\ x\text{ не е точен квадрат}\ \Rightarrow\ f_\Gamma(x) > \sqrt{x}]$.
%   \end{enumerate}
% \end{problem}

% \begin{problem}
%   Нека $g:\Nat \to \Nat$ е тотална функция, за която $(\forall x\in\Nat)[g(x) \leq x]$.
%   Операторът $\Gamma:\F_2 \to \F_2$ е зададен с равенството:
%   \begin{align*}
%     \Gamma(f)(x,y) = 
%     \begin{cases}
%       g(y), & \text{ ако } x = 0\\
%       f(f(x-1,g(y-1)), y-1)+ 1, & \text{ иначе}.
%     \end{cases}
%   \end{align*}
%   Докажете, че:
%   \begin{enumerate}[a)]
%   \item 
%     операторът $\Gamma$ е компактен;
%   \item
%     $(\forall x,y\in\Nat)[!f_\Gamma(x,y)\ \Rightarrow\ f_\Gamma(x,y) \leq \max(x,y)]$.
%   \end{enumerate}
% \end{problem}

% \begin{problem}
%   Нека $g:\Nat \to \Nat$ е тотална функция, за която $(\forall x\in\Nat)[g(x) \leq x]$.
%   Операторът $\Gamma:\F_2 \to \F_2$ е зададен с равенството:
%   \begin{align*}
%     \Gamma(f)(x,y) = 
%     \begin{cases}
%       g(y), & \text{ ако } x = 0\\
%       f(x-1, f(x-1, g(y-1))) + 1, & \text{ иначе}.
%     \end{cases}
%   \end{align*}
%   Докажете, че:
%   \begin{enumerate}[a)]
%   \item 
%     операторът $\Gamma$ е компактен;
%   \item
%     $(\forall x,y\in\Nat)[!f_\Gamma(x,y)\ \Rightarrow\ f_\Gamma(x,y) > \max(x,y)]$.
%   \end{enumerate}
% \end{problem}

\begin{problem}
  Да разгледаме изображението $\Gamma \in \Cont{\F_2}{\F_2}$, където:
  \begin{align*}
    \Gamma(f)(x,y) \simeq
    \begin{cases}
      y, & \text{ ако } y\vert x\\
      f(f(x,y+1)), x), & \text{ иначе}.
    \end{cases}
  \end{align*}
  Докажете, че
  % \begin{enumerate}[a)]
  % \item 
  %   операторът $\Gamma$ е компактен;
  % \item
  \[(\forall x)(\forall y)[!f_\Gamma(x,y)\ \&\ y\not| x\ \Rightarrow\ f_\Gamma(x,y)\ \vert\ x].\]
  % \end{enumerate}  
\end{problem}

\begin{problem}
  Да разгледаме $\Gamma \in \Mapping{\F_1}{\F_1}$, където
  \begin{align*}
    \Gamma(f)(x) \simeq
    \begin{cases}
      1, & \text{ ако } x \leq 1\\
      x/2, & \text{ ако } 2\vert x\ \&\ x > 1\\
      f(f(3\lfloor{x/2}\rfloor+2)), & \text{ иначе}.
    \end{cases}
  \end{align*}
  Докажете, че $\Gamma$ е непрекъснато изображение и ако $f_\Gamma \df \lfp(\Gamma)$, то 
  докажете, че
  \[(\forall x)[!f_\Gamma(x)\ \&\ x > 1\ \Rightarrow\ f_\Gamma(x)\ \leq x/2].\]
\end{problem}

% \begin{problem}
%   Операторът $\Gamma:\F_2 \to \F_2$ е зададен с равенството:
%   \begin{align*}
%     \Gamma(f)(x,y) \simeq
%     \begin{cases}
%       y, & x = 0\\
%       f(x-1,2) + y, & x \equiv 1\ (\bmod\ 2)\\
%       f(\frac{x}{2},f(\frac{x}{2},y)), & \text{ иначе}.
%     \end{cases}
%   \end{align*}
%   Докажете, че:
%   \begin{enumerate}[a)]
%   \item 
%     операторът $\Gamma$ е компактен;
%   \item
%     $(\forall x\in\Nat)[!f_\Gamma(x,0)\ \Rightarrow\ f_\Gamma(x,0) \simeq 2x]$.
%   \end{enumerate}  
% \end{problem}
% \begin{hint}
%   Разгледайте свойството
%   \[P(f) \equiv (\forall x,y\in\Nat)[!f(x,y)\ \Rightarrow\ f(x,y) \simeq 2x+y].\]
% \end{hint}

% \begin{problem}
%   Да разгледаме оператора $\Gamma:\mathcal{F}_1\to \mathcal{F}_1$, където:
%   \begin{align*}
%     \Gamma(f)(x) \simeq 
%     \begin{cases}
%       x/3, & \text{ ако }x \equiv 0\ (\bmod\ 3)\\
%       f(3f(x+1)\dotminus 1), & \text{ ако }x \equiv 1\ (\bmod\ 3)\\
%       f(3f(2x-1)+1), & \text{ ако }x \equiv 2\ (\bmod\ 3),
%     \end{cases}
%   \end{align*}
%   където $x\dotminus y = 0$, ако $x < y$, и $x \dotminus y = x-y$, ако $x \geq y$.
%   \begin{enumerate}[a)]
%   \item 
%     Докажете, че операторът $\Gamma$ е компактен.
%   \item
%     Докажете, че
%     $(\forall x\in\Nat)[!f_\Gamma(x) \implies f_\Gamma(x) \leq x/3]$,
%     където с $f_\Gamma$ означаваме най-малката неподвижна точка на оператора $\Gamma$.
%   \end{enumerate}
% \end{problem}
% \begin{hint}
%   \begin{enumerate}[1)]
%   \item 
%     Най-лесно се решава задачата като намерете явния вид на $f_\Gamma$ с теоремата на Кнастер-Тарски.
%     Оказва се, че 
%     \begin{align*}
%       f_\Gamma(x) \simeq 
%       \begin{cases}
%         x/3, & x \equiv 0\ (\bmod\ 3)\\
%         \neg!, & \text{ иначе}.
%       \end{cases}
%     \end{align*}
    
%     Знаем, че $f_0 = \emptyset^{(1)}$, $f_{i+1} = \Gamma(f_i)$ и $f_\Gamma = \bigcup_i f_i$.
%     \begin{align*}
%       f_1(x) \simeq \Gamma(f_0)(x) \simeq\ & \Gamma(\emptyset^{(1)})(x) \simeq 
%       \begin{cases}
%         x/3, & x \equiv 0\ (\bmod\ 3)\\
%         \neg!, & \text{иначе}\\
%       \end{cases}\\
%       f_2(x) \simeq \Gamma(f_1)(x) \simeq &
%       \begin{cases}
%         x/3, & x \equiv 0\ (\bmod\ 3)\\
%         f_1(3f_1(x+1)\dotminus 1), & x \equiv 1\ (\bmod\ 3)\\
%         f_1(3f_1(2x-1) + 1), & x \equiv 2\ (\bmod\ 3)\\
%       \end{cases}\\
%       \simeq &
%       \begin{cases}
%         x/3, & x \equiv 0\ (\bmod\ 3)\\
%         \neg!, & x \equiv 1\ (\bmod\ 3)\\
%         f_1(3\frac{2x-1}{3} + 1), & x \equiv 2\ (\bmod\ 3)\\
%       \end{cases}\\
%       \simeq &
%       \begin{cases}
%         x/3, & x \equiv 0\ (\bmod\ 3)\\
%         \neg!, & x \equiv 1\ (\bmod\ 3)\\
%         f_1(2x), & x \equiv 2\ (\bmod\ 3)\\
%       \end{cases}\\
%       \simeq &
%       \begin{cases}
%         x/3, & x \equiv 0\ (\bmod\ 3)\\
%         \neg!, & \text{иначе}.
%       \end{cases}
%     \end{align*}
%     Виждаме, че $f_1 = f_2$. 
%     Заключаваме, че за всяко $k \geq 1$, $f_k = f_1$.
%     Тогава $f_\Gamma = f_1$.
%   \item
%     Ако искате да използвате правилото на Скот, възможно е да разгледате непрекъснатото свойство
%     \begin{align*}
%       P(f) \equiv\ & (\forall x\in\Nat)[x \equiv 0\ (\bmod\ 3)\ \&\ !f(x) \implies f(x) \simeq x/3]\ \&\ \\
%       & (\forall x\in\Nat)[x \equiv 1\ (\bmod\ 3)\ \&\ !f(x) \implies f(x) \leq x/12]\ \&\ \\
%       & (\forall x\in\Nat)[x \equiv 2\ (\bmod\ 3)\ \&\ !f(x) \implies f(x) \leq x/6].
%     \end{align*}    
%   \end{enumerate}
% \end{hint}

% \begin{problem}
%   Нека предварително имаме дадена частичната функция $h \in \F_2$.
%   Операторът $\Gamma:\F_2 \to \F_2$ е зададен с равенството:
%   \begin{align*}
%     \Gamma(f)(x,y) \simeq
%     \begin{cases}
%       0, & h(x,y) \simeq 0\\
%       f(x, y+1) + 1, & h(x,y) > 0\\
%       \neg !, & \neg ! h(x,y)
%     \end{cases}
%   \end{align*}  
%   Докажете, че 
%   \[(\forall x,y,z\in\Nat)[!f_\Gamma(x,y) \simeq z \implies h(x, y+z) \simeq 0\ \&\ (\forall t < z)[h(x,y+t) > 0]]\]
% \end{problem}

% \newpage

\section{Задачи}

\begin{problem}
  Да фиксираме $a_0 \in \A$ и да разгледаме $P \subseteq \Cont{\A}{\A}$, където
  \[P(f) \dfff \lfp(f) = a_0.\]
  Проверете дали $P$ е непрекъснато свойство.
\end{problem}


\begin{problem}
  Да разгледаме $\Gamma \in \Mapping{\F_1}{\F_1}$, където
  \begin{align*}
    \Gamma(f)(x,y) \simeq
    \begin{cases}
      3.f(\sqrt{x},y) + 2, & \text{ ако $x$ е точен квадрат}\\
      y, & \text{ ако $x$ не е точен квадрат}\\
    \end{cases}
  \end{align*}
  Съобразете, че $\Gamma$ е непрекъснато изображение.
  Нека $f_{\Gamma} \df \lfp(\Gamma)$. Докажете, че:
  \[(\forall x)(\forall y)[3.f_\Gamma(x,y) + 2 \simeq f_\Gamma(x,3y+2)].\]
\end{problem}
\ifhints
\begin{hint}
  \begin{itemize}
  \item 
    Нека първо да разгледаме операторите $\Gamma_1$ и $\Gamma_2$, където
    \begin{align*}
      & \Delta_1(f) \df 3f(x,y)+2\\
      & \Delta_2(f) \df f(x,3y+2).
    \end{align*}
    Съобразете, че те са непрекъснати.
  \item
    От \Prop{continuous-property} следва, че свойството 
    \[P(f) \dfff \Delta_1(f) = \Delta_2(f)\]
    е непрекъснато.
  \item
    Използвайте правилото на Скот върху $P$ 
    за да докажете, че $P(f_\Gamma)$.
    Това означава, че $3f_\Gamma(x,y) + 2 \simeq f_\Gamma(x,3y+2)$ за всяко $x,y \in \Nat$.
  \end{itemize}  
\end{hint}
\fi

\begin{problem}
  Да разгледаме изображението $\Gamma \in \Mapping{\F_1}{\F_1}$, където:
  \begin{align*}
    \Gamma(f)(x,y) \simeq
    \begin{cases}
      (f(\sqrt{x},y))^2, & \text{ ако $x$ е точен квадрат}\\
      y, & \text{ иначе }
    \end{cases}
  \end{align*}
  Съобразете защо $\Gamma$ е непрекъснато изображение.
  Нека $f_{\Gamma} \df \lfp(\Gamma)$. Докажете, че
  \[(\forall x)(\forall y)[(f_\Gamma(x,y))^2 \simeq f_\Gamma(x,y^2)].\]
\end{problem}

\begin{problem}
  Нека $p_0,p_1,p_2\dots\ $ е редицата от всички прости числа в нарастващ ред.
  Да разгледаме $\Gamma \in \Mapping{\F_3}{\F_3}$, където:
  \begin{align*}
    \Gamma(f)(x,y,z) \simeq
    \begin{cases}
      x^xy, & \text{ ако }p_z = x\\
      f(x+x,y,z+2), & \text{ ако }p_z = x.
    \end{cases}
  \end{align*}
  Съобразете, че $\Gamma$ е непрекъснато изображение. 
  Нека $f_{\Gamma} \df \lfp(\Gamma)$. Докажете, че
  \[(\forall x)(\forall y)(\forall z)[!f_{\Gamma}(x,y,z) \implies (\exists\text{ просто число }p)[p \geq x\ \&\ p^py\ |\ f_\Gamma(x,y,z)].\]
\end{problem}

\begin{problem}
  Да разгледаме изображението $\Gamma \in \Mapping{\F_3}{\F_3}$, където:
  \begin{align*}
    \Gamma(f)(x,y,z) \simeq
    \begin{cases}
      z, & y = 0\ \&\ x,z\in\Nat\\
      f(x,y-1, xy+z), & y > 0\ \&\ x,z \in \Nat.
    \end{cases}
  \end{align*}
  Съобразете, че $\Gamma$ е непрекъснато изображение и ако $f_\Gamma \df \lfp(\Gamma)$, то
  \[(\forall x)(\forall y)[f_\Gamma(x,y,0) \simeq \frac{xy(y+1)}{2}].\]
\end{problem}
\begin{hint}
  Да разгледаме свойството над ествествените числа
  \[P(y) \dfff (\forall x,z\in\Nat)[f_\Gamma(x,y,z) \simeq \frac{xy(y+1)}{2} + z].\]
  Докажете с математическа индукция по $y \in \Nat$, че $(\forall y\in\Nat)[P(y)]$.
  % \begin{itemize}
  % \item 
  %   Нека $y = 0$. Тогава за произволни $x$ и $z$,
  %   \begin{align*}
  %     f_\Gamma(x,0,z) & = \Gamma(f_\Gamma)(x,0,z) \\
  %     & = z.
  %   \end{align*}
  % \item
  %   Нека $y > 0$. Тогава 
  %   \begin{align*}
  %     f_\Gamma(x,y,z) & = \Gamma(f_\Gamma)(x,y,z)\\
  %                     & = f_\Gamma(x,y-1,xy+z) & (\text{от деф. на }\Gamma)\\
  %                     & = \frac{x(y-1)(y-1+1)}{2} + xy + z & (\text{от И.П. за }y-1)\\
  %                     & = \frac{xy(y-1)+2xy}{2} + z \\
  %                     & = \frac{xy(y+1)}{2} + z.
  %   \end{align*}
  % \end{itemize}
\end{hint}

% \begin{hint}
%   \marginpar{Ясно е, че $\Gamma$ е компактен}
%   Да разгледаме свойството
%   \[P(f)\ \dfff\ (\forall x,y,z\in\Nat_\bot)[f(x,y,z) \neq \bot\ \to\ f(x,y,z) = \frac{xy(y+1)}{2}+z].\]
%   Лесно се вижда, че това е свойство от тип частична коректност, защото ако положим
%   \begin{align*}
%     I(x,y,z) & \dfff\ \texttt{True},\\
%     O(x,y,z,r) & \dfff\ r = \frac{xy(y+1)}{2},
%   \end{align*}
%   то можем да представим $P$ във вида:
%   \[P(f)\ \equiv\ (\forall x,y,z\in\Nat_\bot)[I(x,y,z)\ \&\ f(x,y,z) \neq \bot\ \to\ O(x,y,z,f(x,y,z))].\]
  
%   Да приложим правилото на Скот.
%   \marginpar{Очевидно е, че $P(\Omega^{(3)})$}
%   Нека е изпълнено $P(f)$. Ще докажем, че $P(\Gamma(f))$.
%   И така, нека да разгледаме елементи $x,y,z \in \Nat_\bot$, за които $\Gamma(f)(x,y,z) \neq \bot$. 
%   Ясно е от дефиницията на оператора $\Gamma$, че $\bot \not\in \{x,y,z\}$.
%   Ще разгледаме два случая.
%   \begin{itemize}
%   \item 
%     Нека $y = 0$. Тогава $\Gamma(f)(x,y,z) = z = \frac{xy(y+1)}{2} + z$.
%   \item
%     Нека $y > 0$. Тогава 
%     \begin{align*}
%       \Gamma(f)(x,y,z) & = f(x,y-1,xy+z) & (\text{от деф. на }\Gamma)\\
%       & = \frac{x(y-1)(y-1+1)}{2} + xy + z & (\text{от }P(f))\\
%       & = \frac{xy(y-1)+2xy}{2} + z \\
%       & = \frac{xy(y+1)}{2} + z.
%     \end{align*}
%   \end{itemize}
%   Заключаваме, че $P(\Gamma(f))$, откъдето следва, по правилото на Скот, че $P(\lfp(\Gamma))$.
  
%   Накрая, за произволни $x,y\in\Nat$ и $z = 0$, получаваме, че 
%   \[f_\Gamma(x,y,0) = \frac{xy(y+1)}{2}.\]
% \end{hint}

% \begin{remark}
% Да разгледаме програмата на езика \REC:

% \begin{haskellcode}
% h(x,y) = f(x,y,0) where
%   f(x,y,z) = if y == 0 then z
%                else f(x, y-1, x*y + z)
% \end{haskellcode}

% Съобразете, че ние горе на практика доказахме, че
% \[(\forall x,y \in \Nat)[\D_V\val{\vv{h}}(x,y) = \frac{xy(y+1)}{2}].\]
% \end{remark}


% \begin{problem}
%   Нека е дадена програмата на езика хаскел:

%   \begin{minted}[frame=lines,framesep=2mm,baselinestretch=1.2]{haskell}
%     rev :: [a] -> [a]
%     rev x = f(x, []) where 
%       f([], y) = y
%       f(x:xs, y) = f(xs, x:y)
%   \end{minted}

%   \noindent 
%   Докажете, че:
%   \begin{enumerate}[a)]
%   \item 
%     $rev:\Sigma^\star \to \Sigma^\star$ е тотална.
%   \item
%     $(\forall x \in \Sigma^\star)[rev(rev(x)) = x]$.
%   \item
%     $(\forall x \in \Sigma^\star)[rev(x) = x^R]$.
%   \end{enumerate}
% \end{problem}
% \begin{hint}
%   % Ще използваме следното правило:
%   % \begin{prooftree}
%   %   \AxiomC{$P(\varepsilon)$}
%   %   \AxiomC{$(\forall x \in \Sigma^\star)[x\neq\varepsilon\ \&\ P(cdr(x)) \to P(x)]$}
%   %   \RightLabel{\scriptsize(1)}
%   %   \BinaryInfC{$(\forall x\in\Sigma^\star)[P(x)]$}
%   % \end{prooftree}
%   % % \item
%   %   Докажете валидността на правилото $(1)$. % е еквивалентно на структурна индукция върху фундираната наредба
%     % $(\Sigma^\star,\prec)$, където $x \prec y \iff (\exists z\in\Sigma^\star)[z\cdot x = y]$, т.е.
%     % $x$ е суфикс на $y$.
%   Да разгледаме фундираната наредба $L = (\Sigma^\star, \prec)$, където
%   $x \prec y \iff \abs{x} < \abs{y}$.
%   \begin{enumerate}[a)]
%   \item 
%     Да разгледаме свойството 
%     \[P(x) \equiv (\forall y\in \Sigma^\star)[f(x,y)\text{ е дефинирана}].\]
%     Докажете със структурна индукция по $L$, че $(\forall x\in\Sigma^\star)[P(x)]$.
%   \item
%     Да разгледаме свойството 
%     \[P(x) \equiv (\forall y\in \Sigma^\star)[rev(f(x,y)) = f(y,x)].\]
%     Докажете със структурна индукция по $L$, че $(\forall x\in\Sigma^\star)[P(x)]$.
%     Тогава в частния случай $y = \varepsilon$, 
%     \[rev(rev(x)) = rev(f(x,\varepsilon)) = f(\varepsilon,x) = x.\]
%   \item
%     Разгледайте свойството
%     \[P(x) \equiv (\forall y\in\Sigma^\star)[f(x,y) = x^R \cdot y].\]
%   \end{enumerate}
% \end{hint}

% \input{proofs/partial-correctness}



%%% Local Variables:
%%% mode: latex
%%% TeX-master: "../sep"
%%% End:
