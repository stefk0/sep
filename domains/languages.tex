\subsection{Регулярни езици}\index{език!регулярен}

Да фиксираме азбуката $\Sigma = \{a_1,\dots,a_k\}$.
\marginpar{Обърнете внимание, че нямаме звездата на Клини в дефиницията на термовете.}
Следната абстрактна граматика 
\[\tau ::= \emptyset\ |\ \varepsilon\ |\ a \cdot X\ |\ \tau + \tau.\]
описват термовете на езика $REG$, където $a$ означава буква от азбуката $\Sigma$, а $X$ означава променлива. Нека имаме и безкраен набор от променливи $X_0,X_1,\dots$.

Програма на езика езика $REG$ представляват системи от следния вид:
\begin{align*}
  & X_1 = \tau_1[X_1,\dots,X_n]\\
  & \ \vdots\\
  & X_n = \tau_n[X_1,\dots,X_n].
\end{align*}

\begin{example}
  Нека $\Sigma = \{a,b\}$. Да разгледаме програмата $R$:
  \begin{align*}
    & X_1 = a \cdot X_1 + b \cdot X_2\\
    & X_2 = \varepsilon + b \cdot X_1.
  \end{align*}
\end{example}


\subsubsection{Операционна семантика}

С индукция по $\ell$, дефинираме релацията $\tau \Downarrow^\ell \alpha$,
където $\tau$ е терм на езика $REG$ със свободни променливи измежду $X_1,\dots, X_n$.

\marginpar{Тук е важно да се отбележи, че тази дефиниция не можем да я дадем с индукция по построението на термовете.}

\begin{prooftree}
  \AxiomC{}
  \LeftLabel{\scriptsize{(eps)}}
  \UnaryInfC{$\varepsilon \Downarrow^0 \varepsilon$}
\end{prooftree}

\begin{prooftree}
  \AxiomC{$\tau_i \Downarrow^\ell \beta$}
  \AxiomC{$\alpha = a\cdot \beta$}
  \RightLabel{\scriptsize{(concat)}}
  \BinaryInfC{$a \cdot X_i \Downarrow^{\ell+1} \alpha$}
\end{prooftree}

\begin{prooftree}
  \AxiomC{$\tau_1 \Downarrow^\ell \alpha$}
  \LeftLabel{\scriptsize{(left-or)}}
  \UnaryInfC{$\tau_1 + \tau_2 \Downarrow^{\ell+1} \alpha$}
\end{prooftree}

\begin{prooftree}
  \AxiomC{$\tau_2 \Downarrow^\ell \alpha$}
  \RightLabel{\scriptsize{(right-or)}}
  \UnaryInfC{$\tau_1 + \tau_2 \Downarrow^{\ell+1} \alpha$}
\end{prooftree}

Ще пишем $\tau \Downarrow \alpha$ точно тогава, когато съществува $\ell$, за което $\tau \Downarrow^\ell \alpha$.

\begin{example}
  Нека $\Sigma = \{a,b\}$. Да разгледаме програмата $R$, където:
  \begin{align*}
    & X_1 = \overbrace{a \cdot X_1 + b \cdot X_2 + \varepsilon}^{\tau_1[X_1,X_2]}\\
    & X_2 = \underbrace{b \cdot X_2 + \varepsilon}_{\tau_2[X_1,X_2]}.
  \end{align*}
  Съобразете, че за всеки две естествени числа $n$ и $k$ е изпълнено:
  \begin{itemize}
  \item
    $\tau_1 \Downarrow a^k b^n$;
  \item
    $\tau_2 \Downarrow b^n$.
  \end{itemize}
\end{example}


\subsubsection{Денотационна семантика}

\begin{problem}
  Докажете, че $\mathcal{S} = (\mathcal{P}(\Sigma^\star), \subseteq, \emptyset)$ е област на Скот.
\end{problem}

За всеки терм $\tau[X_1,\dots,X_n]$ дефинираме изображението
% \marginpar{Тук за променливите използваме главни букви за да се сещаме, че те приемат стойности множества от думи.}
\[\val{\tau}: \mathcal{P}(\Sigma^\star)^n \to \mathcal{P}(\Sigma^\star)\]
 по следния начин:
\begin{itemize}
\item
    $\val{\emptyset}(L_1,\dots,L_n) = \emptyset$.
\item 
  $\val{\varepsilon}(L_1,\dots,L_n) = \{\varepsilon\}$.
\item 
  $\val{a_i \cdot X_j}(L_1,\dots,L_n) = \{a_i\} \cdot L_j$.
\item
  $\val{\tau_1 + \tau_2}(L_1,\dots,L_n) = \val{\tau_1}(L_1,\dots,L_n) \cup \val{\tau_2}(L_1,\dots,L_n)$.
\end{itemize}

\begin{problem}
  Докажете, че за всеки терм $\tau$ имаме, че $\val{\tau}$ е непрекъснато изображение в областта на Скот
  $\mathcal{S} = ( \mathcal{P}(\Sigma^\star),\subseteq, \emptyset)$.
\end{problem}

\marginpar{Програмата е просто текст. Системата е редица от уравнения на непрекъснати изображения.}

\begin{example}
  Да разгледаме програмата $R$, където:
  \begin{align*}
    & X_1 = \overbrace{b \cdot X_1 + a\cdot X_2}^{\tau_1[X_1,X_2]}\\
    & X_2 = \underbrace{\varepsilon}_{\tau_2[X_1,X_2]}.
  \end{align*}
  Според горната дефиниция на семантика на термове, на програмата $R$ съотвества следната система от непректснати изображения:
  \begin{align*}
    & L = \overbrace{\{b\} \cdot L \cup \{a\} \cdot М}^{\val{\tau_1}(L,М)}\\
    & М = \underbrace{\{\varepsilon\}}_{\val{\tau_2}(L_1,L_2)}.
  \end{align*}
  От \Th{sep:min-solution-system} знаем, че тази система има най-малко решение. Нека да го намерим.
  За тази цел, да дефинираме непрекъснатото изображение $\Gamma:\mathcal{P}(\Sigma^\star)^2 \to \mathcal{P}(\Sigma^\star)^2$ като
  \marginpar{От \ref{} знаем, че $\Gamma$ е непрекъснато изображение.}
  \[\Gamma \df \val{\tau_1} \times \val{\tau_2},\]
  т.е. за произволни езици $L$ и $М$ над азбуката $\Sigma$ е изпълнено, че:
  \[\Gamma(L,M) = (\val{\tau_1}(L,M), \val{\tau_2}(L,M)).\]

  От \hyperref[th:knaster-tarski]{Теоремата на Клини} знаем как можем да намерим най-малката неподвижна точка на $\Gamma$,
  която ще бъде и най-малкото решение на горната система.

  \begin{itemize}
  \item 
    $(L_0,M_0) \df (\emptyset,\emptyset)$;
  \item
    $(L_1,M_1) \df \Gamma(L_0,M_0) = (\val{\tau_1}(L_0,M_0), \val{\tau_2}(L_0,M_0)) = (\emptyset, \{\varepsilon\})$;
  \item
    $(L_2,M_2) \df \Gamma(L_1,M_1) = (\val{\tau_1}(L_1,M_1), \val{\tau_2}(L_1,M_1)) = (\{a\},\{\varepsilon\})$;
  \item
    $(L_3,M_3) \df \Gamma(L_2,M_2) = (\val{\tau_1}(L_2,M_2), \val{\tau_2}(L_2,M_2)) = (\{ba,a\},\{\varepsilon\})$;
  \item
    $(L_4,M_4) \df \Gamma(L_3,M_3) =(\val{\tau_1}(L_3,M_3), \val{\tau_2}(L_3,M_3)) = (\{bba, ba,a\},\{\varepsilon\})$;
  \item
    $(L_5,M_5) \df \Gamma(L_4,M_4) = ( \val{\tau_1}(L_4,M_4), \val{\tau_2}(L_4,M_4)) = (\{bba, bba, ba,a\},\{\varepsilon\})$.
  \end{itemize}
  Лесно се съобразява, че $L_n = \{ b^ka \mid k < n\}$, а $M_n = \{\varepsilon\}$. Тогава
  \[\lfp( \Gamma ) = (\bigcup_n L_n, \bigsqcup_n M_n) = (\{b\}^\star \cdot \{a\}, \{\varepsilon\} ).\]
\end{example}


\begin{problem}
  Докажете, че най-малкото решение на системата 
  \begin{align*}
    & X_1 = a \cdot X_1 + b \cdot X_2 + \varepsilon\\
    & X_2 = b \cdot X_2 + \varepsilon
  \end{align*}
  е двойката $(\{a\}^\star \cdot \{b\}^\star, \{b\}^\star)$.
\end{problem}

\begin{problem}
  Да разгледаме системата от непрекъснати оператори
  \begin{align*}
    & \val{\tau_1}(L_1,\dots,L_n) = L_1\\
    & \ \ \vdots\\
    & \val{\tau_n}(L_1,\dots,L_n) = L_n.
  \end{align*}
  Знаем, че тя притежава най-малко решение $(\hat{L}_1,\dots,\hat{L}_n)$.
  Докажете, че всеки от езиците $\hat{L}_i$ е регулярен.

  Докажете, че всеки регулярен език е елемент от най-малкото решение 
  на някоя система от непрекъснати изображения от горния вид.
\end{problem}

% Да разгледаме програмата $P$
% \begin{align*}
%   & X_1 = \tau_1[X_1,\dots,X_n]\\
%   & \vdots\\
%   & X_n = \tau_n[X_1,\dots,X_n].
% \end{align*}

% \marginpar{Няма как да дефинираме операционна семантика, която да разпознава целия език, защото за това ще са ни необходими безкрайно много стъпки.}

% Тогава можем да дефинираме релацията $\rho \Downarrow^\ell_P \alpha$, за терм $\rho$, който съдържа променливи измежду $X_1,\dots,X_n$,
% по следния начин:
% % \begin{prooftree}
% %   \AxiomC{}
% %   \UnaryInfC{$\emptyset \Downarrow^0_P \emptyset$}
% % \end{prooftree}

% \begin{prooftree}
%   \AxiomC{}
%   \UnaryInfC{$\varepsilon \Downarrow^0_P \varepsilon$}
% \end{prooftree}

% \begin{prooftree}
%   \AxiomC{$\tau_i \Downarrow^\ell_P \alpha$}
%   \UnaryInfC{$a \cdot X_i \Downarrow^{\ell+1}_P a \cdot \alpha$}
% \end{prooftree}

% \begin{prooftree}
%   \AxiomC{$\rho_1 \Downarrow^{\ell_1}_P \alpha_1$}
%   % \AxiomC{$\rho_2 \Downarrow^{\ell_2}_P \alpha_2$}
%   \UnaryInfC{$\rho_1 + \rho_2 \Downarrow^{\ell_1+1}_P \alpha_1$}
% \end{prooftree}

% \begin{prooftree}
%   % \AxiomC{$\rho_1 \Downarrow^{\ell_1}_P \alpha_1$}
%   \AxiomC{$\rho_2 \Downarrow^{\ell_2}_P \alpha_2$}
%   \UnaryInfC{$\rho_1 + \rho_2 \Downarrow^{\ell_2+1}_P \alpha_2$}
% \end{prooftree}


% \begin{problem}
%   Нека $P$ е програма и $\tau$ е терм с променливи измежду тези, които се срещат в $P$.
%   Нека $(L_1,\dots,L_n)$ е най-малкото решение на системата $P$.
%   Докажете, че
%   \[\alpha \in \val{\tau}(L_1,\dots,L_n) \iff \tau \Downarrow_P \alpha.\]
% \end{problem}


\subsubsection{Еквивалентност на денотационната и операционната семантики}

Нека тук да фиксираме една програма $R$
\begin{align*}
  & X_1 = \tau_1[X_1,\dots,X_n]\\
    & \vdots\\
  & X_n = \tau_n[X_1,\dots,X_n].
\end{align*}
Нека $(L_1,\dots,L_n)$ е най-малкото решение на непрекъснатото изображение $\Gamma \df \val{\tau_1}\times \cdots \times \val{\tau_n}$. Целта ни е да докажем, че за всяко $i = 1,\dots,n$ е изпълнено, че:
\[L_i = \{\alpha \in \Sigma^\star \mid \tau_i \Downarrow \alpha\}.\]
Това ще направим на две стъпки.

\marginpar{Да напомним, че $L^0_i = \emptyset$ и $L^{k+1}_i = \val{\tau_i}(L^k_1,\dots,L^k_n)$.}

\begin{lemma}
  За всеки терм $\tau$ с променливи измежду $X_1,\dots,X_n$ и всяко $r$ е изпълнено, че:
  \[\val{\tau}(L^r_1,\dots,L^r_n) \subseteq \{\alpha \in \Sigma^\star \mid \tau \Downarrow \alpha\}.\]
\end{lemma}
\begin{proof}
  Нека да кръстим $Include(r)$ твърдението
  ,,за всеки терм $\tau$ с променливи измежду $X_1,\dots,X_n$ е изпълнено, че:
  \[\val{\tau}(L^r_1,\dots,L^r_n) \subseteq \{\alpha \in \Sigma^\star \mid \tau \Downarrow \alpha\}".\]
  Целта ни е да докажем, че $(\forall r\in\Nat)Include(r)$. Това ще направим с индукция по $r$.
  
  Нека $r = 0$. Тук знаем, че $L^0_i = \emptyset$. Тук ще направим вътрешна индукция по построението на термовете $\tau$ за да докажем, че:
  \begin{equation}
    \label{eq:17}
    \val{\tau}(L^0_1,\dots,L^0_n) \subseteq \{\alpha \in \Sigma^\star \mid \tau \Downarrow \alpha\}
  \end{equation}
  \begin{itemize}
  \item
    Нека $\tau \equiv \emptyset$. Понеже $\val{\tau}(L^0_1,\dots,L^0_n) = \emptyset$, то е ясно, че Свойство~\ref{eq:17} е изпълнено.
  \item
    Нека $\tau \equiv \varepsilon$. Тогава $\val{\tau}(L^0_1,\dots,L^0_n) = \{\varepsilon\}$.
    От правилата на операционната семантика имаме, че $\varepsilon \Downarrow \varepsilon$.
    Заключаваме, че Свойство~\ref{eq:17} е изпълнено.
  \item
    Нека $\tau \equiv a \cdot X_j$, за някой индекс $j$ измежду $1,\dots,n$.
    Тогава $\val{\tau}(L^0_1,\dots,L^0_n) = \{a\}\cdot L^0_j = \{a\} \cdot \emptyset = \emptyset$ и оттук е очевидно, че Свойство~\ref{eq:17} е в сила.
  \item
    Нека $\tau \equiv \rho + \mu$. Тогава $\val{\tau}(L^0_1,\dots,L^0_n) = \val{\rho}(L^0_1,\dots,L^0_n) \cup \val{\mu}(L^0_1,\dots,L^0_n)$. Сега от И.П. за Свойство~\ref{eq:17} получаваме, че
    $\val{\rho}(L^0_1,\dots,L^0_n) \subseteq \{\alpha \in \Sigma^\star \mid \rho \Downarrow \alpha\}$ и
    $\val{\mu}(L^0_1,\dots,L^0_n) \subseteq \{\alpha \in \Sigma^\star \mid \mu \Downarrow \alpha\}$.
    От правилата на операционната семантика имаме, че
    $\{\alpha \in \Sigma^\star \mid \rho \Downarrow \alpha\} \subseteq \{\alpha \in \Sigma^\star \mid \rho \Downarrow \alpha\}$ и 
    $\{\alpha \in \Sigma^\star \mid \mu \Downarrow \alpha\} \subseteq \{\alpha \in \Sigma^\star \mid \mu \Downarrow \alpha\}$.
    Заключаваме, че Свойство~\ref{eq:17} е изпълнено.
  \end{itemize}

  Нека $r > 0$ и да приемем, че е изпълнено $Include(r-1)$. Ще докажем, че е изпълнено $Include(r)$.
  Тук знаем, че $L^r_i = \val{\tau_i}(L^{r-1}_1,\dots,L^{r-1}_n)$.
  Ще направим вътрешна индукция по построението на термовете $\tau$ за да докажем, че:
  \begin{equation}
    \label{eq:18}
    \val{\tau}(L^r_1,\dots,L^r_n) \subseteq \{\alpha \in \Sigma^\star \mid \tau \Downarrow \alpha\}.
  \end{equation}
  \begin{itemize}
  \item
    Нека $\tau \equiv \emptyset$.
  \item
    Нека $\tau \equiv \varepsilon$.
  \item
    Нека $\tau \equiv a \cdot X_j$ за някой индекс $j$ измежду $1,\dots,n$.
    Тогава $\val{\tau}(L^r_1,\dots,L^r_n) = \{a\}\cdot L^r_j$.
    Да напомним, че $L^r_j = \val{\tau_j}(L^{r-1}_1,\dots,L^{r-1}_n)$.
    Понеже $Include(r-1)$ е изпълено, то имаме автоматично, че:
    \[L^r_j = \val{\tau_j}(L^{r-1}_1,\dots,L^{r-1}_n) \subseteq \{\beta \in \Sigma^\star \mid \tau_j \Downarrow \beta\}.\]
    От правилата на операционната семантика имаме, че:
    \begin{prooftree}
      \AxiomC{$\tau_j \Downarrow \beta$}
      \UnaryInfC{$a \cdot X_j \Downarrow a \cdot \beta$}
    \end{prooftree}
    Понеже $\tau \equiv a \cdot X_j$, получаваме следното:
    \begin{equation}
      \label{eq:21}
      \{a\} \cdot \{\beta \in \Sigma^\star \mid \tau_j \Downarrow \beta\} \subseteq \{ \alpha \in \Sigma^\star \mid \tau \Downarrow \alpha\}.
    \end{equation}
    Заключаваме, че:
    \begin{align*}
      \val{\tau}(L^r_1,\dots,L^r_n) & = \{a\} \cdot L^r_j \\
                                    & \subseteq \{a\} \cdot \{\beta \in \Sigma^\star \mid \tau_j \Downarrow \beta\} & \comment\text{от }Include(r-1)\\
      & \subseteq \{ \alpha \in \Sigma^\star \mid \tau \Downarrow \alpha\}. & \comment\text{\ref{eq:21}}
    \end{align*}
  \item
    Нека $\tau \equiv \rho + \mu$. Тук имаме, че
    $\val{\tau}(L^r_1,\dots,L^r_n) = \val{\rho}(L^r_1,\dots,L^r_n) \cup \val{\mu}(L^r_1,\dots,L^r_n)$.
    От вътрешното индукционно предположение, имаме, че
    $\val{\rho}(L^r_1,\dots,L^r_n) \subseteq \{\alpha \in \Sigma^\star \mid \rho \Downarrow \alpha\}$ и
    $\val{\mu}(L^r_1,\dots,L^r_n) \subseteq \{\alpha \in \Sigma^\star \mid \mu \Downarrow \alpha\}$.
    От правилата на операционната семантика имаме, че
    $\{\alpha \in \Sigma^\star \mid \rho \Downarrow \alpha\} \cup \{\alpha \in \Sigma^\star \mid \mu \Downarrow \alpha\} \subseteq \{\alpha \in \Sigma^\star \mid \rho + \mu \Downarrow \alpha\}$.
    Заключаваме, че:
    \begin{align*}
      \val{\tau}(L^r_1,\dots,L^r_n) & = \val{\rho}(L^r_1,\dots,L^r_n) \cup \val{\mu}(L^r_1,\dots,L^r_n)\\
                                    & \subseteq \{\alpha \in \Sigma^\star \mid \rho \Downarrow \alpha\} \cup \{\alpha \in \Sigma^\star \mid \mu \Downarrow \alpha\} \\
                                    & \subseteq \{\alpha \in \Sigma^\star \mid \mu+\rho \Downarrow \alpha\}\\
      & = \{\alpha \in \Sigma^\star \mid \tau \Downarrow \alpha\}.
    \end{align*}
  \end{itemize}
\end{proof}

\begin{corollary}
  Нека $L_1,\dots,L_n$ е най-малкото решение на изображението $\Gamma \df \val{\tau_1}\times\cdots\times\val{\tau_n}$. Тогава за произволен терм $\tau$ с променливи измежду $X_1,\dots,X_n$ е изпълнено, че:
  \[\val{\tau}(L_1,\dots,L_n) \subseteq \{\alpha \in \Sigma^\star \mid \tau \Downarrow \alpha\}.\]
\end{corollary}
\begin{proof}
  Използваме, че $\val{\tau}$ е непрекъснато изображение. Тогава
  \begin{align*}
    \val{\tau}(\bigcup_r L^r_1,\dots,\bigcup_rL^r_n) & = \bigcup_r\val{\tau}(L^r_1,\dots,L^r_n)\\
                                                     & \subseteq \{\alpha \in \Sigma^\star \mid \tau \Downarrow \alpha\}.
  \end{align*}
\end{proof}

\begin{lemma}
  Нека $L_1,\dots,L_n$ е неподвижна точка на $\Gamma \df \val{\tau_1}\times \cdots \times \val{\tau_n}$.
  За всеки терм $\tau$ с променливи измежду $X_1,\dots,X_n$ е  изпълнено, че:
  \[\{\alpha \in \Sigma^\star \mid \tau \Downarrow \alpha\} \subseteq \val{\tau}(L_1,\dots,L_n).\]
\end{lemma}
\begin{proof}
  Да означим с $Include(\ell)$ твърдението ,,за всеки терм $\tau$ с променливи измежду $X_1,\dots,X_n$ е изпълнено, че:
  \[\{\alpha \in \Sigma^\star \mid \tau \Downarrow^\ell \alpha\} \subseteq \val{\tau}(L_1,\dots,L_n)".\]
  Ще докажем, че $(\forall \ell\in\Nat)Include(\ell)$.

  Нека $\ell = 0$. Тогава $\tau = \varepsilon$.

  Нека $\ell > 0$. Тогава имаме няколко случая.
  \begin{itemize}
  \item
    \begin{prooftree}
      \AxiomC{$\tau_j \Downarrow^{\ell-1} \beta$}
      \UnaryInfC{$a \cdot X_j \Downarrow^\ell a \cdot \beta$}
    \end{prooftree}
    Според правилата на операционната семантика имаме, че:
    \begin{equation}
      \label{eq:22}
      \{a\} \cdot \{\beta \in \Sigma^\star \mid \tau_j \Downarrow^{\ell-1} \beta\} = \{\alpha \in \Sigma^\star \mid a \cdot X_j \Downarrow^{\ell} \alpha\}
    \end{equation}
    
    Знаем, че $\val{a \cdot X_j}(L_1,\dots,L_n) = \{a\} \cdot L_j = \{a\} \cdot \val{\tau_j}(L_1,\dots,L_n)$,
    защото $L_1,\dots,L_n$ е неподвижна точка на $\Gamma$.
    От $Include(\ell-1)$ имаме следното:
    \[\{\beta \in \Sigma^\star \mid \tau_j \Downarrow^{\ell-1} \beta\} \subseteq \val{\tau_j}(L_1,\dots,L_n).\]
  \item
    Нека имаме следното:
    \begin{prooftree}
      \AxiomC{$\rho \Downarrow^{\ell-1} \alpha$}
      \UnaryInfC{$\rho + \mu \Downarrow^\ell \alpha$}
    \end{prooftree}

    Знаем, че $\val{\rho+\mu}(L_1,\dots,L_n) = \val{\rho}(L_1,\dots,L_n) \cup \val{}(L_1,\dots,L_n)$.
    Използвайки индукционното предположение, което ни казва, че $Include(\ell-1)$ е изпълнено, получаваме:
    \begin{align*}
      \{\alpha \in \Sigma^\star \mid \rho \Downarrow^{\ell-1} \alpha\} & \subseteq \val{\rho}(L_1,\dots,L_n)\\
                                                                       & \subseteq \val{\rho+\mu}(L_1,\dots,L_n).
    \end{align*}
    
  \item
    \begin{prooftree}
      \AxiomC{$\mu \Downarrow^{\ell-1} \alpha$}
      \UnaryInfC{$\rho + \mu \Downarrow^\ell \alpha$}
    \end{prooftree}
  \end{itemize}
\end{proof}



\begin{theorem}
  Да разгледаме програмата $R$
  \begin{align*}
    & X_1 = \tau_1[X_1,\dots,X_n]\\
    & \vdots\\
    & X_n = \tau_n[X_1,\dots,X_n].
  \end{align*}
  Нека $(L_1,\dots,L_n)$ е най-малкото решение на непрекъснатото изображение $\Gamma \df \val{\tau_1}\times \cdots \times \val{\tau_n}$.
  Тогава за всяко $i = 1,\dots,n$ е изпълнено, че:
  \[L_i = \{\alpha \in \Sigma^\star \mid \tau_i \Downarrow \alpha\}.\]
\end{theorem}
\begin{proof}
  Ще разгледаме двете посоки поотделно.
  Първо ще докажем, че за всяко $i = 1,\dots,n$ е изпълнено, че:
  \[L_i \subseteq \{\alpha \in \Sigma^\star \mid \tau_i \Downarrow \alpha\}.\]
  Да напомним, че $L_i = \bigcup^\infty_{k=0} L^k_i$, където $L^0_i = \emptyset$ и $L^{k+1}_i = \val{\tau_i}(L^k_1,\dots,L^k_n)$.
  Ще докажем с индукция, че за всяко $k$ е изпълнено, че:
  \[L^k_i \subseteq \{\alpha \in \Sigma^\star \mid \tau_i \Downarrow \alpha\}.\]
  \begin{itemize}
  \item
    За $k = 0$ твърдението е очевидно, защото $L^0_i = \emptyset \subseteq \{\alpha \in \Sigma^\star \mid \tau_i \Downarrow \alpha\}$.
  \item
    Нека твърдението е изпълнено за произволен индекс $k$.
  \item
    Ще докажем, че $L^{k+1}_i \subseteq \{\alpha \in \Sigma^\star \mid \tau_i \Downarrow \alpha\}$.
    Понеже $L^{k+1}_i = \val{\tau_i}(L^k_1,\dots,L^k_n)$, то трябва да разгледаме какво представлява терма $\tau_i$.
    \begin{itemize}
    \item
      Ако $\tau_i \equiv \emptyset$, то $L^{k+1}_i = \emptyset$ и всичко е очевидно.
    \item
      Ако $\tau_i \equiv \varepsilon$, то $L^{k+1}_i  = \val{\varepsilon}(L^k_1,\dots,L^k_n) = \{\varepsilon\}$.
      От друга страна, според правилата на операционната семантика имаме, че $\varepsilon \Downarrow \varepsilon$.
      Следователно, $L^{k+1}_i \subseteq \{\alpha \in \Sigma^\star \mid \tau_i \Downarrow \alpha\}$.
    \item
      Ако $\tau_i \equiv a \cdot X_j$, то $L^{k+1}_i = \{a\} \cdot L^k_j$.
      От правилата на операционната семантика имаме следното:
      \begin{prooftree}
        \AxiomC{$\tau_j \Downarrow \beta$}
        \UnaryInfC{$a \cdot X_j \Downarrow a \cdot \beta$}
      \end{prooftree}
      Така получаваме, че:
      \begin{equation}
        \label{eq:11}
        \{a\} \cdot \{\beta \in \Sigma^\star \mid \tau_j \Downarrow \beta\} \subseteq \{ \alpha \in \Sigma^\star \mid a \cdot X_j \Downarrow \alpha\}.
      \end{equation}
      От И.П. имаме следното:
      \[L^k_j \subseteq \{\beta \in \Sigma^\star \mid \tau_j \Downarrow \beta\}.\]
      Оттук следва, че:
      \begin{align*}
        L^{k+1}_i & = \val{\tau_i}(L^k_1,\dots,L^k_n)\\
                  & = \{a\} \cdot L^k_j\\
                  & \subseteq \{a\} \cdot \{\beta \in \Sigma^\star \mid \tau_j \Downarrow \beta\} & \comment\text{от И.П.}\\
                  & \subseteq \{ \alpha \in \Sigma^\star \mid a \cdot X_j \Downarrow \alpha\} & \comment\text{от \ref{eq:11}}\\
                  & = \{ \alpha \in \Sigma^\star \mid \tau_i \Downarrow \alpha\}. & \comment \tau_i \equiv a \cdot X_j
      \end{align*}
    \item
      Ако $\tau \equiv \tau_1 + \tau_2$, то $L^{k+1}_i = \val{\tau_1}(L^k_1,\dots,L^k_n) \cup \val{\tau_2}(L^k_1,\dots,L^k_n)$.
      От правилата на операционната семантика получаваме, че:
      \begin{equation}
        \label{eq:13}
        \{\alpha \in \Sigma^\star \mid \tau_1 \Downarrow \alpha\} \cup \{\alpha \in \Sigma^\star \mid \tau_2 \Downarrow \alpha\} \subseteq \{\alpha \in \Sigma^\star \mid \tau \Downarrow \alpha\}.
      \end{equation}
    \end{itemize}
  \end{itemize}
\end{proof}



\subsection{Безконтекстни езици}\index{език!безконтекстен}

Да фиксираме азбуката $\Sigma = \{a_1,\dots,a_n\}$.
Да дефинираме термове от тип 1 като
\[\tau ::= \varepsilon\ |\ \emptyset\ |\ X\ |\ a\ |\ \tau_1 \cdot \tau_2\ |\ \tau_1 + \tau_2,\]
където $j = 1, \dots,n$, а $X_i$ са изброимо безкрайна редица от променливи.

\subsubsection{Операционна семантика}

\begin{prooftree}
  \AxiomC{}
  \UnaryInfC{$\varepsilon \Downarrow^0 \varepsilon$}
\end{prooftree}

\begin{prooftree}
  \AxiomC{}
  \UnaryInfC{$a \Downarrow^0 a$}
\end{prooftree}

\begin{prooftree}
  \AxiomC{$\tau_i \Downarrow^\ell \alpha$}
  \UnaryInfC{$X_i \Downarrow^{\ell+1} \alpha$}
\end{prooftree}

\begin{prooftree}
  \AxiomC{$\tau_1 \Downarrow^\ell \alpha$}
  \UnaryInfC{$\tau_1 + \tau_2 \Downarrow^{\ell+1} \alpha$}
\end{prooftree}

\begin{prooftree}
  \AxiomC{$\tau_2 \Downarrow^\ell \alpha$}
  \UnaryInfC{$\tau_1 + \tau_2 \Downarrow^{\ell+1} \alpha$}
\end{prooftree}


\begin{prooftree}
  \AxiomC{$\tau_1 \Downarrow^{\ell_1} \alpha_1$}
  \AxiomC{$\tau_2 \Downarrow^{\ell_2} \alpha_2$}
  \BinaryInfC{$\tau_1 \cdot \tau_2 \Downarrow^{\ell_1+\ell_2+1} \alpha_1 \cdot \alpha_2$}
\end{prooftree}
\subsubsection{Денотационна семантика}

За всеки терм $\tau[X_1,\dots,X_n]$ дефинираме оператора 
\[\val{\tau}: (\mathcal{P}(\Sigma^\star))^n \to \mathcal{P}(\Sigma^\star)\]
по следния начин:
\begin{itemize}
\item 
  $\val{X_i}(L_1,\dots,L_n) = L_i$.
\item 
  $\val{a_j}(L_1,\dots,L_n) = \{a_j\}$.
\item 
  $\val{\varepsilon}(L_1,\dots,L_n) = \{\varepsilon\}$.
\item 
  $\val{\emptyset}(L_1,\dots,L_n) = \emptyset$.
\item 
  $\val{\tau_1 \cdot \tau_2}(L_1,\dots,L_n) = \val{\tau_1}(L_1,\dots,L_n) \cdot \val{\tau_2}(L_1,\dots,L_n)$.
\item
  $\val{\tau_1 + \tau_2}(L_1,\dots,L_n) = \val{\tau_1}(L_1,\dots,L_n) \cup \val{\tau_2}(L_1,\dots,L_n)$.
\end{itemize}

\begin{problem}
  Докажете, че за всеки терм $\tau$, $\val{\tau}$ е непрекъснато изображение в областта на Скот
  $\mathcal{S} = ( \mathcal{P}(\Sigma^\star),\subseteq, \emptyset)$.
\end{problem}

\begin{problem}
  Докажете, че $\{a^nb^n \mid n\in \Nat\} = \lfp(\val{\tau})$, където 
  \[\tau[X] \equiv \varepsilon + a \cdot X \cdot b.\]
  С други думи, $\{a^nb^n \mid n \in \Nat\}$ е най-малкото решение на уравнението
  \[X = a \cdot X \cdot b + \varepsilon.\]
\end{problem}

Нека сега да разгледаме термовете $\tau_1[X_1,\dots,X_n], \dots, \tau_n[X_1,\dots,X_n]$.

\begin{problem}
  Да разгледаме системата от непрекъснати оператори
  \begin{align*}
    & \val{\tau_1}(L_1,\dots,L_n) = L_1\\
    & \ \ \vdots\\
    & \val{\tau_n}(L_1,\dots,L_n) = L_n.
  \end{align*}
  Знаем, че тя притежава най-малко решение $(\hat{L}_1,\dots,\hat{L}_n)$.
  Докажете, че всеки от езиците $\hat{L}_i$ е безконтекстен.

  Докажете, че всеки безконтекстен език е елемент от най-малкото решение 
  на някоя система от оператори от горния вид.
\end{problem}

\begin{problem}
  \marginpar{Това е аналог на нормалната форма на Чомски}
  Да дефинираме термове от тип 2 като
  \[\tau ::= a_j\ |\ \varepsilon\ |\ \emptyset\ |\ X_i \cdot X_k\ |\ (\tau_1 + \tau_2),\]
  където $j = 1, \dots,n$, а $X_i$ са изброимо безкрайна редица от променливи.
  Докажете горното твърдение, като замените термовете от тип 1 с тези от тип 2.
\end{problem}

\begin{example}
  Да разгледаме системата
  \begin{align*}
    & X_1 = X_3 \cdot X_2 + \varepsilon\\
    & X_2 = X_1 \cdot X_4\\
    & X_3 = a\\
    & X_4 = b.
  \end{align*}


  % \begin{align*}
  %   & \val{\varepsilon + X_3 \cdot X_2}(L_1, L_2, L_3, L_4) = L_1\\
  %   & \val{X_1 \cdot X_4}(L_1, L_2, L_3, L_4) = L_2\\
  %   & \val{a}(L_1, L_2, L_3, L_4) = L_3\\
  %   & \val{b}(L_1, L_2, L_3, L_4) = L_4\\
  % \end{align*}
  Нека $(\hat{L}_1, \hat{L}_2, \hat{L}_3, \hat{L}_4)$ е най-малкото решение на системата.
  Докажете, че $\hat{L}_1 = \{a^nb^n\mid n \in \Nat\}$ $\hat{L}_2 = \{a^nb^{n+1}\mid n \in \Nat\}$,
  $\hat{L}_3 = \{a\}$ и $\hat{L}_4 = \{b\}$.
\end{example}


%%% Local Variables:
%%% mode: latex
%%% TeX-master: "../sep"
%%% End:
