\begin{definition}\label{def:curry}\index{curry}
  Нека $\A$, $\B$ и $\C$ са области на Скот.
  Изображението 
  \[\curry:\Mapping{\A\times \B}{\C} \to \Mapping{\A}{\Mapping{\B}{\C}},\]
  е дефинирано като
  \[\curry(f)(a)(b) \df f(a,b).\]  
\end{definition}

\begin{proposition}\label{pr:curry}
  \marginpar{\cite[стр. 187]{models-of-computation}}
  Ако $f$ е непрекъснато изображение, то
  $\curry(f)$ е непрекъснато изображение,
  т.е. $\curry(f) \in \Cont{\A}{\Cont{\B}{\C}}$.
  Освен това,
  \[\curry \in \Cont{\Cont{\A\times \B}{\C}}{\Cont{\A}{\Mapping{\B}{\C}}}.\]
\end{proposition}
\begin{proof}
  Първо да фиксираме $f \in \Cont{\A\times\B}{\C}$ и $a \in \A$.
  Ще докажем, че $\curry(f)(a) \in \Cont{\B}{\C}$.
  \marginpar{Ако означим $h \df \curry(f)(a)$, то трябва да докажем, че за произволна верига $\chain{b}{i}$ в $\B$,
  $h(\bigsqcup_i b_i) = \bigsqcup_i\{h(b_i)\}$.}
  Нека фиксираме верига $\chain{b}{i}$ от елементи на $\B$.
  Тогава
  \begin{align*}
    \curry(f)(a)(\bigsqcup_i b_i) & \df f(a, \bigsqcup_i b_i)\\
                                  & = \bigsqcup_i\{f(a,b_i)\} & \comment\text{ $f$ е непр. по втория си аргумент}\\
                                  & = \bigsqcup_i \{\curry(f)(a)(b_i)\}. & \comment\text{опр. на }\curry
  \end{align*}

  Второ, сега пък за фиксирано $f \in \Cont{\A\times\B}{\C}$ трябва да докажем, че
  $\curry(f) \in \Cont{\A}{\Cont{\B}{\C}}$.
  За целта да разгледаме произволна верига $\chain{a}{i}$ от елементи на $\A$.
  Трябва да докажем, че
  \[\curry(f)(\bigsqcup_i a_i) = (\bigsqcup_i \curry(f))(a_i).\]
  \marginpar{Обърнете внимание, че $\curry(f)(a_i)$ образуват верига в $\Cont{\B}{\C}$,
  откъдето следва, че $\bigsqcup_i \{\curry(f)(a_i)\}$ е добре дефиниран елемент.}
  Също така, да напомним, че за произволен елемент $b \in \B$,
  \begin{equation}
    \label{eq:7}
    (\bigsqcup_i \{\curry(f)(a_i)\})(b) \df \bigsqcup_i \{\curry(f)(a_i)(b)\}.
  \end{equation}
  Тогава
  \begin{align*}
    \curry(f)(\bigsqcup_i a_i)(b) & = f(\bigsqcup_i a_i,b) & \comment\text{опр. на }\curry\\
                                  & = \bigsqcup_i f(a_i,b) & \comment\text{ $f$ е непр. по първия аргумент}\\
                                  & = \bigsqcup_i \{\curry(f)(a_i)(b)\} & \comment\text{опр. на }\curry\\
                                  & = (\bigsqcup_i \{\curry(f)(a_i)\})(b). & \comment\text{от (\ref{eq:7})}
  \end{align*}
  Заключаваме, че $\curry(f)(\bigsqcup_i a_i) = (\bigsqcup_i \curry(f))(a_i)$.

  Трето, остава да видим защо за произволна верига $\chain{f}{i}$ от елементи на $\Cont{\A\times\B}{\C}$ е изпълнено, че
  \[\curry(\bigsqcup_i f_i) = \bigsqcup_i \{\curry(f_i)\}.\]
  За произволен елемент $a \in \A$ имаме, че
  \[(\bigsqcup_i \curry(f_i))(a) = \bigsqcup_i \{\curry(f_i)(a)\}.\]
  \marginpar{Ако $h_i \df \curry(f_i)(a)$, то е ясно, че $(\bigsqcup_i h_i)(b) = \bigsqcup_i\{h_i(b)\}$.}
  За произволен елемент $b \in \B$ имаме, че
  \[(\bigsqcup_i \curry(f_i)(a))(b) = \bigsqcup_i \{\curry(f_i)(a)(b)\}.\]
  Комбинирайки предишните две равенства, получаваме, че за произволни $a \in \A$ и $b \in \B$,
  \begin{align*}
    (\bigsqcup_i \curry(f_i))(a)(b) & = \bigsqcup_i\{\curry(f_i)(a)(b)\}\\
                                    & = \bigsqcup_i\{f_i(a,b)\} & \comment\text{от опр. на }\curry\\
                                    & = (\bigsqcup_i f_i)(a,b)\\
                                    & = \curry(\bigsqcup_i f_i)(a)(b). & \comment\text{от опр. на }\curry
  \end{align*}
\end{proof}

\begin{definition}\label{def:uncurry}\index{uncurry}
  Нека $\A$, $\B$ и $\C$ са области на Скот.
  Изображението 
  \[\texttt{uncurry}:\Mapping{\Mapping{\A}{\Mapping{\B}{\C}}}{\Mapping{\A\times \B}{\C}},\]
  е дефинирано като
  \[\texttt{uncurry}(f)(a,b) \df f(a)(b).\]  
\end{definition}

\begin{problem}\label{prob:uncurry}
  % \marginpar{\cite[стр. 187]{models-of-computation}}
  Докажете, че ако $f$ е непрекъснато изображение, то
  $\texttt{uncurry}(f)$ е непрекъснато изображение,
  т.е. $\texttt{uncurry}(f) \in \Cont{\A \times \B}{\C}$.
  Освен това, докажете, че
  \[\texttt{uncurry} \in \Cont{\Cont{\A}{\Cont{\B}{\C}}}{\Cont{\A\times \B}{\C}}.\]
\end{problem}

%%% Local Variables:
%%% mode: latex
%%% TeX-master: "../sep"
%%% End:
