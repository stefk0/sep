\begin{problem}
  \marginpar{\cite[стр. 187]{models-of-computation}}
  Да разгледаме следното изображение
  \[\curry:\Mapping{\A\times \B}{\C} \to \Mapping{\A}{\Mapping{\B}{\C}},\]
  дефинирано като
  \[\curry(f)(a)(b) \dff f(a,b).\]
  Докажете, че ако $f$ е непрекъснато изображение, то
  $\curry(f)$ е непрекъснато изображение,
  т.е. $curry(f) \in \Cont{\A}{\Mapping{\B}{\C}}$.
\end{problem}
\begin{proof}
  Да разгледаме произволна верига $\chain{a}{i}$ от елементи на $\A$.
  Трябва да докажем, че
  \[\curry(f)(\bigsqcup_i a_i) = \bigsqcup_i \curry(f)(a_i).\]
  За целта, да разгледаме произволен елемент $b \in \B$. Тогава
  \begin{align*}
    \curry(f)(\bigsqcup_i a_i)(b) & = f(\bigsqcup_i a_i,b) & \comment\text{опр. на }\curry\\
                                  & = \bigsqcup_i f(a_i,b) & \comment\text{ $f$ е непр. по първия аргумент}\\
                                  & = \bigsqcup_i \{\curry(f)(a_i)(b)\}. & \comment\text{опр. на }\curry
  \end{align*}
\end{proof}

% \begin{hint}
%   Трябва да докажем, че
%   \[\texttt{curry}(\bigsqcup_i f_i)(a)(b) = \bigsqcup_i \{\texttt{curry}(f_i)(a,b)\}.\]
%   \begin{align*}
%     \texttt{curry}(\bigsqcup_i f_i)(a)(b) = (\bigsqcup_i f_i)(a,b)
%   \end{align*}
% \end{hint}


%%% Local Variables:
%%% mode: latex
%%% TeX-master: "../sep"
%%% End:
