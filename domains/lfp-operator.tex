\section{Оператор за най-малка неподвижна точка}

\Stefan{Този раздел може да се сложи по-късно.}

\begin{problem}
  \marginpar{Да напомним, че $f^0(a) = a$ и $f^{n+1}(a) = f(f^n(a))$.}
  Нека $\chain{f}{i}$ е верига от елементи на $\Mon{\A}{\A}$.
  Докажете, че:
  \begin{enumerate}[a)]
  \item
    $\chain{f^n}{i}$ е верига за произволно $n$;
  \item
    ${(f^n_i)}^\infty_{n=0}$ е верига за произволно $i$;
  \item
    ${(\bigsqcup_n f^n_i)}^\infty_{i=0}$ е верига;
  \item
    ${(\bigsqcup_i f^n_i)}^\infty_{n=0}$ е верига.
  \end{enumerate}
\end{problem}


\begin{theorem}\label{th:Y}\index{Y}
  \marginpar{Доказателството в \cite[стр. 188]{models-of-computation} е малко по-различно.}
  % \index{$Y_\A$}
  Нека $\A$ е област на Скот и нека $f \in \Cont{\A}{\A}$.
  \marginpar{Знаем от \Th{knaster-tarski}, че най-малката неподвижна точка на $f$ е елемента $\bigsqcup_n f^n(\bot^\A)$,
    т.е. $\lfp(f) = \bigsqcup_n f^n(\bot^\A)$.}
  Тогава изображението $Y : \Cont{\A}{\A} \to \A$, определено като
  \[Y(f) = \lfp(f),\]
  е непрекъснато, т.е.
  $Y \in \Cont{\Cont{\A}{\A}}{\A}$.
\end{theorem}
\begin{proof}
  Нека да вземем една верига $\chain{f}{n}$ от непрекъснати изображения.
  Нашата цел е да докажем, че
  \[Y(\bigsqcup_n f_n) = \bigsqcup_n Y(f_n).\]
  \marginpar{Записът ще стане много тромав, ако вместо $h$ използваме означението $\bigsqcup_n f_n$.}
  Да означим с $h$ точната горна граница на $\chain{f}{n}$.
  Знаем, че $h(a) = \bigsqcup_n \{f_n(a)\}$, а от \Th{continuous-domain} знаем, че $h$ е непрекъснато изображение.
  \begin{proposition}
    За всяко $k \geq 0$ е изпълнено, че $(\bigsqcup f_i)^k = \bigsqcup_i f^k_i$.
  \end{proposition}
  \begin{proof}
    Ще докажем твърдението с индукция по $k$, като случая $k = 0$ е ясен, защото
    $(\bigsqcup_i f_i)^0 = id = \bigsqcup_i f_i^0 = \bigsqcup_i id$.
    Нека приемем, че твърдението е вярно за произволно $k$.
    Ще докажем, че твърдението е вярно за $k+1$.
    \begin{align*}
      (\bigsqcup_i f_i)^{k+1} & = \compose((\bigsqcup_if_i)^k, \bigsqcup_i f_i)\\
                              & = \compose(\bigsqcup_if^k_i, \bigsqcup_i f_i) & \comment\text{от \IndHyp}\\
                              & = \bigsqcup_i \compose(f^k_i, f_i) & \comment\compose\text{ е непр.}\\
                              & = \bigsqcup_i f^{k+1}_i
    \end{align*}
    
    % \begin{align*}
    %   h^{k+1}(a) & = h(h^k(a)) & \\
    %              & = h(\bigsqcup_n f^k_n(a))& \comment{\text{ от инд. предположение}}\\
    %              & = \bigsqcup_n h(f^k_n(a))& \comment{h \text{ е непрекъснато изображение}}\\
    %              & = \bigsqcup_n (\bigsqcup_m f_m(f^k_n(a))). & 
    % \end{align*}
    
    % Да положим $b_n = f^k_n(a)$, за всяко $n$.
    % Понеже $f_n \sqsubseteq f_{n^\prime}$, лесно се съобразява, че за $n \leq n^\prime$
    % имаме $b_n \sqsubseteq^\A b_{n^\prime}$.

    % Сега да положим $e_{m,n} = f_m(b_n)$.
    % Отново, понеже $\chain{b}{n}$ и $\chain{f}{m}$ са вериги, имаме 
    % \[m \leq m^\prime\ \&\ n\leq n^\prime\ \Rightarrow\ e_{m,n} \sqsubseteq^\A e_{m^\prime,n^\prime}.\]
    % Получаваме, че
    % \begin{align*}
    %   h^{k+1}(a) & = \bigsqcup_n (\bigsqcup_m f_m(f^k_n(a))) & \comment{\text{ от горното равенство за } h^{k+1}}\\
    %              & = \bigsqcup_n (\bigsqcup_m e_{m,n}) & \comment{\text{ от определението на }e_{m,n}}\\
    %              & = \bigsqcup_n e_{n,n} & \comment{\text{ от \Th{double-chain}}}\\
    %              & = \bigsqcup_n f_n(f^k_n(a))  = \bigsqcup_n f^{k+1}_n(a) & 
    % \end{align*}
    С това твърдението е доказано.
  \end{proof}
  Сега вече сме готови да докажем непрекъснатостта на $Y$.
  Имаме, че:
  \begin{align*}
    Y(\bigsqcup_i f_i) & = \bigsqcup_m (\bigsqcup_i f_i)^m(\bot^\A) & \comment{\text{ от опр. на }Y }\\
                       & = \bigsqcup_m (\bigsqcup_i f^m_i(\bot^\A)) & \comment{\text{ от горното твърдение}}
  \end{align*}

  Да положим $e_{m,n} = f^m_n(\bot^\A)$.
  Отново лесно се съобразява, че 
  \[m \leq m^\prime\ \&\ n\leq n^\prime\ \Rightarrow\ e_{m,n} \sqsubseteq^\A e_{m^\prime,n^\prime}.\]
  Получаваме, че
  \begin{align*}
    Y(\bigsqcup_n f_n) & = \bigsqcup_m (\bigsqcup_n f^m_n(\bot^\A)) & \comment{\text{ от горното равенство}}\\
                          & = \bigsqcup_m (\bigsqcup_n e_{m,n}) & \comment{\text{ от опр. на }e_{m,n}}\\
                          & = \bigsqcup_n(\bigsqcup_m e_{m,n}) & \comment{\text{ от \Th{double-chain}}}\\
                          & = \bigsqcup_n (\bigsqcup_m f^m_n(\bot^\A)) = \bigsqcup_n Y(f_n). & \comment{\text{ от опр. на }Y}.
  \end{align*}
\end{proof}

\marginpar{Добре е да погледнете \href{https://en.wikibooks.org/wiki/Haskell/Fix_and_recursion}{това}.}

\begin{haskellcode}
ghci> fact x = if x == 0 then 1 else x * fact(x-1)
ghci> fact 5
120
ghci> fix f = x where x = f x
ghci> :t f
fix :: (t -> t) -> t
ghci> fact' = fix \f x -> if x == 0 then 1 else x * f(x-1)
ghci> fact' 5
120
ghci> gamma f = \x -> if x == 0 then 1 else x * f(x-1)
ghci> :t gamma
(t -> t) -> t -> t
ghci> fact'' = fix gamma
ghci> fact'' 5
120
ghci> fix' f = x where x = f(f(f(x)))
ghci> fact''' = fix' gamma
ghci> fact''' 5
120
\end{haskellcode}


%%% Local Variables:
%%% mode: latex
%%% TeX-master: "../sep"
%%% End:
