\section{Изоморфни области на Скот}
\index{изоморфизъм}

\marginpar{\cite[стр. 124]{winskel}.}

Нека $\A_1 = (A_1,~\sqsubseteq_1,~\bot_1)$ и $\A_2 = (A_2,~\sqsubseteq_2~,~\bot_2)$ 
са области на Скот.
Ще казваме, че $\A_1$ е {\bf изоморфна} на $\A_2$, което ще означаваме като 
\[\A_1 \cong \A_2,\]
ако съществува {\em биективна} функция $F:A_1 \to A_2$ със свойството:
\[(\forall a \in A_1)(\forall b\in A_1)[\ a \sqsubseteq_1 b \iff F(a) \sqsubseteq_2 F(b)\ ].\]
В такъв случай ще казваме, че $F$ задава изоморфизъм между $\A_1$ и $\A_2$.

Когато искаме да означим, че $\A_1$ е изоморфна на $\A_2$ чрез $F$,
то понякога ще пишем $\A_1 \cong_F \A_2$.

\begin{problem}
  Докажете, че ако $\A_1 \cong_F \A_2$, то $F(\bot_1) = \bot_2$.
\end{problem}


\begin{proposition}
  \label{pr:isomorphism-is-continuous}
  Ако $\A_1 \cong_F \A_2$ , то $F \in \Cont{\A_1}{\A_2}$.
\end{proposition}
\begin{hint}
  Да разгледаме произволна верига $\chain{a}{i}$ от елементи на $\A_1$.
  Ще докажем, че 
  \[F(\bigsqcup_i a_i) = \bigsqcup_iF(a_i).\]
  
  \begin{itemize}
  \item 
    Първо, от дефиницията веднага следва, че $F$ е монотонно изображение, защото
    \[a \sqsubseteq_1 b \implies F(a) \sqsubseteq_2 F(b).\]
    Това означава, че $(F(a_i))^\infty_{i=0}$ е монотонно растяща верига от елементи на $\A_2$.
    От \Prop{monotone-chain} получаваме, че 
    \[\bigsqcup_i F(a_i) \sqsubseteq_2 F(\bigsqcup_i a_i).\]
  \item
    За другата посока, нека $b \in \A_2$ е горна граница на веригата $(F(a_i))^\infty_{i=0}$, т.е. 
    \[(\forall i)[\ F(a_i) \sqsubseteq_2 b\ ].\]
    Ще докажем, че $F(\bigsqcup_i a_i) \sqsubseteq_2 b$.
    Понеже $F$ е {\em върху} $A_2$, то съществува елемент $a \in A_1$, такъв че $F(a) = b$.
    Тогава:
    \begin{align*}
      (\forall i)[\ F(a_i) \sqsubseteq_2 F(a)\ ] & \implies (\forall i)[\ a_i \sqsubseteq_1 a\ ] & \comment{F \text{ е изоморфизъм }}\\
                                                 & \implies \bigsqcup_i a_i \sqsubseteq_1 a & \comment{a\text{ е горна граница}}\\
                                                 & \implies F(\bigsqcup_i a_i) \sqsubseteq_1 F(a). & \comment{F\text{ е изоморфизъм }}
    \end{align*}
    Понеже $b = F(a)$, заключаваме, че
    \[F(\bigsqcup_i a_i) \sqsubseteq_2 b.\]
  \end{itemize}
\end{hint}

\begin{proposition}
  \label{pr:isomorphic-pair}
  Нека $f \in \Mon{\A_1}{\A_2}$ и $g \in \Mon{\A_2}{\A_1}$,
  като 
  \begin{itemize}
  \item 
    $f \circ g = \texttt{id}_2$;
  \item
    $g \circ f = \texttt{id}_1$.
  \end{itemize}
  \marginpar{$\texttt{id}_i(a) \df a$ за вс. $a \in \A_i$}
  Тогава са изпълнени свойствата:
  \begin{enumerate}[(1)]
  \item
    $\A_1 \cong_f \A_2$;
  \item
    $\A_2 \cong_g \A_1$;
  \end{enumerate}
\end{proposition}
% \begin{hint}
%   За Свойство $(1)$ трябва да проверим, че $f$ отговаря на дефиницията за изоморфизъм.
%   \begin{itemize}
%   \item
%     Ще докажем, че $f$ е инективна като покажем, че за произволни $a, b\in A_1$,
%     ако $f(a) = f(b)$, то $a = b$.
%     Но това е лесно, защото
%     \[a = \texttt{id}_1(a) = g(f(a)) = g(f(b)) = \texttt{id}_1(b) = b.\]
%   \item
%     Нека сега $b \in \A_2$.
%     Знаем, че $f(g(b)) = \texttt{id}_2(b) = b$. Това означава, че $f$ е {\em сюрективна},
%     защото за всеки елемент $b \in A_2$ съществува елемент $a \in A_1$, а именно $a = g(b)$,
%     за който $f(a) = b$.
%   \item
%     Понеже $f$ е монотонно изображение, то директно имаме, че
%     \[a \sqsubseteq_1 b \implies f(a) \sqsubseteq_2 f(b).\]
%   \item
%     Нека $f(a) \sqsubseteq_2 f(b)$.
%     Сега пък понеже $g$ е монотонно изображение, 
%     \[a = \texttt{id}_1(a) = g(f(a)) \sqsubseteq_1 g(f(b)) = \texttt{id}_1(b) = b.\]
%     Така показахме, че
%     \[f(a) \sqsubseteq_2 f(b)\ \implies\ a \sqsubseteq_1 b.\]
%   \end{itemize}
%   Доказахме Свойство $(1)$, т.е. $\A_1 \cong_f \A_2$.
%   Разсъжденията за Свойство $(2)$ са аналогични.
% \end{hint}


\begin{proposition}
  \label{pr:isomorphic-higher-order}
  Нека $\A_1 \cong_F \A_2$. Тогава:
  \begin{enumerate}[(1)]
  \item 
    $\Cont{\A_1}{\A_1} \cong_G \Cont{\A_2}{\A_2}$, където 
    \[G(f) \df F \circ f \circ F^{-1};\]
    Графично това може да се изобрази така:

    \shorthandoff{"}%
    \begin{center}
    \begin{tikzcd}[sep=large]
      \A_1 \arrow[r, "f"] & \A_1 \arrow[d, "F"]\\
      \A_2 \arrow[u, "F^{-1}"]\arrow[r, dashed, "G(f)"] & \A_2 
    \end{tikzcd}
    \end{center}
    \shorthandon{"}%
  \item
    ако $f \in \Cont{\A_1}{\A_1}$, то 
    \[F(\lfp(f)) = \lfp(G(f)).\]
  \end{enumerate}
\end{proposition}
\begin{hint}
  Ще докажем $(1)$ като използвме \Prop{isomorphic-pair}.

  \begin{itemize}
  \item 
    Ще докажем, че $G$ е монотонно изображение.
    Нека $f,h \in \Cont{\A_1}{\A_1}$ и $f \sqsubseteq h$, т.е.
    \[(\forall a \in \A_1)[\ f(a) \sqsubseteq_1 h(a)\ ].\]
    Ще докажем, че $G(f) \sqsubseteq G(h)$, т.е.
    \[(\forall b \in \A_2)[\ G(f)(b) \sqsubseteq_1 G(h)(b)\ ].\]
    Да разгледаме произволен елемент $b \in \A_2$. 
    Понеже $F$ е биекция, то съществува елемент $a \in A_1$, такъв че $F(a) = b$,
    т.е. $F^{-1}(b) = a$. Тогава:
    \begin{align*}
      G(f)(b) & \df F(f(F^{-1}(b)))\\
              & = F(f(a)) & \comment{F^{-1}(b) = a}\\
              & \sqsubseteq_2 F(h(a)) & \comment{f(a) \sqsubseteq h(a)\text{ и $F$ е изом.}}\\
              & = F(h(F^{-1}(b))) & \comment{F^{-1}(b) =a}\\
              & \df G(h)(b).
    \end{align*}
  \item
    Нека $G(f) \sqsubseteq G(h)$. Ще докажем, че $f \sqsubseteq h$.
    За целта, нека $a \in A_1$.
    Понеже $F$ е сюрективна, то съществува $b \in A_2$, за който $F^{-1}(b) =a$.
    Понеже
    \[G(f) \df F \circ f \circ F^{-1} \sqsubseteq F \circ h \circ F^{-1} = G(h),\]
    то получаваме, че
    \[F(f(F^{-1}(b))) \sqsubseteq_2 F(h(F^{-1}(b))).\]
    Оттук,
    \[F(f(a)) \sqsubseteq_2 F(h(a)) \implies f(a) \sqsubseteq_1 h(a),\]
    защото $F$ е изоморфизъм.
  \end{itemize}
  Сега преминаваме към доказателството на $(2)$.
  Да напомним, че за $f \in \Cont{\A_1}{\A_1}$, означаваме
  \begin{align*}
    f^0  & = \lambda x. \bot_1\\
    f^{n+1} & = f \circ f^n.
  \end{align*}
  Понеже $f$ е непрекъснато изображение е ясно, че $(f^n(\bot_1))^{\infty}_{n=0}$ е верига.
  Също така знаем, че
  \[\lfp(f) = \bigsqcup_n f^n(\bot_1).\]
  След аналогични разсъждения можем да съобразим, че
  \[\lfp(G(f)) = \bigsqcup_n G(f)^n(\bot_2).\]
  Първо ще докажем с индукция по $n$, че 
  \begin{equation}
    \label{eq:2}
    (\forall n)[\ (G(f))^n = G(f^n)\ ].
  \end{equation}
  \begin{itemize}
  \item 
    За $n = 0$ имаме, че за произволен елемент $b \in \A_2$,
    \begin{align*}
      (G(f))^{0}(b) & \df \bot_2\\
                    & = F(\bot_1) & \comment{F \text{ е изом.}}\\
                    & = F(f^{0}(F^{-1}(b))) & \comment{f^{0}(F^{-1}(b)) \df \bot_1}\\
                    & = (F \circ f^{0} \circ F^{-1})(b) \\
                    & \df G(f^{0})(b).
    \end{align*} 
  \item
    Нека да приемем, че твърдението е вярно за $n$.
    Тогава за $n+1$ имаме, че:
    \begin{align*}
      (G(f))^{n+1} & \df G(f) \circ (G(f))^n\\
                   & = G(f) \circ G(f^n) & \comment{\text{ от И.П.}}\\
                   & \df (F \circ f \circ F^{-1}) \circ (F\circ f^n \circ F^{-1})\\
                   & = F \circ f \circ (F^{-1} \circ F)\circ f^n \circ F^{-1} \\
                   & = F \circ f \circ f^{n} \circ F^{-1} & \comment{F^{-1}\circ F = id}\\
                   & = F \circ f^{n+1} \circ F^{-1} & \comment{f\circ f^n = f^{n+1}}\\
                   & \df G(f^{n+1}).
    \end{align*}
  \end{itemize}
  Тогава:
  \begin{align*}
    F(\lfp(f)) & = F(\bigsqcup_n f^n(\bot_1)) & \comment{\lfp(f) = \bigsqcup_n f^n(\bot_1)}\\
               & = \bigsqcup_n F(f^n(\bot_1))& \comment{F\text{ е непр.}}\\
               & = \bigsqcup_n F(f^n(F^{-1}(\bot_2))) & \comment{F^{-1}(\bot_2) = \bot_1}\\
               & = \bigsqcup_n (F \circ f^n \circ F^{-1})(\bot_2) \\
               & \df \bigsqcup_n G(f^n)(\bot_2)\\
               & = \bigsqcup_n G(f)^n(\bot_2) & \comment{\text{от }(\ref{eq:2})}\\
               & = \lfp(G(f)).
  \end{align*}
\end{hint}

\begin{framed}
  \begin{proposition}
    За произволни области на Скот $\A$, $\B$ и $\C$ е изпълнено, че
    \[\Cont{\A}{\Cont{\B}{\C}}\ \cong\ \Cont{\A\times\B}{\C}.\]
  \end{proposition}  
\end{framed}
\marginpar{\cite[стр. 139]{winskel}.}
\begin{hint}
  % \begin{itemize}
  % \item 
    Докажете, че изображението
    \[\texttt{curry}:\Cont{\A\times \B}{\C} \to \Cont{\A}{\Cont{\B}{\C}},\]
    където
    \[\texttt{curry}(f)(a)(b) \df f(a,b)\]
    задава изоморфизъм.
  % \item
  %   Докажете, че изображението
  %   \[\texttt{uncurry}:\Cont{\A}{\Cont{\B}{\C}} \to \Cont{\A\times \B}{\C},\]
  %   където
  %   \[\texttt{uncurry}(f)(a,b) \df f(a)(b)\]
  %   е монотонно.
  % \item
  %   Лесно се съобразява, че
  %   \[\texttt{curry} \circ \texttt{uncurry} = \texttt{id}\]
  %   \[\texttt{uncurry} \circ \texttt{curry} = \texttt{id}.\]    
  % \item
  %   Приложете \Prop{isomorphic-pair}.
  % \end{itemize}
\end{hint}

Когато на хаскел пишем типовата сигнатура на някоя функция като 
\mint{haskell}|f :: a -> b -> c| в действителност се има предвид \mint{haskell}|f :: a -> (b -> c)|

На практика тези две задачи ни казват, че няма значение дали използваме {\em curried}
или {\em uncurried} версията на една функция. На \texttt{хаскел} е по-удобно да използваме {\em curried}
версията, защото като фиксираме първия аргумент на една функция получаваме нова функция наготово.
\marginpar{Това се нарича \emph{partial application}. Вижте \url{https://wiki.haskell.org/Partial_application}.}
Например, 
\begin{haskellcode}
  ghci> let plus x y = x + y
  ghci> :t plus
  plus :: Num a => a -> a -> a
  ghci> let plus1 = plus 1
  ghci> :t plus1
  plus1 :: Num a => a -> a
\end{haskellcode}

Нека да дефинираме
\[\emptyset_\bot = (\{\bot\}, \sqsubseteq, \bot).\]
\begin{problem}
  Докажете, че за произволна област на Скот $\A$ е изпълнено:
  \begin{align*}
    & \Cont{\emptyset_\bot}{\A} \cong \A\\
    & \Cont{\A}{\emptyset_\bot} \cong \emptyset_\bot.
  \end{align*}
\end{problem}

\begin{problem}
  Докажете, че съществуват области на Скот $\A$, $\B$ и $\C$, за които
  \[\Cont{\Cont{\A}{\B}}{\C} \not\cong \Cont{\A}{\Cont{\B}{\C}}.\]  
\end{problem}
\begin{hint}
  Нека изберем $\A = \C = \Nat_\bot$, а $\B = \emptyset_\bot$. Тогава
  \[\Cont{\Cont{\Nat_\bot}{\emptyset_\bot}}{\Nat_\bot} \cong \Cont{\emptyset_\bot}{\Nat_\bot} \cong \Nat_\bot,\]
  но лесно можем да съобразим, че:
  \[\Cont{\Nat_\bot}{\Cont{\emptyset_\bot}{\Nat_\bot}} \cong \Cont{\Nat_\bot}{\Nat_\bot}.\]
  Сега остава да съобразим, че
  \[\Nat_\bot \not\cong \Cont{\Nat_\bot}{\Nat_\bot}.\]
\end{hint}

%%% Local Variables:
%%% mode: latex
%%% TeX-master: "../sep"
%%% End:
