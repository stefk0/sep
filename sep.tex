\documentclass[a4paper,12pt,oneside]{book}

% \renewcommand{\familydefault}{\sfdefault}

\usepackage{cmap}
\usepackage[utf8]{inputenc}
\usepackage[T2A]{fontenc}
\usepackage[bulgarian]{babel}



\usepackage[xindy]{imakeidx}
\def\xindylangopt{-M lang/bulgarian/utf8-lang.xdy}
\makeindex[options = \xindylangopt]
% \makeindex[name=names, title = Именен указател, options = \xindylangopt]
% \makeindex[name=proglangs, title = Изчислителни машини и програмни езици, options = \xindylangopt]
% \makeindex[name=organisations, title = Организации и фирми, options = \xindylangopt]

% \makeindex[options = \xindylangopt]
\makeatletter
\let\original@index\index
\renewcommand{\index}[2][\imki@jobname]{%
  \original@index[#1]{\detokenize{#2}}%
}

\usepackage{amssymb, amsmath, amsthm, latexsym, mathrsfs, bm, mathtools}
\usepackage{stmaryrd}
\usepackage{makeidx}

\usepackage{stackrel}
\usepackage{paralist}
\usepackage{pifont}
\usepackage[shortlabels]{enumitem}
\setlist{leftmargin=*}
\usepackage{algorithm}
\floatname{algorithm}{Алгоритъм}
\usepackage[noend]{algpseudocode}
\usepackage{framed}

\usepackage{bussproofs}
\def\defaultHypSeparation{\hskip .2cm}

\usepackage{wrapfig}
\usepackage{xcolor}
\usepackage{minted}
\usepackage{pdflscape}

\usepackage{listings}% http://ctan.org/pkg/listings
\lstset{
  basicstyle=\ttfamily,
  mathescape
}

\usepackage{tikz, pgf}
\usetikzlibrary{arrows, automata, positioning, backgrounds, decorations.pathmorphing, decorations.markings}

\usepackage{titlesec}
\newcommand{\sectionbreak}{\clearpage}

\usemintedstyle{friendly}
\definecolor{bg}{rgb}{0.9, 0.9, 0.9}
\newminted{haskell}{mathescape,fontsize=\footnotesize,baselinestretch=1.2,frame=single,framesep=5pt} %,bgcolor=bg}
%\usepackage{microtype} ??????

\usepackage{color, soulutf8}
\newcommand{\newauthor}[2]{
  \definecolor{#1}{rgb}{#2}
  \expandafter\newcommand\csname #1\endcsname[1]{{\sethlcolor{#1}\hl{#1: ,,##1''}}}}
 
\definecolor{codegreen}{rgb}{0,0.6,0}
\definecolor{codegray}{rgb}{0.5,0.5,0.5}
\definecolor{codepurple}{rgb}{0.58,0,0.82}
\definecolor{backcolour}{rgb}{0.95,0.95,0.92}

% \newauthor{Stefan}{0.6,0.6,0.6}

\newenvironment{SystemEq}
{% This is the begin code
  \[\left|
      \begin{array}{lcl}
}
{% This is the end code
      \end{array}
   \right.\]
}



\newcommand{\Stefan}[1]{}

% \newauthor{Stela}{0.8,0.8,0}


\setlist[itemize]{leftmargin=*}

\def\dotminus{\mathbin{\ooalign{\hss\raise1ex\hbox{.}\hss\cr\mathsurround=0pt$-$}}}

%%%%%%%%%%%%%%% TIKZ Package %%%%%%%%%%%%%%%%%%%%%%%
\usepackage{tikz, pgf}
\usetikzlibrary{shapes,arrows,cd,positioning}
%%%%%%%%%%%%%%%%%%%%%%%%%%%%%%%%%%%%%%%%%%%%%%%%%%%%
\usepackage{caption}
\usepackage{subcaption}
% \usepackage{subfigure}

\usepackage[pdfencoding=unicode, colorlinks=true, linkcolor=blue, pdfstartview=FitV, citecolor=green, urlcolor=blue]{hyperref}
\hypersetup{pdfauthor={Стефан Вътев},pdftitle={Семантика на езиците за програмиране},pdftex,unicode}

\newcommand{\NN}{\mathbf{Num}}
\newcommand{\BB}{\mathbf{Bool}}

\usepackage{mysymbols}
\usepackage{mybgdef}
\usepackage{sep}


%%\selectlanguage{bulgarian}



% \newcommand{\val}[1]{\llbracket{#1}\rrbracket}
\newcommand{\evalv}[1]{\text{eval}^P_V\val{{#1}}}
\newcommand{\evaln}[1]{\text{eval}^P_N\val{{#1}}}
\newcommand{\eval}[1]{\text{eval}_{\vv{P}}\val{{#1}}}
% \newcommand{\iifelse}[3]{\texttt{if}\ #1\ \texttt{then}\ #2\ \texttt{else}\ #3} 
% \newcommand{\while}[2]{\mathbf{while}\ #1\ \mathbf{do}\ #2}
% \newcommand{\rpt}[2]{\mathbf{repeat}\ #1\ \mathbf{until}\ #2}
\newcommand{\op}{\texttt{op}}
\newcommand{\true}{\mathbf{t}}
\newcommand{\false}{\mathbf{f}}
% \newcommand{\exit}{\mathbf{exit}}
% \newcommand{\skp}{\mathbf{skip}}
\newcommand{\newvar}[3]{\mathbf{newvar}\ #1 := #2\ \mathbf{in}\ #3}
\newcommand{\trycatch}[2]{\mathbf{try}\ #1\ \mathbf{catch}\ #2}
\newcommand{\finally}[3]{\mathbf{try}\ #1\ \mathbf{catch}\ #2\ \mathbf{finally}\ #3}
\newcommand{\for}[4]{\mathbf{for}\ #1 := #2\ \mathbf{to}\ #3\ \mathbf{do}\ #4}
\newcommand{\dfs}{\stackrel{\textnormal{деф}}{=}}
% \newcommand{\dff}{\stackrel{\textnormal{деф}}{=}}
\newcommand{\dfff}{\stackrel{\textnormal{деф}}{\equiv}}
\newcommand{\dffff}{\stackrel{\textnormal{деф}}{\equiv}}
\newcommand{\lfp}{\texttt{lfp}}
\newcommand{\lub}{\texttt{lub}}


\newcommand{\Mapping}[2]{[#1~\to~#2]}
% \newcommand{\Total}[2]{[#1~\stackrel{t}{\to}~#2]}
\newcommand{\Partial}[2]{[#1~\stackrel{\text{ч}}{\to}~#2]}
\newcommand{\Strict}[2]{[#1~\stackrel{\text{т}}{\to}~#2]}
\newcommand{\Mon}[2]{[#1~\stackrel{\text{м}}{\to}~#2]}
\newcommand{\Cont}[2]{[#1~\stackrel{\text{н}}{\to}~#2]}
% \newcommand{\Compact}[2]{[#1~\stackrel{\text{k}}{\to}~#2]}
\newcommand{\DomOpCBN}{\Cont{\Nat^{m_1}_\bot}{\Nat_\bot}\times\cdots\times\Cont{\Nat^{m_k}_\bot}{\Nat_\bot}}
\newcommand{\RanOpCBN}{\Cont{\Nat^{n}_\bot}{\Nat_\bot}}
\newcommand{\DomOpCBV}{\Strict{\Nat^{m_1}_\bot}{\Nat_\bot}\times\cdots\times\Strict{\Nat^{m_k}_\bot}{\Nat_\bot}}
\newcommand{\RanOpCBV}{\Strict{\Nat^{n}_\bot}{\Nat_\bot}}


\newcommand{\Pref}{\texttt{Pref}}                 %% Преднеподвижни точки
\newcommand{\Fenv}{\textbf{Fenv}[\varsf]}
\newcommand{\Venv}{\textbf{Env}[\varsx]}

\newcommand{\Graph}[1]{\texttt{Graph}(#1)}

% \newcommand\nil{\texttt{[]}}
% \newcommand{\pair}[1]{\langle{#1}\rangle}
% \newcommand{\abs}[1]{\lvert{#1}\rvert}
% \newcommand{\ov}[1]{\overline{#1}}
% \renewcommand{\iff}{\Longleftrightarrow}
\newcommand{\Cmd}{\mathbf{Cmd}}
\newcommand{\Arith}{\mathbf{Arith}}

\newcommand{\varx}{\mathbf{x}}
\newcommand{\varsx}{\bar{\texttt{x}}}
\newcommand{\varsf}{\bar{\texttt{f}}}

%%%%%%% LISTS %%%%%%%

\newcommand{\FinL}{\texttt{FinL}}
\newcommand{\PartL}{\texttt{PartL}}
\newcommand{\InfL}{\texttt{InfL}}

%%%%%%%%%%%%%%%%%%%%%

\def\monus{\mathbin{\ooalign{\hss\raise1ex\hbox{.}\hss\cr
      \mathsurround=0pt$-$}}}

\setlength{\marginparsep}{1cm}
\setlength{\oddsidemargin}{0.3cm}
\setlength{\hoffset}{-0.75in}
\setlength{\marginparwidth}{110pt}
\setlength{\textwidth}{420pt}
\setlength{\textheight}{670pt}
\let\oldmarginpar\marginpar

\def\dotminus{\mathbin{\ooalign{\hss\raise1ex\hbox{.}\hss\cr
             \mathsurround=0pt$-$}}}


% \renewcommand\marginpar[1]{\-\oldmarginpar[\raggedleft\footnotesize #1]%
% {\raggedright\footnotesize #1}}
% \renewcommand\marginpar[1]{\-\oldmarginpar[\raggedleft\scriptsize #1]
% {\raggedright\scriptsize #1}}
\renewcommand\marginpar[1]{\oldmarginpar{\scriptsize #1}}

\newif\ifhints
%\proofstrue % comment out to hide answers
\hintstrue

\allowdisplaybreaks
% \makeindex


\usepackage[scaled=0.89]{PTSerif}
\usepackage[scaled=0.89]{PTSans}

\author{Стефан Вътев\footnote{\href{mailto:stefanv@fmi.uni-sofia.bg}{stefanv@fmi.uni-sofia.bg}}}
\title{Семантика на езиците за програмиране - записки}

\begin{document}


% \frontmatter
\maketitle
\tableofcontents


% \chapter{Увод}

% Увод няма.


% \mainmatter

% \begin{otherlanguage}{greek}
% \begin{haskellcode}
%   ΤΕΣΤ
% \end{haskellcode}
% \end{otherlanguage}


% \include{basic}
% \include{imp}
% \include{Operators}
\include{imp}
\chapter{Теоретични основи}
% \index{област на Скот}

\marginpar{\cite[Глава 5]{models-of-computation}}

В тази глава ще разгледаме понятията, които са ни нужни за дефинирането на денотационна семантика на една програма.

\section{Частични наредби}
\index{частична наредба}

Бинарната релация $\sqsubseteq$ върху множеството $A$ се нарича {\bf частична наредба}, ако тя е:
\marginpar{На англ. \emph{partial order}}
\begin{itemize}
\item 
  рефлексивна, т.е. $(\forall a \in A)[a \sqsubseteq a]$;
\item
  транзитивна, т.е. $(\forall a,b,c \in A)[a \sqsubseteq b\ \&\ b \sqsubseteq c \implies a \sqsubseteq c]$;
\item
  антисиметрична, т.е. $(\forall a,b \in A)[a \sqsubseteq b\ \&\ b \sqsubseteq a  \implies a = b]$.
\end{itemize}
Една такава двойка $(A, \sqsubseteq)$ се нарича частично наредено множество.

\begin{example}
  Да означим 
  \[\F_n \df \{f:\Nat^n\to\Nat \mid f\text{ е частична функция}\}.\]
  \marginpar{Често вместо $x_1,\dots,x_n$ ще пишем просто $\ov{x}$.}
  Дефинираме и релацията {\bf включване } между две частични функции по следния начин:
  \begin{align*}
    f \subseteq g \dfff (\forall \ov{x}\in \Nat^n)[& f(\ov{x})\text{ не е деф.}\ \vee\\
                                            & (f(\ov{x})\text{ е деф.}\ \&\ g(\ov{x})\text{ е деф.}\ \&\ f(\ov{x}) = g(\ov{x}))].
  \end{align*}
  Да дефинираме също {\bf графиката} на частичната функция $f$ като
  \[\Graph{f} \df \{\pair{\ov{x},y} \mid f(\ov{x}) = y\}.\]
  Тогава лесно се съобразява, че 
  \[f \subseteq g \iff \Graph{f} \subseteq \Graph{g}.\]
  Съобразете, че двойката $(\Partial{\Nat}{\Nat}, \subseteq)$ е частично наредено множество.
\end{example}

\marginpar{Обърнете внимание, че има и понятие \emph{минимален} елемент, което в общия случай е различно от понятието \emph{най-малък} елемент. Един елемент $a_0$ е минимален за множеството $A$, ако $\neg(\exists a \in A)[a \sqsubseteq a_0\ \&\ a \neq a_0]$. Съобразете, че е възможно едно частично наредено множество да притежава повече от един минимални елементи.}
Казваме, че $a_0$ е {\bf най-малък елемент} на частично нареденото множество $(A, \sqsubseteq)$,
ако $(\forall a \in A)[a_0 \sqsubseteq a]$. Ако такъв елемент съществува, то той е единствен,
защото релацията $\sqsubseteq$ е антисиметрична.
% {\bf Неподвижна точка} на $f:A \to A$ е елемент $a \in A$, такъв че $f(a) = a$.
За по-кратко монотонно-растящите редици от елементи на $A$,
\[a_0 \sqsubseteq a_1 \sqsubseteq \cdots \sqsubseteq a_n \sqsubseteq \cdots,\]
ще наричаме (растящи) {\bf вериги}. 

Един елемент $b$ е {\bf горна граница} на веригата $\chain{a}{n}$, ако 
$(\forall n)[a_n \sqsubseteq b]$.
Един елемент $b$ е {\bf точна горна граница} на веригата $\chain{a}{n}$, ако са изпълнени свойствата:
\begin{itemize}
\item 
  $(\forall n)[a_n \sqsubseteq b]$, т.е. $b$ е горна граница;
\item
  за всяка друга горна граница $c$ е изпълнено, че $b \sqsubseteq c$, т.е.
  $b$ е най-малкият елемент измежду всички горни граници на веригата $\chain{a}{n}$.
\end{itemize}
Не всяка верига притежава точна горна граница.
Обикновено точната горна граница на вергата $\chain{a}{n}$ ще бележим като $\bigsqcup_n a_n$.

\Stefan{Още тук да се даде пример за верига от частични функции и две функции - една, която е точна горна граница на веригата и една, която е просто горна граница.}


\begin{framed}
  \begin{definition}
    Наредена тройка от вида $\A = (A, \sqsubseteq, \bot)$ се нарича {\bf област на Скот}, ако:
    \index{област на Скот}
    \begin{itemize}
    \item
      $\sqsubseteq$ е бинарна релация върху $A$, която задава частична наредба.
    \item
      Всяка растяща верига $\chain{a}{n}$ в $A$ притежава точна горна граница $\bigsqcup_n a_n$.
    \item
      $\bot \in A$ е най-малкият елемент на $A$;
    \end{itemize}
  \end{definition}
\end{framed}

\marginpar{На англ. {\em Scott domain}. Обикновено в литературата, за да се нарече едно частично нареденото множество област на Скот се изискват още допълнителни свойства, но за нашите цели тази дефиниция ще свърши работа.}

Интуицията зад израза $a \sqsubseteq b$ е, че $b$ носи повече информация от $a$, без да противоречи на $a$. Елементът $\bot$ означава липса на информация.

\begin{example}
  Тройката $\Partial{\Nat^n}{\Nat} \df (\ \F_n,\ \subseteq,\ \bm{\emptyset}^{(n)}\ )$ е област на Скот, където:
  \begin{itemize}
  \item
    С $\F_n$ означаваме всички частични функции от $\Nat^n$ в $\Nat$.
  \item
     релацията ,,включване'' между функции е дефинирана по следния начин:
     \[f\subseteq g\ \dffff\ \text{Graph}(f) \subseteq \text{Graph}(g).\]
   \item
     $\bm{\emptyset}^{(n)}$ е функцията с празна дефиниционна област, т.е. $\Dom(\bm{\emptyset}^{(n)}) = \emptyset$.
  \end{itemize}
\end{example}

\marginpar{В хаскел $\bot$ се означава като \vv{undefined}. Повече за денотационна семантика в хаскел може да прочетете \href{https://en.wikibooks.org/wiki/Haskell/Denotational_semantics}{тук}.}

\begin{example}
  Да разгледаме няколко примера, които вече сме срещали.
  \begin{itemize}
  \item
    $(\Ps(\Nat),\subseteq,\emptyset)$ е област на Скот.
  \item
    $(\Nat, \leq, 0)$ не е е област на Скот.
  \item
    $(\Nat\cup\{\infty\}, \leq, 0)$ е област на Скот, където наредбата $\leq$ е зададена като
    \[0 \leq 1 \leq \cdots \leq \infty.\]
  \item
    $(\{0,1\}^\star, \preceq, \varepsilon)$ не е област на Скот, където $\preceq$ е релацията префикс на две думи.
  \end{itemize}
\end{example}

\begin{example}
  Да разгледаме множеството 
  \[Bin^\infty = \{\sigma \mid \sigma :\{0,1,2,\dots,n-1\} \to \{0,1\}\ \&\ n \in \Nat\} \cup 
  \{f \mid f:\Nat \to \{0,1\}\}\]
  съставено от всички крайни и безкрайни двоични низове.
  \begin{itemize}
  \item
    Да разгледаме релацията
    \[\sigma \preceq \tau \iff \abs{\sigma} \leq \abs{\tau}\ \&\ (\forall i < |\sigma|)[\sigma(i) = \tau(i)],\]
    т.е. $\sigma$ е префикс на $\tau$.    
  \item
    Да означим с $\varepsilon$ единствения двоичен низ с дължина $0$. С други думи, $\varepsilon$ е празната функция.
  \end{itemize}
  Тогава $Bin^\infty = (Bin^\infty,\preceq,\varepsilon)$ е област на Скот.
\end{example}

\marginpar{Тези две свойства ще се окажат полезни по-нататък.}
\begin{problem}
  Нека $\chain{a}{i}$ и $\chain{b}{i}$ са вериги в областта на Скот $\A$, за които е изпълнено, че
  $a_i \sqsubseteq b_i$ за всяко $i$.
  Докажете, че $\bigsqcup_i a_i \sqsubseteq \bigsqcup_i b_i$.
\end{problem}

\begin{problem}
  Нека $\chain{a}{i}$ е верига в областта на Скот $\A$ и нека $k$ е естествено число.
  Докажете, че $\bigsqcup_i a_i = \bigsqcup_i a_{i+k}$.
\end{problem}

\begin{problem}
  \marginpar{Езикът $\{a^nb^n\mid n\in\Nat\}$ може да се представи като обединение на безкрайна верига от крайни езици.}
  Нека $\Sigma$ е азбука. Да разгледаме тройката $\mathcal{R} = (Reg, \subseteq, \emptyset)$, където $Reg$ е съвкупността от всички регулярни езици над $\Sigma$.
  Вярно ли е, че $\mathcal{R}$ е област на Скот ?
\end{problem}

\begin{problem}
  Нека $\mathcal{P} = (P, \preceq)$ да бъде частична наредба.
  Да дефинираме частичната наредба $\mathcal{C} = (\texttt{Chain}(P), \sqsubseteq)$, където:
  \begin{itemize}
  \item
    $\texttt{Chain}(P) = \{\ov{x} \mid \ov{x} = \chain{x}{i} \text{ е верига в }P\}$;
  \item
    $\ov{x} \sqsubseteq \ov{x}' \iff (\forall i)[\ x_i \preceq x'_i\ ]$ 
  \end{itemize}
  Докажете, че ако $\mathcal{P}$ формира област на Скот, то
  $\mathcal{C}$ също формира област на Скот.
\end{problem}



%%% Local Variables:
%%% mode: latex
%%% TeX-master: "../sep"
%%% End:


\section{Конструкции на области на Скот}

Ще разгледаме няколко конструкции, с които ще видим как можем да строим все по-сложни области на Скот.

\subsection{Плоска област на Скот}

\Stefan{По принцип плоската област на Скот се дефинира върху друга област на Скот.}

Ще започнем с една проста конструкция, чрез която можем да разширим всяко множество до област на Скот по един почти тривиален начин.

Да фиксираме едно произволно непразно множество $A$ и един елемент $\bot \not \in A$.
Да означим $A_\bot = A \cup \{\bot\}$ и да разгледаме следната бинарна релация $\sqsubseteq$ върху $A_\bot$:
\[a \sqsubseteq b\ \iff\ a = \bot\ \vee\ a = b.\]
Лесно се съобразява, че $\sqsubseteq$ задава {\em частична наредба} върху $A_\bot$:
\marginpar{От деф. на $\sqsubseteq$ следва, че $\bot$ е най-малкият елемент}
\begin{itemize}
\item 
  {\em рефлексивност}: $a \sqsubseteq a$ за всяка $a \in A_\bot$;
\item
  {\em транзитивност}: $a \sqsubseteq b\ \&\ b \sqsubseteq c \implies a\sqsubseteq c$ за всеки $a,b,c \in A_\bot$;
\item
  {\em антисиметричност}: $a \sqsubseteq b\ \&\ b\sqsubseteq a \implies a = b$ за всеки $a,b \in A_\bot$.
\end{itemize}

Наредбата $(A_\bot, \sqsubseteq)$ ще наричаме {\bf плоска наредба}. Тя ще играе важна роля в нашите разглеждания.
\marginpar{$\bot$ се нарича {\em bottom} елемент}
Например, често ще разглеждаме плоската наредба $(\Nat_\bot, \sqsubseteq)$.

\begin{framed}
  \begin{figure}[H]
    \label{fig:flat-nat-1}
    \centering
    \begin{tikzpicture}[shorten >=1pt,->]
      \tikzstyle{vertex}=[circle,minimum size=17pt,inner sep=0pt]
      
      \node[vertex] (bot) at (3,0) {$\bot$};
      \node[vertex] (0) at (0,2) {$0$};
      \node[vertex] (1) at (1,2) {$1$};
      \node[vertex] (2) at (2,2) {$2$};
      \node[vertex] (dots) at (3,2) {$\cdots$};
      \node[vertex] (n) at (4,2) {$n$};
      \node[vertex] (n1) at (5,2) {$n+1$};
      \node[vertex] (ddots) at (6.25,2) {$\cdots$};
      
      \draw (bot) -- node[below left]{$\scriptstyle{\sqsupseteq}$} (0);
      \draw (bot) -- (1);
      \draw (bot) -- (2);
      \draw[dashed] (bot) -- (dots);
      \draw (bot) -- (n);
      \draw (bot) -- (n1);
      \draw[dashed] (bot) -- (ddots);
    \end{tikzpicture}    
    \caption{Графично представяне на плоската наредба $\sqsubseteq$ върху $\Nat_\bot$}
  \end{figure}
  \end{framed}

\begin{proposition}
  \index{област на Скот!плоска}
  Нека $A$ е произволно множество и нека елементът $\bot \not \in A$.
  \marginpar{На англ. {\em flat domain}}
  Определяме наредената тройка $\A_\bot = (A_\bot, \sqsubseteq, \bot)$ като:
  \begin{itemize}
  \item 
    $A_\bot = A\cup\{\bot\}$;
  \item
    $\sqsubseteq$ задава {\em плоската наредба} върху $A_\bot$.
  \end{itemize}
  Тогава $\A_\bot$ е област на Скот, която ще наричаме {\bf плоска област на Скот} за множеството $A$.
\end{proposition}

%%% Local Variables:
%%% mode: latex
%%% TeX-master: "../sep"
%%% End:


\subsection{Крайно произведение}
\label{subsect:domains:product}
\index{област на Скот!крайно произведение}
\marginpar{Това е една от най-простите конструкции върху области на Скот. Вижте например \cite[стр. 125]{winskel} и \cite[176]{models-of-computation}.}

Нека $\A$ и $\B$ са области на Скот.
Тогава $\A \times \B \df (A \times B,\sqsubseteq,\bot)$, където
\begin{itemize}
\item
  $A \times B = \{\pair{a,b} \mid a \in A\ \&\ b \in B\}$;
\item
  $\pair{a,b} \sqsubseteq \pair{a',b'} \iff a \sqsubseteq^\A a'\ \&\ b \sqsubseteq^\B b'$;
\item
  $\bot = \pair{\bot^\A,\bot^\B}$.
\end{itemize}

\begin{framed}
  \begin{proposition}
    \label{pr:cartesian}
    Ако $\A$ и $\B$ са области на Скот, то $\A \times \B$ е област на Скот.
  \end{proposition}  
\end{framed}
\begin{hint}
  Лесно се съобразява, че $\sqsubseteq$ е частична наредба и че $\bot$ е най-малкият елемент.
  Да разгледаме една верига $\{\pair{a_i,b_i}\}^\infty_{i=1}$ в $\A \times \B$.
  Лесно се вижда, че
  \[\bigsqcup_i\pair{a_i,b_i} = \pair{\bigsqcup_ia_i,\bigsqcup_ib_i}.\]
  \begin{itemize}
  \item
    За произволен елемент $\pair{a_i,b_i}$ от веригата е ясно, че $a_i \sqsubseteq^\A \bigsqcup_ia_i$ и $b_i \sqsubseteq^\B \bigsqcup_i b_i$.
    Следователно, $\pair{\bigsqcup_ia_i,\bigsqcup_ib_i}$ е горна граница на веригата.
  \item
    Нека $\pair{c,d}$ е горна граница на веригата, т.е. за всяко $i$,
    $a_i \sqsubseteq^\A c$ и $b_i \sqsubseteq^\B d$.
    Но тогава $c$ е горна граница на веригата $\chain{a}{i}$ и следователно $\bigsqcup_ia_i \sqsubseteq^\A c$.
    Също така, $d$ е горна граница на веригата $\chain{b}{i}$ и следователно $\bigsqcup_i b_i \sqsubseteq^\B d$.
    Заключаваме, че $\pair{\bigsqcup_ia_i,\bigsqcup_ib_i} \sqsubseteq \pair{c,d}$, т.е. $\pair{\bigsqcup_ia_i,\bigsqcup_ib_i}$
    е точна горна граница на веригата.
  \end{itemize}
\end{hint}

Нека $\A_1,\dots,\A_n$, $n \geq 2$, са области на Скот. Дефинираме 
$\prod^n_{i=1}\A_i = (A, \sqsubseteq, \bot)$ по следния начин:
\begin{itemize}
\item
  Ако $n = 2$, то $\prod^2_{i=1} \A_i \df \A_1 \times \A_2$.
\item
  Ако $n > 2$, то $\prod^n_{i=1} \A_i \df (\prod^{n-1}_{i=1}\A_i) \times \A_n$.
\end{itemize}

Използвайки \Prop{cartesian}, лесно се съобразява следното твърдение.
\begin{proposition}
  Ако $\A_1,\dots,\A_n$, $n \geq 2$, са области на Скот, то $\prod^n_{i=1}\A_i$ е област на Скот.
\end{proposition}

\begin{framed}
  \begin{figure}[H]
    \centering
    \begin{tikzpicture}[shorten >=1pt,->]
      \tikzstyle{vertex}=[circle,minimum size=17pt,inner sep=0pt]
      
      \node[vertex] (bot) at (5,-1) {$\pair{\bot,\bot}$};
      
      \node[vertex] (0b) at (0,1) {$\pair{0,\bot}$};
      \node[vertex] (db) at (1.25,1) {$\cdots$};
      \node[vertex] (nb) at (2.5,1) {$\pair{n,\bot}$};
      \node[vertex] (ddb) at (4,1) {$\cdots$};
      
      \node[vertex] (b0) at (6,1) {$\pair{\bot,0}$};
      \node[vertex] (bd) at (7.25,1) {$\cdots$};
      \node[vertex] (bn) at (8.5,1) {$\pair{\bot,n}$};
      \node[vertex] (bdd) at (9.5,1) {$\cdots$};
      
      \node[vertex] (00) at (-1,3) {$\pair{0,0}$};
      \node[vertex] (01) at (0.25,3) {$\pair{0,1}$};
      \node[vertex] (10) at (1.5,3) {$\pair{1,0}$};
      \node[vertex] (ddd) at (2.75,3) {$\cdots$};
      \node[vertex] (0n) at (4,3) {$\pair{0,n}$};
      \node[vertex] (dddd) at (5.25,3) {$\cdots$};
      \node[vertex] (n0) at (6.5,3) {$\pair{n,0}$};
      \node[vertex] (ddddd) at (7.75,3) {$\cdots$};
      \node[vertex] (nn) at (9,3) {$\pair{n,n}$};
      \node[vertex] (dddddd) at (10,3) {$\cdots$};

      \draw (bot) -- node[below left]{$\scriptstyle{\sqsupseteq}$} (0b);
      \draw (bot) -- (nb);
      \draw (bot) -- (b0);
      \draw (bot) -- (bn);

      \draw (0b) -- node[below left]{$\scriptstyle{\sqsupseteq}$} (00);
      \draw (0b) -- (01);
      \draw (0b) -- (0n);

      \draw (b0) -- (00);
      \draw (b0) -- (10);
      \draw (b0) -- (n0);
      \draw (nb) -- (n0);
      
      \draw (nb) -- (nn);
      \draw (bn) -- (0n);
      \draw (bn) -- (nn);

    \end{tikzpicture}    
    \caption{Графично представяне на част от $\sqsubseteq$ върху $\Nat^2_\bot$}
    \label{fig:flat-nat-2}
  \end{figure}
\end{framed}

Вижда се от \Figure{flat-nat-2}, че всяка верига в $\Nat^2_\bot$ има дължина най-много $3$.
Лесно се съобразява, че всяка верига в $\Nat^k_\bot$ има дължина най-много $k+1$.
Свойството, че всяка верига в $\Nat^k_\bot$ има само краен брой различни члена
ще се окаже важно по-нататък. Сега ще въведем понятие, което описва това свойство в произволна област на Скот.
\marginpar{ $\Nat^k_\bot \df \underbrace{\Nat_\bot \times \cdots \times\Nat_\bot}_{k}$. Какво става, когато $k = 0$ ? Това ще бъде важно по-нататък.}

Нека $\A$ е област на Скот и да разгледаме една верига $\chain{a}{n}$ в $\A$.
Ще казваме, че $\chain{a}{n}$ се {\bf стабилизира}, ако съществува индекс $n_0$, за който
\[(\forall n \geq n_0)[a_{n_0} = a_{n}],\]
т.е.
\[a_0 \sqsubseteq a_1 \sqsubseteq a_2 \sqsubseteq \cdots \sqsubseteq a_{n_0} = a_{n_0+1} = a_{n_0+2} = \cdots\]
От казаното по-горе следва, че всяка растяща верига в $\Nat^k_\bot$ се стабилизира.

\marginpar{Едни от основните области на Скот, които ще разглеждаме при дефинирането на денотационната семантика ще бъдат $\Nat_\bot$ и $\Nat^k_\bot$.}


% Добре е още тук да отбележим, че конструкцията за декартово произведение, която разглеждаме тук е опростен вариант на това, което имаме в хаскел.
% Да разгледаме един пример.
% \begin{haskellcode}
% ghci> :set +m -- multiline definitions
% ghci> let
%  | h :: (a,a) -> Int
%  | h _ = 4
%  | 
% ghci> h (2,3)
% 4
% ghci> h (2,undefined)
% 4
% ghci> h (undefined,undefined)
% 4
% ghci> h undefined
% 4
% ghci> 
% \end{haskellcode}

% Това означава, че ако искаме да моделираме по-точно декартово произведение на типове в хаскел чрез области на Скот, то би трябвало в областта на Скот $\A \times \B$
% да добавим \emph{нов} елемент $\bot^{\A\times\B}$, за който $\bot^{\A\times\B} \sqsubseteq (\bot^\A,\bot^\B)$.
% Ние не правим това, за да си улесним живота.

%%% Local Variables:
%%% mode: latex
%%% TeX-master: "../sep"
%%% End:


\section{Изображения в области на Скот}

\index{област на Скот!изображения}
Нека $\A_i = (A_i,\ \sqsubseteq_i,\ \bot_i)$, за $i = 1,2$, са области на Скот.
Ще въведем няколко основни видове изображения между $\A_1$ и $\A_2$, 
които ще използваме често. След това ще разгледаме свойства на тези изображения
и ще видим каква е връзката между тях.
\begin{itemize}
\item
  Всяка тотална функция от вида $f:A_1 \to A_2$ ще наричаме изображение между областите на Скот $\A_1$ и $\A_2$
  и ще записваме $f:\A_1 \to \A_2$.
  \marginpar{Тук $\bm{\bot}$ е различно от $\bot$, макар и трудно да се различават графически. Освен това, $\A$ е различно от $A$.}
  Да въведем означението 
  \[\Mapping{\A_1}{\A_2} \df (\{f \mid f:\A_1 \to \A_2\},\ \sqsubseteq,\ \bm{\bot}),\]
  \marginpar{\todo Сами проверете, че така дефинираната релация $\sqsubseteq$ задава частична наредба, при положение, че $\sqsubseteq_2$ е частична наредба.}
  където имаме следната релация между изображенията $f,g:\A_1 \to \A_2$:
  \[f \sqsubseteq g \dfff (\forall a \in \A_1)[f(a) \sqsubseteq_2 g(a)].\]
  Също така, изображението $\bm{\bot}:\A_1 \to \A_2$ е дефинирано като
  \[(\forall a \in \A_1)[\bm{\bot}(a) = \bot_2].\]
\end{itemize}

На хаскел можем да дефинираме изображението $\bm{\bot}$ по следния начин:
\begin{haskellcode}
ghci> let bottom _ = undefined
\end{haskellcode}

\begin{framed}
  \begin{theorem}
    \label{th:all-mappings-is-domain}
    $\Mapping{\A_1}{\A_2}$ е област на Скот.
  \end{theorem}  
\end{framed}
\begin{proof}
  Нетривиалната част в доказателството е да проверим, че всяка верига $(f_i)^{\infty}_{i=0}$ в $\Mapping{\A_1}{\A_2}$
  притежава точна горна граница.
  % \marginpar{За по-кратко пишем $\bar{a}$ вместо $a_1,\dots,a_n$}
  Да разгледаме изображението $h:\A_1 \to \A_2$, където:
  \begin{equation}
    \label{eq:9}
    h(a) \df \bigsqcup \{f_i(a) \mid i \in \Nat\}.
  \end{equation}
  Ще докажем, че $h$ е тази точна горна граница.
  \begin{itemize}
  \item
    \marginpar{Това задължително трябва да се провери,
      защото например множеството $\{\bot, 0, 3\}$ \emph{няма} точна горна граница относно плоската наредба в $\Nat_\bot$.}
    Първо, трябва да се убедим, че дефиницията на $h$ е ,,смислена'', т.е. $h$ е тотална функция.
    Трябва да докажем, че за всяко $a\in \A_1$, точната горна граница 
    \[\bigsqcup\{f_i(a) \mid i \in \Nat\}\] съществува.
    Да фиксираме произволен елемент $a \in \A_1$.
    Получаваме следната верига в $\A_2$:
    \[f_0(a) \sqsubseteq f_1(a) \sqsubseteq f_2(a) \sqsubseteq \cdots \]
    Понеже $\A_2$ е област на Скот, то тази верига притежава точна горна граница в $\A_2$,
    която означачаваме като $\bigsqcup\{f_i(a) \mid i \in \Nat\}$, а според нашата дефиниция,
    това е точно $h(a)$.
    Това означава, че $h$ е тотална функция.
  \item
    Дотук имаме, че $h \in \Mapping{\A_1}{\A_2}$.
    Лесно се съобразява, че $h$ е горна граница на веригата $\chain{f}{i}$, защото за всеки елемент $a \in \A_1$
    и произволен индекс $k$,
    \[f_k(a) \sqsubseteq \bigsqcup\{f_i(a) | i \in \Nat\} \df h(a).\]
  \item
    Сега остава да проверим, че $h$ е точна горна граница, т.е. $h$ е най-малката измежду всички горни граници на 
    веригата $\chain{f}{i}$.
    Нека $g$ е друга горна граница на $\chain{f}{i}$. Това означава, че за всеки индекс $i$,
    $f_i \sqsubseteq g$. Следователно, за фиксирано $a \in \A_1$,
    $g(a)$ е горна граница за веригата $(f_i(a))^{\infty}_{i=0}$.
    Тогава е ясно, че за разглеждания елемент $a$,
    \[h(a) \df \bigsqcup\{f_i(a) \mid i\in\Nat\} \sqsubseteq g(a).\]
    Понеже елементът $a$ е прозиволен, получаваме, че $h \sqsubseteq g$.
  \item
    Доказахме, че $h$ е горна граница и че $h$ е най-малката измежду всички горни граници.
    Заключваме, че $h$ е {\em точната горна граница} на веригата $\chain{f}{i}$.
    \marginpar{Получаваме, че \[(\bigsqcup_if_i)(a) = \bigsqcup_i\{f_i(a)\}.\] Добре е да свикнете с тези означения, защото по-нататък ще ги използваме често.}
    С други думи,
    \[h = \bigsqcup_i f_i.\]
  \end{itemize}
\end{proof}

Завършваме с един важен частен случай на горната теорема.
\begin{framed}
  \begin{corollary}
    $\Mapping{\Nat^n_\bot}{\Nat_\bot}$ е област на Скот.
  \end{corollary}
\end{framed}


%%% Local Variables:
%%% mode: latex
%%% TeX-master: "../sep"
%%% End:


\subsection{Монотонни изображения}

\index{изображение!монотонно}
Да разгледаме областите на Скот $\A_1 = (A_1,\ \sqsubseteq_1,\ \bot_1)$ и $\A_2 =(A_2,\ \sqsubseteq_2,\ \bot_2)$.
Едно изображение $f:\A_1\to \A_2$ се нарича {\bf монотонно}, ако
\[(\forall a,a'\in\A_1)[a \sqsubseteq_1 a' \implies f(a) \sqsubseteq_2 f(a')].\]
Да въведем означението
\[\Mon{\A_1}{\A_2} \df (\{f: \A_1 \to \A_2 \mid f\text{ е мон. изобр.}\},\ \sqsubseteq,\ \bm{\bot}).\]

\begin{framed}
  \begin{theorem}\label{th:monotone-is-domain}
    $\Mon{\A_1}{\A_2}$ е област на Скот.
  \end{theorem}  
\end{framed}
\begin{hint}
  Да фиксираме една верига $\chain{f}{i}$ в $\Mon{\A_1}{\A_2}$. Трябва да докажем, че тази верига притежава точна горна граница,
  която е монотонно изображение.
  Да разгледаме същото изображение $h:\A_1 \to \A_2$ както в доказателството на \Th{all-mappings-is-domain}, като
  \[h(a) \df \bigsqcup \{f_i(a) \mid i \in \Nat\}.\]
  Оттам знаем, че $h$ е точна горна граница на веригата. 
  Остава да докажем, че $h \in \Mon{\A_1}{\A_2}$.
  Нека $a \sqsubseteq b$. Тогава, за всеки индекс $k$, понеже $f_k$ са монотонни изображения, получаваме следното:
  \[f_k(a) \sqsubseteq f_k(b) \sqsubseteq \bigsqcup \{f_i(b) \mid i \in \Nat\} \df h(b).\]
  Това означава, че $h(b)$ е горна граница за веригата ${(f_i(a))}^{\infty}_{i=0}$.
  Заключаваме, че 
  \begin{align*}
    h(a) & \df \bigsqcup \{f_i(a) \mid i \in \Nat\}\\
         & \sqsubseteq \bigsqcup \{f_i(b) \mid i \in \Nat\} \df h(b).
  \end{align*}
\end{hint}

Оттук като частен случай получаваме следното полезно следствие.
\begin{framed}
  \begin{corollary}\label{cr:flat-monotone-is-domain}
    $\Mon{\Nat^n_\bot}{\Nat_\bot}$ е област на Скот.
  \end{corollary}
\end{framed}

Удобно е да имаме следващото свойство на монотонните изображения като отделно твърдение за да можем да
се позоваваме на него, когато имаме нужда от него по-късно.

\begin{proposition}\label{pr:monotone-chain}
  Нека $f \in \Mon{\A}{\B}$ и $\chain{a}{i}$ е верига от елементи на $\A$.
  Тогава
  \[\bigsqcup_i f(a_i) \sqsubseteq f(\bigsqcup_i a_i).\]
\end{proposition}
\begin{proof}
  Понеже $f$ е монотонно, то ${(f(a_i))}^{\infty}_{i=0}$ е верига от елементи на областта на Скот $\B$
  и тя има точна горна граница.
  Достатъчно е да докажем, че $f(\bigsqcup_i a_i)$ е горна граница на веригата ${(f(a_i))}^{\infty}_{i=0}$.
  Но това е лесно.
  Понеже за произволен индекс $k$, $a_k \sqsubseteq \bigsqcup_i a_i$ и $f$ е монотонно изображение, то веднага получаваме, че
  $f(a_k) \sqsubseteq f(\bigsqcup_i a_i)$.
  Заключаваме, че $\bigsqcup_k f(a_k) \sqsubseteq f(\bigsqcup_i a_i)$.
\end{proof}

%%% Local Variables:
%%% mode: latex
%%% TeX-master: "../sep"
%%% End:


\subsection{Непрекъснати изображения}\index{изображение!непрекъснато}

\marginpar{На англ. {\em continuous}}
Едно изображение $f:\A_1\to \A_2$ се нарича {\bf непрекъснато}, ако са изпълнени свойствата:
\begin{itemize}
\item
  $f$ е монотонно изображение;
\item
  при всеки избор на верига $\chain{a}{i}$ в $\A_1$, имаме равенството
  \marginpar{Понеже $\A_1$ е област на Скот знаем, че $\bigsqcup_i a_i \in \A_1$}
  \marginpar{Понеже $f$ е монотонно, то ${(f(a_i))}^{\infty}_{i=0}$ е верига.}
  \marginpar{$\bigsqcup_i f(a_i) \df \bigsqcup\{f(a_i) \mid i \in \Nat\}$}
  \[f(\bigsqcup_i a_i) = \bigsqcup \{f(a_i) \mid i \in \Nat\}.\]  
\end{itemize}
Да означим
\[\Cont{\A_1}{\A_2} \df (\{f: \A_1 \to \A_2 \mid f\text{ е непр. изобр.}\},\ \sqsubseteq,\ \bm{\bot}).\]

% \begin{framed}
%   \begin{prop}
%     \label{pr:continuous-is-monotone}
%     За произволни области на Скот $\A_1$ и $\A_2$, всяко непрекъснато изображение $f:\A_1 \to \A_2$ е монотонно, т.е.
%     \[\Cont{\A_1}{\A_2}\ \subseteq\ \Mon{\A_1}{\A_2}.\]
%   \end{prop}
% \end{framed}
% \begin{proof}
%   Нека $f \in \Cont{\A_1}{\A_2}$.
%   Да вземем два произволни елемента в $\A_1$, за които $a \sqsubseteq_1 b$.
%   Ще докажем, че $f(a) \sqsubseteq_2 f(b)$.
%   Да разгледаме веригата $\chain{a}{i}$ в $\A_1$, където:
%   \[\underbrace{a_0}_{a} \sqsubseteq \underbrace{a_1}_{b} = \underbrace{a_2}_{b} = \underbrace{a_3}_{b} = \cdots\]
%   Ясно е, че 
%   \[\bigsqcup\{a_i \mid i \in \Nat\} = \bigsqcup\{a,b\} = b.\]
%   Тогава от непрекъснатостта на $f$ имаме, че
%   \begin{align*}
%     f(a) & = f(a_0) & \comment{\text{защото }a_0 \df a}\\
%     & \sqsubseteq_2 \bigsqcup\{f(a_i) \mid i \in \Nat\} & \comment{\text{защото }f(a_0) \in \{f(a_i) \mid i\in\Nat\}}\\
%     & = f(\bigsqcup_i a_i) & \comment{\text{защото $f$ е непр.}}\\
%     & = f(\bigsqcup\{a,b,b,b,\dots\}) & \comment{\text{от избора на веригата }\chain{a}{i}}\\
%     & = f(b) & \comment{\text{защото }a \sqsubseteq_1 b}.
%   \end{align*}
%   Така получихме, че за произволни $a,b\in\A_1$, 
%   \[a \sqsubseteq_1 b \implies f(a) \sqsubseteq_2 f(b).\]
% \end{proof}

Ясно е, че всяко непрекъснато изображение е монотонно.
Ествено е да си зададем въпроса дали имаме и обратното включване.
Оказва се, че в общия случай не е вярно, че всяко монотонно изображение е непрекъснато.

\begin{proposition}\label{pr:continuous-mappings-not-monotone}  
  Съществува област на Скот $\A$, за която
  \[\Cont{\A}{\A} \subsetneqq \Mon{\A}{\A}.\]
\end{proposition}
\begin{hint}
  Нека $A = \{a_n \mid n \in \Nat\} \cup \{a_\omega, b\}$.
  Да разгледаме областта на Скот $\A = (A, \sqsubseteq, a_0)$, където 
  наредбата между елементите е следната:
  \[a_0 \sqsubseteq a_1 \sqsubseteq \cdots \sqsubseteq a_n \sqsubseteq \cdots \sqsubseteq a_\omega \sqsubseteq b. \]
  Нека $f(a_n) = a_{n+1}$, $f(a_{\omega}) = b$ и $f(b) = b$.
  Очевидно е, че $f$ е монотонно изображение.
  Лесно се вижда, че $f$ не е непрекъснато изображение, 
  защото
  \[f(\bigsqcup_n a_n) = f(a_\omega) = b,\]
  но 
  \[\bigsqcup_n f(a_n) = \bigsqcup_n a_{n+1} = a_\omega.\]
\end{hint}

Сега да видим един важен за нас случай, при който имаме и обратното включване.
\marginpar{В Раздел~\ref{subsect:domains:product} дефинирахме какво означава една верига $\chain{n}{a}$ да се стабилизира: съществува индекс $n_0$, за който \[(\forall n \geq n_0)[a_{n_0} = a_{n}].\]}

\begin{framed}
  \begin{proposition}\label{pr:stab-continuous}
    Ако всяка верига в $\A_1$ се {\em стабилизира}, то
    \[\Mon{\A_1}{\A_2} \subseteq \Cont{\A_1}{\A_2}.\]
  \end{proposition}
\end{framed}
\begin{hint}
  Да разгледаме една верига $\chain{a}{i}$ в $\A_1$ и $f \in \Mon{\A_1}{\A_2}$.
  Ще докажем, че \[f(\bigsqcup_i a_i) = \bigsqcup_i f(a_i).\]

  \begin{enumerate}[(1)]
  \item
    % \marginpar{Това включване е вярно за произволна област на Скот $\A_1$.}
    % Ясно е, че за всяко монотонно изображение $f$,
    % понеже $a_i \sqsubseteq \bigsqcup_i a_i$, то $f(a_i) \sqsubseteq f(\bigsqcup_i a_i)$.
    % Това означава, че $f(\bigsqcup_i a_i)$ е горна граница на веригата $(f(a_i))^\infty_{i=0}$ в $\A_2$
    % и следователно
    % \[\bigsqcup_i f(a_i) \sqsubseteq f(\bigsqcup_i a_i).\]
    От \Prop{monotone-chain} веднага получаваме, че
    \[\bigsqcup_i f(a_i) \sqsubseteq f(\bigsqcup_i a_i).\]
  \item
    За другата посока ще използваме свойството, че веригата $\chain{a}{i}$ се стабилизира.
    Нека $n_0$ е индекс, такъв че $(\forall k \geq n_0)[a_k = a_{n_0}]$.
    Това означава, че $\bigsqcup_i a_i = a_{n_0}$.
    Тогава
    \[f(\bigsqcup_i a_i) = f(a_{n_0}) \sqsubseteq \bigsqcup_i f(a_i).\]
  \end{enumerate}
  
  От $(1)$ и $(2)$ следва, че $f(\bigsqcup_i a_i) = \bigsqcup_i f(a_i)$.
\end{hint}

От Раздел~\ref{subsect:domains:product} знаем, че всяка верига в $\Nat^n_\bot$ се {\em стабилизира}, то
получаваме следното важно следствие.
\begin{framed}
\begin{corollary}\label{cr:monotone-is-continuous}
  $\Mon{\Nat^n_\bot}{\Nat_\bot} = \Cont{\Nat^n_\bot}{\Nat_\bot}$.
\end{corollary}  
\end{framed}

% От доказателството на (1) за \Prop{stab-continuous} можем да извлечем следното свойство,
% което ще ни бъде полезно по-нататък.
% \begin{proposition}\label{pr:monotone-chain}
%   За всяко изображение $f \in \Mon{\A}{\B}$ и всяка верига $\chain{a}{i}$, е изпълнено, че
%   \[\bigsqcup_i f(a_i) \sqsubseteq f(\bigsqcup_i a_i).\]
% \end{proposition}

Понеже от \Cor{flat-monotone-is-domain} имаме, че монотонните изображения образуват област на Скот, 
то директно получаваме следната важна теорема.

\begin{framed}
\begin{theorem}
  \label{th:continuous-is-domain}
  $\Cont{\Nat^n_\bot}{\Nat_\bot}$ е област на Скот.
\end{theorem}
\end{framed}
% \begin{proof}
%   От \Cor{monotone-is-continuous} имаме, че 
%   \[\Mon{\Nat^n_\bot}{\Nat_\bot} = \Cont{\Nat^n_\bot}{\Nat_\bot}.\]
%   От \Cor{flat-monotone-is-domain} имаме, че 
%   \[(\Mon{\Nat^n_\bot}{\Nat_\bot}, \sqsubseteq, \Omega^{(n)})\]
%   е област на Скот. 
%   Оттук директно получаваме, че 
%   \[(\Cont{\Nat^n_\bot}{\Nat_\bot}, \sqsubseteq, \Omega^{(n)})\] е област на Скот.
% \end{proof}


% \begin{prop}
%   \label{pr:composition}
%   \index{изображения!композиция}
%   Ако $f \in \Cont{\A}{B}$ и $g \in \Cont{\B}{\C}$, то $g \circ f \in \Cont{\A}{\C}$,
%   където \[(g\circ f)(a) \df g(f(a)).\]
% \end{prop}
% \begin{hint}
%   Нека $\chain{a}{i}$ е верига в $\A$.
%   Да обърнем внимание, че понеже $f \in \Cont{\A}{\B}$,
%   то $f$ е монотонно изображение и тогава $(f(a_i))^\infty_{i=0}$ е верига в $\B$.
%   Тогава:
%   \begin{align*}
%     (g \circ f)(\bigsqcup_i a_i) & = g(f(\bigsqcup_i a_i)) & \comment{\text{от деф.}}\\
%     & = g(\bigsqcup_i f(a_i)) & \comment{f \text{ е непр.}}\\
%     & = \bigsqcup_i g(f(a_i)) & \comment{g \text{ е непр.}}
%   \end{align*}
% \end{hint}


\begin{problem}
  \marginpar{\cite[стр. 124]{reynolds}}
  Нека $f \in \Mon{\A}{\B}$ и $g \in \Mon{\B}{\A}$ имат свойствата:
  \begin{itemize}
  \item 
    $f\circ g = id_\B$;
  \item
    $g \circ f = id_\A$.
  \end{itemize}
  Докажете, че $f$ и $g$ са непрекъснати.
\end{problem}


% \begin{problem}
%   Нека $f_0 \sqsubseteq f_1 \sqsubseteq f_2 \sqsubseteq \cdots$
%   е верига от елементи на $\Cont{\A}{\A}$.
%   Да положим $h = \bigsqcup_n f_n$.
%   Вярно ли е, че 
%   \[h \circ h = \bigsqcup_n (f_n \circ f_n)?\]
%   Обосновете се!
% \end{problem}


\marginpar{Много от задачите са от \cite[стр. 31]{abramsky94}}

\begin{problem}
  Да разгледаме непрекъснатите изображения
  \[\Gamma,\Delta \in \Cont{\Cont{\A}{\A}}{\Cont{\A}{\A}}.\]
  \marginpar{Знаем, че изображението $\Gamma \circ \Delta$ е непрекъснато, където
    \[(\Gamma\circ\Delta)(f) \df \Gamma(\Delta(f)).\]}
  Винаги ли е вярно, че 
  \[\lfp(\Gamma \circ \Delta) \sqsubseteq \lfp(\Gamma) \circ \lfp(\Delta)?\]
  Обосновете се!
\end{problem}
% \ifhints
\begin{hint}
  Ще дадем контрапример.
  Нека $\A = \Nat_\bot$.
  Нека например да разгледаме
  \begin{align*}
    & \Delta(f)(x) \df f(x+1)\\
    & \Gamma(f)(x) \df
      \begin{cases}
        0, & x \neq \bot\\
        \bot, & x = \bot.
      \end{cases}
  \end{align*}
  Да положим $f_\Gamma \df \lfp(\Gamma)$ и $f_\Delta \df \lfp(\Delta)$.
  Ясно е, че 
  \begin{align*}
    & f_\Delta(x) = \bot\\
    & f_\Gamma(x) =
    \begin{cases}
      0, & x \neq \bot\\
      \bot, & x = \bot.
    \end{cases}  
  \end{align*}
  Тогава за произволно $x \in \Nat_\bot$,
  \[(f_\Gamma\circ f_\Delta)(x) = f_\Gamma(f_\Delta(x)) = f_\Gamma(\bot)  = \bot.\]
  От друга страна, понеже $(\Gamma \circ \Delta)(f) = \Gamma(\Delta(f))$, то 
  \begin{align*}
    & (\Gamma \circ \Delta)(f)(x) = \Gamma(\Delta(f))(x) = 
      \begin{cases}
        0, & x \neq \bot\\
        \bot, & x = \bot.
      \end{cases}
  \end{align*}
  Лесно се съобразява, че 
  \[\lfp(\Gamma \circ \Delta)(x) =
  \begin{cases}
    0, & x \neq \bot\\
    \bot, & x = \bot.
  \end{cases}\]
  Заключаваме, че 
  \[\lfp(\Gamma \circ \Delta) \sqsupset \lfp(\Gamma) \circ \lfp(\Delta).\]
\end{hint}
% \fi


\begin{problem}
  Да разгледаме операторите \[\Gamma,\Delta \in \Cont{\Cont{\A}{\A}}{\Cont{\A}{\A}}.\]
  Знаем, че операторът $\Gamma \circ \Delta$ е непрекъснат, където
  \[(\Gamma\circ\Delta)(f) \df \Gamma(\Delta(f)).\]
  Винаги ли е вярно, че:
  \[\lfp(\Gamma \circ \Delta) \sqsupseteq \lfp(\Gamma) \circ \lfp(\Delta)?\]
  Обосновете се!
\end{problem}
% \ifhints
\begin{hint}
  Ще дадем контрапример.
  Нека $\A = \Nat_\bot$.
  Нека например да разгледаме
  \begin{align*}
    & \Delta(f)(x) \df 0\\
    & \Gamma(f)(x) \df
      \begin{cases}
        0, & x = 0\\
        \bot, & \text{ иначе}.
      \end{cases}
  \end{align*}
  Да положим $f_\Gamma \df \lfp(\Gamma)$ и $f_\Delta \df \lfp(\Delta)$.
  Ясно е, че 
  \begin{align*}
    & f_\Delta(x) = 0\\
    & f_\Gamma(x) =
    \begin{cases}
      0, & x = 0\\
      \bot, & \text{ иначе}.
    \end{cases}  
  \end{align*}
  Тогава за произволно $x \in \Nat_\bot$,
  \[(f_\Gamma\circ f_\Delta)(x) = f_\Gamma(f_\Delta(x)) = f_\Gamma(0)  = 0.\]
  От друга страна, понеже $(\Gamma \circ \Delta)(f) = \Gamma(\Delta(f))$, то 
  \begin{align*}
    & (\Gamma \circ \Delta)(f)(x) = \Gamma(\Delta(f))(x) = 
      \begin{cases}
        0, & x = 0\\
        \bot, & \text{ иначе}.
      \end{cases}
  \end{align*}
  Лесно се съобразява, че 
  \[\lfp(\Gamma \circ \Delta)(x) =
  \begin{cases}
    0, & x = 0\\
    \bot, & \text{ иначе}.
  \end{cases}\]
  Заключаваме, че 
  \[\lfp(\Gamma \circ \Delta) \sqsubset \lfp(\Gamma) \circ \lfp(\Delta).\]
\end{hint}
% \fi


%%% Local Variables:
%%% mode: latex
%%% TeX-master: "../sep"
%%% End:


\section{Област на Скот от непрекъснати изображения}

Следващата теорема е важна, защото с нейна помощ се доказват много свойства на непрекъснатите изображения от по-висок ред.

\begin{theorem}\label{th:double-chain}
  \marginpar{\cite[стр. 127]{winskel}}
  \marginpar{\cite[стр. 178]{models-of-computation}}
  Нека $\A = (A,\sqsubseteq,\bot)$ да бъде област на Скот и нека множеството 
  \[\{a_{m,n} \mid m,n \in \Nat\}\] от елементи на $A$ притежава свойството, че 
  \[n \leq n^\prime\ \&\ m \leq m^\prime\ \Rightarrow\ a_{n,m} \sqsubseteq a_{n^\prime,m^\prime}.\]
  Тогава са изпълнени равенствата
  \[\bigsqcup_m(\bigsqcup_n a_{n,m}) = \bigsqcup_n(\bigsqcup_{m} a_{n,m}) = \bigsqcup_n a_{n,n}.\]
\end{theorem}
\begin{proof}
  Първо ще въведем някои означения.
  \begin{itemize}
  \item 
    Да фиксираме произволно $m$. Тогава множеството $\{a_{n,m} \mid n \in \Nat\}$ образува верига:
    \[a_{0,m} \sqsubseteq a_{1,m} \sqsubseteq a_{2,m} \sqsubseteq \cdots\]
    \marginpar{По дефиниция, всяка монотонно растяща редица в област на Скот притежава точна горна граница.}
    Следователно тя има точна горна граница $b_m \df \bigsqcup \{a_{n,m} \mid n \in \Nat\}$.
  \item
    Аналогично, при фиксирано $n$, множеството $\{a_{n,m} \mid m \in \Nat\}$ образува верига:
    \[a_{n,0} \sqsubseteq a_{n,1} \sqsubseteq a_{n,2} \sqsubseteq \ldots,\]
    която притежава точна горна граница $c_n \df \bigsqcup \{a_{n,m} \mid m \in \Nat\}$.
  \end{itemize}
  Това означава, че трябва да докажем следното:
  \[\bigsqcup_m b_m = \bigsqcup_n c_n = \bigsqcup_n a_{n,n}.\]
  \begin{enumerate}[1)]
  \item 
    Първо да съобразим, че множеството $\{b_m \mid m \in \Nat\}$ образува верига в $\A$ и следователно притежава точна горна граница $\bigsqcup_m b_m$.
    Нека да разгледаме произволни $m \leq m^\prime$.
    Тогава \[(\forall n)[a_{n,m} \sqsubseteq a_{n,m^\prime} \sqsubseteq \bigsqcup_k a_{k,m^\prime} = b_{m^\prime}].\]
    Следователно $b_{m^\prime}$ е горна граница на веригата $(a_{n,m})^{\infty}_{n=0}$ и понеже $b_m$ е точна горна граница на $(a_{n,m})^{\infty}_{n=0}$, то получаваме, че \[b_m \sqsubseteq b_{m^\prime}.\]
    Това означава, че $\chain{b}{m}$ е верига в $\A$ и тя притежава точна горна граница $\bigsqcup_m b_m$.  
  \item
    С подобни разсъждения можем да докажем, че множеството $\{c_n \mid n \in \Nat\}$ образува верига в $\A$, която притежава точна горна граница $\bigsqcup_n c_n$.
  \item
    Сега ще докажем, че \[\bigsqcup_m b_m = \bigsqcup_n c_n.\]
    Имаме, че 
    \[(\forall m)(\forall n)[a_{n,m} \sqsubseteq \bigsqcup_{i}a_{i,m} = b_m \sqsubseteq \bigsqcup_i b_i],\]
    което е еквивалентно на 
    \[(\forall n)(\forall m)[a_{n,m} \sqsubseteq b_m \sqsubseteq \bigsqcup_i b_i].\]
    Да фиксираме произволно $n$.
    Тогава $\bigsqcup_i b_i$ е горна граница на веригата $(a_{n,i})^\infty_{i=0}$.
    Следователно, $c_n = \bigsqcup_i a_{n,i} \sqsubseteq \bigsqcup_i b_i$.
    Така получаваме, че $\bigsqcup_i b_i$ е горна граница и на веригата $\chain{c}{n}$
    и тогава \[\bigsqcup_n c_n \sqsubseteq \bigsqcup_i b_i.\]
    С аналогични разсъждения можем да докажем също, че 
    \[\bigsqcup_m b_m \sqsubseteq \bigsqcup_n c_n.\]
    Така доказахме, че \[\bigsqcup_m b_m = \bigsqcup_n c_n.\]
  \item
    Достатъчно е още да докажем, че
    \[\bigsqcup_n a_{n,n} = \bigsqcup_n c_n.\]
    Ясно е, че $a_{n,n}$ е елемент на веригата ${(a_{n,m})}^{\infty}_{m=0}$ и следователно
    $a_{n,n} \sqsubseteq \bigsqcup_m a_{n,m} = c_n \sqsubseteq \bigsqcup_n c_n$.
    Получаваме, че $\bigsqcup_n c_n$ е горна граница на веригата ${(a_{n,n})}^{\infty}_{n=0}$
    и следователно $\bigsqcup_n a_{n,n} \sqsubseteq \bigsqcup_n c_n$.
    
    За другата посока, да разгледаме произволен елемент $a_{n,m}$.
    Нека $k = \max\{n,m\}$.
    Ясно е, че $a_{n,m} \sqsubseteq a_{k,k} \sqsubseteq \bigsqcup_n a_{n,n}$.
    Следователно, $\bigsqcup_n a_{n,n}$ е горна граница на верига ${(a_{m,n})}^{\infty}_{m=0}$
    и оттук получаваме, че за всяко $n$, $c_n \sqsubseteq \bigsqcup_n a_{n,n}$.
    Получаваме, че $\bigsqcup_n a_{n,n}$ е горна граница на веригата $\chain{c}{n}$ и следователно
    $\bigsqcup_n c_n \sqsubseteq \bigsqcup_n a_{n,n}$.
  \end{enumerate}
  С това доказателството на теоремата е завършено.
\end{proof}

\begin{framed}
  \begin{lemma}
    Нека $\A$ и $\B$ са области на Скот.
    Нека $\chain{f}{k}$ е верига от елементи на $\Cont{\A}{\B}$.
    Да дефинираме изображението $h$ на $\A$ в $\B$ по следния начин
    \[h(a) \df \bigsqcup\{f_k(a) \mid k \in \Nat\}.\]
    Изображението $h$ е {\em непрекъснато} и е {\em точна горна граница} на веригата $\chain{f}{k}$,
    т.е. $h = \bigsqcup_k f_k$.
  \end{lemma}
\end{framed}
\marginpar{Ако $b_k = f_k(a)$, то $h(a)$ е точната горна граница на веригата $\chain{b}{k}$ в $\B$}
\begin{proof}
  \ifhints
  Доказателството, че $h$ е точна горна граница на веригата $\chain{f}{k}$ е лесно.
  \begin{itemize}
  \item 
    Да разгледаме произволен елемент $a \in A$.
    Лесно се вижда, че понеже $\chain{f}{k}$ е верига, то $(f_k(a))^\infty_{k=0}$ също е верига.
    % Това е така, защото всяко непрекъснато изображение е също така и монотонно.

    \marginpar{$\bigsqcup_n f_n(a)$ е съкратен запис за $\bigsqcup\{f_n(a) \mid n \in \Nat\}$.}
    Получаваме, че за всяко $k$, $f_k(a) \sqsubseteq^\B \bigsqcup_n f_n(a) \df h(a)$.
    Понеже това е вярно за произволно $a \in A$, $(\forall k)[f_k \sqsubseteq h]$,
    което означава, че $h$ е горна граница на веригата.
  \item
    Да разгледаме произволно изображение $g$, което е горна граница на веригата $\chain{f}{k}$.
    За произволен елемент $a \in A$, 
    \[(\forall k)[f_k(a) \sqsubseteq^\B g(a)].\]
    Това означава, че $g(a)$ е горна граница на веригата ${(f_k(a))}^\infty_{k=0}$.
    Понеже $h(a) = \bigsqcup_k \{f_k(a)\}$ е точната горна граница на веригата ${(f_k(a))}^\infty_{k=0}$,
    то $h(a) \sqsubseteq^\B g(a)$.
    Оттук следва, че $h \sqsubseteq g$.
  \end{itemize}
  \fi
  По-сложната част на доказателството е проверката, че $h$ е непрекъснато изображение.
  Да вземем една монотонно растяща редица $\chain{a}{k}$ от елементи на $A$.
  \marginpar{За момента дори не е ясно дали $\{h(a_k) \mid k \in \Nat\}$ е верига в $\B$}
  Ще докажем, че \[h(\bigsqcup_k a_k) = \bigsqcup_k \{h(a_k)\}.\]
  Нека $e_{n,m} \df f_n(a_m)$.
  Понеже всяко $f_n$ е непрекъснато и следователно монотонно изображение, то имаме
  \[n \leq n^\prime\ \&\ m \leq m^\prime\ \Rightarrow\ e_{n,m} \sqsubseteq^{\B} e_{n^\prime,m^\prime}.\]
  Следователно,
  \begin{align*}
    h(\bigsqcup_m a_m) & = \bigsqcup_n(f_n(\bigsqcup_m a_m)) & \comment{\text{от деф. на }h}\\
                       & = \bigsqcup_n(\bigsqcup_m f_n(a_m)) & \comment{\text{ защото } f_n \text{ е непр.}}\\
                       & = \bigsqcup_n(\bigsqcup_m e_{n,m}) = \bigsqcup_m(\bigsqcup_n e_{n,m}) & \comment{\text{от \Th{double-chain}}}\\
                       & = \bigsqcup_m(\bigsqcup_n f_n(a_m)) & \comment{\text{от деф. на }e_{n.m}}\\
                       & = \bigsqcup_m \{h(a_m)\}. & \comment{\text{от деф. на }h}
  \end{align*}
\end{proof}

Да напомним, че релацията $\sqsubseteq$ между две изображения е дефинирана като
\[f \sqsubseteq g \dfff (\forall a\in A)[f(a) \sqsubseteq^\B g(a)].\]
\begin{framed}
  \begin{theorem}
    \label{th:continuous-domain}
    Ако $\A$ и $\B$ са области на Скот, то $\Cont{\A}{\B}$ е област на Скот.
  \end{theorem}
\end{framed}

Нека $\A_1,\dots,\A_n$ и $\A$ са области на Скот и да разгледаме $f: \A_1\times \dots \times \A_n \to \A$.
Казваме, че $f$ е {\bf непрекъснато изображение по $i$-тия аргумент}, ако 
за всяка верига $\chain{a}{k}$ в $\A_i$, то
\[f(b_1,\dots, b_{i-1}, \bigsqcup_k a_k, b_{i+1},\dots,b_n) = \bigsqcup_k f(b_1,\dots, b_{i-1}, a_k, b_{i+1},\dots,b_n).\]

\begin{proposition}
  \label{pr:continuous-arguments}
  \marginpar{\cite[стр. 184]{models-of-computation}}
  Нека $\A_1,\dots,\A_n$ и $\A$ са области на Скот. Едно изображение
  \[f: \A_1\times \dots \times \A_n \to \A\]
  е непрекъснато точно тогава, когато $f$ е непрекъснато по всеки от аргументите си.
\end{proposition}
\begin{proof}
  \marginpar{\writedown Обобщете това твърдение за $n > 2$.}
  За по-просто изложение, да разгледаме случая $n = 2$.

  $(\Rightarrow)$ Лесно се съобразява, че ако $f$ е непрекъснато изображение, то $f$ е непрекъснато по всеки от аргументите си.
  Да видим например защо $f$ е непрекъснато по първия аргумент.
  Да разгледаме веригата ${(\pair{a_i,b})}^\infty_{i=0}$ от елементи на $\A\times\B$, за някое фиксирано $b$.
  Знаем, че $\bigsqcup_i \pair{a_i,b} = \pair{\bigsqcup_ia_i,b}$.
  Тогава
  \marginpar{Формално погледнато, правилно е да пишем $f(\pair{a,b})$ вместо $f(a,b)$.}
  \begin{align*}
    f(\bigsqcup_i a_i,b) & = f(\bigsqcup_i\pair{a_i,b}) \\
                        & = \bigsqcup_i f(a_i,b) & \comment{f\text{ е непрекъснато}}.
  \end{align*}
    
  $(\Leftarrow)$ Нека сега $f$ е непрекъснато по всеки от аргументите си. Ще докажем, че $f$ е непрекъснато.
  Нека ${\{\pair{a_n,b_n}\}}^\infty_{n=0}$ е верига в $\A_1\times \A_2$.
  Понеже от \Prop{cartesian} знаем, че
  \[\bigsqcup_n\pair{a_n,b_n} = \pair{\bigsqcup_n a_n,\bigsqcup_n b_n},\]
  ще докажем, че 
  \[\bigsqcup_n f(a_n,b_n) = f(\bigsqcup_n a_n,\bigsqcup_n b_n).\]
  Да положим $e_{n,m} = f(a_n,b_m)$.
  Понеже $f$ е непрекъснато по всеки от аргументите си, лесно се вижда, че $f$
  е монотонно изображение по всеки от аргументите си. Следователно, 
  \[n \leq n^\prime\ \&\ m \leq m^\prime\ \Rightarrow\ e_{n,m} \sqsubseteq e_{n^\prime,m^\prime}.\]  
  Получаваме, че
  \begin{align*}
    \bigsqcup_n f(a_n,b_n) & = \bigsqcup_n e_{n,n} & \comment{\text{от опр. на }e_{n,m}}\\
                           & = \bigsqcup_n (\bigsqcup_m e_{n,m}) & \comment{\text{от \Th{double-chain}}}\\
                           & = \bigsqcup_n (\bigsqcup_m f(a_n,b_m)) & \comment{\text{от опр. на }e_{n,m}}\\
                           & = \bigsqcup_n f(a_n,\bigsqcup_m b_m) & \comment{f \text{ е непр. по втория си аргумент}}\\
                           & = f(\bigsqcup_n a_n,\bigsqcup_m b_m) & \comment{f \text{ е непр. по първия си аргумент}}.
  \end{align*}
\end{proof}


%%% Local Variables:
%%% mode: latex
%%% TeX-master: "../sep"
%%% End:


\section{Основни непрекъснати изображения}

Да започнем като първо да дефинираме следните {\em основни изображения}
\begin{align*}
  & \plus : \Nat^2_\bot \to \Nat_\bot\text{, където}\\
  & \plus(a,b) \df
    \begin{cases}
      a+b, & \text{ако }a,b \in \Nat\\
      \bot, & \text{ако }\bot \in \{a,b\}
    \end{cases}\\
  & \minus : \Nat^2_\bot \to \Nat_\bot\text{, където}\\
  & \minus(a,b) \df
    \begin{cases}
      0, & \text{ако }a,b \in \Nat\ \&\ a < b\\
      a-b, & \text{ако }a,b \in \Nat\ \&\ a \geq b\\
      \bot, & \text{ако }\bot \in \{a,b\}
    \end{cases}\\
  & \eq : \Nat^2_\bot \to \Nat_\bot\text{, където}\\
  & \eq(a,b) \df
    \begin{cases}
      1, & \text{ако }a = b\ \&\ a,b \in \Nat\\
      0, & \text{ако }a \neq b\ \&\ a,b \in \Nat\\
      \bot, & \text{ако }\bot \in \{a,b\}
    \end{cases}
\end{align*}

% \marginpar{Озн. $\Nat^+ \df \Nat \setminus \{0\}$}

\begin{problem}\label{prob:basic-operations:continuous}
  Докажете, че изображенията $\texttt{(+)}$, $\texttt{(-)}$ и $\texttt{(==)}$ са непрекъснати.
\end{problem}
\begin{hint}
  Понеже $\Mon{\Nat^n_\bot}{\Nat_\bot} = \Cont{\Nat^n_\bot}{\Nat_\bot}$,
  достатъчно е да докажете, че изображенията са монотонни.
\end{hint}


\begin{problem}
  Докажете, че за всяко $b \in \B$, изображението
  $\texttt{const}_b \in \Cont{\prod^n_{i=1}\A_i}{\B}$ е непрекъснато, където
  \[\texttt{const}_b(a_1,\dots,a_n) = b.\]
\end{problem}


\begin{problem}
  \label{prob:projection}
  Докажете, че за всяко $n$ и $i < n$, изображението
  $\texttt{proj}^n_i \in \Cont{\A^n}{\A}$, където
  \[\texttt{proj}^n_i(a_0,\dots,a_{n-1}) = a_i.\]
\end{problem}

% \begin{definition}
%   \label{def:if}
%   \index{if}
%   Нека $\A$ е област на Скот. Дефинираме следното изображение
%   \begin{align*}
%     & \texttt{if}:\Nat_\bot \times \A \times \A \to \A \text{, където}\\
%     & \texttt{if}(b, a_1,a_2) =
%       \begin{cases}
%         a_1, & \text{ако } b \in \Nat^+\\
%         a_2, & \text{ако } b = 0\\
%         \bot, & \text{ако } b = \bot.
%       \end{cases}
%   \end{align*}
% \end{definition}

\begin{problem}\label{prob:basic-operations:if}
  Докажете, че $\texttt{if} \in \Cont{\Nat_\bot \times \A \times \A}{\A}$, където
  \begin{align*}
    & \texttt{if}(b, a_1,a_2) =
      \begin{cases}
        a_1, & \text{ако } b \in \Nat^+\\
        a_2, & \text{ако } b = 0\\
        \bot, & \text{ако } b = \bot.
      \end{cases}
  \end{align*}
\end{problem}
\begin{hint}
  Докажете, че $\texttt{if}$ е непрекъснато изображение по всеки от аргументите си поотделно.
\end{hint}


\begin{proposition}\label{pr:composition}\index{изображения!композиция}
  Ако $f \in \Cont{\A}{B}$ и $g \in \Cont{\B}{\C}$, то $g \circ f \in \Cont{\A}{\C}$,
  където \[(g\circ f)(a) \df g(f(a)).\]
\end{proposition}
\begin{hint}
  Нека $\chain{a}{i}$ е верига в $\A$.
  Да обърнем внимание, че понеже $f \in \Cont{\A}{\B}$,
  то $f$ е монотонно изображение и тогава ${(f(a_i))}^\infty_{i=0}$ е верига в $\B$.
  Тогава:
  \begin{align*}
    (g \circ f)(\bigsqcup_i a_i) & = g(f(\bigsqcup_i a_i)) & \comment{\text{от деф.}}\\
    & = g(\bigsqcup_i f(a_i)) & \comment{f \text{ е непр.}}\\
    & = \bigsqcup_i g(f(a_i)) & \comment{g \text{ е непр.}}
  \end{align*}
\end{hint}

\begin{proposition}
  Докажете, че изображението
  \[\texttt{comp}:\Cont{\B}{\C}\times\Cont{\A}{\B} \to \Cont{\A}{\C}\]
  е непрекъснато, където $\texttt{comp}(f,g) = f\circ g$.
  С други думи, $\texttt{comp}$ е елемент на областта на Скот
  \[\Cont{\Cont{\B}{\C}\times\Cont{\A}{\B}}{\Cont{\A}{\C}}.\]
\end{proposition}
\begin{hint}
  Използвайте \Prop{continuous-arguments}.
\end{hint}

Композицията на две функции е стандартна операция в \texttt{хаскел}.
\begin{haskellcode}
ghci> :t (.)
(.) :: (b -> c) -> (a -> b) -> a -> c  
\end{haskellcode}

%%% Local Variables:
%%% mode: latex
%%% TeX-master: "../sep"
%%% End:



\begin{proposition}\label{pr:cartesian-product:pair-continuous}
  Нека $f \in \Cont{\A}{\B}$ и $g \in \Cont{\A}{\C}$.
  Тогава $h \in \Cont{\A}{\B\times\C}$, където
  \[h(a) \df \pair{f(a),g(a)}.\]
  В такъв случай ще означаваме $h = f \times g$.
\end{proposition}
\begin{proof}
  Нека ${(a_i)}^{\infty}_{i=0}$ е верига в $\A$. Тогава:
  \begin{align*}
    h(\bigsqcup_i a_i) & = \pair{f(\bigsqcup_i a_i), g(\bigsqcup_i a_i)} & \comment{\text{от деф.}}\\
    & = \pair{\bigsqcup_i f(a_i), \bigsqcup_i g(a_i)} & \comment{\text{$f$ и $g$ са непр.}}\\
    & = \bigsqcup_i \pair{f(a_i),g(a_i)} & \comment{\text{от \Prop{cartesian}}}\\
    & = \bigsqcup_i h(a_i) & \comment{\text{от деф.}}
  \end{align*}
\end{proof}

% Обърнете внимание на следващото твърдение, защото ще го използваме често по-късно.
% То представлява обобщение на предишната задача и има сходно доказателство.
% Първо да въведем следното означение за произволно области на Скот $\B_1,\dots,\B_n$,
% \[\prod^n_{i=1}\B_i \df \B_1\times \B_2 \times \cdots \B_n,\]
% което също е област на Скот, дефинирана в {\em Раздел \ref{subsect:domains:product}}.

\begin{proposition}\label{pr:cartesian-product:general-continuous}
  Нека $f_i \in \Cont{\A}{\B_i}$, за $i = 1,\dots,n$.
  \marginpar{\writedown Докажете сами!}
  Тогава
  \[g \in \Cont{\A}{\prod^n_{i=1}\B_i},\]
  където
  \[g(a) \df \pair{f_1(a),f_2(a),\dots,f_n(a)}.\]
  В такъв случай ще означаваме $g = f_1\times f_2 \cdots \times f_n$.
\end{proposition}

\begin{proposition}\label{pr:cartesian-product:operator-continuous}
  Докажете, че изображението
  \[\texttt{cross}: \Cont{\A}{\B}\times\Cont{\A}{\C} \to \Cont{\A}{\B\times\C}\]
  е непрекъснато, където
  \[\texttt{cross}(f,g) = f \times g.\]
\end{proposition}


\begin{definition}\label{def:eval}\index{eval}
  Нека $\D$ и $\E$ са области на Скот. Дефинираме изображението 
  \[\texttt{eval}: \Cont{\D}{\E} \times \D \to \E,\]
  по следния начин:
  \[\texttt{eval}(f,d) \df f(d).\]  
\end{definition}

\begin{problem}\label{prob:eval}
  \marginpar{\cite[стр. 186]{models-of-computation}}
  Докажете, че $\texttt{eval}$ е непрекъснато изображение, т.е.
  \[\texttt{eval} \in \Cont{\Cont{\D}{\E} \times \D}{\E}.\]
\end{problem}
\begin{proof}
  Според \Prop{continuous-arguments}, достатъчно е да докажем, че $\texttt{eval}$ е непрекъснато
  изображение по всеки от двата си аргумента поотделно.
  
  Първо, нека $\chain{f}{n}$ е верига от елементи на $\Cont{\D}{\E}$ и $d$ е произволен елемент на $\D$.
  Тогава
  \[\texttt{eval}(\bigsqcup_n f_n,d) = (\bigsqcup_n f_n)(d) = \bigsqcup_n \{f_n(d)\} = \bigsqcup_n \texttt{eval}(f_n,d),\]
  т.е. изображението $\texttt{eval}$ е непрекъснато по първия си аргумент.
  
  Нека сега $\chain{d}{n}$ е верига от елементи на $\D$.
  Тогава за произволен елемент $f$ на $\Cont{\D}{\E}$ получаваме, че
  \[\texttt{eval}(f,\bigsqcup_n d_n) = f(\bigsqcup_n d_n) = \bigsqcup_n \{f(d_n)\} = \bigsqcup_n \texttt{eval}(f,d_n).\]
\end{proof}

%%% Local Variables:
%%% mode: latex
%%% TeX-master: "../sep"
%%% End:


\begin{definition}
  \label{def:curry}
  \index{curry}
  Нека $\A$, $\B$ и $\C$ са области на Скот.
  Изображението 
  \[\curry:\Mapping{\A\times \B}{\C} \to \Mapping{\A}{\Mapping{\B}{\C}},\]
  е дефинирано като
  \[\curry(f)(a)(b) \df f(a,b).\]  
\end{definition}

\begin{proposition}
  \label{pr:curry}
  \marginpar{\cite[стр. 187]{models-of-computation}}
  Ако $f$ е непрекъснато изображение, то
  $\curry(f)$ е непрекъснато изображение,
  т.е. $\curry(f) \in \Cont{\A}{\Cont{\B}{\C}}$.
  Освен това,
  \[\curry \in \Cont{\Cont{\A\times \B}{\C}}{\Cont{\A}{\Cont{\B}{\C}}}.\]
\end{proposition}
\begin{proof}
  Първо да фиксираме $f \in \Cont{\A\times\B}{\C}$ и $a \in \A$.
  Ще докажем, че $\curry(f)(a) \in \Cont{\B}{\C}$.
  \marginpar{Ако означим $h \df \curry(f)(a)$, то трябва да докажем, че за произволна верига $\chain{b}{i}$ в $\B$,
  $h(\bigsqcup_i b_i) = \bigsqcup_i\{h(b_i)\}$.}
  Нека фиксираме верига $\chain{b}{i}$ от елементи на $\B$.
  Тогава
  \begin{align*}
    \curry(f)(a)(\bigsqcup_i b_i) & \df f(a, \bigsqcup_i b_i)\\
                                  & = \bigsqcup_i\{f(a,b_i)\} & \comment\text{ $f$ е непр. по втория си аргумент}\\
                                  & = \bigsqcup_i \{\curry(f)(a)(b_i)\}. & \comment\text{опр. на }\curry
  \end{align*}

  Второ, сега пък за фиксирано $f \in \Cont{\A\times\B}{\C}$ трябва да докажем, че
  $\curry(f) \in \Cont{\A}{\Cont{\B}{\C}}$.
  За целта да разгледаме произволна верига $\chain{a}{i}$ от елементи на $\A$.
  Трябва да докажем, че
  \[\curry(f)(\bigsqcup_i a_i) = (\bigsqcup_i \curry(f))(a_i).\]
  \marginpar{Обърнете внимание, че $\curry(f)(a_i)$ образуват верига в $\Cont{\B}{\C}$,
  откъдето следва, че $\bigsqcup_i \{\curry(f)(a_i)\}$ е добре дефиниран елемент.}
  Също така, да напомним, че за произволен елемент $b \in \B$,
  \begin{equation}
    \label{eq:7}
    (\bigsqcup_i \{\curry(f)(a_i)\})(b) \df \bigsqcup_i \{\curry(f)(a_i)(b)\}.
  \end{equation}
  Тогава
  \begin{align*}
    \curry(f)(\bigsqcup_i a_i)(b) & = f(\bigsqcup_i a_i,b) & \comment\text{опр. на }\curry\\
                                  & = \bigsqcup_i f(a_i,b) & \comment\text{ $f$ е непр. по първия аргумент}\\
                                  & = \bigsqcup_i \{\curry(f)(a_i)(b)\} & \comment\text{опр. на }\curry\\
                                  & = (\bigsqcup_i \{\curry(f)(a_i)\})(b). & \comment\text{от (\ref{eq:7})}
  \end{align*}
  Заключаваме, че $\curry(f)(\bigsqcup_i a_i) = (\bigsqcup_i \curry(f))(a_i)$.

  Трето, остава да видим защо за произволна верига $\chain{f}{i}$ от елементи на $\Cont{\A\times\B}{\C}$ е изпълнено, че
  \[\curry(\bigsqcup_i f_i) = \bigsqcup_i \{\curry(f_i)\}.\]
  За произволен елемент $a \in \A$ имаме, че
  \[(\bigsqcup_i \curry(f_i))(a) = \bigsqcup_i \{\curry(f_i)(a)\}.\]
  \marginpar{Ако $h_i \df \curry(f_i)(a)$, то е ясно, че $(\bigsqcup_i h_i)(b) = \bigsqcup_i\{h_i(b)\}$.}
  За произволен елемент $b \in \B$ имаме, че
  \[(\bigsqcup_i \curry(f_i)(a))(b) = \bigsqcup_i \{\curry(f_i)(a)(b)\}.\]
  Комбинирайки предишните две равенства, получаваме, че за произволни $a \in \A$ и $b \in \B$,
  \begin{align*}
    (\bigsqcup_i \curry(f_i))(a)(b) & = \bigsqcup_i\{\curry(f_i)(a)(b)\}\\
                                    & = \bigsqcup_i\{f_i(a,b)\} & \comment\text{от опр. на }\curry\\
                                    & = (\bigsqcup_i f_i)(a,b)\\
                                    & = \curry(\bigsqcup_i f_i)(a)(b). & \comment\text{от опр. на }\curry
  \end{align*}
\end{proof}

\begin{definition}\label{def:uncurry}\index{uncurry}
  Нека $\A$, $\B$ и $\C$ са области на Скот.
  Изображението 
  \[\texttt{uncurry}:\Mapping{\Mapping{\A}{\Mapping{\B}{\C}}}{\Mapping{\A\times \B}{\C}},\]
  е дефинирано като
  \[\texttt{uncurry}(f)(a,b) \df f(a)(b).\]  
\end{definition}

\begin{problem}\label{prob:uncurry}
  % \marginpar{\cite[стр. 187]{models-of-computation}}
  Докажете, че ако $f$ е непрекъснато изображение, то
  $\texttt{uncurry}(f)$ е непрекъснато изображение,
  т.е. $\texttt{uncurry}(f) \in \Cont{\A \times \B}{\C}$.
  Освен това, докажете, че
  \[\texttt{uncurry} \in \Cont{\Cont{\A}{\Cont{\B}{\C}}}{\Cont{\A\times \B}{\C}}.\]
\end{problem}

  Всъщност, хаскел има функциите \texttt{curry} и \texttt{uncurry} вградени в стандартната библиотека:
  \begin{haskellcode}
ghci> :t curry
curry :: ((a, b) -> c) -> a -> b -> c
ghci> :t uncurry
uncurry :: (a -> b -> c) -> (a, b) -> c
  \end{haskellcode}


  \begin{problem}
    \marginpar{\cite[стр. 177]{models-of-computation}.}
    Докажете, че изображенията $\texttt{fst}:\A\times\B \to \A$ и $\texttt{snd}:\A\times\B \to \B$ са непрекъснати, където $\texttt{fst}(a,b) \df a$ и $\texttt{snd}(a,b) \df b$.
  \end{problem}
  
  \begin{haskellcode}
ghci> :t fst
fst :: (a, b) -> a
ghci> :t snd
snd :: (a, b) -> b
  \end{haskellcode}


  
%%% Local Variables:
%%% mode: latex
%%% TeX-master: "../sep"
%%% End:


% 
%%% Local Variables:
%%% mode: latex
%%% TeX-master: "../sep"
%%% End:


\section{Най-малки неподвижни точки}\label{sect:lfp}\index{най-малка неподвижна точка}

\begin{itemize}
\item\index{неподвижна точка}
  Да фиксираме произволна област на Скот $\A = (A, \sqsubseteq, \bot)$ и да разгледаме едно изображение $f:\A\to\A$.
  Казваме, че $a \in \A$ е {\bf неподвижна точка} на $f$, ако $f(a) = a$.
\item\index{най-малка неподвижна точка}
  Казваме, че $a \in \A$ е {\bf най-малката неподвижна точка} на $f$, ако:
  \begin{itemize}
  \item 
    $a$ е неподвижна точка, т.е. $f(a) = a$;
  \item
    за всяко $b \in \A$ със свойството, че $f(b) = b$ имаме $a \sqsubseteq b$.
  \item
    \marginpar{least fixed point}
    Ще означаваме най-малката неподвижна точка на $f$ като $\lfp(f)$.
  \end{itemize}
\end{itemize}

\begin{framed}
\begin{theorem}[Клини]
  \label{th:knaster-tarski}
  \index{Клини}
  Нека $\A$ е област на Скот.
  Всяко $f \in \Cont{\A}{\A}$ притежава най-малка неподвижна точка.
\end{theorem}
\end{framed}
\begin{proof}
  \marginpar{В \cite{ditchev-soskov} се нарича теорема на Кнастер-Тарски. Според \href{https://en.wikipedia.org/wiki/Kleene_fixed-point_theorem}{уикипедия} е теорема на Клини. Теоремата на Кнастер-Тарски е по-силна и говори за монотонни изображения в решетки.}
  Определяме монотонно растяща редица от елементи на $\A$ по следния начин:
  \begin{align*}
    & a_0 \df \bot & \comment = f^0(\bot)\\
    & a_{n+1} \df f(a_n) & \comment = f^{n+1}(\bot).
  \end{align*}

  Първо ще докажем с индукция по $n$, че $\chain{a}{n}$ е верига.
  Ясно е, че $a_0 \sqsubseteq a_1$.
  Да приемем, че $a_n \sqsubseteq a_{n+1}$. Тогава, понеже всяко непрекъснато
  изображение е монотонно, то имаме, че
  \[\underbrace{f(a_n)}_{a_{n+1}} \sqsubseteq \underbrace{f(a_{n+1})}_{a_{n+2}}.\]

  Нека $a \df \bigsqcup_i a_i$. Тогава 
  \begin{align*}
    f(a) & = f(\bigsqcup_i a_i) & \comment a \df \bigsqcup_i a_i\\
         & = \bigsqcup_i f(a_i) & \comment f \text{ е непрекъсната}\\
         & = \bigsqcup_i a_{i+1} & \comment a_{i+1} = f(a_i)\\
         & = \bigsqcup_i a_i & \comment \text{защото }\chain{a}{i}\text{ е верига}\\
         & = a.
  \end{align*}
  Така доказахме, че $a$ е \emph{ неподвижна точка} на $f$.
  Остана да видим, че е най-малката неподвижна точка на $f$.

  Нека $b = f(b)$. С индукция по $n$ ще докажем, свойството $(\forall n)[a_n \sqsubseteq b]$.
  \begin{itemize}
  \item 
    За $n = 0$ е очевидно.
  \item
    Да приемем, че $a_n \sqsubseteq b$.
    Тогава $a_{n+1} \df f(a_n) \sqsubseteq f(b) = b$, защото $f$ е монотонно изображение.    
  \end{itemize}
  Така доказахме, че $b$ е горна граница на веригата $\chain{a}{n}$.
  Заключаваме, че $a \df \bigsqcup_n a_n \sqsubseteq b$.
  Следователно, $a$ е \emph{ най-малката неподвижна точка} на $f$,
  т.е. $a = \lfp(f)$.
\end{proof}

\begin{problem}
  Покажете, че съществува област на Скот $\A$ и изображение $f \in \Mon{\A}{\A}$, което притежава най-малка неподвижна точка, но тя не е $\bigsqcup_n f^n(\bot^\A)$.
\end{problem}
\ifhints\begin{hint}
  Вземете областта на Скот $\A$ и монотонното изображение $f$ както в \Prop{continuous-mappings-not-monotone}.
  % Да разгледаме $\A = (A,\ \sqsubseteq,\ a_0)$, където елементите на $A$ са подредени по следния начин:
  % \[A = \{ a_0 \sqsubset a_1 \sqsubset \cdots \sqsubset a_n \sqsubset \cdots \sqsubset a_\omega \sqsubset b \}.\]
  % т.е. $A$ е съставена от веригата ${(a_n)}^\infty_{n=0}$ веднага следвана от елементите $a_\omega$ и $b$.
  % Да обърнем внимание, че $\bigsqcup_n a_n = a_\omega$.
  % Сега да разгледаме изображението $f:\A\to\A$, където за всяко $n$,
  % \begin{align*}
  %   & f(a_n) = a_{n+1}\\
  %   & f(a_\omega) = b\\
  %   & f(b) = b.
  % \end{align*}
  % Лесно се вижда, че това изображение е монотонно.
  Лесно се вижда, че $f$ не е непрекъснато изображение, защото $\bigsqcup_n a_n = a_\omega$, и тогава:
  \[f(\bigsqcup_n a_n) = f(a_\omega) = b \neq a_\omega = \bigsqcup_n a_{n+1} = \bigsqcup_n \{f(a_n)\}.\]
  Според дефиницията на изображението $f$, единствената неподвижна точка на $f$ е елементът $b$.
  Това означава, че $b$ е също и най-малката неподвижна точка.
  \marginpar{Имаме, че $f^n(a_0) = a_n$.}
  Това е пример за монотонно изображение, което не е непрекъснато, но притежава най-малка неподвижна точка $b$,
  макар и тя да не е $a_\omega = \bigsqcup_n f^n(a_0)$.
\end{hint}
\fi

% \begin{problem}
%   Да разгледаме $\Gamma:\Partial{\Nat}{\Nat} \to \Partial{\Nat}{\Nat}$, където
%   \begin{enumerate}[a)]
%   \item
%     $\Gamma(f)(x) = 5$, за всяко $x \in \Nat$;
%   \item
%     $\Gamma(f)(x)$ не е деф. за всяко $x \in \Nat$;
%   \item
%     $\Gamma(f) = f$;
%   \item
%     $\Gamma(f) = f\circ f$;
%   \item
%     $\Gamma(f)(x) = x * f(x+1)$;
%     \item
%     $\Gamma(f)(x) =
%     \begin{cases}
%       \text{не е деф.}, & \text{ ако }x = 0\\
%       x * f(x-1), & \text{ ако }x > 0.
%     \end{cases}$    
%   \item
%     $\Gamma(f)(x) =
%     \begin{cases}
%       0, & \text{ ако }x = 0\\
%       x * f(x-1), & \text{ ако }x > 0.
%     \end{cases}$
%     \item
%     $\Gamma(f)(x) =
%     \begin{cases}
%       1, & \text{ ако }x = 0\\
%       x * f(x-1), & \text{ ако }x > 0.
%     \end{cases}$    
%   \end{enumerate}
% \end{problem}


% \newpage

% \begin{example}
%   Да разгледаме следното изображение $\Gamma:\Partial{\Nat}{\Nat} \to \Partial{\Nat}{\Nat}$, където
%   \[\Gamma(f)(x) \simeq
%     \begin{cases}
%       0, & \text{ ако }x = 0\\
%       x + f(x-1), & \text{ ако }x > 0.
%     \end{cases}\]
%   Първо да видим, че $\Gamma$ е монотонно изображение.
%   Нека $f \subseteq g$.
%   Трябва да докажем, че за всяко $x$, ако $\Gamma(f)(x) \simeq y$, то $\Gamma(g)(x) \simeq y$.
%   \begin{itemize}
%   \item
%     Нека $x = 0$.
%     Тогава $\Gamma(f)(0) \simeq 0 \simeq \Gamma(g)(0)$.
%   \item
%     Нека $x > 0$ и да приемем, че $\Gamma(f)(x) \simeq x + f(x-1) \simeq y$.
%     Това означава, че $f(x-1) \simeq z$, за някое $z$, и $x + z = y$.
%     От $f \subseteq g$ следва, че имаме също и $g(x-1) \simeq z$.
%     Тогава е ясно, че $\Gamma(g)(x) \simeq x + g(x-1) \simeq x+z = y$.
%   \end{itemize}
%   Разгледахме всички възмножни случаи за естественото число $x$ и
%   \marginpar{$\texttt{Graph}(\Gamma(f)) \subseteq \texttt{Graph}(\Gamma(g))$.}
%   видяхме, че за произволно $x$, ако $\Gamma(f)(x) \simeq y$, то $\Gamma(g)(x) \simeq y$.
%   Заключаваме, че $\Gamma(f) \subseteq \Gamma(g)$, т.е. $\Gamma$ е монотонно изображение.

%   Нека сега да видим, че $\Gamma$ е непрекъснато изображение.
%   Да разгледаме произволна верига $\chain{f}{i}$ от частични функции.
%   Трябва да докажем, че
%   \[\Gamma(\bigsqcup_i f_i) = \bigsqcup_i \Gamma(f_i).\]
%   Щом $\Gamma$ е монотонно, от \Prop{monotone-chain} вече знаем, че
%   \[\bigsqcup_i \Gamma(f_i) \subseteq \Gamma(\bigsqcup_i f_i).\]
%   Остава да докажем обратната посока. И така, нека първо да вземем $x = 0$.
%   Тогава $\Gamma(\bigsqcup_i f_i)(0) \simeq 0$. От дефиницията от $\Gamma$ знаем, че
%   за всяко $i$, $\Gamma(f_i)(0) \simeq 0$ и оттук $(\bigsqcup_i \Gamma(f_i))(0) \simeq 0$.
%   Следователно,
%   \[\Gamma(\bigsqcup_i f_i)(0) \simeq 0 \simeq (\bigsqcup_i \Gamma(f_i))(0).\]
%   Нека сега $x > 0$. Тогава
%   \marginpar{Да напомним, че $(\bigsqcup_i f_i)(u) \simeq v$ точно тогава, когато съществува индекс $i$, за който $f_i(u) = v$.}
%   \[\Gamma(\bigsqcup_i f_i)(x) \simeq x + (\bigsqcup_i f_i)(x-1) \simeq y.\]
%   Ясно е, че съществува $z$, за което $(\bigsqcup_i f_i)(x-1) \simeq z$ и $x + z = y$.
%   Знаем, че съществува индекс $i$, за който $f_i(x-1) \simeq z$.
%   Тогава, понеже $\Gamma(f_i)(x) \simeq x + f_i(x-1) \simeq x+z = y$, то следва, че $(\bigsqcup_i \Gamma(f_i))(x) \simeq y$.

%   Сега вече можем да намерим $\lfp(\Gamma) = \bigsqcup_n \Gamma^n(\bm{\bot})$.
%   Ще докажем, че
%   \[\lfp(\Gamma)(x) \simeq \frac{x(x+1)}{2}\] за всяко естествено число $x$.
%   Нека за улеснение да означим $g_n = \Gamma^n(\bm{\bot})$.
%   Ще докажем, че за всяко $n$,
%   \[g_n(x) \simeq \begin{cases}
%       \sum^x_{i=1}i, & \text{ ако } x < n\\
%       \text{не е деф.}, & \text{ ако } x \geq n.
%     \end{cases}\]
%   Ясно е, че $g_0 = \bm{\bot}$, което може да се запише и така:
%   \marginpar{Да напомним, че $\Gamma^0(g) = g$ и $\Gamma^{n+1}(g) = \Gamma(\Gamma^n(g))$.}
%   \[g_0(x) \simeq \begin{cases}
%       \sum^x_{i=1}i, & \text{ ако } x < 0\\
%       \text{не е деф.}, & \text{ ако } x \geq 0.
%     \end{cases}\]
%   Да приемем, че нашето твърдение е изпълнено за $g_n$.
%   Ще докажем, че то е изпълнено и за $g_{n+1}$. И така,
%   \begin{align*}
%     g_{n+1}(x) & \simeq= \Gamma(g_n)(x)\\
%                &  \simeq \begin{cases}
%                  0, & \text{ ако } x = 0\\
%                  x + g_n(x-1), & \text{ ако }x > 0\\
%                \end{cases}\\
%                & \stackrel{\text{И.П.}}{\simeq} \begin{cases}
%                  0, & \text{ ако } x = 0\\
%                  x + \sum^{x-1}_{i=1}i , & \text{ ако } 0 \leq x-1 < n\\
%                  \text{не е деф.}, & \text{ ако }x-1 \geq n\\
%                \end{cases}\\
%                & \simeq \begin{cases}
%                  0, & \text{ ако } x = 0\\
%                  \sum^{x}_{i=1}i , & \text{ ако } 1 \leq x < n+1\\
%                  \text{не е деф.}, & \text{ ако }x \geq n+1\\
%                \end{cases}\\
%                & \simeq \begin{cases}
%                  \sum^{x}_{i=1}i , & \text{ ако } x < n+1\\
%                  \text{не е деф.}, & \text{ ако }x \geq n+1.
%                \end{cases}
%   \end{align*}
%   Сега можем да заключим, че за всяко естествено число $x$,
%   \[g_{x+1}(x) \simeq \sum^x_{i=1}i = \frac{x(x+1)}{2}.\]
%   Тогава
%   \[\texttt{lfp}(\Gamma)(x) \simeq (\bigsqcup_i g_i)(x) \simeq \frac{x(x+1)}{2}.\]
% \end{example}

% Казваме, че елементът $a$ е {\bf преднеподвижна точка} на $f$,
% ако $f(a) \sqsubseteq a$.

\marginpar{Да се обясни къде се използва това твърдение.}
\begin{proposition}\label{pr:prefix-point}
  За всяко $f \in \Cont{\A}{\A}$ е изпълнено, че 
  \[(\forall a \in \Pref(f))[\lfp(f) \sqsubseteq a],\]
  където
  \index{преднеподвижна точка}
  \[\Pref(f) \df \{a \in \A \mid f(a) \sqsubseteq a\}\]
  е множеството от всички преднеподвижни точки на $f$.
  Това означава, че $\lfp(f)$ е най-малката преднеподвижна точка на $f$.
\end{proposition}
\begin{proof}
  Знаем от \hyperref[th:knaster-tarski]{Теоремата на Клини}, че $\lfp(f) = \bigsqcup_n f^n(\bot)$.
  Също така знаем, че $\chain{b}{n}$ е верига, където за улеснение сме положили $b_n \df f^n(\bot)$. 
  Ясно е също, че $\texttt{Pref}(f) \neq \emptyset$, защото $\lfp(f) \in \texttt{Pref}(f)$.
  Да фиксираме прозиволен елемент $a\in \texttt{Pref}(f)$.
  С индукция по $n$ ще докажем, че $b_n \sqsubseteq a$ за всяко $n$.
  \begin{itemize}
  \item 
    За $n = 0$ е очевидно, защото тогава $b_0 \df \bot \sqsubseteq a$.
  \item
    Да приемем, че $b_n \sqsubseteq a$.
    Ще докажем, че $b_{n+1} \sqsubseteq a$.
    Но това е лесно.
    \begin{align*}
      b_{n+1} & = f(b_n) & \comment \text{от деф. на }b_{n+1}\\
      & \sqsubseteq f(a) & \comment b_n \sqsubseteq a\ \&\ f\text{ е мон.}\\
      & \sqsubseteq a & \comment a \in \texttt{Pref}(f).
    \end{align*}
  \end{itemize}
  Така доказахме, че за всяко $n$, $b_n \sqsubseteq a$,
  откъдето следва, че $a$ е горна граница за веригата $\chain{b}{n}$, откъдето директно получаваме, че
  \[\lfp(f) = \bigsqcup_n b_n \sqsubseteq a.\]
\end{proof}

\begin{problem}
  Нека $\A$ е област на Скот и нека $f \in \Cont{\A}{\A}$.
  Да разгледаме множеството $B = \{a \in \A \mid f(a) = f\}$.
  Вярно ли е, че $\B = (B, \sqsubseteq, \lfp(f))$ е област на Скот.
  Обосновете се!
\end{problem}

\begin{problem}% Gunter textbook
  Нека $f \in \Cont{\A}{\A}$.
  Да разгледаме множеството 
  \[B = \{a \in \A \mid f(a) \sqsubseteq a\}.\]
  Вярно ли е, че 
  \[\B = (B, \sqsubseteq^\A, \lfp(f))\] е област на Скот?
  Обосновете се!
\end{problem}


\begin{problem} % Gunter textbook
  Да разгледаме множеството
  \[B = \{f \in \Mon{\A}{\A} \mid f\circ f = f\}.\]
  Вярно ли е, че 
  \[\B = (B,\ \sqsubseteq,\ \lambda x.\bot^\A)\] е област на Скот,
  където 
  \[f \sqsubseteq g \df (\forall a\in\A)[f(a) \sqsubseteq^\A g(a)] ?\]
  Обосновете се!
\end{problem}

\begin{problem}
  Нека $\A$ е област на Скот и $f \in \Cont{\A}{\A}$.
  Докажете, че $\lfp(f) = \lfp(f\circ f)$.
\end{problem}

\begin{problem}
  \label{prob:domains:lfp:compositon}
  \marginpar{\cite[стр. 131]{nikolova-soskova} и задача в Кеймбридж 2020 г. \cite{cambridge-website}}
  Нека $f \in \Cont{\A}{\B}$ и $g \in \Cont{\B}{\A}$.
  Докажете, че 
  \begin{itemize}
  \item 
    $\lfp(g \circ f) \sqsubseteq g(\lfp(f \circ g))$;
  \item
    $f(\lfp(g \circ f)) \sqsubseteq \lfp(f \circ g)$.
  \end{itemize}
  Оттук заключете, че 
  \[\lfp(g \circ f) = g(\lfp(f \circ g)) \text{ и }f(\lfp(g \circ f)) = \lfp(f \circ g).\]
\end{problem}

\begin{problem}
  \label{prob:domains:lfp:compositon-1}
  \marginpar{Изображения $h$, за които $h(\bot^\A) = \bot^\B$ се наричат точни. На англ. strict.}
  Нека $\A$ и $\B$ са области на Скот и нека $f \in \Cont{\A}{\A}$, $g \in \Cont{\B}{\B}$,
  и $h \in \Cont{\A}{\B}$, за които е изпълнено свойството, че $h \circ f = g \circ h \in \Cont{\A}{\B}$.
  Докажете, че ако $h$ е такава, че $h(\bot^{\A}) = \bot^\B$, то е изпълнено, че:
  \[\lfp(g) = h(\lfp(f)).\]
\end{problem}


%%% Local Variables:
%%% mode: latex
%%% TeX-master: "../sep"
%%% End:


% \subsection{Точни изображения}

\marginpar{На англ. се нарича {\em strict}. ,,Стандартната'' семантика на хаскел е {\em non-strict}}
\index{изображение!точно}
Едно изображение $f:\A^n_1 \to \A_2$ се нарича {\bf точно}, ако 
\[(\forall \bar{a}\in\A^n_1)[\bot_1\in\{a_1,\dots,a_n\} \implies f(a_1,\dots,a_n) = \bot_2].\]
% Тук ще разглеждаме съвкупността от точни изображения:
% \[S_n \dff \{f: \Nat^n_\bot \to \Nat_\bot \mid f\text{ е точна}\}.\]
За произволни области на Скот $\A_1$ и $\A_2$, ще означаваме съвкупността от
точните изображения като $\Strict{\A_1}{\A_2}$.

\begin{example}
  Да разгледаме свойствата на няколко прости изображения.
  \begin{itemize}
  \item 
    Изображението $f:\Nat_\bot \to \Nat_\bot$, дефинирано като 
    \[f(x) = 42,\]
    за всяко $x \in \Nat_\bot$,
    {\bf не е точно}, защото $f(\bot) = 42$. Лесно се вижда, че $f$ е монотонно и непрекъснато изображение.
  \item
    От друга страна, изображението $g:\Nat_\bot \to \Nat_\bot$, дефинирано като 
    \[g(x) = \bot,\]
    за всяко $x \in \Nat_\bot$, {\bf е точно}, защото $g(\bot) = \bot$.
    Освен това, $g$ е монотонно и непрекъснато изображение.
  \end{itemize}
\end{example}

Видяхме, че лесно се намират монотонни изображения, които не са точни.
Сега ще разгледаме обратната посока.

\begin{proposition}
  \label{pr:strict-is-monotone}
  Всяко точно изображение $f:\Nat^n_\bot \to \Nat_\bot$ е също така и монотонно.
  С други думи, 
  \[\Strict{\Nat^n_\bot}{\Nat_\bot} \subseteq \Mon{\Nat^n_\bot}{\Nat_\bot}.\]
\end{proposition}
\begin{proof}
  Нека $f$ е точно и $\bar{a} \sqsubseteq \bar{b}$.
  Ще проверим, че 
  \[f(\bar{a}) = c \sqsubseteq d = f(\bar{b}).\]
  \begin{itemize}
  \item 
    Ако $c = \bot$, то е очевидно, че $c \sqsubseteq d$.
  \item
    Интересният случай е когато $c \neq \bot$. Тогава трябва да видим, че $c = d$.
    Понеже $f$ е точна и $c \neq \bot$, това означава, че $\bot \not\in\{a_1,\dots,a_n\}$.
    Но понеже $\sqsubseteq$ е плоска наредба, и $\bar{a} \sqsubseteq \bar{b}$, това означава, че $\bar{a} = \bar{b}$.
    Следотвателно, наистина $c = d$.
  \end{itemize}
\end{proof}

\begin{framed}
  \begin{theorem}
    \label{th:strict-is-domain}
    $(\Strict{\Nat^n_\bot}{\Nat_\bot},\ \sqsubseteq,\ \bm{\bot}^{(n)})$ е област на Скот.
  \end{theorem}
\end{framed}
\begin{hint}
  Да разгледаме веригата $\chain{f}{i}$ от елементи на $\Strict{\Nat^n_\bot}{\Nat_\bot}$.
  Трябва да докажем, че тази верига притежава точна горна граница, която също е точно изображение.
  % Понеже от \Prop{strict-is-monotone} имаме, че всяка точна функция е монотонна, то наготово от 
  От доказателството на \Th{all-mappings-is-domain} знаем, че изображението $h$ дефинирано за всяко $\bar{a} \in \Nat^n_\bot$ като
  \[h(\bar{a}) \df \bigsqcup \{f_i(\bar{a}) \mid i \in \Nat\}\]
  е точна горна граница на веригата $\chain{f}{i}$.
  Остава да проверим, че $h$ е точно изображение.
  Нека $\bar{a} \in \Nat^n_\bot$ и да приемем, че $\bot \in \{a_1,\dots,a_n\}$.
  Тогава:
  \begin{align*}
    h(\bar{a}) & = \bigsqcup \{f_i(\bar{a}) \mid i \in \Nat\} & \comment{\text{от деф. на }h}\\
               & = \bigsqcup\{\bot,\bot,\dots,\bot,\dots\} & \comment{f_i\text{ са точни}}\\
               & = \bot.
  \end{align*}
\end{hint}

\begin{example}
  \label{ex:simple-non-continuous}
  Нека сега да разгледаме $h:\Nat_\bot \to \Nat_\bot$, където 
  \begin{align*}
    h(x) = &
    \begin{cases}
      0, & \text{ако }x = \bot,\\
      1, & \text{иначе}.
    \end{cases}
  \end{align*}
  Да видим колко ,,лошо'' изображение е $h$:
  \begin{itemize}
  \item 
    $h$ не е точно, защото $h(\bot) \neq \bot$;
  \item
    $h$ не е монотонно, защото $\bot \sqsubseteq 5$, но $h(\bot) = 0 \not\sqsubseteq 1 = h(5)$;
  \item
    $h$ не е непрекъснато, защото $\Cont{\Nat_\bot}{\Nat_\bot} = \Mon{\Nat_\bot}{\Nat_\bot}$.
  \end{itemize}
  Нека да видим, че хаскел не позволява такива ,,лоши'' функции:

  \begin{haskellcode}
ghci> let h(x) = if x == undefined then 0 else 1
ghci> h(5)
*** Exception: Prelude.undefined
ghci> h(undefined)
*** Exception: Prelude.undefined
  \end{haskellcode}
\end{example}

\begin{example}
  \label{ex:non-strict-monotone}
  Да разгледаме $f:\Nat^2_\bot \to \Nat_\bot$ дефинирано като \[f(x,y) = x.\]
  \begin{itemize}
  \item 
    $f$ не е точно, защото $f(0,\bot) = 0$.
  \item
    $f$ е монотонно, защото ако $\pair{x,y} \sqsubseteq \pair{x',y'}$, то $x \sqsubseteq x'$ и
    \[f(x,y) = x \sqsubseteq x' = f(x',y').\]
  \item
    $f$ е непрекъснато, защото от \Cor{monotone-is-continuous} знаем, че 
    $\Mon{\Nat^2_\bot}{\Nat_\bot} = \Cont{\Nat^2_\bot}{\Nat_\bot}$.
  \end{itemize}
\end{example}

Можем да обобщим всичко, което направихме дотук, със следната илюстрация на основните области на Скот, с които ще работим
в \Chapter{rec}.

\begin{framed}
    \begin{align*}
      \Strict{\Nat^n_\bot}{\Nat_\bot} & \subsetneqq \Mon{\Nat^n_\bot}{\Nat_\bot} & \comment{\text{от \Ex{non-strict-monotone} и \Prop{strict-is-monotone}}}\\
      & = \Cont{\Nat^n_\bot}{\Nat_\bot} & \comment{\text{от \Prop{continuous-is-monotone} и \Cor{monotone-is-continuous}}}\\
      & \subsetneqq \Mapping{\Nat^n_\bot}{\Nat_\bot} & \comment{\text{от \Ex{simple-non-continuous}}}.
    \end{align*}
\end{framed}

\begin{example}
  \label{ex:plus}
  Да разгледаме $f:\Nat^2_\bot \to \Nat_\bot$ дефинирана по следния начин:
  \[f(x,y) = 
  \begin{cases}
    \bot, & x = \bot\ \&\ y = \bot\\
    x, & x \neq \bot\ \&\ y = \bot\\
    y, & x = \bot\ \&\ y \neq \bot\\
    x+y, & x \neq \bot\ \&\ y \neq \bot.
  \end{cases}\]
  Изображението $f$ не е точно, защото например $f(\bot,0) \neq \bot$.
  $f$ не е монотонно изображение, защото $\pair{2,\bot} \sqsubseteq \pair{2,3}$, но $2 = f(2,\bot) \not\sqsubseteq f(2,3) = 5$.
  Също така, $f$ не е непрекъснато изображение, защото $\Mon{\Nat^2_\bot}{\Nat_\bot} = \Cont{\Nat^2_\bot}{\Nat_\bot}$.
\end{example}
% \begin{hint}
%   \begin{itemize}
%   \item 
%     $f$ не е точно изображение, защото например
%     $f(\bot,0) \neq \bot$.
%   \item
%     $f$ не е монотонно. Например,
%     \[\pair{\bot,0} \sqsubseteq \pair{1,0}\ \&\ f(\bot,0) = 0 \not\sqsubseteq 1 = f(1,0).\]    
%   \item
%     $f$ не е непрекъснато, защото $\Cont{\Nat^2_\bot}{\Nat_\bot} = \Mon{\Nat^2_\bot}{\Nat_\bot}$.
%   \end{itemize}    
% \end{hint}


% \subsection{Точни продължения}
% \index{изображение!точно продължение}

% Нека $A$ и $B$ са произволни множества, а $\A_\bot$ и $\B_\bot$ са плоските области на Скот получени съответно от $A$ и $B$.
% За еднa частична функция $f:A^n \to B$, определяме точното изображение $f^\star:\A^n_\bot \to \B_\bot$ по следния начин:
% \marginpar{$f^\star \in \Strict{\A^n_\bot}{\B_\bot}$}
% \begin{align*}
%   f^\star(\ov{a}) \dff
%   \begin{cases}
%     f(\ov{a}), & \text{ако }\bot\not\in\{a_1,\dots,a_n\}\ \&\ f(\ov{a})\text{ е деф.}\\
%     \bot, & \text{иначе}.
%   \end{cases}
% \end{align*}
% Изображението $f^\star$ се нарича {\bf точно продължение} на $f$.

% \begin{example}
%   Да разгледаме частичната функция $f:\Nat^2 \to \Nat$, дефинирана като
%   $f(x,y) = x+y$. Тогава точното продължение на $f$ е 
%   \[f^\star(x,y) = 
%   \begin{cases}
%     x+y, & x,y\in\Nat\\
%     \bot, & x = \bot \vee y = \bot.
%   \end{cases}\]
% \end{example}

% \Stefan{За какво ми е точното продължение да е непрекъснато?}
% \noindent Да дефинираме оператор $\Sigma^\star : \Partial{\Nat^n}{\Nat} \to \Strict{\Nat^n_\bot}{\Nat_\bot}$ като
% \[\Sigma^\star(f) = f^\star.\] 
% Ще наричаме $\Sigma^\star$ {\bf продължаващ} оператор, защото на всяка частична функция дава нейното точно продължение.

% \begin{framed}
%   \begin{prop}
%     $\Sigma^\star$ е непрекъснато изображение, т.е.
%     \[\Sigma^\star \in \Cont{\Partial{\Nat^n}{\Nat}}{\Strict{\Nat^n_\bot}{\Nat_\bot}}.\]
%   \end{prop}
% \end{framed}
% \begin{hint}
%   Трябва да докажем, че за произволна верига $(f_i)^{\infty}_{i=0}$ от частични функции, 
%   $\Sigma^\star(\bigcup_i f_i) = \bigsqcup_i \Sigma^\star(f_i)$.
%   \begin{itemize}
%   \item 
%     Лесно се съобразява, че ако $f \subseteq g$, то $\Sigma^\star(f) \sqsubseteq \Sigma^\star(g)$.
%     Оттук следва, че 
%     \[\bigsqcup_i \Sigma^\star(f_i) \sqsubseteq \Sigma^\star(\bigcup_i f_i).\]
%   \item
%     За другата посока,
%     Нека $\Sigma^\star(\bigcup_i f_i)(\ov{x}) = y \neq \bot$.
%     Това означава, че $\ov{x} \in \Nat^n$ и $(\bigcup_i f_i)(\ov{x}) \simeq y$.
%     Оттук следва, че съществува $i_0$, за което $f_{i_0}(\ov{x}) \simeq y$.
%     Тогава $\Sigma^\star(f_{i_0})(\ov{x}) = y$.
%     Понеже $y \neq \bot$, 
%     \[\bigsqcup_i (\Sigma^\star(f_i)(\ov{x})) = (\bigsqcup_i \Sigma^\star(f_i))(\ov{x}) = y.\]
%     Заключаваме, че $\Sigma^\star(\bigcup_i f_i)(\ov{x}) \sqsubseteq \bigsqcup_i (\Sigma^\star(f_i)(\ov{x}))$
%   \end{itemize}
% \end{hint}

% \Stefan{Това означава, че някъде трябва да се каже, че $\F_n$ е област на Скот. Някъде в тази глава трябва да е. Това може да бъде един от първите примери след дефиницията на О.С.}



%%% Local Variables:
%%% mode: latex
%%% TeX-master: "../sep"
%%% End:


% \input{domains-basic/strict-restrictions}


\section{Оператор за най-малка неподвижна точка}

\Stefan{Този раздел може да се сложи по-късно.}

\begin{problem}
  \marginpar{Да напомним, че $f^0(a) = a$ и $f^{n+1}(a) = f(f^n(a))$.}
  Нека $\chain{f}{i}$ е верига от елементи на $\Mon{\A}{\A}$.
  Докажете, че:
  \begin{enumerate}[a)]
  \item
    $\chain{f^n}{i}$ е верига за произволно $n$;
  \item
    ${(f^n_i)}^\infty_{n=0}$ е верига за произволно $i$;
  \item
    ${(\bigsqcup_n f^n_i)}^\infty_{i=0}$ е верига;
  \item
    ${(\bigsqcup_i f^n_i)}^\infty_{n=0}$ е верига.
  \end{enumerate}
\end{problem}


\begin{theorem}\label{th:Y}\index{Y}
  \marginpar{Доказателството в \cite[стр. 188]{models-of-computation} е малко по-различно.}
  % \index{$Y_\A$}
  Нека $\A$ е област на Скот и нека $f \in \Cont{\A}{\A}$.
  \marginpar{Знаем от \Th{knaster-tarski}, че най-малката неподвижна точка на $f$ е елемента $\bigsqcup_n f^n(\bot^\A)$,
    т.е. $\lfp(f) = \bigsqcup_n f^n(\bot^\A)$.}
  Тогава изображението $Y : \Cont{\A}{\A} \to \A$, определено като
  \[Y(f) = \lfp(f),\]
  е непрекъснато, т.е.
  $Y \in \Cont{\Cont{\A}{\A}}{\A}$.
\end{theorem}
\begin{proof}
  Нека да вземем една верига $\chain{f}{n}$ от непрекъснати изображения.
  Нашата цел е да докажем, че
  \[Y(\bigsqcup_n f_n) = \bigsqcup_n Y(f_n).\]
  \marginpar{Записът ще стане много тромав, ако вместо $h$ използваме означението $\bigsqcup_n f_n$.}
  Да означим с $h$ точната горна граница на $\chain{f}{n}$.
  Знаем, че $h(a) = \bigsqcup_n \{f_n(a)\}$, а от \Th{continuous-domain} знаем, че $h$ е непрекъснато изображение.
  \begin{proposition}
    За всяко $k \geq 0$ е изпълнено, че $(\bigsqcup f_i)^k = \bigsqcup_i f^k_i$.
  \end{proposition}
  \begin{proof}
    Ще докажем твърдението с индукция по $k$, като случая $k = 0$ е ясен, защото
    $(\bigsqcup_i f_i)^0 = id = \bigsqcup_i f_i^0 = \bigsqcup_i id$.
    Нека приемем, че твърдението е вярно за произволно $k$.
    Ще докажем, че твърдението е вярно за $k+1$.
    \begin{align*}
      (\bigsqcup_i f_i)^{k+1} & = \compose((\bigsqcup_if_i)^k, \bigsqcup_i f_i)\\
                              & = \compose(\bigsqcup_if^k_i, \bigsqcup_i f_i) & \comment\text{от \IndHyp}\\
                              & = \bigsqcup_i \compose(f^k_i, f_i) & \comment\compose\text{ е непр.}\\
                              & = \bigsqcup_i f^{k+1}_i
    \end{align*}
    
    % \begin{align*}
    %   h^{k+1}(a) & = h(h^k(a)) & \\
    %              & = h(\bigsqcup_n f^k_n(a))& \comment{\text{ от инд. предположение}}\\
    %              & = \bigsqcup_n h(f^k_n(a))& \comment{h \text{ е непрекъснато изображение}}\\
    %              & = \bigsqcup_n (\bigsqcup_m f_m(f^k_n(a))). & 
    % \end{align*}
    
    % Да положим $b_n = f^k_n(a)$, за всяко $n$.
    % Понеже $f_n \sqsubseteq f_{n^\prime}$, лесно се съобразява, че за $n \leq n^\prime$
    % имаме $b_n \sqsubseteq^\A b_{n^\prime}$.

    % Сега да положим $e_{m,n} = f_m(b_n)$.
    % Отново, понеже $\chain{b}{n}$ и $\chain{f}{m}$ са вериги, имаме 
    % \[m \leq m^\prime\ \&\ n\leq n^\prime\ \Rightarrow\ e_{m,n} \sqsubseteq^\A e_{m^\prime,n^\prime}.\]
    % Получаваме, че
    % \begin{align*}
    %   h^{k+1}(a) & = \bigsqcup_n (\bigsqcup_m f_m(f^k_n(a))) & \comment{\text{ от горното равенство за } h^{k+1}}\\
    %              & = \bigsqcup_n (\bigsqcup_m e_{m,n}) & \comment{\text{ от определението на }e_{m,n}}\\
    %              & = \bigsqcup_n e_{n,n} & \comment{\text{ от \Th{double-chain}}}\\
    %              & = \bigsqcup_n f_n(f^k_n(a))  = \bigsqcup_n f^{k+1}_n(a) & 
    % \end{align*}
    С това твърдението е доказано.
  \end{proof}
  Сега вече сме готови да докажем непрекъснатостта на $Y$.
  Имаме, че:
  \begin{align*}
    Y(\bigsqcup_i f_i) & = \bigsqcup_m (\bigsqcup_i f_i)^m(\bot^\A) & \comment{\text{ от опр. на }Y }\\
                       & = \bigsqcup_m (\bigsqcup_i f^m_i(\bot^\A)) & \comment{\text{ от горното твърдение}}
  \end{align*}

  Да положим $e_{m,n} = f^m_n(\bot^\A)$.
  Отново лесно се съобразява, че 
  \[m \leq m^\prime\ \&\ n\leq n^\prime\ \Rightarrow\ e_{m,n} \sqsubseteq^\A e_{m^\prime,n^\prime}.\]
  Получаваме, че
  \begin{align*}
    Y(\bigsqcup_n f_n) & = \bigsqcup_m (\bigsqcup_n f^m_n(\bot^\A)) & \comment{\text{ от горното равенство}}\\
                          & = \bigsqcup_m (\bigsqcup_n e_{m,n}) & \comment{\text{ от опр. на }e_{m,n}}\\
                          & = \bigsqcup_n(\bigsqcup_m e_{m,n}) & \comment{\text{ от \Th{double-chain}}}\\
                          & = \bigsqcup_n (\bigsqcup_m f^m_n(\bot^\A)) = \bigsqcup_n Y(f_n). & \comment{\text{ от опр. на }Y}.
  \end{align*}
\end{proof}

\marginpar{Добре е да погледнете \href{https://en.wikibooks.org/wiki/Haskell/Fix_and_recursion}{това}.}

\begin{haskellcode}
ghci> fact x = if x == 0 then 1 else x * fact(x-1)
ghci> fact 5
120
ghci> fix f = x where x = f x
ghci> :t f
fix :: (t -> t) -> t
ghci> fact' = fix \f x -> if x == 0 then 1 else x * f(x-1)
ghci> fact' 5
120
ghci> gamma f = \x -> if x == 0 then 1 else x * f(x-1)
ghci> :t gamma
(t -> t) -> t -> t
ghci> fact'' = fix gamma
ghci> fact'' 5
120
ghci> fix' f = x where x = f(f(f(x)))
ghci> fact''' = fix' gamma
ghci> fact''' 5
120
\end{haskellcode}


%%% Local Variables:
%%% mode: latex
%%% TeX-master: "../sep"
%%% End:



\section{Най-малко решение на система от уравнения}\index{система}



Нека $\A_1,\dots,\A_n$ са области на Скот 
и да разгледаме изображенията
\[f_i:\prod^n_{k=1}\A_k \to \A_i,\] за $i = 1,\dots,n$.
\index{решение на система}
Казваме, че $\bar{a} = \pair{a_1,\dots,a_n}$ е {\bf решение на системата}

\begin{align*}
  \bigstar = 
  \begin{cases}
    &x_1 = f_1(x_1,\dots,x_n)\\
    & \ \vdots\\
    &x_n = f_n(x_1,\dots,x_n),
  \end{cases}
\end{align*}

ако са в сила равенствата
\begin{SystemEq}
  a_1 & = & f_1(a_1,\dots,a_n)\\
  & \vdots & \\
  a_n & = & f_n(a_1,\dots,a_n).  
\end{SystemEq}

\index{система!най-малко решение}

Казваме, че $\bar{a}$ е {\bf най-малкото решение} на системата $\bigstar$, ако за всяко друго решение $\bar{b}$
е изпълнено, че $\bar{a} \sqsubseteq \bar{b}$.

\begin{framed}
\begin{theorem}
  \label{th:sep:min-solution-system}
  За произволни изображения $f_i \in \Cont{\prod^n_{k=1}\A_k}{\A_i}$, за $i = 1,\dots,n$, системата
  \begin{SystemEq}
    x_1 & = & f_1(x_1,\dots,x_n)\\
    & \vdots & \\
    x_n & = & f_n(x_1,\dots,x_n),    
  \end{SystemEq}
  притежава най-малко решение.
\end{theorem}
\end{framed}
\begin{proof}
  Първо да дефинираме изображението
  \[g \df f_1\times\dots\times f_n : \prod^n_{k=1}\A_k \to  \prod^n_{k=1}\A_k,\]
  като 
  \[g(\ov{a}) \df \pair{f_1(\ov{a}),\dots,f_n(\ov{a})},\]
  което е непрекъснато според \Problem{cartesian-product:continuous}.
  От \hyperref[th:knaster-tarski]{Теоремата на Клини} знаем, че $g$ притежава най-малка неподвижна точка
  $\bar{a} = \pair{a_1,\dots,a_n}$. Ще проверим, че $\ov{a}$ е най-малкото решение на системата.
  \begin{itemize}
  \item 
    Понеже $\ov{a}$ е неподвижна точка на $g$, то
    \begin{align*}
      g(a_1,\dots,a_n) & = \pair{f_1(\ov{a}),\dots,f_n(\ov{a})} & \comment{\text{от деф. на }g}\\
                       & = \pair{a_1,\dots,a_n} & \comment{\ov{a}\text{ е неподвижна точка}}.
    \end{align*}
    Оттук директно следва, че $f_i(\bar{a}) = a_i$, за $i = 1, \dots, n$, и следователно $\ov{a}$ е решение на системата.
  \item
    Нека $\ov{b} = \pair{b_1,\dots,b_n}$ е друго решение на системата, т.е. 
    $f_i(\ov{b}) = b_i$, за $i = 1, \dots, n$. Тогава 
    $g(\ov{b}) = \pair{f_1(\ov{b}),\dots,f_n(\ov{b})} = \ov{b}$.
    Следователно $\bar{b}$ е неподвижна точна на $g$.
    Понеже $\ov{a} = \lfp(g)$, то $\ov{a} \sqsubseteq \ov{b}$.
  \end{itemize}
  Така достигнахме до извода, че $\ov{a}$ е най-малкото решение на системата.
\end{proof}

\begin{remark}
  Да разгледаме изображенията $f\in\Cont{\A\times\B}{\A}$, $g \in \Cont{\B}{\B}$ и
  системата от две уравнения:
  \begin{SystemEq}
    f(x,y) & = & x\\
    g(y) & = & y.    
  \end{SystemEq}
  % \[\left|
  %     \begin{array}{lcl}

  %     \end{array}
  %   \right.\]
  % \begin{align*}
  %   \left|
  %   & f(x,y) = x\\
  %   & g(y) = y.
  %     \right.
  % \end{align*}
  За да можем директно да се позовем на \Th{sep:min-solution-system} и да твърдим, че тази система има най-малко решение,
  ние трябва да разгледаме следната модификация на системата:
  \begin{SystemEq}
    f(x,y)       & = & x\\
    \hat{g}(x,y) & = & y,    
  \end{SystemEq}
  където $\hat{g}(x,y) = g(y)$, т.е. добавяме един фиктивен аргумент, защото искаме всички изображения да имат равен брой аргументи.
\end{remark}


Ще завършим този раздел с две твърдения, които ще улеснят нашите разсъждения при 
решаването на задачи.

\begin{framed}
  \begin{proposition}\label{pr:system:independent}
    Да разгледаме две изображения
    \begin{align*}
      & f \in \Cont{\A\times\B}{\A}\\
      & g \in \Cont{\B}{\B},
    \end{align*}
    за които имаме системата от уравнения
    \begin{align*}
      \bigstar = 
      \begin{cases}
        & f(x,y) = x\\
        & g(y) = y.
      \end{cases}
    \end{align*}  
    Нека $b_0 = \lfp(g)$ и $a_0 = \lfp(\hat{f})$, където $\hat{f}(a) \df f(a,b_0)$.
    Тогава $\pair{a_0,b_0}$ е най-малкото решение на системата $\bigstar$.
  \end{proposition}
\end{framed}
\begin{proof}
  \begin{itemize}
  \item
    Първо, понеже $b_0 = \lfp(g)$, то очевидно $g(b_0) = b_0$.
    Освен това, $a_0 = \lfp(\hat{f})$, откъдето следва, че $a_0 = f(a_0,b_0)$.
    Ясно е, че $\pair{a_0,b_0}$ е решение на системата $\bigstar$.
  \item
    Сега нека $\pair{a,b}$ е произволно решение на системата $\bigstar$.
    Да видим, че $\pair{a_0,b_0} \sqsubseteq \pair{a,b}$.
    \begin{itemize}
    \item 
      Първо, ясно е, че $b = g(b)$. Понеже $b_0 = \lfp(g)$, то $b_0 \sqsubseteq b$.
    \item
      Второ, ясно е, че 
      \begin{align*}
        a & = f(a,b) & \comment{a \text{ е решение на }\bigstar}\\
          & \sqsupseteq f(a,b_0) & \comment{b \sqsupseteq b_0}\\
          & = \hat{f}(a) & \comment{\text{от деф.}}
      \end{align*}
      Получихме, че $a \in \texttt{Pref}(\hat{f})$.
      От \Prop{prefix-point} знаем, че 
      \[a_0 \df \lfp(\hat{f}) \sqsubseteq a.\]
    \end{itemize}
    Заключваваме, че $\pair{a_0,b_0} \sqsubseteq \pair{a,b}$.
  \end{itemize}
\end{proof}

Нещата започнаха да стават прекалено абстрактни, затова нека да видим един прост пример, който показва,
че всъщност горното твърдение е близо до нашата интуиция.
\Stefan{По-долу в примера трябва да се цитира горното твърдение по някакъв разбираем начин.}

\begin{example}
  Нека да разгледаме следната програма на езика \texttt{хаскел}:
\begin{haskellcode}
ghci> let g(x,y) = if x == 0 then 0 else g(x-1,y) + y
ghci> let f(x) = if x == 0 then 1 else g(x,f(x-1))
\end{haskellcode}
Лесно се съобразява, че всъщност
\[g(x,y) = x * y.\]
Това означава, че можем да пренапишем дефиницията на $f$ по следния начин:
\begin{haskellcode}
ghci> let f(x) = if x == 0 then 1 else x * f(x - 1)
\end{haskellcode}
Сега лесно се съобразява, че $f(x) = x!$, за $x \in \Nat$.
Да не забравяме, че в \texttt{хаскел} имаме и константатa {\texttt undefined}.
Това означава, че ако се ограничим до $\Nat_\bot$, то по горния начин сме дефинирали следните две функции:
\begin{align}
  \label{eq:4}
  f(x) = & 
  \begin{cases}
    x!,   & \text{ако }x \in \Nat\\
    \bot, & \text{ако }x = \bot
  \end{cases}
  \\
  \label{eq:5}
  g(x,y) = &
  \begin{cases}
    x\cdot y, & \text{ако }x,y \in \Nat\\
    \bot,     & \text{ако }\bot \in \{x,y\}.
  \end{cases}  
\end{align}

Ясно е, че тези функции са монотонни, а следователно и непрекъснати.
Целта на \Chapter{fun} е да формализираме разсъжданията, които направихме по-горе.
Ще видим, че на тази програма можем да съпоставим система от {\em непрекъснати} изображения.

\marginpar{$x + \bot \df \bot$}
\marginpar{В \Chapter{fun} ще видим, че на всяка програма съпоставяме система от {\em непрекъснати} изображения. В конкретния пример можем директно да докажем, че $\Gamma$ и $\Delta$ са непрекъснати изображения.}

\begin{align*}
  \Gamma(f,g)(x) =
  \begin{cases}
    1, & x = 0\\
    g(x, f(x-1)), & x > 0\\
    \bot, & x = \bot\\
  \end{cases}
  \\
  \Delta(g)(x,y) = 
  \begin{cases}
    0, & x = 0\\
    g(x-1,y) + y, & x > 0\\
    \bot, & x = \bot.
  \end{cases}
\end{align*}

Да видим как можем да дефинираме тези изображения на {\texttt haskell}
и как можем получим редицата от апроксимации на най-малките неподвижни точки по Теоремата на Клини.

\begin{haskellcode}
ghci> let gamma(f, g)(x) = if x == 0 then 1 else g(x, f(x - 1))
ghci> let delta(g)(x, y) = if x == 0 then 0 else g(x - 1, y) + y
-- Започваме да строим редицата от апроксимации по Теоремата на Клини
ghci> let g1 = delta( \(x,y) -> undefined )
ghci> let g2 = delta(g1)
ghci> let g3 = delta(g2)
ghci> g3(2,4)
8
ghci> g3(3,4)
*** Exception: Prelude.undefined
-- Можем да подходим и по-мързеливо, като направо дефинираме безкрайния
-- списък от тези апроксимации.
ghci> let approx = (\(x,y) -> undefined):[delta(g) | g <- approx]
ghci> let g9 = approx !! 9
ghci> g9(8,100)
800
ghci> g9(9,100)
*** Exception: Prelude.undefined
-- най-малката неподвижна точка на delta
ghci> let psi(x) = (approx !! (x+1))(x) 
\end{haskellcode}

Горният пример ни подсказва, че с индукция по $k$, можем да докажем, че 
ако имаме редицата
\begin{align*}
  & g_0 = \bm{\bot}^{(2)}\\
  & g_{k+1} = \Delta(g_k),
\end{align*}
то, за произволнен индекс $k$, имаме
\[g_k(x,y) =
\begin{cases}
  x \cdot y, & \text{ако }x < k\text{ и }y \in \Nat\\
  \bot, & \text{иначе}.
\end{cases}\]
Тогава с помощта на Теоремата на Клини можем да докажем, че
\[\lfp(\Delta)(x,y) =
\begin{cases}
  x \cdot y, & \text{ако }x,y\in\Nat\\
  \bot,      & \text{ако }\bot \in \{x,y\}.
\end{cases}\]

Нека сега да разгледаме изображението
\[\hat\Gamma(f)(x) \df \Gamma(f, \lfp(\Delta))(x) = 
\begin{cases}
  1,              & \text{ако }x = 0\\
  x \cdot f(x-1), & \text{ако }x > 0\\
  \bot,           & \text{ако }x = \bot.
\end{cases}\]

Нека отново да видим как можем да дефинираме това изображение на {\texttt haskell}
и как можем получим редицата от апроксимации на най-малките неподвижни точки по Теоремата на Клини.
\begin{haskellcode}
ghci> let gamma(f,g)(x) = if x == 0 then 1 else g(x,f(x-1))
ghci> let gamma'(f) = gamma(f, \(x, y) -> x * y)
ghci> :t gamma'
gamma' :: (a -> a) -> a -> a
ghci> let approx' = (\x -> undefined):[gamma'(f) | f <- approx']
ghci> let f9 = approx' !! 9
ghci> f9(8)
40320
ghci> f9(9)
*** Exception: Prelude.undefined 
ghci> let phi(x) = (approx' !! (x+1))(x)
ghci> phi(8)    -- phi е най-малката неподвижна точка на gamma'
40320           -- лесно се съобразява, че phi(x) == x!
\end{haskellcode}

Горният пример ни подсказва, че с индукция по $k$, можем да докажем,
че ако имаме редицата
\begin{align*}
  & f_0 = \bm{\bot}^{(1)}\\
  & f_{k+1} = \hat\Gamma(f_k),
\end{align*}
то, за произволен индекс $k$, имаме
\[f_k(x) =
\begin{cases}
  x!, & \text{ако }x < k\\
  \bot, & \text{иначе}.
\end{cases}\]

\noindent Отново по Теоремата на Клини, 
\[\lfp(\hat\Gamma)(x) =
\begin{cases}
  x!, & \text{ако }x \in \Nat\\
  \bot, & \text{ако }x = \bot.
\end{cases}\]
            
От \Prop{system:independent} знаем, че двойката $(\lfp(\hat\Gamma)),\lfp(\Delta))$ е най-малкото решение на системата,
което е точно двойката изображения $(f,g)$ с дефиниции (\ref{eq:4}) и (\ref{eq:5}).
\end{example}

\begin{framed}
  \begin{proposition}\label{pr:system:definition}
    Да разгледаме изображенията $f \in \Cont{\A}{\B}$ и $g \in \Cont{\A}{\A}$
    и системата от две уравнения:
    \begin{align*}
      \bigstar = 
      \begin{cases}
        & f(y) = x\\
        & g(y) = y.
      \end{cases}
    \end{align*}  
    Нека $a_0 = \lfp(g)$.
    Тогава най-малкото решение на системата $\bigstar$ е наредената двойка
    \[\pair{f(a_0), a_0}.\]
  \end{proposition}
\end{framed}
\begin{proof}
  \begin{itemize}
  \item 
    Лесно се съобразява, че $\pair{f(a_0), a_0}$ е решение на системата $\bigstar$.
  \item
    Нека $\pair{c,d}$ е решение на системата $\bigstar$.
    Тогава $g(d) = d$ и понеже $a_0 = \lfp(g)$, то $a_0 \sqsubseteq d$.
    Освен това, $c = f(d) \sqsupseteq f(a_0)$.
    Получихме, че $\pair{f(a_0), a_0} \sqsubseteq \pair{c,d}$.
  \end{itemize}
  Заключаваме, че $\pair{f(a_0), a_0}$ е най-малкото решение на системата $\bigstar$.
\end{proof}

\Stefan{Тук пак трябва да се обясни как горното твърдение се използва в долния пример.}

\begin{example}
Да разгледаме следната програма:
  \begin{haskellcode}
ghci> :{  -- използване на multiline дефиниции
ghci> let g(x, y, z) = if x == y + z then z 
ghci|                    else if z == x + 1 then 0 
ghci|                      else g(x, y, z + 1)
ghci| :}
ghci> let f(x, y) = g(x, y, 0)
  \end{haskellcode}

Лесно се съобразява, че 
\[g(x,y) = 
\begin{cases}
  x - y, & \text{ако }x \geq y\\
  0, & \text{ако }x < y.
\end{cases}\]

Тази функция ще я означаваме като $x \monus y$.
На горната програма можем да съпоставим системата от непрекъснати изображения:

\begin{align*}
  \Gamma(g)(x,y) & = g(x,y,0)\\
  \Delta(g)(x,y,z) & = \begin{cases}
    z, & \text{ако } x = y+z\\
    0, & \text{ако } z = x + 1\\
    g(x,y,z+1), & \text{ иначе и }x,y,z\in\Nat\\
    \bot, & \bot \in \{x,y,z\}.
  \end{cases}
\end{align*}


\begin{haskellcode}
ghci> :{  -- Multiline
ghci> let delta(g)(x, y, z) = if x == y + z then z 
ghci|                           else if z == x + 1 then 0 
ghci|                             else g(x, y, z + 1)
ghci| :}
ghci> :t delta
delta :: ((t, t, t) -> t) -> (t, t, t) -> t
ghci> let approx = (\(x,y,z) -> undefined):[delta(g) | g <- approx]
ghci> let g9 = approx !! 9
ghci> g9(20,11,1)  -- 20-11 \in [1, 10)
9
ghci> g9(20,1,11) -- 20-1 \in [11, 20)
19
ghci> g9(2,11,4)  -- 2+1 \not\in [4, 13)
*** Exception: Prelude.undefined
\end{haskellcode}

Горният пример ни подсказва, че с индукция по $k$, можем да докажем,
че ако имаме редицата
\begin{align*}
  & g_0 = \bm{\bot}^{(3)}\\
  & g_{k+1} = \Delta(g_k),
\end{align*}
то, за произволно $k$, имаме
\[g_k(x,y,z) =
\begin{cases}
  0,   & x + 1\in [z,z+k)\\
  x-y, & x \geq y\ \&\ x-y \in [z,z+k)\\
  \bot, & \text{иначе}.
\end{cases}\]
Тогава можем да приложим Теоремата на Клини и да докажем, че
\[\lfp(\Delta)(x,y,z)  =
\begin{cases}
  x \monus y, & z \leq x+1\\
  \bot, & z > x+1\text{ или } \bot \in \{x,y,z\}.
\end{cases}\]
Тогава от \Prop{system:definition} следва, че
\[\lfp(\Gamma)(x,y) = \lfp(\Delta)(x,y,0) =
\begin{cases}
  x \monus y, & x,y\in\Nat\\
  \bot, & \bot \in \{x,y\}.
\end{cases}\]

Съобразете, че $\lfp(\Gamma) = \bigsqcup_k f_k$,
където $f_k(x,y) = g_k(x,y,0)$.

\end{example}



%%% Local Variables:
%%% mode: latex
%%% TeX-master: "../sep"
%%% End:

% \newpage
% \input{domains-basic/compact}
% \newpage
% \section{Задачи}  

Тук с $\A$, $\B$ и $\C$ ще означаваме области на Скот.

\marginpar{Много от задачите са от \cite[стр. 31]{abramsky94}}

\begin{problem}
  Да разгледаме операторите \[\Gamma,\Delta \in \Cont{\Cont{\A}{\A}}{\Cont{\A}{\A}}.\]
  Знаем, че операторът $\Gamma \circ \Delta$ е непрекъснат, където
  \[(\Gamma\circ\Delta)(f) \df \Gamma(\Delta(f)).\]
  Вярно ли е, че
  \[\lfp(\Gamma \circ \Delta) \sqsubseteq \lfp(\Gamma) \circ \lfp(\Delta)?\]
  Обосновете се!
\end{problem}
\ifhints
\begin{hint}
  Нека $\A = \Nat_\bot$.
  Нека например да разгледаме
  \begin{align*}
    & \Delta(f)(x) \df f(x+1)\\
    & \Gamma(f)(x) \df
      \begin{cases}
        0, & x \neq \bot\\
        \bot, & x = \bot.
      \end{cases}
  \end{align*}
  Да положим $f_\Gamma \df \lfp(\Gamma)$ и $f_\Delta \df \lfp(\Delta)$.
  Ясно е, че 
  \begin{align*}
    & f_\Delta(x) = \bot\\
    & f_\Gamma(x) =
    \begin{cases}
      0, & x \neq \bot\\
      \bot, & x = \bot.
    \end{cases}  
  \end{align*}
  Тогава за произволно $x \in \Nat_\bot$,
  \[(f_\Gamma\circ f_\Delta)(x) = f_\Gamma(f_\Delta(x)) = f_\Gamma(\bot)  = \bot.\]
  От друга страна, понеже $(\Gamma \circ \Delta)(f) = \Gamma(\Delta(f))$, то 
  \begin{align*}
    & (\Gamma \circ \Delta)(f)(x) = \Gamma(\Delta(f))(x) = 
      \begin{cases}
        0, & x \neq \bot\\
        \bot, & x = \bot.
      \end{cases}
  \end{align*}
  Лесно се съобразява, че 
  \[\lfp(\Gamma \circ \Delta)(x) =
  \begin{cases}
    0, & x \neq \bot\\
    \bot, & x = \bot.
  \end{cases}\]
  Заключаваме, че 
  \[\lfp(\Gamma \circ \Delta) \sqsupset \lfp(\Gamma) \circ \lfp(\Delta).\]
\end{hint}
\fi

\begin{problem}
  Да разгледаме операторите \[\Gamma,\Delta \in \Cont{\Cont{\A}{\A}}{\Cont{\A}{\A}}.\]
  Знаем, че операторът $\Gamma \circ \Delta$ е непрекъснат, където
  \[(\Gamma\circ\Delta)(f) \df \Gamma(\Delta(f)).\]
  Вярно ли е, че
  \[\lfp(\Gamma \circ \Delta) \sqsupseteq \lfp(\Gamma) \circ \lfp(\Delta)?\]
  Обосновете се!
\end{problem}
\ifhints
\begin{hint}
  Нека $\A = \Nat_\bot$.
  Нека например да разгледаме
  \begin{align*}
    & \Delta(f)(x) \df 0\\
    & \Gamma(f)(x) \df
      \begin{cases}
        0, & x = 0\\
        \bot, & \text{ иначе}.
      \end{cases}
  \end{align*}
  Да положим $f_\Gamma \df \lfp(\Gamma)$ и $f_\Delta \df \lfp(\Delta)$.
  Ясно е, че 
  \begin{align*}
    & f_\Delta(x) = 0\\
    & f_\Gamma(x) =
    \begin{cases}
      0, & x = 0\\
      \bot, & \text{ иначе}.
    \end{cases}  
  \end{align*}
  Тогава за произволно $x \in \Nat_\bot$,
  \[(f_\Gamma\circ f_\Delta)(x) = f_\Gamma(f_\Delta(x)) = f_\Gamma(0)  = 0.\]
  От друга страна, понеже $(\Gamma \circ \Delta)(f) = \Gamma(\Delta(f))$, то 
  \begin{align*}
    & (\Gamma \circ \Delta)(f)(x) = \Gamma(\Delta(f))(x) = 
      \begin{cases}
        0, & x = 0\\
        \bot, & \text{ иначе}.
      \end{cases}
  \end{align*}
  Лесно се съобразява, че 
  \[\lfp(\Gamma \circ \Delta)(x) =
  \begin{cases}
    0, & x = 0\\
    \bot, & \text{ иначе}.
  \end{cases}\]
  Заключаваме, че 
  \[\lfp(\Gamma \circ \Delta) \sqsubset \lfp(\Gamma) \circ \lfp(\Delta).\]
\end{hint}
\fi

\begin{problem}
  Нека $f_0 \sqsubseteq f_1 \sqsubseteq f_2 \sqsubseteq \cdots$
  е верига от елементи на $\Cont{\A}{\A}$.
  Да положим $h = \bigsqcup_n f_n$.
  Вярно ли е, че 
  \[h \circ h = \bigsqcup_n (f\circ f)?\]
  Обосновете се!
\end{problem}

\begin{problem}
  Да разгледаме едно изображение $f: \A \times \B \to \C$.
  За произволно $a \in \A$, дефинираме изображението $g_a: \B \to \C$, където
  \[g_a(b) \df f(a,b).\]
  Аналогично, за произволно $b \in \B$, дефинираме изображението $h_b: \A \to \C$, където
  \[h_b(a) \df f(a,b).\]
  Докажете, че следните твърдения са еквивалентни:
  \begin{enumerate}[1)]
  \item 
    $f$ е непрекъснато изображение;
  \item
    $g_a$ и $h_b$ са непрекъснати изображения, за всяко $a \in \A$ и $b \in \B$.
  \end{enumerate}
\end{problem}

\begin{problem}
  Да разгледаме $f \in \Cont{\A \times \B}{\C}$.
  За произволно $a \in \A$, дефинираме изображението $g_a: \B \to \C$, където
  \[g_a(b) \df f(a,b).\]
  Вече знаем, че $g_a \in \Cont{\B}{\C}$, за всяко $a \in \A$.
  Да разгледаме изображението $h:\A \to \Cont{\B}{\C}$, където
  \[h(a) \df g_a.\]
  Докажете, че $h$ е непрекъснато изображение.
\end{problem}

\begin{problem}
  \marginpar{\cite[стр. 129]{nikolova-soskova}}
  Да разгледаме $f \in \Cont{\A \times \B}{\B}$.
  За произволно $a \in \A$, дефинираме изображението $g_a: \B \to \B$, където
  \[g_a(b) \df f(a,b).\]
  Вече знаем, че $g_a \in \Cont{\B}{\B}$, за всяко $a \in \A$,
  следователно $\lfp(g_a)$ съществува.
  Да разгледаме изображението $h:\A \to \B$, където
  \[h(a) \df \lfp(g_a).\]
  Докажете, че $h$ е непрекъснато изображение.
\end{problem}


\begin{problem}
  Да разгледаме $f \in \Cont{\A}{\Cont{\B}{\C}}$.
  За произволно $a \in \A$, дефинираме изображението $g_a \in \Cont{\B}{\C}$, където
  \[g_a(b) \df f(a).\]
  Да разгледаме изображението $h:\A\times \B \to \C$, където
  \[h(a,b) \df g_a(b).\]
  Докажете, че $h$ е непрекъснато изображение.
\end{problem}

% \begin{problem}
%   Нека са дадени областите на Скот $\D$ и $\E$ и изображението 
%   \[\texttt{eval}: \Cont{\D}{\E} \times \D \to \E,\]
%   където 
%   \[\texttt{eval}(f,d) \df f(d).\]
%   Докажете, че $\texttt{eval}$ е непрекъснато изображение.
% \end{problem}
% \ifhints
% \begin{hint}
%   Понеже $\bigsqcup_n(f_n,d_n) = (\bigsqcup_m f_m,\bigsqcup_n d_n)$, 
%   ще докажем, че \[\texttt{eval}(\bigsqcup_m f_m, \bigsqcup_n d_n) = \bigsqcup_n \texttt{eval}(f_n,d_n).\]
%   Знаем, че
%   \begin{align*}
%     \texttt{eval}(\bigsqcup_m f_m, \bigsqcup_n d_n) & = (\bigsqcup_m f_m)(\bigsqcup_n d_n) & (\mbox{от опр. на }\texttt{eval})\\
%     & = \bigsqcup_m (f_m(\bigsqcup_n d_n)) & (\mbox{от опр. на }\bigsqcup_mf_m)\\
%     & = \bigsqcup_m (\bigsqcup_n (f_m(d_n)) & (\mbox{всяка }f_m\mbox{ е непр.} )\\
%   \end{align*}
%   Нека да положим $e_{m,n} = f_m(d_n)$.
%   Лесно се съобразява, че
%   \[m \leq m^\prime\ \&\ n \leq n^\prime\ \Rightarrow\ e_{m,n} \sqsubseteq^\E e_{m^\prime,n^\prime}.\]
%   Така получаваме, че 
%   \begin{align*}
%     \texttt{eval}(\bigsqcup_m f_m, \bigsqcup_n d_n) & = \bigsqcup_m (\bigsqcup_n (f_m(d_n)) & (\mbox{от по-горе})\\
%     & = \bigsqcup_{m,n} e_{m,n} = \bigsqcup_{n} e_{n,n} & (\mbox{от \Th{double-chain}})\\
%     & = \bigsqcup_{n} f_n(d_n) & (\text{ от опр. на }e_{m,n})\\
%     & = \bigsqcup_n \texttt{eval}(f_n,d_n).
%   \end{align*}
% \end{hint}
% \fi

\begin{problem}
  Нека са дадени областите на Скот $\D$ и $\E$ и изображението 
  \[\texttt{eval}: \Cont{\D}{\E} \times \D \to \E,\]
  където 
  \[\texttt{eval}(f,d) \df f(d).\]
  Докажете, че $\texttt{eval}$ е непрекъснато изображение.
\end{problem}
\ifhints
\begin{hint}
  Понеже $\bigsqcup_n(f_n,d_n) = (\bigsqcup_m f_m,\bigsqcup_n d_n)$, 
  ще докажем, че \[\texttt{eval}(\bigsqcup_m f_m, \bigsqcup_n d_n) = \bigsqcup_n \texttt{eval}(f_n,d_n).\]
  Знаем, че
  \begin{align*}
    \texttt{eval}(\bigsqcup_m f_m, \bigsqcup_n d_n) & = (\bigsqcup_m f_m)(\bigsqcup_n d_n) & \comment\text{от опр. на }\texttt{eval}\\
                                                    & = \bigsqcup_m (f_m(\bigsqcup_n d_n)) & \comment\text{от опр. на }\bigsqcup_mf_m\\
                                                    & = \bigsqcup_m (\bigsqcup_n (f_m(d_n)) & \comment\text{всяка }f_m\mbox{ е непр.}\\
  \end{align*}
  Нека да положим $e_{m,n} = f_m(d_n)$.
  Лесно се съобразява, че
  \[m \leq m^\prime\ \&\ n \leq n^\prime\ \Rightarrow\ e_{m,n} \sqsubseteq^\E e_{m^\prime,n^\prime}.\]
  Така получаваме, че 
  \begin{align*}
    \texttt{eval}(\bigsqcup_m f_m, \bigsqcup_n d_n) & = \bigsqcup_m (\bigsqcup_n (f_m(d_n)) & \comment\text{от по-горе}\\
    & = \bigsqcup_{m,n} e_{m,n} = \bigsqcup_{n} e_{n,n} & \comment\text{от \Th{double-chain}}\\
    & = \bigsqcup_{n} f_n(d_n) & \comment\text{ от опр. на }e_{m,n}\\
    & = \bigsqcup_n \texttt{eval}(f_n,d_n).
  \end{align*}
\end{hint}
\fi

%%% Local Variables:
%%% mode: latex
%%% TeX-master: "../sep"
%%% End:

\begin{problem}
  Нека изображението \[\texttt{comp}:(\Cont{\B}{\C} \times \Cont{\A}{\B}) \to \Cont{\A}{\C}\]
  е определено като 
  \[\texttt{comp}(g,f) \df g\circ f.\]
  \marginpar{$(g \circ f)(a) = g(f(a))$}
  Докажете, че $\texttt{comp}$ е непрекъснато изображение.
\end{problem}
\ifhints
\begin{hint}
  \marginpar{\cite[стр. 124]{reynolds}}
  Трябва да докажем, че за всяка монотонно растяща редица $\{(g_n,f_n)\}_{n\in\Nat}$,
  \[\Gamma(\bigsqcup_n(g_n,f_n))(a) = \bigsqcup_n\Gamma(g_n,f_n)(a),\]
  за произволно $a \in A$.
  Да фиксираме едно $a\in A$ и да положим $g_n(f_k(a)) = e_{n,k}$.
  Лесно се вижда, че 
  \[n\leq n^\prime\ \&\ k \leq k^\prime\ \Rightarrow\ e_{n,k} \sqsubseteq e_{n^\prime,k^\prime}.\]
  Тогава:
  \begin{align*}
    \Gamma(\bigsqcup_n(g_n,f_n))(a) & = \Gamma(\bigsqcup_n g_n, \bigsqcup_k f_k)(a) & \\
                                    & = (\bigsqcup_n g_n)(\bigsqcup_k f_k(a)) & \comment\text{ по деф. на }\Gamma\\
                                    & = (\bigsqcup_n g_n)(\bigsqcup_k b_k) & \comment\text{ полагаме }b_k = f_k(a)\\
                                    & = \bigsqcup_k (\bigsqcup_n g_n)(b_k) & \comment\bigsqcup_n g_n\text{ е непр.}\\
                                    & = \bigsqcup_k(\bigsqcup_n g_n(b_k)) & \comment\text{ по деф. на }\bigsqcup_n g_n\\
                                    & = \bigsqcup_k (\bigsqcup_n g_n(f_k(a))) & \comment\text{ полагаме }e_{n,k} = g_n(f_k(a))\\
                                    & = \bigsqcup_k\bigsqcup_n e_{n,k} = \bigsqcup_n e_{n,n} & \comment\text{ от \Th{double-chain}}\\
                                    & = \bigsqcup_n g_n(f_n(a)) = \bigsqcup_n \Gamma(g_n, f_n)(a).
  \end{align*}
\end{hint}
\fi

\begin{remark}
  \marginpar{Когато на хаскел пишем $(.)$, означава, че операцията е инфиксна}
  В хаскел има операция композиция на функции.
  \begin{haskellcode}
ghci> :t (.)
(.) :: (b -> c) -> (a -> b) -> a -> c
  \end{haskellcode}
\end{remark}


%%% Local Variables:
%%% mode: latex
%%% TeX-master: "../sep"
%%% End:


% \begin{problem}
%   Нека изображението \[\texttt{comp}:(\Cont{\B}{\C} \times \Cont{\A}{\B}) \to \Cont{\A}{\C}\]
%   е определено като 
%   \[\texttt{comp}(g,f) \df g\circ f.\]
%   \marginpar{$(g \circ f)(a) = g(f(a))$}
%   Докажете, че $\texttt{comp}$ е непрекъснато изображение.
% \end{problem}
% \ifhints
% \begin{hint}
%   \marginpar{\cite[стр. 124]{reynolds}}
%   Трябва да докажем, че за всяка монотонно растяща редица $\{(g_n,f_n)\}_{n\in\Nat}$,
%   \[\Gamma(\bigsqcup_n(g_n,f_n))(a) = \bigsqcup_n\Gamma(g_n,f_n)(a),\]
%   за произволно $a \in A$.
%   Да фиксираме едно $a\in A$ и да положим $g_n(f_k(a)) = e_{n,k}$.
%   Лесно се вижда, че 
%   \[n\leq n^\prime\ \&\ k \leq k^\prime\ \Rightarrow\ e_{n,k} \sqsubseteq e_{n^\prime,k^\prime}.\]
%   Тогава:
%   \begin{align*}
%     \Gamma(\bigsqcup_n(g_n,f_n))(a) & = \Gamma(\bigsqcup_n g_n, \bigsqcup_k f_k)(a) & \\
%     & = (\bigsqcup_n g_n)(\bigsqcup_k f_k(a)) & (\text{ по деф. на }\Gamma )\\
%     & = (\bigsqcup_n g_n)(\bigsqcup_k b_k) & (\text{ полагаме }b_k = f_k(a))\\
%     & = \bigsqcup_k (\bigsqcup_n g_n)(b_k) & (\bigsqcup_n g_n\text{ е непр.})\\
%     & = \bigsqcup_k(\bigsqcup_n g_n(b_k)) & (\text{ по деф. на }\bigsqcup_n g_n)\\
%     & = \bigsqcup_k (\bigsqcup_n g_n(f_k(a))) & (\text{ полагаме }e_{n,k} = g_n(f_k(a)))\\
%     & = \bigsqcup_k\bigsqcup_n e_{n,k} = \bigsqcup_n e_{n,n} & (\text{ от \Th{double-chain}})\\
%     & = \bigsqcup_n g_n(f_n(a)) = \bigsqcup_n \Gamma(g_n, f_n)(a).
%   \end{align*}
% \end{hint}
% \fi

% \begin{remark}
%   \marginpar{Когато пишем $(.)$ означава, че операцията е инфиксна}
%   В хаскел има операция композиция на функции.
%   \begin{haskellcode}
% ghci> :t (.)
% (.) :: (b -> c) -> (a -> b) -> a -> c
%   \end{haskellcode}
% \end{remark}

\begin{problem}
  \marginpar{\cite[стр. 131]{nikolova-soskova}}
  Нека $f \in \Cont{\A}{\B}$ и $g \in \Cont{\B}{\A}$.
  Докажете, че 
  \begin{itemize}
  \item 
    $\lfp(g \circ f) \sqsubseteq g(\lfp(f \circ g))$;
  \item
    $f(\lfp(g \circ f)) \sqsubseteq \lfp(f \circ g)$.
  \end{itemize}
  Оттук заключете, че 
  \[\lfp(g \circ f) = g(\lfp(f \circ g)) \text{ и }f(\lfp(g \circ f)) = \lfp(f \circ g).\]
\end{problem}


% \begin{problem}[Кантор-Шрьодер-Бернщайн]
%   \marginpar{\cite[стр. 639]{hanbook-cs}}
%   Нека имаме две инективни функции $f:A \to B$ и $g:B \to A$.
%   Тогава съществува биективна функция $h: A \to B$.  
% \end{problem}
% \begin{hint}
%   За множеството $B$, да дефинираме областта на Скот 
%   \[\D_B = (\Ps(B),\subseteq,\emptyset).\]
%   \begin{enumerate}[a)]
%   \item 
%     За дадените от условието инективни функции $f$ и $g$,
%     да разгледаме изображението $\Gamma:\D_B \to \D_B$ зададено като
%     \marginpar{Озн. $h(X) = \{h(x) \mid x\in X\}$, $h^{-1}(X) = \{z \mid h(z) \in X\}$}
%     \[\Gamma(X) = B\setminus f(A)\cup f(g(X)).\]
%     Докажете, че $F$ е непрекъснато изображение.
%   \item
%     \Stefan{Използвам, че $X_0$ е неподвижна точка, но не виждам къде използвам, че е най-малката.}
%     Нека $X_0 = \lfp(\Gamma)$. Тогава $X_0 = B\setminus f(A) \cup f(g(X_0))$.
%     Докажете, че 
%     \[B \setminus X_0 = f(A \setminus g(X_0)).\]
%   \item
%     Дефинираме функцията $h:A \to B$ по следния начин:
%     \begin{align*}
%       h(a) = 
%       \begin{cases}
%         g^{-1}(a), & a \in g(X_0)\\
%         f(a), & a \in A \setminus g(X_0).
%       \end{cases}
%     \end{align*}
%     Докажете, че $h$ е биекция.
%   \end{enumerate}
% \end{hint}

% \begin{problem}% Gunter textbook
%   Нека $f \in \Cont{\A}{\A}$.
%   Да разгледаме множеството 
%   \[B = \{a \in \A \mid f(a) = a\}.\]
%   Вярно ли е, че 
%   \[\B = (B, \sqsubseteq^\A, \lfp(f))\] е област на Скот?
%   Обосновете се!
% \end{problem}

\begin{problem}% Gunter textbook
  Нека $f \in \Cont{\A}{\A}$.
  Да разгледаме множеството 
  \[B = \{a \in \A \mid f(a) \sqsubseteq a\}.\]
  Вярно ли е, че 
  \[\B = (B, \sqsubseteq^\A, \lfp(f))\] е област на Скот?
  Обосновете се!
\end{problem}


\begin{problem} % Gunter textbook
  Да разгледаме множеството
  \[B = \{f \in \Mon{\A}{\A} \mid f\circ f = f\}.\]
  Вярно ли е, че 
  \[\B = (B,\ \sqsubseteq,\ \lambda x.\bot^\A)\] е област на Скот,
  където 
  \[f \sqsubseteq g \df (\forall a\in\A)[f(a) \sqsubseteq^\A g(a)] ?\]
  Обосновете се!
\end{problem}

% \begin{problem} % Gunter textbook
%   Да разгледаме множеството
%   \[B = \{f \in \Strict{\A}{\A} \mid f\circ f = f\}.\]
%   Вярно ли е, че 
%   \[\B = (B,\ \sqsubseteq,\ \lambda x.\bot^\A)\] е област на Скот,
%   където 
%   \[f \sqsubseteq g \df (\forall a\in\A)[f(a) \sqsubseteq^\A g(a)] ?\]
%   Обосновете се!
% \end{problem}

\begin{problem} % Gunter textbook
  Да разгледаме множеството
  \[B = \{f \in \Strict{\Nat_\bot}{\Nat_\bot} \mid f\circ f = f\}.\]
  Вярно ли е, че 
  \[\B = (B,\ \sqsubseteq,\ \lambda x.\bot)\] е област на Скот,
  където 
  \[f \sqsubseteq g \df (\forall a\in\Nat_\bot)[f(a) \sqsubseteq g(a)] ?\]
  Обосновете се!
\end{problem}

\begin{problem}
  % \marginpar{\cite[стр. 124]{reynolds}}
  Нека $f \in \Mon{\A}{\B}$ и $g \in \Mon{\B}{\A}$ имат свойствата:
  \begin{itemize}
  \item 
    $f\circ g = id_\B$;
  \item
    $g \circ f = id_\A$.
  \end{itemize}
  Докажете, че $f$ и $g$ са точни и непрекъснати.
\end{problem}


\begin{problem}
  % \marginpar{задачата е \href{http://www.cl.cam.ac.uk/teaching/exams/pastpapers/y2008p8q14.pdf}{оттук} и \href{http://www.cl.cam.ac.uk/teaching/exams/pastpapers/y1998p9q10.pdf}{оттук}}
  Да разгледаме областта на Скот 
  \[\O = (\{\bot,\top\},\sqsubseteq, \bot),\]
  където $\bot \sqsubseteq \top$.
  За произволна област на Скот $\A$ и елемент $a \in \A$, $a \neq \bot$, дефинираме изображенията:
  \begin{enumerate}[a)]
  \item
    $f_a:\A \to \O$, където
    \[f_a(x) \df
    \begin{cases}
      \top, & a \sqsubseteq x\\
      \bot, & a \not\sqsubseteq x.
    \end{cases}\]
    Вярно ли е, че $f_a$ е точно непрекъснато изображение? Обосновете се!
  \item
    $\hat{f}_a:\A \to \O$, където
    \[\hat{f}_a(x) \df
    \begin{cases}
      \bot, & a \sqsubseteq x\\
      \top, & a \not\sqsubseteq x.
    \end{cases}\]
    Вярно ли е, че $\hat{f}_a$ е точно непрекъснато изображение? Обосновете се!
  \item 
    $g_a:\A \to \O$, където
    \[g_a(x) \df
    \begin{cases}
      \bot, & x \sqsubseteq a\\
      \top, & x \not\sqsubseteq a.
    \end{cases}\]
    Вярно ли е, че $g_a$ е точно непрекъснато изображение? Обосновете се!
  \item 
    $\hat{g}_a:\A \to \O$, където
    \[\hat{g}_a(x) \df
    \begin{cases}
      \top, & x \sqsubseteq a\\
      \bot, & x \not\sqsubseteq a.
    \end{cases}\]
    Вярно ли е, че $\hat{g}_a$ е точно непрекъснато изображение? Обосновете се!
  \item
    Докажете, че 
    \[f \in \Cont{\D}{\A} \iff (\forall a \in \A)[g_a \circ f \in \Cont{\D}{\O}].\]
  \end{enumerate}
\end{problem}

\begin{problem}
  Да разгледаме изображението
  \[\Gamma: \Cont{\A}{\B} \times \Cont{\A}{\C} \to \Cont{\A}{\B\times\C},\]
  където $\Gamma(f,g)(a) \df \pair{f(a),g(b)}$.
  \begin{itemize}
  \item
    Докажете, че $\Gamma$ е добре дефинирано изображение, т.е. за всеки непрекъснати $f$ и $g$,
    $\Gamma(f,g)$ е непрекъснато изображение.
  \item 
    Докажете, че $\Gamma$ е непрекъснато изображение.
  \end{itemize}
\end{problem}

\begin{problem}
  \marginpar{задачата е \href{http://www.cl.cam.ac.uk/teaching/exams/pastpapers/y2005p9q15.pdf}{оттук}}
  Докажете, че изображението
  \[\texttt{uncurry}:\Cont{\A}{\Cont{\B}{\C}} \to \Cont{\A\times \B}{\C},\]
  дефинирано като
  \[\texttt{uncurry}(f)(a,b) \df f(a)(b),\]
  е непрекъснато.
\end{problem}

% \begin{problem}
%   Докажете, че изображението
%   \[\texttt{curry}:\Cont{\A\times \B}{\C} \to \Cont{\A}{\Cont{\B}{\C}},\]
%   дефинирано като
%   \[\texttt{curry}(f)(a)(b) \df f(a,b),\]
%   е непрекъснато.
% \end{problem}

\begin{problem}
  Докажете, че изображението
  \[\texttt{curry}:\Cont{\A\times \B}{\C} \to \Cont{\A}{\Cont{\B}{\C}},\]
  дефинирано като
  \[\texttt{curry}(f)(a)(b) \df f(a,b),\]
  е непрекъснато.
\end{problem}

%%% Local Variables:
%%% mode: latex
%%% TeX-master: "../sep"
%%% End:


\newpage
\subsection{Регулярни езици}

Да фиксираме азбуката $\Sigma = \{a_1,\dots,a_k\}$.
Да дефинираме полиномите над $\Sigma$ като
\[\tau ::= \emptyset\ |\ \varepsilon\ |\ a_i \cdot X_j\ |\ \tau_1 + \tau_2.\]
където $i = 1, \dots,k$, а $X$ е променлива.
За всеки полином $\tau[X_1,\dots,X_n]$ дефинираме оператора 
\[\val{\tau}: \mathcal{P}(\Sigma^\star)^n \to \mathcal{P}(\Sigma^\star)\]
 по следния начин:
\begin{itemize}
\item
    $\val{\emptyset}(L_1,\dots,L_n) = \emptyset$.
\item 
  $\val{\varepsilon}(L_1,\dots,L_n) = \varepsilon$.
\item 
  $\val{a_i \cdot X_j}(L_1,\dots,L_n) = \{a_i\} \cdot L_j$.
\item
  $\val{\tau_1 + \tau_2}(L_1,\dots,L_n) = \val{\tau_1}(L_1,\dots,L_n) \cup \val{\tau_2}(L_1,\dots,L_n)$.
\end{itemize}

\begin{problem}
  Докажете, че за всеки полином $\tau$ имаме, че $\val{\tau}$ е непрекъснато изображение в областта на Скот
  $\mathcal{S} = ( \mathcal{P}(\Sigma^\star),\subseteq, \emptyset)$.
\end{problem}


\begin{example}
  Да разгледаме системата 
  \marginpar{$\tau_1[X_1,X_2] \equiv b \cdot X_1 + a \cdot X_2$}
  \marginpar{$\tau_2[X_1,X_2] \equiv \varepsilon$}
  \begin{align*}
    & X_1 = b \cdot X_1 + a\cdot X_2\\
    & X_2 = \varepsilon.
  \end{align*}

  % Понеже $\val{\tau}$ е непрекъснат оператор, то той има най-малка неподвижна точка.
  Дефинираме непрекъснатия оператор 
  \[\Gamma:\mathcal{P}(\Sigma^\star)^2 \to \mathcal{P}(\Sigma^\star)^2,\]
  където:
  \[\Gamma(L_1,L_2) = (\val{\tau_1}(L_1,L_2), \val{\tau_2}(L_1,L_2)).\]

  От Теоремата на Клини ние знам как можем да намерим най-малката неподвижна точка на $\Gamma$,
  която ще бъде и най-малкото решение на горната система.

  \begin{itemize}
  \item 
    $(L_0,M_0) \df (\emptyset,\emptyset)$;
  \item
    $(L_1,M_1) \df \Gamma(L_0,M_0) = (\val{\tau_1}(L_0,M_0), \val{\tau_2}(L_0,M_0)) = (\emptyset, \varepsilon)$;
  \item
    $(L_2,M_2) \df \Gamma(L_1,M_1) = (\val{\tau_1}(L_1,M_1), \val{\tau_2}(L_1,M_1)) = (\{a\},\varepsilon)$;
  \item
    $(L_3,M_3) \df \Gamma(L_2,M_2) = (\val{\tau_1}(L_2,M_2), \val{\tau_2}(L_2,M_2)) = (\{ba,a\},\varepsilon)$;
  \item
    $(L_4,M_4) \df \Gamma(L_3,M_3) =(\val{\tau_1}(L_3,M_3), \val{\tau_2}(L_3,M_3)) = (\{bba, ba,a\},\varepsilon)$;
  \item
    $(L_5,M_5) \df \Gamma(L_4,M_4) = ( \val{\tau_1}(L_4,M_4), \val{\tau_2}(L_4,M_4)) = (\{bba, bba, ba,a\},\varepsilon)$.
  \end{itemize}
  Лесно се съобразява, че $L_n = \{ b^ka \mid k < n\}$.
  Тогава
  \[\lfp( \Gamma ) = (\bigcup_n L_n, \{\varepsilon\}) = (b^\star a, \{\varepsilon\} ).\]
\end{example}


\begin{problem}
  Докажете, че най-малкото решение на системата 
  \begin{align*}
    & X_1 = a \cdot X_1 + b \cdot X_2 + \varepsilon\\
    & X_2 = b \cdot X_2 + \varepsilon
  \end{align*}
  е двойката $(a^\star b^\star, b^\star)$.
\end{problem}

\begin{problem}
  Да разгледаме системата от оператори
  \begin{align*}
    & \val{\tau_1}(L_1,\dots,L_n) = L_1\\
    & \ \ \vdots\\
    & \val{\tau_n}(L_1,\dots,L_n) = L_n.
  \end{align*}
  Знаем, че тя притежава най-малко решение $(\hat{L}_1,\dots,\hat{L}_n)$.
  Докажете, че всеки от езиците $\hat{L}_i$ е регулярен.

  Докажете, че всеки регулярен език е елемент от най-малкото решение 
  на някоя система от оператори от горния вид.
\end{problem}


% \begin{problem}
%   Докажете, че всеки регулярен език е елемент на най-малкото решение на някоя система от $n$
%   полинома с $n$ променливи за някое $n$.
% \end{problem}

% \begin{problem}
%   Докажете, че всяко най-малко решение на система от $n$ полинома с $n$ променливи представлява $n$-орка от 
%   регулярни езици.
% \end{problem}

% \begin{problem}
%   Опишете алгоритъм, по който може от система от $n$ полинома с $n$ променливи да се построи 
%   краен автомат с $n$ състояния.
% \end{problem}

% \begin{problem}
%   Опишете алгоритъм, по който може от краен автомат с $n$ състояния може да се построи 
% \end{problem}



\subsection{Безконтекстни езици}

Да фиксираме азбуката $\Sigma = \{a_1,\dots,a_n\}$.
Да дефинираме термове от тип 1 като
\[\tau ::= X_i\ |\ a_j\ |\ \varepsilon\ |\ \emptyset\ |\ \tau_1 \cdot \tau_2\ |\ (\tau_1 + \tau_2),\]
където $j = 1, \dots,n$, а $X_i$ са изброимо безкрайна редица от променливи.
За всеки терм $\tau[X_1,\dots,X_n]$ дефинираме оператора 
\[\val{\tau}: (\mathcal{P}(\Sigma^\star))^n \to \mathcal{P}(\Sigma^\star)\]
 по следния начин:
\begin{itemize}
\item 
  $\val{X_i}(L_1,\dots,L_n) = L_i$.
\item 
  $\val{a_j}(L_1,\dots,L_n) = \{a_j\}$.
\item 
  $\val{\varepsilon}(L_1,\dots,L_n) = \varepsilon$.
\item 
  $\val{\emptyset}(L_1,\dots,L_n) = \emptyset$.
\item 
  $\val{\tau_1 \cdot \tau_2}(L_1,\dots,L_n) = \val{\tau_1}(L_1,\dots,L_n) \cdot \val{\tau_2}(L_1,\dots,L_n)$.
\item
  $\val{\tau_1 + \tau_2}(L_1,\dots,L_n) = \val{\tau_1}(L_1,\dots,L_n) \cup \val{\tau_2}(L_1,\dots,L_n)$.
\end{itemize}

\begin{problem}
  Докажете, че за всеки терм $\tau$, $\val{\tau}$ е непрекъснато изображение в областта на Скот
  $\mathcal{S} = ( \mathcal{P}(\Sigma^\star),\subseteq, \emptyset)$.
\end{problem}

\begin{problem}
  Докажете, че $\{a^nb^n \mid n\in \Nat\} = \lfp(\val{\tau})$, където 
  \[\tau[X] \equiv \varepsilon + a \cdot X \cdot b.\]
  С други думи, $\{a^nb^n \mid n \in \Nat\}$ е най-малкото решение на уравнението
  \[X = a \cdot X \cdot b + \varepsilon.\]
\end{problem}

Нека сега да разгледаме термовете $\tau_1[X_1,\dots,X_n], \dots, \tau_n[X_1,\dots,X_n]$.

\begin{problem}
  Да разгледаме системата от оператори
  \begin{align*}
    & \val{\tau_1}(L_1,\dots,L_n) = L_1\\
    & \ \ \vdots\\
    & \val{\tau_n}(L_1,\dots,L_n) = L_n.
  \end{align*}
  Знаем, че тя притежава най-малко решение $(\hat{L}_1,\dots,\hat{L}_n)$.
  Докажете, че всеки от езиците $\hat{L}_i$ е безконтекстен.

  Докажете, че всеки безконтекстен език е елемент от най-малкото решение 
  на някоя система от оператори от горния вид.
\end{problem}

\begin{problem}
  \marginpar{Това е аналог на нормалната форма на Чомски}
  Да дефинираме термове от тип 2 като
  \[\tau ::= a_j\ |\ \varepsilon\ |\ \emptyset\ |\ X_i \cdot X_k\ |\ (\tau_1 + \tau_2),\]
  където $j = 1, \dots,n$, а $X_i$ са изброимо безкрайна редица от променливи.
  Докажете горното твърдение, като замените термовете от тип 1 с тези от тип 2.
\end{problem}

\begin{example}
  Да разгледаме системата
  \begin{align*}
    & X_1 = X_3 \cdot X_2 + \varepsilon\\
    & X_2 = X_1 \cdot X_4\\
    & X_3 = a\\
    & X_4 = b.
  \end{align*}


  % \begin{align*}
  %   & \val{\varepsilon + X_3 \cdot X_2}(L_1, L_2, L_3, L_4) = L_1\\
  %   & \val{X_1 \cdot X_4}(L_1, L_2, L_3, L_4) = L_2\\
  %   & \val{a}(L_1, L_2, L_3, L_4) = L_3\\
  %   & \val{b}(L_1, L_2, L_3, L_4) = L_4\\
  % \end{align*}
  Нека $(\hat{L}_1, \hat{L}_2, \hat{L}_3, \hat{L}_4)$ е най-малкото решение на системата.
  Докажете, че $\hat{L}_1 = \{a^nb^n\mid n \in \Nat\}$ $\hat{L}_2 = \{a^nb^{n+1}\mid n \in \Nat\}$,
  $\hat{L}_3 = \{a\}$ и $\hat{L}_4 = \{b\}$.
\end{example}



%%% Local Variables:
%%% mode: latex
%%% TeX-master: "../sep"
%%% End:

\section{Изоморфни области на Скот}
\index{изоморфизъм}

Нека $\A_1 = (A_1,~\sqsubseteq_1,~\bot_1)$ и $\A_2 = (A_2,~\sqsubseteq_2~,~\bot_2)$ 
са области на Скот.
Ще казваме, че $\A_1$ е {\bf изоморфна} на $\A_2$, което ще означаваме като 
\[\A_1 \cong \A_2,\]
ако съществува {\em биективна} функция $F:A_1 \to A_2$ със свойството:
\[(\forall a \in A_1)(\forall b\in A_1)[\ a \sqsubseteq_1 b \iff F(a) \sqsubseteq_2 F(b)\ ].\]
В такъв случай ще казваме, че $F$ задава изоморфизъм между $\A_1$ и $\A_2$.

Когато искаме да означим, че $\A_1$ е изоморфна на $\A_2$ чрез $F$,
то понякога ще пишем $\A_1 \cong_F \A_2$.

\begin{problem}
  Докажете, че ако $\A_1 \cong_F \A_2$, то $F(\bot_1) = \bot_2$.
\end{problem}


\begin{proposition}
  \label{pr:isomorphism-is-continuous}
  Ако $\A_1 \cong_F \A_2$ , то $F \in \Cont{\A_1}{\A_2}$.
\end{proposition}
\begin{hint}
  Да разгледаме произволна верига $\chain{a}{i}$ от елементи на $\A_1$.
  Ще докажем, че 
  \[F(\bigsqcup_i a_i) = \bigsqcup_iF(a_i).\]
  
  \begin{itemize}
  \item 
    Първо, от дефиницията веднага следва, че $F$ е монотонно изображение, защото
    \[a \sqsubseteq_1 b \implies F(a) \sqsubseteq_2 F(b).\]
    Това означава, че $(F(a_i))^\infty_{i=0}$ е монотонно растяща верига от елементи на $\A_2$.
    От \Prop{monotone-chain} получаваме, че 
    \[\bigsqcup_i F(a_i) \sqsubseteq_2 F(\bigsqcup_i a_i).\]
  \item
    За другата посока, нека $b \in \A_2$ е горна граница на веригата $(F(a_i))^\infty_{i=0}$, т.е. 
    \[(\forall i)[\ F(a_i) \sqsubseteq_2 b\ ].\]
    Ще докажем, че $F(\bigsqcup_i a_i) \sqsubseteq_2 b$.
    Понеже $F$ е {\em върху} $A_2$, то съществува елемент $a \in A_1$, такъв че $F(a) = b$.
    Тогава:
    \begin{align*}
      (\forall i)[\ F(a_i) \sqsubseteq_2 F(a)\ ] & \implies (\forall i)[\ a_i \sqsubseteq_1 a\ ] & \comment{F \text{ е изоморфизъм }}\\
                                                 & \implies \bigsqcup_i a_i \sqsubseteq_1 a & \comment{a\text{ е горна граница}}\\
                                                 & \implies F(\bigsqcup_i a_i) \sqsubseteq_1 F(a). & \comment{F\text{ е изоморфизъм }}
    \end{align*}
    Понеже $b = F(a)$, заключаваме, че
    \[F(\bigsqcup_i a_i) \sqsubseteq_2 b.\]
  \end{itemize}
\end{hint}

\begin{proposition}
  \label{pr:isomorphic-pair}
  Нека $f \in \Mon{\A_1}{\A_2}$ и $g \in \Mon{\A_2}{\A_1}$,
  като 
  \begin{itemize}
  \item 
    $f \circ g = \texttt{id}_2$;
  \item
    $g \circ f = \texttt{id}_1$.
  \end{itemize}
  \marginpar{$\texttt{id}_i(a) \df a$ за вс. $a \in \A_i$}
  Тогава са изпълнени свойствата:
  \begin{enumerate}[(1)]
  \item
    $\A_1 \cong_f \A_2$;
  \item
    $\A_2 \cong_g \A_1$;
  \end{enumerate}
\end{proposition}
% \begin{hint}
%   За Свойство $(1)$ трябва да проверим, че $f$ отговаря на дефиницията за изоморфизъм.
%   \begin{itemize}
%   \item
%     Ще докажем, че $f$ е инективна като покажем, че за произволни $a, b\in A_1$,
%     ако $f(a) = f(b)$, то $a = b$.
%     Но това е лесно, защото
%     \[a = \texttt{id}_1(a) = g(f(a)) = g(f(b)) = \texttt{id}_1(b) = b.\]
%   \item
%     Нека сега $b \in \A_2$.
%     Знаем, че $f(g(b)) = \texttt{id}_2(b) = b$. Това означава, че $f$ е {\em сюрективна},
%     защото за всеки елемент $b \in A_2$ съществува елемент $a \in A_1$, а именно $a = g(b)$,
%     за който $f(a) = b$.
%   \item
%     Понеже $f$ е монотонно изображение, то директно имаме, че
%     \[a \sqsubseteq_1 b \implies f(a) \sqsubseteq_2 f(b).\]
%   \item
%     Нека $f(a) \sqsubseteq_2 f(b)$.
%     Сега пък понеже $g$ е монотонно изображение, 
%     \[a = \texttt{id}_1(a) = g(f(a)) \sqsubseteq_1 g(f(b)) = \texttt{id}_1(b) = b.\]
%     Така показахме, че
%     \[f(a) \sqsubseteq_2 f(b)\ \implies\ a \sqsubseteq_1 b.\]
%   \end{itemize}
%   Доказахме Свойство $(1)$, т.е. $\A_1 \cong_f \A_2$.
%   Разсъжденията за Свойство $(2)$ са аналогични.
% \end{hint}


\begin{proposition}
  \label{pr:isomorphic-higher-order}
  Нека $\A_1 \cong_F \A_2$. Тогава:
  \begin{enumerate}[(1)]
  \item 
    $\Cont{\A_1}{\A_1} \cong_G \Cont{\A_2}{\A_2}$, където 
    \[G(f) \df F \circ f \circ F^{-1};\]
    Графично това може да се изобрази така:

    \shorthandoff{"}%
    \begin{center}
    \begin{tikzcd}[sep=large]
      \A_1 \arrow[r, "f"] & \A_1 \arrow[d, "F"]\\
      \A_2 \arrow[u, "F^{-1}"]\arrow[r, dashed, "G(f)"] & \A_2 
    \end{tikzcd}
    \end{center}
    \shorthandon{"}%
  \item
    ако $f \in \Cont{\A_1}{\A_1}$, то 
    \[F(\lfp(f)) = \lfp(G(f)).\]
  \end{enumerate}
\end{proposition}
\begin{hint}
  Ще докажем $(1)$ като използвме \Prop{isomorphic-pair}.

  \begin{itemize}
  \item 
    Ще докажем, че $G$ е монотонно изображение.
    Нека $f,h \in \Cont{\A_1}{\A_1}$ и $f \sqsubseteq h$, т.е.
    \[(\forall a \in \A_1)[\ f(a) \sqsubseteq_1 h(a)\ ].\]
    Ще докажем, че $G(f) \sqsubseteq G(h)$, т.е.
    \[(\forall b \in \A_2)[\ G(f)(b) \sqsubseteq_1 G(h)(b)\ ].\]
    Да разгледаме произволен елемент $b \in \A_2$. 
    Понеже $F$ е биекция, то съществува елемент $a \in A_1$, такъв че $F(a) = b$,
    т.е. $F^{-1}(b) = a$. Тогава:
    \begin{align*}
      G(f)(b) & \df F(f(F^{-1}(b)))\\
              & = F(f(a)) & \comment{F^{-1}(b) = a}\\
              & \sqsubseteq_2 F(h(a)) & \comment{f(a) \sqsubseteq h(a)\text{ и $F$ е изом.}}\\
              & = F(h(F^{-1}(b))) & \comment{F^{-1}(b) =a}\\
              & \df G(h)(b).
    \end{align*}
  \item
    Нека $G(f) \sqsubseteq G(h)$. Ще докажем, че $f \sqsubseteq h$.
    За целта, нека $a \in A_1$.
    Понеже $F$ е сюрективна, то съществува $b \in A_2$, за който $F^{-1}(b) =a$.
    Понеже
    \[G(f) \df F \circ f \circ F^{-1} \sqsubseteq F \circ h \circ F^{-1} = G(h),\]
    то получаваме, че
    \[F(f(F^{-1}(b))) \sqsubseteq_2 F(h(F^{-1}(b))).\]
    Оттук,
    \[F(f(a)) \sqsubseteq_2 F(h(a)) \implies f(a) \sqsubseteq_1 h(a),\]
    защото $F$ е изоморфизъм.
  \end{itemize}
  Сега преминаваме към доказателството на $(2)$.
  Да напомним, че за $f \in \Cont{\A_1}{\A_1}$, означаваме
  \begin{align*}
    f^0  & = \lambda x. \bot_1\\
    f^{n+1} & = f \circ f^n.
  \end{align*}
  Понеже $f$ е непрекъснато изображение е ясно, че $(f^n(\bot_1))^{\infty}_{n=0}$ е верига.
  Също така знаем, че
  \[\lfp(f) = \bigsqcup_n f^n(\bot_1).\]
  След аналогични разсъждения можем да съобразим, че
  \[\lfp(G(f)) = \bigsqcup_n G(f)^n(\bot_2).\]
  Първо ще докажем с индукция по $n$, че 
  \begin{equation}
    \label{eq:2}
    (\forall n)[\ (G(f))^n = G(f^n)\ ].
  \end{equation}
  \begin{itemize}
  \item 
    За $n = 0$ имаме, че за произволен елемент $b \in \A_2$,
    \begin{align*}
      (G(f))^{0}(b) & \df \bot_2\\
                    & = F(\bot_1) & \comment{F \text{ е изом.}}\\
                    & = F(f^{0}(F^{-1}(b))) & \comment{f^{0}(F^{-1}(b)) \df \bot_1}\\
                    & = (F \circ f^{0} \circ F^{-1})(b) \\
                    & \df G(f^{0})(b).
    \end{align*} 
  \item
    Нека да приемем, че твърдението е вярно за $n$.
    Тогава за $n+1$ имаме, че:
    \begin{align*}
      (G(f))^{n+1} & \df G(f) \circ (G(f))^n\\
                   & = G(f) \circ G(f^n) & \comment{\text{ от И.П.}}\\
                   & \df (F \circ f \circ F^{-1}) \circ (F\circ f^n \circ F^{-1})\\
                   & = F \circ f \circ (F^{-1} \circ F)\circ f^n \circ F^{-1} \\
                   & = F \circ f \circ f^{n} \circ F^{-1} & \comment{F^{-1}\circ F = id}\\
                   & = F \circ f^{n+1} \circ F^{-1} & \comment{f\circ f^n = f^{n+1}}\\
                   & \df G(f^{n+1}).
    \end{align*}
  \end{itemize}
  Тогава:
  \begin{align*}
    F(\lfp(f)) & = F(\bigsqcup_n f^n(\bot_1)) & \comment{\lfp(f) = \bigsqcup_n f^n(\bot_1)}\\
               & = \bigsqcup_n F(f^n(\bot_1))& \comment{F\text{ е непр.}}\\
               & = \bigsqcup_n F(f^n(F^{-1}(\bot_2))) & \comment{F^{-1}(\bot_2) = \bot_1}\\
               & = \bigsqcup_n (F \circ f^n \circ F^{-1})(\bot_2) \\
               & \df \bigsqcup_n G(f^n)(\bot_2)\\
               & = \bigsqcup_n G(f)^n(\bot_2) & \comment{\text{от }(\ref{eq:2})}\\
               & = \lfp(G(f)).
  \end{align*}
\end{hint}

\begin{framed}
  \begin{proposition}
    За произволни области на Скот $\A$, $\B$ и $\C$ е изпълнено, че
    \[\Cont{\A}{\Cont{\B}{\C}}\ \cong\ \Cont{\A\times\B}{\C}.\]
  \end{proposition}  
\end{framed}
\begin{hint}
  % \begin{itemize}
  % \item 
    Докажете, че изображението
    \[\texttt{curry}:\Cont{\A\times \B}{\C} \to \Cont{\A}{\Cont{\B}{\C}},\]
    където
    \[\texttt{curry}(f)(a)(b) \df f(a,b)\]
    задава изоморфизъм.
  % \item
  %   Докажете, че изображението
  %   \[\texttt{uncurry}:\Cont{\A}{\Cont{\B}{\C}} \to \Cont{\A\times \B}{\C},\]
  %   където
  %   \[\texttt{uncurry}(f)(a,b) \df f(a)(b)\]
  %   е монотонно.
  % \item
  %   Лесно се съобразява, че
  %   \[\texttt{curry} \circ \texttt{uncurry} = \texttt{id}\]
  %   \[\texttt{uncurry} \circ \texttt{curry} = \texttt{id}.\]    
  % \item
  %   Приложете \Prop{isomorphic-pair}.
  % \end{itemize}
\end{hint}

Когато на хаскел пишем типовата сигнатура на някоя функция като 
\mint{haskell}|f :: a -> b -> c| в действителност се има предвид \mint{haskell}|f :: a -> (b -> c)|

На практика тези две задачи ни казват, че няма значение дали използваме {\em curried}
или {\em uncurried} версията на една функция. На \texttt{хаскел} е по-удобно да използваме {\em curried}
версията, защото като фиксираме първия аргумент на една функция получаваме нова функция наготово.
\marginpar{Това се нарича \emph{partial application}. Вижте \url{https://wiki.haskell.org/Partial_application}.}
Например, 
\begin{haskellcode}
  ghci> let plus x y = x + y
  ghci> :t plus
  plus :: Num a => a -> a -> a
  ghci> let plus1 = plus 1
  ghci> :t plus1
  plus1 :: Num a => a -> a
\end{haskellcode}

Нека да дефинираме
\[\emptyset_\bot = (\{\bot\}, \sqsubseteq, \bot).\]
\begin{problem}
  Докажете, че за произволна област на Скот $\A$ е изпълнено:
  \begin{align*}
    & \Cont{\emptyset_\bot}{\A} \cong \A\\
    & \Cont{\A}{\emptyset_\bot} \cong \emptyset_\bot.
  \end{align*}
\end{problem}

\begin{problem}
  Докажете, че съществуват области на Скот $\A$, $\B$ и $\C$, за които
  \[\Cont{\Cont{\A}{\B}}{\C} \not\cong \Cont{\A}{\Cont{\B}{\C}}.\]  
\end{problem}
\begin{hint}
  Нека изберем $\A = \C = \Nat_\bot$, а $\B = \emptyset_\bot$. Тогава
  \[\Cont{\Cont{\Nat_\bot}{\emptyset_\bot}}{\Nat_\bot} \cong \Cont{\emptyset_\bot}{\Nat_\bot} \cong \Nat_\bot,\]
  но лесно можем да съобразим, че:
  \[\Cont{\Nat_\bot}{\Cont{\emptyset_\bot}{\Nat_\bot}} \cong \Cont{\Nat_\bot}{\Nat_\bot}.\]
  Сега остава да съобразим, че
  \[\Nat_\bot \not\cong \Cont{\Nat_\bot}{\Nat_\bot}.\]
\end{hint}

%%% Local Variables:
%%% mode: latex
%%% TeX-master: "../sep"
%%% End:

\newpage
% \section{Приложение. Регулярни езици}\index{език!регулярен}

Да фиксираме азбуката $\Sigma = \{a_1,\dots,a_k\}$.
\marginpar{Обърнете внимание, че нямаме звездата на Клини в дефиницията на термовете.}
Следната абстрактна граматика 
\[\tau ::= \varepsilon\ |\ a \cdot X\ |\ \tau_1 + \tau_2.\]
описва термовете на езика $REG$, където $a$ означава буква от азбуката $\Sigma$, а $X$ означава променлива. Нека имаме и безкраен набор от променливи $X_0,X_1,\dots$.
\marginpar{Ще използваме малките гръцки букви $\tau,\rho,\mu$, евентуално с индекси, за да означаваме термовете. Думите над азбуката $\Sigma$ ще означаваме с малките гръцки букви $\alpha$, $\beta$, $\gamma$.}
Програмите на езика езика $REG$ имат следния вид:
\begin{align*}
  & X_1 = \tau_1[X_1,\dots,X_n]\\
  & \ \vdots\\
  & X_n = \tau_n[X_1,\dots,X_n].
\end{align*}
Ето един пример за програма:
\begin{align*}
  & X_1 = a \cdot X_1 + b \cdot X_2\\
  & X_2 = \varepsilon + b \cdot X_1.
\end{align*}


\subsection{Операционна семантика}

С индукция по $\ell$, дефинираме релацията $\tau \Downarrow^\ell L$,
където $\tau$ е терм на езика $REG$ със свободни променливи измежду $X_1,\dots, X_n$.

\marginpar{
  \begin{itemize}
  \item 
    Тук е важно да се отбележи, че тази дефиниция не можем да я дадем с индукция по построението на термовете.
  \item
    Тук няма как да дефинираме релация $\tau \Downarrow L$, където $L$ е език, защото все пак искаме всяко изчисление да трае краен брой стъпки, а това няма как да стане, когато $L$ е безкраен език.
  \end{itemize}
}

\begin{prooftree}
  \AxiomC{}
  \LeftLabel{\scriptsize{(eps)}}
  \UnaryInfC{$\varepsilon \Downarrow^0 \varepsilon$}
\end{prooftree}

\begin{prooftree}
  \AxiomC{$\tau_i \Downarrow^\ell \beta$}
  \AxiomC{$\alpha = a\cdot \beta$}
  \RightLabel{\scriptsize{(concat)}}
  \BinaryInfC{$a \cdot X_i \Downarrow^{\ell+1} \alpha$}
\end{prooftree}

\begin{prooftree}
  \AxiomC{$\tau_1 \Downarrow^\ell \alpha$}
  \LeftLabel{\scriptsize{(left-or)}}
  \UnaryInfC{$\tau_1 + \tau_2 \Downarrow^{\ell+1} \alpha$}
\end{prooftree}

\begin{prooftree}
  \AxiomC{$\tau_2 \Downarrow^\ell \alpha$}
  \RightLabel{\scriptsize{(right-or)}}
  \UnaryInfC{$\tau_1 + \tau_2 \Downarrow^{\ell+1} \alpha$}
\end{prooftree}

Ще пишем $\tau \Downarrow \alpha$ точно тогава, когато съществува $\ell$, за което $\tau \Downarrow^\ell \alpha$.

\begin{example}
  \marginpar{На всяка променлива от програмата $R$ съответства състояние на автомата.}
  Нека $\Sigma = \{a,b\}$. Да разгледаме програмата $R$, където:
  \begin{align*}
    & X_1 = \overbrace{a \cdot X_1 + b \cdot X_2 + \varepsilon}^{\tau_1[X_1,X_2]}\\
    & X_2 = \underbrace{b \cdot X_2 + \varepsilon}_{\tau_2[X_1,X_2]}.
  \end{align*}
  Съобразете, че за всеки две естествени числа $n$ и $k$ е изпълнено, че $\tau_1~\Downarrow~a^k b^n$ и $\tau_2~\Downarrow~b^n$.
  \begin{figure}[H]
      \centering
      \begin{tikzpicture}[->,>=stealth,thick,node distance=65pt]
        \tikzstyle{every state}=[circle,minimum size=20pt,auto]
        
        \node[initial, state, accepting]   (0) {$X_1$};
        \node[state, accepting]            (1) [right of=0]{$X_2$};
        
        \path
        (0) edge [loop above]   node [above] {$a$}    (0)
        (0) edge [bend left=15] node [above] {$b$}    (1)
        (1) edge [loop above]   node [above] {$b$}    (1);
        % (1) edge [bend left=15] node [above] {$a$}    (2)
        % (2) edge [bend left=45] node [below] {$a$}    (0)
        % (2) edge [bend left=15] node [above] {$b$}    (3)
        % (3) edge [loop above]   node [above] {$a,b$}  (3);
      \end{tikzpicture}
      \caption{Краен автомат съотвестващ на програмата $R$.}
    \end{figure}
\end{example}

\subsection{Денотационна семантика}

\begin{problem}
  Докажете, че $\mathcal{S} = (\mathcal{P}(\Sigma^\star), \subseteq, \emptyset)$ е област на Скот.
\end{problem}

За всеки терм $\tau[X_1,\dots,X_n]$ дефинираме изображението
% \marginpar{Тук за променливите използваме главни букви за да се сещаме, че те приемат стойности множества от думи.}
\[\val{\tau}: \mathcal{P}(\Sigma^\star)^n \to \mathcal{P}(\Sigma^\star)\]
 по следния начин:
\begin{itemize}
\item 
  $\val{\varepsilon}(L_1,\dots,L_n) = \{\varepsilon\}$.
\item 
  $\val{a_i \cdot X_j}(L_1,\dots,L_n) = \{a_i\} \cdot L_j$.
\item
  $\val{\tau_1 + \tau_2}(L_1,\dots,L_n) = \val{\tau_1}(L_1,\dots,L_n) \cup \val{\tau_2}(L_1,\dots,L_n)$.
\end{itemize}

\begin{problem}
  Докажете, че за всеки терм $\tau$ имаме, че $\val{\tau}$ е непрекъснато изображение в областта на Скот
  $\mathcal{S} = ( \mathcal{P}(\Sigma^\star),\subseteq, \emptyset)$.
\end{problem}

\marginpar{Програмата е просто текст. Системата е редица от уравнения на непрекъснати изображения.}

\begin{example}
  Да разгледаме програмата $R$, където:
  \begin{align*}
    & X_1 = \overbrace{b \cdot X_1 + a\cdot X_2}^{\tau_1[X_1,X_2]}\\
    & X_2 = \underbrace{\varepsilon}_{\tau_2[X_1,X_2]}.
  \end{align*}
  Според горната дефиниция на семантика на термове, на програмата $R$ съотвества следната система от непрекъснати изображения:
  \begin{align*}
    & L = \overbrace{\{b\} \cdot L \cup \{a\} \cdot M}^{\val{\tau_1}(L,M)}\\
    & M = \underbrace{\{\varepsilon\}}_{\val{\tau_2}(L,M)}.
  \end{align*}
  От \Th{sep:min-solution-system} знаем, че тази система има най-малко решение. Нека да го намерим.
  За тази цел, да дефинираме непрекъснатото изображение $\Gamma:\mathcal{P}(\Sigma^\star)^2 \to \mathcal{P}(\Sigma^\star)^2$ като
  \marginpar{От \ref{} знаем, че $\Gamma$ е непрекъснато изображение.}
  \[\Gamma \df \val{\tau_1} \times \val{\tau_2},\]
  т.е. за произволни езици $L$ и $M$ над азбуката $\Sigma$ е изпълнено, че:
  \[\Gamma(L,M) = (\val{\tau_1}(L,M), \val{\tau_2}(L,M)).\]

  От \hyperref[th:knaster-tarski]{Теоремата на Клини} знаем как можем да намерим най-малката неподвижна точка на $\Gamma$,
  която ще бъде и най-малкото решение на горната система.

  \begin{itemize}
  \item 
    $(L_0,M_0) \df (\emptyset,\emptyset)$;
  \item
    $(L_1,M_1) \df \Gamma(L_0,M_0) = (\val{\tau_1}(L_0,M_0), \val{\tau_2}(L_0,M_0)) = (\emptyset, \{\varepsilon\})$;
  \item
    $(L_2,M_2) \df \Gamma(L_1,M_1) = (\val{\tau_1}(L_1,M_1), \val{\tau_2}(L_1,M_1)) = (\{a\},\{\varepsilon\})$;
  \item
    $(L_3,M_3) \df \Gamma(L_2,M_2) = (\val{\tau_1}(L_2,M_2), \val{\tau_2}(L_2,M_2)) = (\{ba,a\},\{\varepsilon\})$;
  \item
    $(L_4,M_4) \df \Gamma(L_3,M_3) =(\val{\tau_1}(L_3,M_3), \val{\tau_2}(L_3,M_3)) = (\{bba, ba,a\},\{\varepsilon\})$;
  \item
    $(L_5,M_5) \df \Gamma(L_4,M_4) = ( \val{\tau_1}(L_4,M_4), \val{\tau_2}(L_4,M_4)) = (\{bba, bba, ba,a\},\{\varepsilon\})$.
  \end{itemize}
  Лесно се съобразява, че $L_n = \{ b^ka \mid k < n\}$, а $M_n = \{\varepsilon\}$. Тогава
  \[\lfp( \Gamma ) = (\bigcup_n L_n, \bigcup_n M_n) = (\{b\}^\star \cdot \{a\}, \{\varepsilon\} ).\]
\end{example}

\begin{example}
  Да разгледаме следния краен детерминиран автомат:
  \marginpar{Означаваме с $\abs{\omega}_a$ броя на срещанията на $a$ в думата $\omega$.}
  \begin{figure}[H]
    \centering
    \begin{tikzpicture}[->,>=stealth,thick,node distance=65pt]
      \tikzstyle{every state}=[circle,minimum size=20pt,auto]
      
      \node[initial, state, accepting]         (0) {$q_1$};
      \node[state]                             (1) [right of=0]{$q_2$};
      \node[state]                             (2) [right of=1]{$q_3$};
      
      \path 
      (0) edge [loop above]   node   [above] {$b$}    (0)
      (0) edge [bend left=15] node   [above] {$a$}    (1)
      (1) edge [loop above]   node   [above] {$b$}    (1)
      (1) edge [bend left=15] node   [above] {$a$}    (2)
      (2) edge [loop above]   node   [above] {$b$}    (2)
      (2) edge [bend left=45] node   [below] {$a$}    (0);
    \end{tikzpicture}
    \caption{Автомат разпознаващ $\{\omega \in \{a,b\}^\star \mid \abs{\omega}_{a} \equiv 0 \bmod\ 3\}$}
  \end{figure}
  На този автомат съответства следната програма $R$:
  \begin{align*}
    & X_1 = a \cdot X_2 + b \cdot X_1 + \varepsilon \\
    & X_2 = a \cdot X_3 + b \cdot X_2\\
    & X_3 = a \cdot X_1 + b \cdot X_3.
  \end{align*}
  На тази програма съответства системата:
  \begin{align*}
    & L_1 = \{a\} \cdot L_2 \cup \{b\} \cdot L_1 \cup \{\varepsilon\}\\
    & L_2 = \{a\} \cdot L_3 \cup \{b\} \cdot L_2\\
    & L_3 = \{a\} \cdot L_1 \cup \{b\} \cdot L_3.
  \end{align*}
  Най-малкото решение на тази система е тройката $(\hat{L}_1,\hat{L}_2,\hat{L}_3)$ от следните езици:
  \begin{align*}
    \hat{L}_1 = & \{\omega \in \{a,b\}^\star \mid \abs{\omega}_{a} \equiv 0 \bmod\ 3\}\\
    \hat{L}_2 = & \{\omega \in \{a,b\}^\star \mid \abs{\omega}_{a} \equiv 2 \bmod\ 3\}\\
    \hat{L}_3 = & \{\omega \in \{a,b\}^\star \mid \abs{\omega}_{a} \equiv 1 \bmod\ 3\}.
  \end{align*}
\end{example}

\begin{example}
  Да разгледаме системата от само едно уравнение
  \[L = \{a\} \cdot L.\]
  Най-малкото решение на тази система е езикът $\emptyset$,
  но тази система има за решение и езикът $\{a\}^\star$.
\end{example}

  
\begin{problem}
  Докажете, че най-малкото решение на системата 
  \begin{align*}
    & X_1 = a \cdot X_1 + b \cdot X_2 + \varepsilon\\
    & X_2 = b \cdot X_2 + \varepsilon
  \end{align*}
  е двойката $(\{a\}^\star \cdot \{b\}^\star, \{b\}^\star)$.
  
\end{problem}

\begin{problem}
  Да разгледаме системата от непрекъснати оператори
  \begin{align*}
    & \val{\tau_1}(L_1,\dots,L_n) = L_1\\
    & \ \ \vdots\\
    & \val{\tau_n}(L_1,\dots,L_n) = L_n.
  \end{align*}
  Знаем, че тя притежава най-малко решение $(\hat{L}_1,\dots,\hat{L}_n)$.
  Докажете, че всеки от езиците $\hat{L}_i$ е регулярен.

  Докажете, че всеки регулярен език е елемент от най-малкото решение 
  на някоя система от непрекъснати изображения от горния вид.
\end{problem}

\subsection{Еквивалентност на денотационната и операционната семантика}

Нека тук да фиксираме една програма $R$
\begin{align*}
  & X_1 = \tau_1[X_1,\dots,X_n]\\
    & \vdots\\
  & X_n = \tau_n[X_1,\dots,X_n].
\end{align*}
Нека $(L_1,\dots,L_n)$ е най-малкото решение на непрекъснатото изображение $\Gamma \df \val{\tau_1}\times \cdots \times \val{\tau_n}$. Целта ни е да докажем, че за всяко $i = 1,\dots,n$ е изпълнено, че:
\[L_i = \{\alpha \in \Sigma^\star \mid \tau_i \Downarrow \alpha\}.\]
Да напомним, че $L_i = \bigcup_k L^k_i$, където $L^0_i = \emptyset$ и $L^{k+1}_i = \val{\tau_i}(L^k_1,\dots,L^k_n)$.
Този резултат ще наречем теорема за еквивалентност.
Това ще направим на две стъпки.

\begin{lemma}
  \marginpar{Тук раборим при фиксирана програма $R$.}
  За всеки индекс $r$ и всеки терм $\tau$ с променливи измежду $X_1,\dots,X_n$ е изпълнено, че:
  \[\val{\tau}(L^r_1,\dots,L^r_n) \subseteq \{\alpha \in \Sigma^\star \mid \tau \Downarrow \alpha\}.\]
\end{lemma}
\begin{proof}
  Нека да кръстим $\textbf{Include}(r)$ твърдението
  ,,за всеки терм $\tau$ с променливи измежду $X_1,\dots,X_n$ е изпълнено, че:
  \[\val{\tau}(L^r_1,\dots,L^r_n) \subseteq \{\alpha \in \Sigma^\star \mid \tau \Downarrow \alpha\}".\]
  Целта ни е да докажем, че $(\forall r\in\Nat)\textbf{Include}(r)$. Това ще направим с индукция по $r$.
  
  Нека $r = 0$. Знаем, че $L^0_i = \emptyset$ за всеки индекс $i$ измежду $1,\dots,n$. Тук ще направим вътрешна индукция по построението на термовете $\tau$ за да докажем, че:
  \begin{equation}
    \label{eq:17}
    \val{\tau}(L^0_1,\dots,L^0_n) \subseteq \{\alpha \in \Sigma^\star \mid \tau \Downarrow \alpha\}
  \end{equation}
  \begin{itemize}
  \item
    \marginpar{Тук е важно, че дефиницията на денотационната семантика следва индуктивното построение на термовете.}
    Нека $\tau \equiv \emptyset$. Понеже $\val{\tau}(L^0_1,\dots,L^0_n) = \emptyset$, то е ясно, че Свойство~\ref{eq:17} е изпълнено.
  \item
    Нека $\tau \equiv \varepsilon$. Тогава $\val{\tau}(L^0_1,\dots,L^0_n) = \{\varepsilon\}$.
    От правилата на операционната семантика имаме, че $\varepsilon \Downarrow \varepsilon$.
    Заключаваме, че Свойство~\ref{eq:17} е изпълнено.
  \item
    Нека $\tau \equiv a \cdot X_j$, за някой индекс $j$ измежду $1,\dots,n$. Тогава
    \[\val{\tau}(L^0_1,\dots,L^0_n) = \{a\}\cdot L^0_j = \{a\} \cdot \emptyset = \emptyset\]
    и оттук е очевидно, че Свойство~\ref{eq:17} е в сила.
  \item
    Нека $\tau \equiv \rho + \mu$. Тогава $\val{\tau}(L^0_1,\dots,L^0_n) = \val{\rho}(L^0_1,\dots,L^0_n) \cup \val{\mu}(L^0_1,\dots,L^0_n)$. Сега от И.П. за Свойство~\ref{eq:17} получаваме, че
    \begin{align*}
      & \val{\rho}(L^0_1,\dots,L^0_n) \subseteq \{\alpha \in \Sigma^\star \mid \rho \Downarrow \alpha\}\\
      & \val{\mu}(L^0_1,\dots,L^0_n) \subseteq \{\alpha \in \Sigma^\star \mid \mu \Downarrow \alpha\}.
    \end{align*}
    От правилата на операционната семантика имаме, че
    \begin{figure}[H]
      \begin{subfigure}[b]{0.5\textwidth}
        \begin{prooftree}
          \AxiomC{$\rho \Downarrow \alpha$}
          \LeftLabel{\scriptsize{(left-or)}}
          \UnaryInfC{$\rho + \mu \Downarrow \alpha$}
        \end{prooftree}
        \vspace*{2mm}
      \end{subfigure}
      ~
      \begin{subfigure}[b]{0.5\textwidth}
        \begin{prooftree}
          \AxiomC{$\mu \Downarrow \alpha$}
          \RightLabel{\scriptsize{(right-or)}}
          \UnaryInfC{$\rho + \mu \Downarrow \alpha$}
        \end{prooftree}
        \vspace*{2mm}
      \end{subfigure}
    \end{figure}
    Оттук веднага получаваме, че:
    \[\{\alpha \in \Sigma^\star \mid \rho \Downarrow \alpha\} \cup \{\alpha \in \Sigma^\star \mid \mu \Downarrow \alpha\} = \{\alpha \in \Sigma^\star \mid \tau \Downarrow \alpha\}.\]
    Заключаваме, че:
    \begin{align*}
      \val{\tau}(L^0_1,\dots,L^0_n) & = \val{\rho}(L^0_1,\dots,L^0_n) \cup \val{\mu}(L^0_1,\dots,L^0_n)\\
                                    & \subseteq \{\alpha \in \Sigma^\star \mid \rho \Downarrow \alpha\} \cup \{\alpha \in \Sigma^\star \mid \mu \Downarrow \alpha\}\\
                                    & = \{\alpha \in \Sigma^\star \mid \tau \Downarrow \alpha\}.
    \end{align*}
    Свойство~\ref{eq:17} е изпълнено за всеки терм $\tau$.
  \end{itemize}

  Нека $r > 0$ и като индукционно предположение да приемем, че е изпълнено $\textbf{Include}(r-1)$. Ще докажем, че е изпълнено $\textbf{Include}(r)$.
  \marginpar{Тук знаем, че \[L^r_i = \val{\tau_i}(L^{r-1}_1,\dots,L^{r-1}_n).\]}
  Ще направим вътрешна индукция по построението на термовете $\tau$ за да докажем, че:
  \begin{equation}
    \label{eq:18}
    \val{\tau}(L^r_1,\dots,L^r_n) \subseteq \{\alpha \in \Sigma^\star \mid \tau \Downarrow \alpha\}.
  \end{equation}
  \begin{itemize}
  \item
    Нека $\tau \equiv \emptyset$. Отново, понеже $\val{\tau}(L^0_1,\dots,L^0_n) = \emptyset$, то е ясно, че Свойство~\ref{eq:18} е изпълнено.
  \item
    Нека $\tau \equiv \varepsilon$. Отново, понеже $\val{\tau}(L^0_1,\dots,L^0_n) = \{\varepsilon\}$,
    то от правилата на операционната семантика имаме, че $\varepsilon \Downarrow \varepsilon$.
    Заключаваме, че Свойство~\ref{eq:18} е изпълнено.
  \item
    Нека $\tau \equiv a \cdot X_j$ за някой индекс $j$ измежду $1,\dots,n$. Тогава
    \[\val{\tau}(L^r_1,\dots,L^r_n) = \{a\}\cdot L^r_j.\]
    Понеже сме приели, че $\textbf{Include}(r-1)$ е изпълено, то имаме автоматично, че:
    \[L^r_j = \val{\tau_j}(L^{r-1}_1,\dots,L^{r-1}_n) \subseteq \{\beta \in \Sigma^\star \mid \tau_j \Downarrow \beta\}.\]
    От правилата на операционната семантика имаме, че:
    \begin{prooftree}
      \AxiomC{$\tau_j \Downarrow \beta$}
      \UnaryInfC{$a \cdot X_j \Downarrow a \cdot \beta$}
    \end{prooftree}
    Понеже $\tau \equiv a \cdot X_j$, получаваме следното:
    \begin{equation}
      \label{eq:21}
      \{a\} \cdot \{\beta \in \Sigma^\star \mid \tau_j \Downarrow \beta\} = \{ \alpha \in \Sigma^\star \mid \tau \Downarrow \alpha\}.
    \end{equation}
    Заключаваме, че:
    \begin{align*}
      \val{\tau}(L^r_1,\dots,L^r_n) & = \{a\} \cdot L^r_j \\
                                    & = \{a\} \cdot \val{\tau_j}(L^{r-1}_1,\dots,L^{r-1}_n)\\
                                    & \subseteq \{a\} \cdot \{\beta \in \Sigma^\star \mid \tau_j \Downarrow \beta\} & \comment\text{от }\textbf{Include}(r-1)\\
      & = \{ \alpha \in \Sigma^\star \mid \tau \Downarrow \alpha\}. & \comment\text{от \ref{eq:21}}
    \end{align*}
  \item
    Нека $\tau \equiv \rho + \mu$. Тук имаме, че
    \[\val{\tau}(L^r_1,\dots,L^r_n) = \val{\rho}(L^r_1,\dots,L^r_n) \cup \val{\mu}(L^r_1,\dots,L^r_n).\]
    Понеже $\tau$ е построен с помощта на $\rho$ и $\mu$, можем да използваме вътрешното индукционно предположение за $\rho$ и $\mu$, откъдето имаме, че:
    \begin{align*}
      & \val{\rho}(L^r_1,\dots,L^r_n) \subseteq \{\alpha \in \Sigma^\star \mid \rho \Downarrow \alpha\}\\
      & \val{\mu}(L^r_1,\dots,L^r_n) \subseteq \{\alpha \in \Sigma^\star \mid \mu \Downarrow \alpha\}.
    \end{align*}
    От правилата на операционната семантика имаме, че
    \[\{\alpha \in \Sigma^\star \mid \rho \Downarrow \alpha\} \cup \{\alpha \in \Sigma^\star \mid \mu \Downarrow \alpha\} = \{\alpha \in \Sigma^\star \mid \rho + \mu \Downarrow \alpha\}.\]
    Заключаваме, че:
    \begin{align*}
      \val{\tau}(L^r_1,\dots,L^r_n) & = \val{\rho}(L^r_1,\dots,L^r_n) \cup \val{\mu}(L^r_1,\dots,L^r_n)\\
                                    & \subseteq \{\alpha \in \Sigma^\star \mid \rho \Downarrow \alpha\} \cup \{\alpha \in \Sigma^\star \mid \mu \Downarrow \alpha\} \\
                                    & = \{\alpha \in \Sigma^\star \mid \mu+\rho \Downarrow \alpha\}\\
      & = \{\alpha \in \Sigma^\star \mid \tau \Downarrow \alpha\}.
    \end{align*}
    Свойство~\ref{eq:18} е изпълнено за всеки терм $\tau$.
  \end{itemize}
\end{proof}

\begin{corollary}
  Нека $L_1,\dots,L_n$ е най-малкото решение на изображението $\Gamma \df \val{\tau_1}\times\cdots\times\val{\tau_n}$. Тогава за произволен терм $\tau$ с променливи измежду $X_1,\dots,X_n$ е изпълнено, че:
  \[\val{\tau}(L_1,\dots,L_n) \subseteq \{\alpha \in \Sigma^\star \mid \tau \Downarrow \alpha\}.\]
\end{corollary}
\begin{proof}
  Използваме, че $\val{\tau}$ е непрекъснато изображение. Тогава
  \begin{align*}
    \val{\tau}(L_1,\dots,L_n) & = \val{\tau}(\bigcup_r L^r_1,\dots,\bigcup_rL^r_n)\\
                              & = \bigcup_r\val{\tau}(L^r_1,\dots,L^r_n)\\
                              & \subseteq \{\alpha \in \Sigma^\star \mid \tau \Downarrow \alpha\}.
  \end{align*}
\end{proof}

\begin{lemma}
  Нека $L_1,\dots,L_n$ е неподвижна точка на $\Gamma \df \val{\tau_1}\times \cdots \times \val{\tau_n}$.
  За всеки терм $\tau$ с променливи измежду $X_1,\dots,X_n$ е  изпълнено, че:
  \[\{\alpha \in \Sigma^\star \mid \tau \Downarrow \alpha\} \subseteq \val{\tau}(L_1,\dots,L_n).\]
\end{lemma}
\begin{proof}
  Да означим с $\textbf{Include}(\ell)$ твърдението ,,за всеки терм $\tau$ с променливи измежду $X_1,\dots,X_n$ е изпълнено, че:
  \[\{\alpha \in \Sigma^\star \mid \tau \Downarrow^\ell \alpha\} \subseteq \val{\tau}(L_1,\dots,L_n)".\]
  Ще докажем, че $(\forall \ell\in\Nat)\textbf{Include}(\ell)$.

  Нека $\ell = 0$. Според правилата на операционната семантика, единственият случай, който имаме тук е $\tau \equiv \varepsilon$.

  Нека $\ell > 0$ и да разгледаме дума $\alpha$ и терм $\tau$, за които $\tau \Downarrow^\ell \alpha$. Тогава имаме три случая в зависимост от това какво е последното правило, което сме приложили.
  \begin{itemize}
  \item
    Първият случай е когато имаме, че $\tau \equiv a \cdot X_j$ и
    \begin{prooftree}
      \AxiomC{$\tau_j \Downarrow^{\ell-1} \beta$}
      \UnaryInfC{$a \cdot X_j \Downarrow^\ell a \cdot \beta$}
    \end{prooftree}
    Според правилата на операционната семантика имаме, че:
    \begin{equation}
      \label{eq:22}
      \{a\} \cdot \{\beta \in \Sigma^\star \mid \tau_j \Downarrow^{\ell-1} \beta\} = \{\alpha \in \Sigma^\star \mid \tau \Downarrow^{\ell} \alpha\}
    \end{equation}
    Знаем, че $\val{\tau}(L_1,\dots,L_n) = \{a\} \cdot L_j = \{a\} \cdot \val{\tau_j}(L_1,\dots,L_n)$,
    защото $L_1,\dots,L_n$ е неподвижна точка на $\Gamma$.
    От $\textbf{Include}(\ell-1)$ имаме следното:
    \[\{\beta \in \Sigma^\star \mid \tau_j \Downarrow^{\ell-1} \beta\} \subseteq \val{\tau_j}(L_1,\dots,L_n).\]
    Заключаваме, че:
    \begin{align*}
      \{\alpha \in \Sigma^\star \mid \tau \Downarrow^\ell \alpha\} & = \{a\} \cdot \{\beta \in \Sigma^\star \mid \tau_j \Downarrow^{\ell-1} \beta\} & \comment\text{от \ref{eq:22}}\\
                                                                   & \subseteq \{a\} \cdot \val{\tau_j}(L_1,\dots,L_n) & \comment\text{от }\textbf{Include}(\ell-1)\\
                                                                   & = \{a\} \cdot L_j & \comment L_1,\dots,L_n\text{ е неподв. точка}\\
      & = \val{\tau}(L_1,\dots,L_n) & \comment\text{деф. на денот. сем.}
    \end{align*}
  \item
    Сега нека $\tau \equiv \rho + \mu$ и последното правило, което сме приложили е следното:
    \begin{prooftree}
      \AxiomC{$\rho \Downarrow^{\ell-1} \alpha$}
      \RightLabel{\scriptsize{(left-or)}}
      \UnaryInfC{$\rho + \mu \Downarrow^\ell \alpha$}
    \end{prooftree}

    Знаем, че $\val{\tau}(L_1,\dots,L_n) = \val{\rho}(L_1,\dots,L_n) \cup \val{\mu}(L_1,\dots,L_n)$.
    Понеже сме приели, че $\textbf{Include}(\ell-1)$ е изпълнено, получаваме:
    \begin{align*}
      \{\alpha \in \Sigma^\star \mid \rho \Downarrow^{\ell-1} \alpha\} & \subseteq \val{\rho}(L_1,\dots,L_n)\\
                                                                       & \subseteq \val{\tau}(L_1,\dots,L_n).
    \end{align*}
    
  \item
    Сега нека $\tau \equiv \rho + \mu$ и последното правило, което сме приложили е следното:
    \begin{prooftree}
      \AxiomC{$\mu \Downarrow^{\ell-1} \alpha$}
      \RightLabel{\scriptsize{(right-or)}}
      \UnaryInfC{$\rho + \mu \Downarrow^\ell \alpha$}
    \end{prooftree}
    Този случай е аналогичен на предния.
  \end{itemize}
\end{proof}

\begin{framed}
  \begin{theorem}[Теорема за еквивалентност]
    Нека $L_1,\dots,L_n$ е най-малката неподвижна точка на $\Gamma$. Тогава за всеки индекс $i$ измежду $1,\dots,n$ е изпълнено, че:
    \[L_i = \{\alpha \in \Sigma^\star \mid \tau_i \Downarrow \alpha\}.\]  
  \end{theorem}
\end{framed}


\subsubsection{Как ще представим празния език?}


Да разгледаме програмата
\[X = \underbrace{a \cdot X}_{\tau[X]}.\]

Тогава лесно се съобразява, че $\emptyset = \{\alpha \in \Sigma^\star \mid \tau \Downarrow \alpha\}$.
Освен това, най-малката неподвижна точка на $\val{\tau}$ e $\emptyset$.
%%% Local Variables:
%%% mode: latex
%%% TeX-master: "../sep"
%%% End:

% \subsection{Регулярни езици}\index{език!регулярен}

Да фиксираме азбуката $\Sigma = \{a_1,\dots,a_k\}$.
\marginpar{Обърнете внимание, че нямаме звездата на Клини в дефиницията на термовете.}
Следната абстрактна граматика 
\[\tau ::= \emptyset\ |\ \varepsilon\ |\ a \cdot X\ |\ \tau + \tau.\]
описват термовете на езика $REG$, където $a$ означава буква от азбуката $\Sigma$, а $X$ означава променлива. Нека имаме и безкраен набор от променливи $X_0,X_1,\dots$.

Програма на езика езика $REG$ представляват системи от следния вид:
\begin{align*}
  & X_1 = \tau_1[X_1,\dots,X_n]\\
  & \ \vdots\\
  & X_n = \tau_n[X_1,\dots,X_n].
\end{align*}

\begin{example}
  Нека $\Sigma = \{a,b\}$. Да разгледаме програмата $R$:
  \begin{align*}
    & X_1 = a \cdot X_1 + b \cdot X_2\\
    & X_2 = \varepsilon + b \cdot X_1.
  \end{align*}
\end{example}


\subsubsection{Операционна семантика}

С индукция по $\ell$, дефинираме релацията $\tau \Downarrow^\ell \alpha$,
където $\tau$ е терм на езика $REG$ със свободни променливи измежду $X_1,\dots, X_n$.

\marginpar{Тук е важно да се отбележи, че тази дефиниция не можем да я дадем с индукция по построението на термовете.}

\begin{prooftree}
  \AxiomC{}
  \LeftLabel{\scriptsize{(eps)}}
  \UnaryInfC{$\varepsilon \Downarrow^0 \varepsilon$}
\end{prooftree}

\begin{prooftree}
  \AxiomC{$\tau_i \Downarrow^\ell \beta$}
  \AxiomC{$\alpha = a\cdot \beta$}
  \RightLabel{\scriptsize{(concat)}}
  \BinaryInfC{$a \cdot X_i \Downarrow^{\ell+1} \alpha$}
\end{prooftree}

\begin{prooftree}
  \AxiomC{$\tau_1 \Downarrow^\ell \alpha$}
  \LeftLabel{\scriptsize{(left-or)}}
  \UnaryInfC{$\tau_1 + \tau_2 \Downarrow^{\ell+1} \alpha$}
\end{prooftree}

\begin{prooftree}
  \AxiomC{$\tau_2 \Downarrow^\ell \alpha$}
  \RightLabel{\scriptsize{(right-or)}}
  \UnaryInfC{$\tau_1 + \tau_2 \Downarrow^{\ell+1} \alpha$}
\end{prooftree}

Ще пишем $\tau \Downarrow \alpha$ точно тогава, когато съществува $\ell$, за което $\tau \Downarrow^\ell \alpha$.

\begin{example}
  Нека $\Sigma = \{a,b\}$. Да разгледаме програмата $R$, където:
  \begin{align*}
    & X_1 = \overbrace{a \cdot X_1 + b \cdot X_2 + \varepsilon}^{\tau_1[X_1,X_2]}\\
    & X_2 = \underbrace{b \cdot X_2 + \varepsilon}_{\tau_2[X_1,X_2]}.
  \end{align*}
  Съобразете, че за всеки две естествени числа $n$ и $k$ е изпълнено:
  \begin{itemize}
  \item
    $\tau_1 \Downarrow a^k b^n$;
  \item
    $\tau_2 \Downarrow b^n$.
  \end{itemize}
\end{example}


\subsubsection{Денотационна семантика}

\begin{problem}
  Докажете, че $\mathcal{S} = (\mathcal{P}(\Sigma^\star), \subseteq, \emptyset)$ е област на Скот.
\end{problem}

За всеки терм $\tau[X_1,\dots,X_n]$ дефинираме изображението
% \marginpar{Тук за променливите използваме главни букви за да се сещаме, че те приемат стойности множества от думи.}
\[\val{\tau}: \mathcal{P}(\Sigma^\star)^n \to \mathcal{P}(\Sigma^\star)\]
 по следния начин:
\begin{itemize}
\item
    $\val{\emptyset}(L_1,\dots,L_n) = \emptyset$.
\item 
  $\val{\varepsilon}(L_1,\dots,L_n) = \{\varepsilon\}$.
\item 
  $\val{a_i \cdot X_j}(L_1,\dots,L_n) = \{a_i\} \cdot L_j$.
\item
  $\val{\tau_1 + \tau_2}(L_1,\dots,L_n) = \val{\tau_1}(L_1,\dots,L_n) \cup \val{\tau_2}(L_1,\dots,L_n)$.
\end{itemize}

\begin{problem}
  Докажете, че за всеки терм $\tau$ имаме, че $\val{\tau}$ е непрекъснато изображение в областта на Скот
  $\mathcal{S} = ( \mathcal{P}(\Sigma^\star),\subseteq, \emptyset)$.
\end{problem}

\marginpar{Програмата е просто текст. Системата е редица от уравнения на непрекъснати изображения.}

\begin{example}
  Да разгледаме програмата $R$, където:
  \begin{align*}
    & X_1 = \overbrace{b \cdot X_1 + a\cdot X_2}^{\tau_1[X_1,X_2]}\\
    & X_2 = \underbrace{\varepsilon}_{\tau_2[X_1,X_2]}.
  \end{align*}
  Според горната дефиниция на семантика на термове, на програмата $R$ съотвества следната система от непректснати изображения:
  \begin{align*}
    & L = \overbrace{\{b\} \cdot L \cup \{a\} \cdot М}^{\val{\tau_1}(L,М)}\\
    & М = \underbrace{\{\varepsilon\}}_{\val{\tau_2}(L_1,L_2)}.
  \end{align*}
  От \Th{sep:min-solution-system} знаем, че тази система има най-малко решение. Нека да го намерим.
  За тази цел, да дефинираме непрекъснатото изображение $\Gamma:\mathcal{P}(\Sigma^\star)^2 \to \mathcal{P}(\Sigma^\star)^2$ като
  \marginpar{От \ref{} знаем, че $\Gamma$ е непрекъснато изображение.}
  \[\Gamma \df \val{\tau_1} \times \val{\tau_2},\]
  т.е. за произволни езици $L$ и $М$ над азбуката $\Sigma$ е изпълнено, че:
  \[\Gamma(L,M) = (\val{\tau_1}(L,M), \val{\tau_2}(L,M)).\]

  От \hyperref[th:knaster-tarski]{Теоремата на Клини} знаем как можем да намерим най-малката неподвижна точка на $\Gamma$,
  която ще бъде и най-малкото решение на горната система.

  \begin{itemize}
  \item 
    $(L_0,M_0) \df (\emptyset,\emptyset)$;
  \item
    $(L_1,M_1) \df \Gamma(L_0,M_0) = (\val{\tau_1}(L_0,M_0), \val{\tau_2}(L_0,M_0)) = (\emptyset, \{\varepsilon\})$;
  \item
    $(L_2,M_2) \df \Gamma(L_1,M_1) = (\val{\tau_1}(L_1,M_1), \val{\tau_2}(L_1,M_1)) = (\{a\},\{\varepsilon\})$;
  \item
    $(L_3,M_3) \df \Gamma(L_2,M_2) = (\val{\tau_1}(L_2,M_2), \val{\tau_2}(L_2,M_2)) = (\{ba,a\},\{\varepsilon\})$;
  \item
    $(L_4,M_4) \df \Gamma(L_3,M_3) =(\val{\tau_1}(L_3,M_3), \val{\tau_2}(L_3,M_3)) = (\{bba, ba,a\},\{\varepsilon\})$;
  \item
    $(L_5,M_5) \df \Gamma(L_4,M_4) = ( \val{\tau_1}(L_4,M_4), \val{\tau_2}(L_4,M_4)) = (\{bba, bba, ba,a\},\{\varepsilon\})$.
  \end{itemize}
  Лесно се съобразява, че $L_n = \{ b^ka \mid k < n\}$, а $M_n = \{\varepsilon\}$. Тогава
  \[\lfp( \Gamma ) = (\bigcup_n L_n, \bigsqcup_n M_n) = (\{b\}^\star \cdot \{a\}, \{\varepsilon\} ).\]
\end{example}


\begin{problem}
  Докажете, че най-малкото решение на системата 
  \begin{align*}
    & X_1 = a \cdot X_1 + b \cdot X_2 + \varepsilon\\
    & X_2 = b \cdot X_2 + \varepsilon
  \end{align*}
  е двойката $(\{a\}^\star \cdot \{b\}^\star, \{b\}^\star)$.
\end{problem}

\begin{problem}
  Да разгледаме системата от непрекъснати оператори
  \begin{align*}
    & \val{\tau_1}(L_1,\dots,L_n) = L_1\\
    & \ \ \vdots\\
    & \val{\tau_n}(L_1,\dots,L_n) = L_n.
  \end{align*}
  Знаем, че тя притежава най-малко решение $(\hat{L}_1,\dots,\hat{L}_n)$.
  Докажете, че всеки от езиците $\hat{L}_i$ е регулярен.

  Докажете, че всеки регулярен език е елемент от най-малкото решение 
  на някоя система от непрекъснати изображения от горния вид.
\end{problem}

% Да разгледаме програмата $P$
% \begin{align*}
%   & X_1 = \tau_1[X_1,\dots,X_n]\\
%   & \vdots\\
%   & X_n = \tau_n[X_1,\dots,X_n].
% \end{align*}

% \marginpar{Няма как да дефинираме операционна семантика, която да разпознава целия език, защото за това ще са ни необходими безкрайно много стъпки.}

% Тогава можем да дефинираме релацията $\rho \Downarrow^\ell_P \alpha$, за терм $\rho$, който съдържа променливи измежду $X_1,\dots,X_n$,
% по следния начин:
% % \begin{prooftree}
% %   \AxiomC{}
% %   \UnaryInfC{$\emptyset \Downarrow^0_P \emptyset$}
% % \end{prooftree}

% \begin{prooftree}
%   \AxiomC{}
%   \UnaryInfC{$\varepsilon \Downarrow^0_P \varepsilon$}
% \end{prooftree}

% \begin{prooftree}
%   \AxiomC{$\tau_i \Downarrow^\ell_P \alpha$}
%   \UnaryInfC{$a \cdot X_i \Downarrow^{\ell+1}_P a \cdot \alpha$}
% \end{prooftree}

% \begin{prooftree}
%   \AxiomC{$\rho_1 \Downarrow^{\ell_1}_P \alpha_1$}
%   % \AxiomC{$\rho_2 \Downarrow^{\ell_2}_P \alpha_2$}
%   \UnaryInfC{$\rho_1 + \rho_2 \Downarrow^{\ell_1+1}_P \alpha_1$}
% \end{prooftree}

% \begin{prooftree}
%   % \AxiomC{$\rho_1 \Downarrow^{\ell_1}_P \alpha_1$}
%   \AxiomC{$\rho_2 \Downarrow^{\ell_2}_P \alpha_2$}
%   \UnaryInfC{$\rho_1 + \rho_2 \Downarrow^{\ell_2+1}_P \alpha_2$}
% \end{prooftree}


% \begin{problem}
%   Нека $P$ е програма и $\tau$ е терм с променливи измежду тези, които се срещат в $P$.
%   Нека $(L_1,\dots,L_n)$ е най-малкото решение на системата $P$.
%   Докажете, че
%   \[\alpha \in \val{\tau}(L_1,\dots,L_n) \iff \tau \Downarrow_P \alpha.\]
% \end{problem}


\subsubsection{Еквивалентност на денотационната и операционната семантики}

Нека тук да фиксираме една програма $R$
\begin{align*}
  & X_1 = \tau_1[X_1,\dots,X_n]\\
    & \vdots\\
  & X_n = \tau_n[X_1,\dots,X_n].
\end{align*}
Нека $(L_1,\dots,L_n)$ е най-малкото решение на непрекъснатото изображение $\Gamma \df \val{\tau_1}\times \cdots \times \val{\tau_n}$. Целта ни е да докажем, че за всяко $i = 1,\dots,n$ е изпълнено, че:
\[L_i = \{\alpha \in \Sigma^\star \mid \tau_i \Downarrow \alpha\}.\]
Това ще направим на две стъпки.

\marginpar{Да напомним, че $L^0_i = \emptyset$ и $L^{k+1}_i = \val{\tau_i}(L^k_1,\dots,L^k_n)$.}

\begin{lemma}
  За всеки терм $\tau$ с променливи измежду $X_1,\dots,X_n$ и всяко $r$ е изпълнено, че:
  \[\val{\tau}(L^r_1,\dots,L^r_n) \subseteq \{\alpha \in \Sigma^\star \mid \tau \Downarrow \alpha\}.\]
\end{lemma}
\begin{proof}
  Нека да кръстим $Include(r)$ твърдението
  ,,за всеки терм $\tau$ с променливи измежду $X_1,\dots,X_n$ е изпълнено, че:
  \[\val{\tau}(L^r_1,\dots,L^r_n) \subseteq \{\alpha \in \Sigma^\star \mid \tau \Downarrow \alpha\}".\]
  Целта ни е да докажем, че $(\forall r\in\Nat)Include(r)$. Това ще направим с индукция по $r$.
  
  Нека $r = 0$. Тук знаем, че $L^0_i = \emptyset$. Тук ще направим вътрешна индукция по построението на термовете $\tau$ за да докажем, че:
  \begin{equation}
    \label{eq:17}
    \val{\tau}(L^0_1,\dots,L^0_n) \subseteq \{\alpha \in \Sigma^\star \mid \tau \Downarrow \alpha\}
  \end{equation}
  \begin{itemize}
  \item
    Нека $\tau \equiv \emptyset$. Понеже $\val{\tau}(L^0_1,\dots,L^0_n) = \emptyset$, то е ясно, че Свойство~\ref{eq:17} е изпълнено.
  \item
    Нека $\tau \equiv \varepsilon$. Тогава $\val{\tau}(L^0_1,\dots,L^0_n) = \{\varepsilon\}$.
    От правилата на операционната семантика имаме, че $\varepsilon \Downarrow \varepsilon$.
    Заключаваме, че Свойство~\ref{eq:17} е изпълнено.
  \item
    Нека $\tau \equiv a \cdot X_j$, за някой индекс $j$ измежду $1,\dots,n$.
    Тогава $\val{\tau}(L^0_1,\dots,L^0_n) = \{a\}\cdot L^0_j = \{a\} \cdot \emptyset = \emptyset$ и оттук е очевидно, че Свойство~\ref{eq:17} е в сила.
  \item
    Нека $\tau \equiv \rho + \mu$. Тогава $\val{\tau}(L^0_1,\dots,L^0_n) = \val{\rho}(L^0_1,\dots,L^0_n) \cup \val{\mu}(L^0_1,\dots,L^0_n)$. Сега от И.П. за Свойство~\ref{eq:17} получаваме, че
    $\val{\rho}(L^0_1,\dots,L^0_n) \subseteq \{\alpha \in \Sigma^\star \mid \rho \Downarrow \alpha\}$ и
    $\val{\mu}(L^0_1,\dots,L^0_n) \subseteq \{\alpha \in \Sigma^\star \mid \mu \Downarrow \alpha\}$.
    От правилата на операционната семантика имаме, че
    $\{\alpha \in \Sigma^\star \mid \rho \Downarrow \alpha\} \subseteq \{\alpha \in \Sigma^\star \mid \rho \Downarrow \alpha\}$ и 
    $\{\alpha \in \Sigma^\star \mid \mu \Downarrow \alpha\} \subseteq \{\alpha \in \Sigma^\star \mid \mu \Downarrow \alpha\}$.
    Заключаваме, че Свойство~\ref{eq:17} е изпълнено.
  \end{itemize}

  Нека $r > 0$ и да приемем, че е изпълнено $Include(r-1)$. Ще докажем, че е изпълнено $Include(r)$.
  Тук знаем, че $L^r_i = \val{\tau_i}(L^{r-1}_1,\dots,L^{r-1}_n)$.
  Ще направим вътрешна индукция по построението на термовете $\tau$ за да докажем, че:
  \begin{equation}
    \label{eq:18}
    \val{\tau}(L^r_1,\dots,L^r_n) \subseteq \{\alpha \in \Sigma^\star \mid \tau \Downarrow \alpha\}.
  \end{equation}
  \begin{itemize}
  \item
    Нека $\tau \equiv \emptyset$.
  \item
    Нека $\tau \equiv \varepsilon$.
  \item
    Нека $\tau \equiv a \cdot X_j$ за някой индекс $j$ измежду $1,\dots,n$.
    Тогава $\val{\tau}(L^r_1,\dots,L^r_n) = \{a\}\cdot L^r_j$.
    Да напомним, че $L^r_j = \val{\tau_j}(L^{r-1}_1,\dots,L^{r-1}_n)$.
    Понеже $Include(r-1)$ е изпълено, то имаме автоматично, че:
    \[L^r_j = \val{\tau_j}(L^{r-1}_1,\dots,L^{r-1}_n) \subseteq \{\beta \in \Sigma^\star \mid \tau_j \Downarrow \beta\}.\]
    От правилата на операционната семантика имаме, че:
    \begin{prooftree}
      \AxiomC{$\tau_j \Downarrow \beta$}
      \UnaryInfC{$a \cdot X_j \Downarrow a \cdot \beta$}
    \end{prooftree}
    Понеже $\tau \equiv a \cdot X_j$, получаваме следното:
    \begin{equation}
      \label{eq:21}
      \{a\} \cdot \{\beta \in \Sigma^\star \mid \tau_j \Downarrow \beta\} \subseteq \{ \alpha \in \Sigma^\star \mid \tau \Downarrow \alpha\}.
    \end{equation}
    Заключаваме, че:
    \begin{align*}
      \val{\tau}(L^r_1,\dots,L^r_n) & = \{a\} \cdot L^r_j \\
                                    & \subseteq \{a\} \cdot \{\beta \in \Sigma^\star \mid \tau_j \Downarrow \beta\} & \comment\text{от }Include(r-1)\\
      & \subseteq \{ \alpha \in \Sigma^\star \mid \tau \Downarrow \alpha\}. & \comment\text{\ref{eq:21}}
    \end{align*}
  \item
    Нека $\tau \equiv \rho + \mu$. Тук имаме, че
    $\val{\tau}(L^r_1,\dots,L^r_n) = \val{\rho}(L^r_1,\dots,L^r_n) \cup \val{\mu}(L^r_1,\dots,L^r_n)$.
    От вътрешното индукционно предположение, имаме, че
    $\val{\rho}(L^r_1,\dots,L^r_n) \subseteq \{\alpha \in \Sigma^\star \mid \rho \Downarrow \alpha\}$ и
    $\val{\mu}(L^r_1,\dots,L^r_n) \subseteq \{\alpha \in \Sigma^\star \mid \mu \Downarrow \alpha\}$.
    От правилата на операционната семантика имаме, че
    $\{\alpha \in \Sigma^\star \mid \rho \Downarrow \alpha\} \cup \{\alpha \in \Sigma^\star \mid \mu \Downarrow \alpha\} \subseteq \{\alpha \in \Sigma^\star \mid \rho + \mu \Downarrow \alpha\}$.
    Заключаваме, че:
    \begin{align*}
      \val{\tau}(L^r_1,\dots,L^r_n) & = \val{\rho}(L^r_1,\dots,L^r_n) \cup \val{\mu}(L^r_1,\dots,L^r_n)\\
                                    & \subseteq \{\alpha \in \Sigma^\star \mid \rho \Downarrow \alpha\} \cup \{\alpha \in \Sigma^\star \mid \mu \Downarrow \alpha\} \\
                                    & \subseteq \{\alpha \in \Sigma^\star \mid \mu+\rho \Downarrow \alpha\}\\
      & = \{\alpha \in \Sigma^\star \mid \tau \Downarrow \alpha\}.
    \end{align*}
  \end{itemize}
\end{proof}

\begin{corollary}
  Нека $L_1,\dots,L_n$ е най-малкото решение на изображението $\Gamma \df \val{\tau_1}\times\cdots\times\val{\tau_n}$. Тогава за произволен терм $\tau$ с променливи измежду $X_1,\dots,X_n$ е изпълнено, че:
  \[\val{\tau}(L_1,\dots,L_n) \subseteq \{\alpha \in \Sigma^\star \mid \tau \Downarrow \alpha\}.\]
\end{corollary}
\begin{proof}
  Използваме, че $\val{\tau}$ е непрекъснато изображение. Тогава
  \begin{align*}
    \val{\tau}(\bigcup_r L^r_1,\dots,\bigcup_rL^r_n) & = \bigcup_r\val{\tau}(L^r_1,\dots,L^r_n)\\
                                                     & \subseteq \{\alpha \in \Sigma^\star \mid \tau \Downarrow \alpha\}.
  \end{align*}
\end{proof}

\begin{lemma}
  Нека $L_1,\dots,L_n$ е неподвижна точка на $\Gamma \df \val{\tau_1}\times \cdots \times \val{\tau_n}$.
  За всеки терм $\tau$ с променливи измежду $X_1,\dots,X_n$ е  изпълнено, че:
  \[\{\alpha \in \Sigma^\star \mid \tau \Downarrow \alpha\} \subseteq \val{\tau}(L_1,\dots,L_n).\]
\end{lemma}
\begin{proof}
  Да означим с $Include(\ell)$ твърдението ,,за всеки терм $\tau$ с променливи измежду $X_1,\dots,X_n$ е изпълнено, че:
  \[\{\alpha \in \Sigma^\star \mid \tau \Downarrow^\ell \alpha\} \subseteq \val{\tau}(L_1,\dots,L_n)".\]
  Ще докажем, че $(\forall \ell\in\Nat)Include(\ell)$.

  Нека $\ell = 0$. Тогава $\tau = \varepsilon$.

  Нека $\ell > 0$. Тогава имаме няколко случая.
  \begin{itemize}
  \item
    \begin{prooftree}
      \AxiomC{$\tau_j \Downarrow^{\ell-1} \beta$}
      \UnaryInfC{$a \cdot X_j \Downarrow^\ell a \cdot \beta$}
    \end{prooftree}
    Според правилата на операционната семантика имаме, че:
    \begin{equation}
      \label{eq:22}
      \{a\} \cdot \{\beta \in \Sigma^\star \mid \tau_j \Downarrow^{\ell-1} \beta\} = \{\alpha \in \Sigma^\star \mid a \cdot X_j \Downarrow^{\ell} \alpha\}
    \end{equation}
    
    Знаем, че $\val{a \cdot X_j}(L_1,\dots,L_n) = \{a\} \cdot L_j = \{a\} \cdot \val{\tau_j}(L_1,\dots,L_n)$,
    защото $L_1,\dots,L_n$ е неподвижна точка на $\Gamma$.
    От $Include(\ell-1)$ имаме следното:
    \[\{\beta \in \Sigma^\star \mid \tau_j \Downarrow^{\ell-1} \beta\} \subseteq \val{\tau_j}(L_1,\dots,L_n).\]
  \item
    Нека имаме следното:
    \begin{prooftree}
      \AxiomC{$\rho \Downarrow^{\ell-1} \alpha$}
      \UnaryInfC{$\rho + \mu \Downarrow^\ell \alpha$}
    \end{prooftree}

    Знаем, че $\val{\rho+\mu}(L_1,\dots,L_n) = \val{\rho}(L_1,\dots,L_n) \cup \val{}(L_1,\dots,L_n)$.
    Използвайки индукционното предположение, което ни казва, че $Include(\ell-1)$ е изпълнено, получаваме:
    \begin{align*}
      \{\alpha \in \Sigma^\star \mid \rho \Downarrow^{\ell-1} \alpha\} & \subseteq \val{\rho}(L_1,\dots,L_n)\\
                                                                       & \subseteq \val{\rho+\mu}(L_1,\dots,L_n).
    \end{align*}
    
  \item
    \begin{prooftree}
      \AxiomC{$\mu \Downarrow^{\ell-1} \alpha$}
      \UnaryInfC{$\rho + \mu \Downarrow^\ell \alpha$}
    \end{prooftree}
  \end{itemize}
\end{proof}



\begin{theorem}
  Да разгледаме програмата $R$
  \begin{align*}
    & X_1 = \tau_1[X_1,\dots,X_n]\\
    & \vdots\\
    & X_n = \tau_n[X_1,\dots,X_n].
  \end{align*}
  Нека $(L_1,\dots,L_n)$ е най-малкото решение на непрекъснатото изображение $\Gamma \df \val{\tau_1}\times \cdots \times \val{\tau_n}$.
  Тогава за всяко $i = 1,\dots,n$ е изпълнено, че:
  \[L_i = \{\alpha \in \Sigma^\star \mid \tau_i \Downarrow \alpha\}.\]
\end{theorem}
\begin{proof}
  Ще разгледаме двете посоки поотделно.
  Първо ще докажем, че за всяко $i = 1,\dots,n$ е изпълнено, че:
  \[L_i \subseteq \{\alpha \in \Sigma^\star \mid \tau_i \Downarrow \alpha\}.\]
  Да напомним, че $L_i = \bigcup^\infty_{k=0} L^k_i$, където $L^0_i = \emptyset$ и $L^{k+1}_i = \val{\tau_i}(L^k_1,\dots,L^k_n)$.
  Ще докажем с индукция, че за всяко $k$ е изпълнено, че:
  \[L^k_i \subseteq \{\alpha \in \Sigma^\star \mid \tau_i \Downarrow \alpha\}.\]
  \begin{itemize}
  \item
    За $k = 0$ твърдението е очевидно, защото $L^0_i = \emptyset \subseteq \{\alpha \in \Sigma^\star \mid \tau_i \Downarrow \alpha\}$.
  \item
    Нека твърдението е изпълнено за произволен индекс $k$.
  \item
    Ще докажем, че $L^{k+1}_i \subseteq \{\alpha \in \Sigma^\star \mid \tau_i \Downarrow \alpha\}$.
    Понеже $L^{k+1}_i = \val{\tau_i}(L^k_1,\dots,L^k_n)$, то трябва да разгледаме какво представлява терма $\tau_i$.
    \begin{itemize}
    \item
      Ако $\tau_i \equiv \emptyset$, то $L^{k+1}_i = \emptyset$ и всичко е очевидно.
    \item
      Ако $\tau_i \equiv \varepsilon$, то $L^{k+1}_i  = \val{\varepsilon}(L^k_1,\dots,L^k_n) = \{\varepsilon\}$.
      От друга страна, според правилата на операционната семантика имаме, че $\varepsilon \Downarrow \varepsilon$.
      Следователно, $L^{k+1}_i \subseteq \{\alpha \in \Sigma^\star \mid \tau_i \Downarrow \alpha\}$.
    \item
      Ако $\tau_i \equiv a \cdot X_j$, то $L^{k+1}_i = \{a\} \cdot L^k_j$.
      От правилата на операционната семантика имаме следното:
      \begin{prooftree}
        \AxiomC{$\tau_j \Downarrow \beta$}
        \UnaryInfC{$a \cdot X_j \Downarrow a \cdot \beta$}
      \end{prooftree}
      Така получаваме, че:
      \begin{equation}
        \label{eq:11}
        \{a\} \cdot \{\beta \in \Sigma^\star \mid \tau_j \Downarrow \beta\} \subseteq \{ \alpha \in \Sigma^\star \mid a \cdot X_j \Downarrow \alpha\}.
      \end{equation}
      От И.П. имаме следното:
      \[L^k_j \subseteq \{\beta \in \Sigma^\star \mid \tau_j \Downarrow \beta\}.\]
      Оттук следва, че:
      \begin{align*}
        L^{k+1}_i & = \val{\tau_i}(L^k_1,\dots,L^k_n)\\
                  & = \{a\} \cdot L^k_j\\
                  & \subseteq \{a\} \cdot \{\beta \in \Sigma^\star \mid \tau_j \Downarrow \beta\} & \comment\text{от И.П.}\\
                  & \subseteq \{ \alpha \in \Sigma^\star \mid a \cdot X_j \Downarrow \alpha\} & \comment\text{от \ref{eq:11}}\\
                  & = \{ \alpha \in \Sigma^\star \mid \tau_i \Downarrow \alpha\}. & \comment \tau_i \equiv a \cdot X_j
      \end{align*}
    \item
      Ако $\tau \equiv \tau_1 + \tau_2$, то $L^{k+1}_i = \val{\tau_1}(L^k_1,\dots,L^k_n) \cup \val{\tau_2}(L^k_1,\dots,L^k_n)$.
      От правилата на операционната семантика получаваме, че:
      \begin{equation}
        \label{eq:13}
        \{\alpha \in \Sigma^\star \mid \tau_1 \Downarrow \alpha\} \cup \{\alpha \in \Sigma^\star \mid \tau_2 \Downarrow \alpha\} \subseteq \{\alpha \in \Sigma^\star \mid \tau \Downarrow \alpha\}.
      \end{equation}
    \end{itemize}
  \end{itemize}
\end{proof}



\subsection{Безконтекстни езици}\index{език!безконтекстен}

Да фиксираме азбуката $\Sigma = \{a_1,\dots,a_n\}$.
Да дефинираме термове от тип 1 като
\[\tau ::= \varepsilon\ |\ \emptyset\ |\ X\ |\ a\ |\ \tau_1 \cdot \tau_2\ |\ \tau_1 + \tau_2,\]
където $j = 1, \dots,n$, а $X_i$ са изброимо безкрайна редица от променливи.

\subsubsection{Операционна семантика}

\begin{prooftree}
  \AxiomC{}
  \UnaryInfC{$\varepsilon \Downarrow^0 \varepsilon$}
\end{prooftree}

\begin{prooftree}
  \AxiomC{}
  \UnaryInfC{$a \Downarrow^0 a$}
\end{prooftree}

\begin{prooftree}
  \AxiomC{$\tau_i \Downarrow^\ell \alpha$}
  \UnaryInfC{$X_i \Downarrow^{\ell+1} \alpha$}
\end{prooftree}

\begin{prooftree}
  \AxiomC{$\tau_1 \Downarrow^\ell \alpha$}
  \UnaryInfC{$\tau_1 + \tau_2 \Downarrow^{\ell+1} \alpha$}
\end{prooftree}

\begin{prooftree}
  \AxiomC{$\tau_2 \Downarrow^\ell \alpha$}
  \UnaryInfC{$\tau_1 + \tau_2 \Downarrow^{\ell+1} \alpha$}
\end{prooftree}


\begin{prooftree}
  \AxiomC{$\tau_1 \Downarrow^{\ell_1} \alpha_1$}
  \AxiomC{$\tau_2 \Downarrow^{\ell_2} \alpha_2$}
  \BinaryInfC{$\tau_1 \cdot \tau_2 \Downarrow^{\ell_1+\ell_2+1} \alpha_1 \cdot \alpha_2$}
\end{prooftree}
\subsubsection{Денотационна семантика}

За всеки терм $\tau[X_1,\dots,X_n]$ дефинираме оператора 
\[\val{\tau}: (\mathcal{P}(\Sigma^\star))^n \to \mathcal{P}(\Sigma^\star)\]
по следния начин:
\begin{itemize}
\item 
  $\val{X_i}(L_1,\dots,L_n) = L_i$.
\item 
  $\val{a_j}(L_1,\dots,L_n) = \{a_j\}$.
\item 
  $\val{\varepsilon}(L_1,\dots,L_n) = \{\varepsilon\}$.
\item 
  $\val{\emptyset}(L_1,\dots,L_n) = \emptyset$.
\item 
  $\val{\tau_1 \cdot \tau_2}(L_1,\dots,L_n) = \val{\tau_1}(L_1,\dots,L_n) \cdot \val{\tau_2}(L_1,\dots,L_n)$.
\item
  $\val{\tau_1 + \tau_2}(L_1,\dots,L_n) = \val{\tau_1}(L_1,\dots,L_n) \cup \val{\tau_2}(L_1,\dots,L_n)$.
\end{itemize}

\begin{problem}
  Докажете, че за всеки терм $\tau$, $\val{\tau}$ е непрекъснато изображение в областта на Скот
  $\mathcal{S} = ( \mathcal{P}(\Sigma^\star),\subseteq, \emptyset)$.
\end{problem}

\begin{problem}
  Докажете, че $\{a^nb^n \mid n\in \Nat\} = \lfp(\val{\tau})$, където 
  \[\tau[X] \equiv \varepsilon + a \cdot X \cdot b.\]
  С други думи, $\{a^nb^n \mid n \in \Nat\}$ е най-малкото решение на уравнението
  \[X = a \cdot X \cdot b + \varepsilon.\]
\end{problem}

Нека сега да разгледаме термовете $\tau_1[X_1,\dots,X_n], \dots, \tau_n[X_1,\dots,X_n]$.

\begin{problem}
  Да разгледаме системата от непрекъснати оператори
  \begin{align*}
    & \val{\tau_1}(L_1,\dots,L_n) = L_1\\
    & \ \ \vdots\\
    & \val{\tau_n}(L_1,\dots,L_n) = L_n.
  \end{align*}
  Знаем, че тя притежава най-малко решение $(\hat{L}_1,\dots,\hat{L}_n)$.
  Докажете, че всеки от езиците $\hat{L}_i$ е безконтекстен.

  Докажете, че всеки безконтекстен език е елемент от най-малкото решение 
  на някоя система от оператори от горния вид.
\end{problem}

\begin{problem}
  \marginpar{Това е аналог на нормалната форма на Чомски}
  Да дефинираме термове от тип 2 като
  \[\tau ::= a_j\ |\ \varepsilon\ |\ \emptyset\ |\ X_i \cdot X_k\ |\ (\tau_1 + \tau_2),\]
  където $j = 1, \dots,n$, а $X_i$ са изброимо безкрайна редица от променливи.
  Докажете горното твърдение, като замените термовете от тип 1 с тези от тип 2.
\end{problem}

\begin{example}
  Да разгледаме системата
  \begin{align*}
    & X_1 = X_3 \cdot X_2 + \varepsilon\\
    & X_2 = X_1 \cdot X_4\\
    & X_3 = a\\
    & X_4 = b.
  \end{align*}


  % \begin{align*}
  %   & \val{\varepsilon + X_3 \cdot X_2}(L_1, L_2, L_3, L_4) = L_1\\
  %   & \val{X_1 \cdot X_4}(L_1, L_2, L_3, L_4) = L_2\\
  %   & \val{a}(L_1, L_2, L_3, L_4) = L_3\\
  %   & \val{b}(L_1, L_2, L_3, L_4) = L_4\\
  % \end{align*}
  Нека $(\hat{L}_1, \hat{L}_2, \hat{L}_3, \hat{L}_4)$ е най-малкото решение на системата.
  Докажете, че $\hat{L}_1 = \{a^nb^n\mid n \in \Nat\}$ $\hat{L}_2 = \{a^nb^{n+1}\mid n \in \Nat\}$,
  $\hat{L}_3 = \{a\}$ и $\hat{L}_4 = \{b\}$.
\end{example}


%%% Local Variables:
%%% mode: latex
%%% TeX-master: "../sep"
%%% End:


% \newpage

\section{Задачи}

\begin{problem}
  Нека $\A$ е област на Скот и нека $f \in \Cont{\A}{\A}$.
  Да разгледаме множеството $B = \{a \in \A \mid f(a) = f\}$.
  Докажете, че $\B = (B, \sqsubseteq, \lfp(f))$ е област на Скот.
\end{problem}




\subsection{Категория на Скот $\mathcal{S}$}


Категория представлява съвкупност от:
\begin{itemize}
\item
  Съвкупност от обекти $a,b,c,\dots$
\item
  Съвкупност от морфизми $f,g,h,\dots$.
\item
  Операции $dom$ и $cod$, за които
  $dom(f)$ и $cod(f)$ са обекти.
  Ще пишем $f:a\to b$ когато $dom(f) = a$ и $cod(f) = b$.
\item
  Имаме операция $\circ$ със свойството, че
  за всеки два морфизма от вида $f:a\to b$ и $g : b \to c$, то
  $g \circ f : a \to c$ също е морфизъм и също така
  \[h \circ (f \circ g) = (h \circ f) \circ g.\]
\item
  За всеки обект $a$ съществува морфизъм $id_a:a\to a$ със свойството, че
  за всеки морфизъм $f:a\to b$, то
  $id_a \circ f = f$ и $f \circ id_b = f$.
\end{itemize}


\begin{itemize}
\item 
  Обектите са всички области на Скот
\item
  Морфизмите са всички непрекъснати изображения между области на Скот
\end{itemize}


Функтор $\F : \mathcal{S} \to \mathcal{S}$ е изображение, за което са изпълнени условията:
\begin{itemize}
\item
  $g \in \Cont{\A}{\B}$, то $\F(g) \in \Cont{\F(\A)}{\F(\B)}$;
\item
  $\F(id_\A) = id_{\F(\A)}$;
\item
  $\F(h\circ g) = \F(h) \circ \F(g)$.
\end{itemize}



\begin{haskellcode}
class Functor f where
  fmap :: (a -> b) -> f a -> f b
\end{haskellcode}

\begin{haskellcode}
data Maybe a = Nothing | Just a

instance Functor Maybe where
  fmap :: (a -> b) -> Maybe a -> Maybe b
  fmap f Nothing = Nothing
  fmap f (Just x) = Just (f x)
\end{haskellcode}


Например, $\texttt{List}$ е функтор, където за произволно $g \in \Cont{\A}{\B}$
\[\texttt{List}(g) \in \Cont{\texttt{List}(\A)}{\texttt{List}(\B)}\]
и
\[\texttt{List}(g) \df \texttt{map}(g).\]


\begin{haskellcode}
instance Functor [] where
  fmap = map
\end{haskellcode}


\begin{haskellcode}
data Tree a = Leaf a | Root a (Tree a) (Tree a)

instance Functor Tree where
  fmap g (Leaf x) = Leaf (g x)
  fmap g (Root x l r) = Root (g x) (fmap g l) (fmap g r)
\end{haskellcode}


Например, $\texttt{Arrow}_\B$ е функтор, където
$\texttt{Arrow}_\B(\A) = \Cont{\B}{\A}$ и за произволно $f \in \Cont{\A}{\C}$,
\[\texttt{Arrow}_\B(f) : \Cont{\B}{\A} \to \Cont{\B}{\C}\]
и 
\[\texttt{Arrow}_\B(f)(g) \df f \circ g.\]

\begin{haskellcode}
instance Functor ((->) r) where
  fmap :: (a->b) -> (a->r) -> (b->r)
  fmap = (.)
\end{haskellcode}



\subsubsection*{Апликативни функтори}



\begin{haskellcode}
class (Functor f) => Applicative f where
  pure :: a -> f a
  (<*>) :: f(a -> b) -> f a -> f b
\end{haskellcode}


\begin{haskellcode}
instance Applicative Maybe where

  fmap :: (a -> b) -> Maybe a -> Maybe b
  fmap _ Nothing = Nothing
  fmap f Just x = Just (f x)

  pure :: a -> Maybe a
  pure x = Just x

  (<*>) :: Maybe(a -> b) -> Maybe a -> Maybe b
  Nothing <*> _ = Nothing
  (Just f) <*> x = fmap f x
\end{haskellcode}

\begin{haskellcode}
instance Applicative [] where

  fmap :: (a->b) -> [a] -> [b]
  fmap f xs = [f x | x <- xs]
  -- fmap = map
  
  pure :: a -> [a]
  pure x = [x]

  <*> :: [a->b] -> [a] -> [b]
  fs <*> xs = [f x | f <- fs, x <- xs]
\end{haskellcode}


\begin{haskellcode}
instance Applicative ((->) r) where

  fmap :: (a -> b) -> (r -> a) -> (r -> b)
  fmap f g = g . f
  -- fmap = (.)

  pure :: a -> (r -> a)
  pure x = (\_ -> x)

  <*> :: (r -> (a -> b)) -> (r -> a) -> (r -> b)
  f <*> g = \x -> f x (g x)  
\end{haskellcode}

<\$> is the infix version of fmap

\begin{haskellcode}

ghci> map (+2) [1,2,3]
[3,4,5]
ghci> fmap (+2) [1,2,3]
[3,4,5]
ghci> (+2) <$> [1,2,3]
[3,4,5]

\end{haskellcode}



\begin{haskellcode}
class Monad m where
  return :: a -> m a
  (>>=) :: m a -> (a -> m b) -> m b
\end{haskellcode}


\begin{haskellcode}
instance Monad [] where
  return :: a -> [a]
  return x = [x]

  (>>=) :: [a] -> (a -> [b]) -> [b]
  xs >>= f = concat (map f xs)
\end{haskellcode}

%%% Local Variables:
%%% mode: latex
%%% TeX-master: "../sep"
%%% End:


%%% Local Variables:
%%% mode: latex
%%% TeX-master: "../sep"
%%% End:

\chapter{Езикът \REG}\index{език!регулярен}

\section{Регулярни изрази}

Да фиксираме азбуката $\Sigma = \{a_1,\dots,a_k\}$.
\marginpar{Обърнете внимание, че нямаме звездата на Клини в дефиницията на термовете.}
Следната абстрактна граматика 
\[\tau ::= \varepsilon\ |\ a \cdot X\ |\ \tau_1 + \tau_2.\]
описва термовете на езика \REG, където $a$ означава буква от азбуката $\Sigma$, а $X$ означава променлива. Нека имаме и безкраен набор от променливи $X_0,X_1,\dots$.
\marginpar{Ще използваме малките гръцки букви $\tau,\rho,\mu$, евентуално с индекси, за да означаваме термовете. Думите над азбуката $\Sigma$ ще означаваме с малките гръцки букви $\alpha$, $\beta$, $\gamma$.}
Програмите на езика езика $\texttt{REG}$ имат следния вид:
\begin{align*}
  & X_1 = \tau_1[X_1,\dots,X_n]\\
  & \ \vdots\\
  & X_n = \tau_n[X_1,\dots,X_n].
\end{align*}
Ето един пример за програма:
\begin{align*}
  & X_1 = a \cdot X_1 + b \cdot X_2\\
  & X_2 = \varepsilon + b \cdot X_1.
\end{align*}


\section{Операционна семантика}

Да фиксираме една програма на езика \REG:
\begin{align*}
  & X_1 = \tau_1[X_1,\dots,X_n]\\
  & \ \vdots\\
  & X_n = \tau_n[X_1,\dots,X_n].
\end{align*}

С индукция по $\ell$, дефинираме релацията $\tau \Downarrow^\ell \alpha$,
където $\tau$ е терм на езика \REG със свободни променливи измежду $X_1,\dots, X_n$.

\marginpar{
  \begin{itemize}
  \item 
    Тук е важно да се отбележи, че дефиницията на операционната семантика не можем да я дадем с индукция по построението на термовете. Защо?
  \item
    Тук няма как да дефинираме релация $\tau \Downarrow L$, където $L$ е език, защото все пак искаме всяко изчисление да трае краен брой стъпки, а това няма как да стане, когато $L$ е безкраен език.
  \end{itemize}
}


\begin{framed}
  \begin{figure}[H]
    \begin{subfigure}[b]{0.5\textwidth}
      \begin{prooftree}
        \AxiomC{}
        \LeftLabel{\scriptsize{(eps)}}
        \UnaryInfC{$\varepsilon \Downarrow^0 \varepsilon$}
      \end{prooftree}
    \end{subfigure}
    ~
    \begin{subfigure}[b]{0.5\textwidth}
      \begin{prooftree}
        \AxiomC{$\tau_i \Downarrow^\ell \beta$}
        \AxiomC{$\alpha = a\cdot \beta$}
        \RightLabel{\scriptsize{(rec)}}
        \BinaryInfC{$a \cdot X_i \Downarrow^{\ell+1} \alpha$}
      \end{prooftree}
    \end{subfigure}
    \begin{subfigure}[b]{0.5\textwidth}
      \begin{prooftree}
        \AxiomC{$\tau_1 \Downarrow^\ell \alpha$}
        \LeftLabel{\scriptsize{(left-or)}}
        \UnaryInfC{$\tau_1 + \tau_2 \Downarrow^{\ell+1} \alpha$}
      \end{prooftree}
    \end{subfigure}
    ~
    \begin{subfigure}[b]{0.5\textwidth}
      \begin{prooftree}
        \AxiomC{$\tau_2 \Downarrow^\ell \alpha$}
        \RightLabel{\scriptsize{(right-or)}}
        \UnaryInfC{$\tau_1 + \tau_2 \Downarrow^{\ell+1} \alpha$}
      \end{prooftree}
    \end{subfigure}
    \caption{Правила на операционната семантика на езика \REG.}
  \end{figure}
\end{framed}
Ще пишем $\tau \Downarrow \alpha$ точно тогава, когато съществува $\ell$, за което $\tau \Downarrow^\ell \alpha$.


\begin{proposition}
  Докажете, че ако $\tau \Downarrow \alpha$ и $\tau \Downarrow \beta$, то $\alpha = \beta$.
\end{proposition}


\begin{example}
  \marginpar{На всяка променлива от програмата $R$ съответства състояние на автомата.}
  Нека $\Sigma = \{a,b\}$. Да разгледаме програмата $R$, където:
  \begin{align*}
    & X_1 = \overbrace{a \cdot X_1 + b \cdot X_2 + \varepsilon}^{\tau_1[X_1,X_2]}\\
    & X_2 = \underbrace{b \cdot X_2 + \varepsilon}_{\tau_2[X_1,X_2]}.
  \end{align*}
  Съобразете, че за всеки две естествени числа $n$ и $k$ е изпълнено, че $\tau_1~\Downarrow~a^k b^n$ и $\tau_2~\Downarrow~b^n$.
  \begin{figure}[H]
      \centering
      \begin{tikzpicture}[->,>=stealth,thick,node distance=65pt]
        \tikzstyle{every state}=[circle,minimum size=20pt,auto]
        
        \node[initial, state, accepting]   (0) {$X_1$};
        \node[state, accepting]            (1) [right of=0]{$X_2$};
        
        \path
        (0) edge [loop above]   node [above] {$a$}    (0)
        (0) edge [bend left=15] node [above] {$b$}    (1)
        (1) edge [loop above]   node [above] {$b$}    (1);
        % (1) edge [bend left=15] node [above] {$a$}    (2)
        % (2) edge [bend left=45] node [below] {$a$}    (0)
        % (2) edge [bend left=15] node [above] {$b$}    (3)
        % (3) edge [loop above]   node [above] {$a,b$}  (3);
      \end{tikzpicture}
      \caption{Краен автомат съотвестващ на програмата $R$.}
    \end{figure}
\end{example}

\section{Денотационна семантика}

\begin{problem}
  Докажете, че $\mathcal{S} = (\mathcal{P}(\Sigma^\star), \subseteq, \emptyset)$ е област на Скот.
\end{problem}

За всеки терм $\tau[X_1,\dots,X_n]$ дефинираме изображението
% \marginpar{Тук за променливите използваме главни букви за да се сещаме, че те приемат стойности множества от думи.}
\[\val{\tau}: \mathcal{P}(\Sigma^\star)^n \to \mathcal{P}(\Sigma^\star)\]
 по следния начин:
\begin{itemize}
\item 
  $\val{\varepsilon}(L_1,\dots,L_n) = \{\varepsilon\}$.
\item 
  $\val{a_i \cdot X_j}(L_1,\dots,L_n) = \{a_i\} \cdot L_j$.
\item
  $\val{\tau_1 + \tau_2}(L_1,\dots,L_n) = \val{\tau_1}(L_1,\dots,L_n) \cup \val{\tau_2}(L_1,\dots,L_n)$.
\end{itemize}

\begin{problem}
  Докажете, че за всеки терм $\tau$ на езика \REG имаме, че $\val{\tau}$ е непрекъснато изображение в областта на Скот
  $\mathcal{S} = ( \mathcal{P}(\Sigma^\star),\subseteq, \emptyset)$.
\end{problem}

\marginpar{Програмата е просто текст. Системата е редица от уравнения на непрекъснати изображения.}

\begin{example}
  Да разгледаме програмата $R$, където:
  \begin{align*}
    & X_1 = \overbrace{b \cdot X_1 + a\cdot X_2}^{\tau_1[X_1,X_2]}\\
    & X_2 = \underbrace{\varepsilon}_{\tau_2[X_1,X_2]}.
  \end{align*}
  Според горната дефиниция на семантика на термове, на програмата $R$ съотвества следната система от непрекъснати изображения:
  \begin{align*}
    & L = \overbrace{\{b\} \cdot L \cup \{a\} \cdot M}^{\val{\tau_1}(L,M)}\\
    & M = \underbrace{\{\varepsilon\}}_{\val{\tau_2}(L,M)}.
  \end{align*}
  От \Th{sep:min-solution-system} знаем, че тази система има най-малко решение. Нека да го намерим.
  За тази цел, да дефинираме непрекъснатото изображение $\Gamma:\mathcal{P}(\Sigma^\star)^2 \to \mathcal{P}(\Sigma^\star)^2$ като
  \marginpar{От \Prop{cartesian-continuous} знаем, че $\Gamma$ е непрекъснато изображение.}
  \[\Gamma \df \val{\tau_1} \times \val{\tau_2},\]
  т.е. за произволни езици $L$ и $M$ над азбуката $\Sigma$ е изпълнено, че:
  \[\Gamma(L,M) = (\val{\tau_1}(L,M), \val{\tau_2}(L,M)).\]

  От \hyperref[th:knaster-tarski]{Теоремата на Клини} знаем как можем да намерим най-малката неподвижна точка на $\Gamma$,
  която ще бъде и най-малкото решение на горната система.

  \begin{itemize}
  \item 
    $(L_0,M_0) \df (\emptyset,\emptyset)$;
  \item
    $(L_1,M_1) \df \Gamma(L_0,M_0) = (\val{\tau_1}(L_0,M_0), \val{\tau_2}(L_0,M_0)) = (\emptyset, \{\varepsilon\})$;
  \item
    $(L_2,M_2) \df \Gamma(L_1,M_1) = (\val{\tau_1}(L_1,M_1), \val{\tau_2}(L_1,M_1)) = (\{a\},\{\varepsilon\})$;
  \item
    $(L_3,M_3) \df \Gamma(L_2,M_2) = (\val{\tau_1}(L_2,M_2), \val{\tau_2}(L_2,M_2)) = (\{ba,a\},\{\varepsilon\})$;
  \item
    $(L_4,M_4) \df \Gamma(L_3,M_3) =(\val{\tau_1}(L_3,M_3), \val{\tau_2}(L_3,M_3)) = (\{bba, ba,a\},\{\varepsilon\})$;
  \item
    $(L_5,M_5) \df \Gamma(L_4,M_4) = ( \val{\tau_1}(L_4,M_4), \val{\tau_2}(L_4,M_4)) = (\{bba, bba, ba,a\},\{\varepsilon\})$.
  \end{itemize}
  Лесно се съобразява, че $L_n = \{ b^ka \mid k < n\}$, а $M_n = \{\varepsilon\}$. Тогава
  \[\lfp( \Gamma ) = (\bigcup_n L_n, \bigcup_n M_n) = (\{b\}^\star \cdot \{a\}, \{\varepsilon\} ).\]
\end{example}

\begin{example}
  Да разгледаме следния краен детерминиран автомат:
  \marginpar{Означаваме с $\abs{\omega}_a$ броя на срещанията на $a$ в думата $\omega$.}
  \begin{figure}[H]
    \centering
    \begin{tikzpicture}[->,>=stealth,thick,node distance=65pt]
      \tikzstyle{every state}=[circle,minimum size=20pt,auto]
      
      \node[initial, state, accepting]         (0) {$q_1$};
      \node[state]                             (1) [right of=0]{$q_2$};
      \node[state]                             (2) [right of=1]{$q_3$};
      
      \path 
      (0) edge [loop above]   node   [above] {$b$}    (0)
      (0) edge [bend left=15] node   [above] {$a$}    (1)
      (1) edge [loop above]   node   [above] {$b$}    (1)
      (1) edge [bend left=15] node   [above] {$a$}    (2)
      (2) edge [loop above]   node   [above] {$b$}    (2)
      (2) edge [bend left=45] node   [below] {$a$}    (0);
    \end{tikzpicture}
    \caption{Автомат разпознаващ $\{\omega \in \{a,b\}^\star \mid \abs{\omega}_{a} \equiv 0 \bmod\ 3\}$}
  \end{figure}
  На този автомат съответства следната програма $R$:
  \begin{align*}
    & X_1 = a \cdot X_2 + b \cdot X_1 + \varepsilon \\
    & X_2 = a \cdot X_3 + b \cdot X_2\\
    & X_3 = a \cdot X_1 + b \cdot X_3.
  \end{align*}
  На тази програма съответства системата:
  \begin{align*}
    & L_1 = \{a\} \cdot L_2 \cup \{b\} \cdot L_1 \cup \{\varepsilon\}\\
    & L_2 = \{a\} \cdot L_3 \cup \{b\} \cdot L_2\\
    & L_3 = \{a\} \cdot L_1 \cup \{b\} \cdot L_3.
  \end{align*}
  Най-малкото решение на тази система е тройката $(\hat{L}_1,\hat{L}_2,\hat{L}_3)$ от следните езици:
  \begin{align*}
    \hat{L}_1 = & \{\omega \in \{a,b\}^\star \mid \abs{\omega}_{a} \equiv 0 \bmod\ 3\}\\
    \hat{L}_2 = & \{\omega \in \{a,b\}^\star \mid \abs{\omega}_{a} \equiv 2 \bmod\ 3\}\\
    \hat{L}_3 = & \{\omega \in \{a,b\}^\star \mid \abs{\omega}_{a} \equiv 1 \bmod\ 3\}.
  \end{align*}
\end{example}

\begin{example}
  Да разгледаме системата от само едно уравнение
  \[L = \{a\} \cdot L.\]
  Най-малкото решение на тази система е езикът $\emptyset$,
  но тази система има за решение и езикът $\{a^n \mid n > 0\}$.
  Да разгледаме сега програмата
  \[X = \underbrace{a \cdot X}_{\tau[X]}.\]
  Лесно се съобразява, че $\emptyset = \{\alpha \in \Sigma^\star \mid \tau \Downarrow \alpha\}$.
\end{example}

  
\begin{problem}
  Докажете, че най-малкото решение на системата 
  \begin{align*}
    & X_1 = a \cdot X_1 + b \cdot X_2 + \varepsilon\\
    & X_2 = b \cdot X_2 + \varepsilon
  \end{align*}
  е двойката $(\{a\}^\star \cdot \{b\}^\star, \{b\}^\star)$.
  
\end{problem}

\begin{problem}[Теорема за характеризация]
  Да разгледаме системата от непрекъснати оператори
  \begin{align*}
    & \val{\tau_1}(L_1,\dots,L_n) = L_1\\
    & \ \ \vdots\\
    & \val{\tau_n}(L_1,\dots,L_n) = L_n.
  \end{align*}
  Знаем, че тя притежава най-малко решение $(\hat{L}_1,\dots,\hat{L}_n)$.
  Докажете, че всеки от езиците $\hat{L}_i$ е регулярен.

  Докажете, че всеки регулярен език е елемент от най-малкото решение 
  на някоя система от непрекъснати изображения от горния вид.
\end{problem}

\section{Еквивалентност на денотационната и операционната семантика}

Нека тук да фиксираме една програма $R$
\begin{align*}
  & X_1 = \tau_1[X_1,\dots,X_n]\\
    & \vdots\\
  & X_n = \tau_n[X_1,\dots,X_n].
\end{align*}
Нека $(L_1,\dots,L_n)$ е най-малкото решение на непрекъснатото изображение $\Gamma \df \val{\tau_1}\times \cdots \times \val{\tau_n}$. Целта ни е да докажем, че за всяко $i = 1,\dots,n$ е изпълнено, че:
\[L_i = \{\alpha \in \Sigma^\star \mid \tau_i \Downarrow \alpha\}.\]
Този резултат ще наречем теорема за еквивалентност. Това ще направим на две стъпки.

\marginpar{Да напомним, че $L_i = \bigcup_k L^k_i$, където $L^0_i = \emptyset$ и $L^{k+1}_i = \val{\tau_i}(L^k_1,\dots,L^k_n)$. }

\begin{lemma}
  \marginpar{Тук раборим при фиксирана програма $R$.}
  За всеки индекс $r$ и всеки терм $\tau$ на езика \REG с променливи измежду $X_1,\dots,X_n$ е изпълнено, че:
  \[\val{\tau}(L^r_1,\dots,L^r_n) \subseteq \{\alpha \in \Sigma^\star \mid \tau \Downarrow \alpha\}.\]
\end{lemma}
\begin{proof}
  Нека да кръстим $\textbf{Include}(r)$ твърдението
  ,,за всеки терм $\tau$ с променливи измежду $X_1,\dots,X_n$ е изпълнено, че:
  \[\val{\tau}(L^r_1,\dots,L^r_n) \subseteq \{\alpha \in \Sigma^\star \mid \tau \Downarrow \alpha\}".\]
  Целта ни е да докажем, че $(\forall r\in\Nat)\textbf{Include}(r)$. Това ще направим с индукция по $r$.
  
  Нека $r = 0$. Знаем, че $L^0_i = \emptyset$ за всеки индекс $i$ измежду $1,\dots,n$. Тук ще направим вътрешна индукция по построението на термовете $\tau$ за да докажем, че:
  \begin{equation}
    \label{eq:17}
    \val{\tau}(L^0_1,\dots,L^0_n) \subseteq \{\alpha \in \Sigma^\star \mid \tau \Downarrow \alpha\}
  \end{equation}
  \begin{itemize}
  % \item
    \marginpar{Тук е важно, че дефиницията на денотационната семантика следва индуктивното построение на термовете.}
    % Нека $\tau \equiv \emptyset$. Понеже $\val{\tau}(L^0_1,\dots,L^0_n) = \emptyset$, то е ясно, че Свойство~\ref{eq:17} е изпълнено.
  \item
    Нека $\tau \equiv \varepsilon$. Тогава $\val{\tau}(L^0_1,\dots,L^0_n) = \{\varepsilon\}$.
    От правилата на операционната семантика имаме, че $\varepsilon \Downarrow \varepsilon$.
    Заключаваме, че Свойство~\ref{eq:17} е изпълнено.
  \item
    Нека $\tau \equiv a \cdot X_j$, за някой индекс $j$ измежду $1,\dots,n$. Тогава
    \[\val{\tau}(L^0_1,\dots,L^0_n) = \{a\}\cdot L^0_j = \{a\} \cdot \emptyset = \emptyset\]
    и оттук е очевидно, че Свойство~\ref{eq:17} е в сила.
  \item
    Нека $\tau \equiv \rho + \mu$. Тогава $\val{\tau}(L^0_1,\dots,L^0_n) = \val{\rho}(L^0_1,\dots,L^0_n) \cup \val{\mu}(L^0_1,\dots,L^0_n)$. Сега от И.П. за Свойство~\ref{eq:17} получаваме, че
    \begin{align*}
      & \val{\rho}(L^0_1,\dots,L^0_n) \subseteq \{\alpha \in \Sigma^\star \mid \rho \Downarrow \alpha\}\\
      & \val{\mu}(L^0_1,\dots,L^0_n) \subseteq \{\alpha \in \Sigma^\star \mid \mu \Downarrow \alpha\}.
    \end{align*}
    От правилата на операционната семантика имаме, че
    \begin{figure}[H]
      \begin{subfigure}[b]{0.5\textwidth}
        \begin{prooftree}
          \AxiomC{$\rho \Downarrow \alpha$}
          \LeftLabel{\scriptsize{(left-or)}}
          \UnaryInfC{$\rho + \mu \Downarrow \alpha$}
        \end{prooftree}
        \vspace*{2mm}
      \end{subfigure}
      ~
      \begin{subfigure}[b]{0.5\textwidth}
        \begin{prooftree}
          \AxiomC{$\mu \Downarrow \alpha$}
          \RightLabel{\scriptsize{(right-or)}}
          \UnaryInfC{$\rho + \mu \Downarrow \alpha$}
        \end{prooftree}
        \vspace*{2mm}
      \end{subfigure}
    \end{figure}
    Оттук веднага получаваме, че:
    \[\{\alpha \in \Sigma^\star \mid \rho \Downarrow \alpha\} \cup \{\alpha \in \Sigma^\star \mid \mu \Downarrow \alpha\} = \{\alpha \in \Sigma^\star \mid \tau \Downarrow \alpha\}.\]
    Заключаваме, че:
    \begin{align*}
      \val{\tau}(L^0_1,\dots,L^0_n) & = \val{\rho}(L^0_1,\dots,L^0_n) \cup \val{\mu}(L^0_1,\dots,L^0_n)\\
                                    & \subseteq \{\alpha \in \Sigma^\star \mid \rho \Downarrow \alpha\} \cup \{\alpha \in \Sigma^\star \mid \mu \Downarrow \alpha\}\\
                                    & = \{\alpha \in \Sigma^\star \mid \tau \Downarrow \alpha\}.
    \end{align*}
    Свойство~\ref{eq:17} е изпълнено за всеки терм $\tau$.
  \end{itemize}

  Нека $r > 0$ и като индукционно предположение да приемем, че е изпълнено $\textbf{Include}(r-1)$. Ще докажем, че е изпълнено $\textbf{Include}(r)$.
  \marginpar{Тук знаем, че \[L^r_i = \val{\tau_i}(L^{r-1}_1,\dots,L^{r-1}_n).\]}
  Ще направим вътрешна индукция по построението на термовете $\tau$ за да докажем, че:
  \begin{equation}
    \label{eq:18}
    \val{\tau}(L^r_1,\dots,L^r_n) \subseteq \{\alpha \in \Sigma^\star \mid \tau \Downarrow \alpha\}.
  \end{equation}
  \begin{itemize}
  \item
    Нека $\tau \equiv \emptyset$. Отново, понеже $\val{\tau}(L^0_1,\dots,L^0_n) = \emptyset$, то е ясно, че Свойство~\ref{eq:18} е изпълнено.
  \item
    Нека $\tau \equiv \varepsilon$. Отново, понеже $\val{\tau}(L^0_1,\dots,L^0_n) = \{\varepsilon\}$,
    то от правилата на операционната семантика имаме, че $\varepsilon \Downarrow \varepsilon$.
    Заключаваме, че Свойство~\ref{eq:18} е изпълнено.
  \item
    Нека $\tau \equiv a \cdot X_j$ за някой индекс $j$ измежду $1,\dots,n$. Тогава
    \[\val{\tau}(L^r_1,\dots,L^r_n) = \{a\}\cdot L^r_j.\]
    Понеже сме приели, че $\textbf{Include}(r-1)$ е изпълено, то имаме автоматично, че:
    \[L^r_j = \val{\tau_j}(L^{r-1}_1,\dots,L^{r-1}_n) \subseteq \{\beta \in \Sigma^\star \mid \tau_j \Downarrow \beta\}.\]
    От правилата на операционната семантика имаме, че:
    \begin{prooftree}
      \AxiomC{$\tau_j \Downarrow \beta$}
      \UnaryInfC{$a \cdot X_j \Downarrow a \cdot \beta$}
    \end{prooftree}
    Понеже $\tau \equiv a \cdot X_j$, получаваме следното:
    \begin{equation}
      \label{eq:21}
      \{a\} \cdot \{\beta \in \Sigma^\star \mid \tau_j \Downarrow \beta\} = \{ \alpha \in \Sigma^\star \mid \tau \Downarrow \alpha\}.
    \end{equation}
    Заключаваме, че:
    \begin{align*}
      \val{\tau}(L^r_1,\dots,L^r_n) & = \{a\} \cdot L^r_j \\
                                    & = \{a\} \cdot \val{\tau_j}(L^{r-1}_1,\dots,L^{r-1}_n)\\
                                    & \subseteq \{a\} \cdot \{\beta \in \Sigma^\star \mid \tau_j \Downarrow \beta\} & \comment\text{от }\textbf{Include}(r-1)\\
      & = \{ \alpha \in \Sigma^\star \mid \tau \Downarrow \alpha\}. & \comment\text{от \ref{eq:21}}
    \end{align*}
  \item
    Нека $\tau \equiv \rho + \mu$. Тук имаме, че
    \[\val{\tau}(L^r_1,\dots,L^r_n) = \val{\rho}(L^r_1,\dots,L^r_n) \cup \val{\mu}(L^r_1,\dots,L^r_n).\]
    Понеже $\tau$ е построен с помощта на $\rho$ и $\mu$, можем да използваме вътрешното индукционно предположение за $\rho$ и $\mu$, откъдето имаме, че:
    \begin{align*}
      & \val{\rho}(L^r_1,\dots,L^r_n) \subseteq \{\alpha \in \Sigma^\star \mid \rho \Downarrow \alpha\}\\
      & \val{\mu}(L^r_1,\dots,L^r_n) \subseteq \{\alpha \in \Sigma^\star \mid \mu \Downarrow \alpha\}.
    \end{align*}
    От правилата на операционната семантика имаме, че
    \[\{\alpha \in \Sigma^\star \mid \rho \Downarrow \alpha\} \cup \{\alpha \in \Sigma^\star \mid \mu \Downarrow \alpha\} = \{\alpha \in \Sigma^\star \mid \rho + \mu \Downarrow \alpha\}.\]
    Заключаваме, че:
    \begin{align*}
      \val{\tau}(L^r_1,\dots,L^r_n) & = \val{\rho}(L^r_1,\dots,L^r_n) \cup \val{\mu}(L^r_1,\dots,L^r_n)\\
                                    & \subseteq \{\alpha \in \Sigma^\star \mid \rho \Downarrow \alpha\} \cup \{\alpha \in \Sigma^\star \mid \mu \Downarrow \alpha\} \\
                                    & = \{\alpha \in \Sigma^\star \mid \mu+\rho \Downarrow \alpha\}\\
      & = \{\alpha \in \Sigma^\star \mid \tau \Downarrow \alpha\}.
    \end{align*}
    Свойство~\ref{eq:18} е изпълнено за всеки терм $\tau$.
  \end{itemize}
\end{proof}

\begin{corollary}
  Нека $L_1,\dots,L_n$ е най-малкото решение на непрекъснатото изображение $\Gamma \df \val{\tau_1}\times\cdots\times\val{\tau_n}$. Тогава за произволен терм $\tau$ с променливи измежду $X_1,\dots,X_n$ е изпълнено, че:
  \[\val{\tau}(L_1,\dots,L_n) \subseteq \{\alpha \in \Sigma^\star \mid \tau \Downarrow \alpha\}.\]
\end{corollary}
\begin{proof}
  Използваме, че $\val{\tau}$ е непрекъснато изображение. Тогава
  \begin{align*}
    \val{\tau}(L_1,\dots,L_n) & = \val{\tau}(\bigcup_r L^r_1,\dots,\bigcup_rL^r_n)\\
                              & = \bigcup_r\val{\tau}(L^r_1,\dots,L^r_n)\\
                              & \subseteq \{\alpha \in \Sigma^\star \mid \tau \Downarrow \alpha\}.
  \end{align*}
\end{proof}

\begin{lemma}
  Нека $L_1,\dots,L_n$ е неподвижна точка на $\Gamma \df \val{\tau_1}\times \cdots \times \val{\tau_n}$.
  За всеки терм $\tau$ с променливи измежду $X_1,\dots,X_n$ е  изпълнено, че:
  \[\{\alpha \in \Sigma^\star \mid \tau \Downarrow \alpha\} \subseteq \val{\tau}(L_1,\dots,L_n).\]
\end{lemma}
\begin{proof}
  Да означим с $\textbf{Include}(\ell)$ твърдението ,,за всеки терм $\tau$ с променливи измежду $X_1,\dots,X_n$ е изпълнено, че:
  \[\{\alpha \in \Sigma^\star \mid \tau \Downarrow^\ell \alpha\} \subseteq \val{\tau}(L_1,\dots,L_n)".\]
  Ще докажем, че $(\forall \ell\in\Nat)\textbf{Include}(\ell)$.

  Нека $\ell = 0$. Според правилата на операционната семантика, единственият случай, който имаме тук е $\tau \equiv \varepsilon$.

  Нека $\ell > 0$ и да разгледаме дума $\alpha$ и терм $\tau$, за които $\tau \Downarrow^\ell \alpha$. Тогава имаме три случая в зависимост от това какво е последното правило, което сме приложили.
  \begin{itemize}
  \item
    Първият случай е когато имаме, че $\tau \equiv a \cdot X_j$ и
    \begin{prooftree}
      \AxiomC{$\tau_j \Downarrow^{\ell-1} \beta$}
      \UnaryInfC{$a \cdot X_j \Downarrow^\ell a \cdot \beta$}
    \end{prooftree}
    Според правилата на операционната семантика имаме, че:
    \begin{equation}
      \label{eq:22}
      \{a\} \cdot \{\beta \in \Sigma^\star \mid \tau_j \Downarrow^{\ell-1} \beta\} = \{\alpha \in \Sigma^\star \mid \tau \Downarrow^{\ell} \alpha\}
    \end{equation}
    Знаем, че $\val{\tau}(L_1,\dots,L_n) = \{a\} \cdot L_j = \{a\} \cdot \val{\tau_j}(L_1,\dots,L_n)$,
    защото $L_1,\dots,L_n$ е \emph{неподвижна точка} на изображението $\Gamma$.
    От $\textbf{Include}(\ell-1)$ имаме следното:
    \begin{equation}
      \label{eq:11}
      \{\beta \in \Sigma^\star \mid \tau_j \Downarrow^{\ell-1} \beta\} \subseteq \val{\tau_j}(L_1,\dots,L_n).
    \end{equation}
    Заключаваме, че:
    \begin{align*}
      \{\alpha \in \Sigma^\star \mid \tau \Downarrow^\ell \alpha\} & = \{a\} \cdot \{\beta \in \Sigma^\star \mid \tau_j \Downarrow^{\ell-1} \beta\} & \comment\text{от (\ref{eq:22})}\\
                                                                   & \subseteq \{a\} \cdot \val{\tau_j}(L_1,\dots,L_n) & \comment\text{от (\ref{eq:11})}\\
                                                                   & = \{a\} \cdot L_j & \comment L_1,\dots,L_n\text{ е неподв. точка}\\
      & = \val{\tau}(L_1,\dots,L_n) & \comment\text{деф. на денот. сем.}
    \end{align*}
  \item
    Сега нека $\tau \equiv \rho + \mu$ и последното правило, което сме приложили е следното:
    \begin{prooftree}
      \AxiomC{$\rho \Downarrow^{\ell-1} \alpha$}
      \RightLabel{\scriptsize{(left-or)}}
      \UnaryInfC{$\rho + \mu \Downarrow^\ell \alpha$}
    \end{prooftree}

    Знаем, че $\val{\tau}(L_1,\dots,L_n) = \val{\rho}(L_1,\dots,L_n) \cup \val{\mu}(L_1,\dots,L_n)$.
    Понеже сме приели, че $\textbf{Include}(\ell-1)$ е изпълнено, получаваме:
    \begin{align*}
      \{\alpha \in \Sigma^\star \mid \rho \Downarrow^{\ell-1} \alpha\} & \subseteq \val{\rho}(L_1,\dots,L_n)\\
                                                                       & \subseteq \val{\tau}(L_1,\dots,L_n).
    \end{align*}
    
  \item
    Сега нека $\tau \equiv \rho + \mu$ и последното правило, което сме приложили е следното:
    \begin{prooftree}
      \AxiomC{$\mu \Downarrow^{\ell-1} \alpha$}
      \RightLabel{\scriptsize{(right-or)}}
      \UnaryInfC{$\rho + \mu \Downarrow^\ell \alpha$}
    \end{prooftree}
    Този случай е аналогичен на предния.
  \end{itemize}
\end{proof}

Така доказахме теоремата за еквивалентност на денотационната и операционната семантика.

\begin{framed}
  \begin{theorem}[Теорема за еквивалентност]
    Нека $L_1,\dots,L_n$ е най-малката неподвижна точка на $\Gamma$. Тогава за всеки индекс $i$ измежду $1,\dots,n$ е изпълнено, че:
    \[L_i = \{\alpha \in \Sigma^\star \mid \tau_i \Downarrow \alpha\}.\]  
  \end{theorem}
\end{framed}

\section{Разширения}


\subsection{Езикът \texttt{REG++}}

Нека тук с $\alpha,\beta,\gamma$ да означаваме произволни регулярни изрази.
Да въведем означението $\alpha \leq \beta$, ако $\mathcal{L}(\alpha) \subseteq \mathcal{L}(\beta)$.

\begin{problem}
  Докажете, че ако $\alpha \cdot \gamma + \beta \leq \gamma$, то $\alpha^\star \beta \leq \gamma$.
\end{problem}
\begin{hint}
  Докажете, че $\alpha^n \cdot \beta \leq \gamma$ за всяко естествено число $n$.
\end{hint}

Да разгледаме следното разширение на езика \REG, който да наречем $\REG\texttt{++}$.
Термовете на езика \REGPP се формират спрямо следната абстрактна граматика:
\[\tau ::= \alpha\ |\ \alpha \cdot X\ |\ \tau_1 + \tau_2,\]
където с $\alpha$ означаваме произволен регулярен израз, т.е.
\[\alpha ::= \emptyset\ |\ \varepsilon\ |\ a\ |\ \alpha_1 + \alpha_2\ |\ \alpha_1 \cdot \alpha_2\ |\ \alpha^\star_1\]
\marginpar{Новото е, че тук позволяваме да имаме термове от вида $\alpha \cdot X$, за произволен регулярен израз $\alpha$, докато в езика \REG\ позволявахме само $a \cdot X$, за произволна буква $a$.}

\begin{problem}[Правило на Ардън \cite{arden}]\label{prob:reg:arden}
  \index{правило на Ардън}
  Да разгледаме програмата на езика \REGPP:
  \[X = \alpha \cdot X + \beta.\]
  Докажете, че най-малкото решение на системата за тази програма е езикът описан с регулярния израз $\alpha^\star \cdot \beta$. Докажете, че ако $\varepsilon \not \in \mathcal{L}(\alpha)$, то това решение е и едиствено.
\end{problem}

\begin{example}
  Да разгледаме следния краен детерминиран автомат:
  \marginpar{Означаваме с $\abs{\omega}_a$ броя на срещанията на $a$ в думата $\omega$.}
  \begin{figure}[H]
    \centering
    \begin{tikzpicture}[->,>=stealth,thick,node distance=65pt]
      \tikzstyle{every state}=[circle,minimum size=20pt,auto]
      
      \node[initial, state, accepting]         (0) {$q_1$};
      \node[state]                             (1) [right of=0]{$q_2$};
      \node[state]                             (2) [right of=1]{$q_3$};
      
      \path 
      (0) edge [loop above]   node   [above] {$b$}    (0)
      (0) edge [bend left=15] node   [above] {$a$}    (1)
      (1) edge [loop above]   node   [above] {$b$}    (1)
      (1) edge [bend left=15] node   [above] {$a$}    (2)
      (2) edge [loop above]   node   [above] {$b$}    (2)
      (2) edge [bend left=45] node   [below] {$a$}    (0);
    \end{tikzpicture}
    \caption{Автомат разпознаващ $\{\omega \in \{a,b\}^\star \mid \abs{\omega}_{a} \equiv 0 \bmod\ 3\}$}
  \end{figure}
  На този автомат съответства следната програма $R$, която е програма и на езика \REGPP:
  \marginpar{Тук $\Sigma = \{a,b\}$.}
  \begin{align*}
    & X_1 = a \cdot X_2 + b \cdot X_1 + \varepsilon \\
    & X_2 = a \cdot X_3 + b \cdot X_2\\
    & X_3 = a \cdot X_1 + b \cdot X_3.
  \end{align*}

  Да видим как можем да намерим най-малкото решение на тази система като използваме само \hyperref[prob:reg:arden]{правилото на Ардън}.
  Нека първо да разгледаме само последния ред на системата като си мислим, че имаме азбука $\Sigma_3 = \{a,b,X_1,X_2\}$. Тогава получаваме следното:
  \[X_3 = \underbrace{b}_{\alpha} \cdot X_3 + \underbrace{a \cdot X_1}_{\beta}.\]
  Според \hyperref[prob:reg:arden]{правилото на Ардън}, най-малкото решение на тази система е $b^\star a X_1$.
  Заместваме $X_3$ с този израз в горните две уравнения и получаваме следното:
  \begin{align*}
    & X_1 = b \cdot X_1 + a \cdot X_2 + \varepsilon \\
    & X_2 = b \cdot X_2 + a b^\star a \cdot X_1.
  \end{align*}
  Ако позволим да имаме регулярни изрази по ребрата на автомат, то новият автомат би бил този:

  \begin{figure}[H]
    \centering
    \begin{tikzpicture}[->,>=stealth,thick,node distance=65pt]
      \tikzstyle{every state}=[circle,minimum size=20pt,auto]
      
      \node[initial, state, accepting]         (0) {$q_1$};
      \node[state]                             (1) [right of=0]{$q_2$};
      % \node[state]                             (2) [right of=1]{$q_3$};
      
      \path 
      (0) edge [loop above]   node   [above] {$b$}    (0)
      (0) edge [bend left=35] node   [above] {$a$}    (1)
      (1) edge [loop above]   node   [above] {$b$}    (1)
      (1) edge [bend left=35] node   [below] {$ab^\star a$}    (0);
      % (2) edge [loop above]   node   [above] {$b$}    (2)
      % (2) edge [bend left=45] node   [below] {$a$}    (0);
    \end{tikzpicture}
    \caption{Автомат разпознаващ $\{\omega \in \{a,b\}^\star \mid \abs{\omega}_{a} \equiv 0 \bmod\ 3\}$}
  \end{figure}
  
  Сега разглеждаме втория ред като си мислим, че имаме азбуката $\Sigma_2 = \{a,b,X_1,X_3\}$. Тогава получаваме следното:
  \[X_2 = \underbrace{b}_{\alpha} \cdot X_2 + \underbrace{a b^\star a X_1}_{\beta}.\]
  Според \hyperref[prob:reg:arden]{правилото на Ардън}, най-малкото решение на тази система е $b^\star a b^\star a X_1$.
  Заместваме $X_2$ с израза $b^\star a b^\star a X_1$ в първия ред и получаваме:
  \[X_1 = b \cdot X_1 + ab^\star a b^\star a \cdot X_1  + \varepsilon.\]
  Оттук имаме, че:
  \[X_1 = \underbrace{(b + ab^\star a b^\star a)}_{\alpha} \cdot X_1 + \underbrace{\varepsilon}_{\beta}.\]
  Така получаваме, че най-малкото решение на първия ред е $(b + ab^\star a b^\star a)^\star$.
  Заключаваме, че най-малкото решение на системата е тройката от езици:
  \marginpar{Тук един от проблемите е, че изобщо не е ясно дали тези регулярни изрази са ,,оптимални'' в някакъв смисъл.}
  \begin{align*}
    \hat{L}_1 & = (b + ab^\star a b^\star a)^\star\\
    \hat{L}_2 & = b^\star a b^\star a (b + ab^\star a b^\star a)^\star\\
    \hat{L}_3 & = b^\star a (b + ab^\star a b^\star a)^\star.
  \end{align*}
  
  Да напомним, че вече намерихме следното представяне на най-малкото решение на системата:
  \begin{align*}
    \hat{L}_1 & = \{\omega \in \{a,b\}^\star \mid \abs{\omega}_{a} \equiv 0 \bmod\ 3\}\\
    \hat{L}_2 & = \{\omega \in \{a,b\}^\star \mid \abs{\omega}_{a} \equiv 2 \bmod\ 3\}\\
    \hat{L}_3 & = \{\omega \in \{a,b\}^\star \mid \abs{\omega}_{a} \equiv 1 \bmod\ 3\}.
  \end{align*}
\end{example}


\subsection{Езикът \CFG}
\index{език!безконтекстен}

Термовете на езика \CFG се описват със следната абстрактна граматика:
\[\tau ::= \varepsilon\ |\ a\ |\ X\ |\ \tau_1 \cdot \tau_2\ |\ \tau_1 + \tau_2.\]
Нека оттук нататък да приемем, че сме фиксирали една програма $\vv{G}$ на езика \CFG:
\begin{SystemEq}
  X_1 & = & \tau_1[X_1,\dots,X_n]\\
  & \vdots & \\
  X_n & = & \tau_n[X_1,\dots,X_n].  
\end{SystemEq}

\subsubsection*{Операционна семантика}

С индукция по $\ell$, дефинираме релацията $\tau \opsemCFG{\ell} \alpha$,
където $\tau$ е терм на езика \CFG със свободни променливи измежду $X_1,\dots, X_n$.

\begin{framed}
  \begin{figure}[H]
    \begin{subfigure}[b]{0.5\textwidth}
      \begin{prooftree}
        \AxiomC{}
        \LeftLabel{\scriptsize{(eps)}}
        \UnaryInfC{$\varepsilon \opsemCFG{0} \varepsilon$}
      \end{prooftree}
    \end{subfigure}
    ~
    \begin{subfigure}[b]{0.5\textwidth}
      \begin{prooftree}
        \AxiomC{}
        \RightLabel{\scriptsize{(letter)}}
        \UnaryInfC{$a \opsemCFG{1} a$}
      \end{prooftree}
    \end{subfigure}

    \vspace{10pt}

    \begin{subfigure}[b]{0.5\textwidth}
      \begin{prooftree}
        \AxiomC{$\tau_1 \opsemCFG{\ell} \alpha$}
        \LeftLabel{\scriptsize{(left-or)}}
        \UnaryInfC{$\tau_1 + \tau_2 \opsemCFG{\ell+1} \alpha$}
      \end{prooftree}
    \end{subfigure}
    ~
    \begin{subfigure}[b]{0.5\textwidth}
      \begin{prooftree}
        \AxiomC{$\tau_2 \opsemCFG{\ell} \alpha$}
        \RightLabel{\scriptsize{(right-or)}}
        \UnaryInfC{$\tau_1 + \tau_2 \opsemCFG{\ell+1} \alpha$}
      \end{prooftree}
    \end{subfigure}

    \vspace{10pt}
    
    \begin{subfigure}[b]{0.5\textwidth}
      \begin{prooftree}
        \AxiomC{$\tau_i \opsemCFG{\ell} \alpha$}
        \LeftLabel{\scriptsize{(rec)}}
        \UnaryInfC{$X_i \opsemCFG{\ell+1} \alpha$}
      \end{prooftree}
    \end{subfigure}
    ~
    \begin{subfigure}[b]{0.5\textwidth}
      \begin{prooftree}
        \AxiomC{$\tau_1 \opsemCFG{\ell_1} \alpha_1$}
        \AxiomC{$\tau_2 \opsemCFG{\ell_2} \alpha_2$}
        \RightLabel{\scriptsize{(concat)}}
        \BinaryInfC{$\tau_1 \cdot \tau_2 \opsemCFG{\ell_1+\ell_2+1} \alpha_1\cdot\alpha_2$}
      \end{prooftree}
    \end{subfigure}
    \caption{Правила на операционната семантика на езика \CFG.}
  \end{figure}
\end{framed}
Ще пишем $\tau \Downarrow_{\vv{P}} \alpha$ точно тогава, когато съществува $\ell$, за което $\tau \opsemCFG{\ell} \alpha$.


\subsubsection*{Денотационна семантика}

За всеки терм $\tau[X_1,\dots,X_n]$ на езика \CFG дефинираме изображението
% \marginpar{Тук за променливите използваме главни букви за да се сещаме, че те приемат стойности множества от думи.}
\[\val{\tau}: \mathcal{P}(\Sigma^\star)^n \to \mathcal{P}(\Sigma^\star)\]
 по следния начин:
\begin{itemize}
\item 
  $\val{\varepsilon}(L_1,\dots,L_n) = \{\varepsilon\}$.
\item 
  $\val{a}(L_1,\dots,L_n) = \{a\}$.
\item 
  $\val{X_j}(L_1,\dots,L_n) = L_j$.
\item
  $\val{\tau_1 + \tau_2}(L_1,\dots,L_n) = \val{\tau_1}(L_1,\dots,L_n) \cup \val{\tau_2}(L_1,\dots,L_n)$.
\item
  $\val{\tau_1 \cdot \tau_2}(L_1,\dots,L_n) = \val{\tau_1}(L_1,\dots,L_n) \cdot \val{\tau_2}(L_1,\dots,L_n)$.
\end{itemize}

\begin{problem}
  Докажете, че за всеки терм $\tau$ на езика \REG имаме, че $\val{\tau}$ е непрекъснато изображение в областта на Скот
  $\mathcal{S} = ( \mathcal{P}(\Sigma^\star),\subseteq, \emptyset)$.
\end{problem}

\marginpar{Тук \hyperref[prob:reg:arden]{правилото на Ардън} не работи!}

\begin{example}
  Да разгледаме следната програма на езика \CFG.
  \begin{align*}
    X & = \overbrace{a\cdot X \cdot c + Y}^{\tau_1[X,Y]}\\
    Y & = \underbrace{b \cdot Y \cdot c + \varepsilon}_{\tau_2[X,Y]}.
  \end{align*}
  На тази програма съответства системата от непрекъснати изображения:
  \begin{align*}
    & L = \val{\tau_1}(L,M) = \{a\}\cdot L_1 \cdot \{c\} \cup L_2\\
    & M = \val{\tau_2}(L_1,L_2) = \{b\} \cdot L_2 \cdot \{c\} \cup \{\varepsilon\}.
  \end{align*}
  Да намерим най-малкото решение $(\hat{L},\hat{M})$ на тази система.
  \begin{align*}
    L_0 & = \emptyset\\
    M_0 & = \emptyset\\
    L_1 & = \val{\tau_1}(L_0,M_0)\\
        & = \{a\}\cdot \emptyset \cdot \{c\} \cup \emptyset = \emptyset\\
    M_1 & = \val{\tau_2}(L_0,M_0)\\
        & = \{b\} \cdot \emptyset \cdot \{c\} \cup \{\varepsilon\} = \{\varepsilon\}\\
    L_2 & = \{a\}\cdot L_1 \cdot \{c\} \cup M_1 = \{\varepsilon\}\\
    M_2 & = \{b\} \cdot M_1 \cdot \{c\} \cup M_1\\
        & = \{b\} \cdot \{\varepsilon\} \cdot \{c\} \cup \{\varepsilon\} = \{bc,\varepsilon\}\\
    L_3 & = \{a\}\cdot L_2 \cdot \{c\} \cup M_2 \\
        & = \{ac,bc,\varepsilon\}\\
        & = \{a^mb^nc^k \mid 2 > m+n = k\}\\
    M_3 & = \{b\} \cdot M_2 \cdot \{c\} \cup M_2\\
        & = \{b\} \cdot \{bc,\varepsilon\} \cdot \{c\} \cup \{bc,\varepsilon\}\\
        & = \{b^nc^n \mid 3 > n\}.
  \end{align*}

  \marginpar{\writedown Довършете доказателството!}
  Сега вече сме готови да формулираме нашето индукционно предположение:
  \begin{align*}
    L_t & = \{a^mb^nc^k \mid t-1 > m+n = k\}\\
    M_t & = \{b^nc^n \mid t > n \}.
  \end{align*}

  Докажете, че $\hat{L} = \{a^mb^nc^k \mid m+n = k\}$ и $\hat{M} = \{b^nc^n \mid n \in \Nat\}$.
  
\end{example}


\begin{problem}[Теорема за характеризация]
  Да разгледаме системата от непрекъснати оператори
  \begin{SystemEq}
    L_1 & = & \val{\tau_1}(L_1,\dots,L_n)\\
    & \vdots & \\
    L_n & = & \val{\tau_n}(L_1,\dots,L_n).
  \end{SystemEq}

  Знаем, че тя притежава най-малко решение $(\hat{L}_1,\dots,\hat{L}_n)$.
  Докажете, че всеки от езиците $\hat{L}_i$ е безконтекстен.

  Докажете, че всеки безконтекстен език е елемент от най-малкото решение 
  на някоя система от непрекъснати изображения от горния вид.
\end{problem}


\subsubsection*{Еквивалентност}

\begin{problem}
  Докажете, че за всеки индекс $r$ и всеки терм $\tau$ на езика \CFG с променливи измежду $X_1,\dots,X_n$ е изпълнено, че:
  \[\val{\tau}(L^r_1,\dots,L^r_n) \subseteq \{\alpha \in \Sigma^\star \mid \tau \Downarrow_{\vv{P}} \alpha\}.\]
  Заключете, че ако $L_1,\dots,L_n$ е най-малкото решение на непрекъснатото изображение $\Gamma \df \val{\tau_1}\times\cdots\times\val{\tau_n}$, то
  за произволен терм $\tau$ с променливи измежду $X_1,\dots,X_n$ е изпълнено, че:
  \[\val{\tau}(L_1,\dots,L_n) \subseteq \{\alpha \in \Sigma^\star \mid \tau \Downarrow_{\vv{P}} \alpha\}.\]
\end{problem}

\begin{problem}
  Нека $L_1,\dots,L_n$ е неподвижна точка на $\Gamma \df \val{\tau_1}\times \cdots \times \val{\tau_n}$.
  Докажете, че за всеки терм $\tau$ на езика \CFG с променливи измежду $X_1,\dots,X_n$ е  изпълнено, че:
  \[\{\alpha \in \Sigma^\star \mid \tau \Downarrow \alpha\} \subseteq \val{\tau}(L_1,\dots,L_n).\]

  Заключете, че ако $L_1,\dots,L_n$ е най-малката неподвижна точка на $\Gamma$, то за всеки индекс $i$ измежду $1,\dots,n$ е изпълнено, че:
  \[L_i = \{\alpha \in \Sigma^\star \mid \tau_i \Downarrow_{\vv{P}} \alpha\}.\]
\end{problem}

%%% Local Variables:
%%% mode: latex
%%% TeX-master: "../sep"
%%% End:


%%% Local Variables:
%%% mode: latex
%%% TeX-master: "../sep"
%%% End:

% \include{domains-extra/domains-extra} 
% \newcommand{\REC}{{\bf REC}}

\chapter{Езикът \FUN}\label{ch:rec}
\marginpar{В тази глава до голяма степен следваме \cite[Глава 9]{winskel}.}

\marginpar{Основно следваме \cite[Глава 9]{winskel}}
\section{Синтаксис}
Ще разглеждаме един много прост език за функционално програмиране.
\begin{itemize}
\item
  \index{константа}
  \marginpar{За разлика от \cite{ditchev-soskov}, няма да въвеждаме термове от тип $\BB$.}
  \marginpar{Константите не са числа! Константите са синтактични обекти, докато числата са семантични обекти. Обърнете внимание, че в езика имаме константи за всяко естествено число, но в езика нямаме константа за $\bot$. Това е една разлика с \texttt{хаскел}, където в езика има константа за $\bot$ и тя е означена с \texttt{undefined}.}
  константи $\vv{n}$, за всяко число $n \in \Nat$;
  \Stefan{Да ги нарека обектови константи. По тази логика, операциите стават функционални константи}
\item
  \index{променлива!обектова}
  \index{променлива!нулев тип}
  \marginpar{Удобно е в нашия език още на синтактично ниво да правим разлика между двата типа променливи, които имаме в езика.}
  изброимо много променливи от тип 0 (или обектови променливи) $\vv{x}, \vv{y}, \vv{z}, \dots$, евентуално с индекси;
\item
  \index{променлива!функционална}
  изброимо много променливи от тип 1 (или функционални променливи) $\vv{f},\vv{g},\vv{h},\dots$, евентуално с индекси. 
  Формално погледнато, трябва на всяка функционална променлива $\vv{f}$
  да съпоставим число - брой аргументи, които приема. Нека да означим с $\sharp\vv{f}$ броя аргументи на $\vv{f}$.
  Обикновено броят аргументи на $\vv{f}$ ще е ясен от контекста.
\item
  \index{терм}
  Термовете, които обикновено ще означаваме с $\tau$, в езика {\bf REC} имат следния синтаксис:
  \marginpar{Тук $m = \sharp\vv{f}$}
  \marginpar{Граматиката е във форма на Бекус-Наур.}
  \[\tau ::= \vv{n} \mid \vv{x} \mid \tau + \tau \mid \tau\ \vv{==}\ \tau \mid \ifelse{\tau}{\tau}{\tau} \mid \vv{f}(\underbrace{\tau,\dots,\tau}_{m}).\]
\item
  Ще записваме $\tau[\vv{x}_1,\dots,\vv{x}_n,\vv{f}_1,\dots,\vv{f}_k]$, когато искаме да означим, че променливите
  на терма $\tau$ са {\em измежду} посочените.
\item
  Ще наричаме един терм {\bf функционален}, ако той не съдържа обектови променливи.
  Обикновено ще означаваме функционалните термове с $\mu$, а произволни термове с $\tau$.
\item
  Най-удобно е да си мислим за един терм като за дърво.
\item
  Ще пишем $\tau_1 \equiv \tau_2$, ако термовете $\tau_1$ и $\tau_2$ представляват едни и същи дървета.
  Например, $2+3 = 3+2$, но $2+3 \not\equiv 2+3$.
\item
  С $\tau[\vv{x}/\mu]$ ще означаваме терма получен от $\tau$, в който всяко срещане на обектовата променлива $\vv{x}$
  е заменена с функционалния терм $\mu$. Можем да дадем формална дефиниция с индукция по построението на термовете:
  \begin{itemize}
  \item
    Ако $\tau \equiv \vv{n}$, то
    \[\vv{n}[\vv{x}/\mu] \equiv \vv{n}.\]
  \item
    Ако $\tau \equiv \vv{x}$, то
    \[\vv{x}[\vv{x}/\mu] \equiv \mu.\]
  \item
    Ако $\tau \equiv \vv{y}$ и $\vv{y} \not\equiv \vv{x}$, то
    \[\vv{y}[\vv{x}/\mu] \equiv \vv{y}.\]
  \item
    Ако $\tau \equiv \tau_1 + \tau_2$, то
    \[\tau[x/\mu] \equiv \tau_1[x/\mu] + \tau_2[x/\mu].\]
  \item
    Ако $\tau \equiv \tau_1\ \vv{==}\ \tau_2$, то
    \[\tau[x/\mu] \equiv \tau_1[x/\mu]\ \vv{==}\ \tau_2[x/\mu].\]
  \item
    Ако $\tau \equiv \ifelse{\tau_1}{\tau_2}{\tau_3}$, то
    \[\tau[x/\mu] \equiv \ifelse{\tau_1[x/\mu]}{\tau_2[x/\mu]}{\tau_3[x/\mu]}.\]
  \item
    Ако $\tau \equiv \vv{f}(\tau_1,\dots,\tau_m)$, то
    \[\tau[x/\mu] \equiv \vv{f}(\tau_1[x/\mu], \dots, \tau_m[x/\mu]).\]
  \end{itemize}  
\end{itemize}

\marginpar{\cite[стр. 141]{winskel}}

\index{рекурсивна програма}
Една {\bf рекурсивна програма} $\vv{P}$ на езика {\bf REC} има следния общ вид:
\marginpar{Една програма е просто текст със специален формат. Важният въпрос е каква функция (семантика) отговаря на този текст (синтаксис)}
\marginpar{Може да си мислите, че $\vv{f}_1$ е $\vv{main}$ функцията на нашата програма}
\begin{align*}
  \vv{P} = 
  \begin{cases}
    & \vv{f}_1(\vv{x}_1,\dots,\vv{x}_{m_1}) = \tau_1[\vv{x}_1,\dots,\vv{x}_{m_1},\vv{f}_1,\dots,\vv{f}_k]\\
    & \vdots\\
    & \vv{f}_k(\vv{x}_1,\dots,\vv{x}_{m_k}) = \tau_k[\vv{x}_1,\dots,\vv{x}_{m_k},\vv{f}_1,\dots,\vv{f}_k]
  \end{cases}
\end{align*}

В такъв случай казваме, че термът $\tau_i$ задава {\em дефиницията} на фунционалната променлива $\vv{f}_i$.

\begin{example}
  \label{ex:minus}
  Да разгледаме програмата $\vv{P}$ на езика {\bf REC}:
  \begin{haskellcode}
h(x) = f(x, 1)
f(x,y) = if x == y then 0 
           else f(x, y+1) + 1
  \end{haskellcode}
  Да положим
  \begin{align*}
    & \tau_1[\vv{x},\vv{h},\vv{f}] \dfff \vv{f}(\vv{x},\vv{1})\\
    & \tau_2[\vv{x},\vv{y},\vv{h},\vv{f}] \dfff \ifelse{\vv{x == y}}{\vv{0}}{\vv{f(x,y+1) + 1}}.
  \end{align*}
  Тогава програмата $\vv{P}$ приема следния вид:
  \begin{align*}
    & \vv{h}(\vv{x}) = \tau_1[\vv{x},\vv{h},\vv{f}]\\
    & \vv{f}(\vv{x},\vv{y}) = \tau_2[\vv{x},\vv{y},\vv{h},\vv{f}].
  \end{align*}
\end{example}

%%% Local Variables:
%%% mode: latex
%%% TeX-master: "../sep"
%%% End:


\section{Денотационна семантика}

\subsection{Термални оператори}\label{subsect:rec:term-value}

Нека първо да дефинираме следните {\em изображения}
\begin{align*}
  & \texttt{plus} : \Nat^2_\bot \to \Nat_\bot\text{, където}\\
  & \texttt{plus}(a,b) =
    \begin{cases}
      a+b, & \text{ако }a,b \in \Nat\\
      \bot, & \text{ако }\bot \in \{a,b\}
    \end{cases}\\
  & \texttt{eq} : \Nat^2_\bot \to \Nat_\bot\text{, където}\\
  & \texttt{eq}(a,b) =
    \begin{cases}
      1, & \text{ако }a = b\ \&\ a,b \in \Nat\\
      0, & \text{ако }a \neq b\ \&\ a,b \in \Nat\\
      \bot, & \text{ако }\bot \in \{a,b\}
    \end{cases}
\end{align*}

\marginpar{Спестяваме си труда от въвеждането на булевия тип променливи}
\marginpar{Озн. $\Nat^+ \df \Nat \setminus \{0\}$}

\begin{problem}\label{prob:rec:if:continuous}
  Докажете, че изображенията $\texttt{plus}$ и $\texttt{eq}$ са непрекъснати.
\end{problem}
\begin{hint}
  Понеже $\Mon{\Nat^n_\bot}{\Nat_\bot} = \Cont{\Nat^n_\bot}{\Nat_\bot}$,
  достатъчно е да докажете, че изображенията са монотонни.
\end{hint}

\index{денотационна семантика!по име}
\marginpar{\cite[стр. 155]{winskel}}
\index{терм!стойност}
\index{оператор!термален}
За всеки терм $\tau[\vv{x}_1,\dots,\vv{x}_{n},\vv{f}_1,\dots,\vv{f}_k]$,
ще разгледаме изображението със сигнатура
\[\val{\tau}:\Cont{\Nat^{m_1}_\bot}{\Nat_\bot}\times\cdots\times\Cont{\Nat^{m_k}_\bot}{\Nat_\bot} \to \Mapping{\Nat^{n}_\bot}{\Nat_\bot},\]
което ще дефинираме с индукция по построението на термовете.
Изображенията от вида $\val{\tau}$ ще наричаме {\bf термални оператори}.

\begin{itemize}
\item
  ако $\tau \equiv \vv{c}$, за някоя константа, то 
  \[\val{\vv{c}}(\ov{\varphi})(\ov{a}) \df c.\]
\item
  ако $\tau \equiv \vv{x}_i$, за някоя обектова променлива, то 
  \[\val{\vv{x}_i}(\ov{\varphi})(\ov{a}) \df a_i.\]
\item
  ако $\tau \equiv \tau_1 + \tau_2$, то
  \[\val{\tau_1 + \tau_2}(\ov{\varphi})(\ov{a}) \df \texttt{plus}(\val{\tau_1}(\ov{\varphi})(\ov{a}), \val{\tau_2}(\ov{\varphi})(\ov{a})).\]
\item
  \marginpar{От \Problem{rec:if:continuous} знаем, че изображенията $\texttt{plus}$, $\texttt{eq}$ са непрекъснати.}
  ако $\tau \equiv \tau_1\ \vv{==}\ \tau_2$, то
  \[\val{\tau_1\ \vv{==}\ \tau_2}(\ov{\varphi})(\ov{a}) \df \texttt{eq}(\val{\tau_1}(\ov{\varphi})(\ov{a}), \val{\tau_2}(\ov{\varphi})(\ov{a})).\]
\item
  \marginpar{За $\texttt{if}$ вижте \Def{if}. От \Problem{if} знаем, че $\texttt{if}$ е непрекъснато изображение.}
  ако $\tau \equiv\ \ifelse{\tau_1}{\tau_2}{\tau_3}$, то
  \[\val{\ifelse{\tau_1}{\tau_2}{\tau_3}}(\ov{\varphi})(\ov{a}) \df \texttt{if}(\val{\tau_1}(\ov{\varphi})(\ov{a}), \val{\tau_2}(\ov{\varphi})(\ov{a}), \val{\tau_3}(\ov{\varphi})(\ov{a})).\]
\item
  ако $\tau \equiv \vv{f}_i(\tau_1,\dots,\tau_{m_i})$, то
  \marginpar{Термовете $\tau_j$ може да имат различен брой променливи. Ако се наложи, разширяваме ги с фиктивни променливи}
  \[\val{\vv{f}_i(\tau_1,\dots,\tau_{m_i})}(\ov{\varphi})(\ov{a}) \df \varphi_i(\val{\tau_1}(\ov{\varphi})(\ov{a}),\dots,\val{\tau_{m_i}}(\ov{\varphi})(\ov{a})).\]
\end{itemize}

\begin{example}
  Да разгледаме следния терм:
  \[\tau[\vv{x},\vv{y},\vv{z}] \df \ifelse{\vv{x}\ \vv{==}\ 5}{\vv{y}}{\vv{z}}.\]
  Да видим каква е негова стойност при произволни стойности $a,b,c \in \Nat_\bot$:
  \begin{align*}
    \val{\tau}(a,b,c) & =
                        \begin{cases}
                          \val{\vv{y}}(a,b,c), & \text{ако }\val{x \ \vv{==}\ 5}(a,b,c) \in \Nat^+\\
                          \val{\vv{z}}(a,b,c), & \text{ако }\val{x \ \vv{==}\  5}(a,b,c) = 0\\
                          \bot,                & \text{ако }\val{x\ \vv{==}\ 5}(a,b,c) = \bot
                        \end{cases}\\
                      & =
                        \begin{cases}
                          b,    & \text{ако }a = 5\\
                          c,    & \text{ако }a \in \Nat\ \&\ a \neq 5\\
                          \bot, & \text{ако }a = \bot.
                        \end{cases}
  \end{align*}
\end{example}
Нека сега да видим следния пример на хаскел.
\begin{haskellcode}
ghci> let f(x) = if x == undefined then 0 else 1
ghci> f(2)
*** Exception: Prelude.undefined
\end{haskellcode}
Това означава, че функцията на хаскел \texttt{==} е точна, т.е. не можем да сравняваме с $\bot$, което съответства на нашата
денотационна семантика, т.е. функцията $\texttt{eq}$.

\begin{framed}
  \begin{lemma}[Лема за замяната]
    \label{lem:rec:substitution}
    Да разгледаме терма $\tau[\vv{x}_1,\dots,\vv{x}_n,\vv{f}_1,\dots,\vv{f}_{k}]$ и {\em функционалните} термове 
    $\mu_1[\vv{f}_1,\dots,\vv{f}_k],\dots, \mu_n[\vv{f}_1,\dots,\vv{f}_k]$.
    Тогава
    \[\val{\tau[\bar{\vv{x}}/\bar{\mu}]}(\bar{\varphi}) = \val{\tau}(\ov{\varphi})(\val{\mu_1}(\bar{\varphi}),\dots,\val{\mu_n}(\bar{\varphi}))\]
    за произволни $\varphi_i \in \Cont{\Nat^{m_i}_\bot}{\Nat_\bot}$, за $i = 1, \dots, k$.
  \end{lemma}
\end{framed}
\begin{proof}
  Доказателството се провежда с {\em индукция по построението на терма $\tau$.}
  \marginpar{Понеже нямаме обектови променливи в $\mu_1,\dots,\mu_n$, то е очевидно как правим замяната.}
  \marginpar{Substitution Lemma, \cite[стр. 149]{winskel}.}
  \marginpar{Това твърдение е без д-во в \cite[стр. 188]{ditchev-soskov}.}
  \begin{itemize}
  \item
    \marginpar{$\vv{c}$ е константа, докато $c \in \Nat_\bot$}
    Нека да започнем с най-лесния случай.
    Нека $\tau \equiv \vv{c}$, за някоя константа.
    Тогава $\tau[\ov{\vv{x}}/\ov{\mu}] \equiv \vv{c}$.
    Това означава, че 
    \[\val{\tau[\ov{\vv{x}}/\ov{\mu}]}(\ov{\varphi}) = \val{\vv{c}}(\ov{\varphi}) \df c.\]
    От друга страна, имаме също, че 
    \[\val{\vv{c}}(\ov{\varphi})(\val{\mu_1}(\ov{\varphi}),\dots,\val{\mu_n}(\ov{\varphi})) \df c.\]
  \item
    Нека $\tau \equiv \vv{x}_i$. Тогава $\vv{x}_i[\varsx/\bar{\mu}] \equiv \mu_i$
    и следователно 
    \[\val{\vv{x}_i[\varsx/\bar{\mu}]}(\bar{\varphi}) = \val{\mu_i}(\bar{\varphi}).\]
    От друга страна, по дефиниция на стойност на терм, 
    \[\val{\vv{x}_i}(\ov{\varphi})(\val{\mu_1}(\bar{\varphi}),\dots,\val{\mu_n}(\bar{\varphi})) = \val{\mu_i}(\bar{\varphi}).\]
  \item
    Нека $\tau \equiv \tau_1 + \tau_2$.
    От {\bf И.П.} имаме, че за $j = 1,2$ е изпъленено следното:
    \[\val{\tau_j[\varsx/\ov{\mu}]}(\ov{\varphi}) = \val{\tau_j}(\ov{\varphi})(\underbrace{\val{\mu_1}(\ov{\varphi})}_{b_1},\dots,\underbrace{\val{\mu_n}(\ov{\varphi})}_{b_n}).\]
    Тогава
    \begin{align*}
      \val{\tau[\varsx/\ov{\mu}]}(\ov{\varphi}) & = \val{\tau_1[\varsx/\ov{\mu}] + \tau_2[\varsx/\ov{\mu}]}(\ov{\varphi})\\
                                                & \df \texttt{plus}(\val{\tau_1[\varsx/\ov{\mu}]}(\ov{\varphi}), \val{\tau_2[\varsx/\ov{\mu}]}(\ov{\varphi}))\\
                                                % & = \texttt{plus}(\val{\tau_1}(\ov{\varphi})(\val{\mu_1}(\ov{\varphi}),\dots,\val{\mu_n}(\ov{\varphi})),\val{\tau_2}(\ov{\varphi})(\val{\mu_1}(\ov{\varphi}),\dots,\val{\mu_n}(\ov{\varphi}))) & \comment{\text{от {\bf И.П.}}}\\
                                                & = \texttt{plus}(\val{\tau_1}(\ov{\varphi})(b_1,\dots,b_n),\val{\tau_2}(\ov{\varphi})(b_1,\dots,b_n)) & \comment{\text{от {\bf И.П.}}}\\
                                                & \df \val{\tau}(\ov{\varphi})(b_1,\dots,b_n)\\
                                                & = \val{\tau}(\ov{\varphi})(\val{\mu_1}(\bar{\varphi}),\dots,\val{\mu_n}(\ov{\varphi})). & \comment{b_j \df \val{\mu_j}(\ov{\varphi})}
    \end{align*}
  \item
    \marginpar{\writedown Докажете сами останалите два случая. Те не крият изненади.}
    Нека $\tau \equiv \tau_1\ \vv{==}\  \tau_2$.
  \item
    Нека $\tau \equiv \ifelse{\tau_0}{\tau_1}{\tau_2}$.
  \item 
    Нека $\tau \equiv \vv{f}_i(\tau_1,\dots,\tau_{m_i})$.
    Имаме, че 
    \begin{equation}
      \label{eq:1}
      \tau[\varsx/\bar{\mu}] \equiv \vv{f}_i(\tau_1[\varsx/\bar{\mu}],\dots,\tau_{m_i}[\varsx/\bar{\mu}]).
    \end{equation}
    Нека за улеснение да означим $b_i \df \val{\mu_i}(\bar{\varphi})$, за $i = 1,\dots,n$.
    Прилагаме {\bf И.П.} за термовете $\tau_1,\dots,\tau_{m_i}$ и получаваме за $j = 1, \dots, m_i$,
    \begin{align*}
      \val{\tau_j[\varsx /\bar{\mu}]}(\bar{\varphi}) & = \val{\tau_j}(\ov{\varphi})(\underbrace{\val{\mu_1}(\bar{\varphi})}_{b_1},\dots,\underbrace{\val{\mu_{n}}(\bar{\varphi})}_{b_{n}}) & \comment{\text{от {\bf И.П.}}}\\
      & = \val{\tau_j}(\ov{\varphi})(b_1,\dots,b_{n}) & \comment{b_i \df \val{\mu_i}(\bar{\varphi})}.
    \end{align*}
    Следователно,
    \begin{align*}
      \val{\tau[\varsx/\ov{\mu}]}(\ov{\varphi}) & = \val{\vv{f}_i(\tau_1[\varsx/\ov{\mu}],\dots,\tau_{m_i}[\varsx/\ov{\mu}])}(\ov{\varphi}) & \comment{\text{от (\ref{eq:1})}}\\
                                                & \df \varphi_i(\val{\tau_1[\varsx/\ov{\mu}]}(\ov{\varphi}),\dots,\val{\tau_{m_i}[\varsx/\ov{\mu}]}(\ov{\varphi}))\\
                                                & = \varphi_i(\val{\tau_1}(\ov{\varphi})(\ov{b}),\dots,\val{\tau_{m_i}}(\ov{\varphi})(\ov{b})) & \comment{\text{от {\bf И.П.}}} \\
                                                & \df \val{\vv{f}_i(\tau_1,\dots,\tau_{m_i})}(\ov{\varphi})(\ov{b})\\
                                                & = \val{\tau}(\ov{\varphi})(\ov{b}) & \comment{\tau \equiv \vv{f}_i(\tau_1,\dots,\tau_{m_i})}\\
      & = \val{\tau}(\ov{\varphi})(\val{\mu_1}(\ov{\varphi}),\dots,\val{\mu_n}(\ov{\varphi})) & \comment{b_i \df \val{\mu_i}(\ov{\varphi})}.
    \end{align*}
  \end{itemize}
\end{proof}

\begin{remark}
  В частния случай, когато функционалните термове $\mu_1,\dots, \mu_n$ са константите $\vv{c}_1, \dots, \vv{c}_n$, получаваме, че 
  \[\val{\tau[\ov{\vv{x}}/\ov{\vv{c}}]}(\ov{\varphi}) = \val{\tau}(\ov{\varphi})(\ov{c}).\]  
\end{remark}

%%% Local Variables:
%%% mode: latex
%%% TeX-master: "../sep"
%%% End:


% % \subsection{Термални оператори}

% Вече дефинирахме как на всеки терм $\tau[\vv{x}_1,\dots,\vv{x}_n,\vv{f}_1,\dots,\vv{f}_k]$
% съпоставяме изображението $\val{\tau}$ със сигнатура
% \[\val{\tau}:\Mapping{\Nat^{m_1}_\bot}{\Nat_\bot}\times\cdots\times\Mapping{\Nat^{m_k}_\bot}{\Nat_\bot} \to \Mapping{\Nat^n_\bot}{\Nat_\bot}\]

% Едно от основните свойства, които трябва да притежават тези изображения е да бъдат {\em непрекъснати} относно областите на Скот,
% за които са дефинирани.
% Причината за това е, че искаме да дефинираме семантиката на една програма като използваме най-малкото решение 
% на система от такива оператори, а според \hyperref[th:knaster-tarski]{Теоремата на Клини}, за да можем да направим това, трябва да работим
% с непрекъснати оператори. Следващият пример ни показва, че трябва да сме по-внимателни върху какви елементи разглеждаждаме тези изображения.

% \begin{example}
%   \label{ex:non-continuous}
%   Да разгледаме терма 
%   \[\tau[\vv{x},\vv{f},\vv{g}] \dfff \vv{f}(\vv{g}(\vv{x})),\]
%   и следните две изображения от тип $\Mapping{\Nat_\bot}{\Nat_\bot}$:
%   \marginpar{Обърнете внимание, че $\varphi$ не е монотонна функция. Тя дори не е точна. 
%     Функцията $\psi$ не е точна, но е монотонна. Знаем, че $\varphi$ не е непрекъсната}
%   \begin{align*}
%     & \varphi(x) \dff
%     \begin{cases}
%       42,   & \text{ако }x = \bot\\
%       \bot, & \text{ако }x \in \Nat
%     \end{cases}\\
%     & \psi(x) \dff 42 \text{, за всяко }x \in \Nat_\bot.
%   \end{align*}
%   Лесно се вижда, че $\pair{\varphi,\varphi} \sqsubseteq \pair{\varphi,\psi}$, но
%   \[\val{\tau}(\varphi,\varphi) \not\sqsubseteq \val{\tau}(\varphi,\psi),\]
%   защото за произволно $a \in \Nat$,
%   \begin{align*}
%     \val{\tau}(\varphi,\varphi)(a) & = \varphi(\varphi(a)) & \comment{\text{стойност на терм}}\\
%                                  & = 42  \\
%                                  & \not\sqsubseteq \bot & \comment{\text{плоска наредба}}\\
%                                  & = \varphi(\psi(a)) & \comment{\text{защото }\psi(a) \neq \bot}\\
%                                  & = \val{\tau}(\varphi,\psi)(a) & \comment{\text{от деф. на }\tau}\\
%   \end{align*}

%   \marginpar{Знаем, че всеки непрекъснат оператор е монотонен. Щом $\val{\tau}$ не е монотонен, то той със сигурност не е непрекъснат.}
%   Това означава, че за конкретния терм $\tau$, изображението $\val{\tau}$ не е монотонно, откъдето следва, че то не е непрекъснато.
% \end{example}

% % Горният пример ни казва, че за $\val{\tau}$, дефинирани върху произволни елементи от $\Mapping{\Nat^{m_i}_\bot}{\Nat_\bot}$
% % {\em не можем да гарантираме свойството непрекъснатост}. Ще видим, че ако се ограничим само до непрекъснатите изображения,
% % то тогава операторите ще бъдат непрекъснати.
% Всъщност на нас е необходимо да можем да подаваме като аргументи на $\val{\tau}$ само такива $\varphi$,
% които можем да дефинираме на езика \REC.
% Според семантиката на термовете, която дефинирахме по-горе, не е ясно дали можем да дефинираме терм на езика \REC, чиято семантика да бъде изображението $\varphi$. Естествено е да разгледаме терма 
% \[\tau[\vv{x}] \dfff \ifelse{\vv{x}\ \vv{==}\ \bot}{42}{\bot}.\]
% \marginpar{Това {\em не} е доказателство, че не можем да дефинираме функцията $\varphi$ на езика \REC. Тук ние твърдим само, че очевидният опит за дефиниция на $\varphi$ се проваля. По-нататък ще видим, че наистина не може да дефинираме $\varphi$ на нашия език, защото тя не е непрекъсната функция.}
%  Каква е семантиката на този терм? Следваме дефиницията на стойност на терм, за произволно $a \in \Nat_\bot$, и получаваме:
%  \begin{align*}
%    \val{\tau}(a) & =
%                    \begin{cases}
%                      42,   & \text{ако }\val{\vv{x}\ \vv{==}\ \bot}(a) \in \Nat^+\\
%                      \bot, & \text{ако }\val{\vv{x}\ \vv{==}\ \bot}(a) = 0\\
%                      \bot, & \text{ако }\val{\vv{x}\ \vv{==}\ \bot}(a) = \bot\\
%                    \end{cases}\\
%                  & = 
%                    \begin{cases}
%                      42,   & \text{ако }\underbrace{\val{\vv{x}}(a)}_{\in\Nat} = \underbrace{\val{\bot}(a)}_{\in \Nat}\\
%                      \bot, & \text{ако }\underbrace{\val{\vv{x}}(a)}_{\in\Nat} \neq \underbrace{\val{\bot}(a)}_{\in \Nat}\\
%                      \bot, & \text{ако }\val{\vv{x}}(a) = \bot\text{ или }\val{\bot}(a) = \bot\\
%                    \end{cases}\\
%                  & = \bot.
%  \end{align*}
 
% Да видим, какво ще стане ако преведем директно горния пример на \vv{хаскел}.
% \begin{haskellcode}
% ghci> let phi(x) = if x == undefined then 42 else undefined
% ghci> let psi(x) = 42
% ghci> phi(0)
% *** Exception: Prelude.undefined
% ghci> phi(undefined)
% *** Exception: Prelude.undefined
% ghci> psi(0)
% 42
% \end{haskellcode}

% Виждаме, че тук \texttt{хаскел} следва нашата семантика за езика {\bf REC}.
% Направените по-горе бележки ни насочват към следната дефиниция на {\bf термален оператор}.

% \begin{framed}
%   \begin{definition}
%     \index{термален оператор}
%     За всеки терм от вида $\tau[\vv{x}_1,\dots,\vv{x}_n, \vv{f}_1,\dots,\vv{f}_k]$
%     дефинираме оператора 
%     \[\Gamma_\tau: \DomOpCBN \to \RanOpCBN,\] % като
%     където
%     \[\Gamma_\tau(\bar{\varphi})(\bar{a}) \dff \val{\tau}(\ov{\varphi})(\ov{a}).\]
%   \end{definition}
% \end{framed}
% \marginpar{Обърнете внимание, че все още не е ясно защо $\Gamma_\tau(\ov{\varphi}) \in \RanOpCBN$}
% В следващия раздел ще се концентрираме върху доказателството на следния основен резултат:

% \begin{framed}
%   За всеки терм $\tau[\varsx, \varsf]$, операторът $\Gamma_\tau$ е непрекъснат.
% \end{framed}

\subsection{Непрекъснатост на термалните оператори}

% Първата ни задача ще бъде да проверим, че операторът $\Gamma_\tau$ е коректно дефиниран.
% Това означава, че трябва да се уверим, че винаги когато подадем като аргументи на $\Gamma_\tau$ непрекъснати изображения,
% то $\Gamma_\tau$ връща непрекъснато изображение.

\marginpar{В този раздел всички доказателства се провеждат с индукция по построението на термовете.}

\begin{proposition}
  \label{pr:term-monotone}
  Да разгледаме един терм $\tau[\vv{x}_1,\dots,\vv{x}_n, \vv{f}_1,\dots,\vv{f}_k]$.
  \marginpar{Озн. $\ov{a}_i = (a^1_i,\dots,a^n_i)$}
  Нека $\ov{a} \sqsubseteq \ov{b}$.
  Тогава 
  \[\val{\tau}(\ov{\varphi})(\ov{a}) \sqsubseteq \val{\tau}(\ov{\varphi})(\ov{b}),\]
  където
  $\varphi_i \in \Cont{\Nat^{m_i}_\bot}{\Nat_\bot}$, за $i = 1,\dots,k$.
\end{proposition}
\begin{hint}
  Индукция по построението на терма $\tau$.
  \begin{itemize}
  \item
    Нека $\tau \equiv \vv{c}$. Този случай е ясен.
  \item
    Нека $\tau \equiv \vv{x}_i$. Тогава
    \begin{align*}
      \val{\tau}(\ov{\varphi})(\ov{a}) & = \val{\vv{x}_i}(\ov{\varphi})(\ov{a})\\
                                       & = a_i & \comment{\text{стойност на терм}}\\
                                       & \sqsubseteq b_i & \comment{\ov{a} \sqsubseteq \ov{b}}\\
                                       &= \val{\vv{x}_i}(\ov{\varphi})(\ov{b}) & \comment{\text{стойност на терм}}\\
                                       & = \val{\tau}(\ov{\varphi})(\ov{b}).
    \end{align*}
  \item
    Нека $\tau \equiv \tau_1 + \tau_2$. 
    Тук ще използваме, че от {\bf И.П.}, за $j = 1,2$, имаме
    \begin{equation}
      \label{eq:10}
      \val{\tau_j}(\ov{\varphi})(\ov{a}) \sqsubseteq \val{\tau_j}(\ov{\varphi})(\ov{b}).
    \end{equation}
    Освен това, понеже изображението $\texttt{plus}$ е непрекъснато, то то е и монотонно.
    \begin{align*}
      \val{\tau}(\ov{\varphi})(\ov{a}) & = \val{\tau_1 + \tau_2}(\ov{\varphi})(\ov{a})\\
                                       & \df \texttt{plus}(\val{\tau_1}(\ov{\varphi})(\ov{a}), \val{\tau_2}(\ov{\varphi})(\ov{a}))\\
                                       & \sqsubseteq \texttt{plus}(\val{\tau_1}(\ov{\varphi})(\ov{b}), \val{\tau_2}(\ov{\varphi})(\ov{b})) & \comment{\text{от (\ref{eq:10}) и мон.}}\\
      & \df \val{\tau_1 + \tau_2}(\ov{\varphi})(\ov{b}).
    \end{align*}
  \item
    \marginpar{\writedown За домашно!}
    Нека $\tau \equiv \tau_1\ \vv{==}\ \tau_2$.
  \item
    Нека $\tau \equiv \ifelse{\tau_0}{\tau_1}{\tau_2}$.  
  \item
    Нека $\tau \equiv \vv{f}_i(\tau_1,\dots,\tau_{m_i})$. 
    Тук ще използваме, че от {\bf И.П.}, за $j = 1,\dots,m_i$, имаме
    \begin{equation}
      \label{eq:6}
      \val{\tau_j}(\ov{\varphi})(\ov{a}) \sqsubseteq \val{\tau_j}(\ov{\varphi})(\ov{b}).
    \end{equation}
    Тогава
    \begin{align*}
      \val{\tau}(\ov{\varphi})(\ov{a}) & = \val{\vv{f}_i(\tau_1,\dots,\tau_{m_i})}(\ov{\varphi})(\ov{a})\\
                                      & = \varphi_i(\val{\tau_1}(\ov{\varphi})(\ov{a}), \dots, \val{\tau_{m_i}}(\ov{\varphi})(\ov{a})) & \comment\text{от деф.}\\
                                      & \sqsubseteq \varphi_i(\val{\tau_1}(\ov{\varphi})(\ov{b}), \dots, \val{\tau_{m_i}}(\ov{\varphi})(\ov{b})) & \comment{\text{от (\ref{eq:6}) и $\varphi_i$ е мон.}}\\
                                      & = \val{\tau}(\ov{\varphi})(\ov{b}). & \comment\text{от деф.}
    \end{align*}
  \end{itemize}
\end{hint}

\begin{corollary}
  \label{cr:tau-preserves-continuous}
  Да разгледаме един терм $\tau[\vv{x}_1,\dots,\vv{x}_n, \vv{f}_1,\dots,\vv{f}_k]$.
  Тогава за произволна верига $\chain{\bar{a}}{i}$ в $\Nat^{n}_\bot$,
  \[\val{\tau}(\ov{\varphi})(\bigsqcup_i\ov{a}_i) = \bigsqcup_i\val{\tau}(\ov{\varphi})(\ov{a}_i),\]
  където
  $\varphi_i \in \Cont{\Nat^{n_i}_\bot}{\Nat_\bot}$, за $i = 1,\dots,k$.
\end{corollary}
\begin{proof}
  Да разгледаме произволен терм $\tau$ и непрекъснати изображения $\ov{\varphi}$.
  От \Prop{term-monotone} следва, че $\val{\tau}(\ov{\varphi}) \in \Mon{\Nat^n_\bot}{\Nat_\bot}$.
  Понеже всяка верига в $\Nat^n_\bot$ се стабилизира, то директно от \Prop{stab-continuous}
  следва, че $\val{\tau}(\ov{\varphi}) \in \Cont{\Nat^n_\bot}{\Nat_\bot}$,
  което означава, че за произволна верига $\chain{\ov{a}}{i}$,
  \[\val{\tau}(\ov{\varphi})(\bigsqcup_i\ov{a}_i) = \bigsqcup_i\val{\tau}(\ov{\varphi})(\ov{a}_i).\]
\end{proof}

\begin{corollary}\label{cr:gamma-preserves-continuous}
  Да разгледаме произволен терм $\tau[\vv{x}_1,\dots,\vv{x}_n, \vv{f}_1,\dots,\vv{f}_k]$.
  Тогава за произволни $\varphi_i \in \Cont{\Nat^{m_i}}{\Nat_\bot}$, за $i = 1,\dots k$,
  имаме, че
  \[\val{\tau}(\bar\varphi) \in \RanOpCBN.\]
\end{corollary}

Щом термалните оператори $\val{\tau}$ са добре дефинирани, ще 
проверим, че са непрекъснати. Първата стъпка ще бъде проверката, че те са монотонни.

\begin{lemma}\label{lem:rec:functional:term:continuous}
  Нека $\tau[\vv{x}_1,\dots,\vv{x}_n,\vv{f}_1,\dots,\vv{f}_k]$ е произволен терм.
  \marginpar{Да напомним, че $m_i \df \sharp\vv{f}_i$}
  Да разгледаме произволна верига $\chain{\ov{\varphi}}{r}$
  от елементи на областта на Скот $\DomOpCBN$.
  Тогава $\val{\tau}$ е непрекъснато изображение, т.е.
  \[\val{\tau}(\bigsqcup_r\ov{\varphi}_r) = \bigsqcup_r \val{\tau}(\ov{\varphi}_r).\]
\end{lemma}
\begin{proof}
  Индукция по построението на терма $\tau$.
  \begin{itemize}
  \item
    Нека $\tau \equiv \vv{x}_i$.
  \item
    Нека $\tau \equiv \vv{c}$. Този случай е очевиден.
  \item
    Нека $\tau \equiv \tau_1 + \tau_2$. Ще използваме, че $\texttt{plus}$ е непрекъснато изображение.
    \begin{align*}
      \val{\tau}(\bigsqcup_r \ov{\varphi}_r)(\ov{a}) & = \texttt{plus}(\val{\tau_1}(\bigsqcup_r\ov{\varphi}_r)(\ov{a}), \val{\tau_2}(\bigsqcup_r\ov{\varphi}_r)(\ov{a})) & \comment\text{от деф.}\\
                                             & = \texttt{plus}(\bigsqcup_r\val{\tau_1}(\ov{\varphi}_r)(\ov{a}), \bigsqcup_r \val{\tau_2}(\ov{\varphi}_r)(\ov{a})) & \comment{\text{от И.П.}}\\
                                             & = \bigsqcup_r \texttt{plus}(\val{\tau_1}(\ov{\varphi}_r)(\ov{a}), \val{\tau_2}(\ov{\varphi}_r)(\ov{a})) & \comment{\texttt{plus}\text{ е непр.}}\\
                                             & = \bigsqcup_r \val{\tau}(\ov{\varphi}_r)(\ov{a}). & \comment\text{от деф.}
    \end{align*}
  \item
    Нека $\tau \equiv \tau_1\ \vv{==}\ \tau_2$.
    Използвайте, че $\texttt{eq}$ е непрекъснато изображение.
  \item
    Нека $\tau \equiv \ifelse{\tau_0}{\tau_1}{\tau_2}$.
    Използвайте \Problem{rec:if:continuous}.
  \item
    Нека $\tau \equiv \vv{f}_i(\tau_1,\dots,\tau_{m_i})$. 
    От И.П. знаем, че $\val{\tau_j}$ са непрекъснати изображения за $j = 1,\dots,m_i$ и следователно са монотонни. Тогава
    за произволни индекси $n \leq n'$ и $r \leq r'$ имаме, че
    \marginpar{$\ov{\varphi}_n \df (\varphi^1_n,\dots,\varphi^{k}_n)$}
    \begin{align*}
      \varphi^i_{n}(\val{\tau_1}(\ov{\varphi}_r)(\ov{a}),\dots,\val{\tau_{m_i}}(\ov{\varphi}_r)(\ov{a})) & \sqsubseteq \varphi^i_{n}(\val{\tau_1}(\ov{\varphi}_{r'})(\ov{a}),\dots,\val{\tau_{m_i}}(\ov{\varphi}_{r'})(\ov{a}))\\
                                                                                       & \sqsubseteq \varphi^i_{n'}(\val{\tau_1}(\ov{\varphi}_{r'})(\ov{a}),\dots,\val{\tau_{m_i}}(\ov{\varphi}_{r'})(\ov{a})).
    \end{align*}
    Това означава, че ако положим
    \[e_{n,r} \df \varphi^i_{n}(\val{\tau_1}(\ov{\varphi}_r)(\ov{a}),\dots,\val{\tau_{m_i}}(\ov{\varphi}_r)(\ov{a})),\]
    то е изпълнена импликацията:
    \[n \leq n'\ \&\ r \leq r' \implies e_{n,r} \sqsubseteq e_{n',r'}.\]
    Сега можем да приложим \Th{double-chain}, според която
    \begin{equation}
      \label{eq:8}
      \bigsqcup_n(\bigsqcup_r e_{n,r}) = \bigsqcup_n e_{n,n}.
    \end{equation}

    Тогава 
    \begin{align*}
      \val{\tau}(\bigsqcup_n\ov{\varphi}_n)(\ov{a}) & = \val{\vv{f}_i(\tau_1,\dots,\tau_{m_i})}(\bigsqcup_n\ov{\varphi}_n)(\ov{a})\\
                                                    & = (\bigsqcup_n\varphi^i_n)(\val{\tau_1}(\bigsqcup_r\ov{\varphi}_r)(\ov{a}),\dots,\val{\tau_{m_i}}(\bigsqcup_r\ov{\varphi}_r)(\ov{a})) & \comment\text{от деф.}\\
                                                    & = \bigsqcup_n \{\varphi^i_n(\val{\tau_1}(\bigsqcup_r\ov{\varphi}_r)(\ov{a}),\dots,\val{\tau_{m_i}}(\bigsqcup_r\ov{\varphi}_r)(\ov{a}))\} & \comment{\text{от \Th{monotone-is-domain}}}\\
                                           & =  \bigsqcup_n \{\varphi^i_n(\bigsqcup_r \val{\tau_1}(\ov{\varphi}_r)(\ov{a}),\dots,\bigsqcup_r \val{\tau_{m_i}}(\ov{\varphi}_r)(\ov{a}))\} & \comment{\text{от И.П. за }\tau_j}\\
                                           & =  \bigsqcup_n \{\bigsqcup_r \underbrace{\varphi^i_n(\val{\tau_1}(\ov{\varphi}_r)(\ov{a}),\dots,\val{\tau_{m_i}}(\ov{\varphi}_r)(\ov{a}))}_{e_{n,r}}\} & \comment{\varphi^i_n \text{ е непр.}}\\
                                           & =  \bigsqcup_n \underbrace{\varphi^i_n(\val{\tau_1}(\ov{\varphi}_n)(\ov{a}),\dots,\val{\tau_{m_i}}(\ov{\varphi}_n)(\ov{a}))}_{e_{n,n}} & \comment{\text{от (\ref{eq:8})}}\\
                                           & = \bigsqcup_n \val{\vv{f}_i(\tau_1,\dots,\tau_{m_i})}(\ov{\varphi}_n)(\ov{a}) & \comment\text{от деф.}\\
                                           & = \bigsqcup_n \val{\tau}(\ov{\varphi}_n)(\ov{a}).
    \end{align*}
  \end{itemize}
\end{proof}

% \begin{corollary}
%   \label{cr:rec:term:continuous}
%   Нека $\tau[\varsx,\varsf]$ е терм.
%   Да разгледаме произволни елементи $\ov{a}$ на $\Nat^n_\bot$ и 
%   произволна верига $\chain{\ov{\varphi}}{r}$
%   от елементи на $\DomOpCBN$.
%   Тогава 
%   \[\val{\tau}(\bigsqcup_r\ov{\varphi}_r)(\ov{a}) = \bigsqcup_r \val{\tau}(\ov{\varphi}_r)(\ov{a}).\]  
% \end{corollary}
% \begin{proof}
%   \marginpar{Тук има проблем! Някои от $\ov{a}$ може да е $\bot$.}
%   \begin{align*}
%     \val{\tau}(\bigsqcup_r\ov{\varphi}_r)(\ov{a}) & = \val{\tau[\varsx/\ov{\vv{a}}]}(\bigsqcup_r \ov{\varphi}_r) & \comment{\text{от \hyperref[lem:rec:substitution]{Лема за замяната}}}\\
%                                                   & = \bigsqcup_r \val{\tau[\varsx/\ov{\vv{a}}]}(\ov{\varphi}_r) & \comment{\text{от \Lem{rec:functional:term:continuous}}}\\
%                                                   & = \bigsqcup_r \val{\tau}(\ov{\varphi}_r)(\ov{a}) &  \comment{\text{от \hyperref[lem:rec:substitution]{Лема за замяната}}}\\
%   \end{align*}
% \end{proof}

Вече имаме всичко необходимо за да се убедим, че термалните оператори са непрекъснати.

\begin{framed}
\begin{theorem}
  \label{th:gamma-is-continuous}
  За всеки терм $\tau[\vv{x}_1,\dots,\vv{x}_n, \vv{f}_1,\dots,\vv{f}_k]$, имаме, че
  \[\val{\tau} \in \Cont{\DomOpCBN}{\RanOpCBN}.\]
\end{theorem}
\end{framed}
% \begin{proof}
%   От \Cor{gamma-preserves-continuous} имаме, че $\Gamma_\tau$ е коректно дефиниран.
%   Сега директно се позоваваме на \Cor{rec:term:continuous}.
% \end{proof}


%%% Local Variables:
%%% mode: latex
%%% TeX-master: "../sep"
%%% End:


% \subsection{Предаване на параметрите по име}

Нека е дадена една рекурсивна програма $\vv{P}[\varsx,\varsf]$, където:
\marginpar{Можем да си мислим, че $\vv{f}_1$ е нещо като \texttt{main} функция за програмата $\vv{P}$.}
\begin{align*}
  \vv{P} = & 
             \begin{cases}
               & \vv{f}_1(\vv{x}_1,\dots,\vv{x}_{m_1}) = \tau_1[\vv{x}_1,\dots,\vv{x}_{m_1},\vv{f}_1,\dots,\vv{f}_k]\\
               & \vv{f}_2(\vv{x}_1,\dots,\vv{x}_{m_2}) = \tau_2[\vv{x}_1,\dots,\vv{x}_{m_2},\vv{f}_1,\dots,\vv{f}_k]\\
               & \vdots\\
               & \vv{f}_k(\vv{x}_1,\dots,\vv{x}_{m_k}) = \tau_k[\vv{x}_1,\dots,\vv{x}_{m_k},\vv{f}_1,\dots,\vv{f}_k]
             \end{cases}
\end{align*}
Нека $\ov{\gamma} \in \DomOpCBN$
е {\em най-малкото решение} на системата
\marginpar{В тази система неизвестните са $\varphi_1,\dots,\varphi_k$.}

\begin{SystemEq}
  \varphi_1 & = & \val{\tau_1}(\varphi_1,\dots,\varphi_k)\\
  & \vdots & \\
  \varphi_k & = & \val{\tau_k}(\varphi_1,\dots,\varphi_k).
\end{SystemEq}
% \begin{align*}
%   & \val{\tau_1}(\varphi_1,\dots,\varphi_k) = \varphi_1\\
%   & \ \vdots \\
%   & \val{\tau_k}(\varphi_1,\dots,\varphi_k) = \varphi_k.
% \end{align*}
От \hyperref[th:knaster-tarski]{Теоремата на Клини} знаем, че такова най-малко решение съществува.

\index{денотационна семантика!по име}
\begin{framed}
  За дадената рекурсивна програма $\vv{P}[\varsx,\varsf]$, 
  определяме {\bf денотационната семантика с предаване на параметрите по име} 
  като изображението $\D\val{\vv{P}} \in \Cont{\Nat^{m_1}_\bot}{\Nat_\bot}$, където:
  \[\D\val{\vv{P}}(a_1,\dots,a_{m_1}) \df
    \begin{cases}
      \gamma_1(a_1,\dots,a_{m_1}), & \text{ако }\bot\not\in\{a_1,\dots,a_{m_1}\}\\
      \bot, & \text{ако }\bot\in\{a_1,\dots,a_{m_1}\}.
    \end{cases}\]
\end{framed}


\begin{example}
  Да разгледаме следната проста програма:
  \begin{haskellcode}
f(x) = if x == 0 then 1 else f(x+1)
  \end{haskellcode}
  \marginpar{Тази програма е дори валидна програма на хаскел.}
  На тази програма съответства системата от едно уравнение с една функционална променлива
  \begin{equation}
    \label{eq:denotational-cbn:example-motivation}
    f = \Gamma(f),
  \end{equation}
  където за изображението $\Gamma$ имаме, че:
  \begin{align*}
    & \Gamma \in \Cont{\Cont{\Nat_\bot}{\Nat_\bot}}{\Cont{\Nat_\bot}{\Nat_\bot}}\\
    & \Gamma(f)(x) \df
      \begin{cases}
        1, & \text{ако }x = 0\\
        f(x+1), & \text{иначе}.
      \end{cases}
  \end{align*}
  Най-малкото решение на системата (\ref{eq:denotational-cbn:example-motivation}) е функцията $\varphi:\Nat_\bot\to\Nat_\bot$, за която
  \[\varphi(x) = \begin{cases}
      1, & \text{ако } x =  0\\
      \bot, & \text{иначе}.
    \end{cases}\]
  % $\varphi(0) = 1$ и $\varphi(x) = \bot$ за $x \neq 0$.
  \marginpar{\todo Съобразете, че системата (\ref{eq:denotational-cbn:example-motivation}) има дори безкрайно много решения!}
  Обаче системата (\ref{eq:denotational-cbn:example-motivation}) има и други решения. Например, функцията $\psi(x) = 1$ за всяко $x \in \Nat_\bot$.
  Този пример ни показва, че най-малкото решение кореспондира точно с нашата интуиция как ,,работи'' тази програма.
\end{example}



%%% Local Variables:
%%% mode: latex
%%% TeX-master: "../sep"
%%% End:


% \subsection{Предаване на параметрите по стойност}
\index{денотационна семантика!по стойност}

Основната разлика между денотационната семантика по име и по стойност е, че при семантиката по стойност
изискваме да работим {\em само с точни функции}.

\begin{remark}
  \marginpar{Сигурни сме единствено, че термалните оператори запазват непрекъснатостта}
  Да обърнем внимание, че термалните оператори $\Gamma_\tau$ не запазват точните функции, т.е.
  възможно е $\varphi \in \Strict{\Nat^n_\bot}{\Nat_\bot}$, но $\Gamma_\tau(\varphi) \not \in \Strict{\Nat^m_\bot}{\Nat_\bot}$.
  Например, да разгледаме терма
  \[\tau[\vv{x},\vv{y},\vv{f}] \equiv \vv{x},\]
  т.е. изобщо не се обръщаме към функционалния аргумент.
  Тогава ако $a \neq \bot$, но $b = \bot$, то $\val{\tau}(\varphi)(a,b) \neq \bot$.

  Да се уверим, че хаскел ,,смята'' по подобен начин.
  \begin{haskellcode}
ghci> let f(x,y,g) = x
ghci> let h(x,y) = f(x,y,(\z -> z))
ghci> h(1,undefined)
1
  \end{haskellcode}
  Вижда се, че макар и да подаваме точна функция като аргумент на $\vv{f}$, то новополучената функция $\vv{h}$ не е точна.
\end{remark}

Поради тази причина, за да дефинираме семантика с предаване на параметрите по стойност, ще използваме най-малки неподвижни точки
на други оператори, {\em които винаги връщат точни функции}.

\begin{framed}
  \begin{theorem}
    За всеки терм $\tau[\varsx,\varsf]$, операторът 
    \[\Delta_\tau:\DomOpCBV\to \RanOpCBV,\]
    дефиниран като
    \[\Delta_\tau \dff \Sigma_\star \circ \Gamma_\tau,\]
    е непрекъснат.
  \end{theorem}
\end{framed}
\begin{proof}
  Първо да съобразим, че $\Delta_\tau$ е коректно дефиниран оператор, т.е. 
  за $\bar{\varphi} \in \DomOpCBV$ имаме, че $\Delta_\tau(\bar{\varphi}) \in \RanOpCBV$.
  Това е ясно, защото $\Sigma_\star$ превръща всяко изображение в точно.
  
  От \Prop{strict-operator} знаем, че $\Sigma_\star$ е непрекъснат оператор, а от \Prop{composition}
  знаем, че композиция на непрекъснати оператори е също непрекъснат оператор.
  Заключаваме, че $\Delta_\tau$ е непрекъснат оператор.    
\end{proof}

\marginpar{Тук отново можем да си мислим за $\vv{f}_1$ като за \texttt{main} функцията на програмата $\vv{P}$.}
Нека разгледаме рекурсивната програма $\vv{P}[\varsx,\varsf]$ на езика $\REC$, където:
\begin{align*}
  \vv{P} = & 
             \begin{cases}
               & \vv{f}_1(\vv{x}_1,\dots,\vv{x}_{m_1}) = \tau_1[\vv{x}_1,\dots,\vv{x}_{m_1},\vv{f}_1,\dots,\vv{f}_k]\\
               & \vv{f}_2(\vv{x}_1,\dots,\vv{x}_{m_2}) = \tau_2[\vv{x}_1,\dots,\vv{x}_{m_2},\vv{f}_1,\dots,\vv{f}_k]\\
               & \vdots\\
               & \vv{f}_k(\vv{x}_1,\dots,\vv{x}_{m_k}) = \tau_k[\vv{x}_1,\dots,\vv{x}_{m_k},\vv{f}_1,\dots,\vv{f}_k]
             \end{cases}
\end{align*}
Нека $\bar{\delta} \in \DomOpCBV$ е {\em най-малкото решение} на системата,
която съответства на програмата $\texttt{P}$:
\begin{align*}
  & \Delta_{\tau_1}(\psi_1,\dots,\psi_k) = \psi_1\\
  & \ \vdots \\
  & \Delta_{\tau_k}(\psi_1,\dots,\psi_k) = \psi_k.
\end{align*}
Понеже $\Delta_{\tau_i}$ са непрекъснати оператори, то \hyperref[th:knaster-tarski]{Теоремата на Клини} знаем, че такива точни изображения $\ov{\delta}$ съществуват.

\index{денотационна семантика!по стойност}
\begin{framed}
  За рекурсивната програма $\vv{P}[\varsx,\varsf]$, определяме {\bf денотационната семантика с предаване на параметрите по стойност} 
  като изображението $\D_V\val{\vv{P}} \in \Strict{\Nat^{m_1}_\bot}{\Nat_\bot}$, където:
  \[\D_V\val{\vv{P}}(a_1,\dots,a_{m_1}) \dff \delta_1(a_1,\dots,a_{m_1}).\]
\end{framed}

%%% Local Variables:
%%% mode: latex
%%% TeX-master: "../sep"
%%% End:


% \subsection{Сравнение между двете семантики}


\begin{example}
  Да разгледаме програмата $\vv{P}$ на езика $\REC$:
  \begin{haskellcode}
f(x) = g(f(x))
g(x) = 0  
  \end{haskellcode}
Да проверим, че
$\D_V\val{\vv{P}} \neq \D_N\val{\vv{P}}$.
Имаме, че
\begin{align*}
  \tau_1[\vv{x},\vv{f},\vv{g}] & \equiv \vv{g}(\vv{f}(\vv{x}))\\
  \tau_2[\vv{x},\vv{f},\vv{g}] & \equiv \vv{0}.
\end{align*}
За $i = 1,2$ дефинираме операторите 
\[\Delta_{\tau_i} \in \Cont{\Strict{\Nat_\bot}{\Nat_\bot} \times \Strict{\Nat_\bot}{\Nat_\bot}}{\Strict{\Nat_\bot}{\Nat_\bot}},\]
  които имат следните дефиниции:  
  \begin{align*}
    \Delta_{\tau_1}(\varphi,\psi)(x) & \dff
    \begin{cases}
      \psi(\varphi(x)), & \text{ако }x \neq \bot\\
      \bot, & \text{ако }x = \bot\\
    \end{cases}\\
    \Delta_{\tau_2}(\varphi,\psi)(x) & \dff
    \begin{cases}
      0, & \text{ако }x \neq \bot,\\
      \bot, & \text{ако }x = \bot.
    \end{cases}
  \end{align*}
  Нека $\Delta \dff \Delta_{\tau_1} \times \Delta_{\tau_2}$. За да намерим най-малкото
  решение на системата, трябва да намерим най-малката неподвижна точка на $\Delta$.
  Ще получим това най-малко решение като точна горна граница на редица от двойки функции $(\varphi_i, \psi_i)$.
  Имаме, че $\varphi_0 \dff \lambda x.\bot$ и $\psi_0 \dff \lambda x.\bot$.
  За произволно $x \in \Nat_\bot$ имаме, че
  \begin{align*}
    \varphi_1(x) & \dff \Delta_{\tau_1}(\varphi_0,\psi_0)(x)\\
                 & =
                   \begin{cases}
                     \psi_0(\varphi_0(x)), & \text{ако } x \neq \bot\\
                     \bot, & \text{ако }x = \bot
                   \end{cases}\\
                 & = \bot\\
    \psi_1(x) & \dff \Delta_{\tau_2}(\varphi_0,\psi_0)(x)\\
                 & = 
                   \begin{cases}
                     0, & \text{ако }x \neq \bot\\
                     \bot, & \text{ако }x = \bot\\
                   \end{cases}
  \end{align*}
  Това означава, че $\Delta(\varphi_0,\psi_0) = (\varphi_1,\psi_1)$.
  При следващата итерация получаваме следното:
  \begin{align*}
    \varphi_2(x) & \dff \Delta_{\tau_1}(\varphi_1,\psi_1)(x)\\
                 & =
                   \begin{cases}
                     \psi_1(\varphi_1(x)) & \text{ако } x \neq \bot\\
                     \bot, & \text{ако } x = \bot\\
                   \end{cases}\\
                 & = 
                   \begin{cases}
                     \psi_1(\bot) & \text{ако } x \neq \bot\\
                     \bot, & \text{ако } x = \bot\\
                   \end{cases}\\
                 & = \bot\\
    \psi_2(x) & \dff \Delta_{\tau_2}(\varphi_1,\psi_1)(x)\\
                 & = \begin{cases}
                   0, & \text{ако }x \neq \bot\\
                   \bot, & \text{ако }x = \bot\\
                 \end{cases}\\
                 & = \psi_1(x).
  \end{align*}
  Получихме, че $\Delta(\varphi_1,\psi_1) = (\varphi_1,\psi_1)$ и
  следователно $\lfp(\Delta) = (\varphi_1,\psi_1)$.
  Тогава за всяко $a \in \Nat_\bot$,
  \[\D_V\val{\vv{P}}(a) = \varphi_1(a) = \bot.\]

  Сега да намерим семантика по име на горната програма.
За $i = 1,2$ дефинираме операторите 
  \[\Gamma_{\tau_i} \in \Cont{\Cont{\Nat_\bot}{\Nat_\bot} \times \Cont{\Nat_\bot}{\Nat_\bot}}{\Cont{\Nat_\bot}{\Nat_\bot}},\]
  които имат следните дефиниции
  \begin{align*}
    \Gamma_{\tau_1}(\varphi,\psi)(x) & \dff \psi(\varphi(x))\\
    \Gamma_{\tau_2}(\varphi,\psi)(x) & \dff 0.
  \end{align*}
  Търсим най-малката неподвижна точна на оператора 
  \[\Gamma(\varphi,\psi) \dff (\Gamma_{\tau_1}(\varphi, \psi), \Gamma_{\tau_2}(\varphi,\psi)).\]
  \marginpar{т.е. това са функциите, които винаги връщат $\bot$}
  Имаме, че $\varphi_0 \dff \lambda x.\bot$ и $\psi_0 \dff \lambda x.\bot$.
  Сега, за произволно $x \in \Nat_\bot$, получаваме:
  \begin{align*}
    \varphi_1(x) & \dff \Gamma_{\tau_1}(\varphi_0,\psi_0)(x) = \psi_0(\varphi_0(x)) = \bot\\
    \psi_1(x) & \dff \Gamma_{\tau_2}(\varphi_0,\psi_0)(x) = 0.
  \end{align*}
  Това означава, че $\Gamma(\varphi_0,\psi_0) = (\varphi_1,\psi_1)$.
  Итерираме процеса още веднъж, и получаваме:
  \marginpar{Единствената разлика е, че тук не разглеждаме отделен случай дали $x = \bot$}
  \begin{align*}
    \varphi_2(x) & \dff \Gamma_{\tau_1}(\varphi_1,\psi_1)(x) = \psi_1(\varphi_1(x)) = 0\\
    \psi_2(x) & \dff \Gamma_{\tau_2}(\varphi_1,\psi_1)(x) = 0.
  \end{align*}
  Това означава, че $\Gamma(\varphi_1,\psi_1) = (\varphi_2,\psi_2)$.
  Отново итерираме процеса:
    \begin{align*}
      \varphi_3(x) & \dff \Gamma_{\tau_1}(\varphi_2,\psi_2)(x) = \psi_2(\varphi_2(x)) = 0 = \varphi_2(x)\\
      \psi_3(x) & \dff \Gamma_{\tau_2}(\varphi_2,\psi_2)(x) = 0 = \psi_2(x).
  \end{align*}
  Получихме, че $\Gamma(\varphi_2,\psi_2) = (\varphi_2,\psi_2)$ и
  следователно $\lfp(\Gamma) = (\varphi_2,\psi_2)$. 
  Тогава, за всяко $a \in \Nat_\bot$,
  \[\D_N\val{\vv{P}}(a) = \varphi_2(a) = 0.\]
  Това означава, че денотационната семантика по име на програмата $\vv{P}$ е константната функция, която винаги връща $0$.
  Обърнете внимание, че също $\D_N\val{\vv{P}}(\bot) = 0$.
\end{example}

Можем да проверим нашите изчисления за най-малките неподвижни точки на операторите $\Gamma$ и $\Delta$
от горния пример като преведем техните дефиниции на \texttt{хаскел}.

\begin{haskellcode}
ghci> let gamma1(f,g)(x) = g(f(x))
ghci> let gamma2(f,g)(x) = 0
ghci> let gamma(f,g) = (gamma1(f,g), gamma2(f,g))
ghci> let omega = \x -> undefined
ghci> let approxCBN = (omega, omega) : [gamma(f,g) | (f,g) <- approxCBN ]
ghci> let (f3, g3) = approxCBN !! 3
ghci> f3(undefined)
0
ghci> f3(55)
0
ghci> :set -XBangPatterns
ghci> let delta1(f,g)(!x) = g(f(x))
ghci> let delta2(f,g)(!x) = 0
ghci> let delta(f,g) = (delta1(f,g), delta2(f,g))
ghci> let approxCBV = (omega, omega) : [delta(f,g) | (f,g) <- approxCBV ]
ghci> let (f3', g3') = approxCBV !! 3
ghci> f3'(undefined)
undefined
ghci> f3'(0)
undefined
ghci> f3'(55)
undefined
\end{haskellcode}



  Горният пример показва, че съществуват програми $\vv{P}$, 
  за които двете денотационни семантики са различни, но все пак $\D_V\val{\vv{P}} \sqsubseteq \D_N\val{\vv{P}}$.
  Ще видим, че това свойство е вярно за всяка програма $\vv{P}$ на езика \REC.


  \begin{proposition}
    \label{pr:delta-in-gamma}
    Да разгледаме произволен терм $\tau[\vv{x}_1,\dots,\vv{x}_n, \vv{f}_1,\dots,\vv{f}_k]$.
    Тогава всеки 
    \[\bar{\varphi} \in \Mapping{\Nat^{m_1}_\bot}{\Nat_\bot}\times\cdots\times\Mapping{\Nat^{m_k}_\bot}{\Nat_\bot}\]
    е изпълнено, че
    \[\Delta_\tau(\bar{\varphi}) \sqsubseteq \Gamma_\tau(\bar{\varphi}).\]
  \end{proposition}
  \begin{hint}
    \marginpar{Обърнете внимание, че ако $\varphi$ е точна функция, то $\Gamma_\tau(\varphi)$ е непрекъсната, но може да не е точна.}
    Понеже за всяка $\varphi \in \Mapping{\Nat^n_\bot}{\Nat_\bot}$,
    \[\Sigma_\star(\varphi) \sqsubseteq \varphi,\]
    по получаваме, че
    \[\Delta_\tau(\bar{\varphi}) \dff \Sigma_\star(\Gamma_\tau(\bar{\varphi})) \sqsubseteq \Gamma_\tau(\bar{\varphi}).\]
\end{hint}

\begin{framed}
  \begin{theorem}
    За всяка рекурсивна програма \vv{P} е изпълнено, че
    \[\D_V\val{\vv{P}} \sqsubseteq \D_N\val{\vv{P}}.\]
  \end{theorem}
\end{framed}
\begin{proof}
  Първо с индукция по $k$ ще докажем, че
  \[(\forall k)[\ \bar{\delta}_k \sqsubseteq \bar{\gamma}_k\ ].\]
  За $k = 0$ е ясно, защото $\delta^i_0(\ov{x}) = \bot = \gamma^i_0(\ov{x})$.
  \marginpar{Озн. $\ov{\delta}_k = (\delta^1_k,\dots,\delta^n_k)$}
  Да приемем, че $\ov{\delta}_k \sqsubseteq \ov{\gamma}_k$. Тогава
  \begin{align*}
    \delta^i_{k+1} & \dff \Delta_{\tau_i}(\bar{\delta}_k)\\
                   & \sqsubseteq \Gamma_{\tau_i}(\bar{\delta}_k) & \comment{\text{\Prop{delta-in-gamma}}}\\
                   & \sqsubseteq \Gamma_{\tau_i}(\bar{\gamma}_k) & \comment{\text{$\Gamma_{\tau_i}$ е мон., а от {\bf И.П.} имаме, че } \ov{\delta}_k \sqsubseteq \ov{\gamma}_k}\\
                   & \dff \gamma^i_{k+1}.
  \end{align*}
  Заключаваме, че $\ov{\delta}_{k+1} \sqsubseteq \ov{\gamma}_{k+1}$.
  Сега е лесно да съобразим, че
  \[\delta^i = \bigsqcup_k \delta^i_k \sqsubseteq \bigsqcup_k \gamma^i_k = \gamma^i.\]
  Оттук следва, че 
  \begin{equation}
    \label{eq:12}
    \bar{\delta} = \bigsqcup_k\bar{\delta}_k \sqsubseteq \bigsqcup_k\bar{\gamma}_k = \bar{\gamma}.
  \end{equation}
  Сега, за произволни елементи $\ov{a}$,
  \begin{align*}
    \D_V\val{\vv{h}}(\bar{a}) & \dff \delta_1(\ov{a})\\
                              & \sqsubseteq \gamma_1(\ov{a})  & \comment{\text{от (\ref{eq:12}) имаме, че }\bar{\delta}\sqsubseteq\bar{\gamma}}\\
                              & \dff \D_N\val{\vv{h}}(\ov{a}).
  \end{align*}
\end{proof}


%%% Local Variables:
%%% mode: latex
%%% TeX-master: "../sep"
%%% End:


% \section{Операционна семантика}\index{операционна семантика}

\subsection{Предаване на параметрите по име}\index{операционна семантика}

% Правилата за извод с предаване на параметрите по име, които означаваме като $\mu \Downarrow^P c$,
% са същите като тези с предаване на параметрите по стойност като 
% единствената разлика е, че вместо правилата $(4)_\Nat$ и $(4)_\bot$ имаме правилото $(4)$.

\marginpar{Това се нарича cost dynamics в \cite[стр. 58]{practical-foundations}.}
% В някои доказателства ще се наложи да правим индукция по дължината на извода $\to_P$.
% Затова дефинираме релацията $\mu \to^\ell_P \vv{a}$, която казва, че функционалният терм $\mu$
% се свежда до константата $\vv{a}$ след $\ell$ на брой прилагания на правилата на операционната семантика.
% Формално дефиницията е следната:


Дефинираме релация $\Downarrow^\ell_P$, която ни казва как един функционален терм $\mu$
се свежда до константата $\vv{a}$ след $\ell$ на брой стъпки следвайки следните правила:

\begin{description}
\item
  % За всяко $a \in \Nat$,
  \begin{figure}[h!]
    \begin{prooftree}
      \AxiomC{}
      \RightLabel{\scriptsize{(const)}}
      \UnaryInfC{$\vv{a}\Downarrow^0_P \vv{a}$}
    \end{prooftree}
  \end{figure}
\item
  \begin{figure}[h!]
    \begin{prooftree}
      \AxiomC{$\mu_1\Downarrow^{\ell_1}_P \vv{a}_1$}
      \AxiomC{$\mu_2\Downarrow^{\ell_2}_P \vv{a}_2$}
      \AxiomC{$a = \texttt{plus}(a_1, a_2)$}
      \RightLabel{\scriptsize{(plus)}}
      \TrinaryInfC{$\mu_1 + \mu_2 \Downarrow^{\ell_1+\ell_2+1}_P \vv{a}$}
    \end{prooftree}
  \end{figure}
\item
  \begin{figure}[h!]
    \begin{prooftree}
      \AxiomC{$\mu_1\Downarrow^{\ell_1}_P \vv{a}_1$}
      \AxiomC{$\mu_2\Downarrow^{\ell_2}_P \vv{a}_2$}
      \AxiomC{$a = \texttt{eq}(a_1, a_2)$}
      \RightLabel{\scriptsize{(eq)}}
      \TrinaryInfC{$\mu_1\ \vv{==}\ \mu_2 \Downarrow^{\ell_1+\ell_2+1}_P \vv{a}$}
    \end{prooftree}
  \end{figure}
\item
  \begin{figure}[h!]
    \begin{prooftree}
      \AxiomC{$\mu_0\Downarrow^{\ell_0}_P \vv{a}_0$}
      \AxiomC{$\mu_1 \Downarrow^{\ell_1}_P \vv{a}_1$}
      \AxiomC{$\vv{a}_0 \not\equiv \vv{0}$}
      \RightLabel{\scriptsize{(if$^+$)}}
      \TrinaryInfC{$\ifelse{\mu_0}{\mu_1}{\mu_2} \Downarrow^{\ell_0+\ell_1+1}_P \vv{a}_1$}
    \end{prooftree}
  \end{figure}  
\item
  \begin{figure}[h!]
    \begin{prooftree}
      \AxiomC{$\mu_0\Downarrow^{\ell_0}_P \vv{0}$}
      \AxiomC{$\mu_2 \Downarrow^{\ell_2}_P \vv{a}_2$}
      \RightLabel{\scriptsize{(if$_0$)}}
      \BinaryInfC{$\ifelse{\mu_0}{\mu_1}{\mu_2} \Downarrow^{\ell_0+\ell_2+1}_P \vv{a}_2$}
    \end{prooftree}
  \end{figure}
\item
  \begin{figure}[h!]
    \begin{prooftree}
      \AxiomC{$\tau_i[\vv{x}_1/\mu_1,\dots,\vv{x}_{m_i}/\mu_{m_i}] \Downarrow^{\ell}_P \vv{a}$}
      \RightLabel{\scriptsize{(cbn)}}
      \UnaryInfC{$\vv{f}_i(\mu_1,\dots,\mu_{m_i}) \Downarrow^{\ell+1}_P \vv{a}$}
    \end{prooftree}
  \end{figure}
\end{description}

Ще пишем $\mu \Downarrow_{\vv{P}} \vv{a}$, ако съществува $\ell$, за което $\mu \Downarrow^\ell_{\vv{P}} \vv{a}$.
Също така, понякога ще пишем $\mu \Downarrow^{<\ell}_P \vv{a}$, когато искаме да кажем, че
$\mu$ се свежда до $\vv{a}$ след прилагане на по-малко от $\ell$ на брой правила от операционната семантика.

\begin{lemma}
  \marginpar{Защо не можем да направим индукция по построението на термовете?}
  Докажете, че за всеки затворен терм $\mu$,
  ако $\mu \Downarrow_P \vv{a}$ и $\mu \Downarrow_P \vv{a}'$, то $\vv{a} \equiv \vv{a}'$.
\end{lemma}
\begin{hint}
  Индукция по дължината на изчислението.
\end{hint}

За фиксирана програма $\vv{P}$, нека за всеки {\em функционален} терм $\mu$ да дефинираме
\[\eval{\mu}\df
  \begin{cases}
    b, & \text{ ако }\mu \Downarrow_P \vv{b}\\
    \bot, & \text{ ако }\mu \text{ няма извод до константа}.
\end{cases}\]

\begin{framed}
  Операционната семантика по име на рекурсивната програма $\vv{P}[\varsx,\varsf]$ представлява
  изображението 
  \[\O\val{\vv{P}} \in \Cont{\Nat^{m_1}_\bot}{\Nat_\bot},\] където
  \[\O\val{\vv{P}}(a_1,\dots,a_{m_1}) \df
    \begin{cases}
      \eval{\vv{f}_1(\vv{a}_1,\dots,\vv{a}_{m_1})}, & \text{ако }\bot \not\in\{a_1,\dots,a_{m_1}\}\\
      \bot, & \text{ако }\bot \in \{a_1,\dots,a_{m_1}\}
    \end{cases}\]
  за произволни $a_1,\dots,a_{m_1} \in \Nat_\bot$.
\end{framed}

\begin{remark}
  Всъщност ние все още няма как да знаем, че за всяка програма $\vv{P}$,
  $\O\val{\vv{P}}$ е непрекъснато изображение.
  Този факт може да се докаже директно, но вместо това, ние ще видим, че
  $\O\val{\vv{P}} = \D\val{\vv{P}}$ и оттам ще получим непрекъснатостта на $\O\val{\vv{P}}$,
  защото от дефиницията на $\D\val{\vv{P}}$ е ясно, че то е непрекъснато изображение.
\end{remark}


% \begin{description}
% \item
%   % За всяко $a \in \Nat$,
%   \begin{figure}[h!]
%     \begin{prooftree}
%       \AxiomC{}
%       \RightLabel{\scriptsize{(const)}}
%       \UnaryInfC{$\vv{a} \Downarrow^0_P \vv{a}$}
%     \end{prooftree}
%   \end{figure}
% \item
%   \begin{figure}[h!]
%     \begin{prooftree}
%       \AxiomC{$\mu_1\Downarrow^{\ell_1}_P \vv{a}_1$}
%       \AxiomC{$\mu_2\Downarrow^{\ell_2}_P \vv{a}_2$}
%       \AxiomC{$a = a_1 + a_2$}
%       \RightLabel{\scriptsize{(plus)}}
%       \TrinaryInfC{$\mu_1 + \mu_2 \Downarrow^{\ell_1+\ell_2+1}_P \vv{a}$}
%     \end{prooftree}
%   \end{figure}
% \item
%   \begin{figure}[h!]
%     \begin{prooftree}
%       \AxiomC{$\mu_1\Downarrow^{\ell_1}_P \vv{a}_1$}
%       \AxiomC{$\mu_2\Downarrow^{\ell_2}_P \vv{a}_2$}
%       \AxiomC{$a = \texttt{eq}(a_1, a_2)$}
%       \RightLabel{\scriptsize{$(eq)$}}
%       \TrinaryInfC{$\mu_1\ \vv{==}\ \mu_2 \Downarrow^{\ell_1+\ell_2+1}_P \vv{a}$}
%     \end{prooftree}
%   \end{figure}
% \item
%   \begin{figure}[h!]
%     \begin{prooftree}
%       \AxiomC{$\mu_1\Downarrow^{\ell_1}_P \vv{a}_1$}
%       \AxiomC{$\mu_2 \Downarrow^{\ell_2}_P \vv{a}_2$}
%       \AxiomC{$\vv{a}_1 \not\equiv \vv{0}$}
%       \RightLabel{\scriptsize{(if$_\true$)}}
%       \TrinaryInfC{$\ifelse{\mu_1}{\mu_2}{\mu_3} \Downarrow^{\ell_1+\ell_2+1}_P \vv{a}_2$}
%     \end{prooftree}
%   \end{figure}  
% \item
%   \begin{figure}[h!]
%     \begin{prooftree}
%       \AxiomC{$\mu_1\Downarrow^{\ell_1}_P \vv{0}$}
%       \AxiomC{$\mu_3 \Downarrow^{\ell_3}_P \vv{a}_3$}
%       \RightLabel{\scriptsize{(if$_\false$)}}
%       \BinaryInfC{$\ifelse{\mu_1}{\mu_2}{\mu_3} \Downarrow^{\ell_1+\ell_3+1}_P \vv{a}_3$}
%     \end{prooftree}
%   \end{figure}
% \item
%   \begin{figure}[h!]
%     \begin{prooftree}
%       \AxiomC{$\tau_i[\vv{x}_1/\mu_1,\dots,\vv{x}_{m_i}/\mu_{m_i}] \Downarrow^{\ell}_P \vv{a}$}
%       \RightLabel{\scriptsize{(cbn)}}
%       \UnaryInfC{$\vv{f}_i(\mu_1,\dots,\mu_{m_i}) \Downarrow^{\ell+1}_P \vv{a}$}
%     \end{prooftree}
%   \end{figure}
% \end{description}

% \newpage

\begin{example}
  Нека за програмата \vv{P}:
  \begin{haskellcode}
    f(x,y) = if x == y then 0 else 1 + f(x,y+1)
  \end{haskellcode}
  да разгледаме няколко извода с правилата на операционната семантика по име.
\end{example}


%%% Local Variables:
%%% mode: latex
%%% TeX-master: "../sep"
%%% End:


\newpage
\def\defaultHypSeparation{\hskip .6cm}

\begin{landscape}

\begin{figure}[h!]
  \begin{framed}
    \begin{prooftree}
      \AxiomC{}
      \LeftLabel{\scriptsize{(const)}}
      \UnaryInfC{$\vv{3} \Downarrow_P \vv{3}$}
      \AxiomC{}
      \RightLabel{\scriptsize{(const)}}
      \UnaryInfC{$\vv{2} \Downarrow_P \vv{2}$}
      \LeftLabel{\scriptsize{(eq)}}
      \BinaryInfC{$\vv{3 == 2} \Downarrow_P \vv{0}$}
      \AxiomC{}
      \LeftLabel{\scriptsize{(const)}}
      \UnaryInfC{$\vv{1} \Downarrow_P \vv{1}$}
      \AxiomC{}
      \LeftLabel{\scriptsize{(const)}}
      \UnaryInfC{$\vv{3} \Downarrow_P \vv{3}$}
      \AxiomC{}
      \RightLabel{\scriptsize{(const)}}
      \UnaryInfC{$\vv{2+1} \Downarrow_P \vv{3}$}
      \LeftLabel{\scriptsize{(eq)}}
      \BinaryInfC{$\vv{3 == 2 + 1} \Downarrow_P \vv{1}$}
      \AxiomC{}
      \RightLabel{\scriptsize{(const)}}
      \UnaryInfC{$\vv{0} \Downarrow_P \vv{0}$}
      \RightLabel{\scriptsize{(if$^+$)}}
      \BinaryInfC{$\ifelse{\vv{3 == 2+1}}{\vv{0}}{\vv{1+f(3,2+1+1)}} \Downarrow_P \vv{0}$}
      \RightLabel{\scriptsize{(cbn)}}
      \UnaryInfC{$\vv{f(3,2+1)} \Downarrow_P \vv{0}$}
      \RightLabel{\scriptsize{(plus)}}
      \BinaryInfC{$\vv{1 + f(3,2+1)} \Downarrow_P \vv{1}$}
      \RightLabel{\scriptsize{(if$_{0}$)}}
      \BinaryInfC{$\ifelse{\vv{3 == 2}}{\vv{0}}{\vv{1 + f(3,2+1)}} \Downarrow_P \vv{1}$ }
      \RightLabel{\scriptsize{(cbn)}}
      \UnaryInfC{$\vv{f(3,2)} \Downarrow_P \vv{1}$}
    \end{prooftree}
  \end{framed}
  \caption{Крайно дърво на извод започващо от функционалния терм $\vv{f(3,2)}$}
\end{figure}

\def\defaultHypSeparation{\hskip .4cm}

\begin{figure}[h!]
  \begin{framed}
    \begin{prooftree}
      \AxiomC{}
      \LeftLabel{\scriptsize{(const)}}
      \UnaryInfC{$\vv{2} \Downarrow_P \vv{2}$}
      \AxiomC{}
      \RightLabel{\scriptsize{(const)}}
      \UnaryInfC{$\vv{3} \Downarrow_P \vv{3}$}
      \RightLabel{\scriptsize{(eq)}}
      \BinaryInfC{$\vv{2 == 3} \Downarrow_P \vv{0}$}
      \AxiomC{}
      \RightLabel{\scriptsize{(const)}}
      \UnaryInfC{$\vv{1} \Downarrow_P \vv{1}$}
      \AxiomC{}
      \LeftLabel{\scriptsize{(const)}}
      \UnaryInfC{$\vv{2} \Downarrow_P \vv{2}$}
      \AxiomC{}
      \LeftLabel{\scriptsize{(const)}}
      \UnaryInfC{$\vv{3} \Downarrow_P \vv{3}$}
      \AxiomC{}
      \RightLabel{\scriptsize{(const)}}
      \UnaryInfC{$\vv{1} \Downarrow_P \vv{1}$}
      \RightLabel{\scriptsize{(plus)}}
      \BinaryInfC{$\vv{3+1} \Downarrow_P \vv{4}$}
      \RightLabel{\scriptsize{(eq)}}
      \BinaryInfC{$\vv{2 == 3+1} \Downarrow_P \vv{0}$}
      \AxiomC{$\vdots$}
      \UnaryInfC{$\vv{1 + f(2,3+1+1)} \Downarrow_P \square$}
      \RightLabel{\scriptsize{(if$_{0}$)}}
      \BinaryInfC{$\ifelse{\vv{2 == 3+1}}{\vv{0}}{\vv{1 + f(2,3+1+1)}} \Downarrow_P \square$}
      \RightLabel{\scriptsize{(cbn)}}
      \UnaryInfC{$\vv{f(2,3+1)} \Downarrow_P \square$}
      \RightLabel{\scriptsize{(plus)}}
      \BinaryInfC{$\vv{1+f(2,3+1)} \Downarrow_P \square$}
      \RightLabel{\scriptsize{(if$_{0}$)}} 
      \BinaryInfC{$\ifelse{\vv{2 == 3}}{\vv{0}}{\vv{1 + f(2,3+1)}} \Downarrow_P \square$ }
      \RightLabel{\scriptsize{(cbn)}}
      \UnaryInfC{$\vv{f(2,3)} \Downarrow_P \square$}
    \end{prooftree}
\end{framed}
\caption{Част от безкрайното дърво на извод започващо от функционалния терм $\vv{f(2,3)}$}
\end{figure}


\end{landscape}


%%% Local Variables:
%%% mode: latex
%%% TeX-master: "../sep"
%%% End:


\subsection{Теорема за еквивалентност}
\marginpar{Винаги с $\mu$ ще означаваме функционални термове, а с $\tau$ произволни термове.}

\begin{proposition}\label{pr:rec:op-name-inclusion1}
  \marginpar{В \cite[стр. 192]{ditchev-soskov}, \cite[стр. 157]{winskel} доказателството е друго.}
  % \marginpar{Сравнете с \Prop{rec:op-value-inclusion1}}
  Да разгледаме една програма \vv{P}.
  Нека $\mu[\vv{f}_1,\dots,\vv{f}_n]$ е {\em функционален} терм. Тогава
  \[\eval{\mu} \sqsubseteq \val{\mu}(\bar{\gamma}),\]
  където $\bar{\gamma} = (\gamma_1,\dots,\gamma_n)$ е 
  \marginpar{Тук използваме наготово \Prop{product-continuous}.}
  някоя неподвижна точка на непрекъснатия оператор
  \[\Gamma \df \val{\tau_1}\times \cdots \times \val{\tau_n},\]
  който съответства на програмата \vv{P}.
\end{proposition}
\begin{proof}
  \marginpar{За това твърдение не е необходимо $\bar{\gamma}$ да бъде най-малката неподвижна точка на $\Gamma$, а просто решение.
    Само за другата посока ще е важно да е най-малкото решение.}
  Ако от терма $\mu$ {\em няма извод} до константа, то
  по дефиниция $\eval{\mu} = \bot$ и в този случай е очевидно, че
  \[\eval{\mu} \sqsubseteq \val{\mu}(\ov{\gamma}).\]
  
  Интересният случай е когато от терма $\mu$ {\em има извод} до константа.
  Тогава ще докажем, че за произволен функционален терм $\mu$ и константа $\vv{a}$,
  \begin{equation}
    \label{eq:16}
    \mu\Downarrow_P \vv{a}\ \implies \val{\mu}(\ov{\gamma}) = a.
  \end{equation}
  Доказателството на (\ref{eq:16}) ще проведем с индукция по дължината $\ell$ на извода $\mu\Downarrow^\ell_P a$.

  \marginpar{Константата $\vv{a}$ има стойност числото $a$.}
  Нека изводът има дължина $0$. Понеже $\mu$ е функционален терм, според правилата на операционната семантика, единствената възможност е $\mu \equiv \vv{a}$.
  Тогава е очевидно, че $\val{\mu}(\bar{\gamma}) = a$. Ясно е, че в този случай,
  \[\vv{a} \Downarrow^0_P \vv{a}\ \implies\ \val{\vv{a}}(\ov{\gamma}) = a.\]
  Нака имаме следното индукционно предположение:
  \begin{equation}
    \label{eq:ind-hyp}
    \mu\Downarrow^{<\ell}_P \vv{a}\ \implies \val{\mu}(\ov{\gamma}) = a.
  \end{equation}
  Ще докажем, че
  \[\mu\Downarrow^{\ell}_P \vv{a}\ \implies \val{\mu}(\ov{\gamma}) = a.\]
  Ще разгледаме всички случаи в зависимост от вида на функционалния терм $\mu$.
  \begin{itemize}
  \item
    Да разгледаме случая, когато $\mu \equiv \mu_1 + \mu_2$.
    Нека $\mu_1 + \mu_2 \Downarrow^\ell_P \vv{a}$.
    Според правилата за извод в операционната семантика по име, имаме следната ситуация:
    \begin{prooftree}
      \AxiomC{$\vdots$}
      % \LeftLabel{\scriptsize{(Извод с дълж. $\ell_1$)}}
      \UnaryInfC{$\mu_1 \Downarrow^{\ell_1}_P \vv{a}_1$}
      \AxiomC{$\vdots$}
      % \RightLabel{\scriptsize{(Извод с дълж. $\ell_2$)}}
      \UnaryInfC{$\mu_2 \Downarrow^{\ell_2}_P \vv{a}_2$}
      \RightLabel{\scriptsize{правило (plus)}}
      \BinaryInfC{$\mu_1 + \mu_2 \Downarrow^{\ell_1+\ell_2+1}_P \vv{a}$}
    \end{prooftree}

    където $a = \texttt{plus}(a_1,a_2)$.
    Ясно е, че изводите на $\mu_1\Downarrow^{<\ell}_P \vv{a}_1$ и $\mu_2 \Downarrow^{<\ell}_P \vv{a}_2$.
    Следователно можем да приложим {\bf И.П.} за $\mu_1$ и $\mu_2$, откъдето получаваме, че
    \begin{align*}
      & \mu_1 \Downarrow^{<\ell}_P \vv{a}_1\ \implies \val{\mu_1}(\ov{\gamma}) = \vv{a}_1\\
      & \mu_2 \Downarrow^{<\ell}_P \vv{a}_2\ \implies \val{\mu_2}(\ov{\gamma}) = \vv{a}_2.
    \end{align*}
    Тогава получаваме, че ако $\mu_1 + \mu_2 \Downarrow^\ell_P \vv{a}$, то
    \begin{align*}
      \val{\mu_1 + \mu_2}(\ov{\gamma}) & \df \texttt{plus}(\val{\mu_1}(\ov{\gamma}), \val{\mu_2}(\ov{\gamma}))\\
                                       & = \texttt{plus}(a_1,a_2)\\
                                       & = a.
    \end{align*}
  \item
    Случаят, когато $\mu \equiv \mu_1\ \vv{==}\ \mu_2$ е лесен.
  \item
    Случаят, когато $\mu \equiv \ifelse{\mu_0}{\mu_1}{\mu_2}$, е лесен.
  \item
    Нека имаме функционалния терм $\mu \equiv \vv{f}_i(\mu_1,\dots,\mu_{m_i})$ и $\mu \Downarrow^\ell_P \vv{a}$.
    Според правилата за извод в операционната семантика по име, имаме следната ситуация:
    \begin{prooftree}
      \AxiomC{$\vdots$}
      % \RightLabel{\scriptsize{Извод с дълж. $(\ell-1)$}}
      \UnaryInfC{$\tau_i[\vv{x}_1/\mu_1,\dots,\vv{x}_{m_i}/\mu_{m_i}] \Downarrow^{\ell-1}_P \vv{a}$}
      \RightLabel{\scriptsize{правило (cbn)}}
      \UnaryInfC{$\vv{f}_i(\mu_1,\dots,\mu_{m_i}) \Downarrow^{\ell}_P \vv{a}$}
    \end{prooftree}
    \marginpar{Индукционното предположение е цялата импликация. Според случая, който разглеждаме и според правилата на операционната семантика, лявата страна на импликацията е изпълнена. Следователно, и дясната страна на импликацията е изпълнена.}
    От {\bf И.П.} имаме, че  
    \[\tau_i[\varsx/\ov{\mu}] \Downarrow^{<\ell}_P \vv{a}\ \implies\ \val{\tau_i[\varsx/\ov{\mu}]}(\ov{\gamma}) = a.\]
    Понеже знаем, че $\tau_i[\varsx/\ov{\mu}] \Downarrow^{<\ell}_P \vv{a}$, то $\val{\tau_i[\varsx/\ov{\mu}]}(\ov{\gamma}) = a$.
    \marginpar{Озн. $\ov{\gamma} = (\gamma_1,\dots,\gamma_n)$. Единствено в този случай се използва, че $\ov{\gamma}$ е решение на системата от оператори, като не е задължително да е най-малкото решение.}
    Лесно се съобразява, че:
    \begin{align*}
      \val{\tau_i[\varsx/\ov{\mu}]}(\ov{\gamma}) & = \val{\tau_i}(\ov{\gamma})(\val{\mu_1}(\ov{\gamma}),\dots,\val{\mu_{m_i}}(\ov{\gamma})) & \comment{\text{\hyperref[lem:rec:substitution]{Лема за замяната}}}\\
                                                 & = \gamma_i(\val{\mu_1}(\ov{\gamma}),\dots, \val{\mu_{m_i}}(\ov{\gamma})) & \comment{\gamma_i = \val{\tau_i}(\ov{\gamma})}\\
                                                 & = \val{\vv{f}_i(\mu_1,\dots,\mu_{m_i})}(\ov{\gamma}). & \comment{\text{деф. на стойност на терм}}\\
    \end{align*}
    Обединявайки всичко, което знаем, получаваме:
    \begin{align*}
      \vv{f}_i(\mu_1,\dots,\mu_{m_i}) \Downarrow^\ell_P \vv{a} & \implies \tau_i[\varsx/\ov{\mu}] \Downarrow^{<\ell}_P \vv{a} & \comment{\text{правило (cbn)}}\\
                                                & \implies \val{\tau_i[\varsx/\ov{\mu}]}(\ov{\gamma}) = a & \comment{\text{{\bf И.П.}}}\\
                                                & \implies  \val{\vv{f}_i(\mu_1,\dots,\mu_{m_i})}(\ov{\gamma}) = a.
    \end{align*}
  \end{itemize}
  С това доказателството на (\ref{eq:16}) е завършено.
\end{proof}

\begin{framed}
  \begin{corollary}[Теорема за коректност]
    \label{cr:on-in-dn}
    За всяка рекурсивна програма $\vv{P}$ на езика \REC имаме, че 
    \[\O\val{\vv{P}} \sqsubseteq \D\val{\vv{P}}.\]
  \end{corollary}
\end{framed}
\begin{proof}
  Нека $\ov{\gamma}$ е най-малкото решение на системата от непрекъснати оператори, която съответства на програмата $\vv{P}$.
  Тогава за произволни $a_1,\dots,a_{m_1} \in \Nat$,
  \begin{align*}
    \O\val{\vv{P}}(a_1,\dots,a_{m_1}) & = \eval{\vv{f}_1(\vv{a}_1,\dots,\vv{a}_{m_1})} & \comment\text{от деф.}\\
                                      & \sqsubseteq \val{\vv{f}_1(\vv{a}_1,\dots,\vv{a}_{m_1})}(\ov{\gamma}) & \comment{\text{\Prop{rec:op-name-inclusion1}}} \\
                                      & = \gamma_1(\val{\vv{a}_1}(\ov{\gamma}),\dots,\val{\vv{a}_{m_1}}(\ov{\gamma})) & \comment{\text{стойност на терм}}\\
                                      & = \gamma_1(a_1,\dots,a_{m_1}) & \comment \val{\vv{a}_i}(\ov{\gamma}) = a_i\\
                                      & = \D\val{\vv{P}}(a_1,\dots,a_{m_1}). & \comment\text{от деф.}
  \end{align*}
\end{proof}

Така получихме едната посока на теоремата за еквивалентност.
Сега преминаваме към доказателството на другата посока.

Нека сега $\ov{\gamma} = (\gamma_1,\dots,\gamma_k)$ е най-малкото решение на системата от оператори, която съответства на програма \vv{P}
за денотационната семантика по име.
Да напомним, че това означава, че $\ov{\gamma} = \bigsqcup_r \ov{\gamma}_r$, 
където $\ov{\gamma}_0 = (\bot^{(m_1)},\dots,\bot^{(m_k)})$ и за всяко $r$,
\begin{align*}
  & \gamma^i_{r+1} = \val{\tau_i}(\ov{\gamma}_r)\\
  & \ov{\gamma}_r = ( \gamma^1_r, \dots, \gamma^k_r).
\end{align*}



\begin{proposition}
  \label{pr:op-name-inclusion2}
  \marginpar{\writedown На пръв поглед не е ясно защо трябва да докажем толкова сложно формулирано твърдение.Съобразете защо не можем да докажем по-простото твърдение, че за всеки функционален терм $\mu$ и всяко $r$,
  $\val{\mu}(\ov{\gamma}_r) \sqsubseteq \eval{\mu}$.}
  Да разгледаме една програма \vv{P} в езика $\REC$, като
  $\ov{\gamma} = \bigsqcup_r\ov{\gamma}_r$ е най-малкото решение на системата от оператори за \vv{P}.
  Тогава за произволен терм $\tau[\vv{x}_1,\dots,\vv{x}_n,\vv{f}_1,\dots,\vv{f}_k]$ и
  произволни функционални термове
  \[\mu_1[\vv{f}_1,\dots,\vv{f}_k],\dots,\mu_n[\vv{f}_1\dots,\vv{f}_k],\]
  и всяко $r$, е изпълнено, че:
  \[\val{\tau}(\ov{\gamma}_r)(\eval{\mu_1},\dots,\eval{\mu_n}) \sqsubseteq \eval{\tau[\varsx/\ov{\mu}]}.\]
\end{proposition}
\begin{proof}
  За произволно естествено число $r$, нека твърдението $\texttt{Include}(r)$ да гласи следното: ,,за произволен терм $\tau[\vv{x}_1,\dots,\vv{x}_n,\vv{f}_1,\dots,\vv{f}_k]$ и
  произволни функционални термове $\mu_1[\vv{f}_1,\dots,\vv{f}_k],\dots,\mu_n[\vv{f}_1\dots,\vv{f}_k]$,
  е изпълнено, че:
  \[\val{\tau}(\ov{\gamma}_r)(\eval{\mu_1},\dots,\eval{\mu_n}) \sqsubseteq \eval{\tau[\varsx/\ov{\mu}]}.\text{''}\]
  Ще докажем с индукция по $r$, че $\texttt{Include}(r)$ е изпълнено за всяко $r$.

  \begin{itemize}
  \item 
    Първо ще докажем $\texttt{Include}(0)$.
    Това ще направим с индукция по построението на терма $\tau$.
    Да разгледаме произволни функционални термове $\mu_i$ и за улеснение да положим $a_i \df \eval{\mu_i}$.
    \begin{itemize}
    \item
      Нека $\tau \equiv \vv{c}$. Тогава:
      \begin{align*}
        \val{\vv{c}}(\ov{\gamma}_0)(\ov{a}) & = c & \comment\text{стойност на терма}\\
                                            & = \eval{\vv{c}} & \comment{\text{правило (1)}}\\
                                            & = \eval{\vv{c}[\varsx/\ov{\mu}]}.
      \end{align*}
    \item
      Нека $\tau \equiv \vv{x}_i$. Тогава:
      \begin{align*}
        \val{\vv{x}_i}(\ov{\gamma}_0)(\ov{a}) & = a_i & \comment\text{стойност на терма}\\
                                              & = \eval{\mu_i} & \comment{a_i \df \eval{\mu_i}}\\
                                              & = \eval{\vv{x}_i[\varsx/\ov{\mu}]}.                    
      \end{align*}
    \item
      Нека $\tau \equiv \tau_1 + \tau_2$.
      В този случай, $\tau$ е съставен от по-простите термове $\tau_1$ и $\tau_2$.
      От {\bf И.П.} за $\tau_1$ и $\tau_2$ следва, че за $i = 1,2$ е изпълнено следното:
      \begin{equation}
        \label{eq:15}
        \val{\tau_i}(\ov{\gamma}_0)(\ov{a}) \sqsubseteq \eval{\tau_i[\varsx/\ov{\mu}]}.
      \end{equation}
      Да напомним, че изображението $\texttt{plus}$ е непрекъснато, откъдето следва, че също така е монотонно. 
      Тогава:
      \begin{align*}
        \val{\tau_1 + \tau_2}(\ov{\gamma}_0)(\ov{a}) & = \texttt{plus}(\val{\tau_1}(\ov{\gamma}_0)(\ov{a}), \val{\tau_2}(\ov{\gamma}_0)(\ov{a})) & \comment\text{стойност на терма}\\
                                                    & \sqsubseteq \texttt{plus}(\eval{\tau_1[\varsx/\ov{\mu}]}, \eval{\tau_2[\varsx/\ov{\mu}]}) & \comment{\text{от (\ref{eq:15})}}\\
                                                    & = \eval{\tau[\varsx/\ov{\mu}]} & \comment{\text{правило }(\texttt{plus})}
      \end{align*}
    \item
      \marginpar{\writedown Тези два случая са за домашно!}
      Нека $\tau \equiv \tau_1\ \vv{==}\ \tau_2$.
    \item
      Нека $\tau \equiv \ifelse{\tau_0}{\tau_1}{\tau_2}$.
    \item
      Нека $\tau \equiv \vv{f}_i(\rho_1,\dots,\rho_{m_i})$. Тогава
      \begin{align*}
        \val{\tau}(\ov{\gamma}_0)(\ov{a}) & = \gamma^i_0(\val{\rho_1}(\ov{\gamma}_0)(\ov{a}), \dots,\val{\rho_{m_i}}(\ov{\gamma}_0)(\ov{a})) & \comment\text{стойност на терма}\\
                                          & = \bot & \comment{\gamma^i_0(\ov{x}) \df \bot}\\
                                          & \sqsubseteq \eval{\tau[\varsx/\ov{\mu}]}.
      \end{align*}
      Така доказахме, че $\texttt{Include}(0)$ е изпълнено.
    \end{itemize}
  \item
    Нека $r > 0$. Да приемем, че $\texttt{Include}(r-1)$ е изпълнено. Ще докажем $\texttt{Include}(r)$
    отново с индукция по построението на термовете.
    \marginpar{\writedown Разгледайте сами останалите случаи за $\tau$ и се убедете, че те наистина се доказват по същия начин както при $r = 0$.} 
    Единственият случай, който заслужава внимание е 
    \[\tau \equiv \vv{f}_i(\rho_1,\dots,\rho_{m_i}).\]
    Доказателствата на всички останали случаи за $\tau$ протичат по абсолютно същия начин както при $r = 0$.

    Понеже термът $\tau$ е построен с помощта на термовете $\rho_j$, за $j = 1, \dots, m_i$,
    можем да приложим {\bf И.П.} за тях и да получим, че 
    \begin{align*}
      \val{\rho_{j}}(\ov{\gamma}_r)(\underbrace{\eval{\mu_1}}_{a_1},\dots,\underbrace{\eval{\mu_n}}_{a_n}) & = \underbrace{\val{\rho_{j}}(\ov{\gamma}_r)(a_1,\dots, a_n)}_{b_j} \\
                                                                                                             & \sqsubseteq \underbrace{\eval{\rho_j[\varsx/\ov{\mu}]}}_{c_j}. & \comment{\text{от {\bf И.П.}}}
    \end{align*}

    \marginpar{Обърнете внимание, че $\rho'_j$ са функционални термове}
    Нека за наше улеснение да положим $\rho'_j \df \rho[\varsx/\ov{\mu}]$.
    Това означава, че до момента имаме следното:
    \[\val{\tau_i}(\ov{\varphi})(\underbrace{\val{\rho_1}(\ov{\gamma}_r)(\ov{a})}_{b_1}, \dots, \underbrace{\val{\rho_{m_i}}(\ov{\gamma}_r)(\ov{a})}_{b_{m_i}}) \sqsubseteq  \val{\tau_i}(\ov{\varphi})(\underbrace{\eval{\rho'_1}}_{c_1},\dots, \underbrace{\eval{\rho'_{m_i}}}_{c_{m_i}}),\]
    за произволни непрекъснати изображения $\ov{\varphi}$.
    
    Като обединим всичко от по-горе, получаваме следното:
    \begin{align*}
      \val{\tau}(\ov{\gamma}_r)(\ov{a}) & = \val{\vv{f}_i(\rho_1,\dots,\rho_{m_i})}(\ov{\gamma}_r)(\ov{a})\\
                                        & = \gamma^i_r(\underbrace{\val{\rho_1}(\ov{\gamma}_r)(\ov{a})}_{b_1}, \dots, \underbrace{\val{\rho_{m_i}}(\ov{\gamma}_r)(\ov{a})}_{b_{m_i}}) & \comment\text{стойност на терма}\\
                                        & = \gamma^i_r(b_1, \dots, b_{m_i})\\
                                        & \sqsubseteq \gamma^i_r(c_1, \dots, c_{m_i}) & \comment{\gamma^i_r\text{ е непр. и следователно мон.}}\\
                                        & = \val{\tau_i}(\ov{\gamma}_{r-1})(c_1,\dots,c_{m_i}) & \comment{\gamma^i_r \df \val{\tau_i}(\ov{\gamma}_{r-1})}\\
                                        & = \val{\tau_i}(\ov{\gamma}_{r-1})(\underbrace{\eval{\rho'_1}}_{c_1},\dots,\underbrace{\eval{\rho'_{m_i}}}_{c_{m_i}}) & \comment{c_i = \eval{\rho'_i}} \\
                                        & \sqsubseteq \eval{\tau_i[\vv{x}_1/\rho'_1,\dots,\vv{x}_{m_i}/\rho'_{m_i}]} & \comment{\text{от }\texttt{Include}(r-1)}\\
                                        & = \eval{\vv{f}_i(\rho'_1,\dots,\rho'_{m_i})} & \comment{\text{от правило (4)}}\\
                                        & = \eval{\vv{f}_i(\rho_1[\varsx/\ov{\mu}],\dots,\rho_{m_i}[\varsx/\ov{\mu}])} & \comment{\rho'_j \df \rho[\varsx/\ov{\mu}]}\\
                                        & = \eval{\vv{f}_i(\rho_1,\dots,\rho_{m_i})[\varsx/\ov{\mu}]} & \comment{\text{правила за замяна}}\\
                                        & = \eval{\tau[\varsx/\ov{\mu}]}.
    \end{align*}
    Заключаваме, че
    \[\val{\tau}(\ov{\gamma}_r)(\underbrace{\eval{\mu_1}}_{a_1},\dots,\underbrace{\eval{\mu_n}}_{a_n}) \sqsubseteq  \eval{\tau[\varsx/\ov{\mu}]}.\]
    Така доказахме $\texttt{Include}(r)$.
  \end{itemize}
  Най-накрая заключаваме, че $(\forall r)\texttt{Include}(r)$.
\end{proof}

\begin{corollary}\label{cr:rec:equivalence-cbn:inclusion2}
  Нека $\mu$ е функционален терм.
  Тогава $\eval{\mu}$ е горна граница на веригата $(\val{\mu}(\ov{\gamma}_r))^{\infty}_{r=0}$.
\end{corollary}

\begin{lemma}
  \marginpar{$f\times g$ е дефинирано в \Prop{cartesian-continuous}.}
  За всяка рекурсивна програма $\vv{P}$,
  произволен {\em функционален} терм $\mu$,
  \[\val{\mu}(\ov{\gamma}) \sqsubseteq \eval{\mu},\]
  където $\ov{\gamma} = \lfp(\Gamma)$, а $\Gamma = \val{\tau_1} \times \val{\tau_2} \times \cdots \times \val{\tau_k}$ е операторът, който съответства на програмата $\vv{P}$.
\end{lemma}
\begin{hint}
  
  \begin{align*}
    \val{\mu}(\ov{\gamma}) & = \val{\mu}(\bigsqcup_r\ov{\gamma}_r) & \comment{\ov{\gamma} = \bigsqcup_r \ov{\gamma}_r}\\
                           & = \bigsqcup_r \val{\mu}(\ov{\gamma}_r) & \comment{\text{от \Lem{rec:functional:term:continuous}}}\\
                           & \sqsubseteq \eval{\mu}. & \comment{\text{от \Cor{rec:equivalence-cbn:inclusion2}}}
  \end{align*}
\end{hint}

\begin{framed}
  \begin{theorem}
    За всяка рекурсивна програма $\vv{P}$ на езика \REC е изпълнено:
    \[\O\val{\vv{P}} = \D\val{\vv{P}}.\]
  \end{theorem}
\end{framed}
\begin{proof}
  \marginpar{Озн. $\ov{\gamma} = (\gamma_1,\dots,\gamma_n)$ е най-малкото решение на системата от оператори съответстващи на програмата \vv{P} }
  Ние вече знаем от \Cor{on-in-dn}, че 
  \[\O\val{\vv{P}} \sqsubseteq \D\val{\vv{P}}.\]
  Остава да докажем обратната посока, а именно 
  \[\D\val{\vv{P}} \sqsubseteq \O\val{\vv{P}}.\]
  За произволни числа $\ov{a}$, имаме следните връзки:
  \begin{align*}
    \D\val{\vv{P}}(\ov{a}) & = \gamma_1(\ov{a}) & \comment\text{деф. на денот. сем.}\\
                           & = \val{\tau_1}(\ov{\gamma})(\ov{a}) & \comment{\gamma_1 \df \val{\tau_1}(\ov{\gamma})}\\
                           & = \val{\tau_1[\varsx/\ov{\vv{a}}]}(\ov{\gamma}) & \comment{\text{\hyperref[lem:rec:substitution]{Лема за замяната}}}\\
                           & \sqsubseteq \eval{\tau_1[\varsx/\ov{\vv{a}}]} & \comment{\text{\Prop{op-name-inclusion2}}}\\
                           & = \eval{\vv{f}_1(\vv{a}_1,\dots,\vv{a}_{m_1})} & \comment{\text{правило (cbn) на опер. сем.}}\\
                           & = \O\val{\vv{P}}(\ov{a}). & \comment\text{деф. на опер. сем.}
  \end{align*}
\end{proof}


%%% Local Variables:
%%% mode: latex
%%% TeX-master: "../sep"
%%% End:


% \subsection{Предаване на параметрите по стойстност}
\index{операционна семантика!по стойност}


В операционната семантика показва как свеждаме един {\em функционален} терм до естествено число или $\bot$.
Да разгледаме една програмата \vv{P}.

За всеки {\em функционален терм} $\mu$ дефинираме {\bf извод $\mu \to^P_V a$ с предаване на параметрите по стойност}
посредством индукция по построението на функционалния терм $\mu$.
\marginpar{Тук разликата, в сравнение с \cite{ditchev-soskov, winskel} е, че $\bot$ е константа в езика}
\marginpar{Да напомним, че функционален терм е терм без свободни обектови променливи}
\marginpar{В \cite{ditchev-soskov} се използва означението $\vv{P} \vdash_V \mu \to n$}
\begin{description}
\item
  % За произволно $a \in \Nat_\bot$,
  \begin{figure}[h!]
    \begin{prooftree}
      \AxiomC{}
      \RightLabel{\scriptsize{(1)}}
      \UnaryInfC{$\vv{a}\to^P_V a$}
    \end{prooftree}
  \end{figure}
\item
  \begin{figure}[h!]
    \begin{prooftree}
      \AxiomC{$\mu_1\to^P_V a_1$}
      \AxiomC{$\mu_2\to^P_V a_2$}
      \AxiomC{$a = \texttt{plus}(a_1, a_2)$}
      \RightLabel{\scriptsize{$(2_+)$}}
      \TrinaryInfC{$\mu_1 + \mu_2 \to^P_V a$}
    \end{prooftree}
  \end{figure}
\item
  \begin{figure}[h!]
    \begin{prooftree}
      \AxiomC{$\mu_1\to^P_N a_1$}
      \AxiomC{$\mu_2\to^P_N a_2$}
      \AxiomC{$a = \texttt{eq}(a_1, a_2)$}
      \RightLabel{\scriptsize{$(2_{\vv{==}})$}}
      \TrinaryInfC{$\mu_1\ \vv{==}\ \mu_2 \to^P_N a$}
    \end{prooftree}
  \end{figure}
\item
  \begin{figure}[h!]
    \begin{prooftree}
      \AxiomC{$\mu_0\to^P_V a_0$}
      \AxiomC{$\mu_1 \to^P_V a_1$}
      \AxiomC{$a_0 \in \Nat^+$}
      \RightLabel{\scriptsize{$(3_\true)$}}
      \TrinaryInfC{$\ifelse{\mu_0}{\mu_1}{\mu_2} \to^P_V a_1$}
    \end{prooftree}
  \end{figure}  
\item
  \begin{figure}[h!]
    \begin{prooftree}
      \AxiomC{$\mu_0\to^P_V 0$}
      \AxiomC{$\mu_2 \to^P_V a_2$}
      \RightLabel{\scriptsize{$(3_\false)$}}
      \BinaryInfC{$\ifelse{\mu_0}{\mu_1}{\mu_2} \to^P_V a_2$}
    \end{prooftree}
  \end{figure}
\item
  \begin{figure}[h!]
    \begin{prooftree}
      \AxiomC{$\mu_0\to^P_V \bot$}
      \RightLabel{\scriptsize{$(3_\bot)$}}
      \UnaryInfC{$\ifelse{\mu_0}{\mu_1}{\mu_2} \to^P_V \bot$}
    \end{prooftree}
  \end{figure}
\item
  \begin{figure}[h!]
    \begin{prooftree}
      \AxiomC{$\mu_1\to^P_V a_1$}
      \AxiomC{$\cdots$}
      \AxiomC{$\mu_{m_i}\to^P_V a_{m_i}$}
      \AxiomC{$\tau_i[\vv{x}_1/\vv{a}_1,\dots,\vv{x}_{m_i}/\vv{a}_{m_i}] \to^P_V a$}
      \AxiomC{$\bot \not\in \{a_1,\dots, a_{m_i}\}$}
      \RightLabel{\scriptsize{$(4_\Nat)$}}
      \QuinaryInfC{$\vv{f}_i(\mu_1,\dots,\mu_{m_i}) \to^P_V a$}
    \end{prooftree}
  \end{figure}
\item
  \begin{figure}[h!]
    \begin{prooftree}
      \AxiomC{$\mu_1\to^P_V a_1$}
      \AxiomC{$\cdots$}
      \AxiomC{$\mu_{m_i}\to^P_V a_{m_i}$}
      \AxiomC{$\bot\in\{a_1,\dots, a_{m_i}\}$}
      \RightLabel{\scriptsize{$(4_\bot)$}}
      \QuaternaryInfC{$\vv{f}_i(\mu_1,\dots,\mu_{m_i}) \to^P_V \bot$}
    \end{prooftree}
  \end{figure}
\end{description}

\Stefan{Всъщност това правило $3_\bot$ за какво ми е ? Ако го махна, всичко ще продължи да си е ОК.}

\marginpar{Докато доказателства за денотационна семантика протичаха с индукция по построението на термовете,
  доказателствата за операционна семантика обикновено протичат с индукция по дължината на извода.}

За фиксираната програма $\vv{P}$, с всеки {\em функционален} терм $\mu[\varsf]$ асоциираме 
\[\evalv{\mu}\dff
\begin{cases}
  b, & \text{ ако }\mu \to^P_V b\\
  \bot, & \text{ ако }\mu \text{ няма извод до константа}.
\end{cases}\]

\begin{framed}
  % \index{$\O_V$}
  Операционната семантика по стойност на рекурсивната програма $\vv{P}[\varsx,\varsf]$ представлява изображението
  $\O_V\val{\vv{P}} \in \Strict{\Nat^{m_0}_\bot}{\Nat_\bot}$
  дефинирано като
  \[\O_V\val{\vv{P}}(a_1,\dots,a_{m_1}) \dff \evalv{\vv{f}_1(\vv{a}_1,\dots,\vv{a}_{m_1})},\]
  за произволни $a_1,\dots,a_{m_1} \in \Nat_\bot$.
\end{framed}

\begin{remark}
  Можем директно да докажем, че $\O_V\val{\vv{P}}$ е точно изображение.
  Но това е излишно. Това свойство ще следва от теоремата за еквивалентност, която ще докажем след малко, 
  защото вече знаем, че денотационната семантика по стойност представлява точно изображение.
\end{remark}

\begin{example}
  Да разгледаме отново програмата \vv{P}, където:
  \begin{haskellcode}
f(x) = g(f(x)) where
  g(x) = 0
  \end{haskellcode}
  За нея вече видяхме, че $\D_V\val{\vv{P}} \sqsubseteq \D_N\val{\vv{P}}$,
  но $\D_V\val{\vv{P}} \neq \D_N\val{\vv{P}}$.
  
  Да разгледаме операционната семантика на тази програма.
  Лесно се съобразява, че $\vv{f(1)} \to^P_N 0$.
  
  \begin{prooftree}
    \AxiomC{}
    \RightLabel{\scriptsize{правило (1)}}
    \UnaryInfC{$\vv{0}[\vv{x}/\vv{g(f(1))}] \to^P_N 0$}
    \RightLabel{\scriptsize{правило (4)}}
    \UnaryInfC{$\vv{g(f(x))}[\vv{x}/\vv{1}] \to^P_N 0$}
    \RightLabel{\scriptsize{правило (4)}}
    \UnaryInfC{$\vv{f(1)} \to^P_N 0$}
  \end{prooftree}

  
  Да видим защо $\vv{f(1)}$ няма извод до елемент в операционната семантика по стойност.

  \begin{prooftree}
    \AxiomC{}
    \LeftLabel{\scriptsize{(1)}}
    \UnaryInfC{$\vv{1} \to^P_V 1$}
    \AxiomC{$\vdots$}
    \LeftLabel{\scriptsize{правило $(4_\Nat)$}}
    \UnaryInfC{$\vv{f(1)} \to^P_V \square$}
    \AxiomC{}
    \LeftLabel{\scriptsize{(1)}}
    \UnaryInfC{$\vv{0}[\vv{x}/\square] \to^P_V 0$}
    \RightLabel{\scriptsize{$(4_\bot)$ или $(4_\Nat)$}}
    \BinaryInfC{$\vv{g(f(x))}[\vv{x}/\vv{1}] \to^P_V \square$}
    \RightLabel{\scriptsize{правило $(4_\Nat)$}}
    \BinaryInfC{$\vv{f(1)} \to^P_V \square$}
  \end{prooftree}

\end{example}


%%% Local Variables:
%%% mode: latex
%%% TeX-master: "../sep-notes"
%%% End:


% \subsection{Теорема за еквивалентност}

\begin{proposition}
  \label{pr:rec:op-value-inclusion1}
  \marginpar{Сравнете с \Prop{rec:op-name-inclusion1}}
  Нека $\mu$ е {\em функционален} терм и
  \[\bar{\delta} \in \DomOpCBV\]
  е решение на системата от уравнения $\Delta_{\tau_i}$ съответстващи на програмата \vv{P}.
  \marginpar{Тук не е необходимо $\bar{\delta}$ да бъде най-малкото решение}
  Тогава 
  \marginpar{\cite[стр. 182]{ditchev-soskov}}
  \[\evalv\mu \sqsubseteq \val{\mu}(\ov{\delta}).\]
\end{proposition}
\begin{proof}
  \marginpar{Тук до голяма степен повтаряме същите разсъждения както в доказателството на \Prop{rec:op-name-inclusion1}}
  Ако от терма $\mu$ {\em няма извод} до елемент на $\Nat_\bot$, то
  по дефиниция $\evaln{\mu} = \bot$ и в този случай е очевидно, че
  \[\evalv{\mu} \sqsubseteq \val{\mu}(\ov{\delta}).\]

  \marginpar{Ако $\mu \to^P_V \bot$, то е ясно, че $\evalv{\mu} \sqsubseteq \val{\mu}(\ov{\delta})$}
  Интересният случай е когато от терма $\mu$ {\em има извод} до елемент на $\Nat$.
  Тогава ще докажем, че за произволен функционален терм $\mu$ и елемент $a \in \Nat$, 
  \begin{equation}
    \label{eq:18}
    \mu\to^P_V a\ \implies \val{\mu}(\ov{\delta}) = a.
  \end{equation}
  Доказателството на (\ref{eq:18}) ще проведем с индукция по дължината $\ell$ на извода $\mu\to^P_V a$.

  Първо, нека $\ell = 1$. Тогава единствения случай, който трябва да разгледаме е $\mu \equiv \vv{a}$.
  Този случай е прекалено лесен.

  Нека сега изводът $\mu \to^P_V a$ има дължина $\ell > 1$.
  Понеже правилата за извод строго следват дефиницията на термовете, 
  трябва да разгледаме следните случаи.
  \begin{itemize}
  \item
    Нека $\mu \equiv \mu_1 + \mu_2$, като $\evalv{\mu} = a$. Тогава:
    \begin{prooftree}
      \AxiomC{$\vdots$}
      \LeftLabel{\scriptsize{(извод с дълж. $\ell_1$)}}
      \UnaryInfC{$\mu_1 \to^P_V a_1$}
      \AxiomC{$\vdots$}
      \RightLabel{\scriptsize{(извод с дълж. $\ell_2$)}}
      \UnaryInfC{$\mu_2 \to^P_V a_2$}
      \RightLabel{\scriptsize{\text{правило }$(2_+)$}}
      \BinaryInfC{$\mu \to^P_V a$}
    \end{prooftree}
    Тук имаме, че $\ell = \ell_1 + \ell_2 + 1$ и $a = \texttt{plus}(a_1,a_2)$.
    Следователно можем да приложим {\bf И.П.} за $\mu_1$ и $\mu_2$, откъдето получаваме, че
    \begin{align*}
      & \mu_1 \to^P_V a_1\ \implies \val{\mu_1}(\ov{\delta}) = a _1\\
      & \mu_2 \to^P_V a_2\ \implies \val{\mu_2}(\ov{\delta}) = a _2.
    \end{align*}
    % Получаваме, че:
    % \begin{align*}
    %   & \evalv{\mu_1} = a_1 \sqsubseteq \val{\mu}(\ov{\delta})\\
    %   & \evalv{\mu_2} = a_2 \sqsubseteq \val{\mu}(\ov{\delta}).
    % \end{align*}
    % Тогава, понеже изображението $\texttt{plus}$ е непрекъснато, а следователно и монотонно, то
    % можем да заключим, че 
    Тогава получаваме, че ако $\mu_1 + \mu_2 \to^P_V a$, то
    \begin{align*}
      \val{\mu_1 + \mu_2}(\ov{\delta}) & \dff \texttt{plus}(\val{\mu_1}(\ov{\gamma}), \val{\mu_2}(\ov{\delta}))\\
                                       & = \texttt{plus}(a_1,a_2)\\
                                       & = a.
    \end{align*}
  \item
    Нека $\mu \equiv \mu_1\ \vv{==}\ \mu_2$. Този случай не крие изненади.
    \marginpar{\writedown Домашно!}
  \item
    Нека $\mu \equiv \ifelse{\tau_0}{\tau_1}{\tau_2}$.    
  \item
    Нека $\mu \equiv \vv{f}_i(\mu_1,\dots,\mu_{m_i})$ и да приемем, че $\mu \to^P_V a$.
    Според правилата на операционната семантика, единствената възможна ситуация е следната.
    Съществуват елементи $a_1,\dots,a_{m_i} \in \Nat$, за които:
    % \begin{itemize}
    % \item 
    %   Ако $\bot \not\in \{a_1,\dots,a_{m_i}\}$, където:
    \begin{prooftree}
      \AxiomC{$\vdots$}
      \LeftLabel{\scriptsize(дълж. $\ell_1$)}
      \UnaryInfC{$\mu_1\to^P_V a_1$}
      \AxiomC{$\cdots$}
      \AxiomC{$\vdots$}
      \LeftLabel{\scriptsize($\ell_{m_i}$)}
      \UnaryInfC{$\mu_{m_i} \to^P_V a_{m_i}$}
      \AxiomC{$\vdots$}
      \RightLabel{\scriptsize(дълж. $\ell_0$)}
      \UnaryInfC{$\tau_i[\varsx/\bar{\vv{a}}] \to^P_V a$}
      \LeftLabel{\scriptsize{(дълж. $\ell$)}}
      \RightLabel{\scriptsize{правило ($4_\Nat$)}}
      \QuaternaryInfC{$\vv{f}_i(\mu_1,\dots,\mu_{m_i}) \to^P_V a$}
    \end{prooftree}

    Понеже за $j = 1,\dots, m_i$ изводът на $\mu_j\to^P_V a_j$ има дължина $\ell_j < \ell$,
    то от {\bf И.П.} за (\ref{eq:18}) имаме, че 
    \begin{equation}
      \label{eq:20}
      \val{\mu_j}(\bar{\delta}) = a_j.
    \end{equation}
    \marginpar{$\ell_0 + \ell_1+\cdots \ell_{m_i} + 1 = \ell$}
    Аналогично, понеже изводът на $\tau_i[\varsx/\bar{\vv{a}}] \to^{\vv{P}}_V a$ има дължина $\ell_0 < \ell$,
    и очевидно термът $\tau_i[\varsx/\bar{\vv{a}}]$ е функционален, то от {\bf И.П.} за (\ref{eq:18}) имаме, че 
    \begin{equation}
      \label{eq:3}
      \val{\tau_i[\varsx/\bar{\vv{a}}]}(\bar{\delta}) = a.
    \end{equation}
    
    Получаваме следното:
    \begin{align*}
      \val{\mu}(\ov{\delta}) & = \val{\vv{f}_i(\mu_1,\dots,\mu_{m_i})}(\ov{\delta}) \\
                             & = \delta_i(\underbrace{\val{\mu_1}(\bar{\delta})}_{a_1},\dots,\underbrace{\val{\mu_{m_i}}(\bar{\delta})}_{a_{m_i}}) & \comment{\text{стойност на терм}}\\
                             & = \delta_i(a_1,\dots,a_{m_i}) & \comment{\text{от (\ref{eq:20})}}\\
                             & = \Delta_{\tau_i}(\bar{\delta})(a_1,\dots,a_{m_i}) & \comment{\delta_i = \Delta_{\tau_i}(\ov{\delta})}\\
                             & = \Sigma_{\star}(\Gamma_{\tau_i}(\ov{\delta}))(\bar{a}) & \comment{\Delta_{\tau_i} \dff \Sigma_{\star} \circ \Gamma_{\tau_i}}\\
                             & = \Gamma_{\tau_i}(\bar{\delta})(\ov{a}) & \comment{\text{защото }\bot\not\in\{a_1,\dots,a_n\}}\\
                             & \dff \val{\tau_i}(\ov{\delta})(\ov{a}) & \comment{\ov{\delta}\text{ са непрекъснати}}\\
                             & = \val{\tau_i[\varsx/\bar{\vv{a}}]}(\bar{\delta}) & \comment{\text{\hyperref[lem:rec:substitution]{Лема за замяната}}}\\
                             & = a & \comment{\text{от (\ref{eq:3})}}
    \end{align*}
  % \item
  %   Ако $\bot \in \{a_1,\dots,a_{m_i}\}$, където:
  %   \begin{prooftree}
  %     \AxiomC{$\vdots$}
  %     \LeftLabel{\scriptsize(дълж. $l_1$)}
  %     \UnaryInfC{$\mu_1\to^P_V a_1$}
  %     \AxiomC{$\cdots$}
  %     \AxiomC{$\vdots$}
  %     \LeftLabel{\scriptsize($l_{m_i}$)}
  %     \UnaryInfC{$\mu_{m_i} \to^P_V a_{m_i}$}
  %     \RightLabel{\scriptsize{правило ($4_\bot$)}}
  %     \TrinaryInfC{$\vv{f}_i(\mu_1,\dots,\mu_{m_i}) \to^P_V \bot$}
  %   \end{prooftree}
  %   \marginpar{$l_0 + l_1+\cdots l_{m_i} = l$ и $l_j \geq 1$}
  %   В този случай е ясно, че $a = \bot$. Тогава понеже за $j = 1,\dots, m_i$ изводът на $\mu_j\to^P_V a_j$ има дължина $l_j < l$,
  %   то от {\bf И.П.} за (\ref{eq:18}) имаме, че 
  %   \begin{equation}
  %     \label{eq:19}
  %     \val{\mu_j}(\bar{\delta}) = a_j.
  %   \end{equation}
  %   Накрая получаваме следното:
  %   \begin{align*}
  %     \val{\mu}(\ov{\delta}) & = \val{\vv{f}_i(\mu_1,\dots,\mu_{m_i})}(\ov{\delta}) \\
  %                            & \dff \delta_i(\underbrace{\val{\mu_1}(\bar{\delta})}_{a_1},\dots,\underbrace{\val{\mu_{m_i}}(\bar{\delta})}_{a_{m_i}}) & \comment{\text{ стойност на терм}}\\
  %                            & = \delta_i(a_1,\dots,a_{m_i}) & \comment{\text{от (\ref{eq:18})}}\\
  %                            & = \bot. & \comment{\delta_i\text{ е точно изображение}}
  %   \end{align*}
  % \end{itemize}
  \end{itemize}
\end{proof}

Обърнете внимание, че дотук използвахме само, че $\bar{\delta}$ е решение на системата от оператори $\Delta_{\tau_i}$ зададена от програмата \vv{P}. 
Не сме използвали, че $\ov{\delta}$ е най-малкото решение.

\begin{framed}
  \begin{cor}
    \label{cr:ov-in-dv}
    За всяка рекурсивна програма $\vv{P}$ на езика $\REC$  имаме, че
    \[\O_V\val{\vv{P}} \sqsubseteq \D_V\val{\vv{P}}.\]
  \end{cor}
\end{framed}
\begin{proof}
  \marginpar{Озн. с $\ov{\delta}$ най-малкото решение на системата от оператори за програмата \vv{P}} Тогава
  \begin{align*}
    \O_V\val{\vv{P}}(\ov{a}) & \dff \evalv{\vv{f}_1(\vv{a}_1,\dots,\vv{a}_{m_1})} \\
                             & \sqsubseteq \val{\vv{f}_1(\vv{a}_1,\dots,\vv{a}_{m_1})}(\ov{\delta}) & \comment{\text{от \Prop{rec:op-value-inclusion1}}}\\
                             & = \delta_1(\val{\vv{a}_1}(\ov{\delta}), \dots, \val{\vv{a}_{m_1}}(\ov{\delta})) \\
                             & = \delta_1(a_1,\dots,a_{m_1})\\
                             & \dff \D_V\val{\vv{P}}(\ov{a}).
  \end{align*}

  
  % \begin{itemize}
  % \item 
  %   Нека $\bot \not\in \{a_1,\dots,a_{m_0}\}$. Тогава:
  %   \begin{align*}
  %     \O_V\val{\vv{P}}(\ov{a}) & \dff \evalv{\vv{f}_0(\vv{a}_1,\dots,\vv{a}_{m_0})} \\
  %                              & = \evalv{\tau_0[\varsx/\ov{\vv{a}}]} & \comment{\text{Правило }(4_\Nat)}\\
  %                              & \sqsubseteq \val{\tau_0[\varsx/\ov{\vv{a}}]}(\ov{\delta})& (\text{от \Prop{rec:op-value-inclusion1}})\\
  %                              & = \val{\tau_0}(\ov{\delta})(\ov{a}) & \comment{\text{\hyperref[lem:rec:substitution]{Лема за замяната}}}\\
  %                              & \dff \Gamma_{\tau_0}(\ov{\delta})(\ov{a})\\
  %                              & = \Sigma_{\star}(\Gamma_{\tau_0}(\ov{\delta})(\ov{a}) & \comment{\bot\not\in\{a_1,\dots,a_{m_0}\}}\\
  %                              & \dff \Delta_{\tau_0}(\ov{\delta})(\ov{a}) & \comment{\Delta_{\tau_0} = \Sigma_{\star} \circ \Gamma_{\tau_0}}\\
  %                              & = \delta_0(\ov{a}) & \comment{\delta_0 = \Delta_{\tau_0}(\ov{\delta})}\\
  %                              & \dff \D_V\val{\vv{P}}(\ov{a}).
  %   \end{align*}
  % \item
  %   Нека $\bot \in \{a_1,\dots,a_{m_0}\}$. Тогава:
  %   \begin{align*}
  %     \O_V\val{\vv{P}}(\ov{a}) & \dff \evalv{\vv{f}_0(\vv{a}_1,\dots,\vv{a}_{m_0})} \\
  %                              & = \bot & \comment{\text{Правило }(4_\bot)}\\
  %                              & \sqsubseteq \D_V\val{\vv{P}}.
  %   \end{align*}
  % \end{itemize}
\end{proof}


Обръщаме нашето внимание към доказателството на обратната посока, т.е. $\D_V\val{h} \sqsubseteq \O_V\val{h}$. 
Нека сега $\ov{\delta}$ е най-малкото решение на системата от оператори, която съответства на програма \vv{P}
за денотационната семантика по стойност.
Да напомним, че това означава, че $\ov{\delta} = \bigsqcup_r \ov{\delta}_r$, 
където $\delta^i_{r+1} = \Delta_{\tau_i}(\ov{\delta}_r)$

\begin{proposition}
  \label{pr:rec:op-value-inclusion2}
  Тогава за всеки {\em функционален} терм $\mu[\vv{f}_1,\dots,\vv{f}_k]$ и всяко $r$,
  \[\val{\mu}(\ov{\delta}_r) \sqsubseteq \evalv{\mu}.\]
\end{proposition}
\begin{hint}
  \marginpar{Сравнете с \Prop{op-name-inclusion2}. Основната разлика е, че тук можем да минем само с функционални термове. Това не можем да направим при семантиката по име поради разликата в правилата за извод. Всъщност можем да го направим, ако имаме лема за симулацията, но тя сега е сложена като задача}
  
  Доказателството на това твърдение следва същата схема като доказателството на \Prop{op-name-inclusion2}.
  За произволно естествено число $r$, нека твърдението $\texttt{Include}(r)$ да гласи следното:

  ,,за произволен функционален терм $\mu[\vv{f}_1,\dots,\vv{f}_l]$
  е изпълнено, че:
  \[\val{\mu}(\ov{\delta}_r) \sqsubseteq \evalv{\mu}.\text{''}\]
  
  Трябва да докажем, че $\texttt{Include}(r)$ е изпълнено за всяко $r$.
  Това ще направим с индукция по $r$.


  \begin{itemize}
  \item 
    Първо ще докажем $\texttt{Include}(0)$.
    Това ще направим с индукция по построението на терма $\tau$.
    Доказателството протича по същия начин ( с индукция по построението на термовете ) както доказателството в този случай на \Prop{op-name-inclusion2}.
    \begin{itemize}
    \item
      Нека $\mu \equiv \vv{a}$.
    \item
      Нека $\mu \equiv \mu_1 + \mu_2$.
      От {\bf И.П.} имаме, че за $i = 1,2$ е изпълнено
      \[\val{\mu_i}(\ov{\delta}_0) \sqsubseteq \evalv{\mu_i}.\]
      Тогава
      \begin{align*}
        \val{\mu_1 + \mu_2}(\ov{\delta}_0) & = \texttt{plus}(\val{\mu_1}(\ov{\delta}_0), \val{\mu_2}(\ov{\delta}_0))\\
                                           & \sqsubseteq \texttt{plus}(\evalv{\mu_1}, \evalv{\mu_2}) & \comment{\text{ от И.П. и мон. на }\texttt{plus}}\\
                                           & = \evalv{\mu_1 + \mu_2} & \comment{\text{ от правило }(2_+)}
      \end{align*}
    \item
      Нека $\mu \equiv \mu_1 \vv{==} \mu_2$.
    \item
      Нека $\mu \equiv \ifelse{\mu_1}{\mu_2}{\mu_3}$.
    \item
      Нека $\mu \equiv \vv{f}_i(\mu_1,\dots,\mu_{m_i})$.
    \end{itemize}
  \item 
    Нека $r > 0$. Да приемем, че $\texttt{Include}(r-1)$ е изпълнено. Ще докажем $\texttt{Include}(r)$
    с индукция по построението на термовете.
    Единственият случай, който заслужава внимание е 
    \[\mu \equiv \vv{f}_i(\mu_1,\dots,\mu_{m_i}).\]
    Доказателствата на всички останали случаи за $\mu$ протичат по абсолютно същия начин както при $r = 0$.
    
    Понеже термът $\mu$ е построен с помощта на термовете $\mu_j$, за $j = 1, \dots, m_i$,
    можем да приложим {\bf И.П.} за тях и да получим, че 
    \[b_j \dff \val{\mu_{j}}(\ov{\delta}_r) \sqsubseteq \evalv{\mu_j}.\]
    Трябва да разгледаме два случая.
    \begin{itemize}
    \item 
      Ако $\bot \not\in \{b_1,\dots,b_{m_i}\}$. Тогава, понеже работим в плоска наредба, \[\evalv{\mu_j} = b_j.\]
      От правилата на операционната семантика, това означава, че в този случай имаме следния извод:
      \begin{figure}[h!]
        \begin{prooftree}
          \AxiomC{$\vdots$}
          \UnaryInfC{$\mu_1\to^P_V b_1$}
          \AxiomC{$\cdots$}
          \AxiomC{$\vdots$}
          \UnaryInfC{$\mu_{m_i}\to^P_V b_{m_i}$}
          \AxiomC{$\vdots$}
          \UnaryInfC{$\tau_i[\vv{x}_1/\vv{b}_1,\dots,\vv{x}_{m_i}/\vv{b}_{m_i}] \to^P_V b$}
          \RightLabel{$(4_\Nat)$}
          \QuaternaryInfC{$\vv{f}_i(\mu_1,\dots,\mu_{m_i}) \to^P_V b$}
        \end{prooftree}
      \end{figure}
      
      Оттук следва, че:
      \begin{equation}
        \label{eq:14}
        \evalv{\vv{f}_i(\mu_1,\dots,\mu_{m_i})} = \evalv{\tau_i[\vv{x}_1/\vv{b}_1,\dots,\vv{x}_{m_i}/\vv{b}_{m_i}]}.
      \end{equation}
      Получаваме, че:
      \begin{align*}
        \val{\mu}(\ov{\delta}_r) & = \val{\vv{f}_i(\mu_1,\dots,\mu_{m_i})}(\ov{\delta}_r)\\
                                 & \dff \delta^i_r(\underbrace{\val{\mu_1}(\ov{\delta}_r)}_{b_1}, \dots, \underbrace{\val{\mu_{m_i}}(\ov{\delta}_r)}_{b_{m_i}}) & \comment{\ov{\delta}_r = (\delta^1_r,\dots,\delta^k_r)}\\
                                 & = \delta^i_r(b_1,\dots,b_{m_i}) \\
                                 & = \Delta_{\tau_i}(\ov{\delta}_{r-1})(b_1,\dots,b_{m_i}) & \comment{\delta^i_r = \Delta_{\tau_i}(\ov{\delta}_{r-1})}\\
                                 & = \Sigma_{\star}(\Gamma_{\tau_i}(\ov{\delta}_{r-1}))(b_1,\dots,b_{m_i}) & \comment{\Delta_{\tau_i} \dff \Sigma_{\star} \circ \Gamma_{\tau_i}}\\
                                 & = \Gamma_{\tau_i}(\ov{\delta}_{r-1})(b_1,\dots,b_{m_i}) & \comment{\text{всяко }b_j \neq \bot}\\
                                 & \dff \val{\tau_i}(\ov{\delta}_{r-1})(b_1,\dots,b_{m_i}) & \comment{\ov{\delta}_{r-1}\text{ са непрекъснати}}\\
                                 & = \val{\tau_i[\vv{x}_1/\vv{b}_1,\dots,\vv{x}_{m_i}/\vv{b}_{m_i}]}(\ov{\delta}_{r-1}) & \comment{\text{\hyperref[lem:rec:substitution]{Лема за замяната}}}\\
                                 & \sqsubseteq \evalv{\tau_i[\vv{x}_1/\vv{b}_1,\dots,\vv{x}_{m_i}/\vv{b}_{m_i}]} & \comment{\text{от }\texttt{Include}(r-1)}\\
                                 & = \evalv{\vv{f}_i(\mu_1,\dots,\mu_{m_i})} & \comment{\text{от (\ref{eq:14})}}\\
                                 & = \evalv{\mu}.
      \end{align*}
    \item
      Нека $\bot \in \{b_1,\dots,b_{m_i}\}$. Тогава е лесно, защото:
      \begin{align*}
        \val{\mu}(\ov{\delta}_r)  & = \val{\vv{f}_i(\mu_1,\dots,\mu_{m_i})}(\ov{\delta}_r)\\
                                  & \dff \delta^i_r(\underbrace{\val{\mu_1}(\ov{\delta}_r)}_{b_1}, \dots, \underbrace{\val{\mu_{m_i}}(\ov{\delta}_r)}_{b_{m_i}})  & \comment{\ov{\delta}_r = (\delta^1_r,\dots,\delta^k_r)}\\
                                  & = \delta^i_r(b_1,\dots,b_{m_i}) & \comment{b_j \dff \val{\mu_j}(\ov{\delta}_r)}\\
                                  & = \bot & \comment{\delta^i_r\text{ е точно изображение}}\\
                                  & \sqsubseteq \evalv{\vv{f}_i(\mu_1,\dots,\mu_{m_i})}\\
                                  & = \evalv{\mu}.
      \end{align*}      
    \end{itemize}
  \end{itemize}
\end{hint}

\begin{cor}
  \label{cr:rec:equivalence-cbv-inclusion2}
  Нека $\mu$ е функционален терм.
  Тогава $\evalv{\mu}$ е горна граница на веригата $(\val{\mu}(\ov{\delta}_r))^{\infty}_{r=0}$.
\end{cor}

\begin{lemma}
  За всяка рекурсивна програма $\vv{P}$,
  произволен {\em функционален} терм $\mu$,
  \[\val{\mu}(\ov{\delta}) \sqsubseteq \evalv{\mu},\]
  където $\delta = \lfp(\Delta)$, а $\Delta$ е операторът, който съответства на системата от 
  уравнения за $\vv{P}$.
\end{lemma}
\begin{hint}
  Знаем, че $\ov{\delta} = \bigsqcup_r \ov{\delta}_r$. Тогава:
  \begin{align*}
    \val{\mu}(\ov{\delta}) & = \val{\mu}(\bigsqcup_r\ov{\delta}_r)\\
                           & = \bigsqcup_r \val{\mu}(\ov{\delta}_r) & \comment{\text{от \Cor{rec:equivalence-cbv-inclusion2}}}\\
                           & \sqsubseteq \evalv{\mu}. & \comment{\text{от \Prop{rec:op-value-inclusion2}}}
  \end{align*}
\end{hint}

\begin{framed}
  \begin{theorem}
    \label{th:dv-equivalent-ov}
    За всяка рекурсивна програма $\vv{P}$ на езика $\REC$ е изпълнено:
    \[\D_V\val{\vv{P}} = \O_V\val{\vv{P}}.\]
  \end{theorem}
\end{framed}
\begin{proof}
  От \Cor{ov-in-dv} знаем, че 
  \[\O_V\val{\vv{P}} \sqsubseteq \D_V\val{\vv{P}}.\]
  Остава да докажем другата посока, а именно, че 
  \[\D_V\val{\vv{P}} \sqsubseteq \O_V\val{\vv{P}}.\]
  Но това е лесно:

  \begin{align*}
    \D_V\val{\vv{P}}(\ov{a}) & \dff \delta_1(\ov{a})\\
                             & = \delta_1(\val{\vv{a}_1}(\ov{\delta}),\dots,\val{\vv{a}_{m_1}}(\ov{\delta})) & \comment{ a_j = \val{\vv{a}_j}(\ov{\delta})}\\
                             & = \val{\vv{f}_1(\vv{a}_1,\dots,\vv{a}_{m_1})}(\ov{\delta}) & \comment{\text{ деф. на стойност на терм}}\\
                             % & = \Delta_{\tau_1}(\ov{\delta})(\ov{a}) & \comment{\delta_1 = \Delta_{\tau_1}(\ov{\delta})}\\
                             % & = \Sigma_{m_1}(\Gamma_{\tau_1}(\ov{\delta}))(\ov{a}) & \comment{\Delta_{\tau_1} \dff \Sigma_{m_1}\circ\Gamma_{\tau_1}}\\
                             % & = \Gamma_{\tau_1}(\ov{\delta})(\ov{a}) &  \comment{\text{защото }a_j \neq \bot}\\
                             % & \dff \val{\tau_1}(\bar{a},\bar{\delta}) \\
                             % & = \val{\tau_1[\varsx/\ov{\vv{a}}]}(\ov{\delta}) & \comment{\text{от \hyperref[lem:rec:substitution]{Лема за замяната}}}\\
                             & \sqsubseteq \evalv{\vv{f}_1(\vv{a}_1,\dots\vv{a}_{m_1})} & \comment{\text{от \Prop{rec:op-value-inclusion2}}}\\
                             & \dff \O_V\val{\vv{P}}(\ov{a}).
  \end{align*}

  
  
  % \begin{itemize}
  % \item 
  %   Ако $\bot \not\in\{a_1,\dots,a_{m_0}\}$, то 
  %   \begin{align*}
  %     \D_V\val{\vv{P}}(\ov{a}) & \dff \delta_0(\ov{a})\\
  %                               & = \Delta_{\tau_1}(\ov{\delta})(\ov{a}) & \comment{\delta_1 = \Delta_{\tau_1}(\ov{\delta})}\\
  %                               & = \Sigma_{m_1}(\Gamma_{\tau_1}(\ov{\delta}))(\ov{a}) & \comment{\Delta_{\tau_1} \dff \Sigma_{m_1}\circ\Gamma_{\tau_1}}\\
  %                               & = \Gamma_{\tau_1}(\ov{\delta})(\ov{a}) &  \comment{\text{защото }a_j \neq \bot}\\
  %                               & \dff \val{\tau_1}(\bar{a},\bar{\delta}) \\
  %                               & = \val{\tau_1[\varsx/\ov{\vv{a}}]}(\ov{\delta}) & \comment{\text{от \hyperref[lem:rec:substitution]{Лема за замяната}}}\\
  %                               & \sqsubseteq \evalv{\tau_1[\varsx/\ov{\vv{a}}]} & \comment{\text{от \Prop{rec:op-value-inclusion2}}}\\
  %                               & = \evalv{\vv{f}_1(\vv{a}_1,\dots,\vv{a}_{m_1})} & \comment{\text{правило }(4_\Nat)}\\
  %                               & \dff \O_V\val{\vv{P}}(\ov{a}).
  %   \end{align*}
  % \item
  %   Ако $\bot \in \{a_1,\dots,a_{m_1}\}$, то в този случай, 
  %   \begin{align*}
  %     \D_V\val{\vv{P}}(\ov{a}) & \dff \delta_1(\ov{a})\\
  %                              & = \bot & \comment{\delta_1 \text{ е точно изображение}}\\
  %                              & \sqsubseteq \O_V\val{\vv{P}}(\ov{a}).
  %   \end{align*}
  % \end{itemize}
\end{proof}


%%% Local Variables:
%%% mode: latex
%%% TeX-master: "../sep"
%%% End:

% \newpage
% \section{Задачи}  

Тук с $\A$, $\B$ и $\C$ ще означаваме области на Скот.

\marginpar{Много от задачите са от \cite[стр. 31]{abramsky94}}

\begin{problem}
  Да разгледаме операторите \[\Gamma,\Delta \in \Cont{\Cont{\A}{\A}}{\Cont{\A}{\A}}.\]
  Знаем, че операторът $\Gamma \circ \Delta$ е непрекъснат, където
  \[(\Gamma\circ\Delta)(f) \df \Gamma(\Delta(f)).\]
  Вярно ли е, че
  \[\lfp(\Gamma \circ \Delta) \sqsubseteq \lfp(\Gamma) \circ \lfp(\Delta)?\]
  Обосновете се!
\end{problem}
\ifhints
\begin{hint}
  Нека $\A = \Nat_\bot$.
  Нека например да разгледаме
  \begin{align*}
    & \Delta(f)(x) \df f(x+1)\\
    & \Gamma(f)(x) \df
      \begin{cases}
        0, & x \neq \bot\\
        \bot, & x = \bot.
      \end{cases}
  \end{align*}
  Да положим $f_\Gamma \df \lfp(\Gamma)$ и $f_\Delta \df \lfp(\Delta)$.
  Ясно е, че 
  \begin{align*}
    & f_\Delta(x) = \bot\\
    & f_\Gamma(x) =
    \begin{cases}
      0, & x \neq \bot\\
      \bot, & x = \bot.
    \end{cases}  
  \end{align*}
  Тогава за произволно $x \in \Nat_\bot$,
  \[(f_\Gamma\circ f_\Delta)(x) = f_\Gamma(f_\Delta(x)) = f_\Gamma(\bot)  = \bot.\]
  От друга страна, понеже $(\Gamma \circ \Delta)(f) = \Gamma(\Delta(f))$, то 
  \begin{align*}
    & (\Gamma \circ \Delta)(f)(x) = \Gamma(\Delta(f))(x) = 
      \begin{cases}
        0, & x \neq \bot\\
        \bot, & x = \bot.
      \end{cases}
  \end{align*}
  Лесно се съобразява, че 
  \[\lfp(\Gamma \circ \Delta)(x) =
  \begin{cases}
    0, & x \neq \bot\\
    \bot, & x = \bot.
  \end{cases}\]
  Заключаваме, че 
  \[\lfp(\Gamma \circ \Delta) \sqsupset \lfp(\Gamma) \circ \lfp(\Delta).\]
\end{hint}
\fi

\begin{problem}
  Да разгледаме операторите \[\Gamma,\Delta \in \Cont{\Cont{\A}{\A}}{\Cont{\A}{\A}}.\]
  Знаем, че операторът $\Gamma \circ \Delta$ е непрекъснат, където
  \[(\Gamma\circ\Delta)(f) \df \Gamma(\Delta(f)).\]
  Вярно ли е, че
  \[\lfp(\Gamma \circ \Delta) \sqsupseteq \lfp(\Gamma) \circ \lfp(\Delta)?\]
  Обосновете се!
\end{problem}
\ifhints
\begin{hint}
  Нека $\A = \Nat_\bot$.
  Нека например да разгледаме
  \begin{align*}
    & \Delta(f)(x) \df 0\\
    & \Gamma(f)(x) \df
      \begin{cases}
        0, & x = 0\\
        \bot, & \text{ иначе}.
      \end{cases}
  \end{align*}
  Да положим $f_\Gamma \df \lfp(\Gamma)$ и $f_\Delta \df \lfp(\Delta)$.
  Ясно е, че 
  \begin{align*}
    & f_\Delta(x) = 0\\
    & f_\Gamma(x) =
    \begin{cases}
      0, & x = 0\\
      \bot, & \text{ иначе}.
    \end{cases}  
  \end{align*}
  Тогава за произволно $x \in \Nat_\bot$,
  \[(f_\Gamma\circ f_\Delta)(x) = f_\Gamma(f_\Delta(x)) = f_\Gamma(0)  = 0.\]
  От друга страна, понеже $(\Gamma \circ \Delta)(f) = \Gamma(\Delta(f))$, то 
  \begin{align*}
    & (\Gamma \circ \Delta)(f)(x) = \Gamma(\Delta(f))(x) = 
      \begin{cases}
        0, & x = 0\\
        \bot, & \text{ иначе}.
      \end{cases}
  \end{align*}
  Лесно се съобразява, че 
  \[\lfp(\Gamma \circ \Delta)(x) =
  \begin{cases}
    0, & x = 0\\
    \bot, & \text{ иначе}.
  \end{cases}\]
  Заключаваме, че 
  \[\lfp(\Gamma \circ \Delta) \sqsubset \lfp(\Gamma) \circ \lfp(\Delta).\]
\end{hint}
\fi

\begin{problem}
  Нека $f_0 \sqsubseteq f_1 \sqsubseteq f_2 \sqsubseteq \cdots$
  е верига от елементи на $\Cont{\A}{\A}$.
  Да положим $h = \bigsqcup_n f_n$.
  Вярно ли е, че 
  \[h \circ h = \bigsqcup_n (f\circ f)?\]
  Обосновете се!
\end{problem}

\begin{problem}
  Да разгледаме едно изображение $f: \A \times \B \to \C$.
  За произволно $a \in \A$, дефинираме изображението $g_a: \B \to \C$, където
  \[g_a(b) \df f(a,b).\]
  Аналогично, за произволно $b \in \B$, дефинираме изображението $h_b: \A \to \C$, където
  \[h_b(a) \df f(a,b).\]
  Докажете, че следните твърдения са еквивалентни:
  \begin{enumerate}[1)]
  \item 
    $f$ е непрекъснато изображение;
  \item
    $g_a$ и $h_b$ са непрекъснати изображения, за всяко $a \in \A$ и $b \in \B$.
  \end{enumerate}
\end{problem}

\begin{problem}
  Да разгледаме $f \in \Cont{\A \times \B}{\C}$.
  За произволно $a \in \A$, дефинираме изображението $g_a: \B \to \C$, където
  \[g_a(b) \df f(a,b).\]
  Вече знаем, че $g_a \in \Cont{\B}{\C}$, за всяко $a \in \A$.
  Да разгледаме изображението $h:\A \to \Cont{\B}{\C}$, където
  \[h(a) \df g_a.\]
  Докажете, че $h$ е непрекъснато изображение.
\end{problem}

\begin{problem}
  \marginpar{\cite[стр. 129]{nikolova-soskova}}
  Да разгледаме $f \in \Cont{\A \times \B}{\B}$.
  За произволно $a \in \A$, дефинираме изображението $g_a: \B \to \B$, където
  \[g_a(b) \df f(a,b).\]
  Вече знаем, че $g_a \in \Cont{\B}{\B}$, за всяко $a \in \A$,
  следователно $\lfp(g_a)$ съществува.
  Да разгледаме изображението $h:\A \to \B$, където
  \[h(a) \df \lfp(g_a).\]
  Докажете, че $h$ е непрекъснато изображение.
\end{problem}


\begin{problem}
  Да разгледаме $f \in \Cont{\A}{\Cont{\B}{\C}}$.
  За произволно $a \in \A$, дефинираме изображението $g_a \in \Cont{\B}{\C}$, където
  \[g_a(b) \df f(a).\]
  Да разгледаме изображението $h:\A\times \B \to \C$, където
  \[h(a,b) \df g_a(b).\]
  Докажете, че $h$ е непрекъснато изображение.
\end{problem}

% \begin{problem}
%   Нека са дадени областите на Скот $\D$ и $\E$ и изображението 
%   \[\texttt{eval}: \Cont{\D}{\E} \times \D \to \E,\]
%   където 
%   \[\texttt{eval}(f,d) \df f(d).\]
%   Докажете, че $\texttt{eval}$ е непрекъснато изображение.
% \end{problem}
% \ifhints
% \begin{hint}
%   Понеже $\bigsqcup_n(f_n,d_n) = (\bigsqcup_m f_m,\bigsqcup_n d_n)$, 
%   ще докажем, че \[\texttt{eval}(\bigsqcup_m f_m, \bigsqcup_n d_n) = \bigsqcup_n \texttt{eval}(f_n,d_n).\]
%   Знаем, че
%   \begin{align*}
%     \texttt{eval}(\bigsqcup_m f_m, \bigsqcup_n d_n) & = (\bigsqcup_m f_m)(\bigsqcup_n d_n) & (\mbox{от опр. на }\texttt{eval})\\
%     & = \bigsqcup_m (f_m(\bigsqcup_n d_n)) & (\mbox{от опр. на }\bigsqcup_mf_m)\\
%     & = \bigsqcup_m (\bigsqcup_n (f_m(d_n)) & (\mbox{всяка }f_m\mbox{ е непр.} )\\
%   \end{align*}
%   Нека да положим $e_{m,n} = f_m(d_n)$.
%   Лесно се съобразява, че
%   \[m \leq m^\prime\ \&\ n \leq n^\prime\ \Rightarrow\ e_{m,n} \sqsubseteq^\E e_{m^\prime,n^\prime}.\]
%   Така получаваме, че 
%   \begin{align*}
%     \texttt{eval}(\bigsqcup_m f_m, \bigsqcup_n d_n) & = \bigsqcup_m (\bigsqcup_n (f_m(d_n)) & (\mbox{от по-горе})\\
%     & = \bigsqcup_{m,n} e_{m,n} = \bigsqcup_{n} e_{n,n} & (\mbox{от \Th{double-chain}})\\
%     & = \bigsqcup_{n} f_n(d_n) & (\text{ от опр. на }e_{m,n})\\
%     & = \bigsqcup_n \texttt{eval}(f_n,d_n).
%   \end{align*}
% \end{hint}
% \fi

\begin{problem}
  Нека са дадени областите на Скот $\D$ и $\E$ и изображението 
  \[\texttt{eval}: \Cont{\D}{\E} \times \D \to \E,\]
  където 
  \[\texttt{eval}(f,d) \df f(d).\]
  Докажете, че $\texttt{eval}$ е непрекъснато изображение.
\end{problem}
\ifhints
\begin{hint}
  Понеже $\bigsqcup_n(f_n,d_n) = (\bigsqcup_m f_m,\bigsqcup_n d_n)$, 
  ще докажем, че \[\texttt{eval}(\bigsqcup_m f_m, \bigsqcup_n d_n) = \bigsqcup_n \texttt{eval}(f_n,d_n).\]
  Знаем, че
  \begin{align*}
    \texttt{eval}(\bigsqcup_m f_m, \bigsqcup_n d_n) & = (\bigsqcup_m f_m)(\bigsqcup_n d_n) & \comment\text{от опр. на }\texttt{eval}\\
                                                    & = \bigsqcup_m (f_m(\bigsqcup_n d_n)) & \comment\text{от опр. на }\bigsqcup_mf_m\\
                                                    & = \bigsqcup_m (\bigsqcup_n (f_m(d_n)) & \comment\text{всяка }f_m\mbox{ е непр.}\\
  \end{align*}
  Нека да положим $e_{m,n} = f_m(d_n)$.
  Лесно се съобразява, че
  \[m \leq m^\prime\ \&\ n \leq n^\prime\ \Rightarrow\ e_{m,n} \sqsubseteq^\E e_{m^\prime,n^\prime}.\]
  Така получаваме, че 
  \begin{align*}
    \texttt{eval}(\bigsqcup_m f_m, \bigsqcup_n d_n) & = \bigsqcup_m (\bigsqcup_n (f_m(d_n)) & \comment\text{от по-горе}\\
    & = \bigsqcup_{m,n} e_{m,n} = \bigsqcup_{n} e_{n,n} & \comment\text{от \Th{double-chain}}\\
    & = \bigsqcup_{n} f_n(d_n) & \comment\text{ от опр. на }e_{m,n}\\
    & = \bigsqcup_n \texttt{eval}(f_n,d_n).
  \end{align*}
\end{hint}
\fi

%%% Local Variables:
%%% mode: latex
%%% TeX-master: "../sep"
%%% End:

\begin{problem}
  Нека изображението \[\texttt{comp}:(\Cont{\B}{\C} \times \Cont{\A}{\B}) \to \Cont{\A}{\C}\]
  е определено като 
  \[\texttt{comp}(g,f) \df g\circ f.\]
  \marginpar{$(g \circ f)(a) = g(f(a))$}
  Докажете, че $\texttt{comp}$ е непрекъснато изображение.
\end{problem}
\ifhints
\begin{hint}
  \marginpar{\cite[стр. 124]{reynolds}}
  Трябва да докажем, че за всяка монотонно растяща редица $\{(g_n,f_n)\}_{n\in\Nat}$,
  \[\Gamma(\bigsqcup_n(g_n,f_n))(a) = \bigsqcup_n\Gamma(g_n,f_n)(a),\]
  за произволно $a \in A$.
  Да фиксираме едно $a\in A$ и да положим $g_n(f_k(a)) = e_{n,k}$.
  Лесно се вижда, че 
  \[n\leq n^\prime\ \&\ k \leq k^\prime\ \Rightarrow\ e_{n,k} \sqsubseteq e_{n^\prime,k^\prime}.\]
  Тогава:
  \begin{align*}
    \Gamma(\bigsqcup_n(g_n,f_n))(a) & = \Gamma(\bigsqcup_n g_n, \bigsqcup_k f_k)(a) & \\
                                    & = (\bigsqcup_n g_n)(\bigsqcup_k f_k(a)) & \comment\text{ по деф. на }\Gamma\\
                                    & = (\bigsqcup_n g_n)(\bigsqcup_k b_k) & \comment\text{ полагаме }b_k = f_k(a)\\
                                    & = \bigsqcup_k (\bigsqcup_n g_n)(b_k) & \comment\bigsqcup_n g_n\text{ е непр.}\\
                                    & = \bigsqcup_k(\bigsqcup_n g_n(b_k)) & \comment\text{ по деф. на }\bigsqcup_n g_n\\
                                    & = \bigsqcup_k (\bigsqcup_n g_n(f_k(a))) & \comment\text{ полагаме }e_{n,k} = g_n(f_k(a))\\
                                    & = \bigsqcup_k\bigsqcup_n e_{n,k} = \bigsqcup_n e_{n,n} & \comment\text{ от \Th{double-chain}}\\
                                    & = \bigsqcup_n g_n(f_n(a)) = \bigsqcup_n \Gamma(g_n, f_n)(a).
  \end{align*}
\end{hint}
\fi

\begin{remark}
  \marginpar{Когато на хаскел пишем $(.)$, означава, че операцията е инфиксна}
  В хаскел има операция композиция на функции.
  \begin{haskellcode}
ghci> :t (.)
(.) :: (b -> c) -> (a -> b) -> a -> c
  \end{haskellcode}
\end{remark}


%%% Local Variables:
%%% mode: latex
%%% TeX-master: "../sep"
%%% End:


% \begin{problem}
%   Нека изображението \[\texttt{comp}:(\Cont{\B}{\C} \times \Cont{\A}{\B}) \to \Cont{\A}{\C}\]
%   е определено като 
%   \[\texttt{comp}(g,f) \df g\circ f.\]
%   \marginpar{$(g \circ f)(a) = g(f(a))$}
%   Докажете, че $\texttt{comp}$ е непрекъснато изображение.
% \end{problem}
% \ifhints
% \begin{hint}
%   \marginpar{\cite[стр. 124]{reynolds}}
%   Трябва да докажем, че за всяка монотонно растяща редица $\{(g_n,f_n)\}_{n\in\Nat}$,
%   \[\Gamma(\bigsqcup_n(g_n,f_n))(a) = \bigsqcup_n\Gamma(g_n,f_n)(a),\]
%   за произволно $a \in A$.
%   Да фиксираме едно $a\in A$ и да положим $g_n(f_k(a)) = e_{n,k}$.
%   Лесно се вижда, че 
%   \[n\leq n^\prime\ \&\ k \leq k^\prime\ \Rightarrow\ e_{n,k} \sqsubseteq e_{n^\prime,k^\prime}.\]
%   Тогава:
%   \begin{align*}
%     \Gamma(\bigsqcup_n(g_n,f_n))(a) & = \Gamma(\bigsqcup_n g_n, \bigsqcup_k f_k)(a) & \\
%     & = (\bigsqcup_n g_n)(\bigsqcup_k f_k(a)) & (\text{ по деф. на }\Gamma )\\
%     & = (\bigsqcup_n g_n)(\bigsqcup_k b_k) & (\text{ полагаме }b_k = f_k(a))\\
%     & = \bigsqcup_k (\bigsqcup_n g_n)(b_k) & (\bigsqcup_n g_n\text{ е непр.})\\
%     & = \bigsqcup_k(\bigsqcup_n g_n(b_k)) & (\text{ по деф. на }\bigsqcup_n g_n)\\
%     & = \bigsqcup_k (\bigsqcup_n g_n(f_k(a))) & (\text{ полагаме }e_{n,k} = g_n(f_k(a)))\\
%     & = \bigsqcup_k\bigsqcup_n e_{n,k} = \bigsqcup_n e_{n,n} & (\text{ от \Th{double-chain}})\\
%     & = \bigsqcup_n g_n(f_n(a)) = \bigsqcup_n \Gamma(g_n, f_n)(a).
%   \end{align*}
% \end{hint}
% \fi

% \begin{remark}
%   \marginpar{Когато пишем $(.)$ означава, че операцията е инфиксна}
%   В хаскел има операция композиция на функции.
%   \begin{haskellcode}
% ghci> :t (.)
% (.) :: (b -> c) -> (a -> b) -> a -> c
%   \end{haskellcode}
% \end{remark}

\begin{problem}
  \marginpar{\cite[стр. 131]{nikolova-soskova}}
  Нека $f \in \Cont{\A}{\B}$ и $g \in \Cont{\B}{\A}$.
  Докажете, че 
  \begin{itemize}
  \item 
    $\lfp(g \circ f) \sqsubseteq g(\lfp(f \circ g))$;
  \item
    $f(\lfp(g \circ f)) \sqsubseteq \lfp(f \circ g)$.
  \end{itemize}
  Оттук заключете, че 
  \[\lfp(g \circ f) = g(\lfp(f \circ g)) \text{ и }f(\lfp(g \circ f)) = \lfp(f \circ g).\]
\end{problem}


% \begin{problem}[Кантор-Шрьодер-Бернщайн]
%   \marginpar{\cite[стр. 639]{hanbook-cs}}
%   Нека имаме две инективни функции $f:A \to B$ и $g:B \to A$.
%   Тогава съществува биективна функция $h: A \to B$.  
% \end{problem}
% \begin{hint}
%   За множеството $B$, да дефинираме областта на Скот 
%   \[\D_B = (\Ps(B),\subseteq,\emptyset).\]
%   \begin{enumerate}[a)]
%   \item 
%     За дадените от условието инективни функции $f$ и $g$,
%     да разгледаме изображението $\Gamma:\D_B \to \D_B$ зададено като
%     \marginpar{Озн. $h(X) = \{h(x) \mid x\in X\}$, $h^{-1}(X) = \{z \mid h(z) \in X\}$}
%     \[\Gamma(X) = B\setminus f(A)\cup f(g(X)).\]
%     Докажете, че $F$ е непрекъснато изображение.
%   \item
%     \Stefan{Използвам, че $X_0$ е неподвижна точка, но не виждам къде използвам, че е най-малката.}
%     Нека $X_0 = \lfp(\Gamma)$. Тогава $X_0 = B\setminus f(A) \cup f(g(X_0))$.
%     Докажете, че 
%     \[B \setminus X_0 = f(A \setminus g(X_0)).\]
%   \item
%     Дефинираме функцията $h:A \to B$ по следния начин:
%     \begin{align*}
%       h(a) = 
%       \begin{cases}
%         g^{-1}(a), & a \in g(X_0)\\
%         f(a), & a \in A \setminus g(X_0).
%       \end{cases}
%     \end{align*}
%     Докажете, че $h$ е биекция.
%   \end{enumerate}
% \end{hint}

% \begin{problem}% Gunter textbook
%   Нека $f \in \Cont{\A}{\A}$.
%   Да разгледаме множеството 
%   \[B = \{a \in \A \mid f(a) = a\}.\]
%   Вярно ли е, че 
%   \[\B = (B, \sqsubseteq^\A, \lfp(f))\] е област на Скот?
%   Обосновете се!
% \end{problem}

\begin{problem}% Gunter textbook
  Нека $f \in \Cont{\A}{\A}$.
  Да разгледаме множеството 
  \[B = \{a \in \A \mid f(a) \sqsubseteq a\}.\]
  Вярно ли е, че 
  \[\B = (B, \sqsubseteq^\A, \lfp(f))\] е област на Скот?
  Обосновете се!
\end{problem}


\begin{problem} % Gunter textbook
  Да разгледаме множеството
  \[B = \{f \in \Mon{\A}{\A} \mid f\circ f = f\}.\]
  Вярно ли е, че 
  \[\B = (B,\ \sqsubseteq,\ \lambda x.\bot^\A)\] е област на Скот,
  където 
  \[f \sqsubseteq g \df (\forall a\in\A)[f(a) \sqsubseteq^\A g(a)] ?\]
  Обосновете се!
\end{problem}

% \begin{problem} % Gunter textbook
%   Да разгледаме множеството
%   \[B = \{f \in \Strict{\A}{\A} \mid f\circ f = f\}.\]
%   Вярно ли е, че 
%   \[\B = (B,\ \sqsubseteq,\ \lambda x.\bot^\A)\] е област на Скот,
%   където 
%   \[f \sqsubseteq g \df (\forall a\in\A)[f(a) \sqsubseteq^\A g(a)] ?\]
%   Обосновете се!
% \end{problem}

\begin{problem} % Gunter textbook
  Да разгледаме множеството
  \[B = \{f \in \Strict{\Nat_\bot}{\Nat_\bot} \mid f\circ f = f\}.\]
  Вярно ли е, че 
  \[\B = (B,\ \sqsubseteq,\ \lambda x.\bot)\] е област на Скот,
  където 
  \[f \sqsubseteq g \df (\forall a\in\Nat_\bot)[f(a) \sqsubseteq g(a)] ?\]
  Обосновете се!
\end{problem}

\begin{problem}
  % \marginpar{\cite[стр. 124]{reynolds}}
  Нека $f \in \Mon{\A}{\B}$ и $g \in \Mon{\B}{\A}$ имат свойствата:
  \begin{itemize}
  \item 
    $f\circ g = id_\B$;
  \item
    $g \circ f = id_\A$.
  \end{itemize}
  Докажете, че $f$ и $g$ са точни и непрекъснати.
\end{problem}


\begin{problem}
  % \marginpar{задачата е \href{http://www.cl.cam.ac.uk/teaching/exams/pastpapers/y2008p8q14.pdf}{оттук} и \href{http://www.cl.cam.ac.uk/teaching/exams/pastpapers/y1998p9q10.pdf}{оттук}}
  Да разгледаме областта на Скот 
  \[\O = (\{\bot,\top\},\sqsubseteq, \bot),\]
  където $\bot \sqsubseteq \top$.
  За произволна област на Скот $\A$ и елемент $a \in \A$, $a \neq \bot$, дефинираме изображенията:
  \begin{enumerate}[a)]
  \item
    $f_a:\A \to \O$, където
    \[f_a(x) \df
    \begin{cases}
      \top, & a \sqsubseteq x\\
      \bot, & a \not\sqsubseteq x.
    \end{cases}\]
    Вярно ли е, че $f_a$ е точно непрекъснато изображение? Обосновете се!
  \item
    $\hat{f}_a:\A \to \O$, където
    \[\hat{f}_a(x) \df
    \begin{cases}
      \bot, & a \sqsubseteq x\\
      \top, & a \not\sqsubseteq x.
    \end{cases}\]
    Вярно ли е, че $\hat{f}_a$ е точно непрекъснато изображение? Обосновете се!
  \item 
    $g_a:\A \to \O$, където
    \[g_a(x) \df
    \begin{cases}
      \bot, & x \sqsubseteq a\\
      \top, & x \not\sqsubseteq a.
    \end{cases}\]
    Вярно ли е, че $g_a$ е точно непрекъснато изображение? Обосновете се!
  \item 
    $\hat{g}_a:\A \to \O$, където
    \[\hat{g}_a(x) \df
    \begin{cases}
      \top, & x \sqsubseteq a\\
      \bot, & x \not\sqsubseteq a.
    \end{cases}\]
    Вярно ли е, че $\hat{g}_a$ е точно непрекъснато изображение? Обосновете се!
  \item
    Докажете, че 
    \[f \in \Cont{\D}{\A} \iff (\forall a \in \A)[g_a \circ f \in \Cont{\D}{\O}].\]
  \end{enumerate}
\end{problem}

\begin{problem}
  Да разгледаме изображението
  \[\Gamma: \Cont{\A}{\B} \times \Cont{\A}{\C} \to \Cont{\A}{\B\times\C},\]
  където $\Gamma(f,g)(a) \df \pair{f(a),g(b)}$.
  \begin{itemize}
  \item
    Докажете, че $\Gamma$ е добре дефинирано изображение, т.е. за всеки непрекъснати $f$ и $g$,
    $\Gamma(f,g)$ е непрекъснато изображение.
  \item 
    Докажете, че $\Gamma$ е непрекъснато изображение.
  \end{itemize}
\end{problem}

\begin{problem}
  \marginpar{задачата е \href{http://www.cl.cam.ac.uk/teaching/exams/pastpapers/y2005p9q15.pdf}{оттук}}
  Докажете, че изображението
  \[\texttt{uncurry}:\Cont{\A}{\Cont{\B}{\C}} \to \Cont{\A\times \B}{\C},\]
  дефинирано като
  \[\texttt{uncurry}(f)(a,b) \df f(a)(b),\]
  е непрекъснато.
\end{problem}

% \begin{problem}
%   Докажете, че изображението
%   \[\texttt{curry}:\Cont{\A\times \B}{\C} \to \Cont{\A}{\Cont{\B}{\C}},\]
%   дефинирано като
%   \[\texttt{curry}(f)(a)(b) \df f(a,b),\]
%   е непрекъснато.
% \end{problem}

\begin{problem}
  Докажете, че изображението
  \[\texttt{curry}:\Cont{\A\times \B}{\C} \to \Cont{\A}{\Cont{\B}{\C}},\]
  дефинирано като
  \[\texttt{curry}(f)(a)(b) \df f(a,b),\]
  е непрекъснато.
\end{problem}

%%% Local Variables:
%%% mode: latex
%%% TeX-master: "../sep"
%%% End:


\newpage
\subsection{Регулярни езици}

Да фиксираме азбуката $\Sigma = \{a_1,\dots,a_k\}$.
Да дефинираме полиномите над $\Sigma$ като
\[\tau ::= \emptyset\ |\ \varepsilon\ |\ a_i \cdot X_j\ |\ \tau_1 + \tau_2.\]
където $i = 1, \dots,k$, а $X$ е променлива.
За всеки полином $\tau[X_1,\dots,X_n]$ дефинираме оператора 
\[\val{\tau}: \mathcal{P}(\Sigma^\star)^n \to \mathcal{P}(\Sigma^\star)\]
 по следния начин:
\begin{itemize}
\item
    $\val{\emptyset}(L_1,\dots,L_n) = \emptyset$.
\item 
  $\val{\varepsilon}(L_1,\dots,L_n) = \varepsilon$.
\item 
  $\val{a_i \cdot X_j}(L_1,\dots,L_n) = \{a_i\} \cdot L_j$.
\item
  $\val{\tau_1 + \tau_2}(L_1,\dots,L_n) = \val{\tau_1}(L_1,\dots,L_n) \cup \val{\tau_2}(L_1,\dots,L_n)$.
\end{itemize}

\begin{problem}
  Докажете, че за всеки полином $\tau$ имаме, че $\val{\tau}$ е непрекъснато изображение в областта на Скот
  $\mathcal{S} = ( \mathcal{P}(\Sigma^\star),\subseteq, \emptyset)$.
\end{problem}


\begin{example}
  Да разгледаме системата 
  \marginpar{$\tau_1[X_1,X_2] \equiv b \cdot X_1 + a \cdot X_2$}
  \marginpar{$\tau_2[X_1,X_2] \equiv \varepsilon$}
  \begin{align*}
    & X_1 = b \cdot X_1 + a\cdot X_2\\
    & X_2 = \varepsilon.
  \end{align*}

  % Понеже $\val{\tau}$ е непрекъснат оператор, то той има най-малка неподвижна точка.
  Дефинираме непрекъснатия оператор 
  \[\Gamma:\mathcal{P}(\Sigma^\star)^2 \to \mathcal{P}(\Sigma^\star)^2,\]
  където:
  \[\Gamma(L_1,L_2) = (\val{\tau_1}(L_1,L_2), \val{\tau_2}(L_1,L_2)).\]

  От Теоремата на Клини ние знам как можем да намерим най-малката неподвижна точка на $\Gamma$,
  която ще бъде и най-малкото решение на горната система.

  \begin{itemize}
  \item 
    $(L_0,M_0) \df (\emptyset,\emptyset)$;
  \item
    $(L_1,M_1) \df \Gamma(L_0,M_0) = (\val{\tau_1}(L_0,M_0), \val{\tau_2}(L_0,M_0)) = (\emptyset, \varepsilon)$;
  \item
    $(L_2,M_2) \df \Gamma(L_1,M_1) = (\val{\tau_1}(L_1,M_1), \val{\tau_2}(L_1,M_1)) = (\{a\},\varepsilon)$;
  \item
    $(L_3,M_3) \df \Gamma(L_2,M_2) = (\val{\tau_1}(L_2,M_2), \val{\tau_2}(L_2,M_2)) = (\{ba,a\},\varepsilon)$;
  \item
    $(L_4,M_4) \df \Gamma(L_3,M_3) =(\val{\tau_1}(L_3,M_3), \val{\tau_2}(L_3,M_3)) = (\{bba, ba,a\},\varepsilon)$;
  \item
    $(L_5,M_5) \df \Gamma(L_4,M_4) = ( \val{\tau_1}(L_4,M_4), \val{\tau_2}(L_4,M_4)) = (\{bba, bba, ba,a\},\varepsilon)$.
  \end{itemize}
  Лесно се съобразява, че $L_n = \{ b^ka \mid k < n\}$.
  Тогава
  \[\lfp( \Gamma ) = (\bigcup_n L_n, \{\varepsilon\}) = (b^\star a, \{\varepsilon\} ).\]
\end{example}


\begin{problem}
  Докажете, че най-малкото решение на системата 
  \begin{align*}
    & X_1 = a \cdot X_1 + b \cdot X_2 + \varepsilon\\
    & X_2 = b \cdot X_2 + \varepsilon
  \end{align*}
  е двойката $(a^\star b^\star, b^\star)$.
\end{problem}

\begin{problem}
  Да разгледаме системата от оператори
  \begin{align*}
    & \val{\tau_1}(L_1,\dots,L_n) = L_1\\
    & \ \ \vdots\\
    & \val{\tau_n}(L_1,\dots,L_n) = L_n.
  \end{align*}
  Знаем, че тя притежава най-малко решение $(\hat{L}_1,\dots,\hat{L}_n)$.
  Докажете, че всеки от езиците $\hat{L}_i$ е регулярен.

  Докажете, че всеки регулярен език е елемент от най-малкото решение 
  на някоя система от оператори от горния вид.
\end{problem}


% \begin{problem}
%   Докажете, че всеки регулярен език е елемент на най-малкото решение на някоя система от $n$
%   полинома с $n$ променливи за някое $n$.
% \end{problem}

% \begin{problem}
%   Докажете, че всяко най-малко решение на система от $n$ полинома с $n$ променливи представлява $n$-орка от 
%   регулярни езици.
% \end{problem}

% \begin{problem}
%   Опишете алгоритъм, по който може от система от $n$ полинома с $n$ променливи да се построи 
%   краен автомат с $n$ състояния.
% \end{problem}

% \begin{problem}
%   Опишете алгоритъм, по който може от краен автомат с $n$ състояния може да се построи 
% \end{problem}



\subsection{Безконтекстни езици}

Да фиксираме азбуката $\Sigma = \{a_1,\dots,a_n\}$.
Да дефинираме термове от тип 1 като
\[\tau ::= X_i\ |\ a_j\ |\ \varepsilon\ |\ \emptyset\ |\ \tau_1 \cdot \tau_2\ |\ (\tau_1 + \tau_2),\]
където $j = 1, \dots,n$, а $X_i$ са изброимо безкрайна редица от променливи.
За всеки терм $\tau[X_1,\dots,X_n]$ дефинираме оператора 
\[\val{\tau}: (\mathcal{P}(\Sigma^\star))^n \to \mathcal{P}(\Sigma^\star)\]
 по следния начин:
\begin{itemize}
\item 
  $\val{X_i}(L_1,\dots,L_n) = L_i$.
\item 
  $\val{a_j}(L_1,\dots,L_n) = \{a_j\}$.
\item 
  $\val{\varepsilon}(L_1,\dots,L_n) = \varepsilon$.
\item 
  $\val{\emptyset}(L_1,\dots,L_n) = \emptyset$.
\item 
  $\val{\tau_1 \cdot \tau_2}(L_1,\dots,L_n) = \val{\tau_1}(L_1,\dots,L_n) \cdot \val{\tau_2}(L_1,\dots,L_n)$.
\item
  $\val{\tau_1 + \tau_2}(L_1,\dots,L_n) = \val{\tau_1}(L_1,\dots,L_n) \cup \val{\tau_2}(L_1,\dots,L_n)$.
\end{itemize}

\begin{problem}
  Докажете, че за всеки терм $\tau$, $\val{\tau}$ е непрекъснато изображение в областта на Скот
  $\mathcal{S} = ( \mathcal{P}(\Sigma^\star),\subseteq, \emptyset)$.
\end{problem}

\begin{problem}
  Докажете, че $\{a^nb^n \mid n\in \Nat\} = \lfp(\val{\tau})$, където 
  \[\tau[X] \equiv \varepsilon + a \cdot X \cdot b.\]
  С други думи, $\{a^nb^n \mid n \in \Nat\}$ е най-малкото решение на уравнението
  \[X = a \cdot X \cdot b + \varepsilon.\]
\end{problem}

Нека сега да разгледаме термовете $\tau_1[X_1,\dots,X_n], \dots, \tau_n[X_1,\dots,X_n]$.

\begin{problem}
  Да разгледаме системата от оператори
  \begin{align*}
    & \val{\tau_1}(L_1,\dots,L_n) = L_1\\
    & \ \ \vdots\\
    & \val{\tau_n}(L_1,\dots,L_n) = L_n.
  \end{align*}
  Знаем, че тя притежава най-малко решение $(\hat{L}_1,\dots,\hat{L}_n)$.
  Докажете, че всеки от езиците $\hat{L}_i$ е безконтекстен.

  Докажете, че всеки безконтекстен език е елемент от най-малкото решение 
  на някоя система от оператори от горния вид.
\end{problem}

\begin{problem}
  \marginpar{Това е аналог на нормалната форма на Чомски}
  Да дефинираме термове от тип 2 като
  \[\tau ::= a_j\ |\ \varepsilon\ |\ \emptyset\ |\ X_i \cdot X_k\ |\ (\tau_1 + \tau_2),\]
  където $j = 1, \dots,n$, а $X_i$ са изброимо безкрайна редица от променливи.
  Докажете горното твърдение, като замените термовете от тип 1 с тези от тип 2.
\end{problem}

\begin{example}
  Да разгледаме системата
  \begin{align*}
    & X_1 = X_3 \cdot X_2 + \varepsilon\\
    & X_2 = X_1 \cdot X_4\\
    & X_3 = a\\
    & X_4 = b.
  \end{align*}


  % \begin{align*}
  %   & \val{\varepsilon + X_3 \cdot X_2}(L_1, L_2, L_3, L_4) = L_1\\
  %   & \val{X_1 \cdot X_4}(L_1, L_2, L_3, L_4) = L_2\\
  %   & \val{a}(L_1, L_2, L_3, L_4) = L_3\\
  %   & \val{b}(L_1, L_2, L_3, L_4) = L_4\\
  % \end{align*}
  Нека $(\hat{L}_1, \hat{L}_2, \hat{L}_3, \hat{L}_4)$ е най-малкото решение на системата.
  Докажете, че $\hat{L}_1 = \{a^nb^n\mid n \in \Nat\}$ $\hat{L}_2 = \{a^nb^{n+1}\mid n \in \Nat\}$,
  $\hat{L}_3 = \{a\}$ и $\hat{L}_4 = \{b\}$.
\end{example}



%%% Local Variables:
%%% mode: latex
%%% TeX-master: "../sep"
%%% End:


%%% Local Variables:
%%% mode: latex
%%% TeX-master: "../sep"
%%% End:

% \section{Задачи}  

Тук с $\A$, $\B$ и $\C$ ще означаваме области на Скот.

\marginpar{Много от задачите са от \cite[стр. 31]{abramsky94}}

\begin{problem}
  Да разгледаме операторите \[\Gamma,\Delta \in \Cont{\Cont{\A}{\A}}{\Cont{\A}{\A}}.\]
  Знаем, че операторът $\Gamma \circ \Delta$ е непрекъснат, където
  \[(\Gamma\circ\Delta)(f) \df \Gamma(\Delta(f)).\]
  Вярно ли е, че
  \[\lfp(\Gamma \circ \Delta) \sqsubseteq \lfp(\Gamma) \circ \lfp(\Delta)?\]
  Обосновете се!
\end{problem}
\ifhints
\begin{hint}
  Нека $\A = \Nat_\bot$.
  Нека например да разгледаме
  \begin{align*}
    & \Delta(f)(x) \df f(x+1)\\
    & \Gamma(f)(x) \df
      \begin{cases}
        0, & x \neq \bot\\
        \bot, & x = \bot.
      \end{cases}
  \end{align*}
  Да положим $f_\Gamma \df \lfp(\Gamma)$ и $f_\Delta \df \lfp(\Delta)$.
  Ясно е, че 
  \begin{align*}
    & f_\Delta(x) = \bot\\
    & f_\Gamma(x) =
    \begin{cases}
      0, & x \neq \bot\\
      \bot, & x = \bot.
    \end{cases}  
  \end{align*}
  Тогава за произволно $x \in \Nat_\bot$,
  \[(f_\Gamma\circ f_\Delta)(x) = f_\Gamma(f_\Delta(x)) = f_\Gamma(\bot)  = \bot.\]
  От друга страна, понеже $(\Gamma \circ \Delta)(f) = \Gamma(\Delta(f))$, то 
  \begin{align*}
    & (\Gamma \circ \Delta)(f)(x) = \Gamma(\Delta(f))(x) = 
      \begin{cases}
        0, & x \neq \bot\\
        \bot, & x = \bot.
      \end{cases}
  \end{align*}
  Лесно се съобразява, че 
  \[\lfp(\Gamma \circ \Delta)(x) =
  \begin{cases}
    0, & x \neq \bot\\
    \bot, & x = \bot.
  \end{cases}\]
  Заключаваме, че 
  \[\lfp(\Gamma \circ \Delta) \sqsupset \lfp(\Gamma) \circ \lfp(\Delta).\]
\end{hint}
\fi

\begin{problem}
  Да разгледаме операторите \[\Gamma,\Delta \in \Cont{\Cont{\A}{\A}}{\Cont{\A}{\A}}.\]
  Знаем, че операторът $\Gamma \circ \Delta$ е непрекъснат, където
  \[(\Gamma\circ\Delta)(f) \df \Gamma(\Delta(f)).\]
  Вярно ли е, че
  \[\lfp(\Gamma \circ \Delta) \sqsupseteq \lfp(\Gamma) \circ \lfp(\Delta)?\]
  Обосновете се!
\end{problem}
\ifhints
\begin{hint}
  Нека $\A = \Nat_\bot$.
  Нека например да разгледаме
  \begin{align*}
    & \Delta(f)(x) \df 0\\
    & \Gamma(f)(x) \df
      \begin{cases}
        0, & x = 0\\
        \bot, & \text{ иначе}.
      \end{cases}
  \end{align*}
  Да положим $f_\Gamma \df \lfp(\Gamma)$ и $f_\Delta \df \lfp(\Delta)$.
  Ясно е, че 
  \begin{align*}
    & f_\Delta(x) = 0\\
    & f_\Gamma(x) =
    \begin{cases}
      0, & x = 0\\
      \bot, & \text{ иначе}.
    \end{cases}  
  \end{align*}
  Тогава за произволно $x \in \Nat_\bot$,
  \[(f_\Gamma\circ f_\Delta)(x) = f_\Gamma(f_\Delta(x)) = f_\Gamma(0)  = 0.\]
  От друга страна, понеже $(\Gamma \circ \Delta)(f) = \Gamma(\Delta(f))$, то 
  \begin{align*}
    & (\Gamma \circ \Delta)(f)(x) = \Gamma(\Delta(f))(x) = 
      \begin{cases}
        0, & x = 0\\
        \bot, & \text{ иначе}.
      \end{cases}
  \end{align*}
  Лесно се съобразява, че 
  \[\lfp(\Gamma \circ \Delta)(x) =
  \begin{cases}
    0, & x = 0\\
    \bot, & \text{ иначе}.
  \end{cases}\]
  Заключаваме, че 
  \[\lfp(\Gamma \circ \Delta) \sqsubset \lfp(\Gamma) \circ \lfp(\Delta).\]
\end{hint}
\fi

\begin{problem}
  Нека $f_0 \sqsubseteq f_1 \sqsubseteq f_2 \sqsubseteq \cdots$
  е верига от елементи на $\Cont{\A}{\A}$.
  Да положим $h = \bigsqcup_n f_n$.
  Вярно ли е, че 
  \[h \circ h = \bigsqcup_n (f\circ f)?\]
  Обосновете се!
\end{problem}

\begin{problem}
  Да разгледаме едно изображение $f: \A \times \B \to \C$.
  За произволно $a \in \A$, дефинираме изображението $g_a: \B \to \C$, където
  \[g_a(b) \df f(a,b).\]
  Аналогично, за произволно $b \in \B$, дефинираме изображението $h_b: \A \to \C$, където
  \[h_b(a) \df f(a,b).\]
  Докажете, че следните твърдения са еквивалентни:
  \begin{enumerate}[1)]
  \item 
    $f$ е непрекъснато изображение;
  \item
    $g_a$ и $h_b$ са непрекъснати изображения, за всяко $a \in \A$ и $b \in \B$.
  \end{enumerate}
\end{problem}

\begin{problem}
  Да разгледаме $f \in \Cont{\A \times \B}{\C}$.
  За произволно $a \in \A$, дефинираме изображението $g_a: \B \to \C$, където
  \[g_a(b) \df f(a,b).\]
  Вече знаем, че $g_a \in \Cont{\B}{\C}$, за всяко $a \in \A$.
  Да разгледаме изображението $h:\A \to \Cont{\B}{\C}$, където
  \[h(a) \df g_a.\]
  Докажете, че $h$ е непрекъснато изображение.
\end{problem}

\begin{problem}
  \marginpar{\cite[стр. 129]{nikolova-soskova}}
  Да разгледаме $f \in \Cont{\A \times \B}{\B}$.
  За произволно $a \in \A$, дефинираме изображението $g_a: \B \to \B$, където
  \[g_a(b) \df f(a,b).\]
  Вече знаем, че $g_a \in \Cont{\B}{\B}$, за всяко $a \in \A$,
  следователно $\lfp(g_a)$ съществува.
  Да разгледаме изображението $h:\A \to \B$, където
  \[h(a) \df \lfp(g_a).\]
  Докажете, че $h$ е непрекъснато изображение.
\end{problem}


\begin{problem}
  Да разгледаме $f \in \Cont{\A}{\Cont{\B}{\C}}$.
  За произволно $a \in \A$, дефинираме изображението $g_a \in \Cont{\B}{\C}$, където
  \[g_a(b) \df f(a).\]
  Да разгледаме изображението $h:\A\times \B \to \C$, където
  \[h(a,b) \df g_a(b).\]
  Докажете, че $h$ е непрекъснато изображение.
\end{problem}

% \begin{problem}
%   Нека са дадени областите на Скот $\D$ и $\E$ и изображението 
%   \[\texttt{eval}: \Cont{\D}{\E} \times \D \to \E,\]
%   където 
%   \[\texttt{eval}(f,d) \df f(d).\]
%   Докажете, че $\texttt{eval}$ е непрекъснато изображение.
% \end{problem}
% \ifhints
% \begin{hint}
%   Понеже $\bigsqcup_n(f_n,d_n) = (\bigsqcup_m f_m,\bigsqcup_n d_n)$, 
%   ще докажем, че \[\texttt{eval}(\bigsqcup_m f_m, \bigsqcup_n d_n) = \bigsqcup_n \texttt{eval}(f_n,d_n).\]
%   Знаем, че
%   \begin{align*}
%     \texttt{eval}(\bigsqcup_m f_m, \bigsqcup_n d_n) & = (\bigsqcup_m f_m)(\bigsqcup_n d_n) & (\mbox{от опр. на }\texttt{eval})\\
%     & = \bigsqcup_m (f_m(\bigsqcup_n d_n)) & (\mbox{от опр. на }\bigsqcup_mf_m)\\
%     & = \bigsqcup_m (\bigsqcup_n (f_m(d_n)) & (\mbox{всяка }f_m\mbox{ е непр.} )\\
%   \end{align*}
%   Нека да положим $e_{m,n} = f_m(d_n)$.
%   Лесно се съобразява, че
%   \[m \leq m^\prime\ \&\ n \leq n^\prime\ \Rightarrow\ e_{m,n} \sqsubseteq^\E e_{m^\prime,n^\prime}.\]
%   Така получаваме, че 
%   \begin{align*}
%     \texttt{eval}(\bigsqcup_m f_m, \bigsqcup_n d_n) & = \bigsqcup_m (\bigsqcup_n (f_m(d_n)) & (\mbox{от по-горе})\\
%     & = \bigsqcup_{m,n} e_{m,n} = \bigsqcup_{n} e_{n,n} & (\mbox{от \Th{double-chain}})\\
%     & = \bigsqcup_{n} f_n(d_n) & (\text{ от опр. на }e_{m,n})\\
%     & = \bigsqcup_n \texttt{eval}(f_n,d_n).
%   \end{align*}
% \end{hint}
% \fi

\begin{problem}
  Нека са дадени областите на Скот $\D$ и $\E$ и изображението 
  \[\texttt{eval}: \Cont{\D}{\E} \times \D \to \E,\]
  където 
  \[\texttt{eval}(f,d) \df f(d).\]
  Докажете, че $\texttt{eval}$ е непрекъснато изображение.
\end{problem}
\ifhints
\begin{hint}
  Понеже $\bigsqcup_n(f_n,d_n) = (\bigsqcup_m f_m,\bigsqcup_n d_n)$, 
  ще докажем, че \[\texttt{eval}(\bigsqcup_m f_m, \bigsqcup_n d_n) = \bigsqcup_n \texttt{eval}(f_n,d_n).\]
  Знаем, че
  \begin{align*}
    \texttt{eval}(\bigsqcup_m f_m, \bigsqcup_n d_n) & = (\bigsqcup_m f_m)(\bigsqcup_n d_n) & \comment\text{от опр. на }\texttt{eval}\\
                                                    & = \bigsqcup_m (f_m(\bigsqcup_n d_n)) & \comment\text{от опр. на }\bigsqcup_mf_m\\
                                                    & = \bigsqcup_m (\bigsqcup_n (f_m(d_n)) & \comment\text{всяка }f_m\mbox{ е непр.}\\
  \end{align*}
  Нека да положим $e_{m,n} = f_m(d_n)$.
  Лесно се съобразява, че
  \[m \leq m^\prime\ \&\ n \leq n^\prime\ \Rightarrow\ e_{m,n} \sqsubseteq^\E e_{m^\prime,n^\prime}.\]
  Така получаваме, че 
  \begin{align*}
    \texttt{eval}(\bigsqcup_m f_m, \bigsqcup_n d_n) & = \bigsqcup_m (\bigsqcup_n (f_m(d_n)) & \comment\text{от по-горе}\\
    & = \bigsqcup_{m,n} e_{m,n} = \bigsqcup_{n} e_{n,n} & \comment\text{от \Th{double-chain}}\\
    & = \bigsqcup_{n} f_n(d_n) & \comment\text{ от опр. на }e_{m,n}\\
    & = \bigsqcup_n \texttt{eval}(f_n,d_n).
  \end{align*}
\end{hint}
\fi

%%% Local Variables:
%%% mode: latex
%%% TeX-master: "../sep"
%%% End:

\begin{problem}
  Нека изображението \[\texttt{comp}:(\Cont{\B}{\C} \times \Cont{\A}{\B}) \to \Cont{\A}{\C}\]
  е определено като 
  \[\texttt{comp}(g,f) \df g\circ f.\]
  \marginpar{$(g \circ f)(a) = g(f(a))$}
  Докажете, че $\texttt{comp}$ е непрекъснато изображение.
\end{problem}
\ifhints
\begin{hint}
  \marginpar{\cite[стр. 124]{reynolds}}
  Трябва да докажем, че за всяка монотонно растяща редица $\{(g_n,f_n)\}_{n\in\Nat}$,
  \[\Gamma(\bigsqcup_n(g_n,f_n))(a) = \bigsqcup_n\Gamma(g_n,f_n)(a),\]
  за произволно $a \in A$.
  Да фиксираме едно $a\in A$ и да положим $g_n(f_k(a)) = e_{n,k}$.
  Лесно се вижда, че 
  \[n\leq n^\prime\ \&\ k \leq k^\prime\ \Rightarrow\ e_{n,k} \sqsubseteq e_{n^\prime,k^\prime}.\]
  Тогава:
  \begin{align*}
    \Gamma(\bigsqcup_n(g_n,f_n))(a) & = \Gamma(\bigsqcup_n g_n, \bigsqcup_k f_k)(a) & \\
                                    & = (\bigsqcup_n g_n)(\bigsqcup_k f_k(a)) & \comment\text{ по деф. на }\Gamma\\
                                    & = (\bigsqcup_n g_n)(\bigsqcup_k b_k) & \comment\text{ полагаме }b_k = f_k(a)\\
                                    & = \bigsqcup_k (\bigsqcup_n g_n)(b_k) & \comment\bigsqcup_n g_n\text{ е непр.}\\
                                    & = \bigsqcup_k(\bigsqcup_n g_n(b_k)) & \comment\text{ по деф. на }\bigsqcup_n g_n\\
                                    & = \bigsqcup_k (\bigsqcup_n g_n(f_k(a))) & \comment\text{ полагаме }e_{n,k} = g_n(f_k(a))\\
                                    & = \bigsqcup_k\bigsqcup_n e_{n,k} = \bigsqcup_n e_{n,n} & \comment\text{ от \Th{double-chain}}\\
                                    & = \bigsqcup_n g_n(f_n(a)) = \bigsqcup_n \Gamma(g_n, f_n)(a).
  \end{align*}
\end{hint}
\fi

\begin{remark}
  \marginpar{Когато на хаскел пишем $(.)$, означава, че операцията е инфиксна}
  В хаскел има операция композиция на функции.
  \begin{haskellcode}
ghci> :t (.)
(.) :: (b -> c) -> (a -> b) -> a -> c
  \end{haskellcode}
\end{remark}


%%% Local Variables:
%%% mode: latex
%%% TeX-master: "../sep"
%%% End:


% \begin{problem}
%   Нека изображението \[\texttt{comp}:(\Cont{\B}{\C} \times \Cont{\A}{\B}) \to \Cont{\A}{\C}\]
%   е определено като 
%   \[\texttt{comp}(g,f) \df g\circ f.\]
%   \marginpar{$(g \circ f)(a) = g(f(a))$}
%   Докажете, че $\texttt{comp}$ е непрекъснато изображение.
% \end{problem}
% \ifhints
% \begin{hint}
%   \marginpar{\cite[стр. 124]{reynolds}}
%   Трябва да докажем, че за всяка монотонно растяща редица $\{(g_n,f_n)\}_{n\in\Nat}$,
%   \[\Gamma(\bigsqcup_n(g_n,f_n))(a) = \bigsqcup_n\Gamma(g_n,f_n)(a),\]
%   за произволно $a \in A$.
%   Да фиксираме едно $a\in A$ и да положим $g_n(f_k(a)) = e_{n,k}$.
%   Лесно се вижда, че 
%   \[n\leq n^\prime\ \&\ k \leq k^\prime\ \Rightarrow\ e_{n,k} \sqsubseteq e_{n^\prime,k^\prime}.\]
%   Тогава:
%   \begin{align*}
%     \Gamma(\bigsqcup_n(g_n,f_n))(a) & = \Gamma(\bigsqcup_n g_n, \bigsqcup_k f_k)(a) & \\
%     & = (\bigsqcup_n g_n)(\bigsqcup_k f_k(a)) & (\text{ по деф. на }\Gamma )\\
%     & = (\bigsqcup_n g_n)(\bigsqcup_k b_k) & (\text{ полагаме }b_k = f_k(a))\\
%     & = \bigsqcup_k (\bigsqcup_n g_n)(b_k) & (\bigsqcup_n g_n\text{ е непр.})\\
%     & = \bigsqcup_k(\bigsqcup_n g_n(b_k)) & (\text{ по деф. на }\bigsqcup_n g_n)\\
%     & = \bigsqcup_k (\bigsqcup_n g_n(f_k(a))) & (\text{ полагаме }e_{n,k} = g_n(f_k(a)))\\
%     & = \bigsqcup_k\bigsqcup_n e_{n,k} = \bigsqcup_n e_{n,n} & (\text{ от \Th{double-chain}})\\
%     & = \bigsqcup_n g_n(f_n(a)) = \bigsqcup_n \Gamma(g_n, f_n)(a).
%   \end{align*}
% \end{hint}
% \fi

% \begin{remark}
%   \marginpar{Когато пишем $(.)$ означава, че операцията е инфиксна}
%   В хаскел има операция композиция на функции.
%   \begin{haskellcode}
% ghci> :t (.)
% (.) :: (b -> c) -> (a -> b) -> a -> c
%   \end{haskellcode}
% \end{remark}

\begin{problem}
  \marginpar{\cite[стр. 131]{nikolova-soskova}}
  Нека $f \in \Cont{\A}{\B}$ и $g \in \Cont{\B}{\A}$.
  Докажете, че 
  \begin{itemize}
  \item 
    $\lfp(g \circ f) \sqsubseteq g(\lfp(f \circ g))$;
  \item
    $f(\lfp(g \circ f)) \sqsubseteq \lfp(f \circ g)$.
  \end{itemize}
  Оттук заключете, че 
  \[\lfp(g \circ f) = g(\lfp(f \circ g)) \text{ и }f(\lfp(g \circ f)) = \lfp(f \circ g).\]
\end{problem}


% \begin{problem}[Кантор-Шрьодер-Бернщайн]
%   \marginpar{\cite[стр. 639]{hanbook-cs}}
%   Нека имаме две инективни функции $f:A \to B$ и $g:B \to A$.
%   Тогава съществува биективна функция $h: A \to B$.  
% \end{problem}
% \begin{hint}
%   За множеството $B$, да дефинираме областта на Скот 
%   \[\D_B = (\Ps(B),\subseteq,\emptyset).\]
%   \begin{enumerate}[a)]
%   \item 
%     За дадените от условието инективни функции $f$ и $g$,
%     да разгледаме изображението $\Gamma:\D_B \to \D_B$ зададено като
%     \marginpar{Озн. $h(X) = \{h(x) \mid x\in X\}$, $h^{-1}(X) = \{z \mid h(z) \in X\}$}
%     \[\Gamma(X) = B\setminus f(A)\cup f(g(X)).\]
%     Докажете, че $F$ е непрекъснато изображение.
%   \item
%     \Stefan{Използвам, че $X_0$ е неподвижна точка, но не виждам къде използвам, че е най-малката.}
%     Нека $X_0 = \lfp(\Gamma)$. Тогава $X_0 = B\setminus f(A) \cup f(g(X_0))$.
%     Докажете, че 
%     \[B \setminus X_0 = f(A \setminus g(X_0)).\]
%   \item
%     Дефинираме функцията $h:A \to B$ по следния начин:
%     \begin{align*}
%       h(a) = 
%       \begin{cases}
%         g^{-1}(a), & a \in g(X_0)\\
%         f(a), & a \in A \setminus g(X_0).
%       \end{cases}
%     \end{align*}
%     Докажете, че $h$ е биекция.
%   \end{enumerate}
% \end{hint}

% \begin{problem}% Gunter textbook
%   Нека $f \in \Cont{\A}{\A}$.
%   Да разгледаме множеството 
%   \[B = \{a \in \A \mid f(a) = a\}.\]
%   Вярно ли е, че 
%   \[\B = (B, \sqsubseteq^\A, \lfp(f))\] е област на Скот?
%   Обосновете се!
% \end{problem}

\begin{problem}% Gunter textbook
  Нека $f \in \Cont{\A}{\A}$.
  Да разгледаме множеството 
  \[B = \{a \in \A \mid f(a) \sqsubseteq a\}.\]
  Вярно ли е, че 
  \[\B = (B, \sqsubseteq^\A, \lfp(f))\] е област на Скот?
  Обосновете се!
\end{problem}


\begin{problem} % Gunter textbook
  Да разгледаме множеството
  \[B = \{f \in \Mon{\A}{\A} \mid f\circ f = f\}.\]
  Вярно ли е, че 
  \[\B = (B,\ \sqsubseteq,\ \lambda x.\bot^\A)\] е област на Скот,
  където 
  \[f \sqsubseteq g \df (\forall a\in\A)[f(a) \sqsubseteq^\A g(a)] ?\]
  Обосновете се!
\end{problem}

% \begin{problem} % Gunter textbook
%   Да разгледаме множеството
%   \[B = \{f \in \Strict{\A}{\A} \mid f\circ f = f\}.\]
%   Вярно ли е, че 
%   \[\B = (B,\ \sqsubseteq,\ \lambda x.\bot^\A)\] е област на Скот,
%   където 
%   \[f \sqsubseteq g \df (\forall a\in\A)[f(a) \sqsubseteq^\A g(a)] ?\]
%   Обосновете се!
% \end{problem}

\begin{problem} % Gunter textbook
  Да разгледаме множеството
  \[B = \{f \in \Strict{\Nat_\bot}{\Nat_\bot} \mid f\circ f = f\}.\]
  Вярно ли е, че 
  \[\B = (B,\ \sqsubseteq,\ \lambda x.\bot)\] е област на Скот,
  където 
  \[f \sqsubseteq g \df (\forall a\in\Nat_\bot)[f(a) \sqsubseteq g(a)] ?\]
  Обосновете се!
\end{problem}

\begin{problem}
  % \marginpar{\cite[стр. 124]{reynolds}}
  Нека $f \in \Mon{\A}{\B}$ и $g \in \Mon{\B}{\A}$ имат свойствата:
  \begin{itemize}
  \item 
    $f\circ g = id_\B$;
  \item
    $g \circ f = id_\A$.
  \end{itemize}
  Докажете, че $f$ и $g$ са точни и непрекъснати.
\end{problem}


\begin{problem}
  % \marginpar{задачата е \href{http://www.cl.cam.ac.uk/teaching/exams/pastpapers/y2008p8q14.pdf}{оттук} и \href{http://www.cl.cam.ac.uk/teaching/exams/pastpapers/y1998p9q10.pdf}{оттук}}
  Да разгледаме областта на Скот 
  \[\O = (\{\bot,\top\},\sqsubseteq, \bot),\]
  където $\bot \sqsubseteq \top$.
  За произволна област на Скот $\A$ и елемент $a \in \A$, $a \neq \bot$, дефинираме изображенията:
  \begin{enumerate}[a)]
  \item
    $f_a:\A \to \O$, където
    \[f_a(x) \df
    \begin{cases}
      \top, & a \sqsubseteq x\\
      \bot, & a \not\sqsubseteq x.
    \end{cases}\]
    Вярно ли е, че $f_a$ е точно непрекъснато изображение? Обосновете се!
  \item
    $\hat{f}_a:\A \to \O$, където
    \[\hat{f}_a(x) \df
    \begin{cases}
      \bot, & a \sqsubseteq x\\
      \top, & a \not\sqsubseteq x.
    \end{cases}\]
    Вярно ли е, че $\hat{f}_a$ е точно непрекъснато изображение? Обосновете се!
  \item 
    $g_a:\A \to \O$, където
    \[g_a(x) \df
    \begin{cases}
      \bot, & x \sqsubseteq a\\
      \top, & x \not\sqsubseteq a.
    \end{cases}\]
    Вярно ли е, че $g_a$ е точно непрекъснато изображение? Обосновете се!
  \item 
    $\hat{g}_a:\A \to \O$, където
    \[\hat{g}_a(x) \df
    \begin{cases}
      \top, & x \sqsubseteq a\\
      \bot, & x \not\sqsubseteq a.
    \end{cases}\]
    Вярно ли е, че $\hat{g}_a$ е точно непрекъснато изображение? Обосновете се!
  \item
    Докажете, че 
    \[f \in \Cont{\D}{\A} \iff (\forall a \in \A)[g_a \circ f \in \Cont{\D}{\O}].\]
  \end{enumerate}
\end{problem}

\begin{problem}
  Да разгледаме изображението
  \[\Gamma: \Cont{\A}{\B} \times \Cont{\A}{\C} \to \Cont{\A}{\B\times\C},\]
  където $\Gamma(f,g)(a) \df \pair{f(a),g(b)}$.
  \begin{itemize}
  \item
    Докажете, че $\Gamma$ е добре дефинирано изображение, т.е. за всеки непрекъснати $f$ и $g$,
    $\Gamma(f,g)$ е непрекъснато изображение.
  \item 
    Докажете, че $\Gamma$ е непрекъснато изображение.
  \end{itemize}
\end{problem}

\begin{problem}
  \marginpar{задачата е \href{http://www.cl.cam.ac.uk/teaching/exams/pastpapers/y2005p9q15.pdf}{оттук}}
  Докажете, че изображението
  \[\texttt{uncurry}:\Cont{\A}{\Cont{\B}{\C}} \to \Cont{\A\times \B}{\C},\]
  дефинирано като
  \[\texttt{uncurry}(f)(a,b) \df f(a)(b),\]
  е непрекъснато.
\end{problem}

% \begin{problem}
%   Докажете, че изображението
%   \[\texttt{curry}:\Cont{\A\times \B}{\C} \to \Cont{\A}{\Cont{\B}{\C}},\]
%   дефинирано като
%   \[\texttt{curry}(f)(a)(b) \df f(a,b),\]
%   е непрекъснато.
% \end{problem}

\begin{problem}
  Докажете, че изображението
  \[\texttt{curry}:\Cont{\A\times \B}{\C} \to \Cont{\A}{\Cont{\B}{\C}},\]
  дефинирано като
  \[\texttt{curry}(f)(a)(b) \df f(a,b),\]
  е непрекъснато.
\end{problem}

%%% Local Variables:
%%% mode: latex
%%% TeX-master: "../sep"
%%% End:


\newpage
\subsection{Регулярни езици}

Да фиксираме азбуката $\Sigma = \{a_1,\dots,a_k\}$.
Да дефинираме полиномите над $\Sigma$ като
\[\tau ::= \emptyset\ |\ \varepsilon\ |\ a_i \cdot X_j\ |\ \tau_1 + \tau_2.\]
където $i = 1, \dots,k$, а $X$ е променлива.
За всеки полином $\tau[X_1,\dots,X_n]$ дефинираме оператора 
\[\val{\tau}: \mathcal{P}(\Sigma^\star)^n \to \mathcal{P}(\Sigma^\star)\]
 по следния начин:
\begin{itemize}
\item
    $\val{\emptyset}(L_1,\dots,L_n) = \emptyset$.
\item 
  $\val{\varepsilon}(L_1,\dots,L_n) = \varepsilon$.
\item 
  $\val{a_i \cdot X_j}(L_1,\dots,L_n) = \{a_i\} \cdot L_j$.
\item
  $\val{\tau_1 + \tau_2}(L_1,\dots,L_n) = \val{\tau_1}(L_1,\dots,L_n) \cup \val{\tau_2}(L_1,\dots,L_n)$.
\end{itemize}

\begin{problem}
  Докажете, че за всеки полином $\tau$ имаме, че $\val{\tau}$ е непрекъснато изображение в областта на Скот
  $\mathcal{S} = ( \mathcal{P}(\Sigma^\star),\subseteq, \emptyset)$.
\end{problem}


\begin{example}
  Да разгледаме системата 
  \marginpar{$\tau_1[X_1,X_2] \equiv b \cdot X_1 + a \cdot X_2$}
  \marginpar{$\tau_2[X_1,X_2] \equiv \varepsilon$}
  \begin{align*}
    & X_1 = b \cdot X_1 + a\cdot X_2\\
    & X_2 = \varepsilon.
  \end{align*}

  % Понеже $\val{\tau}$ е непрекъснат оператор, то той има най-малка неподвижна точка.
  Дефинираме непрекъснатия оператор 
  \[\Gamma:\mathcal{P}(\Sigma^\star)^2 \to \mathcal{P}(\Sigma^\star)^2,\]
  където:
  \[\Gamma(L_1,L_2) = (\val{\tau_1}(L_1,L_2), \val{\tau_2}(L_1,L_2)).\]

  От Теоремата на Клини ние знам как можем да намерим най-малката неподвижна точка на $\Gamma$,
  която ще бъде и най-малкото решение на горната система.

  \begin{itemize}
  \item 
    $(L_0,M_0) \df (\emptyset,\emptyset)$;
  \item
    $(L_1,M_1) \df \Gamma(L_0,M_0) = (\val{\tau_1}(L_0,M_0), \val{\tau_2}(L_0,M_0)) = (\emptyset, \varepsilon)$;
  \item
    $(L_2,M_2) \df \Gamma(L_1,M_1) = (\val{\tau_1}(L_1,M_1), \val{\tau_2}(L_1,M_1)) = (\{a\},\varepsilon)$;
  \item
    $(L_3,M_3) \df \Gamma(L_2,M_2) = (\val{\tau_1}(L_2,M_2), \val{\tau_2}(L_2,M_2)) = (\{ba,a\},\varepsilon)$;
  \item
    $(L_4,M_4) \df \Gamma(L_3,M_3) =(\val{\tau_1}(L_3,M_3), \val{\tau_2}(L_3,M_3)) = (\{bba, ba,a\},\varepsilon)$;
  \item
    $(L_5,M_5) \df \Gamma(L_4,M_4) = ( \val{\tau_1}(L_4,M_4), \val{\tau_2}(L_4,M_4)) = (\{bba, bba, ba,a\},\varepsilon)$.
  \end{itemize}
  Лесно се съобразява, че $L_n = \{ b^ka \mid k < n\}$.
  Тогава
  \[\lfp( \Gamma ) = (\bigcup_n L_n, \{\varepsilon\}) = (b^\star a, \{\varepsilon\} ).\]
\end{example}


\begin{problem}
  Докажете, че най-малкото решение на системата 
  \begin{align*}
    & X_1 = a \cdot X_1 + b \cdot X_2 + \varepsilon\\
    & X_2 = b \cdot X_2 + \varepsilon
  \end{align*}
  е двойката $(a^\star b^\star, b^\star)$.
\end{problem}

\begin{problem}
  Да разгледаме системата от оператори
  \begin{align*}
    & \val{\tau_1}(L_1,\dots,L_n) = L_1\\
    & \ \ \vdots\\
    & \val{\tau_n}(L_1,\dots,L_n) = L_n.
  \end{align*}
  Знаем, че тя притежава най-малко решение $(\hat{L}_1,\dots,\hat{L}_n)$.
  Докажете, че всеки от езиците $\hat{L}_i$ е регулярен.

  Докажете, че всеки регулярен език е елемент от най-малкото решение 
  на някоя система от оператори от горния вид.
\end{problem}


% \begin{problem}
%   Докажете, че всеки регулярен език е елемент на най-малкото решение на някоя система от $n$
%   полинома с $n$ променливи за някое $n$.
% \end{problem}

% \begin{problem}
%   Докажете, че всяко най-малко решение на система от $n$ полинома с $n$ променливи представлява $n$-орка от 
%   регулярни езици.
% \end{problem}

% \begin{problem}
%   Опишете алгоритъм, по който може от система от $n$ полинома с $n$ променливи да се построи 
%   краен автомат с $n$ състояния.
% \end{problem}

% \begin{problem}
%   Опишете алгоритъм, по който може от краен автомат с $n$ състояния може да се построи 
% \end{problem}



\subsection{Безконтекстни езици}

Да фиксираме азбуката $\Sigma = \{a_1,\dots,a_n\}$.
Да дефинираме термове от тип 1 като
\[\tau ::= X_i\ |\ a_j\ |\ \varepsilon\ |\ \emptyset\ |\ \tau_1 \cdot \tau_2\ |\ (\tau_1 + \tau_2),\]
където $j = 1, \dots,n$, а $X_i$ са изброимо безкрайна редица от променливи.
За всеки терм $\tau[X_1,\dots,X_n]$ дефинираме оператора 
\[\val{\tau}: (\mathcal{P}(\Sigma^\star))^n \to \mathcal{P}(\Sigma^\star)\]
 по следния начин:
\begin{itemize}
\item 
  $\val{X_i}(L_1,\dots,L_n) = L_i$.
\item 
  $\val{a_j}(L_1,\dots,L_n) = \{a_j\}$.
\item 
  $\val{\varepsilon}(L_1,\dots,L_n) = \varepsilon$.
\item 
  $\val{\emptyset}(L_1,\dots,L_n) = \emptyset$.
\item 
  $\val{\tau_1 \cdot \tau_2}(L_1,\dots,L_n) = \val{\tau_1}(L_1,\dots,L_n) \cdot \val{\tau_2}(L_1,\dots,L_n)$.
\item
  $\val{\tau_1 + \tau_2}(L_1,\dots,L_n) = \val{\tau_1}(L_1,\dots,L_n) \cup \val{\tau_2}(L_1,\dots,L_n)$.
\end{itemize}

\begin{problem}
  Докажете, че за всеки терм $\tau$, $\val{\tau}$ е непрекъснато изображение в областта на Скот
  $\mathcal{S} = ( \mathcal{P}(\Sigma^\star),\subseteq, \emptyset)$.
\end{problem}

\begin{problem}
  Докажете, че $\{a^nb^n \mid n\in \Nat\} = \lfp(\val{\tau})$, където 
  \[\tau[X] \equiv \varepsilon + a \cdot X \cdot b.\]
  С други думи, $\{a^nb^n \mid n \in \Nat\}$ е най-малкото решение на уравнението
  \[X = a \cdot X \cdot b + \varepsilon.\]
\end{problem}

Нека сега да разгледаме термовете $\tau_1[X_1,\dots,X_n], \dots, \tau_n[X_1,\dots,X_n]$.

\begin{problem}
  Да разгледаме системата от оператори
  \begin{align*}
    & \val{\tau_1}(L_1,\dots,L_n) = L_1\\
    & \ \ \vdots\\
    & \val{\tau_n}(L_1,\dots,L_n) = L_n.
  \end{align*}
  Знаем, че тя притежава най-малко решение $(\hat{L}_1,\dots,\hat{L}_n)$.
  Докажете, че всеки от езиците $\hat{L}_i$ е безконтекстен.

  Докажете, че всеки безконтекстен език е елемент от най-малкото решение 
  на някоя система от оператори от горния вид.
\end{problem}

\begin{problem}
  \marginpar{Това е аналог на нормалната форма на Чомски}
  Да дефинираме термове от тип 2 като
  \[\tau ::= a_j\ |\ \varepsilon\ |\ \emptyset\ |\ X_i \cdot X_k\ |\ (\tau_1 + \tau_2),\]
  където $j = 1, \dots,n$, а $X_i$ са изброимо безкрайна редица от променливи.
  Докажете горното твърдение, като замените термовете от тип 1 с тези от тип 2.
\end{problem}

\begin{example}
  Да разгледаме системата
  \begin{align*}
    & X_1 = X_3 \cdot X_2 + \varepsilon\\
    & X_2 = X_1 \cdot X_4\\
    & X_3 = a\\
    & X_4 = b.
  \end{align*}


  % \begin{align*}
  %   & \val{\varepsilon + X_3 \cdot X_2}(L_1, L_2, L_3, L_4) = L_1\\
  %   & \val{X_1 \cdot X_4}(L_1, L_2, L_3, L_4) = L_2\\
  %   & \val{a}(L_1, L_2, L_3, L_4) = L_3\\
  %   & \val{b}(L_1, L_2, L_3, L_4) = L_4\\
  % \end{align*}
  Нека $(\hat{L}_1, \hat{L}_2, \hat{L}_3, \hat{L}_4)$ е най-малкото решение на системата.
  Докажете, че $\hat{L}_1 = \{a^nb^n\mid n \in \Nat\}$ $\hat{L}_2 = \{a^nb^{n+1}\mid n \in \Nat\}$,
  $\hat{L}_3 = \{a\}$ и $\hat{L}_4 = \{b\}$.
\end{example}



%%% Local Variables:
%%% mode: latex
%%% TeX-master: "../sep"
%%% End:


\chapter{Езикът \PCF}\label{ch:pcf}

\marginpar{\PCF - programming language for computable functions. Plotkin’s language \PCF is often called the \emph{E. coli} of programming languages, the  subject  of  countless  studies  of  language  concept.}

\marginpar{Тази глава се оповава основно на \cite[Глава 19]{practical-foundations} и \cite{cambridge-den-sem}.}



\newcommand{\nat}{\vv{nat}}
\newcommand{\type}[2]{\vv{#1}:\vv{#2}}
\newcommand{\lamb}[3]{\lambda~\type{#1}{#2}~.~{#3}}


\section{Синтаксис}

\newcommand{\rename}[2]{\{\vv{#1}/{#2}\}}

\newcommand{\var}{\texttt{var}}
\newcommand{\fv}{\texttt{fv}}

\begin{itemize}
\item\index{тип}
  Типове
  \[\vv{a} ::= \vv{nat}\ |\ \vv{a} \to \vv{a}.\]
  Когато пишем 
  $\vv{a} \to \vv{b} \to \vv{c}$, то имаме предвид, че
  $\vv{a} \to (\vv{b} \to \vv{c})$.
\item\index{израз}
  \marginpar{Можем да си мислим за изразите като дървета.}
  Изрази
  \begin{align*}
    \tau ::=\ & \vv{n}\ |\ \vv{x}\ |\ \tau_1 + \tau_2\ |\ \tau_1 - \tau_2\ |\  \tau_1\ \vv{==}\ \tau_2\ |\ \ifelse{\tau_1}{\tau_2}{\tau_3}\ |\\
              & \tau_1\tau_2\ |\ \lamb{x}{a}{\tau_1}\ |\ \fix(\tau_1).
  \end{align*}
  Ще означаваме съвкупоността от всички изрази с $\mathcal{E}$, а съвкупността от всички променливи с $\mathcal{V}$.
  Да обърнем внимание, че не всички изрази са ,,смислени''. Например,
  $\lamb{x}{nat}{\vv{xx}}$ не е ясно какво означава.
  След малко ще дефинираме типизираща релация, с чиято помощ ще изхвърлим безслислените изрази.
  
  Понеже тук вече имаме свободни и свързани променливи, трябва да сме внимателни, когато правим замяна на един израз с друг. Да разгледаме функцията $\fv:\mathcal{E} \to \mathcal{P}(\mathcal{V})$, дефинирана със структурна индукция по построението на термовете по следния начин:
  
  \begin{itemize}
  \item
    $\fv(\vv{n}) = \emptyset$;
  \item
    $\fv(\vv{x}) = \{\vv{x}\}$;
  \item
    $\fv(\tau_1 + \tau_2) = \fv(\tau_1\ \vv{==}\ \tau_2) = \fv(\tau_1\tau_2) = \fv(\tau_1) \cup \fv(\tau_2)$;
  \item
    $\fv(\ifelse{\tau_1}{\tau_2}{\tau_3}) = \fv(\tau_1) \cup \fv(\tau_2) \cup \fv(\tau_3)$;
  \item
    $\fv(\lamb{x}{a}{\tau}) = \fv(\tau) \setminus \{x\}$.
  \item
    $\fv(\fix(\tau)) = \fv(\tau)$;
  \end{itemize}
  
  Ще казваме, че един израз $\tau$ е {\bf затворен}, ако $\fv(\tau) = \emptyset$.
  В противен случай, ще казваме, че изразът е {\bf отворен}.
\item
  \marginpar{Стойносттите понякога се наричат и канонични форми \cite{winskel}.}
  Ще казваме, че един израз е {\bf стойност}, ако той е затворен терм, съставен по следния начин:
  \[\vv{v} ::= \vv{n}\ |\ \lamb{x}{a}{\mu}.\]
  Интуицията тук е, че стойностите са затворени термове, в които не е възможно да се правят повече опростявания.
  Например, $\vv{5 + 6}$ не е стойност, защото може да се опрости до $\vv{11}$,
  но $\lamb{x}{nat}{\vv{5+6}}$ е стойност.
\end{itemize}

С $\tau\rename{x}{\rho}$ ще означаваме изразът получен от $\tau$, в който всяко \emph{свободно} срещане на променливата $\vv{x}$
е заменена с израза $\rho$. Можем да дадем формална дефиниция с индукция по построението на изразите:
\marginpar{Операцията $\rename{x}{\rho}$ заменя в дървото за израза $\tau$, всяко листо означено с $x$, което не участва в дърво с корен $\lambda x:a$, с дървото за израза $\rho$.}
\begin{itemize}
\item
  Ако $\tau \equiv \vv{n}$, то $\tau\rename{x}{\rho} \equiv \vv{n}$.
\item
  Ако $\tau \equiv \vv{x}$, то $\tau\rename{x}{\rho} \equiv \rho$.
\item
  Ако $\tau \equiv \vv{y}$, където $\vv{y} \not\equiv \vv{x}$, то $\tau\rename{x}{\rho} \equiv \vv{y}$.
\item
  Ако $\tau \equiv \tau_1 + \tau_2$, то $\tau\rename{x}{y} \equiv \tau_1\rename{x}{\rho} + \tau_2\rename{x}{\rho}$.
\item
  Ако $\tau \equiv \tau_1\ \vv{==}\ \tau_2$, то $\tau\rename{x}{\rho} \equiv \tau_1\rename{x}{\rho}\ \vv{==}\ \tau_2\rename{x}{\rho}$.
\item
  Ако $\tau \equiv \ifelse{\tau_1}{\tau_2}{\tau_3}$, то
  \[\tau\rename{x}{\rho} \equiv \ifelse{\tau_1\rename{x}{\rho}}{\tau_2\rename{x}{\rho}}{\tau_3\rename{x}{\rho}}.\]
\item
  Ако $\tau \equiv \lamb{x}{a}{\tau'}$, то $\tau\rename{x}{\rho} \equiv \tau$;
\item
  Ако $\tau \equiv \lamb{y}{a}{\tau'}$ и $\vv{y} \not\equiv \vv{x}$, то
  $\tau\rename{x}{\rho} \equiv \lamb{y}{a}{\tau'\rename{x}{\rho}}$.
\end{itemize}

Нека $\tau \equiv \lamb{x}{a}{\vv{x+y}}$. Обърнете внимание, че $\tau\rename{y}{\vv{x}} \equiv \lamb{x}{a}{\vv{x+x}}$,
т.е. тук получаваме израз, който ,,смислово'' е доста различен от първоначалния израз $\tau$.
Проблемът се състои в това, че при замяната на $\vv{x}$ с $\rho$, някоя свободна променлива на $\rho$ може да попадне под обхвата на някоя свързана
променлива на $\tau$.

\index{$\alpha$-еквивалентност}
\marginpar{Това е подходът на Хаскел Къри за дефиниране на замяна на променлива с израз \cite[стр. 578]{barendregt-handbook}.}
Сега ще дефинираме бинарна релация между изрази, която ще наричаме $\alpha$-ек\-ви\-ва\-лент\-ност.
Интуитивно, всеки два $\alpha$-еквивалентни израза трябва да бъдат смислово неотличими.
Това е най-малката релация между изрази, която ще означаваме с $\alphaEq$, за която са изпълнени свойствата:
\marginpar{
  В тази дефиниция единствено последният случай е интересен.
  Например,
  \[(\lamb{x}{a}{\vv{x+z}})\vv{x} \equiv_\alpha (\lamb{y}{a}{\vv{y+z}})\vv{x},\]
  но \[(\lamb{x}{a}{\vv{x+y}})\vv{x} \not\equiv_\alpha (\lamb{y}{a}{\vv{x+y}})\vv{x}.\]}
\begin{itemize}
\item
  $\vv{x} \alphaEq \vv{x}$;
\item
  $\vv{n} \alphaEq \vv{n}$;
\item
  ако $\tau_1 \alphaEq \rho_1$ и $\tau_2 \equiv_\alpha \rho_2$, то имаме, че
  \begin{align*}
    & \tau_1 + \tau_2 \alphaEq \rho_1 + \rho_2,\\
    & \tau_1\ \vv{==}\ \tau_2 \alphaEq \rho_1\ \vv{==}\ \rho_2,\\
    & \tau_1 \tau_2 \alphaEq \rho_1 \rho_2;
  \end{align*}
\item
  ако $\tau_1 \alphaEq \rho_1$, $\tau_2 \equiv_\alpha \rho_2$ и $\tau_3 \equiv_\alpha \rho_3$, то
  \[\ifelse{\tau_1}{\tau_2}{\tau_3} \alphaEq \ifelse{\rho_1}{\rho_2}{\rho_3};\]
\item
  ако $\tau \alphaEq \rho$, то $\fix(\tau) \equiv_\alpha \fix(\rho)$;
\item  
  ако $\tau\rename{x}{\vv{z}} \alphaEq \rho\rename{y}{\vv{z}}$, където $\vv{z} \not\in \var(\tau) \cup \var(\rho)$, то
  \[\lamb{x}{a}{\tau} \alphaEq \lamb{y}{a}{\rho}.\]
\end{itemize}

\marginpar{
Например,
\[\lamb{x}{a}{\vv{x+y}} \alphaEq \lamb{z}{a}{\vv{z+y}},\]
но
\[\lamb{x}{a}{\vv{x+y}} \not\alphaEq \lamb{y}{a}{\vv{y+y}},\]
защото
\[(\vv{x+y})\rename{x}{\vv{u}} \not\alphaEq (\vv{y+y})\rename{y}{\vv{u}}.\]}
Също така,
\[\lamb{x}{nat}{\vv{x + }\fix(\lamb{x}{nat}{\vv{x}})} \alphaEq \lamb{y}{nat}{\vv{y + }\fix(\lamb{z}{nat}{\vv{z}})}.\]

\index{терм}
\begin{framed}
\begin{definition}
  PCF терм е клас на еквивалентност от PCF изрази относно релацията $\alpha$-еквивалентност.
\end{definition}
\end{framed}

\marginpar{Това да се премести в нова глава за ламбда смятане.}
Де Бройн има по-програмистки подход за дефинирането на PCF термовете.
Всеки PCF терм има единствено представяне като израз, в който свързаните променливи са заменени с индекси.

{\bf Пример ......}


Сега искаме $\tau\subst{x}{\rho}$ да означава изразът получен от израза $\tau$, в който всяко \emph{свободно} срещане на променливата $\vv{x}$
е заменена с израза $\rho$. Тук имаме потенциален проблем. Искаме да направим тази замяна по такъв начин, че свободни променливи на $\rho$
да не попаднат под обхвата на свързани променливи от $\tau$. За да направим това, операцията $\subst{x}{\rho}$
трябва да работи не върху отделни изрази, а върху термове.
\index{субституция}
Можем да дадем формална дефиниция с индукция по построението на термовете:
\begin{itemize}
\item
  Ако $\tau \equiv \vv{n}$, то $\tau\subst{x}{\rho} \equiv \vv{n}$.
\item
  Ако $\tau \equiv \vv{x}$, то $\tau\subst{x}{\rho} \equiv \rho$.
\item
  Ако $\tau \equiv \vv{y}$ и $\vv{y} \not\equiv \vv{x}$, то $\tau\subst{x}{\rho} \equiv \vv{y}$.
\item
  Ако $\tau \equiv \tau_1 + \tau_2$, то
  \[\tau\subst{x}{y} \equiv \tau_1\subst{x}{\rho} + \tau_2\subst{x}{\rho}.\]
\item
  Ако $\tau \equiv \tau_1\ \vv{==}\ \tau_2$, то
  \[\tau\subst{x}{\rho} \equiv \tau_1\subst{x}{\rho}\ \vv{==}\ \tau_2\subst{x}{\rho}.\]
\item
  Ако $\tau \equiv \ifelse{\tau_1}{\tau_2}{\tau_3}$, то
  \[\tau\subst{x}{\rho} \equiv \ifelse{\tau_1\subst{x}{\rho}}{\tau_2\subst{x}{\rho}}{\tau_3\subst{x}{\rho}}.\]
\item
  Ако $\tau \equiv \lamb{y}{a}{\tau'}$, то
  \[\tau\subst{x}{\rho} \equiv \lamb{z}{a}{(\tau'\subst{y}{\vv{z}}\subst{x}{\rho})},\]
  където $\vv{z} \not\in \fv(\tau') \cup \fv(\rho) \cup \{\vv{x}\}$.
\end{itemize}

В тази дефиниция отново единствено последният случай е интересен.
Обърнете внимание, че според него заместването на $\vv{x}$ с $\rho$ дава като резултат безкрайно много
изрази, всички от които са $\alpha$-еквивалентни, т.е. ако работим на ниво термове операцията е добре дефинирана.

Това означава, че няма значение дали ще говорим за $\lamb{x}{a}{\vv{x+y}}$ или за $\lamb{y}{a}{\vv{y+z}}$.
Тези два израза описват един и същи терм и тези два израза са $\alpha$-еквивалентни.


%%% Local Variables:
%%% mode: latex
%%% TeX-master: "../sep"
%%% End:


\section{Добре типизирани термове}\index{тип}

Типовите контексти представляват крайни редици от двойки от вида $\type{x}{a}$, т.е.
\[\Gamma ::= \emptyset\ |\ \Gamma,\type{x}{a}.\]
Например,
\[\Gamma = \vv{x}:\vv{nat},\ \vv{y}:\vv{nat} \to \vv{nat},\ \vv{z}:\vv{nat}.\]
\marginpar{Обикновено ще означаваме типовите контексти с главните гръцки $\Gamma, \Delta, \dots$.}
На един типов контекст $\Gamma$ може да се гледа и като на \emph{крайна} функция приемаща като аргумент променлива и връщаща тип.
Например, може понякога да пишем $\Gamma(\vv{x}) = \vv{nat}$, когато искаме да кажем, че променливата $\vv{x}$ има тип $\vv{nat}$ в типовия контекст $\Gamma$.
\marginpar{$\Gamma$ също се нарича и type environment.}
Ние искаме да работим само с коректно типизирани термове.
Например, не е ясно какво означава терма $\vv{1 + }\lamb{x}{nat}{\vv{x}}$,
защото трябва да съберем число и функция - два обекта от различен тип.
Ако в изразите имаме свободни променливи, то дали един израз е коректно типизиран ще зависи от типовия контекст.
Например, термът $\lamb{x}{nat}{\vv{z+x}}$ е добре типизиран в контекста $\Gamma = \vv{z}:\vv{nat}$, но не е добре типизиран в контекста
$\Delta = \vv{z}:\vv{nat}\to\vv{nat}$.
Сега ще дефинираме релация $\Gamma \vdash \tau : \vv{a}$, която ще ни казва, че термът $\tau$, относно типовият контекст $\Gamma$,
е добре типизиран и има тип $\vv{a}$.

\marginpar{Обърнете внимание, че можем да правим доказателства свойства на типизиращата релация с индукция по простроението на термовете.}

\begin{figure}[H]
  \centering
\begin{prooftree}
  \AxiomC{}
  \RightLabel{\scriptsize{(const)}}
  \UnaryInfC{$\Gamma \vdash \vv{n} : \vv{nat}$}
\end{prooftree}

\begin{prooftree}
  \AxiomC{$\vv{x} \in \texttt{dom}(\Gamma)$}
  \AxiomC{$\Gamma(\vv{x}) = \vv{a}$}
  \RightLabel{\scriptsize{(var)}}
  \BinaryInfC{$\Gamma \vdash \vv{x} : \vv{a}$}
\end{prooftree}

\begin{prooftree}
  \AxiomC{$\Gamma \vdash \tau_1:\vv{nat}$}
  \AxiomC{$\Gamma \vdash \tau_2:\vv{nat}$}
  \RightLabel{\scriptsize{(plus)}}
  \BinaryInfC{$\Gamma \vdash \tau_1 + \tau_2 : \vv{nat}$}
\end{prooftree}

\begin{prooftree}
  \AxiomC{$\Gamma \vdash \tau_1:\vv{nat}$}
  \AxiomC{$\Gamma \vdash \tau_2:\vv{nat}$}
  \RightLabel{\scriptsize{(eq)}}
  \BinaryInfC{$\Gamma \vdash \tau_1\ \vv{==}\ \tau_2 : \vv{nat}$}
\end{prooftree}

\begin{prooftree}
  \AxiomC{$\Gamma \vdash \tau_1:\vv{nat}$}
  \AxiomC{$\Gamma \vdash \tau_2:\vv{a}$}
  \AxiomC{$\Gamma \vdash \tau_3:\vv{a}$}
  \RightLabel{\scriptsize{(if)}}
  \TrinaryInfC{$\Gamma \vdash \ifelse{\tau_1}{\tau_2}{\tau_3} : \vv{a}$}
\end{prooftree}

\begin{prooftree}
  \AxiomC{$\Gamma \vdash \tau_1:\vv{a}\to\vv{b}$}
  \AxiomC{$\Gamma \vdash \tau_2:\vv{a}$}
  \RightLabel{\scriptsize{(app)}}
  \BinaryInfC{$\Gamma \vdash \tau_1\tau_2 : \vv{b}$}
\end{prooftree}

\begin{prooftree}
  \AxiomC{$\Gamma \vdash \tau:\vv{a}\to\vv{a}$}
  \RightLabel{\scriptsize{(fix)}}
  \UnaryInfC{$\Gamma \vdash \fix(\tau) : \vv{a}$}
\end{prooftree}

\begin{prooftree}
  \AxiomC{$\vv{x} \not\in\vv{dom}(\Gamma)$}
  \AxiomC{$\Gamma, \type{x}{a} \vdash \tau:\vv{b}$}
  \RightLabel{\scriptsize{(lambda)}}
  \BinaryInfC{$\Gamma \vdash \lambda \type{x}{a}\ .\ \tau : \vv{a} \to \vv{b}$}
\end{prooftree}
  
  \caption{Релация за типизиране на термовете от езика \PCF}
  \label{fig:pcf:types:relation}
\end{figure}


Ако имаме затворен израз $\tau$, то ще пишем $\tau : \vv{a}$ вместо $\emptyset \vdash \tau : \vv{a}$.
Да положим
\[\text{PCF}_{\vv{a}} \df \{\tau \text{ е затворен терм}\mid \emptyset \vdash \tau : \vv{a}\}.\]

\begin{example}
  ????
\end{example}



\begin{proposition}
  Ако $\Gamma \vdash \tau : \vv{a}$, то $\fv(\tau) \subseteq \vv{dom}(\Gamma)$.
\end{proposition}

\begin{proposition}
  Ако $\Gamma \vdash \tau : \vv{a}$ и $\Gamma \vdash \tau : \vv{b}$, то $\vv{a} = \vv{b}$.
\end{proposition}

\begin{corollary}
  Всеки затворен терм има най-много един тип.
\end{corollary}

\begin{problem}
  \marginpar{\cite[стр. 104]{types-programming-languages}}
  Докажете или опровергайте дали е възможно да съществува типов контекст $\Gamma$ и тип $\vv{a}$, такива че
  $\Gamma \vdash \type{xx}{a}$.
\end{problem}

%%% Local Variables:
%%% mode: latex
%%% TeX-master: "../sep"
%%% End:


\section{Операционна семантика}\label{pcf:sect:operational-cbn}
\index{операционна семантика}

За затворен терм $\tau : \vv{a}$ и стойност $\vv{v} : \vv{a}$, дефинираме релацията $\tau \opsem{\ell}{a} \vv{v}$ по
следния начин.
\marginpar{Да напомним, че с $\vv{v}$ означаваме стойности, които биват или константи или затворени термове от вида $\lamb{x}{a}{\mu}$.

  Обърнете внимание, че заради случите (fix) и (cbn) не можем да доказваме свойства на операционната семантика с индукция по построението на термовете. Можем да направим това с индукция по броя на стъпки в изчислението.}

\begin{figure}[H]
  \centering
  
\begin{prooftree}
  \AxiomC{$\type{v}{a}$}
  \RightLabel{\scriptsize{(val)}}
  \UnaryInfC{$\vv{v} \opsem{0}{a} \vv{v}$}
\end{prooftree}

\begin{prooftree}
  \AxiomC{$\tau_1 \opsem{\ell_1}{nat} \vv{n}_1$}
  \AxiomC{$\tau_2 \opsem{\ell_2}{nat} \vv{n}_2$}
  \AxiomC{$n = \eq(n_1,n_2)$}
  \RightLabel{\scriptsize{(eq)}}
  \TrinaryInfC{$\tau_1\ \vv{==}\ \tau_2 \opsem{\ell_1+\ell_2+1}{nat} \vv{n}$}
\end{prooftree}

\begin{prooftree}
  \AxiomC{$\tau_1 \opsem{\ell_1}{nat} \vv{n}_1$}
  \AxiomC{$\tau_2 \opsem{\ell_2}{nat} \vv{n}_2$}
  \AxiomC{$n = \plus(n_1,n_2)$}
  \RightLabel{\scriptsize{(plus)}}
  \TrinaryInfC{$\tau_1\ \vv{+}\ \tau_2 \opsem{\ell_1+\ell_2+1}{nat} \vv{n}$}
\end{prooftree}

\begin{prooftree}
  \AxiomC{$\tau_1 \opsem{\ell_1}{nat} \vv{n}_1$}
  \AxiomC{$\tau_2 \opsem{\ell_2}{nat} \vv{n}_2$}
  \AxiomC{$n = \minus(n_1,n_2)$}
  \RightLabel{\scriptsize{(minus)}}
  \TrinaryInfC{$\tau_1\ \vv{-}\ \tau_2 \opsem{\ell_1+\ell_2+1}{nat} \vv{n}$}
\end{prooftree}

\begin{prooftree}
  \AxiomC{$\tau_1 \opsem{\ell_1}{nat} \vv{0}$}
  \AxiomC{$\tau_3 \opsem{\ell_2}{a} \vv{v}$}
  \RightLabel{\scriptsize{(if$_0$)}}
  \BinaryInfC{$\ifelse{\tau_1}{\tau_2}{\tau_3} \opsem{\ell_1+\ell_2+1}{a} \vv{v}$}
\end{prooftree}

\begin{prooftree}
  \AxiomC{$\tau_1 \opsem{\ell_1}{nat} \vv{n}$}
  \AxiomC{$\tau_2 \opsem{\ell_2}{a} \vv{v}$}
  \AxiomC{$\vv{n} \not\equiv \vv{0}$}
  \RightLabel{\scriptsize{(if$^+$)}}
  \TrinaryInfC{$\ifelse{\tau_1}{\tau_2}{\tau_3} \opsem{\ell_1+\ell_2+1}{a} \vv{v}$}
\end{prooftree}

\begin{prooftree}
  \AxiomC{$\tau_1 \opsemGen{\ell_1}{\vv{a}\to\vv{b}} \lamb{x}{a}{\tau'_1}$}
  \AxiomC{$\tau'_1\subst{x}{\tau_2} \opsem{\ell_2}{b} \vv{v}$}
  \RightLabel{\scriptsize{(cbn)}}
  \BinaryInfC{$\tau_1 \tau_2 \opsem{\ell_1+\ell_2+1}{b} \vv{v} $}
\end{prooftree}

\begin{prooftree}
  \AxiomC{$\tau\ \fix(\tau) \opsem{\ell}{a} \vv{v}$}
  \RightLabel{\scriptsize{(fix)}}
  \UnaryInfC{$\fix(\tau) \opsem{\ell+1}{a} \vv{v} $}
\end{prooftree}
\caption{Правила на операционната семантика за езика \PCF}
\end{figure}



\begin{itemize}
\item 
  Ще пишем $\tau \opsem{}{a} \vv{v}$, ако съществува $\ell$, за което $\tau \opsem{\ell}{a} \vv{v}$.  
\item
  Също така, ще пишем $\tau \not\opsem{}{a}$, ако не съществува стойност $\vv{v}$, за която $\tau \opsem{}{a} \vv{v}$.  
\end{itemize}

\begin{lemma}
  За произволен затворен терм $\tau$ и стойности $\vv{v}$ и $\vv{u}$,
  \[\tau \opsem{}{a} \vv{v}\ \&\ \tau \opsem{}{a} \vv{u}\ \implies\ \vv{v} \equiv \vv{u}.\]
\end{lemma}







%%% Local Variables:
%%% mode: latex
%%% TeX-master: "../sep"
%%% End:

\newpage
\begin{example}
  Нека $\vv{a} = \vv{nat} \to (\vv{nat} \to \vv{nat})$ и 
  \[\tau \equiv \fix\vv{(}\underbrace{\lamb{f}{a}{\overbrace{\lamb{x}{nat}{\lamb{y}{nat}{\ifelse{\vv{y == 0}}{\vv{0}}{\vv{x + (f x (y-1))} }}}}^{\tau'_0}}}_{\tau_0}\vv{)}.\]
  Лесно се вижда, че
  \[\emptyset \vdash \tau:\vv{nat} \to (\vv{nat} \to \vv{nat}).\]

  Също така за всяко $n$ и $k$, ако $m = n * k$, то
  \[\tau\ \vv{n}\ \vv{k} \Downarrow_{\vv{nat}} \vv{m}.\]
  Нека сега
  \[\rho \equiv \lamb{g}{a}{\fix(\lambda \vv{f} : \vv{nat}\to\vv{nat}\ .\ \lamb{x}{nat}{\ifelse{\vv{x == 0}}{\vv{1}}{\vv{g x f(x-1)}}}}).\]
  Лесно се вижда, че
  \[\emptyset \vdash \rho : \vv{a} \to (\vv{nat} \to \vv{nat}).\]
  Също така, за всяко $n$, ако $k = n!$, то
  \[ \rho\ \tau\ \vv{n} \opsem{}{nat} \vv{k}.\]
\end{example}

Нека да положим
\[\Gamma \df \vv{f} : \vv{a}, \vv{x} : \vv{nat}, \vv{y} : \vv{nat}.\]


\begin{landscape}
  \begin{framed}
  \begin{figure}[H]
    \centering
    % \begin{subfigure}[b]{1\textwidth}
      \begin{prooftree}
        \AxiomC{$\Gamma(\vv{y}) = \vv{nat}$}
        \UnaryInfC{$\Gamma \vdash \vv{y} : \vv{nat}$}
        \AxiomC{}
        \UnaryInfC{$\Gamma \vdash \vv{0} : \vv{nat}$}
        \BinaryInfC{$\Gamma \vdash \vv{y==0} : \vv{nat}$}
        \AxiomC{}
        \UnaryInfC{$\Gamma \vdash \vv{0} : \vv{nat}$}
        \AxiomC{$\Gamma(\vv{x}) = \vv{nat}$}
        \UnaryInfC{$\Gamma \vdash \vv{x}:\vv{nat}$}
        \AxiomC{$\Gamma(\vv{f}) = \vv{a}$}
        \UnaryInfC{$\Gamma \vdash \vv{f} : \vv{a}$}
        \AxiomC{$\Gamma(\vv{x}) = \vv{nat}$}
        \UnaryInfC{$\Gamma \vdash \vv{x} : \vv{nat}$}
        \BinaryInfC{$\Gamma \vdash \vv{f x} : \vv{nat} \to \vv{nat}$}
        \AxiomC{$\Gamma(\vv{y}) = \vv{nat}$}
        \UnaryInfC{$\Gamma \vdash \vv{y} : \vv{nat}$}
        \AxiomC{}
        \UnaryInfC{$\Gamma \vdash \vv{1} : \vv{nat}$}
        \BinaryInfC{$\Gamma \vdash \vv{y-1} : \vv{nat}$}
        \BinaryInfC{$\Gamma \vdash \vv{f x (y-1)} : \vv{nat}$}
        \BinaryInfC{$\Gamma \vdash \vv{x + f x (y-1)} : \vv{nat}$}
        \TrinaryInfC{$\vv{f} : \vv{a}, \vv{x} : \vv{nat}, \vv{y} : \vv{nat} \vdash \ifelse{\vv{y == 0}}{\vv{0}}{\vv{x + (f x (y-1))}} : \vv{nat}$}
        \UnaryInfC{$\vv{f} : \vv{a}, \vv{x} : \vv{nat} \vdash \lamb{y}{nat}{\ifelse{\vv{y == 0}}{\vv{0}}{\vv{x + (f x (y-1))}}} : \vv{nat} \to \vv{nat}$}
        \UnaryInfC{$\vv{f} : \vv{a} \vdash \tau'_0 : \vv{a}$}
        \UnaryInfC{$\emptyset \vdash \tau_0 : \vv{a} \to \vv{a}$}
        \UnaryInfC{$\emptyset \vdash \fix(\tau_0):\vv{a}$}
      \end{prooftree}
      \caption{Формален извод според правилата на типизиращата релацията, който показва, че $\emptyset \vdash \tau : \vv{a}$.}
    % \end{subfigure}
  \end{figure}
\end{framed}

След като видяхме, че $\tau$ е терм от тип $\vv{a}$, нека да видим
колко стъпки ще са ни нужни, според правилата на операционната семантика, за да проверим, че
$\tau\ \vv{3}\ \vv{2} \opsem{}{nat} \vv{6}$.

Първо да обърнем внимание, че $\tau_0$ е стойност, т.е. $\tau_0 \opsemGen{0}{\vv{a}\to\vv{a}} \tau_0$.
Макар и $\tau'_0$ да има свободна променлива $\vv{f}$, понеже термът $\fix(\tau_0)$ е затворен с тип $\vv{a}$, то термът
\[\tau'_0\subst{f}{\fix(\tau_0)} \equiv \lamb{x}{nat}{\lamb{y}{nat}{\ifelse{\vv{y == 0}}{\vv{0}}{\vv{ x + (}\fix(\tau_0)\vv{ x (y-1))}}}}\]
е стойност и следователно,
\[\tau'_0\subst{f}{\fix(\tau_0)} \opsem{0}{a} \tau'_0\subst{f}{\fix(\tau_0)}.\]

\begin{framed}
  \begin{figure}[H]
    \begin{prooftree}
      \AxiomC{$\tau_0$ е стойност}
      \UnaryInfC{$\tau_0 \opsemGen{0}{\vv{a}\to\vv{a}} \lamb{f}{a}{\tau'_0}$}
      \AxiomC{$\tau'_0\subst{f}{\fix(\tau_0)}$ е стойност}
      \UnaryInfC{$\tau'_0\subst{f}{\fix(\tau_0)} \opsem{0}{a}  \lamb{x}{nat}{\tau_2}$}
      \BinaryInfC{$\tau_0\fix(\tau_0) \opsem{1}{a} \lamb{x}{nat}{\tau_2}$}
      \UnaryInfC{$\fix(\tau_0) \opsem{2}{a} \lamb{x}{nat}{\tau_2}$}
      \AxiomC{$\tau_2\substConst{x}{3}$ е стойност}
      \UnaryInfC{$\tau_2\substConst{x}{3} \opsemGen{0}{\vv{nat}\to \vv{nat}} \lamb{y}{nat}{\tau_1}$}
      \BinaryInfC{$\fix(\tau_0)\ \vv{3} \opsemGen{3}{\vv{nat}\to\vv{nat}} \lamb{y}{nat}{\tau_1}$}
      \AxiomC{(продължава по-долу)}
      \UnaryInfC{$\tau_1[\vv{y}/\vv{2}] \opsem{t}{nat} \vv{6}$}
      \RightLabel{(app)}
      \BinaryInfC{$\fix(\tau_0)\ \vv{3}\ \vv{2} \opsem{t+4}{nat} \vv{6}$}
    \end{prooftree}
    \caption{Първа част от изчислението на $\tau\ \vv{3}\ \vv{2}$ според правилата на операционната семантика.}
  \end{figure}
\end{framed}

Нека за улеснение да положим
\begin{align*}
  \tau_2 & \equiv \lamb{y}{nat}{\ifelse{\vv{y == 0}}{\vv{0}}{\vv{ x + (}\fix(\tau_0)\vv{ x (y-1))}}}\\
  \tau_1 & \equiv \ifelse{\vv{y == 0}}{\vv{0}}{\vv{ 3 + (}\fix(\tau_0)\vv{ 3 (y-1))}}.
\end{align*}

Термът $\tau_2$ не е стойност, защото има свободна променлива $\vv{x}$, но вече термът $\tau_2\subst{x}{3}$ е стойност. Лесно се съобразява, че
\begin{align*}
  & \tau'_0\subst{f}{\fix(\tau_0)} \equiv \lamb{x}{nat}{\tau_2}\\
  & \tau_2 \substConst{x}{3} \equiv \lamb{y}{nat}{\tau_1}.
\end{align*}



% Оттук съобразяваме, че
%   \[\tau_2 \equiv \lamb{y}{nat}{\ifelse{\vv{y == 0}}{\vv{0}}{\vv{ x + (}\fix(\tau_0)\vv{ x (y-1))}}}.\]
%   Това означава, че
%   \[\tau_1 \equiv \ifelse{\vv{y == 0}}{\vv{0}}{\vv{ 3 + (}\fix(\tau_0)\vv{ 3 (y-1))}}\]
%   и
%   \[\tau_1\substConst{y}{2} \equiv \ifelse{\vv{2 == 0}}{\vv{0}}{\vv{ 3 + (}\fix(\tau_0)\vv{ 3 (2-1))}}\]
  Сега продължаваме десния клон на изчислението:
  \begin{framed}
    \begin{figure}[H]
      \begin{prooftree}
        \AxiomC{$\vv{2}$ е стойност}
        \UnaryInfC{$\vv{2} \opsem{0}{nat} \vv{2}$}
        \AxiomC{$\vv{0}$ е стойност}
        \UnaryInfC{$\vv{0} \opsem{0}{nat} \vv{0}$}
        \BinaryInfC{$\vv{2 == 0} \opsem{1}{nat} \vv{0}$}
        \AxiomC{$\vv{3}$ е стойност}
        \UnaryInfC{$\vv{3} \opsem{0}{nat} \vv{3}$}
        \AxiomC{(Повтаря се както по-горе)}
        \UnaryInfC{$\fix(\tau_0)\ \vv{3} \opsemGen{3}{\vv{nat}\to\vv{nat}} \lamb{y}{nat}{\tau_1}$}
        \AxiomC{(продължава по-долу)}
        \UnaryInfC{$\tau_1\substConst{y}{2-1} \opsem{k}{nat} \vv{3}$}
        \BinaryInfC{$\fix(\tau_0)\vv{ 3 (2-1) }\opsem{k+4}{nat} \vv{3}$}
        \BinaryInfC{$\vv{ 3 + (}\fix(\tau_0)\vv{ 3 (2-1))} \opsem{k+5}{nat} \vv{6}$}
        \BinaryInfC{$\underbrace{\ifelse{\vv{2 == 0}}{\vv{0}}{\vv{ 3 + (}\fix(\tau_0)\vv{ 3 (2-1))}}}_{\tau_1\substConst{y}{2}} \opsem{k+6}{nat} \vv{6}$}
      \end{prooftree}
      \caption{Втора част от изчислението, която показва, че $\tau_1\substConst{y}{2} \opsem{k+6}{nat} \vv{6}$.}
    \end{figure}
  \end{framed}
  Отново продължаваме с десния клон на изчислението, като тук вече имаме, че:
  \[\tau_1\substConst{y}{2-1} \equiv \ifelse{\vv{2-1 == 0}}{\vv{0}}{\vv{ 3 + (}\fix(\tau_0)\vv{ 3 (2-1-1))}}\]

  \begin{framed}
    \begin{figure}[H]
      \begin{prooftree}
        \AxiomC{$\vv{2}$ е стойност}
        \UnaryInfC{$\vv{2} \opsem{0}{nat} \vv{2}$}
        \AxiomC{$\vv{1}$ е стойност}
        \UnaryInfC{$\vv{1} \opsem{0}{nat} \vv{1}$}
        \BinaryInfC{$\vv{2-1} \opsem{1}{nat} \vv{1}$}
        \AxiomC{$\vv{0}$ е стойност}
        \UnaryInfC{$\vv{0} \opsem{0}{nat} \vv{0}$}
        \BinaryInfC{$\vv{2-1 == 0} \opsem{2}{nat} \vv{0}$}
        \AxiomC{$\vv{3}$ е стойност}
        \UnaryInfC{$\vv{3} \opsem{0}{nat} \vv{3}$}
        \AxiomC{(Повтаря се както по-горе)}
        \UnaryInfC{$\fix(\tau_0)\ \vv{3} \opsemGen{3}{\vv{nat}\to\vv{nat}} \lamb{y}{nat}{\tau_1}$}
        % \AxiomC{(продължава в (\ref{fig:operational-cbn-example:last-part}))}
        \AxiomC{(продължава по-долу)}
        \UnaryInfC{$\tau_1\substConst{y}{2-1-1} \opsem{\ell}{nat} \vv{0}$}
        \BinaryInfC{$\fix(\tau_0)\vv{ 3 (2-1-1) }\opsem{\ell+4}{nat} \vv{0}$}
        \BinaryInfC{$\vv{ 3 + (}\fix(\tau_0)\vv{ 3 (2-1-1))} \opsem{\ell+5}{nat} \vv{3}$}
        \BinaryInfC{$\underbrace{\ifelse{\vv{2-1 == 0}}{\vv{0}}{\vv{ 3 + (}\fix(\tau_0)\vv{ 3 (2-1-1))}}}_{\tau_1\substConst{y}{2-1}} \opsem{\ell+6}{nat} \vv{3}$}
      \end{prooftree}
      \caption{Трета част от изчислението, която показва, че $\tau_1\substConst{y}{2-1} \opsem{\ell+6}{nat} \vv{3}$ }
    \end{figure}
  \end{framed}
  Сега вече имаме, че
  \[\tau_1\substConst{y}{2-1-1} \equiv \ifelse{\vv{2-1-1 == 0}}{\vv{0}}{\vv{ 3 + (}\fix(\tau_0)\vv{ 3 (2-1-1-1))}}.\]

  Завършваме изчислението така:
  \begin{framed}
    \begin{figure}[H]
      \label{fig:operational-cbn-example:last-part}
      \begin{prooftree}
        \AxiomC{$\vv{2}$ е стойност}
        \UnaryInfC{$\vv{2} \opsem{0}{nat} \vv{2}$}
        \AxiomC{$\vv{1}$ е стойност}
        \UnaryInfC{$\vv{1} \opsem{0}{nat} \vv{1}$}
        \BinaryInfC{$\vv{2-1} \opsem{1}{nat} \vv{1} $}
        \AxiomC{$\vv{1}$ е стойност}
        \UnaryInfC{$\vv{1} \opsem{0}{nat} \vv{1}$}
        \BinaryInfC{$\vv{2-1-1} \opsem{2}{nat} \vv{0}$}
        \AxiomC{$\vv{0}$ е стойност}
        \UnaryInfC{$\vv{0} \opsem{0}{nat} \vv{0}$}
        \BinaryInfC{$\vv{2-1-1 == 0} \opsem{3}{nat} \vv{1}$}
        \AxiomC{$\vv{0}$ е стойност}
        \UnaryInfC{$\vv{0} \opsem{0}{nat} \vv{0}$}
        \BinaryInfC{$\underbrace{\ifelse{\vv{2-1-1 == 0}}{\vv{0}}{\vv{ 3 + (}\fix(\tau_0)\vv{ 3 (2-1-1-1))}}}_{\tau_1\substConst{y}{2-1-1}} \opsem{4}{nat} \vv{0}$}
      \end{prooftree}
      \caption{Последна част от изчислението, което показва, че $\tau_1\substConst{y}{2-1-1} \opsem{4}{nat} \vv{0}$.}
    \end{figure}
  \end{framed}  
\end{landscape}


Можем да напишем директно горния пример и на хаскел:
\begin{haskellcode}
ghci> fix f = f (fix f)
ghci> times = fix(\f x y -> if y == 0 then 0 else x + (f x (y-1)))
ghci> times 2 3
6
ghci> fct = \g -> fix(\f x -> if x == 0 then 1 else g x (f (x-1)))
ghci> fact = fct times
ghci> fact 5
120
\end{haskellcode}


\newpage
\section{Денотационна семантика с предаване на параметрите по име}

Семантиката на всеки тип ще бъде област на Скот както следва:
\begin{align*}
  & \val{\vv{nat}} \df \Nat_\bot\\
  & \val{\vv{a} \to \vv{b}} \df \Cont{\val{\vv{a}}}{\val{\vv{b}}}.
\end{align*}
\marginpar{Да напомним, че \[\emptyset_\bot = (\{\bot\},\sqsubseteq,\bot).\]}
\marginpar{От Раздел~\ref{subsect:domains:product} знаем, че $\val{\Gamma}$ е област на Скот.}
За един типов контекст $\Gamma$, дефинираме $\val{\Gamma}$ по следния начин:
\begin{itemize}
\item
  Ако $\Gamma = \emptyset$, то $\val{\Gamma} = \emptyset_\bot$;
\item
  Ако $\Gamma = \Gamma', \vv{x:a}$, то $\val{\Gamma} = \val{\Gamma'} \times \val{\vv{a}}$.
\end{itemize}
Например, ако $\Gamma = \vv{x}_1 : \vv{a}_1,\ \vv{x}_2 : \vv{a}_2,\ \vv{x}_3 : \vv{a}_3$, то
\[\val{\Gamma} = (\val{\vv{a}_1} \times \val{\vv{a}_2})\times \val{\vv{a}_3}.\]

Сега трябва да дефинираме семантика на термовете.
За всеки терм, за който $\fv(\tau) \subseteq \texttt{dom}(\Gamma)$ и
за произволни $\overline{u} \in \val{\Gamma}$, дефинираме неговата стойност $\val{\tau}_\Gamma(\overline{u})$ по следния начин:
\begin{itemize}
\item
  Нека $\tau \equiv \vv{n}$. Тогава
  \[\val{\vv{n}}_\Gamma(\overline{u}) \df n.\]
\item
  Нека $\tau \equiv \vv{x}_i$. Тогава
  \[\val{\vv{x}_i}_\Gamma(\overline{u}) \df u_i.\]
\item
  \marginpar{За $\texttt{plus}$ вижте Раздел~\ref{subsect:rec:term-value}.}
  Нека $\tau \equiv \tau_1 + \tau_2$. Тогава
  \[\val{\tau_1 + \tau_2}_\Gamma(\overline{u}) \df \texttt{plus}(\val{\tau_1}_\Gamma(\overline{u}), \val{\tau_2}_\Gamma(\overline{u})).\]
\item
  \marginpar{За $\texttt{eq}$ вижте Раздел~\ref{subsect:rec:term-value}.}
  Нека $\tau \equiv \tau_1\ \vv{==}\ \tau_2$. Тогава
  \[\val{\tau_1\ \vv{==}\ \tau_2}_\Gamma(\overline{u}) \df \texttt{eq}(\val{\tau_1}_\Gamma(\overline{u}), \val{\tau_2}_\Gamma(\overline{u})).\]
\item
  \marginpar{За $\texttt{if}$ вижте \Def{if}.}
  Нека $\tau \equiv \ifelse{\tau_1}{\tau_2}{\tau_3}$. Тогава
  \[\val{\ifelse{\tau_1}{\tau_2}{\tau_3}}_\Gamma(\overline{u}) \df \texttt{if}(\val{\tau_1}_\Gamma(\overline{u}),
  \val{\tau_2}_\Gamma(\overline{u}), \val{\tau_3}_\Gamma(\overline{u})).\]
\item
  \marginpar{За $\texttt{eval}$ вижте \Def{eval}.}
  Нека $\tau \equiv \tau_1 \tau_2$. Тогава
  \[\val{\tau_1 \tau_2}_\Gamma(\overline{u}) \df \texttt{eval}(\val{\tau_1}_\Gamma(\overline{u}), \val{\tau_2}_\Gamma(\overline{u})).\]
\item
  \marginpar{За $\lfp$ вижте Раздел~\ref{sect:lfp}.}
  Нека $\tau \equiv \fix(\tau')$. Тогава 
  \[\val{\fix(\tau')}_\Gamma(\overline{u}) \df \lfp(\val{\tau'}_\Gamma(\overline{u})).\]
\item
  \marginpar{За $\curry$ вижте \Def{curry}.}
  Нека $\tau \equiv \lamb{y}{b}{\tau'}$, като $\vv{y} \not \in \texttt{dom}(\Gamma)$.
  Нека $\Gamma' \df \Gamma, \type{y}{b}$. Тогава
  \[\val{\lamb{y}{b}{\tau'}}_\Gamma(\overline{u}) \df \curry(\val{\tau'}_{\Gamma'})(\overline{u}).\]
\end{itemize}

\begin{remark}
  За $\Gamma = \emptyset$, ще пишем $\val{\tau}$ вместо $\val{\tau}_\emptyset$.
\end{remark}

Не е ясно дали винаги горните дефиниции имат смисъл.
Сега ще докажем, че винаги, когато един терм е добре типизиран, то горната дефиниция има смисъл.

\begin{framed}
  \begin{lemma}
    Ако $\Gamma \vdash \tau : \vv{a}$, то $\val{\tau}_\Gamma \in \Cont{\val{\Gamma}}{\val{\vv{a}}}$.
  \end{lemma}  
\end{framed}
\begin{proof}
  Доказателството протича с индукция по построението на термовете
  като съществено използваме \Prop{composition} според което, ако $f \in \Cont{\A}{\B}$ и $g \in \Cont{\B}{\C}$, то
  $g \circ f \in \Cont{\A}{\C}$.
  \marginpar{Изображението $f \times g$ е дефинирано в \Prop{cartesian-continuous}.}
  \begin{itemize}
  \item
    Нека $\tau \equiv \vv{n}$. Щом $\Gamma \vdash \tau : \vv{a}$, то
    по правилата за типизиране следва, че $\vv{a} = \vv{nat}$.
    Сега лесно се съобразява, че изображението $\val{\vv{n}}_\Gamma \in \Cont{\val{\Gamma}}{\val{\vv{nat}}}$, където
    $\val{\vv{n}}_\Gamma(\overline{u}) = n$.
    Това е така, защото за всяка верига $\chain{\overline{u}}{i}$ от елементи на $\val{\Gamma}$,
    \[\val{\vv{n}}_\Gamma(\bigsqcup_i\overline{u}_i) = n = \bigsqcup_i\{ \val{\vv{n}}(\overline{u}_i)\}.\]
  \item
    Нека $\tau \equiv \vv{x}_i$. Щом $\Gamma \vdash \tau : \vv{a}$, то
    по правилата за типизиране следва, че $\vv{a} = \vv{a}_i$.
    Сега лесно се съобразява, че изображението $\val{\vv{n}}_\Gamma \in \Cont{\val{\Gamma}}{\val{\vv{a}_i}}$, където
    $\val{\vv{n}}_\Gamma(\overline{u}) = u_i$.
    \marginpar{$\overline{u}_k = (u_{1,k},\dots,u_{n,k})$.}
    Това е така, защото за всяка верига $\chain{\overline{u}}{n}$ от елементи на $\val{\Gamma}$,
    \[\val{\vv{x}_i}_\Gamma(\bigsqcup_n\overline{u}_n) = \bigsqcup_n u_{i,n} = \bigsqcup_i\{ \val{\vv{x}_i}(\overline{u}_n)\}.\]
  \item
    Нека $\tau \equiv \tau_1 + \tau_2$. Щом $\Gamma \vdash \tau : \vv{a}$, то
    по правилата за типизиране следва, че $\vv{a} = \vv{nat}$, а също и $\Gamma \vdash \tau_1 : \vv{nat}$ и $\Gamma \vdash \tau_2
    : \vv{nat}$.
    От И.П. имаме, че
    \begin{align*}
      & \val{\tau_1}_\Gamma \in \Cont{\val{\Gamma}}{\val{\vv{nat}}};\\
        & \val{\tau_2}_\Gamma \in \Cont{\val{\Gamma}}{\val{\vv{nat}}}.
    \end{align*}
    Това означава, че $(\val{\tau_1} \times \val{\tau_2}) \in \Cont{\val{\Gamma}}{\val{\vv{nat}} \times \val{\vv{nat}}}$.
    Тогава имаме следното равенство
    \marginpar{Използваме, че композиция на непрекъснати изображения е непрекъснато изображение.}
    \[\val{\tau_1 + \tau_2}_\Gamma = \texttt{plus} \circ (\val{\tau_1} \times \val{\tau_2}) \in \Cont{\val{\Gamma}}{\val{\vv{a}}},\]
    защото за произволни $\overline{u} \in \val{\Gamma}$,
    \begin{align*}
      (\texttt{plus} \circ (\val{\tau_1} \times \val{\tau_2}))(\overline{u}) & = \texttt{plus}((\val{\tau_1} \times \val{\tau_2})(\overline{u}))\\ 
                                                                             & = \texttt{plus}(\val{\tau_1}_\Gamma(\overline{u}), \val{\tau_2}_\Gamma(\overline{u}))\\
                                                                             & \df \val{\tau}_\Gamma(\overline{u}).
    \end{align*}
  \item
    Нека $\tau \equiv \tau_1\ \vv{==}\ \tau_2$. Съобразете сами, че 
    \[\val{\tau_1\ \vv{==}\ \tau_2}_\Gamma = \texttt{eq} \circ (\val{\tau_1}_\Gamma \times \val{\tau_2}_\Gamma) \in \Cont{\val{\Gamma}}{\val{\vv{a}}}.\]
  \item
    Нека $\tau \equiv \ifelse{\tau_1}{\tau_2}{\tau_3}$. Съобразете сами, че 
    \[\val{\ifelse{\tau_1}{\tau_2}{\tau_3}}_\Gamma = \texttt{if} \circ (\val{\tau_1}_\Gamma \times \val{\tau_2}_\Gamma \times \val{\tau_3}_\Gamma)  \in \Cont{\val{\Gamma}}{\val{\vv{a}}}.\]
  \item
    Нека $\tau \equiv \tau_1 \tau_2$.
    Щом $\Gamma \vdash \tau_1 \tau_2 : \vv{a}$, то от правилата за типизиране следва, че
    \begin{align*}
      & \Gamma \vdash \tau_1 : \vv{b} \to \vv{a}\\
      & \Gamma \vdash \tau_2 : \vv{b}.
    \end{align*}
    От И.П. за $\tau_1$ и $\tau_2$ знаем, че
    \begin{align*}
      & \val{\tau_1}_\Gamma \in \Cont{\val{\Gamma}}{\Cont{\val{\vv{b}}}{\val{\vv{a}}}} \\
      & \val{\tau_2}_\Gamma \in \Cont{\val{\Gamma}}{\val{\vv{b}}}
    \end{align*}
    Оттук получаваме, че за произволни $\overline{u} \in \val{\Gamma}$,
    \begin{align*}
      & \val{\tau_1}_\Gamma(\overline{u}) \in \Cont{\val{\vv{b}}}{\val{\vv{a}}} \\
      & \val{\tau_2}_\Gamma(\overline{u}) \in \val{\vv{b}}.
    \end{align*}
    Тогава 
    \[\val{\tau_1 \tau_2}_\Gamma = \texttt{eval} \circ (\val{\tau_1}_\Gamma \times \val{\tau_2}_\Gamma) \in \Cont{\val{\Gamma}}{\val{\vv{a}}},\]
    защото за произволни $\overline{u} \in \val{\Gamma}$,
    \begin{align*}
      (\texttt{eval} \circ \val{\tau_1}_\Gamma \times \val{\tau_2}_\Gamma)(\overline{u}) & = \texttt{eval}((\val{\tau_1}_\Gamma \times \val{\tau_2}_\Gamma)(\overline{u}))\\
                                                                                         & = \texttt{eval}(\val{\tau_1}_\Gamma(\overline{u}), \val{\tau_2}_\Gamma(\overline{u}))\\
                                                                                         & \df \val{\tau_1\tau_2}_\Gamma(\overline{u}).
    \end{align*}
    
  \item
    Нека сега $\tau \equiv \fix(\tau')$.
    Понеже $\Gamma \vdash \fix(\tau') : \vv{a}$, то от правилата за типизиране имаме, че
    $\Gamma \vdash \tau' : \vv{a} \to \vv{a}$.
    От И.П. знаем, че
    \[\val{\tau'}_\Gamma \in \Cont{\val{\Gamma}}{\Cont{\val{\vv{a}}}{\val{\vv{a}}}}.\]
    Това означава, че за произволни $\overline{u} \in \val{\Gamma}$,
    \[\val{\tau'}_\Gamma(\overline{u}) \in \Cont{\val{\vv{a}}}{\val{\vv{a}}}.\]
    Следователно
    $\val{\tau'}_\Gamma(\overline{u})$ е изображение, което според \Th{knaster-tarski}
    притежава най-малка неподвижна точка.
    \marginpar{Непрекъснатото изображението $Y$ е дефинирано \Th{Y}.}
    Тогава
    \[\val{\texttt{fix}(\tau')}_\Gamma = Y \circ \val{\tau'}_\Gamma \in \Cont{\val{\Gamma}}{\val{\vv{a}}},\]
    защото за произволни $\overline{u} \in \val{\Gamma}$,
    \begin{align*}
      (Y \circ \val{\tau'}_\Gamma)(\overline{u}) & = Y(\val{\tau'}_\Gamma(\overline{u}))\\
                                                 & = \lfp(\val{\tau'}_\Gamma(\overline{u}))\\
                                                 & \df \val{\fix(\tau')}_\Gamma(\ov{u}).
    \end{align*}
  \item
    Нека $\tau \equiv \lamb{y}{b}{\tau'}$, като $\vv{y} \not \in \texttt{dom}(\Gamma)$.
    Щом $\Gamma \vdash \lamb{y}{b}{\tau'} : \vv{a}$, то от правилата за типизиране следва, че $\vv{a} = \vv{b} \to \vv{c}$
    и 
    \[\Gamma, \type{y}{b} \vdash \tau' : \vv{c}.\]
    
    Нека $\Gamma' = \Gamma, \vv{y}:\vv{b}$. Тогава $\val{\Gamma'} = \val{\Gamma} \times \val{\vv{b}}$, а от И.П. имаме, че
    \[\val{\tau'}_{\Gamma'} \in \Cont{\val{\Gamma} \times \val{\vv{b}}}{\val{\vv{c}}}.\]
    Тогава от \Prop{curry} следва, че
    \[\val{\lamb{y}{b}{\tau}}_\Gamma \df \curry(\val{\tau'}_{\Gamma'}) \in \Cont{\val{\Gamma}}{\Cont{\val{\vv{b}}}{\val{\vv{c}}}}.\]
  \end{itemize}
\end{proof}

\begin{remark}
  В случая $\Gamma = \emptyset$, формално погледнато,
  $\val{\tau}_\emptyset \in \Cont{\emptyset_\bot}{\A}$, за някоя област на Скот $\A$.
  Но ние знаем, че $\Cont{\emptyset_\bot}{\A} \cong \A$.
  Следователно, можем да считаме, че $\val{\tau} \in \A$.
  В противен случай, трябва винаги да пишем $\val{\tau}(\bot)$ вместо $\val{\tau}$.
\end{remark}


\begin{proposition}
  \marginpar{Ясно е, че това твърдение се обобщава за произволна пермутация на индекстите $1,\dots,n$ \cite[стр. 106]{types-programming-languages}.}
  Нека имаме следните типови контексти:
  \begin{align*}
    &\Gamma = \vv{x}_1:\vv{a}_1, \dots, \vv{x}_i:\vv{a}_i, \dots, \vv{x}_j:\vv{a}_j, \dots, \vv{x}_n:\vv{a}_n;\\
    &\Delta = \vv{x}_1:\vv{a}_1, \dots, \vv{x}_j:\vv{a}_j, \dots, \vv{x}_i:\vv{a}_i, \dots, \vv{x}_n:\vv{a}_n,
  \end{align*}
  т.е. $\Delta$ се получава от $\Gamma$ като разменим местата на $i$-тата и $j$-тата двойка.
  Тогава за всеки терм $\tau$, такъв че $\Gamma \vdash \tau : \vv{a}$, е изпълено, че $\Delta \vdash \tau : \vv{a}$ и за всеки $(u_1,\dots,u_n) \in \val{\Gamma}$,
  \[\val{\tau}_\Gamma(u_1,\dots,u_i,\dots,u_j,\dots,u_n) = \val{\tau}_\Delta(u_1,\dots,u_j,\dots,u_i,\dots,u_n).\]
\end{proposition}
\begin{hint}
  Индукция по построението на терма $\tau$.
\end{hint}


%%% Local Variables:
%%% mode: latex
%%% TeX-master: "../sep"
%%% End:

\newpage
\begin{example}
  Нека $\vv{a} = \vv{nat} \to (\vv{nat} \to \vv{nat})$ и 
  \[\tau \equiv \fix\vv{(}\underbrace{\lamb{f}{a}{\overbrace{\lamb{x}{nat}{\lamb{y}{nat}{\ifelse{\vv{y == 0}}{\vv{0}}{\vv{x + (f x (y-1))} }}}}^{\tau'_0}}}_{\tau_0}\vv{)}.\]

  Знаем, че $\val{\tau} \in \Cont{\Nat_\bot}{\Cont{\Nat_\bot}{\Nat_\bot}}$, където
  \[\val{\tau} \df \lfp(\val{\tau_0}).\]

  Ясно е, че $\val{\tau_0} \in \Cont{\val{\vv{a}}}{\val{\vv{a}}}$ и
  $\val{\tau_0} = \curry(\val{\tau'_0}_{\vv{f:a}}) = \val{\tau'_0}_{\vv{f:a}}$ и
  сега пък ако положим
  \begin{align*}
    % & \tau_0 \equiv \lamb{f}{a}{\lamb{x}{nat}{\lamb{y}{nat}{\ifelse{\vv{y == 0}}{\vv{0}}{\vv{x + (f x (y-1))} }}}},\\
    % & \tau'_0 \equiv \lamb{x}{nat}{\lamb{y}{nat}{\ifelse{\vv{y == 0}}{\vv{0}}{\vv{x + (f x (y-1))}}}},\\
    & \tau''_0 \equiv \lamb{y}{nat}{\ifelse{\vv{y == 0}}{\vv{0}}{\vv{x + (f x (y-1))}}},\\
    & \tau'''_0 \equiv \ifelse{\vv{y == 0}}{\vv{0}}{\vv{x + (f x (y-1))}}.
  \end{align*}

  то ще получим, че
  \[\val{\tau'_0}_{\vv{f:a}} = \curry(\val{\tau''_0}_{\vv{f:a,x:nat}}),\]
  и тогава
  \[\val{\tau'_0}_{\vv{f:a}}(\varphi)(m) = \val{\tau''_0}_{\vv{f:a,x:nat}}(\varphi,m).\]
  Сега вече получаваме, че
  \[\val{\tau''_0}_{\vv{f:a,x:nat}} = \curry(\val{\tau'''_0}_{\vv{f:a,x:nat,y:nat}}),\]
  т.е.
  \[\val{\tau''_0}_{\vv{f:a,x:nat}}(\varphi,m)(n) = \val{\tau'''_0}_{\vv{f:a,x:nat,y:nat}}(\varphi,m,n).\]

  Обединявайки всичко получаваме, че:
  \begin{align*}
    \val{\tau_0}(\varphi)(m)(n) & = \val{\tau'_0}_{\vv{f:a}}(\varphi)(m)(n) \\
                                & = \val{\tau''_0}_{\vv{f:a,x:nat}}(\varphi,m)(n)\\
                                & = \val{\tau'''_0}_{\vv{f:a,x:nat,y:nat}}(\varphi,m,n)\\
                                & = \val{\ifelse{\vv{y==0}}{\vv{0}}{\vv{x + (f x (y-1))}}}(\varphi,m,n)\\
                                & = \texttt{if}(\val{\vv{y==0}}(\varphi,m,n), \val{\vv{0}}(\varphi,m,n),\val{\vv{x + (f x (y-1))}}(\varphi,m,n)).
                                % & = \texttt{if}(\eq(n,0),0,\plus(m, \texttt{eval}(\texttt{eval}(\varphi, m),n-1))).
  \end{align*}

  Накрая получаваме, че
  \[\val{\tau_0}(\varphi)(m)(n) = \begin{cases}
      0, & \text{ако }n = 0\\
      \plus(m, \varphi(m)(n-1)), & \text{ако } n > 0\\
      \bot, & \text{ако }n = \bot.
    \end{cases}
  \]
  
                                
  Сега вече знаем как по теоремата на Клини да докажем, че
  \[\lfp(\val{\tau_0})(m)(n) =
    \begin{cases}
      m*n,  & \text{ако }m,n\in\Nat\\
      \bot, & \text{иначе}
    \end{cases}
\]
                                
  
\end{example}

\begin{framed}
\begin{lemma}[Лема за замяната]\label{lem:pcf:substitution}
  Нека $\Gamma$ е типов контекст, $\tau$ и $\rho$ са термове, $\vv{x} \not\in \texttt{dom}(\Gamma)$,
  \begin{align*}
    & \Gamma \vdash \rho : \vv{a}\\
    & \Gamma, \type{x}{a} \vdash \tau : \vv{b}.
  \end{align*}
  Тогава
  \begin{enumerate}[1)]
  \item
    $\Gamma \vdash \tau\subst{x}{\rho} : \vv{b}$;
  \item
    за всяко $\overline{u} \in \val{\Gamma}$,
    \[\val{\tau\subst{x}{\rho}}_\Gamma(\overline{u}) = \val{\tau}_{\Gamma'}(\overline{u},\val{\rho}_\Gamma(\overline{u})),\]
    където $\Gamma' = \Gamma, \type{x}{a}$.  
  \end{enumerate}
\end{lemma}
\end{framed}
\marginpar{Защо да не взема $\rho$ да бъде затворен терм ?}
\begin{proof}
  Индукция по построението на термовете.
  \begin{itemize}
  \item
    Нека $\tau \equiv \vv{x}_i$, където $\vv{x}_i \not\equiv \vv{x}$.
  \item
    Нека $\tau \equiv \vv{x}$.
  \item
    Нека $\tau \equiv \vv{n}$.
  \item
    Нека $\tau \equiv \ifelse{\tau_1}{\tau_2}{\tau_3}$.
  \item
    Нека $\tau \equiv \tau_1 + \tau_2$.
  \item
    Нека $\tau \equiv \tau_1\ \vv{==}\ \tau_2$.
  \item
    Нека $\tau \equiv \tau_1 \tau_2$.
    Тук първата част е лесна. Понеже имаме, че
    \begin{prooftree}
      \AxiomC{$\Gamma, \type{x}{a} \vdash \tau_1: \vv{c} \to \vv{b}$}
      \AxiomC{$\Gamma, \type{x}{a} \vdash \tau_2: \vv{c}$}
      \BinaryInfC{$\Gamma, \type{x}{a} \vdash \tau_1 \tau_2 : \vv{b}$}
    \end{prooftree}
    то можем да приложим И.П. за да получим, че
    \begin{prooftree}
      \AxiomC{$\Gamma \vdash \rho : \vv{a}$}
      \AxiomC{$\Gamma, \type{x}{a} \vdash \tau_1: \vv{c} \to \vv{b}$}
      \LeftLabel{\scriptsize{(И.П.)}}
      \BinaryInfC{$\Gamma \vdash \tau_1\subst{\vv{x}}{\rho} : \vv{b}$}
      \AxiomC{$\Gamma \vdash \rho : \vv{a}$}
      \AxiomC{$\Gamma, \type{x}{a} \vdash \tau_2: \vv{c} \to \vv{b}$}
      \RightLabel{\scriptsize{(И.П.)}}
      \BinaryInfC{$\Gamma \vdash \tau_2\subst{x}{\rho} : \vv{c}$}
      \RightLabel{\scriptsize{(app)}}
      \BinaryInfC{$\Gamma \vdash \tau_1\subst{x}{\rho}(\tau_2\subst{x}{\rho}) : \vv{c}$}
      \RightLabel{\scriptsize{(правила на замяна)}}
      \UnaryInfC{$\Gamma \vdash \tau\subst{x}{\rho} : \vv{c}$}
    \end{prooftree}
    Втората част също е лесна.
    \begin{align*}
      \val{\tau_1\tau_2}_{\Gamma'}(\ov{u},\val{\rho}_\Gamma(\ov{u})) & \df \texttt{eval}(\val{\tau_1}_{\Gamma'}(\ov{u},\val{\rho}_\Gamma(\ov{u})), \val{\tau_2}_{\Gamma'}(\ov{u},\val{\rho}_\Gamma(\ov{u}))) & \comment\text{\Def{eval}}\\
                                                                   & = \texttt{eval}(\val{\tau_1\subst{x}{\rho}}_\Gamma(\ov{u}), \val{\tau_2\subst{x}{\rho}}_\Gamma(\ov{u})) & \comment\text{И.П.}\\
                                                                   & = \val{\tau_1\subst{x}{\rho}(\tau_2\subst{x}{\rho})}_\Gamma(\ov{u})\\
                                                                   & = \val{\tau\subst{x}{\rho}}_\Gamma(\ov{u})
    \end{align*}
  \item
    Нека $\tau \equiv \fix(\tau')$.
    Първо трябва да докажем, че $\Gamma \vdash \tau[\vv{x}/\rho] : \vv{b}$.
    От правилата за типизиране е ясно, че
    \begin{prooftree}
      \AxiomC{$\Gamma, \type{x}{a} \vdash \tau':\vv{b}\to\vv{b}$}
      \RightLabel{\scriptsize{(fix)}}
      \UnaryInfC{$\Gamma, \type{x}{a} \vdash \fix(\tau') : \vv{b}$}
    \end{prooftree}
    Сега можем да приложим И.П. за терма $\tau'$. Получаваме, че
    \begin{prooftree}
      \AxiomC{$\Gamma \vdash \rho: \vv{a}$}
      \AxiomC{$\Gamma, \type{x}{a} \vdash \tau':\vv{b}\to\vv{b}$}
      \RightLabel{\scriptsize{(И.П.)}}
      \BinaryInfC{$\Gamma \vdash \tau'\subst{x}{\rho} : \vv{b} \to \vv{b}$}
      \RightLabel{\scriptsize{(fix)}}
      \UnaryInfC{$\Gamma \vdash \fix(\tau'\subst{x}{\rho}) : \vv{b}$}
    \end{prooftree}
    Понеже $\fix(\tau'\subst{x}{\rho}) \equiv \fix(\tau')\subst{x}{\rho}$, то заключаваме, че
    \[\Gamma \vdash \tau[\vv{x}/\rho] : \vv{b}.\]
    
    Сега трябва да проверим защо $\val{\fix(\tau\subst{x}{\rho})}_{\Gamma}(\ov{u}) = \val{\fix(\tau)}_{\Gamma'})(\ov{u},\val{\rho}_\Gamma)$.
    Получаваме следното:
    \begin{align*}
      \val{\fix(\tau\subst{x}{\rho})}_{\Gamma}(\ov{u}) & = \lfp(\val{\tau\subst{x}{\rho}}_\Gamma(\ov{u})) & \comment\text{от деф.}\\
                                                       & = \lfp(\val{\tau}_{\Gamma'}(\ov{u},\val{\rho}_\Gamma(\ov{u}))) & \comment\text{от И.П. за }\tau\\
                                                       & = \val{\fix(\tau)}_{\Gamma'}(\ov{u},\val{\rho}_\Gamma(\ov{u})).
    \end{align*}
    
    
  \item
    \marginpar{Тук е важно, че $\val{\Delta} = \val{\Gamma} \times \val{\vv{a}_n}$}

    Нека $\tau \equiv \lamb{y}{c}{\tau'}$, където $\vv{y} \not\in \vv{dom}(\Gamma) \cup \{\vv{x}\}$.
    Първо трябва да докажем, че $\Gamma \vdash \tau[\vv{x}/\rho] : \vv{b}$.
    
    От правилата за типизиране е ясно, че щом $\Gamma, \type{x}{a} \vdash \tau : \vv{b}$, то
    $\vv{b} = \vv{c} \to \vv{d}$ за някой тип $\vv{d}$ и
    \begin{prooftree}
      \AxiomC{$\vv{y} \not\in\vv{dom}(\Gamma)\cup\{\vv{x}\}$}
      \AxiomC{$\Gamma, \type{x}{a}, \type{y}{c} \vdash \tau':\vv{d}$}
      \RightLabel{\scriptsize{(lambda)}}
      \BinaryInfC{$\Gamma, \type{x}{a} \vdash \lamb{y}{c}{\tau'} : \vv{c}\to\vv{d}$}
    \end{prooftree}
    Това означава, че можем да използваме И.П. за терма $\tau'$ и така получаваме, че
    \begin{prooftree}
      \AxiomC{$\vv{y} \not\in \vv{dom}(\Gamma)$}
      \AxiomC{$\Gamma \vdash \rho : \vv{a}$}
      \UnaryInfC{$\Gamma,\type{y}{c} \vdash \rho : \vv{a}$}
      \AxiomC{$\Gamma, \type{y}{c}, \type{x}{a} \vdash \tau':\vv{d}$}
      \RightLabel{\scriptsize{(И.П.)}}
      \BinaryInfC{$\Gamma, \type{y}{c} \vdash \tau'\subst{x}{\rho} : \vv{d}$}
      \RightLabel{\scriptsize{(lambda)}}
      \BinaryInfC{$\Gamma \vdash \lamb{y}{c}{\tau'\subst{x}{\rho}}:\vv{c}\to\vv{d}$}
    \end{prooftree}
    Накрая, понеже $\tau\subst{x}{\rho} \equiv \lamb{y}{c}{\tau'\subst{x}{\rho}}$, то
    заключаваме, че
    \[\Gamma \vdash \tau\subst{x}{\rho}:\vv{b}.\]

    Нека $\Delta \df \Gamma,\type{y}{c}$.
    Понеже имаме, че $\Delta \vdash \tau'\subst{x}{\rho} : \vv{d}$,
    то можем да приложим И.П. за $\tau'$ и така получаваме, че за всяко $\overline{u},v \in \val{\Delta}$,
    \begin{align*}
      \curry(\val{\tau'\subst{x}{\rho}}_\Delta)(\ov{u})(v) & \df \val{\tau'\subst{x}{\rho}}_\Delta(\overline{u},v) & \comment\text{\Def{curry}}\\
                                                                 & \stackrel{\text{И.П.}}{=} \val{\tau'}_{\Delta'}(\overline{u},v,\val{\rho}_\Delta(\ov{u},v)) & \comment \Delta' \df \Gamma, \type{y}{c}, \type{x}{a}\\
                                                                 & = \val{\tau'}_{\Delta'}(\ov{u},v,\val{\rho}_\Gamma(\ov{u})) & \comment \fv(\rho) \subseteq \vv{dom}(\Gamma)\\
                                                                 & = \val{\tau'}_{\Delta''}(\ov{u},\val{\rho}_\Gamma(\ov{u}),v) & \comment \Delta'' \df \Gamma, \type{x}{a}, \type{y}{c}\\
                                                                 & = \curry(\val{\tau'}_{\Delta''})(\ov{u},\val{\rho}_\Gamma(\ov{u}))(v) \\
                                                                 & = \val{\lamb{y}{c}{\tau'}}_{\Gamma'}(\ov{u},\val{\rho}_\Gamma(\ov{u}))(v). & \comment \Gamma' \df \Gamma, \type{x}{a}
    \end{align*}    
    Така получихме, че
    \begin{align*}
      \val{\lamb{y}{c}{\tau'\subst{x}{\rho}}}_\Gamma(\ov{u}) & \df \curry(\val{\tau'\subst{x}{\rho}}_\Delta)(\ov{u})\\
                                                                   & = \val{\lamb{y}{c}{\tau'}}_{\Gamma'}(\ov{u},\val{\rho}_\Gamma(\ov{u})).
    \end{align*}
    
  \end{itemize}
\end{proof}



%%% Local Variables:
%%% mode: latex
%%% TeX-master: "../sep"
%%% End:

\newpage
\section{Коректност}

\marginpar{Да напомним, че когато термът $\tau$ е затворен, то ще пишем $\val{\tau}$ вместо $\val{\tau}_\emptyset(\bot)$.}
Понеже вече имаме дефинирани операционна и денотационна семантика на термовете,
следващата стъпка е да разгледаме каква е връзката между тях.
В този раздел ще докажем едната (по-лесната) посока.
\marginpar{На англ. {\em soundness}.}
\begin{framed}
  \begin{theorem}[Теорема за коректност]\label{th:pcf:soundness}
    За всеки затворен терм $\tau : \vv{b}$ и стойност $\type{v}{b}$, е изпълнена импликацията:
    \[\tau \opsem{}{b} \vv{v}\ \implies\ \val{\tau} = \val{\vv{v}} \in \val{\vv{b}}.\]
  \end{theorem}  
\end{framed}
\begin{proof}
  Индукция по дължината $\ell$ на извода $\opsem{\ell}{b}$ за всеки тип $\vv{b}$.
  Нека $\ell = 0$. Имаме два случая, защото имаме два вида стойности.
  \begin{itemize}
  \item
    Нека $\tau \equiv \vv{n}$.
    Ясно е, че $\vv{b} = \vv{nat}$ и от правилата на операционната семантика имаме, че:
    \begin{prooftree}
      \AxiomC{}
      \RightLabel{\scriptsize{(val)}}
      \UnaryInfC{$\tau \opsem{0}{nat} \vv{n}$}
    \end{prooftree}
    От дефиницията на семантика на терм, директно получаваме, че
    $\val{\tau} = n = \val{\vv{n}}$.    
  \item
    Нека $\tau \equiv \lamb{x}{c}{\tau'}$. Тогава $\vv{b} = \vv{a}\to\vv{c}$ и от правилата на операционната семантика имаме, че:
    \begin{prooftree}
      \AxiomC{}
      \RightLabel{\scriptsize{(val)}}
      \UnaryInfC{$\tau \opsemGen{0}{\vv{a}\to\vv{c}} \tau$}
    \end{prooftree}
    Ясно е, че $\val{\tau} = \val{\tau} \in \val{b}$.
  \end{itemize}
  Така доказахме, че
  \[\tau \opsem{0}{b} \vv{v}\ \implies\ \val{\tau} = \val{\vv{v}} \in \val{\vv{b}}.\]
  Нека сега $\ell > 0$ и да приемем, че имаме следното индукционно предположение:
  \[\tau \opsem{<\ell}{b} \vv{v}\ \implies\ \val{\tau} = \val{\vv{v}} \in \val{\vv{b}}.\]
  Ще докажем, че
  \[\tau \opsem{\ell}{b} \vv{v}\ \implies\ \val{\tau} = \val{\vv{v}} \in \val{\vv{b}}.\]
  \begin{itemize}
  \item
    Нека $\tau \equiv \tau_1 + \tau_2$. Тогава от правилата на операционната семантика имаме, че:
    \begin{prooftree}
      \AxiomC{$\tau_1 \opsem{\ell_1}{nat} \vv{n}_1$}
      \AxiomC{$\tau_2 \opsem{\ell_2}{nat} \vv{n}_2$}
      \LeftLabel{\scriptsize{($\ell=\ell_1+\ell_2+1$)}}
      \RightLabel{\scriptsize{(plus)}}
      \BinaryInfC{$\tau_1 + \tau_2 \opsem{\ell_1+\ell_2+1}{nat} \vv{n},$}
    \end{prooftree}
    където $n = n_1 + n_2$. От \IndHyp получаваме, че
    \begin{align*}
      & \val{\tau_1} = \val{\vv{n}_1} = n_1\\
      & \val{\tau_2} = \val{\vv{n}_2} = n_2.
    \end{align*}
    Тогава
    \begin{align*}
      \val{\tau_1 + \tau_2} & = \plus(\val{\tau_1}, \val{\tau_2}) & \comment\text{от деф.}\\
                            & = n_1 + n_2 & \comment\text{\IndHyp}\\
                            & = n.
    \end{align*}
  \item
    Случаите $\tau \equiv \tau_1 - \tau_2$ и $\tau \equiv \tau_1\ \vv{==}\ \tau_2$ са аналогични. Оставяме ги на читателя.
  \item
    Нека $\tau \equiv \ifelse{\tau_1}{\tau_2}{\tau_3}$. Тогава от правилата на операционната семантика имаме, че:
    \begin{prooftree}
      \AxiomC{$\tau_1 \opsem{\ell_1}{nat} \vv{n}_1$}
      \AxiomC{$\tau_2 \opsem{\ell_2}{a} \vv{v}_2$}
      \AxiomC{$\vv{n}_1 \not\equiv \vv{0}$}
      \LeftLabel{\scriptsize{($\ell=\ell_1+\ell_2+1$)}}
      \RightLabel{\scriptsize{(if$^+$)}}
      \TrinaryInfC{$\ifelse{\tau_1}{\tau_2}{\tau_3} \opsem{\ell_1+\ell_2+1}{a} \vv{v}_2,$}
    \end{prooftree}
    Тогава от \IndHyp получаваме, че:
    \begin{align*}
      & \val{\tau_1} = n_1\\
      & \val{\tau_2} = \val{\vv{v}_2}.
    \end{align*}
    Тогава
    \begin{align*}
      \val{\ifelse{\tau_1}{\tau_2}{\tau_3}} & = \texttt{if}(\val{\tau_1}, \val{\tau_2}, \val{\tau_3})\\
                                            & = \texttt{if}(n_1,\val{\tau_2}, \val{\tau_3}) & \comment\text{от \IndHyp}\\
                                            & = \val{\tau_2} & \comment\text{от деф. на }\texttt{if}\\
                                            & = \val{\vv{v}_2}. & \comment\text{от \IndHyp}
    \end{align*}
    
    Случаят, когато $\vv{n}_1 \equiv \vv{0}$ е аналогичен.
  \item
    Нека $\tau \equiv \tau_1 \tau_2$. Тогава от правилата на операционната семантика имаме, че:
    \begin{prooftree}
      \AxiomC{$\tau_1 \opsemGen{\ell_1}{\vv{a}\to\vv{b}} \lamb{x}{a}{\tau'_1}$}
      \AxiomC{$\tau'_1[x/\tau_2] \opsem{\ell_2}{b} \vv{v}$}
      \LeftLabel{\scriptsize{($\ell=\ell_1+\ell_2+1$)}}
      \RightLabel{\scriptsize{(cbn)}}
      \BinaryInfC{$\tau_1 \tau_2 \opsem{\ell_1+\ell_2+1}{b} \vv{v} $}
    \end{prooftree}
    Тогава от \IndHyp получаваме, че:    
    \begin{align*}
      & \val{\tau_1} = \val{\lamb{x}{a}{\tau'_1}} \in \Cont{\val{\vv{a}}}{\val{\vv{b}}}\\
      & \val{\tau'_1\subst{x}{\tau_2}} = \val{\vv{v}} \in \val{\vv{b}}.
    \end{align*}
    Тогава
    \begin{align*}
      \val{\tau_1\tau_2} & = \texttt{eval}(\val{\tau_1},\val{\tau_2}) & \comment\text{от деф.}\\ 
                         & = \val{\tau_1}(\val{\tau_2}) & \comment \val{\tau_1} \in \Cont{\val{\vv{a}}}{\val{\vv{b}}}\\
                         & = \val{\lamb{x}{a}{\tau'_1}}(\val{\tau_2}) & \comment\text{\IndHyp}\\
                         & = \val{\tau'_1}_{\type{x}{a}}(\val{\tau_2})\\
                         & = \val{\tau'_1\subst{x}{\tau_2}} & \comment\text{от \hyperref[lem:pcf:substitution]{Лема за замяната}}\\
                         & = \val{\vv{v}} & \comment\text{\IndHyp}
    \end{align*}
  \item
    Нека $\tau \equiv \fix(\tau')$. Тогава от правилата на операционната семантика имаме, че:
    \begin{prooftree}
      \AxiomC{$\tau'\ \fix(\tau') \opsem{\ell-1}{a} \vv{v}$}
      \RightLabel{\scriptsize{(fix)}}
      \UnaryInfC{$\fix(\tau') \opsem{\ell}{a} \vv{v} $}
    \end{prooftree}
    Тогава от \IndHyp имаме, че:
    \[\val{\tau'\ \fix(\tau')} = \val{\vv{v}}.\]
    Тогава
    \begin{align*}
      \val{\tau} & = \lfp(\val{\tau'})\\
                 & = \val{\tau'}(\lfp(\val{\tau'}))\\
                 & = \val{\tau'}(\val{\fix(\tau')})\\
                 & = \val{\tau'\fix(\tau')}\\
                 & = \val{\vv{v}}. & \comment\text{\IndHyp}
    \end{align*}
  \end{itemize}
\end{proof}

\hyperref[th:pcf:soundness]{Теоремата за коректност}\ частично потвърждава нашата интуиция, че за типа $\vv{nat}$
можем да си мислим за $\bot^{\val{\vv{nat}}}$ като за изчисление, което никога не завършва.

\marginpar{Другата посока ще я получим след малко.}

\begin{corollary}
  Нека $\tau$ е затворен терм от тип $\vv{nat}$. Тогава
  \[\val{\tau} = \bot^{\val{\vv{nat}}}\ \implies\ \tau \not\opsem{}{nat}.\]
\end{corollary}

За жалост, тази наша интуиция се ,,губи'', когато се интересуваме от термове от по-висок от $\nat$ тип.
\marginpar{За дискусия по този въпрос вижте \cite[стр. 213]{models-of-computation}.}

Да разгледаме един пример. Нека 
\[\tau \equiv \lamb{y}{nat}{\fix(\lamb{x}{nat}{\vv{x}})}.\]
Лесно се съобразява, че $\tau : \nat\to\nat$.
За произволен елемент $a \in \Nat_\bot$ е изпъленено следното:
\begin{align*}
  \val{\tau}(a) & = \val{\lamb{y}{nat}{\fix(\lamb{x}{nat}{\vv{x}})}}(a)\\
                & = \curry(\val{\fix(\lamb{x}{nat}{\vv{x}})}_{\type{y}{nat}})(a)\\
                & = \val{\fix(\lamb{x}{nat}{\vv{x}})}_{\type{y}{nat}}(a)\\
                & = \lfp(\val{\lamb{x}{nat}{\vv{x}}}_{\type{y}{nat}}(a))\\
                & = \lfp(\underbrace{\curry(\val{x}_{\type{y}{nat},\type{x}{nat}})(a)}_{\texttt{id}\text{ за }\Nat_\bot})\\
                & = \bot^{\val{\nat}}
\end{align*}
С други думи, получаваме, че
\[\val{\tau} = \bot^{\val{\nat\to\nat}}.\]
От друга страна, обаче, $\tau$ представлява стойност. Следователно,
\[\tau \opsemGen{0}{\nat\to\nat} \tau.\]


%%% Local Variables:
%%% mode: latex
%%% TeX-master: "../sep"
%%% End:

\newpage
\section{Адекватност}
\marginpar{Adequacy ???}
Нашата цел в този раздел е да докажем следната теорема.
\begin{framed}
  \begin{theorem}[Теорема за адекватност]
    За всеки затворен терм $\tau : \vv{nat}$ е изпълнена импликацията
    \[\val{\tau} = n \neq \bot^{\val{\nat}} \implies \tau \Downarrow_{\vv{nat}} \vv{n}.\]
  \end{theorem}
\end{framed}
\marginpar{Тук $n$ е число, а $\vv{n}$ е константа.}

Оказва се, че доказателството на тази теорема не е леко.
Ще започнем като дефинираме за всеки тип $\vv{a}$ релацията 
$\triangleleft_{\vv{a}} \subseteq \val{\vv{a}} \times \vv{PCF}_{\vv{a}}$
с индукция по построението на типовете.

\begin{itemize}
\item
  \marginpar{Съобразете, че теоремата за адекватност на практика гласи, че $\val{\tau} \triangleleft_{\vv{nat}} \tau$.}
  Нека $\vv{a} = \vv{nat}$. Тогава 
  \marginpar{Обикновено $\triangleleft_{\vv{a}}$ се нарича \emph{логическа релация}.
    В \cite[стр. 210]{models-of-computation} е обяснено защо имаме нужда от тези релации за да докажем теоремата за адекватност. В \cite[стр. 134]{gunter} е представен синтактичен подход към решаването на този проблем.}
  \[n \triangleleft_{\vv{nat}} \tau \dff ( n\neq\bot^{\val{\vv{nat}}} \implies \tau \Downarrow_{\vv{nat}} \vv{n}).\]
\item
  Нека $\vv{a} = \vv{b} \to \vv{c}$. Тогава 
  \[f \triangleleft_{\vv{b}\to\vv{c}} \tau \dff (\forall e\in \val{\vv{b}})(\forall \mu \in \vv{PCF}_{\vv{b}})[\ e \triangleleft_{\vv{b}} \mu \implies f(e) \triangleleft_{\vv{c}} \tau(\mu)\ ].\]
\item
  Нека $\Gamma = \vv{x}_1:\vv{a}_1, \dots, \vv{x}_n:\vv{a}_n$. Тогава 
  \[(u_1,\dots,u_n) \triangleleft_\Gamma (\tau_1,\dots,\tau_n) \dff u_1 \triangleleft_{\vv{a}_1} \tau_1\ \&\ \cdots\ \&\ u_n \triangleleft_{\vv{a}_n} \tau_n.\]
\end{itemize}

\begin{example}
  Да проверим внимателно защо е изпълнено, че:
  \[\texttt{id}_{\val{\nat}} \triangleleft_{\vv{nat}\to\vv{nat}} \lamb{x}{nat}{\vv{x + 0}}.\]
  Според дефиницията трябва да проверим импликацията
  \[e \triangleleft_{\nat} \mu \implies \texttt{id}_{\val{\nat}}(e) \triangleleft_{\nat} \tau(\mu),\]
  за произволен елемент $e \in \Nat_\bot$ и произволен затворен терм $\mu : \nat$.
  \marginpar{Аналогично можем да видим, че $\texttt{id}_{\val{\nat}} \triangleleft_{\nat} \lamb{x}{nat}{x}$.}
  \begin{itemize}
  \item
    Ако $e = \bot$, то от дефиницията на $\triangleleft_{\nat}$ ведната следва, че за произволен затворен терм $\mu : \nat$, то
    $\bot \triangleleft_{\nat} \mu$. Понеже $\texttt{id}_{\val{\nat}}(\bot) = \bot$, то отново от дефиницията веднага следва, че
    $\bot \triangleleft_{\nat} \tau(\mu)$, за произволен затворен терм $\mu : \nat$.
  \item
    Нека $e = n\in\Nat$ и да разгледаме затворен терм $\mu : \nat$, за който $n \triangleleft_{\nat} \mu$.
    Според дефиницията на $\triangleleft_{\nat}$, това означава, че $\mu \opsem{}{nat} \vv{n}$.
    Сега да видим защо $\texttt{id}_{\val{\nat}}(n) = n \triangleleft_{\nat} \tau(\mu)$ или с други думи,
    трябва да проверим, че $\tau(\mu) \opsem{}{nat} \vv{n}$. Тук се позоваваме на правилата от операционната семантика:
    \begin{prooftree}
      \AxiomC{$\tau$ е стойност}
      \LeftLabel{\scriptsize{(val)}}
      \UnaryInfC{$\tau \opsemGen{}{\nat\to\nat} \lamb{x}{nat}{\vv{x + 0}}$}
      \AxiomC{$n \triangleleft_{\nat} \mu$}
      \UnaryInfC{$\mu \opsem{}{nat} \vv{n}$}
      \AxiomC{$\vv{0}$ е стойност}
      \RightLabel{\scriptsize{(val)}}
      \UnaryInfC{$\vv{0} \opsem{}{nat} \vv{0}$}
      \RightLabel{\scriptsize{(plus)}}
      \BinaryInfC{$\vv{(x+0)}\subst{x}{\mu} \opsem{}{nat} \vv{n}$}
      \RightLabel{\scriptsize{(app)}}
      \BinaryInfC{$\tau(\mu) \opsem{}{nat} \vv{n}$}
    \end{prooftree}
  \end{itemize}

\end{example}

\begin{problem}
  Нека положим $\tau \equiv \lamb{x}{nat}{\lamb{y}{nat}{x-y}}$.
  Проверете, че $f \triangleleft_{\nat\to\nat\to\nat} \tau$, където:
  \begin{itemize}
  \item
    $f = \curry(\minus)$;
  \item
    $f(a)(b) =
    \begin{cases}
      a-b, & \text{ако }a \geq b\\
      \bot, & \text{иначе}
    \end{cases}$;
  \item
    $f(a)(b) = \bot$ за произволни $a,b\in\Nat_\bot$.
  \end{itemize}
\end{problem}


Нека първо да разгледаме някои основни свойства на релацията $\triangleleft_{\vv{a}}$.
Тук доказателствата протичат с индукция по построението на типовете.
\marginpar{\cite[стр. 197]{gunter}}

\begin{proposition}\label{pr:pcf:adequacy:bottom}
  За всеки тип $\vv{a}$ и всеки затворен терм $\tau : \vv{a}$ е изпълнено, че $\bot^{\val{\vv{a}}} \triangleleft_{\vv{a}} \tau$.
\end{proposition}
\begin{proof}
  Индукция по построението на типовете $\vv{a}$.
  Първо, нека $\vv{a} = \vv{nat}$. По тривиални съображения имаме, че за произволен терм $\tau:\vv{a}$ е изпълнено, че $\bot^{\val{\vv{nat}}} \triangleleft_{\vv{nat}} \tau$.
  
  Второ, нека $\vv{a} = \vv{b} \to \vv{c}$ и да фиксираме произволен терм $\tau : \vv{b} \to \vv{c}$.
  Тук имаме, че $\bot^{\val{\vv{a}}} \in \Cont{\val{\vv{b}}}{\val{\vv{c}}}$ е изображение,
  за което $\bot^{\val{\vv{a}}}(e) =  \bot^{\val{\vv{c}}}$ за всеки елемент $e \in \val{\vv{b}}$.
  Нека $e \triangleleft_{\vv{b}} \mu$, където $\mu : \vv{b}$.
  Щом $\tau : \vv{b}\to\vv{c}$, от правилата за типизиране е ясно, че $\tau(\mu) : \vv{c}$.
  Сега от \IndHyp за типа $\vv{c}$ е ясно, че $\bot^{\val{\vv{a}}}(e) = \bot^{\val{\vv{c}}} \triangleleft_{\vv{c}} \tau(\mu)$.
\end{proof}


\begin{proposition}\label{pr:pcf:adequacy:chain}
  Нека за произволен тип $\vv{a}$ и произволен терм $\tau:\vv{a}$ да разгледаме множеството $D \df \{d \in \val{\vv{a}} \mid d \triangleleft_{\vv{a}} \tau\}$.
  Тогава ако $\chain{d}{i}$ е верига от елементи на $D$, то $\bigsqcup_i d_i$ също принадлежи на $D$.
\end{proposition}
\begin{proof}
  Индукция по построението на типовете $\vv{a}$.
  Първо, нека $\vv{a} = \vv{nat}$.
  \marginpar{Да напомним, че $\val{\vv{nat}} = \Nat_\bot$. Ясно е, че всяка верига от елементи на $\Nat_\bot$ се стабилизира.}
  Нека $\chain{d}{i}$ е верига от елементи на $\Nat_\bot$ и за всеки индекс $i$, $d_i \triangleleft_{\vv{nat}} \tau$.
  Ако за всяко $i$, $d_i = \bot$, то $\bigsqcup_i d_i = \bot^{\val{\vv{nat}}}$ и следователно $\bigsqcup_i d_i
  \triangleleft_{\vv{nat}} \tau$.
  Ако съществува индекс $i_0$, за който $d_{i_0} = n \neq \bot$, то е ясно, че за всяко $i > i_0$, $d_i = n$.
  Оттук следва, че $\bigsqcup_i d_i = n = d_{i_0}$.
  Понеже $d_{i_0} \triangleleft_{\vv{nat}} \tau$, то директно следва, че $\bigsqcup_i d_i \triangleleft_{\vv{nat}} \tau$.

  Второ, нека $\vv{a} = \vv{b} \to \vv{c}$ и да фиксираме произволен терм $\tau : \vv{b} \to \vv{c}$.
  Нека $\chain{f}{i}$ е верига от елементи на $\Cont{\val{\vv{b}}}{\val{\vv{c}}}$,
  за които е изпълнено, че $f_i \triangleleft_{\vv{a}} \tau$. Трябва да докажем, че $\bigsqcup_i f_i \triangleleft_{\vv{a}} \tau$,
  т.е. за произволен елемент $e \in \val{\vv{b}}$ и произволен затворен терм $\mu : \vv{b}$, за който $e \triangleleft_{\vv{b}} \mu$, то
  $(\bigsqcup_if)(e) \triangleleft_{\vv{c}} \tau(\mu)$.
  Но ние знаем от \Lem{double-chain:lub}, че $(\bigsqcup_if)(e) = \bigsqcup_i\{f_i(e)\}$.
  Щом $f_i \triangleleft_{\vv{b}\to\vv{c}} \tau$, то за разглежданите $e$ и $\mu$ имаме, че $f_i(e) \triangleleft_{\vv{c}} \tau(\mu)$.
  Ние знаем, че ${(f_i(e))}^\infty_{i=0}$ е верига и от \IndHyp за типа $\vv{c}$ следва, че $\bigsqcup_i\{f_i(e)\} \triangleleft_{\vv{c}} \tau(\mu)$.
\end{proof}


\begin{proposition}\label{pr:pcf:adequacy:implication}
  Да разгледаме произволен тип $\vv{a}$ и произволен затворен терм $\tau : \vv{a}$.
  \marginpar{Да напомним, че с $\vv{v}$ означаваме термове стойности.}
  Тогава е изпълнено следното:
  \begin{prooftree}
    \AxiomC{$d \triangleleft_{\vv{a}} \tau$}
    \AxiomC{$(\forall \vv{v})[\tau \Downarrow_{\vv{a}} \vv{v} \implies \rho \Downarrow_{\vv{a}} \vv{v}]$}
    \BinaryInfC{$d \triangleleft_{\vv{a}} \rho$}
  \end{prooftree}
\end{proposition}
\begin{proof}
  Индукция по построението на типовете $\vv{a}$.
  Първо, нека $\vv{a} = \vv{nat}$. 
  Нека $d \triangleleft_{\vv{nat}} \tau$ и $(\forall \vv{v})[\tau \Downarrow_{\vv{nat}} \vv{v} \implies \rho \Downarrow_{\vv{nat}} \vv{v}]$.
  Понеже $\sqsubseteq$ е плоската наредба в $\Nat_\bot$, то имаме два случая.
  Ако $d = \bot^{\val{\vv{nat}}}$, то е ясно от \Prop{pcf:adequacy:bottom}, че $d \triangleleft_{\vv{nat}} \rho$.
  Нека сега $d \neq \bot^{\val{\vv{nat}}}$. 
  Понеже $d \triangleleft_{\vv{nat}} \tau$, то $\tau \Downarrow_{\vv{nat}} \vv{d}$.
  Оттук следва, че $\rho \Downarrow_{\vv{nat}} \vv{d}$. Заключаваме, че $d \triangleleft_{\vv{nat}} \rho$.

  Второ, нека $\vv{a} = \vv{b} \to \vv{c}$ и да фиксираме произволен терм $\tau : \vv{b} \to \vv{c}$.
  Нека
  \begin{align}
    & f \triangleleft_{\vv{b}\to\vv{c}} \tau \label{eq:pcf:adequacy:implication:f-tau}\\
    & (\forall \vv{v})[\tau \Downarrow_{\vv{b}\to\vv{c}} \vv{v} \implies \rho \Downarrow_{\vv{b}\to\vv{c}} \vv{v}] \label{eq:pcf:adequacy:implication:value}
  \end{align}
  % $f \triangleleft_{\vv{b}\to\vv{c}} \tau$ и $(\forall \vv{v})[\tau \Downarrow_{\vv{b}\to\vv{c}} \vv{v} \implies \rho \Downarrow_{\vv{b}\to\vv{c}} \vv{v}]$.
  Ще докажем, че $f \triangleleft_{\vv{b} \to \vv{c}} \rho$.
  За целта, да разгледаме произволен елемент $e \in \val{\vv{b}}$ и произволен затворен терм $\mu : \vv{b}$, за който $e \triangleleft_{\vv{b}} \mu$.
  Достатъчно е да докажем, че $f(e) \triangleleft_{\vv{c}} \rho(\mu)$.
  За момента от Свойство~(\ref{eq:pcf:adequacy:implication:f-tau}) знаем само, че $f(e) \triangleleft_{\vv{c}} \tau(\mu)$.
  Понеже имаме следното правило в операционната семантика:
  \marginpar{Имаме от (\ref{eq:pcf:adequacy:implication:value}), че 
    \[\tau \Downarrow_{\vv{b}\to\vv{c}} \vv{v} \implies \rho \Downarrow_{\vv{b}\to\vv{c}} \vv{v},\] а тук $\vv{v} \equiv \lamb{x}{b}{\tau'}$.}
  \begin{prooftree}
    \AxiomC{$\tau \Downarrow_{\vv{b}\to\vv{c}} \overbrace{\lamb{x}{b}{\tau'}}^{\vv{v}}$}
    \AxiomC{$\tau'\subst{x}{\mu} \Downarrow_{\vv{c}} \vv{v}'$}
    \RightLabel{\scriptsize{(app)}}
    \BinaryInfC{$\tau(\mu) \Downarrow_{\vv{c}} \vv{v}'$}
  \end{prooftree}
  то от Свойство~(\ref{eq:pcf:adequacy:implication:value}) получаваме, че
  \begin{prooftree}
    \AxiomC{$\rho \Downarrow_{\vv{b}\to\vv{c}} \overbrace{\lamb{x}{b}{\tau'}}^{\vv{v}}$}
    \AxiomC{$\tau'\subst{x}{\mu} \Downarrow_{\vv{c}} \vv{v}'$}
    \RightLabel{\scriptsize{(app)}}
    \BinaryInfC{$\rho(\mu) \Downarrow_{\vv{c}} \vv{v}'$}
  \end{prooftree}
  Оттук следва, че
  \begin{equation}
    \label{eq:pcf:adequacy:implication:final}
    (\forall \vv{v}')[\tau(\mu) \Downarrow_{\vv{c}} \vv{v}' \implies \rho(\mu) \Downarrow_{\vv{c}} \vv{v}'].
  \end{equation}
  Сега от \IndHyp за типа $\vv{c}$ директно следва, че щом $f(e) \triangleleft_{\vv{c}}\tau(\mu)$ и Свойство (\ref{eq:pcf:adequacy:implication:final}), то \[f(e) \triangleleft_{\vv{c}} \rho(\mu).\]
\end{proof}




% \begin{lemma}\label{lem:pcf:relation}
%   Нека $\tau : \vv{a}$. Тогава:
%   \begin{enumerate}[1)]
%   \item
%     $\bot^{\val{\vv{a}}} \triangleleft_{\vv{a}} \tau$;
%   \item
%     $D = \{d \in \val{\vv{a}} \mid d \triangleleft_{\vv{a}} \tau\}$ е непрекъснато свойство в областта на Скот $\val{\vv{a}}$, т.е.
%     за всяка верига $\chain{d}{i}$ от елементи на $D$ е изпълнено, че $\bigsqcup_i d_i \in D$.
%   \item
%     \marginpar{Не ми трябва $u \sqsubseteq d$.}
%     Ако $d \triangleleft_{\vv{a}} \tau$ и $(\forall \vv{v})[\tau \Downarrow_{\vv{a}} \vv{v} \implies \rho
%     \Downarrow_{\vv{a}} \vv{v}]$, то $d \triangleleft_{\vv{a}} \rho$.
%   \end{enumerate}
% \end{lemma}
% \begin{proof}
%   Индукция по построението на типовете $\vv{a}$.
%   Първо, нека $\vv{a} = \vv{nat}$ и да фиксираме произволен терм $\tau : \vv{nat}$.
%   \marginpar{Да напомним, че $\val{\vv{nat}} \df \Nat_\bot$.}
%   \begin{enumerate}[1)]
%   \item
%     По тривиални съображения имаме, че
%     \[\bot^{\val{\vv{nat}}} \triangleleft_{\vv{nat}} \tau.\]
%   \item
%     Нека $\chain{d}{i}$ е верига от елементи на $\Nat_\bot$ и за всяко $i$, $d_i \triangleleft_{\vv{nat}} \tau$.
%     Ако за всяко $i$, $d_i = \bot$, то $\bigsqcup_i d_i = \bot^{\val{\vv{nat}}}$ и следователно $\bigsqcup_i d_i
%     \triangleleft_{\vv{nat}} \tau$.
%     Ако съществува $i_0$, за което $d_{i_0} = n \neq \bot$, то е ясно, че за всяко $i > i_0$, $d_i = n$.
%     Оттук следва, че $\bigsqcup_i d_i = n = d_{i_0}$.
%     Понеже $d_{i_0} \triangleleft_{\vv{nat}} \tau$, то директно следва, че $\bigsqcup_i d_i \triangleleft_{\vv{nat}} \tau$.
%   \item
%     Нека $d \triangleleft_{\vv{nat}} \tau$ и $(\forall \vv{v})[\tau \Downarrow_{\vv{nat}} \vv{v} \implies \rho
%     \Downarrow_{\vv{nat}} \vv{v}]$. Понеже $\sqsubseteq$ е плоската наредба в $\Nat_\bot$, то имаме два случая.
%     Ако $d = \bot^{\val{\vv{nat}}}$, то е ясно от 1), че $d \triangleleft_{\vv{nat}} \rho$.
%     Нека сега $d \neq \bot^{\val{\vv{nat}}}$. 
%     Понеже $d \triangleleft_{\vv{nat}} \tau$, то $\tau \Downarrow_{\vv{nat}} \vv{d}$.
%     Оттук следва, че $\rho \Downarrow_{\vv{nat}} \vv{d}$. Заключаваме, че $d \triangleleft_{\vv{nat}} \rho$.    
%   \end{enumerate}
  
%   Второ, нека $\vv{a} = \vv{b} \to \vv{c}$ и да фиксираме произволен терм $\tau : \vv{b} \to \vv{c}$.
%   \marginpar{Да напомним, че \[\val{\vv{b} \to \vv{c}} \df \Cont{\val{\vv{b}}}{\val{\vv{c}}}.\]}
%   \begin{enumerate}[1)]
%   \item
%     Тук имаме, че $\bot^{\val{\vv{a}}} \in \Cont{\val{\vv{b}}}{\val{\vv{c}}}$ е изображение,
%     за което $\bot^{\val{\vv{a}}}(e) =  \bot^{\val{\vv{c}}}$ за всеки елемент $e \in \val{\vv{b}}$.
%     Нека $e \triangleleft_{\vv{b}} \mu$, където $\mu : \vv{b}$.
%     От правилата за типизиране е ясно, че $\tau(\mu) : \vv{c}$.
%     Сега от И.П. е ясно, че $\bot^{\val{\vv{a}}}(e) = \bot^{\val{\vv{c}}} \triangleleft_{\vv{c}} \tau(\mu)$.
%   \item
%     Нека $\chain{f}{i}$ е верига от елементи на $\Cont{\val{\vv{b}}}{\val{\vv{c}}}$,
%     за които е изпълнено, че $f_i \triangleleft_{\vv{a}} \tau$. Трябва да докажем, че $\bigsqcup_i f_i \triangleleft_{\vv{a}} \tau$,
%     т.е. за произволни $e \in \val{\vv{b}}$ и произволни $\mu : \vv{b}$, за които $e \triangleleft_{\vv{b}} \mu$, то
%     $(\bigsqcup_if)(e) \triangleleft_{\vv{c}} \tau(\mu)$.
%     Но ние знаем, че $(\bigsqcup_if)(e) = \bigsqcup_i\{f_i(e)\}$.
%     Щом $f_i \triangleleft_{\vv{b}\to\vv{c}} \tau$, то за разглежданите $e$ и $\mu$ имаме, че $f_i(e) \triangleleft_{\vv{c}} \tau(\mu)$.
%     Ние знаем, че ${(f_i(e))}^\infty_{i=0}$ е верига и от И.П. следва, че $\bigsqcup_i\{f_i(e)\} \triangleleft_{\vv{c}} \tau(\mu)$.
%   \item
%     Нека $f \triangleleft_{\vv{b}\to\vv{c}} \tau$ и $\tau \Downarrow_{\vv{b}\to\vv{c}} \vv{v} \implies \rho
%     \Downarrow_{\vv{b}\to\vv{c}} \vv{v}$. Ще докажем, че $f \triangleleft_{\vv{b} \to \vv{c}} \rho$.
%     За целта, нека $e \in \val{\vv{b}}$, $\mu : \vv{b}$ и $e \triangleleft_{\vv{b}} \mu$.
%     Ще докажем, че $f(e) \triangleleft_{\vv{c}} \rho(\mu)$.
%     За момента знаем само, че $f(e) \triangleleft_{\vv{c}} \tau(\mu)$.
%     Понеже имаме следното правило в операционната семантика:
%     \marginpar{Имаме по условие, че 
%       \[\tau \Downarrow_{\vv{b}\to\vv{c}} \vv{v} \implies \rho \Downarrow_{\vv{b}\to\vv{c}} \vv{v},\] а тук $\vv{v} \equiv \lamb{x}{b}{\tau'}$.}
%     \begin{prooftree}
%       \AxiomC{$\tau \Downarrow_{\vv{b}\to\vv{c}} \lamb{x}{b}{\tau'}$}
%       \AxiomC{$\tau'\subst{x}{\mu} \Downarrow_{\vv{c}} \vv{v}'$}
%       \RightLabel{\scriptsize{(app)}}
%       \BinaryInfC{$\tau(\mu) \Downarrow_{\vv{c}} \vv{v}'$}
%     \end{prooftree}
%     то получаваме, че
%     \begin{prooftree}
%       \AxiomC{$\rho \Downarrow_{\vv{b}\to\vv{c}} \lamb{x}{b}{\tau'}$}
%       \AxiomC{$\tau'\subst{x}{\mu} \Downarrow_{\vv{c}} \vv{v}'$}
%       \RightLabel{\scriptsize{(app)}}
%       \BinaryInfC{$\rho(\mu) \Downarrow_{\vv{c}} \vv{v}'$}
%     \end{prooftree}
%     Оттук следва, че
%     \[(\forall \vv{v}')[\tau(\mu) \Downarrow_{\vv{c}} \vv{v}' \implies \rho(\mu) \Downarrow_{\vv{c}} \vv{v}'].\]
%     Сега от И.П. директно следва, че $f(e) \triangleleft_{\vv{c}} \rho(\mu)$.
%   \end{enumerate}
% \end{proof}

За да докажем, че $\val{\tau} \triangleleft_{\vv{nat}} \tau$, то трябва да докажем, че за всеки тип $\vv{a}$ и всеки затворен терм $\tau$ от тип $\vv{a}$, че е изпълнено
$\val{\tau} \triangleleft_{\vv{a}} \tau$ с индукция по построението на термовете.
Тук обаче имаме проблем. Ако $\tau \equiv \lamb{y}{b}{\tau_1}$, то трябва да използваме индукционно предположение за $\tau_1$,
в който вече има свободна променлива $\vv{y}$. Поради тази причина, ние трябва да разгледаме едно по-общо твърдение, при което позволяваме в термовете да се срещат свободни променливи.

\begin{framed}
  \begin{lemma}[Фундаментално свойство на $\triangleleft_{\vv{a}}$]\label{lem:pcf:fundamental}
    Нека $\Gamma = \vv{x}_1:\vv{a}_1,\dots,\vv{x}_n:\vv{a}_n$. Тогава
    \begin{prooftree}
      \AxiomC{$\Gamma \vdash \tau : \vv{a}$}
      \AxiomC{$(u_1,\dots,u_n) \triangleleft_\Gamma (\mu_1,\dots,\mu_n)$}
      \BinaryInfC{$\val{\tau}_\Gamma(\ov{u}) \triangleleft_{\vv{a}} \tau[\ov{\vv{x}}/\ov{\mu}]$}
    \end{prooftree}
  \end{lemma}  
\end{framed}
\begin{proof}
  Индукция по построението на термовете.
  \begin{itemize}
  \item
    Нека $\tau \equiv \vv{n}$. Тук директно от дефиницията на релацията $\triangleleft_{\vv{nat}}$ имаме, че
    \[n \triangleleft_{\vv{nat}} \vv{n}.\]
  \item
    \marginpar{Понеже $\Gamma \vdash \tau : \vv{a}$, то няма как $\tau$ да е променлива, която да не е някоя измежду $\vv{x}_1,\dots,\vv{x}_n$.}
    Нека $\tau \equiv \vv{x}_i$. Този случай също е много лесен.
    Имаме, че $\tau[\ov{x}/\ov{\mu}] \equiv \mu_i$ и $\val{\tau}_\Gamma(\ov{u}) = u_i$.
    Тогава, понеже $(u_1,\dots,u_n) \triangleleft_\Gamma (\mu_1,\dots,\mu_n)$,
    то директно получаваме, че
    \[\val{\tau}_\Gamma(\ov{u}) \triangleleft_{\vv{a}} \tau[\ov{x}/\ov{\mu}].\]
  \item
    Нека $\tau \equiv \tau_1 + \tau_2$. Тогава $\vv{a} = \vv{nat}$. Имаме също, че
    \begin{align*}
      & \tau[\ov{x}/\ov{\mu}] \equiv \tau_1[\ov{x}/\ov{\mu}] + \tau_2[\ov{x}/\ov{\mu}];\\
      & \val{\tau}_\Gamma(\ov{u}) = \plus(\underbrace{\val{\tau_1}_\Gamma(\ov{u})}_{n_1},\underbrace{\val{\tau_2}_\Gamma(\ov{u})}_{n_2}) = n.
    \end{align*}
    Можем да приемем, че $n \neq \bot$, защото в противен случай този случай е тривиален заради дефиницията на $\triangleleft_{\vv{nat}}$.
    От \IndHyp имаме, че $\val{\tau_1}_\Gamma(\ov{u}) \triangleleft_{\vv{nat}} \tau_1[\ov{x}/\ov{\mu}]$
    и $\val{\tau_2}_\Gamma(\ov{u}) \triangleleft_{\vv{nat}} \tau_2[\ov{x}/\ov{\mu}]$.
    Това означава, че $\tau_1[\ov{x}/\ov{\mu}] \Downarrow_{\vv{nat}} \vv{n}_1$ и $\tau_2[\ov{x}/\ov{\mu}] \Downarrow_{\vv{nat}}
    \vv{n}_2$.
    От правилата на операционната семантика е ясно, че $\tau[\ov{x}/\ov{\mu}] \Downarrow_{\vv{nat}} \vv{n}$ и следователно
    \[\val{\tau}_\Gamma(\ov{u}) \triangleleft_{\vv{a}} \tau[\ov{\vv{x}}/\ov{\mu}].\]
  \item
    Нека $\tau \equiv \tau_1\ \vv{-}\ \tau_2$. 
  \item
    Нека $\tau \equiv \tau_1\ \vv{==}\ \tau_2$. 
  \item
    Нека $\tau \equiv \ifelse{\tau_1}{\tau_2}{\tau_3}$.
  \item
    Нека $\tau \equiv \tau_1\tau_2$. От правилата за типизиране имаме, че
    \begin{prooftree}
      \AxiomC{$\Gamma \vdash \tau_1 : \vv{b} \to \vv{a}$}
      \AxiomC{$\Gamma \vdash \tau_2 : \vv{b}$}
      \BinaryInfC{$\Gamma \vdash \tau_1\tau_2 : \vv{a}$}
    \end{prooftree}
    Да напомним, че
    \[\val{\tau_1\tau_2}_\Gamma(\ov{u}) \df \texttt{eval}(\val{\tau_1}_\Gamma(\ov{u}), \val{\tau_2}_\Gamma(\ov{u})).\]
    От \IndHyp имаме следното:
    \begin{align*}
      & \val{\tau_1}_\Gamma(\ov{u}) \triangleleft_{\vv{b}\to\vv{a}} \tau_1[\ov{x}/\ov{\mu}];\\
      & \val{\tau_2}_\Gamma(\ov{u}) \triangleleft_{\vv{b}} \tau_2[\ov{x}/\ov{\mu}].
    \end{align*}
    Тогава директно следва, че
    \[\texttt{eval}(\val{\tau_1}_\Gamma(\ov{u}), \val{\tau_2}_\Gamma(\ov{u})) \triangleleft_{\vv{a}} \tau_1[\ov{x}/\ov{\mu}](\tau_2[\ov{x}/\ov{\mu}])\]
  \item
    Нека $\tau = \lamb{y}{b}{\tau'}$. Тогава от правилата за типизиране следва, че $\vv{a} = \vv{b} \to \vv{c}$ и
    $\Gamma' \vdash \tau' : \vv{c}$, където $\Gamma' = \Gamma, \type{y}{b}$.
    Да напомним, че
    \[\val{\tau}_\Gamma(\ov{u}) \df \texttt{curry}(\val{\tau'}_{\Gamma'})(\ov{u}) \in \Cont{\val{\vv{b}}}{\val{\vv{c}}}.\]
    Да положим $f \df \val{\tau}_\Gamma(\ov{u})$. Тогава $f(e) = \val{\tau'}_{\Gamma'}(\ov{u},e)$.
    
    Трябва да докажем, че $f \triangleleft_{\vv{b} \to \vv{c}} \tau[\ov{x}/\ov{\mu}]$.
    Това означава, че за произволни $e \in \val{\vv{b}}$ и $\rho : \vv{b}$, за които $e \triangleleft_{\vv{b}} \rho$,
    трябва да докажем, че $f(e) \triangleleft_{\vv{c}} \tau[\ov{x}/\ov{\mu}](\rho)$.
    Имаме, че
    \begin{prooftree}
      \AxiomC{$\Gamma' \vdash \tau' : \vv{c}$}
      \AxiomC{$(u_1,\dots,u_n,e) \triangleleft_{\Gamma'} (\mu_1,\dots,\mu_n,\rho)$}
      \RightLabel{\scriptsize{\IndHyp}}
      \BinaryInfC{$\val{\tau'}(\ov{u},e) \triangleleft_{\vv{c}} \tau'[\ov{x}/\ov{\mu}][y/\rho]$}
      \UnaryInfC{$f(e) \triangleleft_{\vv{c}} \tau'[\ov{x}/\ov{\mu}][y/\rho]$}
    \end{prooftree}
    От правилата на операционната семантика имаме следното:
    \begin{prooftree}
      \AxiomC{$\emptyset \vdash \rho : \vv{b}$}
      \AxiomC{$\tau'[\ov{x}/\ov{\mu}][y/\rho] \Downarrow_{\vv{c}} \vv{v}$}
      \BinaryInfC{$(\lamb{y}{b}{\tau'[\ov{x}/\ov{\mu}]})(\rho) \Downarrow_{\vv{c}} \vv{v}$}
      \UnaryInfC{$\tau[\ov{x}/\ov{\mu}](\rho) \Downarrow_{\vv{c}} \vv{v}$}
    \end{prooftree}
    От \Prop{pcf:adequacy:implication} веднага заключаваме, че $f(e) \triangleleft_{\vv{c}} \tau[\ov{x}/\ov{\mu}](\rho)$.
  \item
    Нека $\tau \equiv \fix(\tau')$. Тогава от правилата за типизиране имаме, че $\Gamma \vdash \tau' : \vv{a} \to \vv{a}$.
    От \IndHyp, приложено за $\tau'$, имаме, че
    \[\val{\tau'}_\Gamma(\ov{u}) \triangleleft_{\vv{a}\to\vv{a}} \tau'[\ov{x}/\ov{\mu}].\]
    Нека за улеснение да положим $f \df \val{\tau'}_\Gamma(\ov{u}) \in \Cont{\val{\vv{a}}}{\val{\vv{a}}}$.
    Да напомним, че
    \[\val{\fix(\tau')}_\Gamma(\ov{u}) = \lfp(f)\]
    От \Prop{pcf:adequacy:chain} знаем, че за всяка верига $\chain{d}{i}$ от елементи на
    \[D \df \{d \in \val{\vv{a}} \mid d \triangleleft_{\vv{a}} \fix(\tau'[\ov{x}/\ov{\mu}])\}\]
    е изпълнено, че $\bigsqcup_i d_i \in D$. Целта ни е да докажем, че
    $\lfp(f) \in D$. Да напомним, че $\lfp(f) = \bigsqcup_n f^n(\bot^{\val{\vv{a}}})$.
    С индукция по $n$ ще докажем, че за всяко $n$, $f^n(\bot^{\val{\vv{a}}}) \in D$.


    % Ще направим това като приложим правилото на Скот.
    % Да напомним, че
    % \begin{prooftree}
    %   \AxiomC{$\bot^{\val{\vv{a}}} \in D$}
    %   \AxiomC{$d \in D \implies f(d) \in D$}
    %   \BinaryInfC{$\lfp(f) \in D$}
    % \end{prooftree}
    За $n = 0$, понеже от \Prop{pcf:adequacy:bottom} имаме, че $f^0(\bot^{\val{\vv{a}}}) = \bot^{\val{\vv{a}}} \in D$.
    Нека $n > 0$ и от \IndHyp имаме, че $f^{n-1}(\bot^{\val{\vv{a}}}) \in D$.
    % нека вземем произволен елемент $d \in D$. Ще докажем, че $f(d) \in D$.

    Понеже $f \triangleleft_{\vv{a}\to\vv{a}} \tau'[\ov{x}/\ov{\mu}]$, то
    за произволно $e \triangleleft_{\vv{a}} \rho$ е изпълнено, че
    $f(e) \triangleleft_{\vv{a}} \tau'[\ov{x}/\ov{\mu}](\rho)$.
    Нека изберем $\rho = \fix(\tau'[\ov{x}/\ov{\mu}])$ и $e = f^{n-1}(\bot^{\val{\vv{a}}})$.
    Щом $f^{n-1}(\bot^{\val{\vv{a}}}) \in D$, то $f^{n-1}(\bot^{\val{\vv{a}}}) \triangleleft_{\vv{a}} \rho$ и следователно
    \[f(f^{n-1}(\bot^{\val{\vv{a}}})) \triangleleft_{\vv{a}} \tau'[\ov{x}/\ov{\mu}](\underbrace{\fix(\tau'[\ov{x}/\ov{\mu}])}_{\rho}).\]
    От правилата на операционната семантика имаме, че:
    \begin{prooftree}
      \AxiomC{$\tau'[\ov{x}/\ov{\mu}](\fix(\tau'[\ov{x}/\ov{\mu}])) \Downarrow_{\vv{a}} \vv{v}$}
      \RightLabel{\scriptsize{(fix)}}
      \UnaryInfC{$\fix(\tau'[\ov{x}/\ov{\mu}]) \Downarrow_{\vv{a}} \vv{v}$}
    \end{prooftree}
    Тогава от \Prop{pcf:adequacy:implication} следва, че
    \[f(f^{n-1}(\bot^{\val{\vv{a}}})) \triangleleft_{\vv{a}} \fix(\tau'[\ov{x}/\ov{\mu}]).\]
    Така доказахме, че $f^n(\bot^{\val{\vv{a}}}) \in D$.
    Заключаваме, че $\lfp(f) \in D$, т.е.
    \[\val{\fix(\tau')}_\Gamma(\ov{u}) \triangleleft_{\vv{a}} \fix(\tau'[\ov{x}/\ov{\mu}])\]
  \end{itemize}
\end{proof}

\begin{corollary}\label{cr:pcf:fundamental}
  За всеки тип $\vv{a}$ и за всеки затворен терм $\tau$ е изпълнено свойството:
  \[\tau \in \text{PCF}_{\vv{a}} \implies \val{\tau} \triangleleft_{\vv{a}} \tau.\]
\end{corollary}

Така на практика доказахме теоремата за адекватност.

\begin{framed}
  \begin{theorem}[Теорема за адекватност]\label{th:pcf:adequacy}
    За всеки затворен терм $\tau : \vv{nat}$, 
    \[\val{\tau} = n \neq \bot \implies \tau \Downarrow_{\vv{nat}} \vv{n}.\]
  \end{theorem}
\end{framed}
\begin{proof}
  Да разгледаме произволен затворен терм $\tau : \vv{nat}$.
  Нека $\val{\tau} = n \neq \bot$.
  От \Cor{pcf:fundamental} имаме, че $\val{\tau} \triangleleft_{\vv{nat}} \tau$.
  Тогава от дефиницията на $\triangleleft_{\vv{nat}}$ получаваме, че $\tau \Downarrow_{\vv{nat}} \vv{n}$.
\end{proof}

Да разгледаме термовете
\begin{align*}
  & \tau \equiv \lamb{x}{nat}{\vv{x + 0}}\\
  & \rho \equiv \lamb{x}{nat}{\vv{x}}.
\end{align*}


Ясно е, че $\val{\tau} = \val{\rho}$, но според правилата на операционната семантика, понеже $\tau$ е стойност, то
$\tau \not\Downarrow_{\vv{nat}\to\vv{nat}} \rho$.
Оттук веднага е ясно, че няма как да имаме теорема за адекватност за по-високи типове от $\vv{nat}$.


\begin{corollary}
  За всеки $\tau \in \text{PCF}_{\nat}$ е изпълнена импликацията:
  \[\tau \not\Downarrow_{\nat}\ \implies\ \val{\tau} = \bot^{\val{\nat}}.\]  
\end{corollary}
\begin{proof}
  Да допуснем, че $\tau \not\Downarrow_{\nat}$, но $\val{\tau} = n \neq \bot$.
  Но тогава от \hyperref[th:pcf:adequacy]{теоремата за адекватност} следва, че $\tau \Downarrow_{\nat} \vv{n}$, което е противоречие.
\end{proof}

\begin{framed}
  \begin{corollary}
    Нека $\tau$ е затворен терм от тип $\nat$. Тогава 
    \[\val{\tau} = \bot^{\val{\nat}} \text{ точно тогава, когато } \tau \not\Downarrow_{\nat}.\]
  \end{corollary}
\end{framed}

%%% Local Variables:
%%% mode: latex
%%% TeX-master: "../sep"
%%% End:

\newpage
\section{Езикът \texttt{PCF(bool)}}

\marginpar{Всъщност в \cite[Глава 4.1]{gunter} това е ,,истинската'' дефиниция на езика \PCF.}

\begin{itemize}
\item
  Типове
  \[\vv{a} ::= \bool\ |\ \nat\ |\ \vv{a}\to\vv{a}\]
\item
  Изрази
  \begin{align*}
    \tau ::=\ & \tru\ |\ \fls\ |\ \vv{n}\ |\ \vv{x}\ |\ \tau_1 + \tau_2\ |\ \tau_1 - \tau_2\ |\  \tau_1\ \vv{==}\ \tau_2\ |\\
              & \ifelse{\tau_1}{\tau_2}{\tau_3}\ |\ \tau_1\tau_2\ |\ \lamb{x}{a}{\tau_1}\ |\ \fix(\tau_1).
  \end{align*}
  % Термове
  % \begin{align*}
  %   \tau ::=\ & \vv{0}\ |\ \tru\ |\ \fls\ |\ \vv{x}\ |\\
  %            & \scc(\tau_1)\ |\ \prd(\tau_1)\ |\ \texttt{iszero}(\tau)\ |\ \ifelse{\tau_1}{\tau_2}{\tau_3}\ |\\
  %            &\tau_1\tau_2\ |\ \lamb{x}{a}{\tau_1}\ |\ \fix(\tau_1).
  % \end{align*}
\item
  Стойностите са затворени термове от следния вид:
  \[\vv{v} ::= \tru\ |\ \fls\ |\ \vv{n}\ |\ \lamb{x}{a}{\mu}\]
\end{itemize}

\subsection{Типизираща релация}

Релацията $\Gamma \vdash \tau : \vv{a}$ за езика \PCFBOOL е почти същата
както за езика \PCF. Имаме две нови правила:

\begin{figure}[H]
  \begin{subfigure}[b]{0.5\textwidth}
    \begin{prooftree}
      \AxiomC{}
      \RightLabel{\scriptsize{(true)}}
      \UnaryInfC{$\Gamma \vdash \tru : \bool$}
    \end{prooftree}
  \end{subfigure}
  ~
  \begin{subfigure}[b]{0.5\textwidth}
    \begin{prooftree}
      \AxiomC{}
      \RightLabel{\scriptsize{(false)}}
      \UnaryInfC{$\Gamma \vdash \fls : \bool$}
    \end{prooftree}
  \end{subfigure}
\end{figure}
Имаме и две променени правила:
\begin{prooftree}
  \AxiomC{$\Gamma \vdash \tau_1:\nat$}
  \AxiomC{$\Gamma \vdash \tau_2:\nat$}
  \RightLabel{\scriptsize{(eq)}}
  \BinaryInfC{$\Gamma \vdash \tau_1\ \vv{==}\ \tau_2 : \bool$}
\end{prooftree}
\begin{prooftree}
  \AxiomC{$\Gamma \vdash \tau_1:\bool$}
  \AxiomC{$\Gamma \vdash \tau_2:\vv{a}$}
  \AxiomC{$\Gamma \vdash \tau_3:\vv{a}$}
  \RightLabel{\scriptsize{(if)}}
  \TrinaryInfC{$\Gamma \vdash \ifelse{\tau_1}{\tau_2}{\tau_3} : \vv{a}$}
\end{prooftree}
Всички останали правила са същите.

% \end{subfigure}
% ~
% \begin{subfigure}[b]{0.5\textwidth}
% \begin{prooftree}
%   \AxiomC{$\Gamma \vdash \tau:\vv{a}\to\vv{a}$}
%   \RightLabel{\scriptsize{(fix)}}
%   \UnaryInfC{$\Gamma \vdash \fix(\tau) : \vv{a}$}
% \end{prooftree}
% \end{subfigure}

% \vspace{10pt}

% \begin{subfigure}[b]{0.5\textwidth}
% \begin{prooftree}
%   \AxiomC{$\Gamma \vdash \tau_1:\vv{a}\to\vv{b}$}
%   \AxiomC{$\Gamma \vdash \tau_2:\vv{a}$}
%   \RightLabel{\scriptsize{(app)}}
%   \BinaryInfC{$\Gamma \vdash \tau_1\tau_2 : \vv{b}$}
% \end{prooftree}
% \end{subfigure}
% ~
% \begin{subfigure}[b]{0.5\textwidth}
% \begin{prooftree}
%   \AxiomC{$\vv{x} \not\in\vv{dom}(\Gamma)$}
%   \AxiomC{$\Gamma, \type{x}{a} \vdash \tau:\vv{b}$}
%   \RightLabel{\scriptsize{(lambda)}}
%   \BinaryInfC{$\Gamma \vdash \lambda \type{x}{a}\ .\ \tau : \vv{a} \to \vv{b}$}
% \end{prooftree}
% \end{subfigure}

% \caption{Релация за типизиране на термовете от езика \texttt{PCF++}}
% \label{fig:pcf:extensions:relation}
% \end{figure}


\subsection{Операционна семантика}

% \marginpar{\cite[стр. 109]{gunter}}

Тук отново всичко е почти същото както преди със следните разлики:

\begin{figure}[H]
  % \begin{subfigure}[b]{0.5\textwidth}
  %   \begin{prooftree}
  %     \AxiomC{$\type{v}{a}$}
  %     \RightLabel{\scriptsize{(val)}}
  %     \UnaryInfC{$\vv{v} \Downarrow^0_{\vv{a}} \vv{v}$}
  %   \end{prooftree}    
  % \end{subfigure}
  % ~
  % \begin{subfigure}[b]{0.5\textwidth}
  %   \begin{prooftree}
  %     \AxiomC{$\tau \Downarrow^\ell_{\nat} \vv{0}$}
  %     \UnaryInfC{$\prd(\tau) \Downarrow^{\ell+1}_{\nat} \vv{0}$}
  %   \end{prooftree}
  % \end{subfigure}

  % \vspace{10pt}

  % \begin{subfigure}[b]{0.5\textwidth}
  %   \begin{prooftree}
  %     \AxiomC{$\tau \Downarrow^\ell_{\nat} \scc(\vv{v})$}
  %     \UnaryInfC{$\prd(\tau) \Downarrow^{\ell+1}_{\nat} \vv{v}$}
  %   \end{prooftree}
  % \end{subfigure}
  % ~
  % \begin{subfigure}[b]{0.5\textwidth}
  %   \begin{prooftree}
  %     \AxiomC{$\tau \Downarrow^\ell_{\nat} \vv{v}$}
  %     \UnaryInfC{$\scc(\tau) \Downarrow^{\ell+1}_{\nat} \scc(\vv{v})$}
  %   \end{prooftree}
  % \end{subfigure}

  % \vspace{10pt}

  % \begin{subfigure}[b]{0.5\textwidth}
  %   \begin{prooftree}
  %     \AxiomC{$\tau \Downarrow^\ell_{\nat} \vv{0}$}
  %     \UnaryInfC{$\iszero(\tau) \Downarrow^{\ell+1}_{\bool} \tru$}
  %   \end{prooftree}
  % \end{subfigure}
  % ~
  % \begin{subfigure}[b]{0.5\textwidth}
  %   \begin{prooftree}
  %     \AxiomC{$\tau \Downarrow^\ell_{\nat} \scc(\vv{v})$}
  %     \UnaryInfC{$\iszero(\tau) \Downarrow^{\ell+1}_{\bool} \fls$}
  %   \end{prooftree}
  % \end{subfigure}

  % \vspace{10pt}
  
  \begin{subfigure}[b]{0.5\textwidth}
    \begin{prooftree}
      \AxiomC{$\tau_1 \opsem{\ell_1}{nat} \vv{v}_1$}
      \AxiomC{$\tau_3 \opsem{\ell_2}{a} \vv{v}_2$}
      \AxiomC{$\vv{v}_1 \equiv \vv{v}_2$}
      % \RightLabel{\scriptsize{(if$_\fls$)}}
      \TrinaryInfC{$\tau_1\ \vv{==}\ \tau_2 \opsem{\ell_1+\ell_2+1}{bool} \tru$}
    \end{prooftree}
  \end{subfigure}
  ~
  \begin{subfigure}[b]{0.5\textwidth}
    \begin{prooftree}
      \AxiomC{$\tau_1 \opsem{\ell_1}{nat} \vv{v}_1$}
      \AxiomC{$\tau_3 \opsem{\ell_2}{a} \vv{v}_2$}
      \AxiomC{$\vv{v}_1 \not\equiv \vv{v}_2$}
      % \RightLabel{\scriptsize{(if$_\fls$)}}
      \TrinaryInfC{$\tau_1\ \vv{==}\ \tau_2 \opsem{\ell_1+\ell_2+1}{bool} \fls$}
    \end{prooftree}
  \end{subfigure}

  \vspace{10pt}
  
  \begin{subfigure}[b]{0.5\textwidth}
    \begin{prooftree}
      \AxiomC{$\tau_1 \opsem{\ell_1}{bool} \fls$}
      \AxiomC{$\tau_3 \opsem{\ell_2}{a} \vv{v}$}
      % \RightLabel{\scriptsize{(if$_\fls$)}}
      \BinaryInfC{$\ifelse{\tau_1}{\tau_2}{\tau_3} \opsem{\ell_1+\ell_2+1}{a} \vv{v}$}
    \end{prooftree}
  \end{subfigure}
  ~
  \begin{subfigure}[b]{0.5\textwidth}
    \begin{prooftree}
      \AxiomC{$\tau_1 \opsem{\ell_1}{bool} \tru$}
      \AxiomC{$\tau_2 \opsem{\ell_2}{a} \vv{v}$}
      % \RightLabel{\scriptsize{(if$_\tru$)}}
      \BinaryInfC{$\ifelse{\tau_1}{\tau_2}{\tau_3} \opsem{\ell_1+\ell_2+1}{a} \vv{v}$}
    \end{prooftree}
  \end{subfigure}

%   \vspace{10pt}

%   \begin{subfigure}[b]{0.5\textwidth}
%     \begin{prooftree}
%       \AxiomC{$\tau_1 \Downarrow^{\ell_1}_{\vv{a}\to\vv{b}} \lamb{x}{a}{\tau'_1}$}
%       \AxiomC{$\tau'_1[x/\tau_2] \Downarrow^{\ell_2}_{\vv{b}} \vv{v}$}
%       % \RightLabel{\scriptsize{(cbn)}}
%       \BinaryInfC{$\tau_1 \tau_2 \Downarrow^{\ell_1+\ell_2+1}_{\vv{b}} \vv{v} $}
%     \end{prooftree}
%   \end{subfigure}
%   ~
%   \begin{subfigure}[b]{0.5\textwidth}
%   \begin{prooftree}
%     \AxiomC{$\tau\ \fix(\tau) \Downarrow^{\ell}_{\vv{a}} \vv{v}$}
%     \RightLabel{\scriptsize{(fix)}}
%     \UnaryInfC{$\fix(\tau) \Downarrow^{\ell+1}_{\vv{a}} \vv{v} $}
%   \end{prooftree}
% \end{subfigure}
% \caption{Правила на операционната семантика за езика \PCFPP}
\end{figure}




% \begin{lemma}
%   Нека $\tau$ е затворен терм от тип $\vv{a}$.
%   Тогава ако $\tau \Downarrow_{\vv{a}} \vv{v}$ и $\tau \Downarrow_{\vv{a}} \vv{u}$, то
%   $\vv{v} \equiv_\alpha \vv{u}$.
% \end{lemma}


\subsection{Денотационна семантика}

\marginpar{В \cite[Глава 4.3]{gunter} се нарича \emph{standard fixed-point semantics} of PCF.}

Семантиката на всеки тип ще бъде област на Скот както следва:
% \begin{align*}
\[\val{\bool} \df \Bool = \{true, false\}_\bot.\]
%   & \val{\nat} \df \Nat_\bot\\
%   & \val{\vv{a} \to \vv{b}} \df \Cont{\val{\vv{a}}}{\val{\vv{b}}}.
% \end{align*}

\begin{itemize}
% \item
%   Нека $\tau \equiv \vv{0}$. Тогава
%   \[\val{\vv{0}}_\Gamma(\overline{u}) \df 0.\]
\item
  Нека $\tau \equiv \tru$. Тогава
  \[\val{\tru}_\Gamma(\overline{u}) \df true.\]
\item
  Нека $\tau \equiv \fls$. Тогава
  \[\val{\fls}_\Gamma(\overline{u}) \df false.\]
% \item
%   Нека $\tau \equiv \vv{x}_i$. Тогава
%   \[\val{\vv{x}_i}_\Gamma(\overline{u}) \df u_i.\]
% \item
%   Нека $\tau \equiv \scc(\tau_1)$. Тогава
%   \[\val{\scc(\tau_1)}_\Gamma(\ov{u}) \df
%   \begin{cases}
%     \val{\tau_1}_\Gamma(\ov{u}) + 1, & \text{ ако }\val{\tau_1}_\Gamma(\ov{u}) \neq \bot\\
%     \bot, & \text{ ако }\val{\tau_1}_\Gamma(\ov{u}) = \bot.
%   \end{cases}\]

% \item
%   Нека $\tau \equiv \prd(\tau_1)$. Тогава
%   \[\val{\prd(\tau_1)}_\Gamma(\ov{u}) \df
%   \begin{cases}
%     0, & \text{ ако }\val{\tau_1}(\ov{u}) = 0\\
%     \val{\tau_1}_\Gamma(\ov{u}) - 1, & \text{ ако }\val{\tau_1}_\Gamma(\ov{u}) \neq 0, \bot\\
%     \bot, & \text{ ако }\val{\tau_1}_\Gamma(\ov{u}) = \bot.
%   \end{cases}\]

% \item
%   Нека $\tau \equiv \iszero(\tau_1)$. Тогава
%   \[\val{\iszero(\tau_1)}_\Gamma(\ov{u}) \df
%   \begin{cases}
%     true, &  \text{ ако }\val{\tau_1}_\Gamma(\ov{u}) = 0\\
%     false, & \text{ ако }\val{\tau_1}_\Gamma(\ov{u}) \neq 0,\bot\\
%     \bot, &  \text{ ако }\val{\tau_1}_\Gamma(\ov{u}) = \bot.
%   \end{cases}\]


% \item
%   \marginpar{За $\texttt{eq}$ вижте Раздел~\ref{subsect:rec:term-value}.}
%   Нека $\tau \equiv \tau_1\ \vv{==}\ \tau_2$. Тогава
%   \[\val{\tau_1\ \vv{==}\ \tau_2}_\Gamma(\overline{u}) \df \texttt{eq}(\val{\tau_1}_\Gamma(\overline{u}), \val{\tau_2}_\Gamma(\overline{u})).\]

\item
  Нека $\tau \equiv \tau_1\ \vv{==}\ \tau_2$. Тогава
  \[\val{\tau_1\ \vv{==}\ \tau_2}_\Gamma(\overline{u}) \df
    \begin{cases}
      true, & \text{ ако }\val{\tau_1}_\Gamma(\ov{u}) = \val{\tau_1}_\Gamma(\ov{u}) \in \Nat\\
      false, & \text{ ако }\val{\tau_1}_\Gamma(\ov{u}) \neq \val{\tau_1}_\Gamma(\ov{u}) \in \Nat\\
      \bot, & \text{ ако } \val{\tau_1}_\Gamma(\ov{u}) = \bot \text{ или } \val{\tau_1}_\Gamma(\ov{u}) = \bot.
    \end{cases}\]
\item
  % \marginpar{За $\texttt{if}$ вижте \Def{if}.}
  Нека $\tau \equiv \ifelse{\tau_1}{\tau_2}{\tau_3}$. Тогава
  \[\val{\ifelse{\tau_1}{\tau_2}{\tau_3}}_\Gamma(\overline{u}) \df
    \begin{cases}
      \val{\tau_2}_\Gamma(\ov{u}), & \text{ ако }\val{\tau_1}_\Gamma(\ov{u}) = true\\
      \val{\tau_3}_\Gamma(\ov{u}), & \text{ ако }\val{\tau_1}_\Gamma(\ov{u}) = false\\
      \bot, & \text{ ако } \val{\tau_1}_\Gamma(\ov{u}) = \bot.
    \end{cases}\]
% \item
%   \marginpar{За $\texttt{eval}$ вижте \Def{eval}.}
%   Нека $\tau \equiv \tau_1 \tau_2$. Тогава
%   \[\val{\tau_1 \tau_2}_\Gamma(\overline{u}) \df \texttt{eval}(\val{\tau_1}_\Gamma(\overline{u}), \val{\tau_2}_\Gamma(\overline{u})).\]
% \item
%   \marginpar{За $\lfp$ вижте Раздел~\ref{sect:lfp}.}
%   Нека $\tau \equiv \fix(\tau')$. Тогава 
%   \[\val{\fix(\tau')}_\Gamma(\overline{u}) \df \lfp(\val{\tau'}_\Gamma(\overline{u})).\]
% \item
%   \marginpar{За $\curry$ вижте \Def{curry}.}
%   Нека $\tau \equiv \lamb{y}{b}{\tau'}$, като $\vv{y} \not \in \texttt{dom}(\Gamma)$.
%   Нека $\Gamma' \df \Gamma, \type{y}{b}$. Тогава
%   \[\val{\lamb{y}{b}{\tau'}}_\Gamma(\overline{u}) \df \curry(\val{\tau'}_{\Gamma'})(\overline{u}).\]
\end{itemize}

% \begin{problem}
%   \marginpar{Аналог на \cite[Лема 4.19]{gunter}.}
%   Докажете, че ако $\Gamma \vdash \tau : \vv{a}$, то $\val{\tau}_\Gamma \in \Cont{\val{\Gamma}}{\val{\vv{a}}}$.
% \end{problem}

% \begin{problem}
%   Нека $\Gamma$ е типов контекст, $\tau$ и $\rho$ са термове, $\vv{x} \not\in \texttt{dom}(\Gamma)$,
%   \begin{align*}
%     & \Gamma \vdash \rho : \vv{a}\\
%     & \Gamma, \type{x}{a} \vdash \tau : \vv{b}.
%   \end{align*}
%   Докажете, че тогава:
%   \begin{enumerate}[1)]
%   \item
%     $\Gamma \vdash \tau\subst{x}{\rho} : \vv{b}$;
%   \item
%     за всяко $\overline{u} \in \val{\Gamma}$,
%     \[\val{\tau\subst{x}{\rho}}_\Gamma(\overline{u}) = \val{\tau}_{\Gamma'}(\overline{u},\val{\rho}_\Gamma(\overline{u})),\]
%     където $\Gamma' = \Gamma, \type{x}{a}$.  
%   \end{enumerate}
% \end{problem}

\begin{theorem}[Теорема за коректност за езика $\texttt{PCF(bool)}$]
  % \marginpar{Аналог на \cite[Твърдение 4.23]{gunter}.}
  Докажете, че за всеки затворен терм $\tau$ от тип $\vv{a}$ и стойност $\vv{v}$, е изпълнена импликацията:
  \[\tau \Downarrow_{\vv{a}} \vv{v}\ \implies\ \val{\tau} = \val{\vv{v}} \in \val{\vv{b}}.\]
\end{theorem}

\begin{theorem}[Теорема за адекватност за езика $\texttt{PCF(bool)}$]
  Нека разгледаме тип $\vv{a} = \nat$ или $\vv{a} = \bool$.
  За всеки затворен терм $\tau$ от тип $\vv{a}$ е изпълнена импликацията
  \[\val{\tau} = v \neq \bot^{\val{\vv{a}}} \implies \tau \Downarrow_{\vv{a}} \vv{v}.\]
\end{theorem}


%%% Local Variables:
%%% mode: latex
%%% TeX-master: "../sep"
%%% End:

\newpage
\section{Контексти}\label{pcf:sect:context}

\index{контекст}
\marginpar{\cite[Глава 6.1]{gunter}}
\marginpar{\cite[Глава 48]{practical-foundations}}
\marginpar{Интуитивно, контекстите са програмни фрагменти. Не са пълни програми, защото имат празни места означени с $-$.}
\[\C ::= -\ |\ \vv{n}\ |\ \vv{x}\ |\ \C + \C\ |\ \C\ \vv{==}\ \C\ |\ \ifelse{\C}{\C}{\C}\ |\ \C\C\ |\ \lamb{x}{a}{\C}\ |\ \fix(\C).\]
Контекстите са на практика изрази, като позволяваме да имат специална свободна променлива, която означаваме с $-$.
% За произволен израз $\tau$, с $\C\{\tau\}$ означаваме израза $\C\rename{-}{\tau}$, където считаме $-$ като свободна променлива на $\C$.
% Това означава, че заместваме всички срещания на $-$ с $\tau$ като правим заместването директно, т.е.
% не се интересуваме дали някоя свободна променлива на $\tau$ няма да попадне под обхвата на свързана променлива в $\C$.
% Например, ако $\C = \lamb{x}{a}{-}$, то $\C\rename{-}{\vv{x}} =
% \lamb{x}{a}{\vv{x}}$.
За краткост, вместо $\C\{-/\tau\}$ ще пишем $\C[\tau]$.

\begin{proposition}
  % \marginpar{Да напомним, че с $\equiv$ означаваме релацията $\alpha$-еквивалентност.}
  Ако $\tau \equiv_\alpha \tau'$, то $\C[\tau] \equiv_\alpha \C[\tau']$.
\end{proposition}
\begin{hint}
  Индукция по построението на контекстите.
\end{hint}

\marginpar{Някои наричат тази релация observational equivalence. Тук наричаме релацията contextual equivalence.}
\begin{definition}\label{df:context:equivalence}
  За затворени термове $\tau_1$ и $\tau_2$, дефинираме
  $\tau_1 \leq_{ctx} \tau_2 : \vv{a}$, ако
  \begin{enumerate}[1)]
  \item
    $\emptyset \vdash \tau_1 : \vv{a}$ и $\emptyset \vdash \tau_2 : \vv{a}$
  \item
    За \emph{всички} контексти $C[-]$, за които $\emptyset \vdash C[\tau_1] : \nat$ и $\emptyset \vdash C[\tau_2] : \nat$, то
    \[C[\tau_1] \Downarrow_{\nat} \vv{n} \implies C[\tau_2] \Downarrow_{\nat} \vv{n}.\]
  \end{enumerate}
  Ще пишем $\tau_1 \cong_{ctx} \tau_2 : \vv{a}$, ако
  $\tau_1 \leq_{ctx} \tau_2 : \vv{a}$ и $\tau_2 \leq_{ctx} \tau_1 : \vv{a}$.
\end{definition}


% \begin{itemize}
% \item 
  % Ще пишем $\tau_1 \cong_{ctx} \tau_2 : \vv{a}$, ако
  % $\tau_1 \leq_{ctx} \tau_2 : \vv{a}$ и $\Gamma \vdash \tau_2 \leq_{ctx} \tau_1 : \vv{a}$.
% \item
%   Ако $\Gamma = \emptyset$, то ще пишем $\tau_1 \cong_{ctx} \tau_2 : \vv{a}$ вместо $\emptyset \vdash \tau_1 \cong_{ctx} \tau_2 : \vv{a}$.  
% \end{itemize}

\marginpar{Two phrases of a programming language are contextually equivalent if any occurrences of the first phrase in a complete
  program can be replaced by the second phrase without affecting the observable results of executing the program.
  This kind of program equivalence is also known as operational, or observational equivalence.}

\begin{framed}
  Денотационната семантика $\val{.}$ се нарича {\bf напълно абстрактна}, ако денотационната и операционната наредба съвпадат, т.е. $\val{\tau_1} \sqsubseteq \val{\tau_2}$ точно тогава, когато
  $\tau_1 \leq_{ctx} \tau_2$.  Един от основните ранни резултати в изучаването на семантиката на езици за програмиране е, че нашата денотационна семантика не е напълно абстрактна.
\end{framed}

Практически, ако имаме два терма, които са контекстно еквивалентни, то можем да заменим единия с другия в произволна програма,
без да има видими разлики на ниво изпълнение на програмата.
% Това представлява опит да се формализира математически практиката за тестване на програми.
Проблемът е, че с този формализъм е трудно да се работи, защото в дефиницията имаме квантор
за всеобщност относно всички контексти (програмни фрагменти).

\begin{proposition}\label{pr:pcf:context:simple}
  За произволни затворени термове $\tau_1$ и $\tau_2$ е изпълнено, че:
  \begin{enumerate}[(1)]
  \item
    $\tau_1 \leq_{ctx} \tau_2 : \nat \implies (\forall \vv{n})[\tau_1\Downarrow_{\nat}\vv{n} \implies \tau_2 \Downarrow_{\nat} \vv{n}]$.
  \item
    $\tau_1 \leq_{ctx} \tau_2 : \vv{b}\to\vv{c} \implies (\forall \rho\in\text{PCF}_{\vv{b}})[\tau_1\rho \leq_{ctx} \tau_2 \rho : \vv{c}]$.
  \end{enumerate}
\end{proposition}
\begin{hint}
  \begin{enumerate}[(1)]
  \item
    Просто вземете контекст $\C \df -$. Ясно е, че $\C[\tau_i] \equiv \tau_i$.
  \item
    Да фиксираме произволен терм $\rho\in\text{PCF}_{\vv{b}}$.
    Да разгледаме произволен контекст $\C[-]$, за който $\C[\tau_1\rho]$ и $\C[\tau_2\rho]$
    са затворени термове от тип $\nat$.
    Разглеждаме контекста $\C' \df \C\rename{-}{(-\rho)}$,
    т.е. заменяме $-$ с $-\rho$ в контекста.
    Лесно се съобразява, че \[\C'[\tau_i] \equiv \C[\tau_i\rho].\]
    Понеже $\tau_1 \leq_{ctx} \tau_2 : \vv{b}\to\vv{c}$,
    то за контекста $\C'$ имаме, че
    \[\C'[\tau_1] \Downarrow_{\nat} \vv{n} \implies \C'[\tau_2] \Downarrow_{\nat} \vv{n}.\]
    Оттук веднага следва, че 
    \[\C[\tau_1\rho] \Downarrow_{\nat} \vv{n} \implies \C[\tau_2\rho] \Downarrow_{\nat} \vv{n}.\]
  \end{enumerate}
\end{hint}

\begin{proposition}\label{pr:pcf:context:terms}
  За произволни затворени термове $\tau_1$ и $\tau_2$ е изпълнено, че:
  \[\tau_1 \leq_{ctx} \tau_2 : \vv{a} \iff (\forall \rho \in \text{PCF}_{\vv{a}\to\nat})[\ \rho\tau_1 \Downarrow_{\nat} \vv{n} \implies \rho\tau_2 \Downarrow_{\nat} \vv{n}\ ].\]
\end{proposition}
\begin{hint}
  За $(\Rightarrow)$, при даден терм $\rho \in \text{PCF}_{\vv{a}\to\nat}$, просто вземете контекст $\C~=~\rho~-$.
  За $(\Leftarrow)$, нека имаме контекст $\C[-]$.
  Нека $\vv{z}$ е променлива, която не се среща в $\C[-]$.
  Разгледайте $\rho \df \lamb{z}{a}{\C\rename{-}{\vv{z}}}$.
  Тогава, понеже $\tau_i$ са затворени термове, то
  $\C\rename{-}{\tau_i} \equiv \rho\tau_i$.
\end{hint}

\begin{proposition}\label{pr:pcf:context:relation}
  За произволни затворени термове $\tau_1$ и $\tau_2$ от тип $\vv{a}$,
  \[(d \triangleleft_{\vv{a}} \tau_1\ \&\ \tau_1 \leq_{ctx} \tau_2 : \vv{a}) \implies d \triangleleft_{\vv{a}} \tau_2.\]
\end{proposition}
\begin{proof}
  Индукция по построението на типовете $\vv{a}$.
  Нека $\vv{a} = \nat$.
  Нека $d \neq \bot$, защото е ясно, че $\bot \triangleleft_{\nat} \tau_2$.
  Понеже $d \triangleleft_{\nat} \tau_1$, то $\tau_1 \Downarrow_{\nat} \vv{d}$.
  Понеже $\tau_1 \leq_{ctx} \tau_2 : \nat$, то $\tau_2 \Downarrow_{\nat} \vv{d}$.
  Заключаваме, че $d \triangleleft_{\nat} \tau_2$.
  
  Нека $\vv{a} = \vv{b} \to \vv{c}$.
  Нека $d \triangleleft_{\vv{b} \to \vv{c}} \tau_1$ и $\tau_1 \leq_{ctx} \tau_2 : \vv{b}\to\vv{c}$.
  Трябва да докажем, че $d \triangleleft_{\vv{b}\to\vv{c}} \tau_2$.
  Нека $u \in \val{\vv{b}}$ и $\rho \in \text{PCF}_{\vv{b}}$,
  за които $u \triangleleft_{\vv{b}} \rho$. Трябва да докажем, че $d(u) \triangleleft_{\vv{c}} \tau_2\rho$.
  От $d \triangleleft_{\vv{b} \to \vv{c}} \tau_1$ имаме, че $d(u) \triangleleft_{\vv{c}} \tau_1 \rho$.
  От (2) на \Prop{pcf:context:simple} имаме, че $\tau_1 \rho \leq_{ctx} \tau_2 \rho : \vv{c}$.
  От И.П. заключаваме, че $d(u) \triangleleft_{\vv{c}} \tau_2 \rho$.
\end{proof}

\begin{proposition}\label{pr:pcf:context:relation-characterization}
  За произволни затворени термове $\tau_1$ и $\tau_2$,
  \[\tau_1 \leq_{ctx} \tau_2 : \vv{a} \iff \val{\tau_1} \triangleleft_{\vv{a}} \tau_2.\]
\end{proposition}
\begin{proof}
  $(\Rightarrow)$ Нека $\tau_1 \leq_{ctx} \tau_2 : \vv{a}$.
  От \Prop{pcf:adequacy:implication} имаме, че
  \[\val{\tau_1} \triangleleft_{\vv{a}} \tau_1.\]
  Тогава от предишното твърдение директно следва, че $\val{\tau_1} \leq_{\vv{a}} \tau_2$.

  $(\Leftarrow)$ Нека сега $\val{\tau_1} \triangleleft_{\vv{a}} \tau_2$.
  Тук ще използваме характеризацията на $\leq_{ctx}$ от \Prop{pcf:context:terms}.
  Да разгледаме произволен терм $\rho \in \text{PCF}_{\vv{a} \to \nat}$.
  Отново \Prop{pcf:adequacy:implication} имаме, че $\val{\rho} \triangleleft_{\vv{a} \to \nat} \rho$.
  Тогава от дефиницията на релацията $\triangleleft_{\vv{a}\to\nat}$ следва, че:
  \[\val{\rho\tau_1} = \val{\rho}(\val{\tau_1}) \triangleleft_{\nat} \rho\tau_2.\]
  Тогава:
  \begin{align*}
    \rho\tau_1 \Downarrow_{\nat} \vv{n} & \implies \val{\rho\tau_1} = n & \comment\text{\hyperref[th:pcf:soundness]{Теорема за коректност}}\\
                                            & \implies \rho\tau_2 \Downarrow_{\nat} \vv{n}. & \comment\val{\rho\tau_1} \triangleleft_{\nat} \rho\tau_2
  \end{align*}
  Заключаваме, че $\tau_1 \leq_{ctx} \tau_2 : \vv{a}$.
\end{proof}

\begin{proposition}\label{pr:pcf:context:extensionality}
  За произволни затворени термове $\tau_1$ и $\tau_2$ е изпълнено, че:
  \begin{enumerate}[(1)]
  \item
    $\tau_1 \leq_{ctx} \tau_2 : \nat \iff (\forall \vv{n})[\tau_1 \Downarrow_{\nat} \vv{n} \implies \tau_2 \Downarrow_{\nat} \vv{n}]$;
  \item
    $\tau_1 \leq_{ctx} \tau_2 : \vv{a}\to\vv{b} \iff (\forall \rho \in \text{PCF}_{\vv{a}})[\ \tau_1\rho \leq_{ctx} \tau_2 \rho : \vv{b}\ ]$.
  \end{enumerate}
\end{proposition}
\begin{proof}
  \begin{enumerate}[(1)]
  \item
    Посоката $(\Rightarrow)$ следва директно от \Prop{pcf:context:simple}.
    За посоката $(\Leftarrow)$, понеже
    \begin{align*}
      \val{\tau_1} = \val{\vv{n}} & \implies \tau_1 \Downarrow_{\nat} \vv{n} & \comment\text{\hyperref[th:pcf:adequacy]{Теорема за адекватност}}\\
                                  & \implies \tau_2 \Downarrow_{\nat} \vv{n}, & \comment\text{от условието}
    \end{align*}
    то получаваме, че $\val{\tau_1} \triangleleft_{\nat} \tau_2$.
    Тогава от предишното твърдение директно имаме, че $\tau_1 \leq_{ctx} \tau_2 : \nat$.
  \item
    Посоката $(\Rightarrow)$ следва директно от дефиницията на $\leq_{ctx}$.
    За посоката $(\Leftarrow)$, според предишното твърдение,
    достатъчно е да докажем, че $\val{\tau_1} \triangleleft_{\vv{a}\to\vv{b}} \tau_2$.
    Нека $u \in \val{\vv{a}}$ и $\rho \in \text{PCF}_{\vv{a}}$, за които $u \triangleleft_{\vv{a}} \rho$.
    Ще докажем, че $\val{\tau_1}(u) \triangleleft_{\vv{b}} \tau_2\rho$.
    Понеже $\val{\tau_1} \triangleleft_{\vv{a}\to\vv{b}} \tau_1$, то
    $\val{\tau_1}(u) \triangleleft_{\vv{b}} \tau_1\rho$.
    Но понеже $\tau_1\rho \leq_{ctx} \tau_2 \rho : \vv{b}$, то от \Prop{pcf:context:relation} следва, че
    $\val{\tau_1}(u) \triangleleft_{\vv{b}} \tau_2 \rho$.
  \end{enumerate}
\end{proof}

Естсвено е да се запитаме дали можем да разширим (1) на \Prop{pcf:context:extensionality} за по-сложни от $\nat$ типове $\vv{a}$, т.е. възможно ли е
\[\tau_1 \leq_{ctx} \tau_2 : \vv{a} \iff (\forall \vv{v} : \vv{a})[\tau_1 \Downarrow_{\vv{a}} \vv{v} \implies \tau_2 \Downarrow_{\vv{a}} \vv{v}]?\]
Първо в \Prop{context:op-left-right} ще видим, че винаги имаме импликацията $(\Leftarrow)$, но по-късно в \Prop{context:op-right-left} ще видим, че дори за типа $\vv{a} = \nat\to\nat$ нямаме импликация $(\Rightarrow)$.

\begin{proposition}\label{pr:context:op-left-right}
  Докажете, че за всеки два затворени терма $\tau_1, \tau_2 \in \text{PCF}_{\vv{a}}$,
  \marginpar{$(\forall \vv{v}:\vv{a})$ означава за всяка стойност $\vv{v}$ от тип $\vv{a}$.}
  \[(\forall \vv{v}:\vv{a})[\tau_1 \Downarrow_{\vv{a}} \vv{v} \implies \tau_2 \Downarrow_{\vv{a}} \vv{v} ] \implies \tau_1 \leq_{ctx} \tau_2 : \vv{a}.\]
\end{proposition}
\begin{hint}
  Според \Prop{pcf:context:relation-characterization}, достатъчно е да докажем, че $\val{\tau_1} \triangleleft_{\vv{a}} \tau_2$.
  Но това е лесно.
  От условието имаме, че $(\forall \vv{v}:\vv{a})[\tau_1 \Downarrow_{\vv{a}} \vv{v} \implies \tau_2 \Downarrow_{\vv{a}} \vv{v}]$.
  Знаем, че $\val{\tau_1} \triangleleft_{\vv{a}} \tau_1$.
  Тогава \Prop{pcf:adequacy:implication} следва, че $\val{\tau_1} \triangleleft_{\vv{a}} \tau_2$.
\end{hint}

\begin{proposition}\label{pr:context:den-left-right}
  За произволни затворени термове $\tau_1, \tau_2$ от тип $\vv{a}$,
  \[\val{\tau_1} \sqsubseteq \val{\tau_2} \implies \tau_1 \leq_{ctx} \tau_2 : \vv{a}.\]
\end{proposition}  
\begin{proof}
  Според \Prop{pcf:context:terms}, достатъчно е да докажем, че за произволен терм $\rho \in \text{PCF}_{\vv{a}\to\nat}$,
  $\rho\tau_1 \Downarrow_{\nat} \vv{n} \implies \rho\tau_2 \Downarrow_{\nat} \vv{n}$.
  \begin{align*}
    \rho\tau_1 \Downarrow_{\nat} \vv{n} & \implies \val{\rho\tau_1} = \val{\vv{n}} & \comment\text{\hyperref[th:pcf:soundness]{Теорема за коректност}}\\
                                            & \implies \val{\rho}(\val{\tau_1}) = \val{\vv{n}} & \comment\text{\hyperref[lem:pcf:substitution]{Лема за замяната}}\\
                                            & \implies \val{\rho}(\val{\tau_2}) = \val{\vv{n}} & \comment\text{монотонност на }\val{\rho}\\
                                            & \implies \val{\rho\tau_2} = \val{\vv{n}} & \comment\text{\hyperref[lem:pcf:substitution]{Лема за замяната}}\\
                                            & \implies \rho\tau_2 \Downarrow_{\nat} \vv{n}. & \comment\text{\hyperref[th:pcf:adequacy]{Теорема за адекватност}}
  \end{align*}
\end{proof}

\begin{framed}
  \begin{theorem}\label{th:pcf:context:connection}
    За всички затворени термове $\tau_1$ и $\tau_2$ от тип $\vv{a}$ е изпълнено, че:
    \begin{enumerate}[(1)]
    \item 
      $(\forall \vv{v}:\vv{a})[\tau_1 \Downarrow_{\vv{a}} \vv{v} \iff \tau_2 \Downarrow_{\vv{a}} \vv{v} ] \implies \tau_1 \cong_{ctx} \tau_2 : \vv{a}$;
    \item
      $\val{\tau_1} = \val{\tau_2} \implies \tau_1 \cong_{ctx} \tau_2 : \vv{a}$.
    \end{enumerate}
  \end{theorem}
\end{framed}

От \Prop{pcf:context:extensionality} и от \hyperref[th:pcf:soundness]{теоремата за коректност} имаме и обратните импликации за типа $\nat$.
За съжаление, ще видим, че нямаме обратните импликации за по-високи типове от $\nat$.


\begin{proposition}\label{pr:context:op-right-left}
  $\Omega'_{\nat} \cong_{ctx} \Omega_{\nat\to\nat} : \nat \to \nat$.
\end{proposition}
\begin{proof}
  Ще използваме (2) на \Prop{pcf:context:extensionality}.
  Първо да разгледаме посоката $(\Rightarrow)$. За произволно $\rho:\nat$, ще докажем, че $\Omega'_{\nat}\rho \leq_{ctx} \Omega_{\nat\to\nat}\rho : \nat$,
  което според (1) на \Prop{pcf:context:extensionality} означава да докажем, че
  \[(\forall \vv{n})[\ \Omega'_{\nat}\rho \opsem{}{nat} \vv{n} \implies \Omega_{\nat\to\nat} \rho \opsem{}{nat} \vv{n}\ ].\]
  Да отбележим, че $\Omega'_{\nat} \rho \not\opsem{}{nat}$, защото
  \begin{prooftree}
    \AxiomC{$\Omega'_{\nat}$ е стойност}
    \UnaryInfC{$\Omega'_{\nat} \opsem{0}{nat} \Omega'_{\nat}$}
    \AxiomC{$\Omega_{\nat} \not\opsem{}{nat}$}
    \UnaryInfC{$\Omega_{\nat}\subst{x}{\rho} \not\opsem{}{nat}$}
    \BinaryInfC{$\Omega'_{\nat} \rho \not\opsem{}{nat}$}
  \end{prooftree}
  Тогава заключаваме, че за всяко такова $\rho$,
  \[(\forall \vv{n})[\ \Omega'_{\nat}\rho \Downarrow_{\nat} \vv{n} \implies \Omega_{\nat\to\nat} \rho \Downarrow_{\nat} \vv{n}\ ].\]
  Сега да разгледаме посоката $(\Leftarrow)$.
  Тук правим сходни разсъждения, защото понеже $\Omega_{\nat\to\nat} \not\Downarrow_{\nat\to\nat}$, то и $\Omega_{\nat\to\nat} \rho \not\opsem{}{nat}$.
\end{proof}


% \begin{remark}
Знаем, че $\Omega_{\nat\to\nat} \not\opsemGen{}{\nat\to\nat}$, но $\Omega'_{\nat} \opsemGen{}{\nat\to\nat} \Omega'_{\nat}$.
Това означава, че според горната задача термовете $\Omega_{\nat\to\nat}$ и $\Omega'_{\nat}$ ни дават пример кога нямаме обратната импликация в (1) на \Th{pcf:context:connection}.
Обърнете внимание, че $\val{\Omega_{\nat\to\nat}} = \val{\Omega'_{\nat}}$.
Следователно, трябва да продължим да търсим термове $\tau_1$ и $\tau_2$ от тип $\vv{a}$, за които $\val{\tau_1} \neq \val{\tau_2}$
и $\tau_1 \cong_{ctx} \tau_2 : \vv{a}$.
% \end{remark}


%%% Local Variables:
%%% mode: latex
%%% TeX-master: "../sep"
%%% End:

\section{Пълна абстракция}\label{pcf:sect:full-abstraction}
\marginpar{Full abstraction на англ.}
\begin{definition}
  \marginpar{\cite[стр. 179]{gunter}}
  Денотационната семантика $\val{.}$ се нарича {\bf напълно абстрактна}, ако
  контекстната (операционната) и денотационната наредба съвпадат, т.е.
  за произволни термове $\tau_1,\tau_2$ от тип $\vv{a}$ е изпълнено, че
  \[\val{\tau_1} \sqsubseteq \val{\tau_2} \iff \tau_1 \leq_{ctx} \tau_2 : \vv{a}.\]
\end{definition}

\begin{framed}
  \begin{theorem}[Гордън Плоткин 1977]
    Денотационната семантика $\val{.}$ за езика PCF {\bf не е} напълно абстрактна.
  \end{theorem}
\end{framed}
Да напомним, че от \Th{pcf:context:connection} винаги имаме следното:
\[ \val{\tau_1} = \val{\tau_2} \implies \tau_1 \cong_{ctx} \tau_2 : \vv{a}.\]
Сега ще се захванем с търсенето на термове $\tau_1$ и $\tau_2$, за които
$\val{\tau_1} \neq \val{\tau_2}$ и $\tau_1 \cong \tau_2 : \vv{a}$.


\begin{problem}
  Да дефинираме функцията $sor:\Nat_\bot \to (\Nat_\bot \to \Nat_\bot)$ по следния начин:
  \marginpar{$sor$ идва от sequential or.}
  
  \begin{tabular}{|c|c|c|c|}
    \hline
    $sor$ & $\bot$ & $0$ & $y>0$ \\
    \hline
    $\bot$ & $\bot$ & $\bot$ & $\bot$\\
    \hline
    $0$ & $\bot$ & $0$ & $1$\\
    \hline
    $x>0$ & $1$ & $1$ & $1$\\
    \hline
  \end{tabular}
  
  Докажете, че $\texttt{sor}$ е определима в PCF.
\end{problem}
\begin{hint}
  Разгледайте затворения терм
  \[\tau \df \lamb{x}{nat}{\lamb{y}{nat}{\ifelse{\vv{x}}{\vv{1}}{\ifelse{\vv{y}}{\vv{1}}{\vv{0}}}}}\]
  Докажете, че $\val{\tau} = sor$.
\end{hint}


\begin{problem}

Да дефинираме изображението $por:\Nat_\bot\to(\Nat_\bot \to \Nat_\bot)$ по следния начин:

\begin{tabular}{|c|c|c|c|}
  \hline
  $por$ & $\bot$ & $0$ & $y>0$\\
  \hline
  $\bot$ & $\bot$ & $\bot$ & $1$\\
  \hline
  $0$ & $\bot$ & $0$ & $1$\\
  \hline
  $x>0$ & $1$ & $1$ & $1$\\
  \hline
\end{tabular}
\marginpar{$por$ идва от parallel or.}

  Докажете, че $por$ е непрекъснато изображение.
\end{problem}
\begin{hint}
  Достатъчно е да се съобрази, че $por$ е монотонно изображение.
\end{hint}

\begin{framed}
  \begin{lemma}[Гордън Плоткин 1977]
    Изображението $por$ не е определимо в PCF, т.е. не съществува затворен терм $\rho$,
    за който $\val{\rho} = por$.
  \end{lemma}
\end{framed}

\begin{example}
Да видим, че операторът ,,или'' в хаскел не е паралелен.
\begin{haskellcode}
ghci> True || undefined
True
ghci> undefined || True
*** Exception: Prelude.undefined
\end{haskellcode}
\end{example}

\begin{problem}\label{prob:pcf:full-abstraction:por}
  Да разгледаме $f \in \Cont{\Nat_\bot}{\Cont{\Nat_\bot}{\Nat_\bot}}$, за което имаме ограниченията:

  \begin{tabular}{|c|c|c|c|}
    \hline
    $f$ & $\bot$ & $0$ & $y>0$\\
    \hline
    $\bot$ & $?$ & $?$ & $1$\\
    \hline
    $0$ & $?$ & $0$ & $?$\\
    \hline
    $x>0$ & $1$ & $?$ & $?$\\
    \hline
  \end{tabular}

  Докажете, че $f = \texttt{por}$.
  
\end{problem}
\begin{hint}
  Използвайте монотонността на $f$.
\end{hint}

\begin{problem}\label{prob:pcf:full-abstraction:not-definable}
  Да разгледаме изображението $f \in \Cont{\Nat_\bot}{\Cont{\Nat_\bot}{\Nat_\bot}}$, за което
  \[f(0)(0) = 0\text{ и } f(1)(\bot) = f(\bot)(1) = 1.\]
  Докажете, че $f$ не е определимо в PCF.
\end{problem}
\begin{hint}
  Да допуснем, че $f$ е определима в PCF.
  Тогава $f = \val{\tau}$, за някой затворен терм $\tau : \nat \to \nat \to \nat$.

  За произволна променлива $\vv{z}$, да положим
  \[\rho_{\vv{z}} \df \ifelse{\vv{z == 0}}{\vv{0}}{\vv{1}}.\]
  Нека също положим
  \begin{align*}
    \tau' & \df \tau\rho_{\vv{x}};\\
    \tau'' & \df \tau'\rho_{\vv{y}}.
  \end{align*}
  Нека също така $\Gamma \df \type{x}{nat}$ и $\Delta = \type{y}{nat}$.
  Ясно, че $\val{\tau'}_\Gamma \in \Cont{\Nat_\bot}{\Cont{\Nat_\bot}{\Nat_\bot}}$ и
  \begin{align*} 
    \val{\tau'}_\Gamma(u)  & = \val{\tau\rho_{\vv{x}}}_\Gamma(u)\\
                       & = \texttt{eval}(\val{\tau}, \val{\rho_{\vv{x}}}_\Gamma(u))\\
                       & = \val{\tau}(\val{\rho_{\vv{x}}}_\Gamma(u))
  \end{align*}
  Нека $f' = \val{\tau'}_\Gamma$. Получаваме следното за $f'$.
  \[f'(u) = f(\val{\rho_{\vv{x}}}_\Gamma(u)) =
    \begin{cases}
      f(u), & \text{ако } u = \bot \text{ или } u = 0\\
      f(1), & \text{ако } u > 0.
    \end{cases}\]
  Аналогично, ясно е, че $\val{\tau''}_{\Gamma,\Delta} \in \Cont{\Nat_\bot\times\Nat_\bot}{\Nat_\bot}$ и
  \begin{align*} 
    \val{\tau''}_{\Gamma,\Delta}(u,v)  & = \val{\tau'\rho_{\vv{y}}}_{\Gamma,\Delta}(u,v)\\
                                  & = \texttt{eval}(\val{\tau'}_\Gamma(u), \val{\rho_{\vv{y}}}_\Delta(v))\\
                                  & = \val{\tau'}_\Gamma(u)(\val{\rho_{\vv{y}}}_\Delta(v))\\
                                  & = \val{\tau}(\val{\rho_{\vv{x}}}_\Gamma(u))(\val{\rho_{\vv{y}}}_\Delta(v)).
  \end{align*}
  Нека сега $f'' = \val{\tau''}_{\Gamma,\Delta}$. Тогава
  \begin{align*}
    f''(u,v) & = f'(u)(\val{\rho_{\vv{y}}}_\Delta(v))\\
             & = \begin{cases}
               f'(u)(v), & \text{ако } v = \bot\text{ или } v = 0\\
               f'(u)(1), & \text{ако } v > 0
             \end{cases}
  \end{align*}
  Така получаваме следната характеризация на $f''$:

  \begin{tabular}{|c|c|c|c|}
    \hline
    $f''$ & $\bot$ & $0$ & $y>0$ \\
    \hline
    $\bot$ & $?$ & $?$ & $1$ \\
    \hline
    $0$ & $?$ & $0$ & $?$ \\
    \hline
    $x>0$ & $1$ & $?$ & $?$\\
    \hline
  \end{tabular}

  Нека сега $\rho \df \lamb{x}{nat}{\lamb{y}{nat}{\tau''}}$.
  Тогава за $g = \val{\rho}$ имаме, че
  \[g(x)(y) = f''(x,y).\]
  От \Problem{pcf:full-abstraction:por} получаваме, че $g = por$.
  Достигаме до противоречие, защото $por$ не е определимо изображение.
\end{hint}

В следващите твърдения ще използваме типовете
\begin{align*}
  & \vv{a} \df \nat \to (\nat \to \nat)\\
  & \vv{b} \df (\nat \to (\nat \to \nat))\to\nat.
\end{align*}
За $n = 0,1$, нека дефинираме затворените термове

\begin{lstlisting}
  $\tau_n \equiv \lambda \vv{f:a}$.if (f 1 $\Omega_\nat$) == 1 then
              if (f $\Omega_\nat$ 1) == 1 then
                if (f 0 0) == 0 then n
                  else $\Omega_\nat$
                else $\Omega_\nat$
              else $\Omega_\nat$
\end{lstlisting}

Лесно се съобразява, че $\tau_0$ и $\tau_1$ са добре типизирани термове от тип $\vv{b}$.

\begin{problem}
  Докажете, че 
  \[\val{\tau_0} \neq \val{\tau_1}.\]
\end{problem}
\begin{hint}
  Докажете, че за $n = 0,1$ е изпълнено, че
  \[\val{\tau_n}(por) = n.\]  
\end{hint}

\begin{proposition}
  $\tau_1 \cong_{ctx} \tau_2 : \vv{b}$.
\end{proposition}
\begin{proof}
  Понеже $\vv{b} = \vv{a} \to \nat$, от \Prop{pcf:context:extensionality} следва, че е достатъчно да докажем, че
  за всеки затворен терм $\rho:\vv{a}$ е изпълнено, че
  \[\tau_1\rho \Downarrow_{\nat} \vv{n} \iff \tau_2\rho \Downarrow_{\nat} \vv{n}.\]
  Да видим какво означава $\tau_i \rho \Downarrow_{\nat}$ за $i = 0,1$.
  Това означава, че трябва да са изпълнени и трите свойства:
  \begin{itemize}
  \item
    $\rho\ \vv{1}\ \Omega_{\nat} \Downarrow_{\nat} \vv{1}$;% , за някое $\vv{k} \not\equiv \vv{0}$;
  \item
    $\rho\ \Omega_{\nat}\ \vv{1} \Downarrow_{\nat} \vv{1}$;% , за някое $\vv{m} \not\equiv \vv{0}$;
  \item
    $\rho\ \vv{0}\ \vv{0} \Downarrow_{\nat} \vv{0}$.
  \end{itemize}
  Понеже $\val{\Omega_{\nat}} = \bot$, от \hyperref[th:pcf:soundness]{теоремата за коректност} получаваме, че трябва да са изпълнени следните три свойства:
  \begin{itemize}
  \item
    $\val{\rho}(1)(\bot) = 1$;
  \item
    $\val{\rho}(\bot)(1) = 1$;
  \item
    $\val{\rho}(0)(0) = 0$.
  \end{itemize}
  Но тогава $\val{\rho} = por$, което е противоречие с \Problem{pcf:full-abstraction:not-definable}.
\end{proof}

Доказателството на следващата теорема излиза извън обхата на този курс.
\index{Плоткин}
\begin{framed}
  \begin{theorem}[Плоткин 1977]
    Денотационната семантика $\val{.}$ за езика PCF+\texttt{por} е напълно абстрактна.
  \end{theorem}
\end{framed}
\marginpar{\cite[стр. 188]{gunter}}

% \index{Плоткин}
% \index{Милнър}
% \begin{theorem}[Милнър,Плоткин]
%   A continuous, order-extensional model of PCF is fully abstract if and only if for every type $\sigma$, $\val{\sigma}$ is a domain whose finite elements are definable.
% \end{theorem}

%%% Local Variables:
%%% mode: latex
%%% TeX-master: "../sep"
%%% End:



%%% Local Variables:
%%% mode: latex
%%% TeX-master: "../sep"
%%% End:

\include{lambda-calculus/lambda-calculus}
\chapter{Доказване на свойства на програми}

След като вече имаме точна дефиниция на семантиката на една рекурсивна програма, можем да видим как можем да доказваме
формално свойства на рекурсивни програми.

\section{Свойства}
\begin{itemize}
\item 
  Да фиксираме една област на Скот $\A$. Подмножествата $P \subseteq \A$ ще наричаме {\bf свойства}.
\item
  \index{непрекъснато свойство}
  \marginpar{В \cite[стр. 166]{winskel} се наричат {\em inclusive subsets}. В \cite{bird-haskell} се наричат {\em chain-complete assertions}}
  Казваме, че $P$ е {\bf непрекъснато (или допустимо, индуктивно) свойство} над областта на Скот $\A$, ако за всяка верига $(a_i)^{\infty}_{i=0}$ от елементи на $\A$, е изпълнено:
  \begin{prooftree}
    \AxiomC{$P(a_0)$}
    \AxiomC{$P(a_1)$}
    \AxiomC{$P(a_2)$}
    \AxiomC{$P(a_3)$}
    \AxiomC{$\ldots$}
    \QuinaryInfC{$P(\bigsqcup_i a_i)$}
  \end{prooftree}
  % за която $P(a_i)$, то е изпълнено и $P(\bigsqcup_i a_i)$.
  \marginpar{Понеже $\A$ е област на Скот, ние знаем, че $\bigsqcup_i a_i$ съществува}
\end{itemize}

Нека да видим, че има свойства, който не са непрекъснати.

\begin{example}
  \label{ex:complement-not-inclusive}
  % Нека да разгледаме областта на Скот от точните изображения $\Strict{\Nat_\bot}{\Nat_\bot}$.
  Да разгледаме свойството $P \subseteq \Cont{\Nat_\bot}{\Nat_\bot}$, което е дефинирано по следния начин:
  \[P(f) \dfff (\exists x \in \Nat)[f(x) = \bot].\]
  Да разгледаме изображенията $f_i$, дефинирани по следния начин:
  \begin{align*}
    f_i(x) & =
    \begin{cases}
      42, & x \in \Nat\ \&\ x \leq i\\
      \bot, & x \in \Nat\ \&\ x > i\\
      \bot, & x = \bot.
    \end{cases}
  \end{align*}
  Лесно се съобразява, че $(f_i)^{\infty}_{i=0}$ е верига в $\Strict{\Nat_\bot}{\Nat_\bot}$ и че 
  за всяко $i$, $P(f_i)$. Да разгледаме точната горна граница $g$ на тази верига, за която знаем, че
  за всяко $y \in \Nat$,
  \[g(x) = y \iff (\exists i)[f_i(x) = y].\]
  Според конструкцията на $g$, $g(x) = 42$ за всяко $x \in \Nat$.
  Оттук директно получаваме, че $\neg P(g)$.
  Така видяхме, че $P$ {\em не е непрекъснато свойство}.
\end{example}

% \begin{example}
%   Нека да разгледаме областта на Скот $\S_1$, т.е. точните функции на един аргумент.
%   Да разгледаме свойството $P \subseteq \S_1$, което е дефинирано по следния начин:
%   \[P(f) \dfff (\exists x \in \Nat)[f(x) = \bot].\]
%   Да разгледаме функциите $f_i$, дефинирани по следния начин:
%   \begin{align*}
%     f_i(x) & =
%     \begin{cases}
%       42, & x \leq i\\
%       \bot, & x > i\\
%       \bot, & x = \bot.
%     \end{cases}
%   \end{align*}
%   Ясно е, че $P(f_i)$ е изпълнено за всяко $i$.
%   Да разгледаме функцията $g = \bigsqcup_i f_i$, която е дефинирана като:
%   \[g(x) = \bigsqcup_i\{f_i(x)\}.\]
%   Лесно се съобразява, че за всяко $x \in \Nat$, $g(x) \neq \bot$. Следователно, $\neg P(g)$.
%   Така видяхме, че $P$ не е непрекъснато свойство.
% \end{example}

\begin{problem}
  \label{prob:inclusive-property}
  Докажете, че свойството $Q$ над $\Cont{\Nat_\bot}{\Nat_\bot}$, където
  \[Q(f) \iff (\forall x \in \Nat)[f(x) \neq \bot],\]
  е непрекъснато.
\end{problem}

\subsection{Основни свойства}

\begin{proposition}
  \label{pr:continuous-property}
  Нека $\A$ и $\B$ са области на Скот и $f_1,f_2 \in \Cont{\A}{\B}$.
  Тогава следните свойства над $\A$ са непрекъснати:
  \begin{itemize}
  \item 
    $P(a) \dfff f_1(a) \sqsubseteq f_2(a)$;
  \item
    $P(a) \dfff f_1(a) = f_2(a)$;
  \end{itemize}
\end{proposition}

\begin{proposition}
  \label{pr:fixed-element-property}
  Нека $\A$ е област на Скот.
  Да фиксираме произволен елемент $a_0 \in \A$.
  Тогава следните свойства над $\A$ са непрекъснати:
  \begin{itemize}
  \item 
    $P(a) \dfff a \sqsubseteq a_0$;
  \item
    $P(a) \dfff a = a_0$;
  \end{itemize}
\end{proposition}

\subsection{Сечение}
\begin{proposition}
  Нека $P_1$ и $P_2$ са непрекъснати свойства над областта на Скот $\A$.
  Тогава $P_1 \cap P_2$ също е непрекъснато свойство.
\end{proposition}

\subsection{Обединение}

\begin{proposition}
  Нека $P_1$ и $P_2$ са непрекъснати свойства над областта на Скот $\A$.
  Тогава $P_1 \cup P_2$ също е непрекъснато свойство.
\end{proposition}

\subsection{Допълнение}

\begin{proposition}
  Съществува непрекъснато свойство $P$ над областта на Скот $\A$,
  за което $\A \setminus P$ {\bf не} е непрекъснато свойство.
\end{proposition}
\begin{proof}
  Да вземем $\A = \Cont{\Nat_\bot}{\Nat_\bot}$.
  Свойството $Q$ от \Problem{inclusive-property} е непрекъснато, 
  докато $\Cont{\Nat_\bot}{\Nat_\bot} \setminus Q = P$, което е точно свойството от \Ex{complement-not-inclusive}, а 
  за него знаем, че не е непрекъснато.
\end{proof}

% \subsection{Образ}

% \begin{problem}
%   Нека $P$ е непрекъснато свойство в областта на Скот $\B$.
%   Нека $f \in \Cont{\A}{\B}$.
%   Да разгледаме свойството 
%   \[f[P] \dff \{f(a) \mid f(a) \in P\}.\]
%   Не винаги $f[P]$ е непрекъснато свойство в $\A$.
% \end{problem}
% \ifhints
% \begin{hint}
%   Нека $(b_n)^\infty_{n=0}$ е верига в $\B$,
%   като $\bigsqcup_n b_n$ не е елемент на веригата.
%   Нека дефинираме изображението $f$ по следния начин:
%   $f(\bot) \dff \bot$ и $f(n) \dff b_n$.
%   Лесно се съобразява, че $f \in \Cont{\Nat_\bot}{\B}$.
%   Нека $P = \Nat$, което очевидно е непрекъснато свойство, защото елементите на $\Nat$ 
%   не са сравними относно плоската наредба.
%   Тогава $f[P] = \{b_n \mid n \in \Nat\}$.
%   Лесно се проверява, че $f[P]$ не е непрекъснато свойство.
% \end{hint}
% \fi

% \subsection{Първообраз}

% \begin{problem}
%   Нека $P$ е непрекъснато свойство в областта на Скот $\B$.
%   Нека $f \in \Cont{\A}{\B}$.
%   Да разгледаме свойството 
%   \[f^{-1}[P] \dff \{a \in \A \mid f(a) \in P\}.\]
%   Докажете, че $f^{-1}[P]$ е непрекъснато свойство в $\A$.
% \end{problem}

% \subsection{Композиция}

% \begin{problem}
%   Нека $P$ е непрекъснато свойство в областта на Скот $\A \times \B$,
%   а $Q$ е непрекъснато свойство в областта на Скот $\B \times \C$.
%   Композицията 
%   \[Q \circ P = \{\pair{a,c} \in \A\times \C \mid (\exists b\in\B)[\pair{a,b} \in P\ \&\ \pair{b,c} \in Q]\}.\]
%   Докажете, че {\bf не винаги} $Q \circ P$ е непрекъснато свойство.
% \end{problem}
% \ifhints
% \begin{hint}
%   Нека $(a_n)^\infty_{n=0}$ е верига в $\A$, а $(c_n)^{\infty}_{n=0}$ е верига в $\C$,
%   като и двете вериги са такива, че $\bigsqcup_n a_n$ не е елемент на $(a_n)^{\infty}_{n=0}$
%   и $\bigsqcup_n c_n$ не е елемент на $(c_n)^\infty_{n=0}$.
%   Нека $\B = \Nat_\bot$.  
%   Тогава дефинираме свойствата по следния начин:
%   \begin{align*}
%     & P \dff \{\pair{a_n,n} \mid n \in \Nat\};\\
%     & Q \dff \{\pair{n,c_n} \mid n \in \Nat\}.
%   \end{align*}
%   Лесно се проверява, че тези свойства са непрекъснати.
%   Тогава,
%   \[Q \circ P = \{\pair{a_n,c_n} \mid n \in \Nat\},\]
%   което очевидно не е непрекъснато свойство.
% \end{hint}
% \fi

% \subsection{Проекции}

% \begin{problem}
%   Нека $P$ е непрекъснато свойство в областта на Скот $\A \times \B$.
%   Нека за произволно $a$ да дефинираме свойството 
%   \[P_a \dff \{b \in \B \mid \pair{a,b} \in P\}.\]
%   Тогава $P_a$ е непрекъснато свойство.
%   Наричаме $P_a$ проекция на $P$ по първата компонента.
% \end{problem}

% Ние знаем, че едно изображение $f:\A\times \B \to \C$ е непрекъснато точно тогава, когато
% $f$ е непрекъснато по всеки от аргументите си.
% Ако $P$ е непрекъснато свойство в $\A\times\B$, то е ясно, че $P$ е непрекъснато по всяка от проекциите си.

% \begin{problem}
%   Да разгледаме свойството $P$ в $\A \times \B$, за което 
%   имаме, че $P$ е непрекъснато свойство по всяка от проекциите.
%   Вярно ли е, че тогава $P$ е непрекъснатото?
% \end{problem}
% \ifhints
% \begin{hint}
%   Нека $\A$ и $\B$ са такива области на Скот, в които има вериги съответно $(a_n)^\infty_{n=0}$ и $(b_n)^\infty_{n=0}$,
%   за които $\bigsqcup_n a_n$ и $\bigsqcup_n a_n$ не са елементи на съответните вериги.
%   Нека $P = \{\pair{a_n,b_n} \mid n \in \Nat\}$.
%   Тогава за всяко $n$, $P_{a_n}$ и $P_{b_n}$ са непрекъснати, защото
%   $P_{a_n} = \{b_n\}$ и $P_{b_n} = \{a_n\}$,
%   но е очевидно, че $P$ не е непрекъснато свойство.
% \end{hint}
% \fi

% \subsection{Универсално затваряне}

% \begin{problem}
%   Нека $P$ е непрекъснато свойство в областта на Скот $\A \times \B$.
%   Нека за произволно $a$ да дефинираме свойството 
%   \[Q \dff \{b \in \B \mid (\forall a \in \A)[\pair{a,b} \in P]\}.\]
%   Вярно ли е, че $Q$ е непрекъснато свойство?
%   Обосновете отговора си!
% \end{problem}
% \ifhints
% \begin{hint}
%   Вярно е.
% \end{hint}
% \fi
% \ifhints
% \begin{hint}
%   Нека $\A$ и $\B$ са такива области на Скот, в които има вериги съответно $(a_n)^\infty_{n=0}$ и $(b_n)^\infty_{n=0}$,
%   за които $\bigsqcup_n a_n$ и $\bigsqcup_n a_n$ не са елементи на съответните вериги.
%   Нека 
%   \[P \dff \{\pair{a_n,b_k} \mid n,k \in \Nat\} \cup \{\pair{\bigsqcup_n a_n, b_k} \mid k \in \Nat\} \cup \{\pair{\bigsqcup_n a_n, \bigsqcup_k b_k}\}.\]
%   Тогава $Q = \{a_n \mid n \in \Nat\}$.
% \end{hint}
% \fi

% \subsection{Екзистенциално затваряне}

% \begin{problem}
%   Нека $P$ е непрекъснато свойство в областта на Скот $\A \times \B$.
%   Нека за произволно $a$ да дефинираме свойството 
%   \[Q \dff \{b \in \B \mid (\exists a \in \A)[\pair{a,b} \in P]\}.\]
%   Вярно ли е, че $Q$ е непрекъснато свойство?
% \end{problem}
% \ifhints
% \begin{hint}
%   Разгледайте $\A = \B = \Int \cup\{-\infty,+\infty\}$.
%   Нека 
%   \[P = \{\pair{-n,n} \mid n \in \Nat\}.\]
%   Тогава $P$ е непрекъснато свойство.
%   Но тогава $Q = \{n \mid n \in \Nat\}$ не е непрекъснато свойство.
% \end{hint}
% \fi

% \subsection{Непрекъснати изображения}

% Знаем, че ако $\A$ и $\B$ са области на Скот, то съвкупността от всички непрекъснати изображения $\Cont{\A}{\B}$ образува
% област на Скот. Сега ще разгледаме аналог на тази теорема за непрекъснати свойства.

% \begin{proposition}
%   Нека $P$ и $Q$ са свойства съответно в $\A$ и $\B$.
%   Да разгледаме свойството $R$ в $\Cont{\A}{\B}$ дефинирано като:
%   \[R \dff \{f \in \Cont{\A}{\B} \mid (\forall a \in \A)[P(a) \implies Q(f(a))]\}.\]
%   Докажете, че ако $Q$ е непрекъснато свойство, то $R$ е непрекъснато свойство.
% \end{proposition}

\subsection{Частична коректност}

\begin{itemize}
\item
  Да разгледаме едно свойство $I$ в областта на Скот $\A$, което наричаме условие за входа, и
  свойство $O$ в областта на Скот $\A \times \B$, което наричаме условие за изхода.
\item
  \index{частична коректност}
  {\bf Свойство от тип частична коректност} относно $I$ и $O$ представлява 
  свойство $P \subseteq \Mapping{\A}{\B}$ със следната дефиниция
  \[P(f) \dfff (\forall a \in \A)[\ I(a)\ \&\ f(a) \neq \bot \implies O(a,f(a))\ ].\]
\end{itemize}

\begin{proposition}
  Нека $I \subseteq \A$, а $O$ е непрекъснато свойство в $\A \times \B$.
  \[P(f) \dfff (\forall a \in \A)[\ I(a)\ \&\ f(a) \neq \bot \implies O(a,f(a))\ ].\]
  Тогава свойството $P$ е непрекъснато в областта на Скот $\Mapping{\A}{\B}$.
\end{proposition}

% \begin{example}
%   Нека $\A = \B = \Nat_\bot$ и $I(x) \dfff x > 1$, а $O(x,y) \dfff x,y\in\Nat\ \&\ x = y^2$.
%   Ясно е, че $O$ е непрекъснато свойство в $\Nat^2_\bot$. Да разгледаме свойството
%   \[P(f) \dfff (\forall x \in \Nat_\bot)[x > 1\ \&\ f(x) \neq \bot \implies O(x,f(x))].\]
%   Знаем, че $P$ е от тип частична коректно и следователно е непрекъснато в областта на Скот 
%   $\Mapping{\Nat^2_\bot}{\Nat_\bot}$.
%   Понеже $\Sigma^\star:\Cont{\Partial{\Nat^2}{\Nat}}{\Strict{\Nat^2_\bot}{\Nat_\bot}}$, то
%   $Q \dfff (\Sigma^\star)^{-1}[P]$ е непрекъснато свойство в областта на Скот $\Partial{\Nat^2}{\Nat}$.
%   Ясно е, че
%   \[Q(f) \equiv (\forall x \in \Nat)[x > 1\ \&\ !f(x) \implies O(x,f(x))].\]
% \end{example}


% \subsection{Тотална коректност}

% \index{тотална коректност}
% {\bf Свойство от тип тотална коректност} относно $I$ и $O$ представлява 
% свойство $P \subseteq \Mapping{\A}{\B}$ със следната дефиниция
% \[P(f) \dfff (\forall a \in \A)[I(a) \implies (f(a) \neq \bot\ \&\  O(a,f(a)))].\]

% \begin{proposition}
%   Нека $I$ е свойство в $\A$, а $O$ е непрекъснато свойство в $\A \times \B$.
%   Тогава свойството
%   \[P(f) \dfff (\forall a \in \A)[I(a) \implies (f(a) \neq \bot\ \&\ O(a,f(a)))]\]
%   е непрекъснато в $\Mapping{\A}{\B}$.
% \end{proposition}

%%% Local Variables:
%%% mode: latex
%%% TeX-master: "../sep"
%%% End:


% \newpage
% \setcounter{problem}{0}

% \begin{problem}
%   Нека е даден следния оператор $\Gamma:\F_1\to\F_1$:
%   \begin{align*}
%     \Gamma(f)(x) \simeq 
%     \begin{cases}
%       0, & f\text{ е крайна функция}\\
%       1, & \text{ иначе }.
%     \end{cases}
%   \end{align*}
%   Проверете дали:
%   \begin{enumerate}[a)]
%   \item 
%     $\Gamma$ е монотонен оператор;
%   \item
%     $\Gamma$ е компактен оператор.
%   \end{enumerate}
% \end{problem}
% \begin{proof}
%   \begin{enumerate}[a)]
%   \item 
%     Трябва да проверим дали:
%     \[(\forall f,g\in\F_1)[f \subseteq g\ \Rightarrow \Gamma(f) \subseteq \Gamma(g)].\]
%     Да изберем $f \subseteq g$ да бъдат такива функции, че $f$ е крайна функция, но $g$ не е крайна функция.
%     Тогава $(\forall x \in \Nat)[\Gamma(f)(x) \simeq 0\ \&\ \Gamma(g)(x) \simeq 1]$.
%     Очевидно е, че за така избраните $f$ и $g$, $\Gamma(f) \not\subseteq \Gamma(g)$.
%   \item
%     Знаем, че всеки компактен оператор е монотонен.
%     Щом $\Gamma$ не е монотонен, то със сигурност $\Gamma$ не е компактен.
%   \end{enumerate}
% \end{proof}

% \begin{problem}
%   Нека е даден следния оператор $\Gamma:\F_1\to\F_1$:
%   \begin{align*}
%     \Gamma(f)(x) \simeq 
%     \begin{cases}
%       \neg !, & f\text{ е крайна функция}\\
%       1, & \text{ иначе }.
%     \end{cases}
%   \end{align*}
%   Проверете дали:
%   \begin{enumerate}[a)]
%   \item 
%     $\Gamma$ е монотонен оператор;
%   \item
%     $\Gamma$ е компактен оператор.
%   \end{enumerate}
% \end{problem}
% \begin{proof}
%   \begin{enumerate}[a)]
%   \item 
%     Трябва да проверим дали:
%     \[(\forall f,g\in\F_1)[f \subseteq g\ \Rightarrow \Gamma(f) \subseteq \Gamma(g)].\]
%     Нека $f \subseteq g$ са произволни функции от $\F_1$.
%     Ще разгледаме два случая.
%     \begin{itemize}
%     \item 
%       $f$ е крайна функция. Тогава $\Gamma(f) = \emptyset^{(1)}$ и е очевидно, че 
%       \[\Gamma(f) = \emptyset^{(1)} \subseteq \Gamma(g).\]
%     \item
%       $f$ не е крайна функция. Щом $f \subseteq g$, то $g$ също не е крайна функция.
%       Тогава 
%       \[(\forall x \in \Nat)[\Gamma(f)(x) \simeq 1 \simeq \Gamma(g)(x)],\]
%       от което следва, че 
%       \[\Gamma(f) \subseteq \Gamma(g).\]
%     \end{itemize}
%     Разгледахме всички възможни случаи за $f$ и във всеки от тях получихме, че $\Gamma(f) \subseteq \Gamma(g)$.
%     Следователно, $\Gamma$ е монотонен оператор.
%   \item
%     Сега ще проверим дали е изпълнено, че:
%     \begin{equation}
%       \label{eq:compact}
%       (\forall f \in \F_1)(\forall x,y\in\Nat)[\Gamma(f)(x)\simeq y\ \Rightarrow\ (\exists \theta \subseteq f)[\Gamma(\theta)(x) \simeq y]].
%     \end{equation}
%     Тук с $\theta$ означаваме крайна функция.
%     Нека $f$ е не е крайна функция.% , например, $f(x) = x^2$ за всяко $x \in \Nat$.
%     Тогава е ясно, че за всяко $x \in \Nat$, $\Gamma(f)(x) \simeq 1$.
%     От друга страна, понеже $\theta$ е крайна, $\neg ! \Gamma(\theta)(x)$ за всяко $x \in \Nat$.
    
%     Така видяхме, че ако $f$ не е крайна, то за произволно $x$, $\Gamma(f)(x) \simeq 1$,
%     но не съществува крайна $\theta \subseteq f$, за която $\Gamma(\theta)(x) \simeq 1$.
%     От това следва, че Формула (\ref{eq:compact}) не е изпълнена.
%     Следователно, $\Gamma$ не е компактен оператор.
%   \end{enumerate}
% \end{proof}


% \begin{problem}
%   Нека е даден следния оператор $\Gamma:\F_2\to\F_2$:
%   \begin{align*}
%     \Gamma(f)(x,y) \simeq &
%     \begin{cases}
%       y, & x = 0\\
%       f(x, f(x-1,y)), & x > 0.
%     \end{cases}
%   \end{align*}
%   \begin{enumerate}[a)]
%   \item 
%     Докажете, че $\Gamma$ е компактен оператор.
%   \item
%     Коя е най-малката неподвижна точка на $\Gamma$?
%   \item
%     Има ли $\Gamma$ други неподвижни точки ?
%   \end{enumerate}
% \end{problem}


% \begin{problem}
%   Нека е даден следния оператор $\Gamma:\F_1\to\F_1$:
%   \begin{align*}
%     \Gamma(f)(x) \simeq &
%     \begin{cases}
%       0, & x = 0\\
%       f(f(x-1)+1)), & x > 0.
%     \end{cases}
%   \end{align*}
%   \begin{enumerate}[a)]
%   \item 
%     Докажете, че $\Gamma$ е компактен оператор.
%   \item
%     Коя е най-малката неподвижна точка на $\Gamma$?
%   \item
%     Има ли $\Gamma$ други неподвижни точки ?
%   \end{enumerate}
% \end{problem}

% \section{Структурна индукция}

% % \Stefan{Да се махне оттук.}
% \begin{problem}
%   Да разгледаме програмите $\texttt{fib}$ и $\texttt{fib'}$:
  
%   \begin{haskellcode}
% fib(n) = f(n) where
%   f(n) = if n == 0 then 0
%            else if n == 1 then 1
%              else f(n-1) + f(n-2)

% fib'(n) = g(0,1,n) where
%   g(a,b,n) = if n == 0 then a
%                else g(b, a+b, n-1)
%   \end{haskellcode}
  
%   Докажете, че $\D_V\val{\texttt{fib}} = \D_V\val{\texttt{fib'}}$.
% \end{problem}
% \begin{hint}
%   Да разгледаме операторите:
%   \begin{align*}
%     \Gamma(f)(x) =
%     \begin{cases}
%       0, & x = 0\\
%       1, & x = 1\\
%       f(x-1) + f(x-2), & x \geq 2\\
%       \bot, & x = \bot.
%     \end{cases}
%   \end{align*}

%   \begin{align*}
%     \Delta(g)(x,y,z) =
%     \begin{cases}
%       x, & z = 0\\
%       g(y,x+y,z-1), & z \geq 1\\
%       \bot, & z = \bot.
%     \end{cases}
%   \end{align*}
%   \Stefan{Малко е объркващо, че от една страна работим с $\Nat_\bot$, където имаме плоска наредба, а 
%   от друга правим индукция по наредбата на естествените числа}

%   Очевидно е, че тези оператори са непрекъснати.
%   Нека $\gamma$ е най-малката неподвижна точка на $\Gamma$ и
%   $\delta$ е най-малката неподвижна точка на $\Delta$.

%   Докажете, че с индукция по $n \in \Nat$, че 
%   \[(\forall i \in \Nat)[\delta(\gamma(i), \gamma(i+1), n) = \gamma(n+i)].\]
% \end{hint}

\section{Правило на Скот}
\marginpar{\cite[стр. 166]{winskel}}
\index{правило на Скот}
\begin{itemize}
\item 
  Нека $\A$ е област на Скот и нека $f \in \Cont{\A}{\A}$.
\item
  Всяко $P \subseteq A$ ще наричаме свойство.
\item
  \marginpar{С $\texttt{lfp}(f)$ означаваме най-малката неподвижна точка на $f$ (от англ. least fixed point)}
  Тогава {\bf правилото на Скот} гласи следното:
  \begin{prooftree}
  \AxiomC{$P(\bot)$}
  \AxiomC{$(\forall a \in \A)[P(a) \implies P(f(a))]$}
  \BinaryInfC{$P(\texttt{lfp}(f))$}
\end{prooftree}
\end{itemize}
\begin{proof}
  ....
\end{proof}

\begin{problem}
  \marginpar{Сравнете с \Prop{prefix-point}}
  Нека $f \in \Cont{\A}{\A}$.
  Да означим множеството от преднеподвижни точки на $f$ като
  \[\texttt{Pref}(f) \df \{a \in \A \mid f(a) \sqsubseteq a\}.\]
  Тогава 
  \[(\forall a \in \A)[a \in \texttt{Pref}(f) \implies \lfp(f) \sqsubseteq a].\]
\end{problem}
\begin{proof}
  Да фиксираме елемент $a \in \texttt{Pref}(f)$.
  Да разгледаме непрекъснатото свойство
  \marginpar{Сами се убедете, че $P$ е непрекъснато свойство!}
  \[P(b) \df b \sqsubseteq a.\]
  Ясно е, че $P(\bot)$.
  Нека $b\in \A$, за който $P(b)$. Ще докажем, че $P(f(b))$.
  \begin{align*}
    b \sqsubseteq a & \implies f(b) \sqsubseteq f(a) & \comment{f \text{ е мон.}}\\
    & \implies f(b) \sqsubseteq f(a) \sqsubseteq a & \comment{a \in \texttt{Pref}(f)}\\
    & \implies f(b) \sqsubseteq a & \comment{\sqsubseteq \text{ е транз. рел.}}.
  \end{align*}
  От правилото на Скот, заключаваме, че $P(\lfp(f))$, т.е.
  $\lfp(f) \sqsubseteq a$.
\end{proof}
  
\begin{problem}
  Нека $f,g \in \Cont{\A}{\A}$ като имаме свойствата:
  \begin{itemize}
  \item
    $f(\bot) \sqsubseteq g(\bot)$;
  \item
    $f \circ g \sqsubseteq g \circ f$.
  \end{itemize}
  Докажете, че $\lfp(f) \sqsubseteq \lfp(g)$.
\end{problem}
\begin{proof}
  Разгледайте непрекъснатото свойството 
  \[P(a) \df f(a) \sqsubseteq g(a).\]
  От условието имаме, че $P(\bot)$.
  Нека $P(a)$. Ще докажем, че $P(g(a))$.
  \begin{align*}
    P(a) & \iff f(a) \sqsubseteq g(a)\\
         & \implies g(f(a)) \sqsubseteq g(g(a)) & \comment{g \text{ е мон.}}\\
         & \implies f(g(a)) \sqsubseteq g(g(a)) & \comment{ f\circ g \sqsubseteq g\circ f}\\
         & \iff P(g(a)).
  \end{align*}
  Тогава по правилото на Скот ще заключим, че $P(\lfp(g))$, откъдето
  \[f(\lfp(g)) \sqsubseteq g(\lfp(g)) = \lfp(g).\]
  Това означава, че $\lfp(g)$ е преднеподвижна точка на $f$, т.е.
  \[\lfp(g) \in \texttt{Pref}(f).\]
  Понеже $\lfp(f)$ е най-малката преднеподвижна точка на $f$,
  то заключаваме, че $\lfp(f) \sqsubseteq \lfp(g)$.
\end{proof}

\begin{problem}
  Нека $h \in \Cont{\A}{\B}$, $f \in \Cont{\A}{\A}$, $g \in \Cont{\B}{\B}$,
  като имаме свойствата:
  \begin{itemize}
  \item 
    $h$ е точна, т.е. $h(\bot^\A) = \bot^\B$;
  \item
    $g\circ h = h \circ f$.
  \end{itemize}
  Докажете, че $\lfp(g) = h(\lfp(f))$.
\end{problem}
\ifhints
\begin{hint}
  \begin{itemize}
  \item 
    Разгледайте непрекъснатото свойство 
    \[P_1(a) \df h(a) \sqsubseteq g(h(a)).\]
    Докажете с правилото на Скот, че $P_1(\lfp(f))$.
    Тогава
    \begin{align*}
      h(\lfp(f)) \sqsubseteq g(h(\lfp(f)) & \iff h(f(\lfp(f))) \sqsubseteq g(h(\lfp(f)\\
                                          & \iff h(f(\lfp(f))) \sqsubseteq h(f(\lfp(f)))\\
                                          & \iff g(h(\lfp(f))) \sqsubseteq h(\lfp(f)).
    \end{align*}
    Това означава, че $h(\lfp(f))$ е преднеподвижна точка на $g$, т.е.
    \[h(\lfp(f)) \in \texttt{Pref}(g).\]
    Заключаваме, че $\lfp(g) \sqsubseteq h(\lfp(f))$.
  \item
    Другата посока е по-лесна. Разгледайте непрекъснатото свойство
    \[P_2(a) \df h(a) \sqsubseteq \lfp(g).\]
    Докажете, че $P_2(\lfp(f))$.
  \end{itemize}
\end{hint}
\fi

\begin{problem}
  Нека $f,g \in \Cont{\A}{\A}$ като имаме свойствата:
  \begin{itemize}
  \item
    $f(\bot) = g(\bot)$;
  \item
    $f \circ g = g \circ f$.
  \end{itemize}
  Докажете, че $\lfp(f \circ g) = \lfp(g \circ f)$.
\end{problem}
\ifhints
\begin{hint}
  Разгледайте непрекъснатото свойство
  \marginpar{Лесно се вижда, че $P$ е непрекъснато свойство, защото $f$ и $g$ са непрекъснати изображения.}
  \[P(a) \df g(f(a)) \sqsubseteq a.\]
  Ясно е, че $P(\bot)$.
  Нека $P(a)$. Ще докажем, че $P(f(g(a))$.
  \begin{align*}
    g(f(a)) \sqsubseteq a & \implies g(f(g(f(a)))) \sqsubseteq g(f(a))\\
    & \implies g(f(f(g(a)))) \sqsubseteq f(g(a))\\
    & \implies P(f(g(a))).
  \end{align*}
  От правилото на Скот заключваме, че $P(\lfp(f\circ g))$.
  Това означава, че 
  \[g(f(\lfp(f\circ g))) \sqsubseteq \lfp(f\circ g),\] т.е.
  $\lfp(f\circ g) \in \texttt{Pref}(g \circ f)$.
  Следователно,
  \[\lfp(g \circ f) \sqsubseteq \lfp(f\circ g).\]

  За другата посока разсъждаваме аналогично.
\end{hint}
\fi


\begin{problem}
  Нека $p \in \Cont{\A}{\Nat_\bot}$ и $h \in \Cont{\A}{\A}$, като $h$ е точна, т.е. $h(\bot) = \bot$.
  Да разгледаме 
  \[\Gamma \in \Cont{\Cont{\A\times\A}{\A}}{\Cont{\A\times\A}{\A}},\]
  където
  \begin{align*}
    \Gamma(f)(x,y) =
    \begin{cases}
      y, & p(x) = 0\\
      h(f(h(x),y)), & p(x) \in \Nat^+\\
      \bot, & p(x) = \bot.
    \end{cases}
  \end{align*}
  Докажете, че ако $f_\Gamma \df \lfp(\Gamma)$, то
  \[(\forall a,b\in\A)[h(f_\Gamma(a,b)) = f_\Gamma(a,h(b))].\]
\end{problem}
\ifhints
\begin{hint}
  Разгледайте непрекъснатото свойство
  \[P(g) \df (\forall a,b\in\A)[h(g(a,b)) = g(a,h(b))].\]
\end{hint}
\fi

\begin{problem}
  Нека $p \in \Cont{\A}{\Nat_\bot}$ и $h \in \Cont{\A}{\A}$, като $p$ е точна, т.е. $p(\bot) = \bot$.
  Да разгледаме 
  \[\Gamma \in \Cont{\Cont{\A}{\A}}{\Cont{\A}{\A}},\] 
  където:
  \begin{align*}
    \Gamma(f)(x) =
    \begin{cases}
      x, & p(x) = 0\\
      f(f(h(x))), & p(x) \in \Nat^+\\
      \bot, & p(x) = \bot.
    \end{cases}
  \end{align*}
  Докажете, че ако $f_\Gamma \df \lfp(\Gamma)$, то
  \[(\forall a\in\A)[f_\Gamma(f_\Gamma(a)) = f_\Gamma(a)].\]
\end{problem}
\ifhints
\begin{hint}
  Разгледайте непрекъснатото свойство
  \[P(g) \df (\forall a \in \A)[f_\Gamma(g(a)) = g(a)].\]
\end{hint}
\fi

\begin{problem}
  Нека $p \in \Cont{\A}{\Nat_\bot}$ и $h,k \in \Cont{\A}{\A}$, като $h$ е точна, т.е. $h(\bot) = \bot$.
  Да разгледаме $\Gamma_{1,2} \in \Cont{\Cont{\A\times\A}{\A}}{\Cont{\A\times\A}{\A}}$, където:
  \begin{align*}
    & \Gamma_1(f)(x,y) =
    \begin{cases}
      y, & p(x) = 0\\
      h(f(k(x),y)), & p(x) \in \Nat^+\\
      \bot, & p(x) = \bot;\\
    \end{cases}\\
   & \Gamma_2(f)(x,y) =
    \begin{cases}
      y, & p(x) = 0\\
      f(k(x),h(y)), & p(x) \in \Nat^+\\
      \bot, & p(x) = \bot;
    \end{cases}
  \end{align*}
  Докажете, че ако $f_1 \df \lfp(\Gamma_1)$ и $f_2 = \lfp(\Gamma_2)$, то
  $f_1 = f_2$.
\end{problem}
\ifhints
\begin{hint}
  Разгледайте непрекъснатото изображение $\Delta$, където
  \[\Delta(f,g) = \pair{\Gamma_1(f),\Gamma_2(g)}.\]
  Разгледайте свойството:
  \[P(f,g) \dff f = g\ \&\ (\forall a,b \in \A)[h(f(a,b))) = f(a,h(b))].\]
  Първо трябва да се съобрази, че това свойство е непрекъснато, което не е трудно.
  Ясно е, че $P(\bot,\bot)$.
  Докажете, че $P(f,g) \implies P(\Delta(f,g))$.
\end{hint}
\fi


%%% Local Variables:
%%% mode: latex
%%% TeX-master: "../sep"
%%% End:



% \begin{problem}
%   Операторът $\Gamma:\mathcal{F}_2\rightarrow \mathcal{F}_2$ действа по правилото:
%   \begin{equation*}
%     \Gamma(f)(x,y)\simeq 
%     \begin{cases} 
%       1, & \text{ ако } x + y \text{ е просто},\\
%       f(x+y,y)+1, & \text{ иначе.}
%     \end{cases}
%   \end{equation*}
%   Да се докаже, че:
%   \begin{enumerate}[a)]
%   \item
%     операторът $\Gamma$ е компактен.
%   \item 
%     ако $f_{\Gamma}$ е най-малката неподвижна точка на $\Gamma$, то:
%     \begin{equation*}
%       (\forall x,y \in \Nat)[!f_{\Gamma}(x,y) \Rightarrow x + y.f_\Gamma(x,y) \text{ е просто}].
%     \end{equation*}
%   \end{enumerate}
% \end{problem}
% \begin{solution}
%   \begin{enumerate}[a)]
%   \item
%     Да се убедим, че $\Gamma$ е компактен.
%     \begin{itemize}
%     \item 
%       Първо да видим, че $\Gamma$ е монотонен.
%       Нека $f \subseteq g$. Ще докажем, че $\Gamma(f) \subseteq \Gamma(g)$, т.е.
%       \[(\forall x,y,z\in\Nat)[\Gamma(f)(x,y) \simeq z\ \rightarrow\ \Gamma(g)(x,y) \simeq z].\]
%       И така, нека $\Gamma(f)(x,y) \simeq z$. Гледайки дефиницията на $\Gamma$, трябва да разгледаме два случая:
%       \begin{itemize}
%       \item 
%         ако $x+y$ е просто число, то по дефиницията на $\Gamma$,
%         \[\Gamma(f)(x,y) \simeq 1 \simeq \Gamma(g)(x,y).\]
%       \item
%         ако $x+y$ не е просто число, то 
%         \[\Gamma(f)(x,y) \simeq f(x+y,y)+1 \simeq z.\]
%         Това означава, че съществува число $u$, такова че $f(x+y,y) \simeq u$ и $z = u+1$.
%         Понеже $f \subseteq g$, то $g(x+y,y) \simeq u$.
%         Тогава 
%         \begin{align*}
%           \Gamma(g)(x,y) & \simeq g(x+y,y) + 1 & (\text{от деф. на }\Gamma)\\
%           & \simeq u+1 & (\text{защото }f \subseteq g)\\
%           & \simeq z.
%         \end{align*}
%       \end{itemize}
%       За всички възможни двойки $x$, $y$ доказахме, че ако $!\Gamma(f)(x,y)\simeq z$, то
%       $\Gamma(g)(x,y) \simeq z$.
%       Следователно, $\Gamma$ е монотонен.
%     \item
%       За втората част, нека $\Gamma(f)(x,y) \simeq z$, за някои $f \in \F_2$ и $x,y,z \in \Nat$.
%       Ще докажем, че съществува крайна функция $\theta \subseteq f$, за която $\Gamma(\theta)(x,y) \simeq z$.
%       За целта ще разгледаме два случая за $x$ и $y$.
%       \begin{itemize}
%       \item 
%         $x+y$ е просто число. Тогава $\Gamma(f)(x,y) \simeq 1$. 
%         Да вземем крайната функция $\theta = \emptyset^{(2)}$.
%         Очевидно е, че $\Gamma(\emptyset^{(2)})(x,y) \simeq 1$.
%       \item
%         $x+y$ не е просто число. Тогава 
%         \[\Gamma(f)(x,y) \simeq f(x+y,y)+1 \simeq z.\]
%         Да положим $u \simeq f(x+y,y)$. Тогава $z = u+1$.
%         В този случай, избираме крайната функция $\theta \subseteq f$ да бъде такава, че
%         \begin{align*}
%           \theta(a,b) \simeq 
%           \begin{cases}
%             u, & a = x+y\ \&\ b = y\\
%             \neg!, & \text{ иначе}.
%           \end{cases}
%         \end{align*}
%         Тогава 
%         \begin{align*}
%           \Gamma(\theta)(x,y) & \simeq \theta(x+y,y) + 1 & (\text{от деф. на }\theta)\\
%           & \simeq u + 1 & (\text{от деф. на }\theta)\\
%           & \simeq z.
%         \end{align*}
%       \end{itemize}
%       Така видяхме, че във всички случаи за $x$ и $y$  съществува крайна $\theta \subseteq f$, 
%       за която $\Gamma(\theta)(x,y) \simeq z$.
%     \end{itemize}
%     От всичко по-горе следва, че операторът $\Gamma$ е компактен.
%   \item
%     Да дефинираме свойството $P$ като
%     \[P(f)\ \iff\ (\forall x,y \in \Nat)[!f(x,y) \Rightarrow x + y.f(x,y) \text{ е просто}].\]
%     Това е свойство от тип частична коректност, защото ако положим
%     \begin{align*}
%       I(x,y) & \equiv\ x,y\in\Nat,\\
%       O(x,y,r) & \equiv\ x + yr\text{ е просто число},
%     \end{align*}
%     то можем да представим $P$ като
%     \[P(f)\ \equiv\ (\forall x,y)[I(x,y)\ \&\ !f(x,y)\ \Rightarrow\ O(x,y,f(x,y))].\]
%     Щом $P$ е от тип частична коректност, то $P$ е непрекъснато свойство.
%     Понеже $\Gamma$ е компактен оператор, а $P$ е непрекъснато, можем да приложим индукционното
%     правило на Скот. Така ще докажем, че $P(f_\Gamma)$.
%     \begin{itemize}
%     \item 
%       \marginpar{няма $x,y$, за които $!\emptyset^{(2)}(x,y)$}
%       $P(\emptyset^{(2)})$ е изпълнено, защото лява страна на импликацията в дефиницията на $P$
%       винаги е неистина и следователно импликацията винаги е истина.
%     \item
%       Нека приемем, че за някое $f \in \F_2$ е изпълнено $P(f)$.
%       Ще докажем, че $P(\Gamma(f))$, т.е.
%       \[(\forall x,y \in \Nat)[!\Gamma(f)(x,y) \Rightarrow x + y.\Gamma(f)(x,y) \text{ е просто}]\]
%       Нека $!\Gamma(f)(x,y)$. Отново ще разгледаме два случая за $x$ и $y$.
%       \begin{itemize}
%       \item 
%         $x+y$ е просто число. Тогава от дефиницията на $\Gamma$ имаме, че $\Gamma(f)(x,y) \simeq 1$
%         и е ясно, че \[x + y.\Gamma(f)(x,y) \simeq x+y.1 = x+y\] е просто число.
%       \item
%         $x+y$ не е просто число. Тогава $\Gamma(f)(x,y) \simeq f(x+y,y)+1$.
%         Да положим $u \simeq f(x+y,y)$. Получаваме, че:
%         \begin{align*}
%           x + y.\Gamma(f)(x,y) & \simeq x+y.(f(x+y,y)+1) & (\text{от деф. на }\Gamma)\\
%           & \simeq x+y.(u+1) \\
%           & \simeq (x+y) + y.u
%         \end{align*}
       
%         Сега използваме, че $P(f)$ е изпълнено. Тогава:
%         \[!f(x+y,y) \simeq u\ \Rightarrow\ (x+y) + y.u+ \text{ е просто число}. \]
%         Заключаваме, че 
%         \[x + y.\Gamma(f)(x,y) \simeq (x+y) + y.u\]
%         е просто число.
%       \end{itemize}      
%     \end{itemize}
%     Разгледахме всички случаи за $x$ и $y$, и следователно, $P(\Gamma(f))$ е изпълнено.    
%     От индукционното правило на Скот получаваме, че $P(f_\Gamma)$, което ни дава точно това, което 
%     трябваше да докажем.
%   \end{enumerate}
% \end{solution}

% \begin{problem}
%   Да рагледаме оператора $\Gamma:\F_3 \to \F_3$, където:
%   \begin{align*}
%     \Gamma(f)(x,y,z) \simeq 
%     \begin{cases}
%       y, & x = 0\\
%       f(x-1, y+2z, y), & x > 0.
%     \end{cases}
%   \end{align*}
%   Докажете, че 
%   \[(\forall x,u\in\Nat)[u\geq 1\ \&\ !f_\Gamma(x,2^u,2^{u-1})\ \Rightarrow\ f_\Gamma(x,2^u,2^{u-1}) \simeq 2^{x+u}].\]
% \end{problem}
% \begin{hint}
%   Лесно се съобразява, че $\Gamma$ е компактен (непрекъснат) оператор.
%   Целта ни е да дефинираме непрекъснато свойство $P$, за което да докажем с индукционното
%   правило на Скот, че $P(f_\Gamma)$.
%   Разгледайте правилото:
%   \begin{align*}
%     P(f)\ \equiv\ (\forall x,y,z\in\Nat)[& (\exists u \geq 1)[y = 2^u\ \&\ z = 2^{u-1}]\ \&\ !f(x,y,z) \Rightarrow\ \\
%     & (\exists u \geq 1)[y = 2^u\ \&\ z = 2^{u-1}\ \&\ f(x,y,z) \simeq 2^{x+u}]].
%   \end{align*}
%   $P$ е свойство от тип частична коректност, защото използвайки предикатите:
%   \begin{align*}
%     I(x,y,z) & \equiv x,y,z\in\Nat\ \&\ (\exists u \geq 1)[y = 2^u\ \&\ z = 2^{u-1}],\\
%     O(x,y,z,r) & \equiv (\exists u \geq 1)[y = 2^u\ \&\ z = 2^{u-1}\ \&\ r = 2^{x+u}],
%   \end{align*}
%   можем да представим свойството $P$ по следния начин:
%   \[P(f) \equiv  (\forall x,y,z)[I(x,y,z)\ \&\ !f(x,y,z) \Rightarrow\ O(x,y,z,f(x,y,z))].\]
  
%   \marginpar{Очевидно е, че $P(\emptyset^{(3)})$}
%   Да приемем, че имаме $P(f)$. Ще докажем $P(\Gamma(f))$.
%   Нека $x,y,z\in\Nat$ са такива, че $!\Gamma(f)(x,y,z)$ и да фиксираме $u \geq 1$, за което $y = 2^u$, $z = 2^{u-1}$.
%   Ще докажем, че $\Gamma(f)(x,y,z) \simeq 2^{x+u}$.
%   Според дефиницията на $\Gamma$, трябва да разгледаме два случая.
%   \begin{itemize}
%   \item 
%     $x = 0$. Тогава $\Gamma(x,y,z) \simeq y = 2^{u+0}$.
%   \item
%     $x > 0$. Тогава $\Gamma(x,y,z) \simeq f(x-1,y+2z,y)$.
%     Понеже $y = 2^u$ и $z = 2^{u-1}$, $y+2z = 2^u+2.2^{u-1} = 2^{u+1}$.
%     Имаме, че $I(x-1,y+2z,y)$ е истина и $!f(x-1,y+2z,y)$.
%     От $P(f)$ следва, че $O(x-1,y+2z,y,f(x-1,y+2z,y)$, т.е. 
%     \[y+2z = 2^{u+1}\ \&\ y = 2^{(u+1)-1}\ \&\ f(x-1,y+2z,y) \simeq 2^{(x-1)+(u+1)}.\]
%     Като обединим всичко, което знаем до момента, получаваме:
%     \begin{align*}
%       \Gamma(f)(x,y,z) & \simeq f(x-1,y+2z,y) & (\text{от деф. на }\Gamma)\\
%       & \simeq f(x-1,2^{u+1},2^{u}) & (y+2z = 2^{u+1},y=2^{(u+1)-1})\\
%       & \simeq 2^{(x-1)+(u+1)} & (\text{от }P(f))\\
%       & = 2^{x+u}
%     \end{align*}
%     Заключаваме, че $O(x,y,z,\Gamma(f)(x,y,z))$, откъдето следва, че $P(\Gamma(f))$.
%   \end{itemize}
%   От правилото на Скот следва, че $P(f_\Gamma)$.
% \end{hint}




% \begin{problem}
%   Операторът $\Gamma:\F_1 \to \F_1$ е зададен с равенството:
%   \begin{align*}
%     \Gamma(f)(x) \simeq
%     \begin{cases}
%       \sqrt{x}, & \text{ ако } x \text{ е точен квадрат}\\
%       f(f(x(x+5))), & \text{ иначе}.
%     \end{cases}
%   \end{align*}
%   Докажете, че:
%   \begin{enumerate}[a)]
%   \item 
%     операторът $\Gamma$ е компактен;
%   \item
%     $(\forall x\in\Nat)[!f_\Gamma(x)\ \&\ x\text{ не е точен квадрат}\ \Rightarrow\ f_\Gamma(x) > \sqrt{x}]$.
%   \end{enumerate}
% \end{problem}

% \begin{problem}
%   Нека $g:\Nat \to \Nat$ е тотална функция, за която $(\forall x\in\Nat)[g(x) \leq x]$.
%   Операторът $\Gamma:\F_2 \to \F_2$ е зададен с равенството:
%   \begin{align*}
%     \Gamma(f)(x,y) = 
%     \begin{cases}
%       g(y), & \text{ ако } x = 0\\
%       f(f(x-1,g(y-1)), y-1)+ 1, & \text{ иначе}.
%     \end{cases}
%   \end{align*}
%   Докажете, че:
%   \begin{enumerate}[a)]
%   \item 
%     операторът $\Gamma$ е компактен;
%   \item
%     $(\forall x,y\in\Nat)[!f_\Gamma(x,y)\ \Rightarrow\ f_\Gamma(x,y) \leq \max(x,y)]$.
%   \end{enumerate}
% \end{problem}

% \begin{problem}
%   Нека $g:\Nat \to \Nat$ е тотална функция, за която $(\forall x\in\Nat)[g(x) \leq x]$.
%   Операторът $\Gamma:\F_2 \to \F_2$ е зададен с равенството:
%   \begin{align*}
%     \Gamma(f)(x,y) = 
%     \begin{cases}
%       g(y), & \text{ ако } x = 0\\
%       f(x-1, f(x-1, g(y-1))) + 1, & \text{ иначе}.
%     \end{cases}
%   \end{align*}
%   Докажете, че:
%   \begin{enumerate}[a)]
%   \item 
%     операторът $\Gamma$ е компактен;
%   \item
%     $(\forall x,y\in\Nat)[!f_\Gamma(x,y)\ \Rightarrow\ f_\Gamma(x,y) > \max(x,y)]$.
%   \end{enumerate}
% \end{problem}

\begin{problem}
  Да разгледаме изображението $\Gamma \in \Cont{\F_2}{\F_2}$, където:
  \begin{align*}
    \Gamma(f)(x,y) \simeq
    \begin{cases}
      y, & \text{ ако } y\vert x\\
      f(f(x,y+1)), x), & \text{ иначе}.
    \end{cases}
  \end{align*}
  Докажете, че
  % \begin{enumerate}[a)]
  % \item 
  %   операторът $\Gamma$ е компактен;
  % \item
  \[(\forall x)(\forall y)[!f_\Gamma(x,y)\ \&\ y\not| x\ \Rightarrow\ f_\Gamma(x,y)\ \vert\ x].\]
  % \end{enumerate}  
\end{problem}

\begin{problem}
  Да разгледаме $\Gamma \in \Mapping{\F_1}{\F_1}$, където
  \begin{align*}
    \Gamma(f)(x) \simeq
    \begin{cases}
      1, & \text{ ако } x \leq 1\\
      x/2, & \text{ ако } 2\vert x\ \&\ x > 1\\
      f(f(3\lfloor{x/2}\rfloor+2)), & \text{ иначе}.
    \end{cases}
  \end{align*}
  Докажете, че $\Gamma$ е непрекъснато изображение и ако $f_\Gamma \df \lfp(\Gamma)$, то 
  докажете, че
  \[(\forall x)[!f_\Gamma(x)\ \&\ x > 1\ \Rightarrow\ f_\Gamma(x)\ \leq x/2].\]
\end{problem}

% \begin{problem}
%   Операторът $\Gamma:\F_2 \to \F_2$ е зададен с равенството:
%   \begin{align*}
%     \Gamma(f)(x,y) \simeq
%     \begin{cases}
%       y, & x = 0\\
%       f(x-1,2) + y, & x \equiv 1\ (\bmod\ 2)\\
%       f(\frac{x}{2},f(\frac{x}{2},y)), & \text{ иначе}.
%     \end{cases}
%   \end{align*}
%   Докажете, че:
%   \begin{enumerate}[a)]
%   \item 
%     операторът $\Gamma$ е компактен;
%   \item
%     $(\forall x\in\Nat)[!f_\Gamma(x,0)\ \Rightarrow\ f_\Gamma(x,0) \simeq 2x]$.
%   \end{enumerate}  
% \end{problem}
% \begin{hint}
%   Разгледайте свойството
%   \[P(f) \equiv (\forall x,y\in\Nat)[!f(x,y)\ \Rightarrow\ f(x,y) \simeq 2x+y].\]
% \end{hint}

% \begin{problem}
%   Да разгледаме оператора $\Gamma:\mathcal{F}_1\to \mathcal{F}_1$, където:
%   \begin{align*}
%     \Gamma(f)(x) \simeq 
%     \begin{cases}
%       x/3, & \text{ ако }x \equiv 0\ (\bmod\ 3)\\
%       f(3f(x+1)\dotminus 1), & \text{ ако }x \equiv 1\ (\bmod\ 3)\\
%       f(3f(2x-1)+1), & \text{ ако }x \equiv 2\ (\bmod\ 3),
%     \end{cases}
%   \end{align*}
%   където $x\dotminus y = 0$, ако $x < y$, и $x \dotminus y = x-y$, ако $x \geq y$.
%   \begin{enumerate}[a)]
%   \item 
%     Докажете, че операторът $\Gamma$ е компактен.
%   \item
%     Докажете, че
%     $(\forall x\in\Nat)[!f_\Gamma(x) \implies f_\Gamma(x) \leq x/3]$,
%     където с $f_\Gamma$ означаваме най-малката неподвижна точка на оператора $\Gamma$.
%   \end{enumerate}
% \end{problem}
% \begin{hint}
%   \begin{enumerate}[1)]
%   \item 
%     Най-лесно се решава задачата като намерете явния вид на $f_\Gamma$ с теоремата на Кнастер-Тарски.
%     Оказва се, че 
%     \begin{align*}
%       f_\Gamma(x) \simeq 
%       \begin{cases}
%         x/3, & x \equiv 0\ (\bmod\ 3)\\
%         \neg!, & \text{ иначе}.
%       \end{cases}
%     \end{align*}
    
%     Знаем, че $f_0 = \emptyset^{(1)}$, $f_{i+1} = \Gamma(f_i)$ и $f_\Gamma = \bigcup_i f_i$.
%     \begin{align*}
%       f_1(x) \simeq \Gamma(f_0)(x) \simeq\ & \Gamma(\emptyset^{(1)})(x) \simeq 
%       \begin{cases}
%         x/3, & x \equiv 0\ (\bmod\ 3)\\
%         \neg!, & \text{иначе}\\
%       \end{cases}\\
%       f_2(x) \simeq \Gamma(f_1)(x) \simeq &
%       \begin{cases}
%         x/3, & x \equiv 0\ (\bmod\ 3)\\
%         f_1(3f_1(x+1)\dotminus 1), & x \equiv 1\ (\bmod\ 3)\\
%         f_1(3f_1(2x-1) + 1), & x \equiv 2\ (\bmod\ 3)\\
%       \end{cases}\\
%       \simeq &
%       \begin{cases}
%         x/3, & x \equiv 0\ (\bmod\ 3)\\
%         \neg!, & x \equiv 1\ (\bmod\ 3)\\
%         f_1(3\frac{2x-1}{3} + 1), & x \equiv 2\ (\bmod\ 3)\\
%       \end{cases}\\
%       \simeq &
%       \begin{cases}
%         x/3, & x \equiv 0\ (\bmod\ 3)\\
%         \neg!, & x \equiv 1\ (\bmod\ 3)\\
%         f_1(2x), & x \equiv 2\ (\bmod\ 3)\\
%       \end{cases}\\
%       \simeq &
%       \begin{cases}
%         x/3, & x \equiv 0\ (\bmod\ 3)\\
%         \neg!, & \text{иначе}.
%       \end{cases}
%     \end{align*}
%     Виждаме, че $f_1 = f_2$. 
%     Заключаваме, че за всяко $k \geq 1$, $f_k = f_1$.
%     Тогава $f_\Gamma = f_1$.
%   \item
%     Ако искате да използвате правилото на Скот, възможно е да разгледате непрекъснатото свойство
%     \begin{align*}
%       P(f) \equiv\ & (\forall x\in\Nat)[x \equiv 0\ (\bmod\ 3)\ \&\ !f(x) \implies f(x) \simeq x/3]\ \&\ \\
%       & (\forall x\in\Nat)[x \equiv 1\ (\bmod\ 3)\ \&\ !f(x) \implies f(x) \leq x/12]\ \&\ \\
%       & (\forall x\in\Nat)[x \equiv 2\ (\bmod\ 3)\ \&\ !f(x) \implies f(x) \leq x/6].
%     \end{align*}    
%   \end{enumerate}
% \end{hint}

% \begin{problem}
%   Нека предварително имаме дадена частичната функция $h \in \F_2$.
%   Операторът $\Gamma:\F_2 \to \F_2$ е зададен с равенството:
%   \begin{align*}
%     \Gamma(f)(x,y) \simeq
%     \begin{cases}
%       0, & h(x,y) \simeq 0\\
%       f(x, y+1) + 1, & h(x,y) > 0\\
%       \neg !, & \neg ! h(x,y)
%     \end{cases}
%   \end{align*}  
%   Докажете, че 
%   \[(\forall x,y,z\in\Nat)[!f_\Gamma(x,y) \simeq z \implies h(x, y+z) \simeq 0\ \&\ (\forall t < z)[h(x,y+t) > 0]]\]
% \end{problem}

% \newpage

\section{Задачи}

\begin{problem}
  Да фиксираме $a_0 \in \A$ и да разгледаме $P \subseteq \Cont{\A}{\A}$, където
  \[P(f) \dfff \lfp(f) = a_0.\]
  Проверете дали $P$ е непрекъснато свойство.
\end{problem}


\begin{problem}
  Да разгледаме $\Gamma \in \Mapping{\F_1}{\F_1}$, където
  \begin{align*}
    \Gamma(f)(x,y) \simeq
    \begin{cases}
      3.f(\sqrt{x},y) + 2, & \text{ ако $x$ е точен квадрат}\\
      y, & \text{ ако $x$ не е точен квадрат}\\
    \end{cases}
  \end{align*}
  Съобразете, че $\Gamma$ е непрекъснато изображение.
  Нека $f_{\Gamma} \df \lfp(\Gamma)$. Докажете, че:
  \[(\forall x)(\forall y)[3.f_\Gamma(x,y) + 2 \simeq f_\Gamma(x,3y+2)].\]
\end{problem}
\ifhints
\begin{hint}
  \begin{itemize}
  \item 
    Нека първо да разгледаме операторите $\Gamma_1$ и $\Gamma_2$, където
    \begin{align*}
      & \Delta_1(f) \df 3f(x,y)+2\\
      & \Delta_2(f) \df f(x,3y+2).
    \end{align*}
    Съобразете, че те са непрекъснати.
  \item
    От \Prop{continuous-property} следва, че свойството 
    \[P(f) \dfff \Delta_1(f) = \Delta_2(f)\]
    е непрекъснато.
  \item
    Използвайте правилото на Скот върху $P$ 
    за да докажете, че $P(f_\Gamma)$.
    Това означава, че $3f_\Gamma(x,y) + 2 \simeq f_\Gamma(x,3y+2)$ за всяко $x,y \in \Nat$.
  \end{itemize}  
\end{hint}
\fi

\begin{problem}
  Да разгледаме изображението $\Gamma \in \Mapping{\F_1}{\F_1}$, където:
  \begin{align*}
    \Gamma(f)(x,y) \simeq
    \begin{cases}
      (f(\sqrt{x},y))^2, & \text{ ако $x$ е точен квадрат}\\
      y, & \text{ иначе }
    \end{cases}
  \end{align*}
  Съобразете защо $\Gamma$ е непрекъснато изображение.
  Нека $f_{\Gamma} \df \lfp(\Gamma)$. Докажете, че
  \[(\forall x)(\forall y)[(f_\Gamma(x,y))^2 \simeq f_\Gamma(x,y^2)].\]
\end{problem}

\begin{problem}
  Нека $p_0,p_1,p_2\dots\ $ е редицата от всички прости числа в нарастващ ред.
  Да разгледаме $\Gamma \in \Mapping{\F_3}{\F_3}$, където:
  \begin{align*}
    \Gamma(f)(x,y,z) \simeq
    \begin{cases}
      x^xy, & \text{ ако }p_z = x\\
      f(x+x,y,z+2), & \text{ ако }p_z = x.
    \end{cases}
  \end{align*}
  Съобразете, че $\Gamma$ е непрекъснато изображение. 
  Нека $f_{\Gamma} \df \lfp(\Gamma)$. Докажете, че
  \[(\forall x)(\forall y)(\forall z)[!f_{\Gamma}(x,y,z) \implies (\exists\text{ просто число }p)[p \geq x\ \&\ p^py\ |\ f_\Gamma(x,y,z)].\]
\end{problem}

\begin{problem}
  Да разгледаме изображението $\Gamma \in \Mapping{\F_3}{\F_3}$, където:
  \begin{align*}
    \Gamma(f)(x,y,z) \simeq
    \begin{cases}
      z, & y = 0\ \&\ x,z\in\Nat\\
      f(x,y-1, xy+z), & y > 0\ \&\ x,z \in \Nat.
    \end{cases}
  \end{align*}
  Съобразете, че $\Gamma$ е непрекъснато изображение и ако $f_\Gamma \df \lfp(\Gamma)$, то
  \[(\forall x)(\forall y)[f_\Gamma(x,y,0) \simeq \frac{xy(y+1)}{2}].\]
\end{problem}
\begin{hint}
  Да разгледаме свойството над ествествените числа
  \[P(y) \dfff (\forall x,z\in\Nat)[f_\Gamma(x,y,z) \simeq \frac{xy(y+1)}{2} + z].\]
  Докажете с математическа индукция по $y \in \Nat$, че $(\forall y\in\Nat)[P(y)]$.
  % \begin{itemize}
  % \item 
  %   Нека $y = 0$. Тогава за произволни $x$ и $z$,
  %   \begin{align*}
  %     f_\Gamma(x,0,z) & = \Gamma(f_\Gamma)(x,0,z) \\
  %     & = z.
  %   \end{align*}
  % \item
  %   Нека $y > 0$. Тогава 
  %   \begin{align*}
  %     f_\Gamma(x,y,z) & = \Gamma(f_\Gamma)(x,y,z)\\
  %                     & = f_\Gamma(x,y-1,xy+z) & (\text{от деф. на }\Gamma)\\
  %                     & = \frac{x(y-1)(y-1+1)}{2} + xy + z & (\text{от И.П. за }y-1)\\
  %                     & = \frac{xy(y-1)+2xy}{2} + z \\
  %                     & = \frac{xy(y+1)}{2} + z.
  %   \end{align*}
  % \end{itemize}
\end{hint}

% \begin{hint}
%   \marginpar{Ясно е, че $\Gamma$ е компактен}
%   Да разгледаме свойството
%   \[P(f)\ \dfff\ (\forall x,y,z\in\Nat_\bot)[f(x,y,z) \neq \bot\ \to\ f(x,y,z) = \frac{xy(y+1)}{2}+z].\]
%   Лесно се вижда, че това е свойство от тип частична коректност, защото ако положим
%   \begin{align*}
%     I(x,y,z) & \dfff\ \texttt{True},\\
%     O(x,y,z,r) & \dfff\ r = \frac{xy(y+1)}{2},
%   \end{align*}
%   то можем да представим $P$ във вида:
%   \[P(f)\ \equiv\ (\forall x,y,z\in\Nat_\bot)[I(x,y,z)\ \&\ f(x,y,z) \neq \bot\ \to\ O(x,y,z,f(x,y,z))].\]
  
%   Да приложим правилото на Скот.
%   \marginpar{Очевидно е, че $P(\Omega^{(3)})$}
%   Нека е изпълнено $P(f)$. Ще докажем, че $P(\Gamma(f))$.
%   И така, нека да разгледаме елементи $x,y,z \in \Nat_\bot$, за които $\Gamma(f)(x,y,z) \neq \bot$. 
%   Ясно е от дефиницията на оператора $\Gamma$, че $\bot \not\in \{x,y,z\}$.
%   Ще разгледаме два случая.
%   \begin{itemize}
%   \item 
%     Нека $y = 0$. Тогава $\Gamma(f)(x,y,z) = z = \frac{xy(y+1)}{2} + z$.
%   \item
%     Нека $y > 0$. Тогава 
%     \begin{align*}
%       \Gamma(f)(x,y,z) & = f(x,y-1,xy+z) & (\text{от деф. на }\Gamma)\\
%       & = \frac{x(y-1)(y-1+1)}{2} + xy + z & (\text{от }P(f))\\
%       & = \frac{xy(y-1)+2xy}{2} + z \\
%       & = \frac{xy(y+1)}{2} + z.
%     \end{align*}
%   \end{itemize}
%   Заключаваме, че $P(\Gamma(f))$, откъдето следва, по правилото на Скот, че $P(\lfp(\Gamma))$.
  
%   Накрая, за произволни $x,y\in\Nat$ и $z = 0$, получаваме, че 
%   \[f_\Gamma(x,y,0) = \frac{xy(y+1)}{2}.\]
% \end{hint}

% \begin{remark}
% Да разгледаме програмата на езика \REC:

% \begin{haskellcode}
% h(x,y) = f(x,y,0) where
%   f(x,y,z) = if y == 0 then z
%                else f(x, y-1, x*y + z)
% \end{haskellcode}

% Съобразете, че ние горе на практика доказахме, че
% \[(\forall x,y \in \Nat)[\D_V\val{\vv{h}}(x,y) = \frac{xy(y+1)}{2}].\]
% \end{remark}


% \begin{problem}
%   Нека е дадена програмата на езика хаскел:

%   \begin{minted}[frame=lines,framesep=2mm,baselinestretch=1.2]{haskell}
%     rev :: [a] -> [a]
%     rev x = f(x, []) where 
%       f([], y) = y
%       f(x:xs, y) = f(xs, x:y)
%   \end{minted}

%   \noindent 
%   Докажете, че:
%   \begin{enumerate}[a)]
%   \item 
%     $rev:\Sigma^\star \to \Sigma^\star$ е тотална.
%   \item
%     $(\forall x \in \Sigma^\star)[rev(rev(x)) = x]$.
%   \item
%     $(\forall x \in \Sigma^\star)[rev(x) = x^R]$.
%   \end{enumerate}
% \end{problem}
% \begin{hint}
%   % Ще използваме следното правило:
%   % \begin{prooftree}
%   %   \AxiomC{$P(\varepsilon)$}
%   %   \AxiomC{$(\forall x \in \Sigma^\star)[x\neq\varepsilon\ \&\ P(cdr(x)) \to P(x)]$}
%   %   \RightLabel{\scriptsize(1)}
%   %   \BinaryInfC{$(\forall x\in\Sigma^\star)[P(x)]$}
%   % \end{prooftree}
%   % % \item
%   %   Докажете валидността на правилото $(1)$. % е еквивалентно на структурна индукция върху фундираната наредба
%     % $(\Sigma^\star,\prec)$, където $x \prec y \iff (\exists z\in\Sigma^\star)[z\cdot x = y]$, т.е.
%     % $x$ е суфикс на $y$.
%   Да разгледаме фундираната наредба $L = (\Sigma^\star, \prec)$, където
%   $x \prec y \iff \abs{x} < \abs{y}$.
%   \begin{enumerate}[a)]
%   \item 
%     Да разгледаме свойството 
%     \[P(x) \equiv (\forall y\in \Sigma^\star)[f(x,y)\text{ е дефинирана}].\]
%     Докажете със структурна индукция по $L$, че $(\forall x\in\Sigma^\star)[P(x)]$.
%   \item
%     Да разгледаме свойството 
%     \[P(x) \equiv (\forall y\in \Sigma^\star)[rev(f(x,y)) = f(y,x)].\]
%     Докажете със структурна индукция по $L$, че $(\forall x\in\Sigma^\star)[P(x)]$.
%     Тогава в частния случай $y = \varepsilon$, 
%     \[rev(rev(x)) = rev(f(x,\varepsilon)) = f(\varepsilon,x) = x.\]
%   \item
%     Разгледайте свойството
%     \[P(x) \equiv (\forall y\in\Sigma^\star)[f(x,y) = x^R \cdot y].\]
%   \end{enumerate}
% \end{hint}

% \input{proofs/partial-correctness}



%%% Local Variables:
%%% mode: latex
%%% TeX-master: "../sep"
%%% End:

%\include{cat/ccc}
% \chapter{Лениви спъсъци}
\marginpar{До известна степен следваме \cite[Глава 6]{bird-haskell}}

\Stefan{Дали не е по-добре вместо $[x,y,z]$ е съкратен запис за $x:y:z:[]$ ?}

\section{Основни понятия}
\begin{itemize}
\item
  Нека с $\nil$ да означаваме празния спъсък, т.е. единствения краен списък с дължина $0$.
\item
  {\bf Краен спъсък} с елементи естествените числа $a_0,\dots,a_{k-1}$
  ще записваме като $\pair{a_0,\dots,a_{k-1},\nil}$, т.е. това е елемент придандлежащ на множеството  
  \[\underbrace{\Nat \times \Nat \cdots \times \Nat}_{k} \times \{\nil\}.\]
  Дефинираме множеството от всички крайни списъци като
  \[\FinL \df \bigcup_{k\geq 0} \Nat^k \times \{\nil\}.\]
\end{itemize}

Нека $P(l)$ е свойство на крайните списъци.
\begin{prooftree}
  \AxiomC{$P(\nil)$}
  \AxiomC{$(\forall a\in\Nat)(\forall l\in\FinL)[P(l) \implies P(\pair{a,l})]$}
  \BinaryInfC{$(\forall l\in \FinL)[P(l)]$}
\end{prooftree}

% \subsection{Частични списъци}

\marginpar{Това на практика са недовършени списъци}
{\bf Частичен списък} с елементи естествените числа $a_0,\dots,a_{k-1}$
ще записваме като $\pair{a_0,a_1,a_2,\dots,a_{k-1},\bot}$, т.е. това е елемент придандлежащ на множеството  
\[\underbrace{\Nat \times \Nat \cdots \times \Nat}_{k} \times \{\bot\}.\]
Дефинираме множеството от всички частични списъци като
\[\PartL \df \bigcup_{k\geq 0} \Nat^k \times \{\bot\}.\]
Тогава $\bot$ може да се интерпретира като единствения частичен списък с дължина $0$.
Да видим какво ще стане ако изпълним следния ред:

\begin{haskellcode}
ghci> filter (<4) [1..]
[1,2,3                      -- == $(1:2:3:\bot)$
\end{haskellcode}
  
Програмата работи безкрайно дълго време след като вече е отпечатала първите три числа,
защото хаскел не знае, че след $3$ в безкрайния списък няма други числа по-малки от $4$.
Това означава, че резултатът от изпълнението на тази програма е частичния списък $\pair{1,2,3,\bot}$,

Нека $P(l)$ е свойство на частичните списъци.
\begin{prooftree}
  \AxiomC{$P(\bot)$}
  \AxiomC{$(\forall a\in\Nat)(\forall l\in\PartL)[P(l) \implies P( \pair{a,l} ) ]$}
  \BinaryInfC{$(\forall l\in \PartL)[P(l)]$}
\end{prooftree}


% \subsection*{Безкрайни списъци}

{\bf Безкраен списък} с елементи естествените числа $a_0,a_1,\dots$
ще записваме като $\pair{a_0,a_1,\dots}$, т.е. това е елемент придандлежащ на множеството  
\[\InfL \df \Nat^\Nat = \{f:\Nat \to \Nat \mid f \text{ е тотална}\}.\]
Както вече знаем, на хаскел е много лесно да работим с безкрайни списъци.
\begin{haskellcode}
ghci> take 10 [1,3..]                 -- $a_n = 1 + (3-1)*n$
[1,3,5,7,9,11,13,15,17,19]
ghci> take 10 [x*x | x <- [0..]]      -- $a_n = n^2$
[0,1,4,9,16,25,36,49,64,81]
\end{haskellcode}

\marginpar{chain-complete property \cite[стр. 218]{bird-haskell}}
Нека $P(l)$ е {\em непрекъснато свойство} на частичните и безкрайни списъци, ако 
за всяка верига от частични списъци $(l_n)^\infty_{n=0}$ имаме, че $(\forall n)[P(l_n)]$,
то тогава имаме, че $P(\bigsqcup_n l_n)$.

\begin{prooftree}
  \AxiomC{$P(\bot)$}
  \AxiomC{$(\forall a\in\Nat)(\forall l\in\PartL)[P(l) \implies P(\pair{a,l})]$}
  \BinaryInfC{$(\forall l\in \PartL)[P(l)]$}
\end{prooftree}



% \begin{itemize}
% \item 
%   Нека с $\nil$ да означаваме празния спъсък, т.е. единствения краен списък с дължина $0$.
% \item
%   {\bf Краен спъсък} с елементи естествените числа $a_0,\dots,a_{k-1}$
%   ще записваме като $\pair{a_0,\dots,a_{k-1},\nil}$, т.е. това е елемент придандлежащ на множеството  
%   \[\underbrace{\Nat \times \Nat \cdots \times \Nat}_{k} \times \{\nil\}.\]
%   Дефинираме множеството от всички крайни списъци като
%   \[\FinL \df \bigcup_{k\geq 0} \Nat^k \times \{\nil\}.\]
% \item
%   \marginpar{Това на практика са недовършени списъци}
%   {\bf Частичен списък} с елементи естествените числа $a_0,\dots,a_{k-1}$
%   ще записваме като $\pair{a_0,\dots,a_{k-1},\bot}$, т.е. това е елемент придандлежащ на множеството  
%   \[\underbrace{\Nat \times \Nat \cdots \times \Nat}_{k} \times \{\bot\}.\]
%   Дефинираме множеството от всички частични списъци като
%   \[\PartL \df \bigcup_{k\geq 0} \Nat^k \times \{\bot\}.\]
%   Тогава $\bot$ може да се интерпретира като единствения частичен списък с дължина $0$.
%   Да видим какво ще стане ако изпълним следния ред:
  
%   \begin{haskellcode}
% ghci> filter (<4) [1..]
% [1,2,3
%   \end{haskellcode}
  
% Програмата работи безкрайно дълго време след като вече е отпечатала първите три числа,
% защото хаскел не знае, че след $3$ в безкрайния списък няма други числа по-малки от $4$.
% Това означава, че резултатът от изпълнението на тази програма е частичния списък $\pair{1,2,3,\bot}$,
% \item
%   {\bf Безкраен списък} с елементи естествените числа $a_0,a_1,\dots$
%   ще записваме като $\pair{a_0,a_1,\dots}$, т.е. това е елемент придандлежащ на множеството  
%   \[\InfL \df \Nat^\Nat = \{f:\Nat \to \Nat \mid f \text{ е тотална}\}.\]
%   Както вече знаем, на хаскел е много лесно да работим с безкрайни списъци.
%   \begin{haskellcode}
% ghci> take 10 [1,3..]
% [1,3,5,7,9,11,13,15,17,19]
% ghci> take 10 [x*x | x <- [1..]]
% [1,4,9,16,25,36,49,64,81,100]
%   \end{haskellcode}
% \end{itemize}



\section{Област на Скот от списъци}


Дефинираме областта на Скот $L = (L,\sqsubseteq, \bot)$, където
\[L = \FinL \cup \PartL \cup \InfL,\]
а частичната наредба $\sqsubseteq$ е дефинирана следвайки правилата:
\begin{itemize}
\item
  $(a_0 : a_1 : \dots: a_{n-1} : \bot) \sqsubseteq (a_0 : a_1 : \dots : a_{n-1} : b_0 : \dots : b_{k-1} : \bot)$;
\item
  $(a_0 : a_1 : \dots : a_{n-1} : \bot) \sqsubseteq (a_0 : \dots : a_{n-1} : b_0 : \dots : b_{k-1} : \nil)$;
\item
  $(a_0 : \dots : a_{n-1} : \bot) \sqsubseteq (a_0 : \dots : a_{n-1} : b_0 : b_1 : \cdots)$.
\end{itemize}

% \begin{haskellcode}
% data LazyList = Nil | Cons Int LazyList deriving (Show)
% \end{haskellcode}

Според правилата, когато $n = 0$, получаваме, че $\bot$ е най-малкият елемент на $L$.
Обърнете внимание, че
\[(\forall \alpha_1,\alpha_2 \in \FinL \cup \InfL)[\alpha_1 \sqsubseteq \alpha_2 \iff \alpha_1 = \alpha_2].\]
Това означава, че например,
\begin{itemize}
% \item 
%   $\pair{0,1,2,\bot} \sqsubseteq \pair{0,1,2,\nil}$;
\item
  $(0:1:\nil) \not\sqsubseteq (0:1:2:\nil)$;
\item
  $\bot \sqsubseteq (0:\bot) \sqsubseteq (0:1:\bot) \sqsubseteq \cdots \sqsubseteq [0,1,\dots]$.
\end{itemize}

\begin{framed}
  \begin{figure}[H]
    \label{fig:lazy-list}
    \centering
    \begin{tikzpicture}[shorten >=1pt,->]
      \tikzstyle{vertex}=[circle,minimum size=17pt,inner sep=0pt]
      
      \node[vertex] (bot) at (3,0) {$\bot$};
      \node[vertex] (0) at (2,1) {$\nil$};
      \node[vertex] (1) at (4,1) {$(0:\bot)$};

      \node[vertex] (11) at (3,2.5) {$(0:\nil)$};
      \node[vertex] (12) at (5,2.5) {$(0:1:\bot)$};

      \node[vertex] (122) at (3.5,4.3) {$(0:1:\nil)$};
      \node[vertex] (123) at (7,4) {$(0:1:2:\bot)$};
      
      \node[vertex] (1233) at (4.6,6) {$(0:1:2:\nil)$};
      \node[vertex] (dots) at (9,5.5) {};

      \draw (bot) -- node[below left]{$\scriptstyle{\sqsupseteq}$} (0);
      \draw (bot) -- node[below right]{$\scriptstyle{\sqsubseteq}$} (1);
      \draw (1) -- node[below left]{$\scriptstyle{\sqsupseteq}$} (11);
      \draw (1) -- node[below right]{$\scriptstyle{\sqsubseteq}$} (12);
      \draw (12) -- node[below left]{$\scriptstyle{\sqsupseteq}$} (122);
      \draw (12) -- node[below right]{$\scriptstyle{\sqsubseteq}$} (123);
      \draw (123) -- node[below left]{$\scriptstyle{\sqsupseteq}$} (1233);
      \draw[dashed] (123) -- node[below right]{$\scriptstyle{\sqsubseteq}$} (dots);
    \end{tikzpicture}    
    \caption{Графично представяне на $\sqsubseteq$ върху част от $L$}
  \end{figure}
\end{framed}

\subsection*{Канонични апроксимации}

\marginpar{Тук следваме \cite[стр. 218]{bird-haskell}}
\index{канонична апроксимация}
За всеки елемент $l \in L$, дефинираме неговата {\bf $n$-та канонична апроксимация} $l\upharpoonright{n}$, където:
\begin{align*}
  & l \upharpoonright{0} \df \bot\\
  & \nil \upharpoonright{(n+1)} \df \nil\\
  % & l \upharpoonright{(n+1)} \df \texttt{cons}(\texttt{car}(l), l\upharpoonright{n}).
  & (a:l) \upharpoonright{(n+1)} \df a:(l\upharpoonright{n}).
\end{align*}
Имаме свойството, че за всеки списък $l \in L$, 
\[n \leq n' \implies l\upharpoonright n \sqsubseteq l\upharpoonright n'.\]
Това означава, че $(l\upharpoonright n)^{\infty}_{n=0}$ е верига.
Лесно се съобразява, че 
\[l = \bigsqcup_n (l \upharpoonright n).\]
Друго важно свойство, което имаме е, че 
\[l_1 \sqsubseteq l_2 \implies l_1 \upharpoonright n \sqsubseteq l_2 \upharpoonright n.\]
\begin{itemize}
\item 
  Да обърнем внимание на апроксимациите на крайни списъци.
  Нека $l = \pair{a_0,a_1,\dots,a_{n-1},\nil}$. Тогава според дефиницията:
  \begin{align*}
    & l \upharpoonright {0} = \bot,\\
    & l \upharpoonright k = \pair{a_0,\dots,a_{k-1},\bot} \text{, за }k = 1,\dots, n,\\
    & l \upharpoonright k = l \text{, за }k\geq n+1.
  \end{align*}

\item
  Да разгледаме няколко примера.
  \begin{itemize}
  \item 
    Нека $l = (0:1:\nil) \in \FinL$. Тогава 
    \[\underbrace{\bot}_{l\upharpoonright 0} \sqsubseteq \underbrace{(0:\bot)}_{l\upharpoonright 1} \sqsubseteq \underbrace{(0:1:\bot)}_{l\upharpoonright 2} \sqsubseteq \underbrace{(0:1:\nil)}_{l\upharpoonright 3} = \underbrace{(0:1:\nil)}_{l\upharpoonright 4} = \cdots\]
  \item
    Нека $l = (0:1:\bot) \in \PartL$. Тогава
    \[\underbrace{\bot}_{l\upharpoonright 0} \sqsubseteq \underbrace{(0:\bot)}_{l\upharpoonright 1} \sqsubseteq \underbrace{(0:1:\bot)}_{l\upharpoonright 2} = \underbrace{(0:1:\bot)}_{l\upharpoonright 3} = \underbrace{(0:1:\bot)}_{l\upharpoonright 4} = \cdots\]
  \item
    Нека $l = (0:1:2:\cdots) \in \InfL$. Тогава
    \[\underbrace{\bot}_{l\upharpoonright 0} \sqsubseteq \underbrace{(0:\bot)}_{l\upharpoonright 1} \sqsubseteq \underbrace{(0:1:\bot)}_{l\upharpoonright 2} \sqsubseteq \underbrace{(0:1:2:\bot)}_{l\upharpoonright 3} \sqsubseteq \underbrace{(0:1:2:3:\bot)}_{l\upharpoonright 4} \sqsubseteq \cdots\]
  \end{itemize}
\end{itemize}


Можем да дефинираме апроксимацията на списъци по следния начин на хаскел:
\begin{haskellcode}
approx _      0         = undefined         -- $l \upharpoonright 0 = \bot$
approx []     n | n > 0 = []                -- $\nil \upharpoonright (n+1) = \nil$
approx (x:xs) n | n > 0 = x:approx xs (n-1) -- $(a:l) \upharpoonright (n+1) = (a: (l\upharpoonright n))$
\end{haskellcode}

\begin{problem}
  \label{prob:approx}
  Да дефинираме изображението 
  \[\texttt{approx}_n(l) \df l \upharpoonright n.\]
  Докажете, че $\texttt{approx}_n \in \Cont{L}{L}$.
\end{problem}

\begin{haskellcode}
ghci> approx [1..10] 0    --  == $\bot$
*** Exception: Prelude.undefined
ghci> approx [1..10] 10   --  == (1:2:3:4:5:6:7:8:9:10:$\bot$)
[1,2,3,4,5,6,7,8,9,10*** Exception: Prelude.undefined  
ghci> approx [1..10] 11   --  == (1:2:3:4:5:6:7:8:9:10:$\nil$)
[1,2,3,4,5,6,7,8,9,10]
ghci> approx [1..10] 101  --  == (1:2:3:4:5:6:7:8:9:10:$\nil$)
[1,2,3,4,5,6,7,8,9,10]
\end{haskellcode}

\Stefan{Да се обасни каква е разликата между take и approx}

\begin{proposition}
  Областта на Скот $L$ е алгебрична като крайните елементи $K(L) = \FinL$ и $\PartL$.
\end{proposition}
% \begin{hint}
%   Банално.
% \end{hint}

\section{Алгебричност на непрекъснатите изображения}

Целта ни тук ще бъде да докажем, че $\Cont{L}{L}$ е алгебрична област на Скот.
За да направим това трябва да намерим {\em крайните елементи} на $\Cont{L}{L}$.
Понеже крайните елементи на $L$ са $\FinL$ и $\PartL$, то е логично да предположим,
че крайните елементи на $\Cont{L}{L}$ са дефинирани чрез крайно много елементи на $K(L)$.

\begin{problem}
  Докажете, че за всяко $f \in \Cont{L}{L}$ е изпълнено, че:
  \[f(l) = \bigsqcup_n \{f(l \upharpoonright n)\upharpoonright n\}.\]
\end{problem}
\begin{hint}
  Използвайте \Th{double-chain}.
\end{hint}

\begin{problem}
  За произволно изображение $f \in \Cont{L}{L}$ и число $n$, да ознчим с $f \upharpoonright n$ изображението, където
  \[(f\upharpoonright n)(l) \df f(l\upharpoonright n)\upharpoonright n.\]
  Докажете, че $f\upharpoonright n \in \Cont{L}{L}$.
\end{problem}
\begin{hint}
  Използвайте \Problem{approx}.
\end{hint}

\Stefan{Това не е ясно!}
Можем да считаме изображенията $f\upharpoonright n$ като апроксимации на $f$.
Обаче те не са крайни апроксимации, защото за да дефинираме $f \upharpoonright n$
се нуждаем от безкрайна информация. Това е така, защото има безкрайно много списъци с дължина $\leq n$.
Оттук следва, че изображенията $f \upharpoonright n$ не са крайните елементи на $\Cont{L}{L}$.

\begin{problem}
  Нека $a \in K(\A)$ и $b \in K(\B)$.
  Да дефинираме изображението
  \[\theta(x) \df
  \begin{cases}
    b, & \text{ ако } a \sqsubseteq x\\
    \bot, & \text{ иначе}.
  \end{cases}\]
  Докажете, че:
  \begin{enumerate}[1)]
  \item 
    $\theta \in \Cont{\A}{\B}$.
  \item
    ако $f \in \Cont{\A}{\B}$, то $\theta \sqsubseteq f \iff b \sqsubseteq f(a)$.
  \end{enumerate}
\end{problem}
\begin{hint}
  
\end{hint}

Сега трябва да видим кога и по какъв начин можем да правим крайни обединения на такива изображения.
Това не е толкова просто, защото искаме полученото изображение също да бъде непрекъснато.

Две редици от елементи на $K(L)$
\begin{align*}
  \bar{a} & = (a_0,\dots,a_{n-1}),\\
  \bar{b} & = (b_0,\dots,b_{n-1})
\end{align*}
се наричат {\bf съвместими}, ако е изпълнено свойството:
\begin{equation}
  \label{eq:mon}
  (\forall i,j < n)[a_i \sqsubseteq a_j \implies b_i \sqsubseteq b_j].
\end{equation}

\begin{example}
  Редиците от крайни елементи
  \begin{align*}
    \bar{a} & = (\ (0:\bot),\ (1:\bot),\ (0:1:\nil)\ ),\\
    \bar{b} & = (\ (1:2:\nil),\ \bot,\ (1:2:3:\nil)\ ),
  \end{align*}
  не са съвместими, защото
  \[(0:\bot) \sqsubseteq (0:1:\nil)\text{, но } (1:2:\nil) \not\sqsubseteq (1:2:3:\nil).\]
\end{example}

Възможно е някои от елементите на $\bar{a}$ да са несравними помежду си, но обърнете внимание, че за произволен елемент $x$, 
всички елементи на множеството 
\[A \df \{a_i \mid i < n\ \&\ a_i \sqsubseteq x\}\]
са сравними помежду си.
Това означава, че 
\[A = \{a_{i_0} \sqsubseteq a_{i_1} \sqsubseteq \cdots \sqsubseteq a_{i_k}\},\] за някое $k$,
и следователно, $A$ притежава точна горна граница.
Естествено, множеството $A$ може и да е празно. Тогава точната горна граница на $A$ ще бъде $\bot$,
защото по дефиниция $\bigsqcup\emptyset = \bot$.

Да разгледаме изображението 
\begin{align*}
  \theta_{\bar{a},\bar{b}}(x) & \df \bigsqcup_{i<n}\{b_i \mid a_i \sqsubseteq x\}\\
  & = \begin{cases}
    b_0, & \text{ ако } a_0 = \bigsqcup_{i<n}\{a_i \mid a_i \sqsubseteq x\}\\
    \vdots & \\
    b_{n-1}, & \text{ ако } a_{n-1} = \bigsqcup_{i<n}\{a_i \mid a_i \sqsubseteq x\}\\
    \bot, & \text{ иначе}.
  \end{cases}
\end{align*}

\Stefan{На картинка нещата не изглеждат толкова сложни.}

\begin{proposition}
  Ако $\bar{a}$ и $\bar{b}$ са съвместими редици от крайни елементи на $L$, то имаме свойствата:
  \begin{enumerate}[1)]
  \item 
    $\theta_{\bar{a},\bar{b}} \in \Cont{L}{L}$;
  \item
    За всяко $f \in \Cont{L}{L}$, 
    \[\theta_{\bar{a},\bar{b}} \sqsubseteq f\ \iff\ (\forall i < n)[b_i \sqsubseteq f(a_i)];\]
  \item
    $\theta_{\bar{a},\bar{b}}$ е компактен елемент в $\Cont{L}{L}$.
  \end{enumerate}
\end{proposition}
\begin{proof}
  \begin{enumerate}[1)]
  \item 
    Лесно. Използва се, че $\bar{a}$ са крайни елементи.
  \item
    Използва се само монотонността на $f$.
  \item
    Тук съществено се използва, че $\bar{b}$ са крайни елементи.

    Нека $(f_r)^\infty_{r=0}$ е верига в $\Cont{L}{L}$ и 
    $\theta_{\bar{a},\bar{b}} \sqsubseteq \bigsqcup_r f_r$.
    От $2)$, това означава, че за $i < n$,
    \[b_i \sqsubseteq (\bigsqcup_r f_r)(a_i) = \bigsqcup_r \{f_r(a_i)\}.\]
    Понеже $b_i$ е краен елемент, то съществува индекс $r_i$, за който
    \[b_i \sqsubseteq f_{r_i}(a_i).\]
    Накрая взимаме $r = \max\{r_1,\dots,r_n\}$.
    Получаваме, че за $i < n$,
    \[b_i \sqsubseteq f_{r}(a_i).\]
    Тогава отново от 2) следва, че $\theta_{\bar{a},\bar{b}} \sqsubseteq f_r$.
  \end{enumerate}
\end{proof}

\begin{framed}
\begin{proposition}
  Областта на Скот $\Cont{L}{L}$ е алгебрична.
\end{proposition}
\end{framed}
\begin{proof}
  Да разгледаме произволен елемент $f \in \Cont{L}{L}$.
  Ще покажем, че съществува верига от крайни елементи $(\theta_n)^\infty_{n=0}$, за която $f = \bigsqcup_n \theta_n$.
  Нека подредим елементите на $K(L)$ в една редица $a_0,a_1,\dots$,
  което можем да направим, защото те са изброимо много.
  Да означим 
  \begin{align*}
    \bar{a}_n & = (a_0,a_1,\dots,a_{n})\\
    \bar{b}_n & = (f(a_0)\upharpoonright n, \dots, f(a_{n})\upharpoonright n).
  \end{align*}
  Нека да положим
  \[\theta_n \df \theta_{\bar{a}_n,\bar{b}_n}.\]

  Лесно се проверява, че $(\theta_n)^\infty_{n=0}$ е верига, защото $f$ е монотонно изображение.
  
  Трябва да проверим, че $f = \bigsqcup_n \theta_n$. 
  От дефиницията на $\theta_n$ е ясно, че $\theta_n \sqsubseteq f$ за всяко $n$.
  Следователно, $\bigsqcup_n \theta_n \sqsubseteq f$.
  За другата посока, нека разгледаме произволен елемент $l \in L$.
  Ще докажем, че $f(l) \sqsubseteq (\bigsqcup_n \theta_n)(l)$.
  Да разгледаме верига от елементи на $K(L)$, $(a_{i_n})^\infty_{n=0}$, за която $i_0 < i_1 < \cdots$
  и $\bigsqcup_n a_{i_n} = l$. Знаем, че такава съществува, защото $L$ е алгебрична област на Скот.
  \Stefan{Трябва някъде да сложа твърдение, че ако $\bigsqcup_n a_n = b$, то за произволна подредица
  $\bigsqcup_n a_{i_n} = b$.}
  Тогава 
  \begin{align*}
    f(l) & = \bigsqcup_n \{f(a_{i_n}) \upharpoonright n\}\\
         & \sqsubseteq \bigsqcup_n \{f(a_{i_n}) \upharpoonright i_n \} & (i_n \geq n)\\
         & = \bigsqcup_n \{\theta_{i_n}(a_{i_n})\} & (\text{от деф. на }\theta_{i_n})\\
         & = \bigsqcup_n\{ \bigsqcup_m \{\theta_{i_n}(a_{i_m})\}\} & (\text{\Th{double-chain}})\\
         & = \bigsqcup_n \{\theta_{i_n}(\bigsqcup_m a_{i_m})\} & (\theta_{i_n} \in \Cont{L}{L})\\
         & = \bigsqcup_n \{\theta_{i_n}(l)\} & (l = \bigsqcup_m a_{i_m})\\
         & = (\bigsqcup_n \theta_{i_n})(l) \\
         & \sqsubseteq (\bigsqcup_n \theta_n)(l).
  \end{align*}
\end{proof}


\section{Задачи}

\Stefan{Дали да дефинираме изображение cons, което да докажа, че е непрекъснато?}

\begin{problem}
  Да се даде пример за $f \in \Mon{L}{L}$, но $f \not\in \Cont{L}{L}$.
\end{problem}

\begin{problem}
  Да се даде пример за $f \in \Strict{L}{L}$, но $f \not\in \Mon{L}{L}$.
\end{problem}

\begin{problem}
  Да разгледаме изображението $f \in \Cont{L}{L}$, където
  \[f(l) \df (0:l).\]
  Намерете $\lfp(f)$.
\end{problem}
\begin{solution}
  Прилагаме \Th{knaster-tarski}.
  \begin{itemize}
  \item 
    $l_0 = \bot$;
  \item
    $l_1 = f(\bot) = (0:\bot)$;
  \item
    $l_2 = f(l) = (0:0:\bot)$;
  \end{itemize}
  Така получаваме, че
  \[l_n = \pair{\underbrace{0,0,\dots,0}_{n},\bot} \in \PartL.\]
  Тогава 
  \[\lfp(f) = \bigsqcup_n l_n = \pair{0,0,\dots}.\]



\begin{haskellcode}
ghci> let f(l) = (0:l)
ghci> let approx = undefined : [f(x) | x <- approx]
ghci> approx !! 10
[0,0,0,0,0,0,0,0,0,0*** Exception: Prelude.undefined  
\end{haskellcode}

\end{solution}

\begin{problem}
  Да разгледаме изображението 
  \[\Gamma \in \Cont{\Cont{\Nat_\bot\times L}{L}}{\Cont{\Nat_\bot\times L}{L}},\] където
  \[\Gamma(f)(n,l) =
  \begin{cases}
    \bot, & n = \bot\\
    \nil, & n = 0\\
    \nil, & n > 0\ \&\ l = \nil\\
    \bot, & n > 0\ \&\ l = \bot\\
    a : f(l'), & n > 0\ \&\ l = (a:l')\\
  \end{cases}\]
  Намерете $\lfp(\Gamma)$.
\end{problem}


\begin{haskellcode}
ghci> let l = (0:l)
ghci> take 10 l
[0,0,0,0,0,0,0,0,0,0]
ghci> take 10 [1,2,3]
[1,2,3]
ghci> take 0 undefined
[]
ghci> take 2 [1,2,undefined]
[1,2]
ghci> take 3 [1,2,undefined]
[1,2,*** Exception: Prelude.undefined
ghci> take undefined []
*** Exception: Prelude.undefined
\end{haskellcode}


\Stefan{Да се разгледа функцията drop n l}

% \begin{problem}
%   Да разгледаме оператора $\Gamma:[\Nat\to L] \to [\Nat \to L]$, където
%   \[\Gamma(f)(a,b) = \pair{a, f(b,a+b)}.\]
%   Намерете безкрайния списък $\lfp(\Gamma)(1,1)$.
% \end{problem}

\begin{problem}
  Да разгледаме оператора $\Gamma:[L\times L \to L] \to [L\times L \to L]$, където
  \[\Gamma(f)(\ell_1,\ell_2) = 
  \begin{cases}
    \bot, & \text{ ако }\ell_1 = \bot \\
    \ell_2, & \text{ ако }\ell_1 = \nil\\
    a:f(\ell'_1,\ell_2) & \text{ ако } \ell_1 = a:\ell'_1.
  \end{cases}\]
  Да означим $\texttt{conc} \df \lfp(\Gamma)$.
  Докажете, че
  \[(\forall \ell_1 \in \FinL)(\forall \ell_2,\ell_3 \in L)[\texttt{conc}(\ell_1, \texttt{conc}(\ell_2,\ell_3)) = \texttt{conc}(\texttt{conc}(\ell_1, \ell_2), \ell_3))]\]
\end{problem}

%%% Local Variables:
%%% mode: latex
%%% TeX-master: "../sep-notes"
%%% End:


%%% Local Variables:
%%% mode: latex
%%% TeX-master: "../sep"
%%% End:


% \include{call-by-value}
% \include{call-by-name}
% \chapter{Верификация на програми по метода на Флойд}

\tikzstyle{decision} = [diamond, draw, fill=green!10, text width=4em, text badly centered, node distance=3cm, inner sep=0pt]
\tikzstyle{block} = [rectangle, draw, fill=red!10, text width=5em, text centered, rounded corners, minimum height=2em]
\tikzstyle{tallblock} = [rectangle, draw, fill=red!10, text width=4em, rounded corners, minimum height=3em]
\tikzstyle{line} = [draw, -latex']
\tikzstyle{bigblock} = [rectangle, draw, fill=red!10, text width=7em, rounded corners, minimum height=3em]
\tikzstyle{label} = [draw,circle,fill=yellow!20,node distance=3.2cm]
\tikzstyle{cloud} = [draw, ellipse, text width=4em, text centered, fill=blue!10, node distance=2cm, minimum height=2em]


Верификация на итеративни програми по метода на индуктивните твърдения на Флойд \cite{floyd-verification}.

% \section{Метод на Флойд}

\marginpar{Този метод е описан за първи път от Робърт Флойд \cite{floyd-verification}}
\marginpar{Тук на практика следваме \cite[Глава 3]{manna} и \cite[Глава 1]{nikolova-soskova}}

\section{Програми с един цикъл}

\subsection{Решени задачи}

\subsection*{Намиране на НОД}

Да дефинираме функцията $\texttt{gcd}:\Nat\times\Nat \to \Nat$, където
\[\texttt{gcd}(x,y) = (\max z)[\ z\ |\ x\ \&\ z\ |\ y\ ].\]
\begin{itemize}
\item 
  Понеже $0\ |\ 0$, то $\text{НОД}(0,0) = 0$.
\item
  Понеже всяко естествено число дели $0$, то $\text{НОД}(0,z) = z$;
\item
  Ясно е от дефиницията, че $\text{НОД}(x,y) = \text{НОД}(y,x)$;
\end{itemize}

\begin{prop}
  \label{pr:gcd}
  $(\forall x,y\in\Nat)[\text{НОД}(x,y) = \text{НОД}(y, x \bmod y)]$.
\end{prop}
\begin{proof}
  Ще разгледаме няколко случая.
  \begin{itemize}
  \item 
    Ако $x = y$, то всичко е ясно, защото $x \bmod x = x$.
  \item
    Ако $x < y$, то всичко е ясно, защото $x \bmod y = x$ и
    \[\texttt{gcd}(x,y) = \texttt{gcd}(y,x) = \texttt{gcd}(y, x \bmod y).\]
  \item
    Нека $x > y$ и $z = \texttt{gcd}(x,y)$.
    Тогава $z$ е най-голямото естествено число, което $z\ |\ x$ и $z\ |\ y$.
    Нека $r = x \bmod y$. Тогава $x = ky + r$, $0 \leq r < y$.
    Щом $z$ дели $x$ и $y$, то е ясно, че $z$ дели $r$.
    Остана да съобразим защо $z = \texttt{gcd}(y,r)$.
    Да допуснем, че съществува естествено число $z'$, такова че $z < z'$ и $z'$ дели $y$ и $r$.
    Тогава $z'$ също дели и $x = ky + r$. Достигнахме до противоречие с факта, че $z = \texttt{gcd}(x,y)$.
    Следователно,
    \[z = \texttt{gcd}(x,y) = \texttt{gcd}(y,x \bmod y).\]
  \end{itemize}
\end{proof}

\begin{figure}[H]
  \begin{subfigure}[b]{0.6\textwidth}
  \begin{tikzpicture}[node distance = 3cm,auto,scale=0.6, every node/.style={scale=0.9}]
    % Place nodes
    \node [cloud] (init) {вход: $x,y$};
    \node [label, below of=init, node distance=1.5cm] (n1) {\scriptsize{1}};
    \node [tallblock, below of=init] (identify) {$z := x$\\$t := y$};
    \node [label, below of=identify,node distance=1.5cm] (n2) {\scriptsize{2}};
    \node [decision, below of=identify] (evaluate) {$t = 0$};
    \node [bigblock, below of=evaluate, node distance=2cm] (ji) {$v := t$\\$t := z \bmod t$\\$z := v$};
    \coordinate[right of=evaluate,node distance=2cm] (inc);
    \node [label, left of=evaluate, node distance=2cm] (n3) {\scriptsize{3}};
    \node [cloud, left of=n3] (exit) {изход: $z$};

  % Draw edges
    \path [line] (init) -- (n1);
    \path [line] (n1) -- (identify);
    \path [line] (identify) -- (n2);
    \path [line] (n2) -- (evaluate);
    \path [line] (evaluate) -- node {не} (ji);
    \draw [-] (ji) -| node {} (inc);
    \path [line] (inc) |- node  {} (n2);
    \path [line] (evaluate) -- node {да} (n3);
    \path [line] (n3) -- node {} (exit);
  \end{tikzpicture}
  \caption{Алгоритъм за намиране на $\text{НОД}(x,y)$}
  \label{fig:gcd}
  \end{subfigure}
  ~
  \begin{subfigure}[b]{0.6\textwidth}
    \footnotesize{
      Да разгледаме свойствата:
      \begin{align*}
        A_1(x,y,z,t,v) \dfff & x,y \in \Nat\\
        A_2(x,y,z,t,v) \dfff & \text{НОД}(x,y) = \text{НОД}(z,t)\ \&\\
        & x,y,z,t,v\in\Nat\\
        A_3(x,y,z,t,v) \dfff & z = \text{НОД}(x,y)
      \end{align*}
      Докажете, че за всеки преход $(k) \to (l)$ имаме
      \[(\forall x,y,z,t,v)[A_k(x,y,z,t,v) \implies A_l(x,y,f_{kl}(z,t,v))].\]
    }
    \end{subfigure}
\end{figure}

\begin{prop}
  Докажете, че програмата на Фигура \ref{fig:gcd} е тотално коректна относно входното условие $x,y\in\Nat$
  и изходното условие $z = \text{НОД}(x,y)$.
\end{prop}
\begin{hint}
  Доказателството на $A_2(x,y,z,t,v) \implies A_2(x,y,f_{22}(z,t,v))$ следва директно от \Prop{gcd}.
  За доказателството на $A_2(x,y,z,t,v) \implies A_3(x,y,f_{23}(z,t,v))$ е достатъчно да съобразим, че ако имаме $A_2(x,y,z,t,v)$ и $t = 0$, то
  $\text{НОД}(x,y) = \text{НОД}(z,0) = z$.
  
  Остана да докажем, че програмата винаги завършва при входни данни $x,y \in \Nat$.
  Да означим с $t_i$ стойността на променливата $t$ след $i$-тото преминаване през етикет $(2)$.
  Това означава, че $t_0 = y$ и $t_{i+1} = z \bmod t_i$, за някое $z$.
  Следователно, $t_{i+1} < t_i$. От $A_2$ имаме, че всички $t_i \geq 0$.
  Така получаваме строго монотонно намаляваща редица $t_0 > t_1 > \cdots > t_i > \cdots \geq 0$, която е ограничена отдолу от $0$.
  Заключаваме, че съществува $i$, за което $t_i = 0$. Следователно, програмата завършва.
\end{hint}


%%% Local Variables:
%%% mode: latex
%%% TeX-master: "../sep-problems"
%%% End:


\subsection*{Търсене на максимална сума}

\begin{problem}
  % Да разгледаме следния алгоритъм:
  % \begin{algorithm}[H]
  %   \caption{}
  %   \label{alg:useless}
  %   \begin{algorithmic}[1]
  %     \State $y := x[0]$
  %     \State $z := x[0]$
  %     \State $i := 1$
  %     \For{$i := 1; i < n; i++$}
  %     \State $z := \max\{x[i], z + x[i]\}$
  %     \State $y := \max\{x[i], z + x[i]\}$
  %     \EndFor
  %     \State \Return $y$
  %   \end{algorithmic}
  % \end{algorithm}
  

  Докажете, че програмата $P$, описана с блок-схемата на \Fig{max-sum}, 
  пресмята $\max\{\sum^l_{i=k}x_i \mid 0 \leq k \leq l \leq n\}$.
\end{problem}

\begin{figure}[H]
  \begin{tikzpicture}[node distance = 3cm,auto,scale=0.6, every node/.style={scale=0.9}]
    % Place nodes
    \node [cloud] (init) {вход: $x_0,\dots,x_n$};
    \node [label, below of=init, node distance=1.5cm] (n1) {\scriptsize{1}};
    \node [tallblock, below of=init] (identify) {$y := x_0$\\$z := x_0$\\$i := 1$};
    \node [label, below of=identify,node distance=1.5cm] (n2) {\scriptsize{2}};
    \node [decision, below of=identify] (evaluate) {$i \leq n$};
    \node [bigblock, below of=evaluate, node distance=2cm] (ji) {$\scriptstyle{z := \max\{x_i,z+x_i\}}$\\$\scriptstyle{y := \max\{y,z\}}$};
    \node [block, right of=evaluate, node distance=2.5cm] (inc) {$i := i+1$};
    \node [label, left of=evaluate, node distance=2cm] (n3) {\scriptsize{3}};
    \node [cloud, left of=n3] (exit) {изход: $y$};

  % Draw edges
    \path [line] (init) -- (n1);
    \path [line] (n1) -- (identify);
    \path [line] (identify) -- (n2);
    \path [line] (n2) -- (evaluate);
    \path [line] (evaluate) -- node {да} (ji);
    \path [line] (ji) -| node {} (inc);
    \path [line] (inc) |- node  {} (n2);
    \path [line] (evaluate) -- node {не} (n3);
    \path [line] (n3) -- node {} (exit);
  \end{tikzpicture}
  \caption{Ще докажем, че $y = \max\{\sum^l_{i=k}x_i \mid 0 \leq k \leq l \leq n\}$}
% \end{wrapfigure}
  \label{fig:max-sum}
\end{figure}

Първо ще разгледаме две твърдения, които ще ни подскажат какво представляват междинните стойности на променливите $z$ и $y$ в програмата на \Fig{max-sum}.

\begin{prop}
  \label{pr:Z}
  Нека е дадена редицата $x_0,\dots,x_n \in \Int$.
  Да дефинираме:
  \begin{itemize}
  \item
    $Z(\bar{x},0) = x_0$;
  \item
    $Z(\bar{x},i+1) = \max\{x_{i+1}, Z(\bar{x},i)+x_{i+1}\}$.
  \end{itemize}
  Тогава за всяко $i \leq n$, $Z(\bar{x},i) = \max\{\sum^i_{j  = k}x_j \mid 0 \leq k \leq i\}$.
\end{prop}
\begin{proof}
  Индукция по $i$.
  За $i = 0$ е ясно, защото 
  \[Z(\bar{x},0) = x_0 = \max\{\sum^0_{j=k}x_j \mid 0\leq k\leq 0\}.\]
  Ще докажем твърдението за $i+1$.
  \begin{align*}
    Z(\bar{x},i+1) & = \max\{Z(\bar{x},i)+x_{i+1},x_{i+1}\} & (\text{от деф.})\\
    & = \max\{\max\{\sum^{i}_{j  = k}x_j \mid 0 \leq k \leq i\}+x_{i+1}, x_{i+1}\} & (\text{от И.П.})\\
    & = \max\{\max\{\sum^{i+1}_{j  = k}x_j \mid 0 \leq k \leq i\}, \sum^{i+1}_{j=i+1}x_j\}\\
    & = \max\{\sum^{i+1}_{j  = k}x_j \mid 0 \leq k \leq i+1\}.
  \end{align*}
\end{proof}

\begin{prop}
  \label{pr:Y}
  Нека е дадена редицата $x_0,\dots,x_n \in \Int$.
  Да дефинираме:
  \begin{itemize}
  \item 
    $Y(\bar{x},0) = x_0$;
  \item
    $Y(\bar{x},i+1) = \max\{Y(\bar{x},i), Z(\bar{x},i+1)\}$.
  \end{itemize}
  Тогава за всяко $i \leq n$, $Y(\bar{x},i) = \max\{Z(\bar{x},l) \mid 0 \leq l \leq i\}$.
\end{prop}
\begin{proof}
  Отново индукция по $i$.
  За $i = 0$ е очевидно, защото
  \[Y(\bar{x},0) = x_0 = Z(\bar{x},0) = \max\{Z(\bar{x},l) \mid 0 \leq l \leq 0\}.\]
  Ще докажем твърдението за $i+1$.
  \begin{align*}
    Y(\bar{x},i+1) & = \max\{Y(\bar{x},i), Z(\bar{x},i+1)\} & (\text{от деф.})\\
    & = \max\{\max\{Z(\bar{x},l) \mid 0 \leq l \leq i\}, Z(\bar{x},i+1)\} & (\text{от И.П.})\\
    & = \max\{Z(\bar{x},l) \mid 0 \leq l \leq i+1\}.
  \end{align*}
\end{proof}

\begin{cor}
  \label{cr:Y}
  $Y(\bar{x},n) = \max\{\sum^l_{i=k}x_i \mid 0\leq k \leq l \leq n\}$.
\end{cor}

Сега сме готови да дефинираме свойствата $A_l$ за етикетите $l = 1,2,3$:
\begin{align*}
  & A_1(\bar{x},i,y,z,n) \equiv \bar{x} \in \Int^{n+1};\\
  & A_2(\bar{x},i,y,z,n) \equiv y = Y(\bar{x},i-1)\ \&\ z = Z(\bar{x},i-1)\ \&\ 1 \leq i \leq n+1;\\
  & A_3(\bar{x},i,y,z,n) \equiv y = Y(\bar{x},n).
\end{align*}  

За всеки директен преход $(k) \to (l)$ между етикети в блок схемата, асоциираме функция $f_{kl}$,
която показва как се изменят стойностите на променливите участващи в програмата на \Fig{max-sum}:
\begin{align*}
  & f_{12}(\bar{x},i,y,z,n) = (\bar{x},1,x_0,x_0,n);\\
  & f_{22}(\bar{x},i,y,z,n) = (\bar{x},i+1,max\{x_i,z+x_i\},\max\{y,z\},n);\\
  & f_{23}(\bar{x},i,y,z,n) = (\bar{x},i,y,z,n).
\end{align*}

\begin{prop}
  За всеки директен преход между етикети $(k) \to (l)$ е изпълнена импликацията:
  \[(\forall\bar{x}\in\Int^{n+1})(\forall i,j\in\Int)[A_k(\bar{x},i,j,n) \implies A_l(f_{kl}(\bar{x},i,j,n))].\]
\end{prop}
\begin{proof}
  \begin{description}
  \item[($1 \to 2$)] 
    Следва директно от дефинициите на $Y(\bar{x},0)$ и $Z(\bar{x},0)$.
  \item[($2 \to 2$)] 
    Следва директно от \Prop{Z} и \Prop{Y}.
    Ясно е също, щом преминаваме $2 \to 2$, то $1 \leq i+1 \leq n+1$.
  \item[($2 \to 3$)] 
    Понеже $i \leq n+1$ и прехода $2\to 3$ ни дава, че $i > n$, 
    то следва, че $i = n+1$. Тогава $Y(i-1) = Y(n)$.
    Сега прилагаме \Cor{Y}.
  \end{description}
\end{proof}
\begin{cor}
  Програмата от \Fig{max-sum} е частично коректна относно 
  входното условие $I(\bar{x},n) \equiv \bar{x} \in \Int^{n+1}$ и изходното условие $O(\bar{x},n,y) \equiv y = \max\{\sum^l_{i=k}x_i \mid 0\leq k \leq l \leq n\}$.
\end{cor}


%%% Local Variables:
%%% mode: latex
%%% TeX-master: "../sep-problems"
%%% End:


\subsection{Задачи с упътване}


\begin{figure}[H]
  \begin{subfigure}[b]{0.6\textwidth}
  \begin{tikzpicture}[node distance = 3cm,auto,scale=0.6, every node/.style={scale=0.9}]
    % Place nodes
    \node [cloud] (init) {вход: $n$};
    \node [label, below of=init, node distance=1.5cm] (n1) {\scriptsize{1}};
    \node [tallblock, below of=init] (identify) {$x := 0$\\$y := 1$\\$s := 1$};
    \node [label, below of=identify,node distance=1.5cm] (n2) {\scriptsize{2}};
    \node [decision, below of=identify] (evaluate) {$s \leq n$};
    \node [bigblock, below of=evaluate, node distance=2cm] (ji) {$x := x + 1$\\$y := y + 2$\\$s := s + y$};
    \coordinate[right of=evaluate,node distance=2cm] (inc);
    \node [label, left of=evaluate, node distance=2cm] (n3) {\scriptsize{3}};
    \node [cloud, left of=n3] (exit) {изход: $x$};

  % Draw edges
    \path [line] (init) -- (n1);
    \path [line] (n1) -- (identify);
    \path [line] (identify) -- (n2);
    \path [line] (n2) -- (evaluate);
    \path [line] (evaluate) -- node {да} (ji);
    \draw [-] (ji) -| node {} (inc);
    \path [line] (inc) |- node  {} (n2);
    \path [line] (evaluate) -- node {не} (n3);
    \path [line] (n3) -- node {} (exit);
  \end{tikzpicture}
  \caption{Алгоритъм за намиране на $\lfloor{\sqrt{n}}\rfloor$}
  \label{fig:sqrt}
  \end{subfigure}
  ~
  \begin{subfigure}[b]{0.6\textwidth}
    \footnotesize{
      Да разгледаме свойствата:
      \begin{align*}
        A_1(x,y,s,n) \dfff & n \in \Nat\\
        A_2(x,y,s,n) \dfff & x,n\in\Nat\ \&\ x^2 \leq n\ \&\\
        & y = 2x+1\ \&\ s = (x+1)^2\\
        A_3(x,y,s,n) \dfff & x^2 \leq n < (x+1)^2
      \end{align*}
      Съобразете, че $A_3(x,y,s,n) \implies x = \lfloor{\sqrt{n}}\rfloor$.
      Докажете, че за всеки преход $(k) \to (l)$ имаме
      \[(\forall x,y,s,n)[A_k(x,y,s,n) \implies A_l(f_{kl}(x,y,s),n)].\]
    }
    \end{subfigure}
\end{figure}

\begin{problem}
  Докажете, че програмата $P$ описана с блок-схемата на Фигура \ref{fig:sqrt} е тотално коректна относно входното 
  условие $n \in \Nat$ и изходното условие $\lfloor{\sqrt{n}}\rfloor$.
\end{problem}


%%% Local Variables:
%%% mode: latex
%%% TeX-master: "../sep-problems"
%%% End:


\section{Програми с два цикъла}

% \subsection*{Сортиране на масив}
\subsection{Решени задачи}


\subsection*{Сортиране чрез вмъкване}
\begin{problem}
  Докажете, че програмата, описана с блок-схемата на \Fig{insertion-sort}, 
  сортира входния масив във възходящ ред.
\end{problem}

%\begin{wrapfigure}{r}{0.5\textwidth}
\begin{figure}[H]
  \begin{tikzpicture}[node distance = 2.75cm,auto,scale=0.6, every node/.style={scale=0.9}]
    % Place nodes
    \node [cloud] (init) {вход: $z_0,\dots,z_n$};
    \node [label, below of=init, node distance=1.25cm] (n1) {\scriptsize{1}};
    \node [block, below of=init, node distance=2.25cm] (identify) {$\bar{x} := \bar{z}$\\$i := 1$};
    \node [label, below of=identify,node distance=1.5cm] (n2) {\scriptsize{2}};
    \node [decision, below of=identify] (evaluate) {$i \leq n$};
    \node [block, below of=evaluate, node distance=2cm] (ji) {$j := i$};
    \node [block, right of=ji, node distance=2.5cm] (inc) {$i := i+1$};
    \node [label, below of=evaluate] (n3) {\scriptsize{3}};
    \node [decision, below of=ji] (loop) {$\scriptstyle{j\geq 1\ \&}$\\$\scriptstyle{x_j < x_{j-1}}$};
    \node [block, left of=loop] (swap) {$\scriptstyle{swap(x_j,x_{j-1})}$\\$\scriptstyle{j := j-1}$};
    \node [cloud, left of=evaluate, node distance=4cm] (exit) {изход: $x_0,\dots,x_n$};
    \node [label, left of=evaluate, node distance=2cm] (n4) {\scriptsize{4}};
  % Draw edges
    \path [line] (init) -- (n1);
    \path [line] (n1) -- (identify);
    \path [line] (identify) -- (n2);
    \path [line] (n2) -- (evaluate);
    \path [line] (evaluate) -- node {да} (ji);
    \path [line] (ji) -- node {} (n3);
    \path [line] (n3) -- node {} (loop);
    \path [line] (loop) -- node {да} (swap);
    \path [line] (swap) |- node {} (n3);
    \path [line] (loop) -| node [below] {не} (inc);
    \path [line] (inc) |- node  {} (n2);
    \path [line] (evaluate) -- node {не} (n4);
    \path [line] (n4) -- node {} (exit);
  \end{tikzpicture}
  \caption{По даден входен масив $\bar{x}\in\Int^{n+1}$, програмата го сортира във възходящ ред (\href{http://en.wikipedia.org/wiki/Insertion_sort}{insertion sort})}
% \end{wrapfigure}
  \label{fig:insertion-sort}
\end{figure}

\noindent Удобно е да означим:
\[\texttt{Ord}(\bar{x},i,j) \dfff (\forall k)[i \leq k < j \implies x_k \leq x_{k+1}],\]
което ни казва, че елемените в интервала $[i,j]$ на масива $\bar{x}$ са подредени във възходящ ред.
Също така, да означим:
\begin{align*}
  \texttt{Perm}(\bar{z},\bar{x},n) \dfff \bar{z},\bar{x} \in \Int^{n+1}\ \&\ (\exists f)[&f:\{0,\dots,n\}\to\{0,\dots,n\} \text{ е биекция }\&\\
  & (\forall i)[0\leq i \leq n \implies z_i = x_{f(i)}]].
\end{align*}
Това означава, че $\bar{z}$ е пермутация на елементите на $\bar{x}$.

\marginpar{Понеже масивът $\bar{z}$ и $n$ са константни, то няма нужда да описваме как се променят с функциите $f_{kl}$}
За всеки директен преход $(k) \to (l)$ между етикети в блок схемата, асоциираме функция $f_{kl}$,
която показва как се изменят стойностите на променливите участващи в програмата.
\begin{align*}
  & f_{12}(\bar{x},i,j) \dff (\bar{x},1,j)\\
  & f_{23}(\bar{x},i,j) \dff (\bar{x}, i, i)\\
  & f_{24}(\bar{x},i,j) \dff (\bar{x}, i, j)\\
  & f_{32}(\bar{x},i,j) \dff (\bar{x}, i+1, j)\\
  & f_{33}(\bar{x},i,j) \dff (\bar{x}',i,j-1),
\end{align*}
където $\bar{x}' = (x'_0,\dots,x'_n)$ е променения масив, за който
$x'_{j-1} = x_{j}$, $x'_j = x_{j-1}$, а $x'_k = x_k$ за всеки индекс $k$ в интервала $[0,n] \setminus\{j-1,j\}$.

Към всеки етикет $(l)$ в блок схемата на програмата $P$ на \Fig{insertion-sort}, асоциираме предиката $A_l$, където:
\begin{align*}
  A_1(\bar{z},\bar{x},i,j,n) \dfff\ & \bar{z} \in \Int^{n+1}\ \&\ n \geq 0\\
  A_2(\bar{z},\bar{x},i,j,n) \dfff\ & \texttt{Perm}(\bar{z},\bar{x},n)\ \&\ \texttt{Ord}(\bar{x},0,i-1)\ \&\ 0 \leq i \leq n+1\\
  A_3(\bar{z},\bar{x},i,j,n) \dfff\ & \texttt{Perm}(\bar{z},\bar{x},n)\ \&\ \texttt{Ord}(\bar{x},0,j-1)\ \&\ \texttt{Ord}(\bar{x},j,i)\ \&\\
  & 0\leq j \leq i \leq n\ \&\ (0 < j < i \implies x_{j-1} \leq x_{j+1})\\
  A_4(\bar{z},\bar{x},i,j,n) \dfff\ & \texttt{Perm}(\bar{z},\bar{x},n)\ \&\ \texttt{Ord}(\bar{x},0,n).
\end{align*}

\begin{prop}
  За всеки директен преход между етикети $(k) \to (l)$ е изпълнена импликацията:
  \[(\forall\bar{z},\bar{x},i,j,n)[A_k(\bar{z},\bar{x},i,j,n) \implies A_l(\bar{z},f_{kl}(\bar{x},i,j),n)].\]
\end{prop}
\begin{proof}
  \begin{description}
  \item[($1\to 2$)]
    Очевидно е, че имаме $A_2(\bar{z},\bar{x},1,j,n)$, т.е. $\texttt{Ord}(\bar{x},0,1-1)$ и $1 \leq n+1$.
  \item[($2 \to 3$)]
    Нека $A_2(\bar{z},\bar{x},i,j,n)$.
    Ще докажем, че имаме $A_3(\bar{z},\bar{x},i,i,n)$. Това е съвсем лесно:
    \begin{itemize}
    \item 
      От $A_2$ е ясно, че имаме $\texttt{Ord}(\bar{x},0,i-1)$.
    \item
      Очевидно е, че имаме $\texttt{Ord}(\bar{x},i,i)$.
    \item
      От $A_2$ имаме, че $i < n+1$. Но понеже от етикет 2 сме отишли в етикет 3, 
      то $i \neq n+1$. Следователно, $0 \leq i \leq i \leq n$.
    \item
      Импликацията $0 < i < i \implies x_{i-1}\leq x_{i+1}$  
      е изпълнена по тривиални причини.
    \end{itemize}
  \item[($3\to 3$)]
  Нека $A_3(\bar{z},\bar{x},i,j,n)$.
  Ще докажем, че $A_3(\bar{z},\bar{x}',i,j-1,n)$.
  Понеже сме направили преход $3 \to 3$,
  то имаме също свойството, че $j \geq 1$ и $x_{j} < x_{j-1}$.
  Това означава, че:
  \begin{align*}
    \bar{x} =\ & \overbrace{x_0 \leq \dots \leq x_{j-2}\leq x_{j-1}}^{\texttt{Ord}} > \overbrace{x_j \leq x_{j+1} \leq \cdots \leq x_i}^{\texttt{Ord}}\\
    \bar{x}' =\ & \underbrace{x_0 \leq \dots \leq x_{j-2}}_{\texttt{Ord}} \square \underbrace{x'_{j-1}}_{x_j} < \underbrace{x'_j}_{x_{j-1}} \square \underbrace{x_{j+1} \leq \cdots \leq x_i}_{\texttt{Ord}}
  \end{align*}
  
  \begin{itemize}
  \item 
    Ясно е, че $0 \leq j-1 \leq i$;
  \item
    От $\texttt{Ord}(\bar{x},0,j-1)$ следва, че $\texttt{Ord}(\bar{x}',0,j-2)$,
    защото единствената промяна в масива е размяната на стойностите
    на $x_{j-1}$ и $x_j$.
  \item
    За да докажем, че $\texttt{Ord}(\bar{x}',j-1,i)$, трябва да разгледаме два случая.
    \begin{itemize}
    \item
      Ако $j = i$, тогава $x_i < x_{i-1}$. Да проверим, че $\texttt{Ord}(\bar{x}',i-1,i)$.
      Имаме, че:
      \[\texttt{Ord}(\bar{x}',i-1,i)\ \iff\ x'_{i-1} \leq x'_{i}.\]
      Ние имаме дясната страна на еквивалентността, защото от размяната на елементите $x_{i-1}$ и $x_{i}$
      имаме \[x'_{i-1} = x_i < x_{i-1} = x'_{i}.\]
    \item
      Ако $j < i$, тогава от $A_3(\bar{x},i,j,n)$ имаме, че $\texttt{Ord}(\bar{x},j,i)$,
      и  $x_{j-1} \leq x_{j+1}$, защото $0 < j < i$.
      % а от последния конюнкт на $A_3$ е изпълнено, че $x_{j-1} \leq x_{j+1}$.
      Щом имаме $\texttt{Ord}(\bar{x},j,i)$, за да докажем, че $\texttt{Ord}(\bar{x}',j-1,i)$ e достатъчно да проверим, че
      $x'_{j-1}\leq x'_{j}$ и $x'_j \leq x'_{j+1}$. Понеже сме извършили размяна на стойностите на $x_{j-1}$ и $x_j$,
      а останалите елементи на $\bar{x}$ остават непроменени, получаваме:
      \begin{itemize}
      \item 
        $ x'_{j-1} = x_j < x_{j-1} = x'_j$,
      \item
        $x'_{j} = x_{j-1} \leq x_{j+1} = x'_{j+1}$.
      \end{itemize}
    \end{itemize}
  \item
    Остана да проверим, че ако $0 < j - 1 < i$, то $x'_{j-2} \leq x'_{j}$, т.е. дали
    \[x'_{j-2} = x_{j-2} \leq x_{j-1} = x'_{j}.\] Това е изпълнено, защото от 
    $A_3(\bar{x},i,j,n)$ имаме, че $\texttt{Ord}(\bar{x},0,j-1)$ и следователно $x_{j-2} \leq x_{j-1}$.
  \end{itemize}
  
\item[($3\to 2$)]
  Нека $A_3(\bar{z},\bar{x},i,j,n)$.
  Щом сме отишли в етикет 2, значи имаме \[\neg(1 \leq j\ \&\ x_j < x_{j-1}).\]
  Ще докажем $A_2(\bar{z},\bar{x},i+1,j,n)$.
  \begin{itemize}
  \item 
    $A_3(\bar{z},\bar{x},i,j,n) \implies i \leq n \implies i+1 \leq n+1$.
  \item
    Трябва да проверим, че $\texttt{Ord}(\bar{x},0,i+1-1)$.
    Да видим защо сме преминали от 3 към 2.
    \begin{itemize}
    \item 
      Ако $j < 1$, то $j = 0$, защото от $A_3$ имаме, че $0\leq j$.
      Освен това, от $A_3$ имаме $\texttt{Ord}(\bar{x},0,i)$.
      Оттук ведната следва, че $\texttt{Ord}(\bar{x},0,i-1)$
    \item
      Ако $j \geq 1$, но $x_{j-1} \leq x_j$.
      От $A_3$ имаме, че $\texttt{Ord}(\bar{x},0,j-1)$ и $\texttt{Ord}(\bar{x},j,i)$.
      От всичко това имаме, че 
      \[x_0 \leq \dots \leq x_{j-1} \leq x_j \leq x_{j+1} \leq \dots \leq x_i.\]
      Заключаваме, че $\texttt{Ord}(\bar{x},0,i)$.
    \end{itemize}
  \end{itemize}
\item[($2 \to 4$)]
  Нека $A_2(\bar{z},\bar{x},i,j,n)$ е изпълнено. Щом се достигнали етикета 4, значи имаме и $i > n$.
  От $A_2$ пък имаме $i \leq n+1$.
  Следователно, $i = n+1$ и тогава $A_2(\bar{z},\bar{x},i,j,n) \implies \texttt{Ord}(\bar{x},0,n+1-1)$.
\end{description}
\end{proof}

\begin{cor}
  Програмата $P$ от \Fig{insertion-sort} 
  е частично коректна относно входното условие $I(\bar{z},n) \dff \bar{z}\in\Int^{n+1}$ и изходното условие $O(\bar{z},\bar{x},n) \equiv \texttt{Ord}(\bar{x},0,n)\ \&\ \texttt{Perm}(\bar{z},\bar{x},n)$.
\end{cor}

%%% Local Variables:
%%% mode: latex
%%% TeX-master: "../sep-problems"
%%% End:


\subsection{Задачи с упътване}

\subsection*{Сортиране с избор}

\begin{figure}[H]
  \begin{subfigure}[b]{0.6\textwidth}
  \begin{tikzpicture}[node distance = 2.5cm,auto,scale=0.6, every node/.style={scale=0.9}]
    % Place nodes
    \node [cloud] (init) {вход: $z_0,\dots,z_n$};
    \node [label, below of=init, node distance=1.5cm] (n1) {\scriptsize{1}};
    \node [block, below of=init] (identify) {$\scriptsize{\bar{x}:=\bar{z}}$\\$\scriptstyle{i := 0}$};
    \node [label, below of=identify,node distance=1.2cm] (n2) {\scriptsize{2}};
    \node [decision, below of=identify,node distance=2.7cm] (evaluate) {$i < n$};
    \node [block, below of=evaluate, node distance=2cm] (ji) {$\scriptstyle{m := i}$\\$\scriptstyle{j := i+1}$};
    \node [label, below of=evaluate] (n3) {\scriptsize{3}};
    \node [decision, below of=ji] (loop) {$j > n$};
    \node [decision, below of=loop,node distance=2.5cm] (comp) {$x_j < x_m$};
    \node [label, right of=comp, node distance=1.8cm] (n4) {\scriptsize{4}};
    \node [block, right of=n4,node distance=1.8cm] (jinc) {$\scriptstyle{j++}$};
    \node [block, below of=jinc] (mset) {$\scriptstyle{m := j}$};
    \node [label, above of=mset, node distance=1.3cm] (n5) {\scriptsize{5}};
    \node [block, left of=loop] (swap) {$\scriptstyle{swap(x_i,x_m)}$\\$\scriptstyle{i++}$};
    \node [label, right of=evaluate, node distance=1.8cm] (n6) {\scriptsize{6}};
    \node [cloud, right of=n6, node distance=2cm] (exit) {изход: $x_0,\dots,x_n$};

  % Draw edges
    \path [line] (init) -- (n1);
    \path [line] (n1) -- (identify);
    \path [line] (identify) -- (n2);
    \path [line] (n2) -- (evaluate);
    \path [line] (evaluate) -- node {да} (ji);
    \path [line] (ji) -- node {} (n3);
    \path [line] (n3) -- node {} (loop);
    \path [line] (loop) -- node {не} (comp);
    \path [line] (comp) -- node {не} (n4);
    \path [line] (n4) -- node {} (jinc);
    \path [line] (comp) |- node [left] {да} (mset);
    \path [line] (mset) -- node {} (n5);
    \path [line] (n5) -- node {} (jinc);
    \path [line] (jinc) |- node {} (n3);
    x\path [line] (loop) -- node [below] {да} (swap);
    \path [line] (swap) |- node  {} (n2);
    \path [line] (evaluate) -- node {не} (n6);
    \path [line] (n6) -- node {} (exit);
  \end{tikzpicture}
  \caption{Блок схема за алгоритъм, който сортира входния масив възходящ ред (\href{https://en.wikipedia.org/wiki/Selection_sort}{Selection sort})}
  \end{subfigure}
  ~
  \qquad
  \begin{subfigure}[b]{0.6\textwidth}
    \footnotesize{
      Трябва да докажем, че програмата е частично коректна относно
      \begin{align*}
        & I(\bar{z},n) \dfff \bar{z} \in \Int^{n+1}\\
        & O(\bar{z},\bar{x},n) \dfff \texttt{Ord}(\bar{x},0,n)\ \&\ \texttt{Perm}(\bar{z},\bar{x},n).
      \end{align*}     
      За да направим това, разгледайте свойствата:
      \begin{itemize}
      \item 
        $A_1(\bar{z},\bar{x},i,j,m,n) \dfff \bar{z} \in \Int^{n+1}\ \&\ n \geq 0$;
      \item
        $A_2(\bar{z},\bar{x},i,j,m,n)~\dfff~\texttt{Perm}(\bar{z},\bar{x},n)\ \&\ \texttt{Ord}(\bar{x},0,i)\ \&$\\
        $i\leq n\ \&\ (0 < i \implies x_{i-1} = \min\{x_{i-1},\dots,x_n\})$;
      \item
        $A_3(\bar{z},\bar{x},i,j,m,n) \dfff A_2(\bar{z},\bar{x},i,j,m,n)\ \&\ x_m = \min\{x_i,\dots,x_{j-1}\}\ \&\ i \leq m \leq j \leq n+1$;
      \item
        $A_4(\bar{z},\bar{x},i,j,m,n) \dfff A_3(\bar{z},\bar{x},i,j+1,m,n)$;
      \item
        $A_5(\bar{z},\bar{x},i,j,m,n) \dfff A_4(\bar{z},\bar{x},i,j,m,n)$;
      \item
        $A_6(\bar{z},\bar{x},i,j,m,n) \dfff \texttt{Ord}(\bar{x},0,n)\ \&\ \texttt{Perm}(\bar{z},\bar{x},n)$.
      \end{itemize}
      Докажете, че за всеки преход $(k) \to (l)$ имаме
      \[A_k(\bar{z},\bar{x},i,j,m,n) \implies A_l(\bar{z},f_{kl}(\bar{x},i,j,m),n).\]
    }
    % \caption{}
  \end{subfigure}
  % \end{wrapfigure}
\end{figure}

%%% Local Variables:
%%% mode: latex
%%% TeX-master: "../sep-problems"
%%% End:

\subsection*{Обръщане на масив}

\begin{figure}[H]
  \begin{subfigure}[b]{0.6\textwidth}
  \begin{tikzpicture}[node distance = 2.5cm,auto,scale=0.6, every node/.style={scale=0.9}]
    % Place nodes
    \node [cloud] (init) {вход: $z_0,\dots,z_n$};
    \node [label, below of=init, node distance=1.5cm] (n1) {\scriptsize{1}};
    \node [block, below of=init] (identify) {$\scriptsize{\bar{x}:=\bar{z}}$\\$\scriptstyle{i := 0}$};
    \node [label, below of=identify,node distance=1.2cm] (n2) {\scriptsize{2}};
    \node [decision, below of=identify,node distance=2.7cm] (evaluate) {$i = n+1$};
    \node [block, below of=evaluate, node distance=2cm] (ji) {$\scriptstyle{y := x_0}$\\$\scriptstyle{j := 0}$};
    \node [label, below of=evaluate] (n3) {\scriptsize{3}};
    \node [decision, below of=ji] (loop) {$j = n - i$};
    \node [block, right of=loop] (jinc) {$\scriptstyle{x_j := x_{j+1}}$\\$\scriptstyle{j++}$};
    \node [block, left of=loop] (mset) {$\scriptstyle{x_{n-i} := y}$\\$\scriptstyle{i++}$};
    \node [label, right of=evaluate, node distance=1.8cm] (n4) {\scriptsize{4}};
    \node [cloud, right of=n6, node distance=2cm] (exit) {изход: $x_0,\dots,x_n$};

  % Draw edges
    \path [line] (init) -- (n1);
    \path [line] (n1) -- (identify);
    \path [line] (identify) -- (n2);
    \path [line] (n2) -- (evaluate);
    \path [line] (evaluate) -- node {не} (ji);
    \path [line] (ji) -- node {} (n3);
    \path [line] (n3) -- node {} (loop);
    \path [line] (loop) -- node {не} (jinc);
    \path [line] (loop) -- node [above] {да} (mset);
    \path [line] (mset) |- node {} (n2);
    \path [line] (jinc) |- node {} (n3);
    \path [line] (evaluate) -- node {да} (n4);
    \path [line] (n4) -- node {} (exit);
  \end{tikzpicture}
  \caption{Ще докажем, че по даден входен масив, програмата го обръща}
  \end{subfigure}
  ~
  \qquad
  \begin{subfigure}[b]{0.6\textwidth}
    \footnotesize{
      Да положим предикатите:
      \begin{align*}
        \texttt{Shift}(\bar{x},\bar{z},i,j,s) & \dfff  (\forall k)[i \leq k \leq j \to x_k = z_{k+s}];\\
        \texttt{Inv}(\bar{x},\bar{z},i,j) & \dfff (\forall k)[i \leq k \leq j \to x_k = z_{n-k}].
      \end{align*}
      Трябва да докажем, че програмата е частично коректна относно 
      \begin{align*}
        & I(\bar{z},n) \dfff \bar{z} \in \Int^{n+1}\ \&\ n \geq 0;\\
        & O(\bar{z},\bar{x},n) \dfff \texttt{Inv}(\bar{x},\bar{z},0,n).
      \end{align*}
      Разгледайте свойствата:
      \begin{align*}
        A_1(\bar{z},\bar{x},i,j,y,n) \dfff & \bar{z} \in \Int^{n+1}\ \&\ n \geq 0\\
        A_2(\bar{z},\bar{x},i,j,y,n) \dfff & 0 \leq i \leq n+1\ \&\\
        & \texttt{Inv}(\bar{x},\bar{z},n+1-i,n)\ \&\\
        & \texttt{Shift}(\bar{x},0,n-i,i);\\
        A_3(\bar{z},\bar{x},i,j,y,n) \dfff & y = z_i\ \&\ 0 \leq j \leq n-i\ \&\ 0 \leq i\\
        & \texttt{Inv}(\bar{x},\bar{z},n+1-i,n)\ \&\\
        & \texttt{Shift}(\bar{x},0,j-1,i+1)\ \&\\
        & \texttt{Shift}(\bar{x},j,n-i,i);\\
        A_4(\bar{z},\bar{x},i,j,y,n) \dfff & \texttt{Inv}(\bar{x},\bar{z},0,n).
      \end{align*}
      Докажете, че за всеки преход $(k) \to (l)$ имаме
      \[A_k(\bar{z},\bar{x},i,j,y,n) \implies A_l(\bar{z},f_{kl}(\bar{x},i,j,y),n).\]
    }
  \end{subfigure}
% \end{wrapfigure}
\end{figure}


%%% Local Variables:
%%% mode: latex
%%% TeX-master: "../sep-problems"
%%% End:

\subsection*{Функцията 91 на Макарти}

\begin{figure}[H]
  \begin{subfigure}[b]{0.6\textwidth}
  \begin{tikzpicture}[node distance = 3cm,auto,scale=0.6, every node/.style={scale=0.9}]
    % Place nodes
    \node [cloud] (init) {вход: $x$};
    \node [label, below of=init, node distance=1.5cm] (n1) {\scriptsize{1}};
    \node [tallblock, below of=init] (identify) {$y := x$\\$t := 1$};
    \node [label, below of=identify,node distance=1.5cm] (n2) {\scriptsize{2}};
    \node [decision, below of=identify] (evaluate) {$y \geq 101$};
    \node [bigblock, below of=evaluate, node distance=2cm] (min) {$y := y - 10$\\$t := t - 1$};
    \node [label, below of=min, node distance=1.5cm] (n3) {\scriptsize{3}};
    \node [bigblock, right of=evaluate] (pls) {$y := y + 11$\\$t := t + 1$};
    \node [decision, below of=n3, node distance=1.5cm] (ready) {$t = 0$};
    \node [label, right of=ready, node distance=2cm] (n4) {\scriptsize{4}};
    \node [cloud, right of=n4] (exit) {изход: $y$};
    % \coordinate[left of=ready] (dummy);

  % Draw edges
    \path [line] (init) -- (n1);
    \path [line] (n1) -- (identify);
    \path [line] (identify) -- (n2);
    \path [line] (n2) -- (evaluate);
    \path [line] (evaluate) -- node {да} (min);
    \path [line] (evaluate) -- node {не} (pls);
    \path [line] (pls) |- node {} (n2);
    \path [line] (min) -- node  {} (n3);
    \path [line] (n3) -- node  {} (ready);
    \path [line] (ready) -- node {да} (n4);
    \path [line] (n4) -- node {} (exit);
    % \path [line] (ready) -- node {} ();
    % \path [line] (dummy) |- node {} (n2);
    \path [line] (ready) -- node [above] {не} (-3,-16.5) -- (-3,-6.75) -- (n2);
  \end{tikzpicture}
  \caption{Итеративна версия на функцията 91 на Макарти}
  \label{fig:gcd}
  \end{subfigure}
  ~
  \begin{subfigure}[b]{0.6\textwidth}
    \footnotesize{
      Да разгледаме свойствата:
      \begin{align*}
        A_1(x,y,t) \dfff & x \in \Nat\ \&\ x \leq 100\\
        A_2(x,y,t) \dfff & (t \geq 2 \implies y \leq 111)\ \&\\
                         & (t = 1 \implies y \leq 101)\\
        A_3(x,y,t) \dfff & (t \geq 1 \implies y \leq 101)\ \&\\
                         & (t = 0 \implies y = 91)\ \& \\
        & 91 \leq y \leq 101\\
        A_4(x,y,t) \dfff & y = 91.
      \end{align*}
      Докажете, че за всеки преход $(k) \to (l)$ имаме
      \[(\forall x,y,z,t,v)[A_k(x,y,z,t,v) \implies A_l(x,y,f_{kl}(z,t,v))].\]
    }
  \end{subfigure}
\end{figure}

Нека да означим $y_i$, $t_i$ стойностите на променливите $y$ и $t$ точно след $i$-тото преминаване през $(2)$.
За да докажете тотална коректност, използвайте, че редицата
\[\{(101 - y_i + 10t_i, t_i) \mid i = 0,1,\dots\}\]
е строго намаляваща относно лексикографската наредба.

% С други думи, $(\Nat\times\Nat, \prec)$ е фундирана наредба, където 
% \[(a,b) \prec (a',b') \iff \]

%%% Local Variables:
%%% mode: latex
%%% TeX-master: "../sep-problems"
%%% End:


% \begin{problem}
%   Нека е дадена програмата на езика хаскел:

%   \begin{minted}[frame=lines,framesep=2mm,baselinestretch=1.2]{haskell}
%     rev :: [a] -> [a]
%     rev x = f(x, []) where 
%       f([], y) = y
%       f(x:xs, y) = f(xs, x:y)
%   \end{minted}

%   \noindent 
%   Докажете, че:
%   \begin{enumerate}[a)]
%   \item 
%     $rev:\Sigma^\star \to \Sigma^\star$ е тотална.
%   \item
%     $(\forall x \in \Sigma^\star)[rev(rev(x)) = x]$.
%   \item
%     $(\forall x \in \Sigma^\star)[rev(x) = x^R]$.
%   \end{enumerate}
% \end{problem}
% \begin{hint}
%   % Ще използваме следното правило:
%   % \begin{prooftree}
%   %   \AxiomC{$P(\varepsilon)$}
%   %   \AxiomC{$(\forall x \in \Sigma^\star)[x\neq\varepsilon\ \&\ P(cdr(x)) \to P(x)]$}
%   %   \RightLabel{\scriptsize(1)}
%   %   \BinaryInfC{$(\forall x\in\Sigma^\star)[P(x)]$}
%   % \end{prooftree}
%   % % \item
%   %   Докажете валидността на правилото $(1)$. % е еквивалентно на структурна индукция върху фундираната наредба
%     % $(\Sigma^\star,\prec)$, където $x \prec y \iff (\exists z\in\Sigma^\star)[z\cdot x = y]$, т.е.
%     % $x$ е суфикс на $y$.
%   Да разгледаме фундираната наредба $L = (\Sigma^\star, \prec)$, където
%   $x \prec y \iff \abs{x} < \abs{y}$.
%   \begin{enumerate}[a)]
%   \item 
%     Да разгледаме свойството 
%     \[P(x) \equiv (\forall y\in \Sigma^\star)[f(x,y)\text{ е дефинирана}].\]
%     Докажете със структурна индукция по $L$, че $(\forall x\in\Sigma^\star)[P(x)]$.
%   \item
%     Да разгледаме свойството 
%     \[P(x) \equiv (\forall y\in \Sigma^\star)[rev(f(x,y)) = f(y,x)].\]
%     Докажете със структурна индукция по $L$, че $(\forall x\in\Sigma^\star)[P(x)]$.
%     Тогава в частния случай $y = \varepsilon$, 
%     \[rev(rev(x)) = rev(f(x,\varepsilon)) = f(\varepsilon,x) = x.\]
%   \item
%     Разгледайте свойството
%     \[P(x) \equiv (\forall y\in\Sigma^\star)[f(x,y) = x^R \cdot y].\]
%   \end{enumerate}
% \end{hint}

% \begin{problem}
%   Нека е дадена програмата на езика хаскел:

%   \begin{minted}[]{haskell}
%     concat :: ([a], [a]) -> [a]
%     concat([], y) = y
%     concat(x:xs, y) = x:concat(xs, y)
%   \end{minted}
%   \noindent Докажете, че:
%   \begin{enumerate}[a)]
%   \item
%     $(\forall x,y\in\Sigma^\star)[concat(x, y) = x \cdot y]$;
%   \item 
%     $(\forall x,y,z\in\Sigma^\star)[concat(concat(x, y), z) = concat(x, concat(y, z))]$;
%   \item
%     $(\forall x,y\in\Sigma^\star)[concat(x, y)^R = concat(y^R, x^R)]$.
%   \end{enumerate}
% \end{problem}
% \begin{hint}
%   Да разгледаме фундираната наредба $L = (\Sigma^\star, \prec)$, където
%   $x \prec y \iff \abs{x} < \abs{y}$.
%   \begin{enumerate}[a)]
%   \item
%     Разгледайте
%     \[P(x) \equiv (\forall y\in\Sigma^\star)[concat(x, y) = x \cdot y].\]
%   \item 
%     Разгледайте 
%     \[P(x) \equiv (\forall y,z\in\Sigma^\star)[concat(concat(x, y), z) = concat(x, concat(y, z))].\]
%   \item
%     Разгледайте
%     \[P(x) \equiv (\forall y\in\Sigma^\star)[concat(x, y)^R = concat(y^R, x^R)].\]
%   \end{enumerate}
% \end{hint}

% \begin{problem}
%   Да разгледаме следната програма
%   \begin{haskellcode}
%     f :: Int -> Int
%     f(x) = if x > 100 then x - 10
%              else f(f(x + 11))
%   \end{haskellcode}
%   Докажете, че 
%   \[f(x) = \begin{cases}
%     x - 10, & x > 100\\
%     91, & x \leq 100.
%   \end{cases}\]
% \end{problem}
% \begin{hint}
%   Разгледайте строгата частична наредба $\A = (\{x \in \Int \mid x \leq 100\}, \prec)$, където
%   \[x \prec y \iff y < x.\]
%   Лесно се съобразява, че наредбата $\A$ е фундирана.
%   Да разгледаме свойството \[P(x) \equiv f(x) = 91.\]
%   \begin{itemize}
%   \item 
%     Лесно се вижда, че $P(100)$.
%   \item
%     Нека $x < 100$. 
%     Да приемем, че $(\forall z \prec x)[P(z)]$. Ще докажем, че $P(x)$.
%     Тук трябва да разгледаме два подслучая в зависимост от това дали $x+11 \leq 100$ или $x+11 > 100$.
%   \end{itemize}
% \end{hint}

% \begin{problem}
%   Да разгледаме следната програма:
%   \begin{minted}[]{haskell}
%     f :: (Int, Int) -> Int
%     f(x, y) = if x == y then 1
%              else (y+2)*(y+1)*f(x, y + 2)
%   \end{minted}
%   Докажете, че \[(\forall x,y\in\Nat)[x \geq y\ \&\ (x-y) \equiv 0 \bmod 2 \implies f(x,y) = \frac{x!}{y!}].\]
% \end{problem}
% \begin{hint}
%   Ще използваме структурна индукция върху $(\Nat, <)$.
%   Да разгледаме свойството
%   \[P(z) \equiv (\forall x,y\in\Nat)[z = x - y \geq 0\ \&\ z \equiv 0 \bmod 2 \implies f(x,y) = \frac{x!}{y!}].\]
% \end{hint}



% \input{verification/induction}
% \input{verification/permutations}


%%% Local Variables: 
%%% mode: latex
%%% TeX-master: "sep-problems"
%%% End: 

% include{dataflow}
% \include{abstract-interpretation}
%\include{extra}
% \include{struct-ind}

% \include{tail}
%\backmatter

\bibliographystyle{plain}
\bibliography{sep}

\printindex

\end{document}


%%% Local Variables:
%%% mode: latex
%%% TeX-master: t
%%% End:
