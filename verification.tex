\chapter{Верификация на програми по метода на Флойд}

\tikzstyle{decision} = [diamond, draw, fill=green!10, text width=4em, text badly centered, node distance=3cm, inner sep=0pt]
\tikzstyle{block} = [rectangle, draw, fill=red!10, text width=5em, text centered, rounded corners, minimum height=2em]
\tikzstyle{tallblock} = [rectangle, draw, fill=red!10, text width=4em, rounded corners, minimum height=3em]
\tikzstyle{line} = [draw, -latex']
\tikzstyle{bigblock} = [rectangle, draw, fill=red!10, text width=7em, rounded corners, minimum height=3em]
\tikzstyle{label} = [draw,circle,fill=yellow!20,node distance=3.2cm]
\tikzstyle{cloud} = [draw, ellipse, text width=4em, text centered, fill=blue!10, node distance=2cm, minimum height=2em]


Верификация на итеративни програми по метода на индуктивните твърдения на Флойд \cite{floyd-verification}.

% \section{Метод на Флойд}

\marginpar{Този метод е описан за първи път от Робърт Флойд \cite{floyd-verification}}
\marginpar{Тук на практика следваме \cite[Глава 3]{manna} и \cite[Глава 1]{nikolova-soskova}}

\section{Програми с един цикъл}

\subsection{Решени задачи}

\subsection*{Намиране на НОД}

Да дефинираме функцията $\texttt{gcd}:\Nat\times\Nat \to \Nat$, където
\[\texttt{gcd}(x,y) = (\max z)[\ z\ |\ x\ \&\ z\ |\ y\ ].\]
\begin{itemize}
\item 
  Понеже $0\ |\ 0$, то $\text{НОД}(0,0) = 0$.
\item
  Понеже всяко естествено число дели $0$, то $\text{НОД}(0,z) = z$;
\item
  Ясно е от дефиницията, че $\text{НОД}(x,y) = \text{НОД}(y,x)$;
\end{itemize}

\begin{prop}
  \label{pr:gcd}
  $(\forall x,y\in\Nat)[\text{НОД}(x,y) = \text{НОД}(y, x \bmod y)]$.
\end{prop}
\begin{proof}
  Ще разгледаме няколко случая.
  \begin{itemize}
  \item 
    Ако $x = y$, то всичко е ясно, защото $x \bmod x = x$.
  \item
    Ако $x < y$, то всичко е ясно, защото $x \bmod y = x$ и
    \[\texttt{gcd}(x,y) = \texttt{gcd}(y,x) = \texttt{gcd}(y, x \bmod y).\]
  \item
    Нека $x > y$ и $z = \texttt{gcd}(x,y)$.
    Тогава $z$ е най-голямото естествено число, което $z\ |\ x$ и $z\ |\ y$.
    Нека $r = x \bmod y$. Тогава $x = ky + r$, $0 \leq r < y$.
    Щом $z$ дели $x$ и $y$, то е ясно, че $z$ дели $r$.
    Остана да съобразим защо $z = \texttt{gcd}(y,r)$.
    Да допуснем, че съществува естествено число $z'$, такова че $z < z'$ и $z'$ дели $y$ и $r$.
    Тогава $z'$ също дели и $x = ky + r$. Достигнахме до противоречие с факта, че $z = \texttt{gcd}(x,y)$.
    Следователно,
    \[z = \texttt{gcd}(x,y) = \texttt{gcd}(y,x \bmod y).\]
  \end{itemize}
\end{proof}

\begin{figure}[H]
  \begin{subfigure}[b]{0.6\textwidth}
  \begin{tikzpicture}[node distance = 3cm,auto,scale=0.6, every node/.style={scale=0.9}]
    % Place nodes
    \node [cloud] (init) {вход: $x,y$};
    \node [label, below of=init, node distance=1.5cm] (n1) {\scriptsize{1}};
    \node [tallblock, below of=init] (identify) {$z := x$\\$t := y$};
    \node [label, below of=identify,node distance=1.5cm] (n2) {\scriptsize{2}};
    \node [decision, below of=identify] (evaluate) {$t = 0$};
    \node [bigblock, below of=evaluate, node distance=2cm] (ji) {$v := t$\\$t := z \bmod t$\\$z := v$};
    \coordinate[right of=evaluate,node distance=2cm] (inc);
    \node [label, left of=evaluate, node distance=2cm] (n3) {\scriptsize{3}};
    \node [cloud, left of=n3] (exit) {изход: $z$};

  % Draw edges
    \path [line] (init) -- (n1);
    \path [line] (n1) -- (identify);
    \path [line] (identify) -- (n2);
    \path [line] (n2) -- (evaluate);
    \path [line] (evaluate) -- node {не} (ji);
    \draw [-] (ji) -| node {} (inc);
    \path [line] (inc) |- node  {} (n2);
    \path [line] (evaluate) -- node {да} (n3);
    \path [line] (n3) -- node {} (exit);
  \end{tikzpicture}
  \caption{Алгоритъм за намиране на $\text{НОД}(x,y)$}
  \label{fig:gcd}
  \end{subfigure}
  ~
  \begin{subfigure}[b]{0.6\textwidth}
    \footnotesize{
      Да разгледаме свойствата:
      \begin{align*}
        A_1(x,y,z,t,v) \dfff & x,y \in \Nat\\
        A_2(x,y,z,t,v) \dfff & \text{НОД}(x,y) = \text{НОД}(z,t)\ \&\\
        & x,y,z,t,v\in\Nat\\
        A_3(x,y,z,t,v) \dfff & z = \text{НОД}(x,y)
      \end{align*}
      Докажете, че за всеки преход $(k) \to (l)$ имаме
      \[(\forall x,y,z,t,v)[A_k(x,y,z,t,v) \implies A_l(x,y,f_{kl}(z,t,v))].\]
    }
    \end{subfigure}
\end{figure}

\begin{prop}
  Докажете, че програмата на Фигура \ref{fig:gcd} е тотално коректна относно входното условие $x,y\in\Nat$
  и изходното условие $z = \text{НОД}(x,y)$.
\end{prop}
\begin{hint}
  Доказателството на $A_2(x,y,z,t,v) \implies A_2(x,y,f_{22}(z,t,v))$ следва директно от \Prop{gcd}.
  За доказателството на $A_2(x,y,z,t,v) \implies A_3(x,y,f_{23}(z,t,v))$ е достатъчно да съобразим, че ако имаме $A_2(x,y,z,t,v)$ и $t = 0$, то
  $\text{НОД}(x,y) = \text{НОД}(z,0) = z$.
  
  Остана да докажем, че програмата винаги завършва при входни данни $x,y \in \Nat$.
  Да означим с $t_i$ стойността на променливата $t$ след $i$-тото преминаване през етикет $(2)$.
  Това означава, че $t_0 = y$ и $t_{i+1} = z \bmod t_i$, за някое $z$.
  Следователно, $t_{i+1} < t_i$. От $A_2$ имаме, че всички $t_i \geq 0$.
  Така получаваме строго монотонно намаляваща редица $t_0 > t_1 > \cdots > t_i > \cdots \geq 0$, която е ограничена отдолу от $0$.
  Заключаваме, че съществува $i$, за което $t_i = 0$. Следователно, програмата завършва.
\end{hint}


%%% Local Variables:
%%% mode: latex
%%% TeX-master: "../sep-problems"
%%% End:


\subsection*{Търсене на максимална сума}

\begin{problem}
  % Да разгледаме следния алгоритъм:
  % \begin{algorithm}[H]
  %   \caption{}
  %   \label{alg:useless}
  %   \begin{algorithmic}[1]
  %     \State $y := x[0]$
  %     \State $z := x[0]$
  %     \State $i := 1$
  %     \For{$i := 1; i < n; i++$}
  %     \State $z := \max\{x[i], z + x[i]\}$
  %     \State $y := \max\{x[i], z + x[i]\}$
  %     \EndFor
  %     \State \Return $y$
  %   \end{algorithmic}
  % \end{algorithm}
  

  Докажете, че програмата $P$, описана с блок-схемата на \Fig{max-sum}, 
  пресмята $\max\{\sum^l_{i=k}x_i \mid 0 \leq k \leq l \leq n\}$.
\end{problem}

\begin{figure}[H]
  \begin{tikzpicture}[node distance = 3cm,auto,scale=0.6, every node/.style={scale=0.9}]
    % Place nodes
    \node [cloud] (init) {вход: $x_0,\dots,x_n$};
    \node [label, below of=init, node distance=1.5cm] (n1) {\scriptsize{1}};
    \node [tallblock, below of=init] (identify) {$y := x_0$\\$z := x_0$\\$i := 1$};
    \node [label, below of=identify,node distance=1.5cm] (n2) {\scriptsize{2}};
    \node [decision, below of=identify] (evaluate) {$i \leq n$};
    \node [bigblock, below of=evaluate, node distance=2cm] (ji) {$\scriptstyle{z := \max\{x_i,z+x_i\}}$\\$\scriptstyle{y := \max\{y,z\}}$};
    \node [block, right of=evaluate, node distance=2.5cm] (inc) {$i := i+1$};
    \node [label, left of=evaluate, node distance=2cm] (n3) {\scriptsize{3}};
    \node [cloud, left of=n3] (exit) {изход: $y$};

  % Draw edges
    \path [line] (init) -- (n1);
    \path [line] (n1) -- (identify);
    \path [line] (identify) -- (n2);
    \path [line] (n2) -- (evaluate);
    \path [line] (evaluate) -- node {да} (ji);
    \path [line] (ji) -| node {} (inc);
    \path [line] (inc) |- node  {} (n2);
    \path [line] (evaluate) -- node {не} (n3);
    \path [line] (n3) -- node {} (exit);
  \end{tikzpicture}
  \caption{Ще докажем, че $y = \max\{\sum^l_{i=k}x_i \mid 0 \leq k \leq l \leq n\}$}
% \end{wrapfigure}
  \label{fig:max-sum}
\end{figure}

Първо ще разгледаме две твърдения, които ще ни подскажат какво представляват междинните стойности на променливите $z$ и $y$ в програмата на \Fig{max-sum}.

\begin{prop}
  \label{pr:Z}
  Нека е дадена редицата $x_0,\dots,x_n \in \Int$.
  Да дефинираме:
  \begin{itemize}
  \item
    $Z(\bar{x},0) = x_0$;
  \item
    $Z(\bar{x},i+1) = \max\{x_{i+1}, Z(\bar{x},i)+x_{i+1}\}$.
  \end{itemize}
  Тогава за всяко $i \leq n$, $Z(\bar{x},i) = \max\{\sum^i_{j  = k}x_j \mid 0 \leq k \leq i\}$.
\end{prop}
\begin{proof}
  Индукция по $i$.
  За $i = 0$ е ясно, защото 
  \[Z(\bar{x},0) = x_0 = \max\{\sum^0_{j=k}x_j \mid 0\leq k\leq 0\}.\]
  Ще докажем твърдението за $i+1$.
  \begin{align*}
    Z(\bar{x},i+1) & = \max\{Z(\bar{x},i)+x_{i+1},x_{i+1}\} & (\text{от деф.})\\
    & = \max\{\max\{\sum^{i}_{j  = k}x_j \mid 0 \leq k \leq i\}+x_{i+1}, x_{i+1}\} & (\text{от И.П.})\\
    & = \max\{\max\{\sum^{i+1}_{j  = k}x_j \mid 0 \leq k \leq i\}, \sum^{i+1}_{j=i+1}x_j\}\\
    & = \max\{\sum^{i+1}_{j  = k}x_j \mid 0 \leq k \leq i+1\}.
  \end{align*}
\end{proof}

\begin{prop}
  \label{pr:Y}
  Нека е дадена редицата $x_0,\dots,x_n \in \Int$.
  Да дефинираме:
  \begin{itemize}
  \item 
    $Y(\bar{x},0) = x_0$;
  \item
    $Y(\bar{x},i+1) = \max\{Y(\bar{x},i), Z(\bar{x},i+1)\}$.
  \end{itemize}
  Тогава за всяко $i \leq n$, $Y(\bar{x},i) = \max\{Z(\bar{x},l) \mid 0 \leq l \leq i\}$.
\end{prop}
\begin{proof}
  Отново индукция по $i$.
  За $i = 0$ е очевидно, защото
  \[Y(\bar{x},0) = x_0 = Z(\bar{x},0) = \max\{Z(\bar{x},l) \mid 0 \leq l \leq 0\}.\]
  Ще докажем твърдението за $i+1$.
  \begin{align*}
    Y(\bar{x},i+1) & = \max\{Y(\bar{x},i), Z(\bar{x},i+1)\} & (\text{от деф.})\\
    & = \max\{\max\{Z(\bar{x},l) \mid 0 \leq l \leq i\}, Z(\bar{x},i+1)\} & (\text{от И.П.})\\
    & = \max\{Z(\bar{x},l) \mid 0 \leq l \leq i+1\}.
  \end{align*}
\end{proof}

\begin{cor}
  \label{cr:Y}
  $Y(\bar{x},n) = \max\{\sum^l_{i=k}x_i \mid 0\leq k \leq l \leq n\}$.
\end{cor}

Сега сме готови да дефинираме свойствата $A_l$ за етикетите $l = 1,2,3$:
\begin{align*}
  & A_1(\bar{x},i,y,z,n) \equiv \bar{x} \in \Int^{n+1};\\
  & A_2(\bar{x},i,y,z,n) \equiv y = Y(\bar{x},i-1)\ \&\ z = Z(\bar{x},i-1)\ \&\ 1 \leq i \leq n+1;\\
  & A_3(\bar{x},i,y,z,n) \equiv y = Y(\bar{x},n).
\end{align*}  

За всеки директен преход $(k) \to (l)$ между етикети в блок схемата, асоциираме функция $f_{kl}$,
която показва как се изменят стойностите на променливите участващи в програмата на \Fig{max-sum}:
\begin{align*}
  & f_{12}(\bar{x},i,y,z,n) = (\bar{x},1,x_0,x_0,n);\\
  & f_{22}(\bar{x},i,y,z,n) = (\bar{x},i+1,max\{x_i,z+x_i\},\max\{y,z\},n);\\
  & f_{23}(\bar{x},i,y,z,n) = (\bar{x},i,y,z,n).
\end{align*}

\begin{prop}
  За всеки директен преход между етикети $(k) \to (l)$ е изпълнена импликацията:
  \[(\forall\bar{x}\in\Int^{n+1})(\forall i,j\in\Int)[A_k(\bar{x},i,j,n) \implies A_l(f_{kl}(\bar{x},i,j,n))].\]
\end{prop}
\begin{proof}
  \begin{description}
  \item[($1 \to 2$)] 
    Следва директно от дефинициите на $Y(\bar{x},0)$ и $Z(\bar{x},0)$.
  \item[($2 \to 2$)] 
    Следва директно от \Prop{Z} и \Prop{Y}.
    Ясно е също, щом преминаваме $2 \to 2$, то $1 \leq i+1 \leq n+1$.
  \item[($2 \to 3$)] 
    Понеже $i \leq n+1$ и прехода $2\to 3$ ни дава, че $i > n$, 
    то следва, че $i = n+1$. Тогава $Y(i-1) = Y(n)$.
    Сега прилагаме \Cor{Y}.
  \end{description}
\end{proof}
\begin{cor}
  Програмата от \Fig{max-sum} е частично коректна относно 
  входното условие $I(\bar{x},n) \equiv \bar{x} \in \Int^{n+1}$ и изходното условие $O(\bar{x},n,y) \equiv y = \max\{\sum^l_{i=k}x_i \mid 0\leq k \leq l \leq n\}$.
\end{cor}


%%% Local Variables:
%%% mode: latex
%%% TeX-master: "../sep-problems"
%%% End:


\subsection{Задачи с упътване}


\begin{figure}[H]
  \begin{subfigure}[b]{0.6\textwidth}
  \begin{tikzpicture}[node distance = 3cm,auto,scale=0.6, every node/.style={scale=0.9}]
    % Place nodes
    \node [cloud] (init) {вход: $n$};
    \node [label, below of=init, node distance=1.5cm] (n1) {\scriptsize{1}};
    \node [tallblock, below of=init] (identify) {$x := 0$\\$y := 1$\\$s := 1$};
    \node [label, below of=identify,node distance=1.5cm] (n2) {\scriptsize{2}};
    \node [decision, below of=identify] (evaluate) {$s \leq n$};
    \node [bigblock, below of=evaluate, node distance=2cm] (ji) {$x := x + 1$\\$y := y + 2$\\$s := s + y$};
    \coordinate[right of=evaluate,node distance=2cm] (inc);
    \node [label, left of=evaluate, node distance=2cm] (n3) {\scriptsize{3}};
    \node [cloud, left of=n3] (exit) {изход: $x$};

  % Draw edges
    \path [line] (init) -- (n1);
    \path [line] (n1) -- (identify);
    \path [line] (identify) -- (n2);
    \path [line] (n2) -- (evaluate);
    \path [line] (evaluate) -- node {да} (ji);
    \draw [-] (ji) -| node {} (inc);
    \path [line] (inc) |- node  {} (n2);
    \path [line] (evaluate) -- node {не} (n3);
    \path [line] (n3) -- node {} (exit);
  \end{tikzpicture}
  \caption{Алгоритъм за намиране на $\lfloor{\sqrt{n}}\rfloor$}
  \label{fig:sqrt}
  \end{subfigure}
  ~
  \begin{subfigure}[b]{0.6\textwidth}
    \footnotesize{
      Да разгледаме свойствата:
      \begin{align*}
        A_1(x,y,s,n) \dfff & n \in \Nat\\
        A_2(x,y,s,n) \dfff & x,n\in\Nat\ \&\ x^2 \leq n\ \&\\
        & y = 2x+1\ \&\ s = (x+1)^2\\
        A_3(x,y,s,n) \dfff & x^2 \leq n < (x+1)^2
      \end{align*}
      Съобразете, че $A_3(x,y,s,n) \implies x = \lfloor{\sqrt{n}}\rfloor$.
      Докажете, че за всеки преход $(k) \to (l)$ имаме
      \[(\forall x,y,s,n)[A_k(x,y,s,n) \implies A_l(f_{kl}(x,y,s),n)].\]
    }
    \end{subfigure}
\end{figure}

\begin{problem}
  Докажете, че програмата $P$ описана с блок-схемата на Фигура \ref{fig:sqrt} е тотално коректна относно входното 
  условие $n \in \Nat$ и изходното условие $\lfloor{\sqrt{n}}\rfloor$.
\end{problem}


%%% Local Variables:
%%% mode: latex
%%% TeX-master: "../sep-problems"
%%% End:


\section{Програми с два цикъла}

% \subsection*{Сортиране на масив}
\subsection{Решени задачи}


\subsection*{Сортиране чрез вмъкване}
\begin{problem}
  Докажете, че програмата, описана с блок-схемата на \Fig{insertion-sort}, 
  сортира входния масив във възходящ ред.
\end{problem}

%\begin{wrapfigure}{r}{0.5\textwidth}
\begin{figure}[H]
  \begin{tikzpicture}[node distance = 2.75cm,auto,scale=0.6, every node/.style={scale=0.9}]
    % Place nodes
    \node [cloud] (init) {вход: $z_0,\dots,z_n$};
    \node [label, below of=init, node distance=1.25cm] (n1) {\scriptsize{1}};
    \node [block, below of=init, node distance=2.25cm] (identify) {$\bar{x} := \bar{z}$\\$i := 1$};
    \node [label, below of=identify,node distance=1.5cm] (n2) {\scriptsize{2}};
    \node [decision, below of=identify] (evaluate) {$i \leq n$};
    \node [block, below of=evaluate, node distance=2cm] (ji) {$j := i$};
    \node [block, right of=ji, node distance=2.5cm] (inc) {$i := i+1$};
    \node [label, below of=evaluate] (n3) {\scriptsize{3}};
    \node [decision, below of=ji] (loop) {$\scriptstyle{j\geq 1\ \&}$\\$\scriptstyle{x_j < x_{j-1}}$};
    \node [block, left of=loop] (swap) {$\scriptstyle{swap(x_j,x_{j-1})}$\\$\scriptstyle{j := j-1}$};
    \node [cloud, left of=evaluate, node distance=4cm] (exit) {изход: $x_0,\dots,x_n$};
    \node [label, left of=evaluate, node distance=2cm] (n4) {\scriptsize{4}};
  % Draw edges
    \path [line] (init) -- (n1);
    \path [line] (n1) -- (identify);
    \path [line] (identify) -- (n2);
    \path [line] (n2) -- (evaluate);
    \path [line] (evaluate) -- node {да} (ji);
    \path [line] (ji) -- node {} (n3);
    \path [line] (n3) -- node {} (loop);
    \path [line] (loop) -- node {да} (swap);
    \path [line] (swap) |- node {} (n3);
    \path [line] (loop) -| node [below] {не} (inc);
    \path [line] (inc) |- node  {} (n2);
    \path [line] (evaluate) -- node {не} (n4);
    \path [line] (n4) -- node {} (exit);
  \end{tikzpicture}
  \caption{По даден входен масив $\bar{x}\in\Int^{n+1}$, програмата го сортира във възходящ ред (\href{http://en.wikipedia.org/wiki/Insertion_sort}{insertion sort})}
% \end{wrapfigure}
  \label{fig:insertion-sort}
\end{figure}

\noindent Удобно е да означим:
\[\texttt{Ord}(\bar{x},i,j) \dfff (\forall k)[i \leq k < j \implies x_k \leq x_{k+1}],\]
което ни казва, че елемените в интервала $[i,j]$ на масива $\bar{x}$ са подредени във възходящ ред.
Също така, да означим:
\begin{align*}
  \texttt{Perm}(\bar{z},\bar{x},n) \dfff \bar{z},\bar{x} \in \Int^{n+1}\ \&\ (\exists f)[&f:\{0,\dots,n\}\to\{0,\dots,n\} \text{ е биекция }\&\\
  & (\forall i)[0\leq i \leq n \implies z_i = x_{f(i)}]].
\end{align*}
Това означава, че $\bar{z}$ е пермутация на елементите на $\bar{x}$.

\marginpar{Понеже масивът $\bar{z}$ и $n$ са константни, то няма нужда да описваме как се променят с функциите $f_{kl}$}
За всеки директен преход $(k) \to (l)$ между етикети в блок схемата, асоциираме функция $f_{kl}$,
която показва как се изменят стойностите на променливите участващи в програмата.
\begin{align*}
  & f_{12}(\bar{x},i,j) \dff (\bar{x},1,j)\\
  & f_{23}(\bar{x},i,j) \dff (\bar{x}, i, i)\\
  & f_{24}(\bar{x},i,j) \dff (\bar{x}, i, j)\\
  & f_{32}(\bar{x},i,j) \dff (\bar{x}, i+1, j)\\
  & f_{33}(\bar{x},i,j) \dff (\bar{x}',i,j-1),
\end{align*}
където $\bar{x}' = (x'_0,\dots,x'_n)$ е променения масив, за който
$x'_{j-1} = x_{j}$, $x'_j = x_{j-1}$, а $x'_k = x_k$ за всеки индекс $k$ в интервала $[0,n] \setminus\{j-1,j\}$.

Към всеки етикет $(l)$ в блок схемата на програмата $P$ на \Fig{insertion-sort}, асоциираме предиката $A_l$, където:
\begin{align*}
  A_1(\bar{z},\bar{x},i,j,n) \dfff\ & \bar{z} \in \Int^{n+1}\ \&\ n \geq 0\\
  A_2(\bar{z},\bar{x},i,j,n) \dfff\ & \texttt{Perm}(\bar{z},\bar{x},n)\ \&\ \texttt{Ord}(\bar{x},0,i-1)\ \&\ 0 \leq i \leq n+1\\
  A_3(\bar{z},\bar{x},i,j,n) \dfff\ & \texttt{Perm}(\bar{z},\bar{x},n)\ \&\ \texttt{Ord}(\bar{x},0,j-1)\ \&\ \texttt{Ord}(\bar{x},j,i)\ \&\\
  & 0\leq j \leq i \leq n\ \&\ (0 < j < i \implies x_{j-1} \leq x_{j+1})\\
  A_4(\bar{z},\bar{x},i,j,n) \dfff\ & \texttt{Perm}(\bar{z},\bar{x},n)\ \&\ \texttt{Ord}(\bar{x},0,n).
\end{align*}

\begin{prop}
  За всеки директен преход между етикети $(k) \to (l)$ е изпълнена импликацията:
  \[(\forall\bar{z},\bar{x},i,j,n)[A_k(\bar{z},\bar{x},i,j,n) \implies A_l(\bar{z},f_{kl}(\bar{x},i,j),n)].\]
\end{prop}
\begin{proof}
  \begin{description}
  \item[($1\to 2$)]
    Очевидно е, че имаме $A_2(\bar{z},\bar{x},1,j,n)$, т.е. $\texttt{Ord}(\bar{x},0,1-1)$ и $1 \leq n+1$.
  \item[($2 \to 3$)]
    Нека $A_2(\bar{z},\bar{x},i,j,n)$.
    Ще докажем, че имаме $A_3(\bar{z},\bar{x},i,i,n)$. Това е съвсем лесно:
    \begin{itemize}
    \item 
      От $A_2$ е ясно, че имаме $\texttt{Ord}(\bar{x},0,i-1)$.
    \item
      Очевидно е, че имаме $\texttt{Ord}(\bar{x},i,i)$.
    \item
      От $A_2$ имаме, че $i < n+1$. Но понеже от етикет 2 сме отишли в етикет 3, 
      то $i \neq n+1$. Следователно, $0 \leq i \leq i \leq n$.
    \item
      Импликацията $0 < i < i \implies x_{i-1}\leq x_{i+1}$  
      е изпълнена по тривиални причини.
    \end{itemize}
  \item[($3\to 3$)]
  Нека $A_3(\bar{z},\bar{x},i,j,n)$.
  Ще докажем, че $A_3(\bar{z},\bar{x}',i,j-1,n)$.
  Понеже сме направили преход $3 \to 3$,
  то имаме също свойството, че $j \geq 1$ и $x_{j} < x_{j-1}$.
  Това означава, че:
  \begin{align*}
    \bar{x} =\ & \overbrace{x_0 \leq \dots \leq x_{j-2}\leq x_{j-1}}^{\texttt{Ord}} > \overbrace{x_j \leq x_{j+1} \leq \cdots \leq x_i}^{\texttt{Ord}}\\
    \bar{x}' =\ & \underbrace{x_0 \leq \dots \leq x_{j-2}}_{\texttt{Ord}} \square \underbrace{x'_{j-1}}_{x_j} < \underbrace{x'_j}_{x_{j-1}} \square \underbrace{x_{j+1} \leq \cdots \leq x_i}_{\texttt{Ord}}
  \end{align*}
  
  \begin{itemize}
  \item 
    Ясно е, че $0 \leq j-1 \leq i$;
  \item
    От $\texttt{Ord}(\bar{x},0,j-1)$ следва, че $\texttt{Ord}(\bar{x}',0,j-2)$,
    защото единствената промяна в масива е размяната на стойностите
    на $x_{j-1}$ и $x_j$.
  \item
    За да докажем, че $\texttt{Ord}(\bar{x}',j-1,i)$, трябва да разгледаме два случая.
    \begin{itemize}
    \item
      Ако $j = i$, тогава $x_i < x_{i-1}$. Да проверим, че $\texttt{Ord}(\bar{x}',i-1,i)$.
      Имаме, че:
      \[\texttt{Ord}(\bar{x}',i-1,i)\ \iff\ x'_{i-1} \leq x'_{i}.\]
      Ние имаме дясната страна на еквивалентността, защото от размяната на елементите $x_{i-1}$ и $x_{i}$
      имаме \[x'_{i-1} = x_i < x_{i-1} = x'_{i}.\]
    \item
      Ако $j < i$, тогава от $A_3(\bar{x},i,j,n)$ имаме, че $\texttt{Ord}(\bar{x},j,i)$,
      и  $x_{j-1} \leq x_{j+1}$, защото $0 < j < i$.
      % а от последния конюнкт на $A_3$ е изпълнено, че $x_{j-1} \leq x_{j+1}$.
      Щом имаме $\texttt{Ord}(\bar{x},j,i)$, за да докажем, че $\texttt{Ord}(\bar{x}',j-1,i)$ e достатъчно да проверим, че
      $x'_{j-1}\leq x'_{j}$ и $x'_j \leq x'_{j+1}$. Понеже сме извършили размяна на стойностите на $x_{j-1}$ и $x_j$,
      а останалите елементи на $\bar{x}$ остават непроменени, получаваме:
      \begin{itemize}
      \item 
        $ x'_{j-1} = x_j < x_{j-1} = x'_j$,
      \item
        $x'_{j} = x_{j-1} \leq x_{j+1} = x'_{j+1}$.
      \end{itemize}
    \end{itemize}
  \item
    Остана да проверим, че ако $0 < j - 1 < i$, то $x'_{j-2} \leq x'_{j}$, т.е. дали
    \[x'_{j-2} = x_{j-2} \leq x_{j-1} = x'_{j}.\] Това е изпълнено, защото от 
    $A_3(\bar{x},i,j,n)$ имаме, че $\texttt{Ord}(\bar{x},0,j-1)$ и следователно $x_{j-2} \leq x_{j-1}$.
  \end{itemize}
  
\item[($3\to 2$)]
  Нека $A_3(\bar{z},\bar{x},i,j,n)$.
  Щом сме отишли в етикет 2, значи имаме \[\neg(1 \leq j\ \&\ x_j < x_{j-1}).\]
  Ще докажем $A_2(\bar{z},\bar{x},i+1,j,n)$.
  \begin{itemize}
  \item 
    $A_3(\bar{z},\bar{x},i,j,n) \implies i \leq n \implies i+1 \leq n+1$.
  \item
    Трябва да проверим, че $\texttt{Ord}(\bar{x},0,i+1-1)$.
    Да видим защо сме преминали от 3 към 2.
    \begin{itemize}
    \item 
      Ако $j < 1$, то $j = 0$, защото от $A_3$ имаме, че $0\leq j$.
      Освен това, от $A_3$ имаме $\texttt{Ord}(\bar{x},0,i)$.
      Оттук ведната следва, че $\texttt{Ord}(\bar{x},0,i-1)$
    \item
      Ако $j \geq 1$, но $x_{j-1} \leq x_j$.
      От $A_3$ имаме, че $\texttt{Ord}(\bar{x},0,j-1)$ и $\texttt{Ord}(\bar{x},j,i)$.
      От всичко това имаме, че 
      \[x_0 \leq \dots \leq x_{j-1} \leq x_j \leq x_{j+1} \leq \dots \leq x_i.\]
      Заключаваме, че $\texttt{Ord}(\bar{x},0,i)$.
    \end{itemize}
  \end{itemize}
\item[($2 \to 4$)]
  Нека $A_2(\bar{z},\bar{x},i,j,n)$ е изпълнено. Щом се достигнали етикета 4, значи имаме и $i > n$.
  От $A_2$ пък имаме $i \leq n+1$.
  Следователно, $i = n+1$ и тогава $A_2(\bar{z},\bar{x},i,j,n) \implies \texttt{Ord}(\bar{x},0,n+1-1)$.
\end{description}
\end{proof}

\begin{cor}
  Програмата $P$ от \Fig{insertion-sort} 
  е частично коректна относно входното условие $I(\bar{z},n) \dff \bar{z}\in\Int^{n+1}$ и изходното условие $O(\bar{z},\bar{x},n) \equiv \texttt{Ord}(\bar{x},0,n)\ \&\ \texttt{Perm}(\bar{z},\bar{x},n)$.
\end{cor}

%%% Local Variables:
%%% mode: latex
%%% TeX-master: "../sep-problems"
%%% End:


\subsection{Задачи с упътване}

\subsection*{Сортиране с избор}

\begin{figure}[H]
  \begin{subfigure}[b]{0.6\textwidth}
  \begin{tikzpicture}[node distance = 2.5cm,auto,scale=0.6, every node/.style={scale=0.9}]
    % Place nodes
    \node [cloud] (init) {вход: $z_0,\dots,z_n$};
    \node [label, below of=init, node distance=1.5cm] (n1) {\scriptsize{1}};
    \node [block, below of=init] (identify) {$\scriptsize{\bar{x}:=\bar{z}}$\\$\scriptstyle{i := 0}$};
    \node [label, below of=identify,node distance=1.2cm] (n2) {\scriptsize{2}};
    \node [decision, below of=identify,node distance=2.7cm] (evaluate) {$i < n$};
    \node [block, below of=evaluate, node distance=2cm] (ji) {$\scriptstyle{m := i}$\\$\scriptstyle{j := i+1}$};
    \node [label, below of=evaluate] (n3) {\scriptsize{3}};
    \node [decision, below of=ji] (loop) {$j > n$};
    \node [decision, below of=loop,node distance=2.5cm] (comp) {$x_j < x_m$};
    \node [label, right of=comp, node distance=1.8cm] (n4) {\scriptsize{4}};
    \node [block, right of=n4,node distance=1.8cm] (jinc) {$\scriptstyle{j++}$};
    \node [block, below of=jinc] (mset) {$\scriptstyle{m := j}$};
    \node [label, above of=mset, node distance=1.3cm] (n5) {\scriptsize{5}};
    \node [block, left of=loop] (swap) {$\scriptstyle{swap(x_i,x_m)}$\\$\scriptstyle{i++}$};
    \node [label, right of=evaluate, node distance=1.8cm] (n6) {\scriptsize{6}};
    \node [cloud, right of=n6, node distance=2cm] (exit) {изход: $x_0,\dots,x_n$};

  % Draw edges
    \path [line] (init) -- (n1);
    \path [line] (n1) -- (identify);
    \path [line] (identify) -- (n2);
    \path [line] (n2) -- (evaluate);
    \path [line] (evaluate) -- node {да} (ji);
    \path [line] (ji) -- node {} (n3);
    \path [line] (n3) -- node {} (loop);
    \path [line] (loop) -- node {не} (comp);
    \path [line] (comp) -- node {не} (n4);
    \path [line] (n4) -- node {} (jinc);
    \path [line] (comp) |- node [left] {да} (mset);
    \path [line] (mset) -- node {} (n5);
    \path [line] (n5) -- node {} (jinc);
    \path [line] (jinc) |- node {} (n3);
    x\path [line] (loop) -- node [below] {да} (swap);
    \path [line] (swap) |- node  {} (n2);
    \path [line] (evaluate) -- node {не} (n6);
    \path [line] (n6) -- node {} (exit);
  \end{tikzpicture}
  \caption{Блок схема за алгоритъм, който сортира входния масив възходящ ред (\href{https://en.wikipedia.org/wiki/Selection_sort}{Selection sort})}
  \end{subfigure}
  ~
  \qquad
  \begin{subfigure}[b]{0.6\textwidth}
    \footnotesize{
      Трябва да докажем, че програмата е частично коректна относно
      \begin{align*}
        & I(\bar{z},n) \dfff \bar{z} \in \Int^{n+1}\\
        & O(\bar{z},\bar{x},n) \dfff \texttt{Ord}(\bar{x},0,n)\ \&\ \texttt{Perm}(\bar{z},\bar{x},n).
      \end{align*}     
      За да направим това, разгледайте свойствата:
      \begin{itemize}
      \item 
        $A_1(\bar{z},\bar{x},i,j,m,n) \dfff \bar{z} \in \Int^{n+1}\ \&\ n \geq 0$;
      \item
        $A_2(\bar{z},\bar{x},i,j,m,n)~\dfff~\texttt{Perm}(\bar{z},\bar{x},n)\ \&\ \texttt{Ord}(\bar{x},0,i)\ \&$\\
        $i\leq n\ \&\ (0 < i \implies x_{i-1} = \min\{x_{i-1},\dots,x_n\})$;
      \item
        $A_3(\bar{z},\bar{x},i,j,m,n) \dfff A_2(\bar{z},\bar{x},i,j,m,n)\ \&\ x_m = \min\{x_i,\dots,x_{j-1}\}\ \&\ i \leq m \leq j \leq n+1$;
      \item
        $A_4(\bar{z},\bar{x},i,j,m,n) \dfff A_3(\bar{z},\bar{x},i,j+1,m,n)$;
      \item
        $A_5(\bar{z},\bar{x},i,j,m,n) \dfff A_4(\bar{z},\bar{x},i,j,m,n)$;
      \item
        $A_6(\bar{z},\bar{x},i,j,m,n) \dfff \texttt{Ord}(\bar{x},0,n)\ \&\ \texttt{Perm}(\bar{z},\bar{x},n)$.
      \end{itemize}
      Докажете, че за всеки преход $(k) \to (l)$ имаме
      \[A_k(\bar{z},\bar{x},i,j,m,n) \implies A_l(\bar{z},f_{kl}(\bar{x},i,j,m),n).\]
    }
    % \caption{}
  \end{subfigure}
  % \end{wrapfigure}
\end{figure}

%%% Local Variables:
%%% mode: latex
%%% TeX-master: "../sep-problems"
%%% End:

\subsection*{Обръщане на масив}

\begin{figure}[H]
  \begin{subfigure}[b]{0.6\textwidth}
  \begin{tikzpicture}[node distance = 2.5cm,auto,scale=0.6, every node/.style={scale=0.9}]
    % Place nodes
    \node [cloud] (init) {вход: $z_0,\dots,z_n$};
    \node [label, below of=init, node distance=1.5cm] (n1) {\scriptsize{1}};
    \node [block, below of=init] (identify) {$\scriptsize{\bar{x}:=\bar{z}}$\\$\scriptstyle{i := 0}$};
    \node [label, below of=identify,node distance=1.2cm] (n2) {\scriptsize{2}};
    \node [decision, below of=identify,node distance=2.7cm] (evaluate) {$i = n+1$};
    \node [block, below of=evaluate, node distance=2cm] (ji) {$\scriptstyle{y := x_0}$\\$\scriptstyle{j := 0}$};
    \node [label, below of=evaluate] (n3) {\scriptsize{3}};
    \node [decision, below of=ji] (loop) {$j = n - i$};
    \node [block, right of=loop] (jinc) {$\scriptstyle{x_j := x_{j+1}}$\\$\scriptstyle{j++}$};
    \node [block, left of=loop] (mset) {$\scriptstyle{x_{n-i} := y}$\\$\scriptstyle{i++}$};
    \node [label, right of=evaluate, node distance=1.8cm] (n4) {\scriptsize{4}};
    \node [cloud, right of=n6, node distance=2cm] (exit) {изход: $x_0,\dots,x_n$};

  % Draw edges
    \path [line] (init) -- (n1);
    \path [line] (n1) -- (identify);
    \path [line] (identify) -- (n2);
    \path [line] (n2) -- (evaluate);
    \path [line] (evaluate) -- node {не} (ji);
    \path [line] (ji) -- node {} (n3);
    \path [line] (n3) -- node {} (loop);
    \path [line] (loop) -- node {не} (jinc);
    \path [line] (loop) -- node [above] {да} (mset);
    \path [line] (mset) |- node {} (n2);
    \path [line] (jinc) |- node {} (n3);
    \path [line] (evaluate) -- node {да} (n4);
    \path [line] (n4) -- node {} (exit);
  \end{tikzpicture}
  \caption{Ще докажем, че по даден входен масив, програмата го обръща}
  \end{subfigure}
  ~
  \qquad
  \begin{subfigure}[b]{0.6\textwidth}
    \footnotesize{
      Да положим предикатите:
      \begin{align*}
        \texttt{Shift}(\bar{x},\bar{z},i,j,s) & \dfff  (\forall k)[i \leq k \leq j \to x_k = z_{k+s}];\\
        \texttt{Inv}(\bar{x},\bar{z},i,j) & \dfff (\forall k)[i \leq k \leq j \to x_k = z_{n-k}].
      \end{align*}
      Трябва да докажем, че програмата е частично коректна относно 
      \begin{align*}
        & I(\bar{z},n) \dfff \bar{z} \in \Int^{n+1}\ \&\ n \geq 0;\\
        & O(\bar{z},\bar{x},n) \dfff \texttt{Inv}(\bar{x},\bar{z},0,n).
      \end{align*}
      Разгледайте свойствата:
      \begin{align*}
        A_1(\bar{z},\bar{x},i,j,y,n) \dfff & \bar{z} \in \Int^{n+1}\ \&\ n \geq 0\\
        A_2(\bar{z},\bar{x},i,j,y,n) \dfff & 0 \leq i \leq n+1\ \&\\
        & \texttt{Inv}(\bar{x},\bar{z},n+1-i,n)\ \&\\
        & \texttt{Shift}(\bar{x},0,n-i,i);\\
        A_3(\bar{z},\bar{x},i,j,y,n) \dfff & y = z_i\ \&\ 0 \leq j \leq n-i\ \&\ 0 \leq i\\
        & \texttt{Inv}(\bar{x},\bar{z},n+1-i,n)\ \&\\
        & \texttt{Shift}(\bar{x},0,j-1,i+1)\ \&\\
        & \texttt{Shift}(\bar{x},j,n-i,i);\\
        A_4(\bar{z},\bar{x},i,j,y,n) \dfff & \texttt{Inv}(\bar{x},\bar{z},0,n).
      \end{align*}
      Докажете, че за всеки преход $(k) \to (l)$ имаме
      \[A_k(\bar{z},\bar{x},i,j,y,n) \implies A_l(\bar{z},f_{kl}(\bar{x},i,j,y),n).\]
    }
  \end{subfigure}
% \end{wrapfigure}
\end{figure}


%%% Local Variables:
%%% mode: latex
%%% TeX-master: "../sep-problems"
%%% End:

\subsection*{Функцията 91 на Макарти}

\begin{figure}[H]
  \begin{subfigure}[b]{0.6\textwidth}
  \begin{tikzpicture}[node distance = 3cm,auto,scale=0.6, every node/.style={scale=0.9}]
    % Place nodes
    \node [cloud] (init) {вход: $x$};
    \node [label, below of=init, node distance=1.5cm] (n1) {\scriptsize{1}};
    \node [tallblock, below of=init] (identify) {$y := x$\\$t := 1$};
    \node [label, below of=identify,node distance=1.5cm] (n2) {\scriptsize{2}};
    \node [decision, below of=identify] (evaluate) {$y \geq 101$};
    \node [bigblock, below of=evaluate, node distance=2cm] (min) {$y := y - 10$\\$t := t - 1$};
    \node [label, below of=min, node distance=1.5cm] (n3) {\scriptsize{3}};
    \node [bigblock, right of=evaluate] (pls) {$y := y + 11$\\$t := t + 1$};
    \node [decision, below of=n3, node distance=1.5cm] (ready) {$t = 0$};
    \node [label, right of=ready, node distance=2cm] (n4) {\scriptsize{4}};
    \node [cloud, right of=n4] (exit) {изход: $y$};
    % \coordinate[left of=ready] (dummy);

  % Draw edges
    \path [line] (init) -- (n1);
    \path [line] (n1) -- (identify);
    \path [line] (identify) -- (n2);
    \path [line] (n2) -- (evaluate);
    \path [line] (evaluate) -- node {да} (min);
    \path [line] (evaluate) -- node {не} (pls);
    \path [line] (pls) |- node {} (n2);
    \path [line] (min) -- node  {} (n3);
    \path [line] (n3) -- node  {} (ready);
    \path [line] (ready) -- node {да} (n4);
    \path [line] (n4) -- node {} (exit);
    % \path [line] (ready) -- node {} ();
    % \path [line] (dummy) |- node {} (n2);
    \path [line] (ready) -- node [above] {не} (-3,-16.5) -- (-3,-6.75) -- (n2);
  \end{tikzpicture}
  \caption{Итеративна версия на функцията 91 на Макарти}
  \label{fig:gcd}
  \end{subfigure}
  ~
  \begin{subfigure}[b]{0.6\textwidth}
    \footnotesize{
      Да разгледаме свойствата:
      \begin{align*}
        A_1(x,y,t) \dfff & x \in \Nat\ \&\ x \leq 100\\
        A_2(x,y,t) \dfff & (t \geq 2 \implies y \leq 111)\ \&\\
                         & (t = 1 \implies y \leq 101)\\
        A_3(x,y,t) \dfff & (t \geq 1 \implies y \leq 101)\ \&\\
                         & (t = 0 \implies y = 91)\ \& \\
        & 91 \leq y \leq 101\\
        A_4(x,y,t) \dfff & y = 91.
      \end{align*}
      Докажете, че за всеки преход $(k) \to (l)$ имаме
      \[(\forall x,y,z,t,v)[A_k(x,y,z,t,v) \implies A_l(x,y,f_{kl}(z,t,v))].\]
    }
  \end{subfigure}
\end{figure}

Нека да означим $y_i$, $t_i$ стойностите на променливите $y$ и $t$ точно след $i$-тото преминаване през $(2)$.
За да докажете тотална коректност, използвайте, че редицата
\[\{(101 - y_i + 10t_i, t_i) \mid i = 0,1,\dots\}\]
е строго намаляваща относно лексикографската наредба.

% С други думи, $(\Nat\times\Nat, \prec)$ е фундирана наредба, където 
% \[(a,b) \prec (a',b') \iff \]

%%% Local Variables:
%%% mode: latex
%%% TeX-master: "../sep-problems"
%%% End:


% \begin{problem}
%   Нека е дадена програмата на езика хаскел:

%   \begin{minted}[frame=lines,framesep=2mm,baselinestretch=1.2]{haskell}
%     rev :: [a] -> [a]
%     rev x = f(x, []) where 
%       f([], y) = y
%       f(x:xs, y) = f(xs, x:y)
%   \end{minted}

%   \noindent 
%   Докажете, че:
%   \begin{enumerate}[a)]
%   \item 
%     $rev:\Sigma^\star \to \Sigma^\star$ е тотална.
%   \item
%     $(\forall x \in \Sigma^\star)[rev(rev(x)) = x]$.
%   \item
%     $(\forall x \in \Sigma^\star)[rev(x) = x^R]$.
%   \end{enumerate}
% \end{problem}
% \begin{hint}
%   % Ще използваме следното правило:
%   % \begin{prooftree}
%   %   \AxiomC{$P(\varepsilon)$}
%   %   \AxiomC{$(\forall x \in \Sigma^\star)[x\neq\varepsilon\ \&\ P(cdr(x)) \to P(x)]$}
%   %   \RightLabel{\scriptsize(1)}
%   %   \BinaryInfC{$(\forall x\in\Sigma^\star)[P(x)]$}
%   % \end{prooftree}
%   % % \item
%   %   Докажете валидността на правилото $(1)$. % е еквивалентно на структурна индукция върху фундираната наредба
%     % $(\Sigma^\star,\prec)$, където $x \prec y \iff (\exists z\in\Sigma^\star)[z\cdot x = y]$, т.е.
%     % $x$ е суфикс на $y$.
%   Да разгледаме фундираната наредба $L = (\Sigma^\star, \prec)$, където
%   $x \prec y \iff \abs{x} < \abs{y}$.
%   \begin{enumerate}[a)]
%   \item 
%     Да разгледаме свойството 
%     \[P(x) \equiv (\forall y\in \Sigma^\star)[f(x,y)\text{ е дефинирана}].\]
%     Докажете със структурна индукция по $L$, че $(\forall x\in\Sigma^\star)[P(x)]$.
%   \item
%     Да разгледаме свойството 
%     \[P(x) \equiv (\forall y\in \Sigma^\star)[rev(f(x,y)) = f(y,x)].\]
%     Докажете със структурна индукция по $L$, че $(\forall x\in\Sigma^\star)[P(x)]$.
%     Тогава в частния случай $y = \varepsilon$, 
%     \[rev(rev(x)) = rev(f(x,\varepsilon)) = f(\varepsilon,x) = x.\]
%   \item
%     Разгледайте свойството
%     \[P(x) \equiv (\forall y\in\Sigma^\star)[f(x,y) = x^R \cdot y].\]
%   \end{enumerate}
% \end{hint}

% \begin{problem}
%   Нека е дадена програмата на езика хаскел:

%   \begin{minted}[]{haskell}
%     concat :: ([a], [a]) -> [a]
%     concat([], y) = y
%     concat(x:xs, y) = x:concat(xs, y)
%   \end{minted}
%   \noindent Докажете, че:
%   \begin{enumerate}[a)]
%   \item
%     $(\forall x,y\in\Sigma^\star)[concat(x, y) = x \cdot y]$;
%   \item 
%     $(\forall x,y,z\in\Sigma^\star)[concat(concat(x, y), z) = concat(x, concat(y, z))]$;
%   \item
%     $(\forall x,y\in\Sigma^\star)[concat(x, y)^R = concat(y^R, x^R)]$.
%   \end{enumerate}
% \end{problem}
% \begin{hint}
%   Да разгледаме фундираната наредба $L = (\Sigma^\star, \prec)$, където
%   $x \prec y \iff \abs{x} < \abs{y}$.
%   \begin{enumerate}[a)]
%   \item
%     Разгледайте
%     \[P(x) \equiv (\forall y\in\Sigma^\star)[concat(x, y) = x \cdot y].\]
%   \item 
%     Разгледайте 
%     \[P(x) \equiv (\forall y,z\in\Sigma^\star)[concat(concat(x, y), z) = concat(x, concat(y, z))].\]
%   \item
%     Разгледайте
%     \[P(x) \equiv (\forall y\in\Sigma^\star)[concat(x, y)^R = concat(y^R, x^R)].\]
%   \end{enumerate}
% \end{hint}

% \begin{problem}
%   Да разгледаме следната програма
%   \begin{haskellcode}
%     f :: Int -> Int
%     f(x) = if x > 100 then x - 10
%              else f(f(x + 11))
%   \end{haskellcode}
%   Докажете, че 
%   \[f(x) = \begin{cases}
%     x - 10, & x > 100\\
%     91, & x \leq 100.
%   \end{cases}\]
% \end{problem}
% \begin{hint}
%   Разгледайте строгата частична наредба $\A = (\{x \in \Int \mid x \leq 100\}, \prec)$, където
%   \[x \prec y \iff y < x.\]
%   Лесно се съобразява, че наредбата $\A$ е фундирана.
%   Да разгледаме свойството \[P(x) \equiv f(x) = 91.\]
%   \begin{itemize}
%   \item 
%     Лесно се вижда, че $P(100)$.
%   \item
%     Нека $x < 100$. 
%     Да приемем, че $(\forall z \prec x)[P(z)]$. Ще докажем, че $P(x)$.
%     Тук трябва да разгледаме два подслучая в зависимост от това дали $x+11 \leq 100$ или $x+11 > 100$.
%   \end{itemize}
% \end{hint}

% \begin{problem}
%   Да разгледаме следната програма:
%   \begin{minted}[]{haskell}
%     f :: (Int, Int) -> Int
%     f(x, y) = if x == y then 1
%              else (y+2)*(y+1)*f(x, y + 2)
%   \end{minted}
%   Докажете, че \[(\forall x,y\in\Nat)[x \geq y\ \&\ (x-y) \equiv 0 \bmod 2 \implies f(x,y) = \frac{x!}{y!}].\]
% \end{problem}
% \begin{hint}
%   Ще използваме структурна индукция върху $(\Nat, <)$.
%   Да разгледаме свойството
%   \[P(z) \equiv (\forall x,y\in\Nat)[z = x - y \geq 0\ \&\ z \equiv 0 \bmod 2 \implies f(x,y) = \frac{x!}{y!}].\]
% \end{hint}



% \input{verification/induction}
% \input{verification/permutations}


%%% Local Variables: 
%%% mode: latex
%%% TeX-master: "sep-problems"
%%% End: 
