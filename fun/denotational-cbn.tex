% \subsection{Предаване на параметрите по име}

Нека е дадена една рекурсивна програма $\vv{P}[\varsx,\varsf]$, където:
\marginpar{Можем да си мислим, че $\vv{f}_1$ е нещо като \texttt{main} функция за програмата $\vv{P}$.}
\begin{align*}
  \vv{P} = & 
             \begin{cases}
               & \vv{f}_1(\vv{x}_1,\dots,\vv{x}_{m_1}) = \tau_1[\vv{x}_1,\dots,\vv{x}_{m_1},\vv{f}_1,\dots,\vv{f}_k]\\
               & \vv{f}_2(\vv{x}_1,\dots,\vv{x}_{m_2}) = \tau_2[\vv{x}_1,\dots,\vv{x}_{m_2},\vv{f}_1,\dots,\vv{f}_k]\\
               & \vdots\\
               & \vv{f}_k(\vv{x}_1,\dots,\vv{x}_{m_k}) = \tau_k[\vv{x}_1,\dots,\vv{x}_{m_k},\vv{f}_1,\dots,\vv{f}_k]
             \end{cases}
\end{align*}
Нека $\ov{\gamma} \in \DomOpCBN$
е {\em най-малкото решение} на системата
\marginpar{В тази система неизвестните са $\varphi_1,\dots,\varphi_k$.}

\begin{SystemEq}
  \varphi_1 & = & \val{\tau_1}(\varphi_1,\dots,\varphi_k)\\
  & \vdots & \\
  \varphi_k & = & \val{\tau_k}(\varphi_1,\dots,\varphi_k).
\end{SystemEq}
% \begin{align*}
%   & \val{\tau_1}(\varphi_1,\dots,\varphi_k) = \varphi_1\\
%   & \ \vdots \\
%   & \val{\tau_k}(\varphi_1,\dots,\varphi_k) = \varphi_k.
% \end{align*}
От \hyperref[th:knaster-tarski]{Теоремата на Клини} знаем, че такова най-малко решение съществува.

\index{денотационна семантика!по име}
\begin{framed}
  За дадената рекурсивна програма $\vv{P}[\varsx,\varsf]$, 
  определяме {\bf денотационната семантика с предаване на параметрите по име} 
  като изображението $\D\val{\vv{P}} \in \Cont{\Nat^{m_1}_\bot}{\Nat_\bot}$, където:
  \[\D\val{\vv{P}}(a_1,\dots,a_{m_1}) \df
    \begin{cases}
      \gamma_1(a_1,\dots,a_{m_1}), & \text{ако }\bot\not\in\{a_1,\dots,a_{m_1}\}\\
      \bot, & \text{ако }\bot\in\{a_1,\dots,a_{m_1}\}.
    \end{cases}\]
\end{framed}


\begin{example}
  Да разгледаме следната проста програма:
  \begin{haskellcode}
f(x) = if x == 0 then 1 else f(x+1)
  \end{haskellcode}
  \marginpar{Тази програма е дори валидна програма на хаскел.}
  На тази програма съответства системата от едно уравнение с една функционална променлива
  \begin{equation}
    \label{eq:denotational-cbn:example-motivation}
    f = \Gamma(f),
  \end{equation}
  където за изображението $\Gamma$ имаме, че:
  \begin{align*}
    & \Gamma \in \Cont{\Cont{\Nat_\bot}{\Nat_\bot}}{\Cont{\Nat_\bot}{\Nat_\bot}}\\
    & \Gamma(f)(x) \df
      \begin{cases}
        1, & \text{ако }x = 0\\
        f(x+1), & \text{иначе}.
      \end{cases}
  \end{align*}
  Най-малкото решение на системата (\ref{eq:denotational-cbn:example-motivation}) е функцията $\varphi:\Nat_\bot\to\Nat_\bot$, за която
  \[\varphi(x) = \begin{cases}
      1, & \text{ако } x =  0\\
      \bot, & \text{иначе}.
    \end{cases}\]
  % $\varphi(0) = 1$ и $\varphi(x) = \bot$ за $x \neq 0$.
  \marginpar{\todo Съобразете, че системата (\ref{eq:denotational-cbn:example-motivation}) има дори безкрайно много решения!}
  Обаче системата (\ref{eq:denotational-cbn:example-motivation}) има и други решения. Например, функцията $\psi(x) = 1$ за всяко $x \in \Nat_\bot$.
  Този пример ни показва, че най-малкото решение кореспондира точно с нашата интуиция как ,,работи'' тази програма.
\end{example}



%%% Local Variables:
%%% mode: latex
%%% TeX-master: "../sep"
%%% End:
