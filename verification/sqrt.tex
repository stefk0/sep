
\begin{figure}[H]
  \begin{subfigure}[b]{0.6\textwidth}
  \begin{tikzpicture}[node distance = 3cm,auto,scale=0.6, every node/.style={scale=0.9}]
    % Place nodes
    \node [cloud] (init) {вход: $n$};
    \node [label, below of=init, node distance=1.5cm] (n1) {\scriptsize{1}};
    \node [tallblock, below of=init] (identify) {$x := 0$\\$y := 1$\\$s := 1$};
    \node [label, below of=identify,node distance=1.5cm] (n2) {\scriptsize{2}};
    \node [decision, below of=identify] (evaluate) {$s \leq n$};
    \node [bigblock, below of=evaluate, node distance=2cm] (ji) {$x := x + 1$\\$y := y + 2$\\$s := s + y$};
    \coordinate[right of=evaluate,node distance=2cm] (inc);
    \node [label, left of=evaluate, node distance=2cm] (n3) {\scriptsize{3}};
    \node [cloud, left of=n3] (exit) {изход: $x$};

  % Draw edges
    \path [line] (init) -- (n1);
    \path [line] (n1) -- (identify);
    \path [line] (identify) -- (n2);
    \path [line] (n2) -- (evaluate);
    \path [line] (evaluate) -- node {да} (ji);
    \draw [-] (ji) -| node {} (inc);
    \path [line] (inc) |- node  {} (n2);
    \path [line] (evaluate) -- node {не} (n3);
    \path [line] (n3) -- node {} (exit);
  \end{tikzpicture}
  \caption{Алгоритъм за намиране на $\lfloor{\sqrt{n}}\rfloor$}
  \label{fig:sqrt}
  \end{subfigure}
  ~
  \begin{subfigure}[b]{0.6\textwidth}
    \footnotesize{
      Да разгледаме свойствата:
      \begin{align*}
        A_1(x,y,s,n) \dfff & n \in \Nat\\
        A_2(x,y,s,n) \dfff & x,n\in\Nat\ \&\ x^2 \leq n\ \&\\
        & y = 2x+1\ \&\ s = (x+1)^2\\
        A_3(x,y,s,n) \dfff & x^2 \leq n < (x+1)^2
      \end{align*}
      Съобразете, че $A_3(x,y,s,n) \implies x = \lfloor{\sqrt{n}}\rfloor$.
      Докажете, че за всеки преход $(k) \to (l)$ имаме
      \[(\forall x,y,s,n)[A_k(x,y,s,n) \implies A_l(f_{kl}(x,y,s),n)].\]
    }
    \end{subfigure}
\end{figure}

\begin{problem}
  Докажете, че програмата $P$ описана с блок-схемата на Фигура \ref{fig:sqrt} е тотално коректна относно входното 
  условие $n \in \Nat$ и изходното условие $\lfloor{\sqrt{n}}\rfloor$.
\end{problem}


%%% Local Variables:
%%% mode: latex
%%% TeX-master: "../sep-problems"
%%% End:
