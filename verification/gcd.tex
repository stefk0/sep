\subsection*{Намиране на НОД}

Да дефинираме функцията $\texttt{gcd}:\Nat\times\Nat \to \Nat$, където
\[\texttt{gcd}(x,y) = (\max z)[\ z\ |\ x\ \&\ z\ |\ y\ ].\]
\begin{itemize}
\item 
  Понеже $0\ |\ 0$, то $\text{НОД}(0,0) = 0$.
\item
  Понеже всяко естествено число дели $0$, то $\text{НОД}(0,z) = z$;
\item
  Ясно е от дефиницията, че $\text{НОД}(x,y) = \text{НОД}(y,x)$;
\end{itemize}

\begin{prop}
  \label{pr:gcd}
  $(\forall x,y\in\Nat)[\text{НОД}(x,y) = \text{НОД}(y, x \bmod y)]$.
\end{prop}
\begin{proof}
  Ще разгледаме няколко случая.
  \begin{itemize}
  \item 
    Ако $x = y$, то всичко е ясно, защото $x \bmod x = x$.
  \item
    Ако $x < y$, то всичко е ясно, защото $x \bmod y = x$ и
    \[\texttt{gcd}(x,y) = \texttt{gcd}(y,x) = \texttt{gcd}(y, x \bmod y).\]
  \item
    Нека $x > y$ и $z = \texttt{gcd}(x,y)$.
    Тогава $z$ е най-голямото естествено число, което $z\ |\ x$ и $z\ |\ y$.
    Нека $r = x \bmod y$. Тогава $x = ky + r$, $0 \leq r < y$.
    Щом $z$ дели $x$ и $y$, то е ясно, че $z$ дели $r$.
    Остана да съобразим защо $z = \texttt{gcd}(y,r)$.
    Да допуснем, че съществува естествено число $z'$, такова че $z < z'$ и $z'$ дели $y$ и $r$.
    Тогава $z'$ също дели и $x = ky + r$. Достигнахме до противоречие с факта, че $z = \texttt{gcd}(x,y)$.
    Следователно,
    \[z = \texttt{gcd}(x,y) = \texttt{gcd}(y,x \bmod y).\]
  \end{itemize}
\end{proof}

\begin{figure}[H]
  \begin{subfigure}[b]{0.6\textwidth}
  \begin{tikzpicture}[node distance = 3cm,auto,scale=0.6, every node/.style={scale=0.9}]
    % Place nodes
    \node [cloud] (init) {вход: $x,y$};
    \node [label, below of=init, node distance=1.5cm] (n1) {\scriptsize{1}};
    \node [tallblock, below of=init] (identify) {$z := x$\\$t := y$};
    \node [label, below of=identify,node distance=1.5cm] (n2) {\scriptsize{2}};
    \node [decision, below of=identify] (evaluate) {$t = 0$};
    \node [bigblock, below of=evaluate, node distance=2cm] (ji) {$v := t$\\$t := z \bmod t$\\$z := v$};
    \coordinate[right of=evaluate,node distance=2cm] (inc);
    \node [label, left of=evaluate, node distance=2cm] (n3) {\scriptsize{3}};
    \node [cloud, left of=n3] (exit) {изход: $z$};

  % Draw edges
    \path [line] (init) -- (n1);
    \path [line] (n1) -- (identify);
    \path [line] (identify) -- (n2);
    \path [line] (n2) -- (evaluate);
    \path [line] (evaluate) -- node {не} (ji);
    \draw [-] (ji) -| node {} (inc);
    \path [line] (inc) |- node  {} (n2);
    \path [line] (evaluate) -- node {да} (n3);
    \path [line] (n3) -- node {} (exit);
  \end{tikzpicture}
  \caption{Алгоритъм за намиране на $\text{НОД}(x,y)$}
  \label{fig:gcd}
  \end{subfigure}
  ~
  \begin{subfigure}[b]{0.6\textwidth}
    \footnotesize{
      Да разгледаме свойствата:
      \begin{align*}
        A_1(x,y,z,t,v) \dfff & x,y \in \Nat\\
        A_2(x,y,z,t,v) \dfff & \text{НОД}(x,y) = \text{НОД}(z,t)\ \&\\
        & x,y,z,t,v\in\Nat\\
        A_3(x,y,z,t,v) \dfff & z = \text{НОД}(x,y)
      \end{align*}
      Докажете, че за всеки преход $(k) \to (l)$ имаме
      \[(\forall x,y,z,t,v)[A_k(x,y,z,t,v) \implies A_l(x,y,f_{kl}(z,t,v))].\]
    }
    \end{subfigure}
\end{figure}

\begin{prop}
  Докажете, че програмата на Фигура \ref{fig:gcd} е тотално коректна относно входното условие $x,y\in\Nat$
  и изходното условие $z = \text{НОД}(x,y)$.
\end{prop}
\begin{hint}
  Доказателството на $A_2(x,y,z,t,v) \implies A_2(x,y,f_{22}(z,t,v))$ следва директно от \Prop{gcd}.
  За доказателството на $A_2(x,y,z,t,v) \implies A_3(x,y,f_{23}(z,t,v))$ е достатъчно да съобразим, че ако имаме $A_2(x,y,z,t,v)$ и $t = 0$, то
  $\text{НОД}(x,y) = \text{НОД}(z,0) = z$.
  
  Остана да докажем, че програмата винаги завършва при входни данни $x,y \in \Nat$.
  Да означим с $t_i$ стойността на променливата $t$ след $i$-тото преминаване през етикет $(2)$.
  Това означава, че $t_0 = y$ и $t_{i+1} = z \bmod t_i$, за някое $z$.
  Следователно, $t_{i+1} < t_i$. От $A_2$ имаме, че всички $t_i \geq 0$.
  Така получаваме строго монотонно намаляваща редица $t_0 > t_1 > \cdots > t_i > \cdots \geq 0$, която е ограничена отдолу от $0$.
  Заключаваме, че съществува $i$, за което $t_i = 0$. Следователно, програмата завършва.
\end{hint}


%%% Local Variables:
%%% mode: latex
%%% TeX-master: "../sep-problems"
%%% End:
