
\subsection*{Сортиране чрез вмъкване}
\begin{problem}
  Докажете, че програмата, описана с блок-схемата на \Fig{insertion-sort}, 
  сортира входния масив във възходящ ред.
\end{problem}

%\begin{wrapfigure}{r}{0.5\textwidth}
\begin{figure}[H]
  \begin{tikzpicture}[node distance = 2.75cm,auto,scale=0.6, every node/.style={scale=0.9}]
    % Place nodes
    \node [cloud] (init) {вход: $z_0,\dots,z_n$};
    \node [label, below of=init, node distance=1.25cm] (n1) {\scriptsize{1}};
    \node [block, below of=init, node distance=2.25cm] (identify) {$\bar{x} := \bar{z}$\\$i := 1$};
    \node [label, below of=identify,node distance=1.5cm] (n2) {\scriptsize{2}};
    \node [decision, below of=identify] (evaluate) {$i \leq n$};
    \node [block, below of=evaluate, node distance=2cm] (ji) {$j := i$};
    \node [block, right of=ji, node distance=2.5cm] (inc) {$i := i+1$};
    \node [label, below of=evaluate] (n3) {\scriptsize{3}};
    \node [decision, below of=ji] (loop) {$\scriptstyle{j\geq 1\ \&}$\\$\scriptstyle{x_j < x_{j-1}}$};
    \node [block, left of=loop] (swap) {$\scriptstyle{swap(x_j,x_{j-1})}$\\$\scriptstyle{j := j-1}$};
    \node [cloud, left of=evaluate, node distance=4cm] (exit) {изход: $x_0,\dots,x_n$};
    \node [label, left of=evaluate, node distance=2cm] (n4) {\scriptsize{4}};
  % Draw edges
    \path [line] (init) -- (n1);
    \path [line] (n1) -- (identify);
    \path [line] (identify) -- (n2);
    \path [line] (n2) -- (evaluate);
    \path [line] (evaluate) -- node {да} (ji);
    \path [line] (ji) -- node {} (n3);
    \path [line] (n3) -- node {} (loop);
    \path [line] (loop) -- node {да} (swap);
    \path [line] (swap) |- node {} (n3);
    \path [line] (loop) -| node [below] {не} (inc);
    \path [line] (inc) |- node  {} (n2);
    \path [line] (evaluate) -- node {не} (n4);
    \path [line] (n4) -- node {} (exit);
  \end{tikzpicture}
  \caption{По даден входен масив $\bar{x}\in\Int^{n+1}$, програмата го сортира във възходящ ред (\href{http://en.wikipedia.org/wiki/Insertion_sort}{insertion sort})}
% \end{wrapfigure}
  \label{fig:insertion-sort}
\end{figure}

\noindent Удобно е да означим:
\[\texttt{Ord}(\bar{x},i,j) \dfff (\forall k)[i \leq k < j \implies x_k \leq x_{k+1}],\]
което ни казва, че елемените в интервала $[i,j]$ на масива $\bar{x}$ са подредени във възходящ ред.
Също така, да означим:
\begin{align*}
  \texttt{Perm}(\bar{z},\bar{x},n) \dfff \bar{z},\bar{x} \in \Int^{n+1}\ \&\ (\exists f)[&f:\{0,\dots,n\}\to\{0,\dots,n\} \text{ е биекция }\&\\
  & (\forall i)[0\leq i \leq n \implies z_i = x_{f(i)}]].
\end{align*}
Това означава, че $\bar{z}$ е пермутация на елементите на $\bar{x}$.

\marginpar{Понеже масивът $\bar{z}$ и $n$ са константни, то няма нужда да описваме как се променят с функциите $f_{kl}$}
За всеки директен преход $(k) \to (l)$ между етикети в блок схемата, асоциираме функция $f_{kl}$,
която показва как се изменят стойностите на променливите участващи в програмата.
\begin{align*}
  & f_{12}(\bar{x},i,j) \dff (\bar{x},1,j)\\
  & f_{23}(\bar{x},i,j) \dff (\bar{x}, i, i)\\
  & f_{24}(\bar{x},i,j) \dff (\bar{x}, i, j)\\
  & f_{32}(\bar{x},i,j) \dff (\bar{x}, i+1, j)\\
  & f_{33}(\bar{x},i,j) \dff (\bar{x}',i,j-1),
\end{align*}
където $\bar{x}' = (x'_0,\dots,x'_n)$ е променения масив, за който
$x'_{j-1} = x_{j}$, $x'_j = x_{j-1}$, а $x'_k = x_k$ за всеки индекс $k$ в интервала $[0,n] \setminus\{j-1,j\}$.

Към всеки етикет $(l)$ в блок схемата на програмата $P$ на \Fig{insertion-sort}, асоциираме предиката $A_l$, където:
\begin{align*}
  A_1(\bar{z},\bar{x},i,j,n) \dfff\ & \bar{z} \in \Int^{n+1}\ \&\ n \geq 0\\
  A_2(\bar{z},\bar{x},i,j,n) \dfff\ & \texttt{Perm}(\bar{z},\bar{x},n)\ \&\ \texttt{Ord}(\bar{x},0,i-1)\ \&\ 0 \leq i \leq n+1\\
  A_3(\bar{z},\bar{x},i,j,n) \dfff\ & \texttt{Perm}(\bar{z},\bar{x},n)\ \&\ \texttt{Ord}(\bar{x},0,j-1)\ \&\ \texttt{Ord}(\bar{x},j,i)\ \&\\
  & 0\leq j \leq i \leq n\ \&\ (0 < j < i \implies x_{j-1} \leq x_{j+1})\\
  A_4(\bar{z},\bar{x},i,j,n) \dfff\ & \texttt{Perm}(\bar{z},\bar{x},n)\ \&\ \texttt{Ord}(\bar{x},0,n).
\end{align*}

\begin{prop}
  За всеки директен преход между етикети $(k) \to (l)$ е изпълнена импликацията:
  \[(\forall\bar{z},\bar{x},i,j,n)[A_k(\bar{z},\bar{x},i,j,n) \implies A_l(\bar{z},f_{kl}(\bar{x},i,j),n)].\]
\end{prop}
\begin{proof}
  \begin{description}
  \item[($1\to 2$)]
    Очевидно е, че имаме $A_2(\bar{z},\bar{x},1,j,n)$, т.е. $\texttt{Ord}(\bar{x},0,1-1)$ и $1 \leq n+1$.
  \item[($2 \to 3$)]
    Нека $A_2(\bar{z},\bar{x},i,j,n)$.
    Ще докажем, че имаме $A_3(\bar{z},\bar{x},i,i,n)$. Това е съвсем лесно:
    \begin{itemize}
    \item 
      От $A_2$ е ясно, че имаме $\texttt{Ord}(\bar{x},0,i-1)$.
    \item
      Очевидно е, че имаме $\texttt{Ord}(\bar{x},i,i)$.
    \item
      От $A_2$ имаме, че $i < n+1$. Но понеже от етикет 2 сме отишли в етикет 3, 
      то $i \neq n+1$. Следователно, $0 \leq i \leq i \leq n$.
    \item
      Импликацията $0 < i < i \implies x_{i-1}\leq x_{i+1}$  
      е изпълнена по тривиални причини.
    \end{itemize}
  \item[($3\to 3$)]
  Нека $A_3(\bar{z},\bar{x},i,j,n)$.
  Ще докажем, че $A_3(\bar{z},\bar{x}',i,j-1,n)$.
  Понеже сме направили преход $3 \to 3$,
  то имаме също свойството, че $j \geq 1$ и $x_{j} < x_{j-1}$.
  Това означава, че:
  \begin{align*}
    \bar{x} =\ & \overbrace{x_0 \leq \dots \leq x_{j-2}\leq x_{j-1}}^{\texttt{Ord}} > \overbrace{x_j \leq x_{j+1} \leq \cdots \leq x_i}^{\texttt{Ord}}\\
    \bar{x}' =\ & \underbrace{x_0 \leq \dots \leq x_{j-2}}_{\texttt{Ord}} \square \underbrace{x'_{j-1}}_{x_j} < \underbrace{x'_j}_{x_{j-1}} \square \underbrace{x_{j+1} \leq \cdots \leq x_i}_{\texttt{Ord}}
  \end{align*}
  
  \begin{itemize}
  \item 
    Ясно е, че $0 \leq j-1 \leq i$;
  \item
    От $\texttt{Ord}(\bar{x},0,j-1)$ следва, че $\texttt{Ord}(\bar{x}',0,j-2)$,
    защото единствената промяна в масива е размяната на стойностите
    на $x_{j-1}$ и $x_j$.
  \item
    За да докажем, че $\texttt{Ord}(\bar{x}',j-1,i)$, трябва да разгледаме два случая.
    \begin{itemize}
    \item
      Ако $j = i$, тогава $x_i < x_{i-1}$. Да проверим, че $\texttt{Ord}(\bar{x}',i-1,i)$.
      Имаме, че:
      \[\texttt{Ord}(\bar{x}',i-1,i)\ \iff\ x'_{i-1} \leq x'_{i}.\]
      Ние имаме дясната страна на еквивалентността, защото от размяната на елементите $x_{i-1}$ и $x_{i}$
      имаме \[x'_{i-1} = x_i < x_{i-1} = x'_{i}.\]
    \item
      Ако $j < i$, тогава от $A_3(\bar{x},i,j,n)$ имаме, че $\texttt{Ord}(\bar{x},j,i)$,
      и  $x_{j-1} \leq x_{j+1}$, защото $0 < j < i$.
      % а от последния конюнкт на $A_3$ е изпълнено, че $x_{j-1} \leq x_{j+1}$.
      Щом имаме $\texttt{Ord}(\bar{x},j,i)$, за да докажем, че $\texttt{Ord}(\bar{x}',j-1,i)$ e достатъчно да проверим, че
      $x'_{j-1}\leq x'_{j}$ и $x'_j \leq x'_{j+1}$. Понеже сме извършили размяна на стойностите на $x_{j-1}$ и $x_j$,
      а останалите елементи на $\bar{x}$ остават непроменени, получаваме:
      \begin{itemize}
      \item 
        $ x'_{j-1} = x_j < x_{j-1} = x'_j$,
      \item
        $x'_{j} = x_{j-1} \leq x_{j+1} = x'_{j+1}$.
      \end{itemize}
    \end{itemize}
  \item
    Остана да проверим, че ако $0 < j - 1 < i$, то $x'_{j-2} \leq x'_{j}$, т.е. дали
    \[x'_{j-2} = x_{j-2} \leq x_{j-1} = x'_{j}.\] Това е изпълнено, защото от 
    $A_3(\bar{x},i,j,n)$ имаме, че $\texttt{Ord}(\bar{x},0,j-1)$ и следователно $x_{j-2} \leq x_{j-1}$.
  \end{itemize}
  
\item[($3\to 2$)]
  Нека $A_3(\bar{z},\bar{x},i,j,n)$.
  Щом сме отишли в етикет 2, значи имаме \[\neg(1 \leq j\ \&\ x_j < x_{j-1}).\]
  Ще докажем $A_2(\bar{z},\bar{x},i+1,j,n)$.
  \begin{itemize}
  \item 
    $A_3(\bar{z},\bar{x},i,j,n) \implies i \leq n \implies i+1 \leq n+1$.
  \item
    Трябва да проверим, че $\texttt{Ord}(\bar{x},0,i+1-1)$.
    Да видим защо сме преминали от 3 към 2.
    \begin{itemize}
    \item 
      Ако $j < 1$, то $j = 0$, защото от $A_3$ имаме, че $0\leq j$.
      Освен това, от $A_3$ имаме $\texttt{Ord}(\bar{x},0,i)$.
      Оттук ведната следва, че $\texttt{Ord}(\bar{x},0,i-1)$
    \item
      Ако $j \geq 1$, но $x_{j-1} \leq x_j$.
      От $A_3$ имаме, че $\texttt{Ord}(\bar{x},0,j-1)$ и $\texttt{Ord}(\bar{x},j,i)$.
      От всичко това имаме, че 
      \[x_0 \leq \dots \leq x_{j-1} \leq x_j \leq x_{j+1} \leq \dots \leq x_i.\]
      Заключаваме, че $\texttt{Ord}(\bar{x},0,i)$.
    \end{itemize}
  \end{itemize}
\item[($2 \to 4$)]
  Нека $A_2(\bar{z},\bar{x},i,j,n)$ е изпълнено. Щом се достигнали етикета 4, значи имаме и $i > n$.
  От $A_2$ пък имаме $i \leq n+1$.
  Следователно, $i = n+1$ и тогава $A_2(\bar{z},\bar{x},i,j,n) \implies \texttt{Ord}(\bar{x},0,n+1-1)$.
\end{description}
\end{proof}

\begin{cor}
  Програмата $P$ от \Fig{insertion-sort} 
  е частично коректна относно входното условие $I(\bar{z},n) \dff \bar{z}\in\Int^{n+1}$ и изходното условие $O(\bar{z},\bar{x},n) \equiv \texttt{Ord}(\bar{x},0,n)\ \&\ \texttt{Perm}(\bar{z},\bar{x},n)$.
\end{cor}

%%% Local Variables:
%%% mode: latex
%%% TeX-master: "../sep-problems"
%%% End:
