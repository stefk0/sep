\section{Област на Скот от непрекъснати изображения}

Следващата теорема е важна, защото с нейна помощ се доказват много свойства на непрекъснатите изображения.

\begin{thm}
  \label{th:double-chain}
  \marginpar{\cite[стр. 127]{winskel}}
  \marginpar{\cite[стр. 178]{models-of-computation}}
  Нека $\A = (A,\sqsubseteq,\bot)$ да бъде област на Скот и нека множеството 
  \[E = \{a_{m,n} \mid m,n \in \Nat\}\]
  от елементи на $A$ притежава свойството, че 
  \[n \leq n^\prime\ \&\ m \leq m^\prime\ \Rightarrow\ a_{n,m} \sqsubseteq a_{n^\prime,m^\prime}.\]
  Тогава множеството $E$ има точна горна граница, която означаваме с $\bigsqcup E$, и са изпълнени равенствата
  \[\bigsqcup E = \bigsqcup_m(\bigsqcup_n a_{n,m}) = \bigsqcup_n(\bigsqcup_{m} a_{n,m}) = \bigsqcup_n a_{n,n}.\]
\end{thm}
\begin{proof}
  Първо ще въведем някои означения.
  \begin{itemize}
  \item 
    Да фиксираме произволно $m$. Тогава можем да подредим елементите на множеството $\{a_{n,m} \mid n \in \Nat\}$ във възходящ ред: 
    \[a_{0,m} \sqsubseteq a_{1,m} \sqsubseteq a_{2,m} \sqsubseteq \cdots\]
    \marginpar{По дефиниция, всяка монотонно растяща редица в област на Скот притежава точна горна граница.}
    Следователно тя има точна горна граница $b_m \dff \bigsqcup \{a_{n,m} \mid n \in \Nat\}$.
  \item
    Аналогично, при фиксирано $n$, можем да подредим елементите на множеството $\{a_{n,m} \mid m \in \Nat\}$ в монотонно растяща редица:
    \[a_{n,0} \sqsubseteq a_{n,1} \sqsubseteq a_{n,2} \sqsubseteq \ldots,\]
    която притежава точна горна граница $c_n \dff \bigsqcup \{a_{n,m} \mid m \in \Nat\}$.
  \end{itemize}
  Това означава, че трябва да докажем следното:
  \[\bigsqcup E = \bigsqcup_mb_m = \bigsqcup_n c_n = \bigsqcup_n a_{n,n}.\]
  \begin{enumerate}[1)]
  \item 
    Първо да съобразим, че множеството $\{b_m \mid m \in \Nat\}$ образува верига в $\A$ и следователно притежава точна горна граница $\bigsqcup_m b_m$.
    Нека да разгледаме произволни $m \leq m^\prime$.
    Тогава \[(\forall n)[a_{n,m} \sqsubseteq a_{n,m^\prime} \sqsubseteq \bigsqcup_k a_{k,m^\prime} = b_{m^\prime}].\]
    Следователно $b_{m^\prime}$ е горна граница на веригата $(a_{n,m})^{\infty}_{n=0}$ и понеже $b_m$ е точна горна граница на $(a_{n,m})^{\infty}_{n=0}$, то получаваме, че \[b_m \sqsubseteq b_{m^\prime}.\]
    Това означава, че $\chain{b}{m}$ е верига в $\A$ и тя притежава точна горна граница $\bigsqcup_m b_m$.  
  \item
    С подобни разсъждения можем да докажем, че множеството $\{c_n \mid n \in \Nat\}$ образува верига в $\A$, която притежава точна горна граница $\bigsqcup_n c_n$.
  \item
    Сега ще докажем, че \[\bigsqcup_m b_m = \bigsqcup_n c_n.\]
    Имаме, че 
    \[(\forall m)(\forall n)[a_{n,m} \sqsubseteq \bigsqcup_na_{n,m} = b_m \sqsubseteq \bigsqcup_m b_m],\]
    което е еквивалентно на 
    \[(\forall n)(\forall m)[a_{n,m} \sqsubseteq b_m \sqsubseteq \bigsqcup_m b_m].\]
    Да фиксираме произволно $n$.
    Тогава $\bigsqcup_m b_m$ е горна граница на веригата $(a_{n,m})^\infty_{m=0}$.
    Следователно, $c_n = \bigsqcup_m a_{n,m} \sqsubseteq \bigsqcup_m b_m$.
    Така получаваме, че $\bigsqcup_m b_m$ е горна граница и на веригата $\chain{c}{n}$
    и тогава \[\bigsqcup_n c_n \sqsubseteq \bigsqcup_m b_m.\]
    С аналогични разсъждения можем да докажем също, че 
    \[\bigsqcup_m b_m \sqsubseteq \bigsqcup_n c_n.\]
    Така доказахме, че \[\bigsqcup_m b_m = \bigsqcup_n c_n.\]
  \item
    Сега ще докажем, че \[\bigsqcup E = \bigsqcup_m b_m.\]
    За целта ще проверим следното:
    \begin{enumerate}[a)]
    \item 
      $\bigsqcup_m b_m$ е горна граница на $E$.
    \item
      Ако $d$ е друга горна граница на $E$, то $\bigsqcup_m b_m \sqsubseteq d$.
    \end{enumerate}
    \begin{itemize}
    \item 
      $\bigsqcup_m b_m$ е горна граница на $E$, защото
      \[(\forall m)(\forall n)[a_{n,m} \sqsubseteq b_m \sqsubseteq \bigsqcup_m b_m].\]
    \item
      Нека $d$ е друга горна граница на $E$, т.е.
      \[(\forall m)(\forall n)[a_{n,m} \sqsubseteq d].\]
      Да фиксираме произволно $m$.
      Тогава $d$ е горна граница на веригата $(a_{n,m})^\infty_{n=0}$.
      Това означава, че $b_m = \bigsqcup_n a_{n,m} \sqsubseteq d$.
      Получаваме, че $d$ е горна граница на веригата $\chain{b}{m}$,
      откъдето следва, че $\bigsqcup_m b_m \sqsubseteq d$.
    \end{itemize}
    Заключаваме, че $\bigsqcup_m b_m$ е точната горна граница на $E$.
    Обобщавайки всичко от по-горе, следва, че:
    \[\bigsqcup E = \bigsqcup_m b_m = \bigsqcup_n c_n.\]
  \item
    Остана да видим, че 
    \[\bigsqcup E = \bigsqcup_n a_{n,n}.\]
    \begin{itemize}
    \item 
      Да разгледаме произволен елемент $a_{n,m} \in E$.
      Нека $k = \max\{n,m\}$.
      Ясно е, че $a_{n,m} \sqsubseteq a_{k,k} \sqsubseteq \bigsqcup_na_{n,n}$.
      Следователно, $\bigsqcup_n a_{n,n}$ е горна граница на $E$, откъдето получаваме
      \[\bigsqcup E \sqsubseteq \bigsqcup_n a_{n,n}.\]
    \item
      Нека $d$ е горна граница на $E$.
      Тогава $(\forall n)(\forall m)[a_{n,m} \sqsubseteq d]$
      и в частност, $(\forall n)[a_{n,n} \sqsubseteq d]$.
      Сега можем да заключим, че $\bigsqcup_n a_{n,n} \sqsubseteq d$.    
    \end{itemize}
    Така доказахме, че $\bigsqcup_n a_{n,n}$ е точна горна граница на $E$.
    \end{enumerate}
  С това доказателството на теоремата е завършено.
\end{proof}

\begin{framed}
  \begin{lemma}
    Нека $\A$ и $\B$ са области на Скот.
    Нека $\chain{f}{k}$ е верига от елементи на $\Cont{\A}{\B}$.
    Да дефинираме изображението $h$ на $\A$ в $\B$ по следния начин
    \[h(a) \dff \bigsqcup\{f_k(a) \mid k \in \Nat\}.\]
    Изображението $h$ е {\em непрекъснато} и е {\em точна горна граница} на веригата $\chain{f}{k}$,
    т.е. $h = \bigsqcup_k f_k$.
  \end{lemma}
\end{framed}
\marginpar{Ако $b_k = f_k(a)$, то $h(a)$ е точната горна граница на веригата $\chain{b}{k}$ в $\B$}
\begin{proof}
  \ifhints
  Доказателството, че $h$ е точна горна граница на веригата $\chain{f}{k}$ е лесно.
  \begin{itemize}
  \item 
    Да разгледаме произволен елемент $a \in A$.
    Лесно се вижда, че понеже $\chain{f}{k}$ е верига, то $(f_k(a))^\infty_{k=0}$ също е верига.
    Това е така, защото всяко непрекъснато изображение е също така и монотонно.

    \marginpar{$\bigsqcup_n f_n(a)$ е съкратен запис за $\bigsqcup\{f_n(a) \mid n \in \Nat\}$.}
    Получаваме, че за всяко $k$, $f_k(a) \sqsubseteq^\B \bigsqcup_n f_n(a) \dff h(a)$.
    Понеже това е вярно за произволно $a \in A$, $(\forall k)[f_k \sqsubseteq h]$,
    което означава, че $h$ е горна граница на веригата.
  \item
    Да разгледаме произволно изображение $g$, което е горна граница на веригата $\chain{f}{k}$.
    За произволен елемент $a \in A$, 
    \[(\forall k)[f_k(a) \sqsubseteq^\B g(a)].\]
    Това означава, че $g(a)$ е горна граница на веригата $(f_k(a))^\infty_{k=0}$.
    Понеже $h(a) = \bigsqcup_k \{f_k(a)\}$ е точната горна граница на веригата $(f_k(a))^\infty_{k=0}$,
    то $h(a) \sqsubseteq^\B g(a)$.
    Оттук следва, че $h \sqsupseteq g$.
  \end{itemize}
  \fi
  По-сложната част на доказателството е проверката, че $h$ е непрекъснато изображение.
  Да вземем една монотонно растяща редица $\chain{a}{k}$ от елементи на $A$.
  \marginpar{За момента дори не е ясно дали $\{h(a_k) \mid k \in \Nat\}$ е верига в $\B$}
  Ще докажем, че \[h(\bigsqcup_k a_k) = \bigsqcup_k \{h(a_k)\}.\]
  Нека $e_{n,m} \dff f_n(a_m)$.
  Понеже всяко $f_n$ е непрекъснато и следователно монотонно изображение, то имаме
  \[n \leq n^\prime\ \&\ m \leq m^\prime\ \Rightarrow\ e_{n,m} \sqsubseteq^{\B} e_{n^\prime,m^\prime}.\]
  Следователно,
  \begin{align*}
    h(\bigsqcup_m a_m) & = \bigsqcup_n(f_n(\bigsqcup_m a_m)) & \comment{\text{от деф. на }h}\\
                       & = \bigsqcup_n(\bigsqcup_m f_n(a_m)) & \comment{\text{ защото } f_n \text{ е непр.}}\\
                       & = \bigsqcup_n(\bigsqcup_m e_{n,m}) = \bigsqcup_m(\bigsqcup_n e_{n,m}) & \comment{\text{от \Th{double-chain}}}\\
                       & = \bigsqcup_m(\bigsqcup_n f_n(a_m)) & \comment{\text{от деф. на }e_{n.m}}\\
                       & = \bigsqcup_m \{h(a_m)\}. & \comment{\text{от деф. на }h}
  \end{align*}
\end{proof}

Да напомним, че релацията $\sqsubseteq$ между две изображения е дефинирана като
\[f \sqsubseteq g \dfff (\forall a\in A)[f(a) \sqsubseteq^\B g(a)].\]
\begin{framed}
  \begin{cor}
    $(\Cont{\A}{\B}, \sqsubseteq, \bm{\bot} )$ е област на Скот.
  \end{cor}
\end{framed}



% \begin{remark}
%   Интересен въпрос е при какви условия областта на Скот $\Cont{\A_1}{\A_2}$ е алгебрична.
%   Ние по-късно ще разгледаме този въпрос в един частен, но достатъчно общ, случай, 
%   а именно когато $\A_1$ и $\A_2$ са областта на Скот от лениви списъци.
% \end{remark}

% \begin{problem}
%   Нека $\A = (A, \sqsubseteq^\A, \bot^\A)$ е област на Скот, а $X$ е произволно непразно множество.
%   Определяме $\A^X = (A^X,\sqsubseteq^X,\bot^X)$ като:
%   \begin{itemize}
%   \item 
%     $A^X \dff \{f:X \to A\mid f\mbox{ е тотална}\}$;
%   \item
%     $f \sqsubseteq^X g\ \iff\ (\forall x \in X)[f(x) \sqsubseteq^\A g(x)]$;
%   \item
%     $\bot^X \in A^X$, като $(\forall x \in X)[\bot^X(x) = \bot^\A]$.
%   \end{itemize}
%   Докажете, че $\A^X$ е област на Скот.
% \end{problem}
% \begin{proof}
%   \begin{enumerate}[1)]
%   \item 
%     Лесно се вижда, че $\sqsubseteq^X$ е частична наредба.
%   \item
%     Също така, $\bot^X$ е най-малкият елемент на $A^X$ относно $\sqsubseteq^X$.
%   \item
%     Нека $(f_n)^\infty_{n=0}$ е верига от елементи на $A^X$. 
%     Ще докажем, че функцията $h$, дефинирана като $h(x) = \bigsqcup_n f_n(x)$,
%     е точна горна граница на $(f_n)^\infty_{n=0}$.
%     Но преди това, първо да съобразим, че $h \in A^X$.
%     Това следва от факта, че $\A$ е област на Скот и за всяко $x\in X$,
%     $(f_n(x))^\infty_{n=0}$ е верига в $\A$ 
%     и следователно има точна горна граница, която е равна на $h(x)$.
    
%     Сега ще покажем, че $h$ е горна граница на $(f_n)^\infty_{n=0}$.
%     Имаме, че:
%     \begin{align*}
%       (\forall x\in X)(\forall n)[f_n(x) \sqsubseteq^\A \bigsqcup_n f_n(x) = h(x)] & \iff (\forall n)(\forall x\in X)[f_n(x) \sqsubseteq^\A h(x)] \\
%       & \iff (\forall n)[f_n \sqsubseteq^\A h].
%     \end{align*}
    
%     Остана да видим, че $h$ е най-малката измежду горните граници на редицата $(f_n)^\infty_{n=0}$.
%     Нека $g$ е произволна горна граница на $(f_n)^\infty_{n=0}$. Това означава, че:
%     \begin{align*}
%       (\forall n)[f_n \sqsubseteq^X g] & \iff (\forall n)(\forall x\in X)[f_n(x) \sqsubseteq^\A g(x)]\\
%       & \iff (\forall x\in X)(\forall n)[f_n(x) \sqsubseteq^\A g(x)]
%     \end{align*}
%     Да разгледаме произволно $x \in X$. Тогава:
%     \[(\forall n)[f_n(x) \sqsubseteq^\A \bigsqcup_n f_n(x) \sqsubseteq^\A g(x)].\]
%     Понеже $h(x) = \bigsqcup_n f_n(x)$, получаваме, че за всяко $x \in X$,
%     \[h(x) \sqsubseteq^\A g(x),\]
%     което означава, че
%     $h \sqsubseteq^X g$.
%   \end{enumerate}
% \end{proof}

Нека $\A_1,\dots,\A_n$ и $\A$ са области на Скот и да разгледаме $f: \A_1\times \dots \times \A_n \to \A$.
Казваме, че $f$ е {\bf непрекъснато изображение по $i$-тия аргумент}, ако 
за всяка верига $\chain{a}{k}$ в $\A_i$, то
\[f(b_1,\dots, b_{i-1}, \bigsqcup_k a_k, b_{i+1},\dots,b_n) = \bigsqcup_kf(b_1,\dots, b_{i-1}, a_k, b_{i+1},\dots,b_n).\]

\begin{prop}
  Нека $\A_1,\dots,\A_n$ и $\A$ са области на Скот.
  Едно изображение $f: \A_1\times \dots \times \A_n \to \A$ 
  е непрекъснато точно тогава, когато $f$ е непрекъснато по всеки от аргументите си.
\end{prop}
\Stefan{Интересно е, че това твърдение не е вярно за непрекъснати свойства. Добре е да се обясни някъде.}

\begin{proof}
  \marginpar{\writedown Обобщете това твърдение за $n > 2$.}
  За по-просто изложение, да разгледаме случая $n = 2$.

  $(\Rightarrow)$ Лесно се съобразява, че ако $f$ е непрекъснато изображение, то $f$ е непрекъснато по всеки от аргументите си.
  \Stefan{Все пак е добре тук да се напише нещо.}
  
  $(\Leftarrow)$ Нека сега $f$ е непрекъснато по всеки от аргументите си. Ще докажем, че $f$ е непрекъснато.
  Нека $\{\pair{a_n,b_n}\}^\infty_{n=0}$ е верига в $\A_1\times \A_2$.
  Понеже от \Prop{cartesian} знаем, че
  \[\bigsqcup_n\pair{a_n,b_n} = \pair{\bigsqcup_na_n,\bigsqcup_n b_n},\]
  ще докажем, че 
  \[\bigsqcup_n f(a_n,b_n) = f(\bigsqcup_n a_n,\bigsqcup_n b_n).\]
  Да положим $e_{n,m} = f(a_n,b_m)$.
  Понеже $f$ е непрекъснато по всеки от аргументите си, лесно се вижда, че $f$
  е монотонно изображение по всеки от аргументите си. Следователно, 
  \[n \leq n^\prime\ \&\ m \leq m^\prime\ \Rightarrow\ e_{n,m} \sqsubseteq e_{n^\prime,m^\prime}.\]  
  Получаваме, че
  \begin{align*}
    \bigsqcup_n f(a_n,b_n) & = \bigsqcup_n e_{n,n} & \comment{\text{от опр. на }e_{n,m}}\\
                           & = \bigsqcup_n(\bigsqcup_m e_{n,m}) & \comment{\text{от \Th{double-chain}}}\\
                           & = \bigsqcup_n(\bigsqcup_m f(a_n,b_m)) & \comment{\text{от опр. на }e_{n,m}}\\
                           & = \bigsqcup_nf(a_n,\bigsqcup_m b_m) & \comment{f \text{ е непр. по втория си аргумент}}\\
                           & = f(\bigsqcup_n a_n,\bigsqcup_m b_m) & \comment{f \text{ е непр. по първия си аргумент}}.
  \end{align*}
\end{proof}


%%% Local Variables:
%%% mode: latex
%%% TeX-master: "../sep"
%%% End:
