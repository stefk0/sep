\section{Задачи}  

Тук с $\A$, $\B$ и $\C$ ще означаваме области на Скот.

\marginpar{Много от задачите са от \cite[стр. 31]{abramsky94}}

\begin{problem}
  Да разгледаме операторите \[\Gamma,\Delta \in \Cont{\Cont{\A}{\A}}{\Cont{\A}{\A}}.\]
  Знаем, че операторът $\Gamma \circ \Delta$ е непрекъснат, където
  \[(\Gamma\circ\Delta)(f) \dff \Gamma(\Delta(f)).\]
  Вярно ли е, че
  \[\lfp(\Gamma \circ \Delta) \sqsubseteq \lfp(\Gamma) \circ \lfp(\Delta)?\]
  Обосновете се!
\end{problem}
\ifhints
\begin{hint}
  Нека $\A = \Nat_\bot$.
  Нека например да разгледаме
  \begin{align*}
    & \Delta(f)(x) \dff f(x+1)\\
    & \Gamma(f)(x) \dff
      \begin{cases}
        0, & x \neq \bot\\
        \bot, & x = \bot.
      \end{cases}
  \end{align*}
  Да положим $f_\Gamma \dff \lfp(\Gamma)$ и $f_\Delta \dff \lfp(\Delta)$.
  Ясно е, че 
  \begin{align*}
    & f_\Delta(x) = \bot\\
    & f_\Gamma(x) =
    \begin{cases}
      0, & x \neq \bot\\
      \bot, & x = \bot.
    \end{cases}  
  \end{align*}
  Тогава за произволно $x \in \Nat_\bot$,
  \[(f_\Gamma\circ f_\Delta)(x) = f_\Gamma(f_\Delta(x)) = f_\Gamma(\bot)  = \bot.\]
  От друга страна, понеже $(\Gamma \circ \Delta)(f) = \Gamma(\Delta(f))$, то 
  \begin{align*}
    & (\Gamma \circ \Delta)(f)(x) = \Gamma(\Delta(f))(x) = 
      \begin{cases}
        0, & x \neq \bot\\
        \bot, & x = \bot.
      \end{cases}
  \end{align*}
  Лесно се съобразява, че 
  \[\lfp(\Gamma \circ \Delta)(x) =
  \begin{cases}
    0, & x \neq \bot\\
    \bot, & x = \bot.
  \end{cases}\]
  Заключаваме, че 
  \[\lfp(\Gamma \circ \Delta) \sqsupset \lfp(\Gamma) \circ \lfp(\Delta).\]
\end{hint}
\fi

\begin{problem}
  Да разгледаме операторите \[\Gamma,\Delta \in \Cont{\Cont{\A}{\A}}{\Cont{\A}{\A}}.\]
  Знаем, че операторът $\Gamma \circ \Delta$ е непрекъснат, където
  \[(\Gamma\circ\Delta)(f) \dff \Gamma(\Delta(f)).\]
  Вярно ли е, че
  \[\lfp(\Gamma \circ \Delta) \sqsupseteq \lfp(\Gamma) \circ \lfp(\Delta)?\]
  Обосновете се!
\end{problem}
\ifhints
\begin{hint}
  Нека $\A = \Nat_\bot$.
  Нека например да разгледаме
  \begin{align*}
    & \Delta(f)(x) \dff 0\\
    & \Gamma(f)(x) \dff
      \begin{cases}
        0, & x = 0\\
        \bot, & \text{ иначе}.
      \end{cases}
  \end{align*}
  Да положим $f_\Gamma \dff \lfp(\Gamma)$ и $f_\Delta \dff \lfp(\Delta)$.
  Ясно е, че 
  \begin{align*}
    & f_\Delta(x) = 0\\
    & f_\Gamma(x) =
    \begin{cases}
      0, & x = 0\\
      \bot, & \text{ иначе}.
    \end{cases}  
  \end{align*}
  Тогава за произволно $x \in \Nat_\bot$,
  \[(f_\Gamma\circ f_\Delta)(x) = f_\Gamma(f_\Delta(x)) = f_\Gamma(0)  = 0.\]
  От друга страна, понеже $(\Gamma \circ \Delta)(f) = \Gamma(\Delta(f))$, то 
  \begin{align*}
    & (\Gamma \circ \Delta)(f)(x) = \Gamma(\Delta(f))(x) = 
      \begin{cases}
        0, & x = 0\\
        \bot, & \text{ иначе}.
      \end{cases}
  \end{align*}
  Лесно се съобразява, че 
  \[\lfp(\Gamma \circ \Delta)(x) =
  \begin{cases}
    0, & x = 0\\
    \bot, & \text{ иначе}.
  \end{cases}\]
  Заключаваме, че 
  \[\lfp(\Gamma \circ \Delta) \sqsubset \lfp(\Gamma) \circ \lfp(\Delta).\]
\end{hint}
\fi

\begin{problem}
  Нека $f_0 \sqsubseteq f_1 \sqsubseteq f_2 \sqsubseteq \cdots$
  е верига от елементи на $\Cont{\A}{\A}$.
  Да положим $h = \bigsqcup_n f_n$.
  Вярно ли е, че 
  \[h \circ h = \bigsqcup_n (f\circ f)?\]
  Обосновете се!
\end{problem}

\begin{problem}
  Да разгледаме едно изображение $f: \A \times \B \to \C$.
  За произволно $a \in \A$, дефинираме изображението $g_a: \B \to \C$, където
  \[g_a(b) \dff f(a,b).\]
  Аналогично, за произволно $b \in \B$, дефинираме изображението $h_b: \A \to \C$, където
  \[h_b(a) \dff f(a,b).\]
  Докажете, че следните твърдения са еквивалентни:
  \begin{enumerate}[1)]
  \item 
    $f$ е непрекъснато изображение;
  \item
    $g_a$ и $h_b$ са непрекъснати изображения, за всяко $a \in \A$ и $b \in \B$.
  \end{enumerate}
\end{problem}

\begin{problem}
  Да разгледаме $f \in \Cont{\A \times \B}{\C}$.
  За произволно $a \in \A$, дефинираме изображението $g_a: \B \to \C$, където
  \[g_a(b) \dff f(a,b).\]
  Вече знаем, че $g_a \in \Cont{\B}{\C}$, за всяко $a \in \A$.
  Да разгледаме изображението $h:\A \to \Cont{\B}{\C}$, където
  \[h(a) \dff g_a.\]
  Докажете, че $h$ е непрекъснато изображение.
\end{problem}

\begin{problem}
  \marginpar{\cite[стр. 129]{nikolova-soskova}}
  Да разгледаме $f \in \Cont{\A \times \B}{\B}$.
  За произволно $a \in \A$, дефинираме изображението $g_a: \B \to \B$, където
  \[g_a(b) \dff f(a,b).\]
  Вече знаем, че $g_a \in \Cont{\B}{\B}$, за всяко $a \in \A$,
  следователно $\lfp(g_a)$ съществува.
  Да разгледаме изображението $h:\A \to \B$, където
  \[h(a) \dff \lfp(g_a).\]
  Докажете, че $h$ е непрекъснато изображение.
\end{problem}


\begin{problem}
  Да разгледаме $f \in \Cont{\A}{\Cont{\B}{\C}}$.
  За произволно $a \in \A$, дефинираме изображението $g_a \in \Cont{\B}{\C}$, където
  \[g_a(b) \dff f(a).\]
  Да разгледаме изображението $h:\A\times \B \to \C$, където
  \[h(a,b) \dff g_a(b).\]
  Докажете, че $h$ е непрекъснато изображение.
\end{problem}

\begin{problem}
  Нека са дадени областите на Скот $\D$ и $\E$ и изображението 
  \[\texttt{eval}: \Cont{\D}{\E} \times \D \to \E,\]
  където 
  \[\texttt{eval}(f,d) \dff f(d).\]
  Докажете, че $\texttt{eval}$ е непрекъснато изображение.
\end{problem}
\ifhints
\begin{hint}
  Понеже $\bigsqcup_n(f_n,d_n) = (\bigsqcup_m f_m,\bigsqcup_n d_n)$, 
  ще докажем, че \[\texttt{eval}(\bigsqcup_m f_m, \bigsqcup_n d_n) = \bigsqcup_n \texttt{eval}(f_n,d_n).\]
  Знаем, че
  \begin{align*}
    \texttt{eval}(\bigsqcup_m f_m, \bigsqcup_n d_n) & = (\bigsqcup_m f_m)(\bigsqcup_n d_n) & (\mbox{от опр. на }\texttt{eval})\\
    & = \bigsqcup_m (f_m(\bigsqcup_n d_n)) & (\mbox{от опр. на }\bigsqcup_mf_m)\\
    & = \bigsqcup_m (\bigsqcup_n (f_m(d_n)) & (\mbox{всяка }f_m\mbox{ е непр.} )\\
  \end{align*}
  Нека да положим $e_{m,n} = f_m(d_n)$.
  Лесно се съобразява, че
  \[m \leq m^\prime\ \&\ n \leq n^\prime\ \Rightarrow\ e_{m,n} \sqsubseteq^\E e_{m^\prime,n^\prime}.\]
  Така получаваме, че 
  \begin{align*}
    \texttt{eval}(\bigsqcup_m f_m, \bigsqcup_n d_n) & = \bigsqcup_m (\bigsqcup_n (f_m(d_n)) & (\mbox{от по-горе})\\
    & = \bigsqcup_{m,n} e_{m,n} = \bigsqcup_{n} e_{n,n} & (\mbox{от \Th{double-chain}})\\
    & = \bigsqcup_{n} f_n(d_n) & (\text{ от опр. на }e_{m,n})\\
    & = \bigsqcup_n \texttt{eval}(f_n,d_n).
  \end{align*}
\end{hint}
\fi


% \begin{problem}
%   Нека фиксираме $f \in \Cont{\A}{\B}$.
%   Да образуваме изображението 
%   \[\Gamma: \Cont{\B}{\C} \to \Cont{\A}{\C},\]
%   където
%   \[\Gamma(g) \dff g \circ f.\]
%   Докажете, че $\Gamma$ е непрекъснато изображение.
% \end{problem}

\begin{problem}
  Нека изображението \[\texttt{comp}:(\Cont{\B}{\C} \times \Cont{\A}{\B}) \to \Cont{\A}{\C}\]
  е определено като 
  \[\texttt{comp}(g,f) \dff g\circ f.\]
  \marginpar{$(g \circ f)(a) = g(f(a))$}
  Докажете, че $\texttt{comp}$ е непрекъснато изображение.
\end{problem}
\ifhints
\begin{hint}
  \marginpar{\cite[стр. 124]{reynolds}}
  Трябва да докажем, че за всяка монотонно растяща редица $\{(g_n,f_n)\}_{n\in\Nat}$,
  \[\Gamma(\bigsqcup_n(g_n,f_n))(a) = \bigsqcup_n\Gamma(g_n,f_n)(a),\]
  за произволно $a \in A$.
  Да фиксираме едно $a\in A$ и да положим $g_n(f_k(a)) = e_{n,k}$.
  Лесно се вижда, че 
  \[n\leq n^\prime\ \&\ k \leq k^\prime\ \Rightarrow\ e_{n,k} \sqsubseteq e_{n^\prime,k^\prime}.\]
  Тогава:
  \begin{align*}
    \Gamma(\bigsqcup_n(g_n,f_n))(a) & = \Gamma(\bigsqcup_n g_n, \bigsqcup_k f_k)(a) & \\
    & = (\bigsqcup_n g_n)(\bigsqcup_k f_k(a)) & (\text{ по деф. на }\Gamma )\\
    & = (\bigsqcup_n g_n)(\bigsqcup_k b_k) & (\text{ полагаме }b_k = f_k(a))\\
    & = \bigsqcup_k (\bigsqcup_n g_n)(b_k) & (\bigsqcup_n g_n\text{ е непр.})\\
    & = \bigsqcup_k(\bigsqcup_n g_n(b_k)) & (\text{ по деф. на }\bigsqcup_n g_n)\\
    & = \bigsqcup_k (\bigsqcup_n g_n(f_k(a))) & (\text{ полагаме }e_{n,k} = g_n(f_k(a)))\\
    & = \bigsqcup_k\bigsqcup_n e_{n,k} = \bigsqcup_n e_{n,n} & (\text{ от \Th{double-chain}})\\
    & = \bigsqcup_n g_n(f_n(a)) = \bigsqcup_n \Gamma(g_n, f_n)(a).
  \end{align*}
\end{hint}
\fi

\begin{remark}
  \marginpar{Когато пишем $(.)$ означава, че операцията е инфиксна}
  В хаскел има операция композиция на функции.
  \begin{haskellcode}
ghci> :t (.)
(.) :: (b -> c) -> (a -> b) -> a -> c
  \end{haskellcode}
\end{remark}

\begin{problem}
  \marginpar{\cite[стр. 131]{nikolova-soskova}}
  Нека $f \in \Cont{\A}{\B}$ и $g \in \Cont{\B}{\A}$.
  Докажете, че 
  \begin{itemize}
  \item 
    $\lfp(g \circ f) \sqsubseteq g(\lfp(f \circ g))$;
  \item
    $f(\lfp(g \circ f)) \sqsubseteq \lfp(f \circ g)$.
  \end{itemize}
  Оттук заключете, че 
  \[\lfp(g \circ f) = g(\lfp(f \circ g)) \text{ и }f(\lfp(g \circ f)) = \lfp(f \circ g).\]
\end{problem}


% \begin{problem}[Кантор-Шрьодер-Бернщайн]
%   \marginpar{\cite[стр. 639]{hanbook-cs}}
%   Нека имаме две инективни функции $f:A \to B$ и $g:B \to A$.
%   Тогава съществува биективна функция $h: A \to B$.  
% \end{problem}
% \begin{hint}
%   За множеството $B$, да дефинираме областта на Скот 
%   \[\D_B = (\Ps(B),\subseteq,\emptyset).\]
%   \begin{enumerate}[a)]
%   \item 
%     За дадените от условието инективни функции $f$ и $g$,
%     да разгледаме изображението $\Gamma:\D_B \to \D_B$ зададено като
%     \marginpar{Озн. $h(X) = \{h(x) \mid x\in X\}$, $h^{-1}(X) = \{z \mid h(z) \in X\}$}
%     \[\Gamma(X) = B\setminus f(A)\cup f(g(X)).\]
%     Докажете, че $F$ е непрекъснато изображение.
%   \item
%     \Stefan{Използвам, че $X_0$ е неподвижна точка, но не виждам къде използвам, че е най-малката.}
%     Нека $X_0 = \lfp(\Gamma)$. Тогава $X_0 = B\setminus f(A) \cup f(g(X_0))$.
%     Докажете, че 
%     \[B \setminus X_0 = f(A \setminus g(X_0)).\]
%   \item
%     Дефинираме функцията $h:A \to B$ по следния начин:
%     \begin{align*}
%       h(a) = 
%       \begin{cases}
%         g^{-1}(a), & a \in g(X_0)\\
%         f(a), & a \in A \setminus g(X_0).
%       \end{cases}
%     \end{align*}
%     Докажете, че $h$ е биекция.
%   \end{enumerate}
% \end{hint}

\begin{problem}% Gunter textbook
  Нека $f \in \Cont{\A}{\A}$.
  Да разгледаме множеството 
  \[B = \{a \in \A \mid f(a) = a\}.\]
  Вярно ли е, че 
  \[\B = (B, \sqsubseteq^\A, \lfp(f))\] е област на Скот?
  Обосновете се!
\end{problem}

\begin{problem}% Gunter textbook
  Нека $f \in \Cont{\A}{\A}$.
  Да разгледаме множеството 
  \[B = \{a \in \A \mid f(a) \sqsubseteq a\}.\]
  Вярно ли е, че 
  \[\B = (B, \sqsubseteq^\A, \lfp(f))\] е област на Скот?
  Обосновете се!
\end{problem}


\begin{problem} % Gunter textbook
  Да разгледаме множеството
  \[B = \{f \in \Mon{\A}{\A} \mid f\circ f = f\}.\]
  Вярно ли е, че 
  \[\B = (B,\ \sqsubseteq,\ \lambda x.\bot^\A)\] е област на Скот,
  където 
  \[f \sqsubseteq g \dfff (\forall a\in\A)[f(a) \sqsubseteq^\A g(a)] ?\]
  Обосновете се!
\end{problem}

% \begin{problem} % Gunter textbook
%   Да разгледаме множеството
%   \[B = \{f \in \Strict{\A}{\A} \mid f\circ f = f\}.\]
%   Вярно ли е, че 
%   \[\B = (B,\ \sqsubseteq,\ \lambda x.\bot^\A)\] е област на Скот,
%   където 
%   \[f \sqsubseteq g \dfff (\forall a\in\A)[f(a) \sqsubseteq^\A g(a)] ?\]
%   Обосновете се!
% \end{problem}

\begin{problem} % Gunter textbook
  Да разгледаме множеството
  \[B = \{f \in \Strict{\Nat_\bot}{\Nat_\bot} \mid f\circ f = f\}.\]
  Вярно ли е, че 
  \[\B = (B,\ \sqsubseteq,\ \lambda x.\bot)\] е област на Скот,
  където 
  \[f \sqsubseteq g \dfff (\forall a\in\Nat_\bot)[f(a) \sqsubseteq g(a)] ?\]
  Обосновете се!
\end{problem}

\begin{problem}
  % \marginpar{\cite[стр. 124]{reynolds}}
  Нека $f \in \Mon{\A}{\B}$ и $g \in \Mon{\B}{\A}$ имат свойствата:
  \begin{itemize}
  \item 
    $f\circ g = id_\B$;
  \item
    $g \circ f = id_\A$.
  \end{itemize}
  Докажете, че $f$ и $g$ са точни и непрекъснати.
\end{problem}


\begin{problem}
  % \marginpar{задачата е \href{http://www.cl.cam.ac.uk/teaching/exams/pastpapers/y2008p8q14.pdf}{оттук} и \href{http://www.cl.cam.ac.uk/teaching/exams/pastpapers/y1998p9q10.pdf}{оттук}}
  Да разгледаме областта на Скот 
  \[\O = (\{\bot,\top\},\sqsubseteq, \bot),\]
  където $\bot \sqsubseteq \top$.
  За произволна област на Скот $\A$ и елемент $a \in \A$, $a \neq \bot$, дефинираме изображенията:
  \begin{enumerate}[a)]
  \item
    $f_a:\A \to \O$, където
    \[f_a(x) \dff
    \begin{cases}
      \top, & a \sqsubseteq x\\
      \bot, & a \not\sqsubseteq x.
    \end{cases}\]
    Вярно ли е, че $f_a$ е точно непрекъснато изображение? Обосновете се!
  \item
    $\hat{f}_a:\A \to \O$, където
    \[\hat{f}_a(x) \dff
    \begin{cases}
      \bot, & a \sqsubseteq x\\
      \top, & a \not\sqsubseteq x.
    \end{cases}\]
    Вярно ли е, че $\hat{f}_a$ е точно непрекъснато изображение? Обосновете се!
  \item 
    $g_a:\A \to \O$, където
    \[g_a(x) \dff
    \begin{cases}
      \bot, & x \sqsubseteq a\\
      \top, & x \not\sqsubseteq a.
    \end{cases}\]
    Вярно ли е, че $g_a$ е точно непрекъснато изображение? Обосновете се!
  \item 
    $\hat{g}_a:\A \to \O$, където
    \[\hat{g}_a(x) \dff
    \begin{cases}
      \top, & x \sqsubseteq a\\
      \bot, & x \not\sqsubseteq a.
    \end{cases}\]
    Вярно ли е, че $\hat{g}_a$ е точно непрекъснато изображение? Обосновете се!
  \item
    Докажете, че 
    \[f \in \Cont{\D}{\A} \iff (\forall a \in \A)[g_a \circ f \in \Cont{\D}{\O}].\]
  \end{enumerate}
\end{problem}

\begin{problem}
  Да разгледаме изображението
  \[\Gamma: \Cont{\A}{\B} \times \Cont{\A}{\C} \to \Cont{\A}{\B\times\C},\]
  където $\Gamma(f,g)(a) \dff \pair{f(a),g(b)}$.
  \begin{itemize}
  \item
    Докажете, че $\Gamma$ е добре дефинирано изображение, т.е. за всеки непрекъснати $f$ и $g$,
    $\Gamma(f,g)$ е непрекъснато изображение.
  \item 
    Докажете, че $\Gamma$ е непрекъснато изображение.
  \end{itemize}
\end{problem}

\begin{problem}
  \marginpar{задачата е \href{http://www.cl.cam.ac.uk/teaching/exams/pastpapers/y2005p9q15.pdf}{оттук}}
  Докажете, че изображението
  \[\texttt{uncurry}:\Cont{\A}{\Cont{\B}{\C}} \to \Cont{\A\times \B}{\C},\]
  дефинирано като
  \[\texttt{uncurry}(f)(a,b) \dff f(a)(b),\]
  е непрекъснато.
\end{problem}

\begin{problem}
  Докажете, че изображението
  \[\texttt{curry}:\Cont{\A\times \B}{\C} \to \Cont{\A}{\Cont{\B}{\C}},\]
  дефинирано като
  \[\texttt{curry}(f)(a)(b) \dff f(a,b),\]
  е непрекъснато.
\end{problem}

\newpage
\subsection{Регулярни езици}

Да фиксираме азбуката $\Sigma = \{a_1,\dots,a_k\}$.
Да дефинираме полиномите над $\Sigma$ като
\[\tau ::= \emptyset\ |\ \varepsilon\ |\ a_i \cdot X_j\ |\ \tau_1 + \tau_2.\]
където $i = 1, \dots,k$, а $X$ е променлива.
За всеки полином $\tau[X_1,\dots,X_n]$ дефинираме оператора 
\[\val{\tau}: \mathcal{P}(\Sigma^\star)^n \to \mathcal{P}(\Sigma^\star)\]
 по следния начин:
\begin{itemize}
\item
    $\val{\emptyset}(L_1,\dots,L_n) = \emptyset$.
\item 
  $\val{\varepsilon}(L_1,\dots,L_n) = \varepsilon$.
\item 
  $\val{a_i \cdot X_j}(L_1,\dots,L_n) = \{a_i\} \cdot L_j$.
\item
  $\val{\tau_1 + \tau_2}(L_1,\dots,L_n) = \val{\tau_1}(L_1,\dots,L_n) \cup \val{\tau_2}(L_1,\dots,L_n)$.
\end{itemize}

\begin{problem}
  Докажете, че за всеки полином $\tau$ имаме, че $\val{\tau}$ е непрекъснато изображение в областта на Скот
  $\mathcal{S} = ( \mathcal{P}(\Sigma^\star),\subseteq, \emptyset)$.
\end{problem}


\begin{example}
  Да разгледаме системата 
  \marginpar{$\tau_1[X_1,X_2] \equiv b \cdot X_1 + a \cdot X_2$}
  \marginpar{$\tau_2[X_1,X_2] \equiv \varepsilon$}
  \begin{align*}
    & X_1 = b \cdot X_1 + a\cdot X_2\\
    & X_2 = \varepsilon.
  \end{align*}

  % Понеже $\val{\tau}$ е непрекъснат оператор, то той има най-малка неподвижна точка.
  Дефинираме непрекъснатия оператор 
  \[\Gamma:\mathcal{P}(\Sigma^\star)^2 \to \mathcal{P}(\Sigma^\star)^2,\]
  където:
  \[\Gamma(L_1,L_2) = (\val{\tau_1}(L_1,L_2), \val{\tau_2}(L_1,L_2)).\]

  От Теоремата на Клини ние знам как можем да намерим най-малката неподвижна точка на $\Gamma$,
  която ще бъде и най-малкото решение на горната система.

  \begin{itemize}
  \item 
    $(L_0,M_0) \df (\emptyset,\emptyset)$;
  \item
    $(L_1,M_1) \df \Gamma(L_0,M_0) = (\val{\tau_1}(L_0,M_0), \val{\tau_2}(L_0,M_0)) = (\emptyset, \varepsilon)$;
  \item
    $(L_2,M_2) \df \Gamma(L_1,M_1) = (\val{\tau_1}(L_1,M_1), \val{\tau_2}(L_1,M_1)) = (\{a\},\varepsilon)$;
  \item
    $(L_3,M_3) \df \Gamma(L_2,M_2) = (\val{\tau_1}(L_2,M_2), \val{\tau_2}(L_2,M_2)) = (\{ba,a\},\varepsilon)$;
  \item
    $(L_4,M_4) \df \Gamma(L_3,M_3) =(\val{\tau_1}(L_3,M_3), \val{\tau_2}(L_3,M_3)) = (\{bba, ba,a\},\varepsilon)$;
  \item
    $(L_5,M_5) \df \Gamma(L_4,M_4) = ( \val{\tau_1}(L_4,M_4), \val{\tau_2}(L_4,M_4)) = (\{bba, bba, ba,a\},\varepsilon)$.
  \end{itemize}
  Лесно се съобразява, че $L_n = \{ b^ka \mid k < n\}$.
  Тогава
  \[\lfp( \Gamma ) = (\bigcup_n L_n, \{\varepsilon\}) = (b^\star a, \{\varepsilon\} ).\]
\end{example}


\begin{problem}
  Докажете, че най-малкото решение на системата 
  \begin{align*}
    & X_1 = a \cdot X_1 + b \cdot X_2 + \varepsilon\\
    & X_2 = b \cdot X_2 + \varepsilon
  \end{align*}
  е двойката $(a^\star b^\star, b^\star)$.
\end{problem}

\begin{problem}
  Да разгледаме системата от оператори
  \begin{align*}
    & \val{\tau_1}(L_1,\dots,L_n) = L_1\\
    & \ \ \vdots\\
    & \val{\tau_n}(L_1,\dots,L_n) = L_n.
  \end{align*}
  Знаем, че тя притежава най-малко решение $(\hat{L}_1,\dots,\hat{L}_n)$.
  Докажете, че всеки от езиците $\hat{L}_i$ е регулярен.

  Докажете, че всеки регулярен език е елемент от най-малкото решение 
  на някоя система от оператори от горния вид.
\end{problem}


% \begin{problem}
%   Докажете, че всеки регулярен език е елемент на най-малкото решение на някоя система от $n$
%   полинома с $n$ променливи за някое $n$.
% \end{problem}

% \begin{problem}
%   Докажете, че всяко най-малко решение на система от $n$ полинома с $n$ променливи представлява $n$-орка от 
%   регулярни езици.
% \end{problem}

% \begin{problem}
%   Опишете алгоритъм, по който може от система от $n$ полинома с $n$ променливи да се построи 
%   краен автомат с $n$ състояния.
% \end{problem}

% \begin{problem}
%   Опишете алгоритъм, по който може от краен автомат с $n$ състояния може да се построи 
% \end{problem}



\subsection{Безконтекстни езици}

Да фиксираме азбуката $\Sigma = \{a_1,\dots,a_n\}$.
Да дефинираме термове от тип 1 като
\[\tau ::= X_i\ |\ a_j\ |\ \varepsilon\ |\ \emptyset\ |\ \tau_1 \cdot \tau_2\ |\ (\tau_1 + \tau_2),\]
където $j = 1, \dots,n$, а $X_i$ са изброимо безкрайна редица от променливи.
За всеки терм $\tau[X_1,\dots,X_n]$ дефинираме оператора 
\[\val{\tau}: (\mathcal{P}(\Sigma^\star))^n \to \mathcal{P}(\Sigma^\star)\]
 по следния начин:
\begin{itemize}
\item 
  $\val{X_i}(L_1,\dots,L_n) = L_i$.
\item 
  $\val{a_j}(L_1,\dots,L_n) = \{a_j\}$.
\item 
  $\val{\varepsilon}(L_1,\dots,L_n) = \varepsilon$.
\item 
  $\val{\emptyset}(L_1,\dots,L_n) = \emptyset$.
\item 
  $\val{\tau_1 \cdot \tau_2}(L_1,\dots,L_n) = \val{\tau_1}(L_1,\dots,L_n) \cdot \val{\tau_2}(L_1,\dots,L_n)$.
\item
  $\val{\tau_1 + \tau_2}(L_1,\dots,L_n) = \val{\tau_1}(L_1,\dots,L_n) \cup \val{\tau_2}(L_1,\dots,L_n)$.
\end{itemize}

\begin{problem}
  Докажете, че за всеки терм $\tau$, $\val{\tau}$ е непрекъснато изображение в областта на Скот
  $\mathcal{S} = ( \mathcal{P}(\Sigma^\star),\subseteq, \emptyset)$.
\end{problem}

\begin{problem}
  Докажете, че $\{a^nb^n \mid n\in \Nat\} = \lfp(\val{\tau})$, където 
  \[\tau[X] \equiv \varepsilon + a \cdot X \cdot b.\]
  С други думи, $\{a^nb^n \mid n \in \Nat\}$ е най-малкото решение на уравнението
  \[X = a \cdot X \cdot b + \varepsilon.\]
\end{problem}

Нека сега да разгледаме термовете $\tau_1[X_1,\dots,X_n], \dots, \tau_n[X_1,\dots,X_n]$.

\begin{problem}
  Да разгледаме системата от оператори
  \begin{align*}
    & \val{\tau_1}(L_1,\dots,L_n) = L_1\\
    & \ \ \vdots\\
    & \val{\tau_n}(L_1,\dots,L_n) = L_n.
  \end{align*}
  Знаем, че тя притежава най-малко решение $(\hat{L}_1,\dots,\hat{L}_n)$.
  Докажете, че всеки от езиците $\hat{L}_i$ е безконтекстен.

  Докажете, че всеки безконтекстен език е елемент от най-малкото решение 
  на някоя система от оператори от горния вид.
\end{problem}

\begin{problem}
  \marginpar{Това е аналог на нормалната форма на Чомски}
  Да дефинираме термове от тип 2 като
  \[\tau ::= a_j\ |\ \varepsilon\ |\ \emptyset\ |\ X_i \cdot X_k\ |\ (\tau_1 + \tau_2),\]
  където $j = 1, \dots,n$, а $X_i$ са изброимо безкрайна редица от променливи.
  Докажете горното твърдение, като замените термовете от тип 1 с тези от тип 2.
\end{problem}

\begin{example}
  Да разгледаме системата
  \begin{align*}
    & X_1 = X_3 \cdot X_2 + \varepsilon\\
    & X_2 = X_1 \cdot X_4\\
    & X_3 = a\\
    & X_4 = b.
  \end{align*}


  % \begin{align*}
  %   & \val{\varepsilon + X_3 \cdot X_2}(L_1, L_2, L_3, L_4) = L_1\\
  %   & \val{X_1 \cdot X_4}(L_1, L_2, L_3, L_4) = L_2\\
  %   & \val{a}(L_1, L_2, L_3, L_4) = L_3\\
  %   & \val{b}(L_1, L_2, L_3, L_4) = L_4\\
  % \end{align*}
  Нека $(\hat{L}_1, \hat{L}_2, \hat{L}_3, \hat{L}_4)$ е най-малкото решение на системата.
  Докажете, че $\hat{L}_1 = \{a^nb^n\mid n \in \Nat\}$ $\hat{L}_2 = \{a^nb^{n+1}\mid n \in \Nat\}$,
  $\hat{L}_3 = \{a\}$ и $\hat{L}_4 = \{b\}$.
\end{example}


%%% Local Variables:
%%% mode: latex
%%% TeX-master: "../sep-notes"
%%% End:
