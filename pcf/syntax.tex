\section{Синтаксис}

\newcommand{\fix}{\texttt{fix}}
\newcommand{\fv}{\texttt{fv}}

\[\vv{a} ::= \vv{nat}\ |\ \vv{a} \to \vv{a}.\]


Изрази
\[\tau ::= \vv{n}\ |\ \vv{x}\ |\ \tau + \tau\ |\ \tau\ \vv{==}\ \tau\ |\ \ifelse{\tau}{\tau}{\tau}\ |\ \tau\tau\ |\ \lambda \vv{x:a}.\tau\ |\ \fix(\tau).\]


Да обърнем внимание, че ние няма да правим разлика между два израза, които са $\alpha$-еквивалентни, т.е.
ще считаме, че $\lambda x:a. (x+1)$ е същият израз като $\lambda y:a. (y+1)$.

Понеже тук вече имам свободни и свързани променливи, трябва да дефинираме точно какво означава това.
\[\fv:\mathcal{E} \to \mathcal{V}\]
Ще дефинираме функцията $\texttt{fv}$ със структурна индукция по построението на термовете.

\begin{itemize}
\item
  $\fv(\vv{n}) = \emptyset$;
\item
  $\fv(\vv{x}) = \{\vv{x}\}$;
\item
  $\fv(\fix(\tau)) = \fv(\tau)$;
\item
  $\fv(\tau_1 + \tau_2) = \fv(\tau_1\ \vv{==}\ \tau_2) = \fv(\tau_1\tau_2) = \fv(\tau_1) \cup \fv(\tau_2)$;
\item
  $\fv(\ifelse{\tau_1}{\tau_2}{\tau_3}) = \fv(\tau_1) \cup \fv(\tau_2) \cup \fv(\tau_3)$;
\item
  $\fv(\lambda x.a:\tau) = \fv(\tau) \setminus \{x\}$.
\end{itemize}

Ще казваме, че един израз $\tau$ е {\bf затворен}, ако $\fv(\tau) = \emptyset$.
В противен случай, ще казваме, че изразът е {\bf отворен}.

%%% Local Variables:
%%% mode: latex
%%% TeX-master: "../sep"
%%% End:
