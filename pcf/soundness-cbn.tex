\section{Коректност}

\marginpar{Да напомним, че когато термът $\tau$ е затворен, то ще пишем $\val{\tau}$ вместо $\val{\tau}_\emptyset(\bot)$.}
Понеже вече имаме дефинирани операционна и денотационна семантика на термовете,
следващата стъпка е да разгледаме каква е връзката между тях.
В този раздел ще докажем едната (по-лесната) посока.
\marginpar{На англ. {\em soundness}.}
\begin{framed}
  \begin{theorem}[Теорема за коректност]\label{th:pcf:soundness}
    За всеки затворен терм $\tau : \vv{b}$ и стойност $\type{v}{b}$, е изпълнена импликацията:
    \[\tau \opsem{}{b} \vv{v}\ \implies\ \val{\tau} = \val{\vv{v}} \in \val{\vv{b}}.\]
  \end{theorem}  
\end{framed}
\begin{proof}
  Индукция по дължината $\ell$ на извода $\opsem{\ell}{b}$ за всеки тип $\vv{b}$.
  Нека $\ell = 0$. Имаме два случая, защото имаме два вида стойности.
  \begin{itemize}
  \item
    Нека $\tau \equiv \vv{n}$.
    Ясно е, че $\vv{b} = \vv{nat}$ и от правилата на операционната семантика имаме, че:
    \begin{prooftree}
      \AxiomC{}
      \RightLabel{\scriptsize{(val)}}
      \UnaryInfC{$\tau \opsem{0}{nat} \vv{n}$}
    \end{prooftree}
    От дефиницията на семантика на терм, директно получаваме, че
    $\val{\tau} = n = \val{\vv{n}}$.    
  \item
    Нека $\tau \equiv \lamb{x}{c}{\tau'}$. Тогава $\vv{b} = \vv{a}\to\vv{c}$ и от правилата на операционната семантика имаме, че:
    \begin{prooftree}
      \AxiomC{}
      \RightLabel{\scriptsize{(val)}}
      \UnaryInfC{$\tau \opsemGen{0}{\vv{a}\to\vv{c}} \tau$}
    \end{prooftree}
    Ясно е, че $\val{\tau} = \val{\tau} \in \val{b}$.
  \end{itemize}
  Така доказахме, че
  \[\tau \opsem{0}{b} \vv{v}\ \implies\ \val{\tau} = \val{\vv{v}} \in \val{\vv{b}}.\]
  Нека сега $\ell > 0$ и да приемем, че имаме следното индукционно предположение:
  \[\tau \opsem{<\ell}{b} \vv{v}\ \implies\ \val{\tau} = \val{\vv{v}} \in \val{\vv{b}}.\]
  Ще докажем, че
  \[\tau \opsem{\ell}{b} \vv{v}\ \implies\ \val{\tau} = \val{\vv{v}} \in \val{\vv{b}}.\]
  \begin{itemize}
  \item
    Нека $\tau \equiv \tau_1 + \tau_2$. Тогава от правилата на операционната семантика имаме, че:
    \begin{prooftree}
      \AxiomC{$\tau_1 \opsem{\ell_1}{nat} \vv{n}_1$}
      \AxiomC{$\tau_2 \opsem{\ell_2}{nat} \vv{n}_2$}
      \LeftLabel{\scriptsize{($\ell=\ell_1+\ell_2+1$)}}
      \RightLabel{\scriptsize{(plus)}}
      \BinaryInfC{$\tau_1 + \tau_2 \opsem{\ell_1+\ell_2+1}{nat} \vv{n},$}
    \end{prooftree}
    където $n = n_1 + n_2$. От \IndHyp получаваме, че
    \begin{align*}
      & \val{\tau_1} = \val{\vv{n}_1} = n_1\\
      & \val{\tau_2} = \val{\vv{n}_2} = n_2.
    \end{align*}
    Тогава
    \begin{align*}
      \val{\tau_1 + \tau_2} & = \plus(\val{\tau_1}, \val{\tau_2}) & \comment\text{от деф.}\\
                            & = n_1 + n_2 & \comment\text{\IndHyp}\\
                            & = n.
    \end{align*}
  \item
    Случаите $\tau \equiv \tau_1 - \tau_2$ и $\tau \equiv \tau_1\ \vv{==}\ \tau_2$ са аналогични. Оставяме ги на читателя.
  \item
    Нека $\tau \equiv \ifelse{\tau_1}{\tau_2}{\tau_3}$. Тогава от правилата на операционната семантика имаме, че:
    \begin{prooftree}
      \AxiomC{$\tau_1 \opsem{\ell_1}{nat} \vv{n}_1$}
      \AxiomC{$\tau_2 \opsem{\ell_2}{a} \vv{v}_2$}
      \AxiomC{$\vv{n}_1 \not\equiv \vv{0}$}
      \LeftLabel{\scriptsize{($\ell=\ell_1+\ell_2+1$)}}
      \RightLabel{\scriptsize{(if$^+$)}}
      \TrinaryInfC{$\ifelse{\tau_1}{\tau_2}{\tau_3} \opsem{\ell_1+\ell_2+1}{a} \vv{v}_2,$}
    \end{prooftree}
    Тогава от \IndHyp получаваме, че:
    \begin{align*}
      & \val{\tau_1} = n_1\\
      & \val{\tau_2} = \val{\vv{v}_2}.
    \end{align*}
    Тогава
    \begin{align*}
      \val{\ifelse{\tau_1}{\tau_2}{\tau_3}} & = \texttt{if}(\val{\tau_1}, \val{\tau_2}, \val{\tau_3})\\
                                            & = \texttt{if}(n_1,\val{\tau_2}, \val{\tau_3}) & \comment\text{от \IndHyp}\\
                                            & = \val{\tau_2} & \comment\text{от деф. на }\texttt{if}\\
                                            & = \val{\vv{v}_2}. & \comment\text{от \IndHyp}
    \end{align*}
    
    Случаят, когато $\vv{n}_1 \equiv \vv{0}$ е аналогичен.
  \item
    Нека $\tau \equiv \tau_1 \tau_2$. Тогава от правилата на операционната семантика имаме, че:
    \begin{prooftree}
      \AxiomC{$\tau_1 \opsemGen{\ell_1}{\vv{a}\to\vv{b}} \lamb{x}{a}{\tau'_1}$}
      \AxiomC{$\tau'_1[x/\tau_2] \opsem{\ell_2}{b} \vv{v}$}
      \LeftLabel{\scriptsize{($\ell=\ell_1+\ell_2+1$)}}
      \RightLabel{\scriptsize{(cbn)}}
      \BinaryInfC{$\tau_1 \tau_2 \opsem{\ell_1+\ell_2+1}{b} \vv{v} $}
    \end{prooftree}
    Тогава от \IndHyp получаваме, че:    
    \begin{align*}
      & \val{\tau_1} = \val{\lamb{x}{a}{\tau'_1}} \in \Cont{\val{\vv{a}}}{\val{\vv{b}}}\\
      & \val{\tau'_1\subst{x}{\tau_2}} = \val{\vv{v}} \in \val{\vv{b}}.
    \end{align*}
    Обядиняваме всичко и получаваме равенствата:
    \begin{align*}
      \val{\tau_1\tau_2} & = \texttt{eval}(\val{\tau_1},\val{\tau_2}) & \comment\text{от деф.}\\ 
                         & = \val{\tau_1}(\val{\tau_2}) & \comment \val{\tau_1} \in \Cont{\val{\vv{a}}}{\val{\vv{b}}}\\
                         & = \val{\lamb{x}{a}{\tau'_1}}(\val{\tau_2}) & \comment\text{\IndHyp}\\
                         & = \val{\tau'_1}_{\type{x}{a}}(\val{\tau_2})\\
                         & = \val{\tau'_1\subst{x}{\tau_2}} & \comment\text{от \hyperref[lem:pcf:substitution]{Лема за замяната}}\\
                         & = \val{\vv{v}} & \comment\text{\IndHyp}
    \end{align*}
  \item
    Нека $\tau \equiv \fix(\tau')$. Тогава от правилата на операционната семантика имаме, че:
    \begin{prooftree}
      \AxiomC{$\tau'\ \fix(\tau') \opsem{\ell-1}{a} \vv{v}$}
      \RightLabel{\scriptsize{(fix)}}
      \UnaryInfC{$\fix(\tau') \opsem{\ell}{a} \vv{v} $}
    \end{prooftree}
    Тогава от \IndHyp имаме, че:
    \[\val{\tau'\ \fix(\tau')} = \val{\vv{v}}.\]
    От правилата за типизиране знаем, че $\tau' : \vv{a}\to\vv{a}$.
    Сега остава да съобразим, че щом $\val{\tau'} \in \Cont{\val{\vv{a}}}{\val{\vv{a}}}$,
    то изображението $\val{\tau'}$ притежава най-малка неподвижна точка $\lfp(\val{\tau'})$.
    Знаем, че според дефиницията на неподвижна точка, $\val{\tau'}(\lfp(\val{\tau'}) = \lfp(\val{\tau'})$.
    Сега сме готови да завършим доказателството:
    \begin{align*}
      \val{\tau} & = \val{\fix(\tau')} \\
                 & = \lfp(\val{\tau'})& \comment\text{от деф.}\\
                 & = \val{\tau'}(\lfp(\val{\tau'})) & \comment\text{неподв. точка}\\
                 & = \val{\tau'}(\val{\fix(\tau')}) & \comment\text{от деф.}\\
                 & = \texttt{eval}(\val{\tau'}, \val{\fix(\tau')})\\
                 & = \val{\tau'\fix(\tau')} & \comment\text{от деф.}\\
                 & = \val{\vv{v}}. & \comment\text{\IndHyp}
    \end{align*}
  \end{itemize}
\end{proof}

\hyperref[th:pcf:soundness]{Теоремата за коректност}\ частично потвърждава нашата интуиция, че за типа $\vv{nat}$
можем да си мислим за $\bot^{\val{\vv{nat}}}$ като за изчисление, което никога не завършва.

\marginpar{Другата посока ще я получим след малко.}

\begin{framed}
  \begin{corollary}
    \label{cor:pcf:soundness}
    Нека $\tau$ е затворен терм от тип $\vv{nat}$. Тогава е изпълнена импликацията
    \[\val{\tau} = \bot^{\val{\vv{nat}}}\ \implies\ \tau \not\opsem{}{nat}.\]
  \end{corollary}
\end{framed}

За жалост, тази наша интуиция се ,,губи'', когато се интересуваме от термове от по-висок от $\nat$ тип.
\marginpar{За дискусия по този въпрос вижте \cite[стр. 213]{models-of-computation}.}
Нека просто да разгледаме терма $\Omega'_{\vv{a}}$ от тип $\vv{a}\to\vv{a}$.
Имаме, че $\val{\Omega'_{\vv{a}}} = \bot^{\val{\vv{a}\to\vv{a}}}$, но
$\Omega'_{\vv{a}}\opsemGen{}{\vv{a}\to\vv{a}} \Omega'_{\vv{a}}$.



% Да разгледаме един пример. Нека 
% \[\tau \equiv \lamb{y}{nat}{\fix(\lamb{x}{nat}{\vv{x}})}.\]
% Лесно се съобразява, че $\tau : \nat\to\nat$.
% За произволен елемент $a \in \Nat_\bot$ е изпъленено следното:
% \begin{equation*}
%   \setlength{\jot}{10pt}
%   \begin{split}
%     \val{\tau}(a) & = \val{\lamb{y}{nat}{\fix(\lamb{x}{nat}{\vv{x}})}}(a)\\
%     & = \curry(\val{\fix(\lamb{x}{nat}{\vv{x}})}_{\type{y}{nat}})(a)\\
%     & = \val{\fix(\lamb{x}{nat}{\vv{x}})}_{\type{y}{nat}}(a)\\
%     & = \lfp(\val{\lamb{x}{nat}{\vv{x}}}_{\type{y}{nat}}(a))\\
%     & = \lfp(\curry(\val{\vv{x}}_{\type{y}{nat},\type{x}{nat}})(a)).
%   \end{split}
% \end{equation*}
% Сега, ясно е, че
% \begin{align*}
%   & \val{\vv{x}}_{\type{y}{nat},\type{x}{nat}} \in \Cont{\Nat_\bot\times\Nat_\bot}{\Nat_\bot},\\
%   & \curry(\val{\vv{x}}_{\type{y}{nat},\type{x}{nat}}) \in \Cont{\Nat_\bot}{\Cont{\Nat_\bot}{\Nat_\bot}}.
% \end{align*}
% За произволни елементи $a,b\in\Nat_\bot$, имаме следното:
% \begin{align*}
%   \curry(\val{\vv{x}}_{\type{y}{nat},\type{x}{nat}})(a)(b) & = \val{\vv{x}}_{\type{y}{nat},\type{x}{nat}}(a,b)\\
%                                                            & = b.
% \end{align*}
% \marginpar{$\texttt{id}_{\val{\nat}}(b) = b$ за всяко $b \in \Nat_\bot$.}
% Следователно, ако означим с $\texttt{id}_{\val{\nat}}$ функцията идентитет върху $\Nat_\bot$, то за всяко $a \in \Nat_\bot$,
% \[\curry(\val{\vv{x}}_{\type{y}{nat},\type{x}{nat}})(a) = \texttt{id}_{\val{\nat}}.\]
% Сега завършваме горната верига от равенства така:
% \begin{equation*}
%   \setlength{\jot}{10pt}
%   \begin{split}
%     \val{\tau}(a) & = \lfp(\curry(\val{\vv{x}}_{\type{y}{nat},\type{x}{nat}})(a))\\
%     & = \lfp(\texttt{id}_{\val{\nat}})\\
%     & = \bot^{\val{\nat}}.
%   \end{split}
% \end{equation*}
\marginpar{Това означава, че ако искаме денотационната семантика да кореспондира по-точно с операционната семантика, ние трябва да въведем нов елемент за $\bot$ за области на Скот съответстващи на типове по-високи от $\nat$.}
% С други думи, получаваме, че
% \[\val{\tau} = \bot^{\val{\nat\to\nat}}.\]
% От друга страна, обаче, $\tau$ представлява стойност. Следователно,
% \[\tau \opsemGen{0}{\nat\to\nat} \tau.\]


%%% Local Variables:
%%% mode: latex
%%% TeX-master: "../sep"
%%% End:
