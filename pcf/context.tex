\section{Контекстна еквивалентност}\label{pcf:sect:context}

\index{контекст}
\marginpar{\cite[Глава 6.1]{gunter}}
\marginpar{\cite[Глава 48]{practical-foundations}}
\marginpar{Интуитивно, контекстите са програмни фрагменти. Не са пълни програми, защото имат празни места означени с $-$.}
\[\C ::= -\ |\ \vv{n}\ |\ \vv{x}\ |\ \C - \C\ |\ \C + \C\ |\ \C\ \vv{==}\ \C\ |\ \ifelse{\C}{\C}{\C}\ |\ \C\C\ |\ \lamb{x}{a}{\C}\ |\ \fix(\C).\]
Контекстите са на практика изрази, като позволяваме да имат специална свободна променлива, която означаваме с $-$.
% За произволен израз $\tau$, с $\C\{\tau\}$ означаваме израза $\C\rename{-}{\tau}$, където считаме $-$ като свободна променлива на $\C$.
% Това означава, че заместваме всички срещания на $-$ с $\tau$ като правим заместването директно, т.е.
% не се интересуваме дали някоя свободна променлива на $\tau$ няма да попадне под обхвата на свързана променлива в $\C$.
% Например, ако $\C = \lamb{x}{a}{-}$, то $\C\rename{-}{\vv{x}} =
% \lamb{x}{a}{\vv{x}}$.
За краткост, вместо $\C[-/\tau]$ ще пишем $\C[\tau]$.

% \begin{proposition}
%   % \marginpar{Да напомним, че с $\equiv$ означаваме релацията $\alpha$-еквивалентност.}
%   Ако $\tau \equiv_\alpha \tau'$, то $\C[\tau] \equiv_\alpha \C[\tau']$.
% \end{proposition}
% \begin{hint}
%   Индукция по построението на контекстите.
% \end{hint}

Ще казваме, че {\bf завършена програма} е затворен терм от тип $\nat$.


\marginpar{Някои наричат тази релация observational equivalence. Тук наричаме релацията contextual equivalence.}
\begin{framed}
  \begin{definition}\label{df:context:equivalence}
    За затворени термове $\tau_1$ и $\tau_2$, дефинираме
    $\tau_1 \leq_{ctx} \tau_2 : \vv{a}$, ако
    \begin{enumerate}[1)]
    \item
      $\emptyset \vdash \tau_1 : \vv{a}$ и $\emptyset \vdash \tau_2 : \vv{a}$
    \item
      За \emph{всички} контексти $\C[-]$, за които $\emptyset \vdash \C[\tau_1] : \nat$ и $\emptyset \vdash \C[\tau_2] : \nat$, то
      \[(\forall \vv{n})[\ \C[\tau_1] \Downarrow_{\nat} \vv{n} \implies \C[\tau_2] \Downarrow_{\nat} \vv{n}\ ].\]
    \end{enumerate}
    Ще пишем $\tau_1 \cong_{ctx} \tau_2 : \vv{a}$, ако
    $\tau_1 \leq_{ctx} \tau_2 : \vv{a}$ и $\tau_2 \leq_{ctx} \tau_1 : \vv{a}$.
  \end{definition}
\end{framed}

Възможно е по-общо да се дефинира релацията $\leq_{ctx}$ не само за затворени, но и за произволни добре типизирани термове. За нашите цели е достатъчно да се ограничим само до затворени термове.

% \begin{itemize}
% \item 
  % Ще пишем $\tau_1 \cong_{ctx} \tau_2 : \vv{a}$, ако
  % $\tau_1 \leq_{ctx} \tau_2 : \vv{a}$ и $\Gamma \vdash \tau_2 \leq_{ctx} \tau_1 : \vv{a}$.
% \item
%   Ако $\Gamma = \emptyset$, то ще пишем $\tau_1 \cong_{ctx} \tau_2 : \vv{a}$ вместо $\emptyset \vdash \tau_1 \cong_{ctx} \tau_2 : \vv{a}$.  
% \end{itemize}

\marginpar{Two phrases of a programming language are contextually equivalent if any occurrences of the first phrase in a complete
  program can be replaced by the second phrase without affecting the observable results of executing the program.
  This kind of program equivalence is also known as operational, or observational equivalence \cite[стр. 44]{cambridge-den-sem}.}

\begin{framed}
  Денотационната семантика $\val{.}$ се нарича {\bf напълно абстрактна}, ако денотационната и операционната наредба съвпадат, т.е. за всеки два затворени терма $\tau_1$ и $\tau_2$ от тип $\vv{a}$ е изпълнено, че:
  \[\val{\tau_1} \sqsubseteq \val{\tau_2}\text{ точно тогава, когато }\tau_1 \leq_{ctx} \tau_2 : \vv{a}.\]
\end{framed}

Един от основните ранни резултати в изучаването на семантиката на езици за програмиране е, че нашата денотационна семантика не е напълно абстрактна.
Практически, ако имаме два терма, които са контекстно еквивалентни, то можем да заменим единия с другия в произволна програма,
без да има видими разлики на ниво изпълнение на програмата.
% Това представлява опит да се формализира математически практиката за тестване на програми.
Проблемът е, че с този формализъм е трудно да се работи, защото в дефиницията имаме квантор
за всеобщност относно всички контексти (програмни фрагменти).

\begin{proposition}\label{pr:pcf:context:simple}
  За произволни добре типизирани затворени термове $\tau_1$ и $\tau_2$ е изпълнено, че:
  \begin{enumerate}[(1)]
  \item
    \label{pr:pcf:context:simple:base}
    $\tau_1 \leq_{ctx} \tau_2 : \nat \implies (\forall \vv{n})[\tau_1\Downarrow_{\nat}\vv{n} \implies \tau_2 \Downarrow_{\nat} \vv{n}]$.
  \item
    \label{pr:pcf:context:simple:step}
    $\tau_1 \leq_{ctx} \tau_2 : \arr{b}{c} \implies (\forall \rho\in\text{PCF}_{\vv{b}})[\tau_1\rho \leq_{ctx} \tau_2 \rho : \vv{c}]$.
  \end{enumerate}
\end{proposition}
\begin{hint}
  \begin{enumerate}[(1)]
  \item
    Просто вземете контекст $\C \df -$. Ясно е, че $\C[\tau_i] \equiv \tau_i$.
  \item
    Да фиксираме произволен терм $\rho\in\text{PCF}_{\vv{b}}$.
    Да разгледаме произволен контекст $\C[-]$, за който $\C[\tau_1\rho]$ и $\C[\tau_2\rho]$
    са затворени термове от тип $\nat$.
    Разглеждаме контекста $\C' \df \C[-\rho]$,
    т.е. заменяме $-$ с $-\rho$ в контекста $\C$.
    Лесно се съобразява, че \[\C'[\tau_i] \equiv \C[\tau_i\rho].\]
    Понеже $\tau_1 \leq_{ctx} \tau_2 : \vv{b}\to\vv{c}$,
    то за специалния контекст $\C'$ имаме, че
    \[(\forall \vv{n})[\ \C'[\tau_1] \Downarrow_{\nat} \vv{n} \implies \C'[\tau_2] \Downarrow_{\nat} \vv{n}\ ].\]
    Оттук веднага следва, че за първоначалния контекст $\C$ имаме, че
    \[(\forall \vv{n})[\ \C[\tau_1\rho] \Downarrow_{\nat} \vv{n} \implies \C[\tau_2\rho] \Downarrow_{\nat} \vv{n}\ ].\]
  \end{enumerate}
\end{hint}

\begin{proposition}\label{pr:pcf:context:terms}
  За произволни затворени термове $\tau_1$ и $\tau_2$ е изпълнено, че:
  \[\tau_1 \leq_{ctx} \tau_2 : \vv{a} \iff (\forall \rho \in \text{PCF}_{\vv{a}\to\nat})(\forall \vv{n})[\ \rho\tau_1 \Downarrow_{\nat} \vv{n} \implies \rho\tau_2 \Downarrow_{\nat} \vv{n}\ ].\]
\end{proposition}
\begin{proof}
  За $(\Rightarrow)$, при даден терм $\rho \in \text{PCF}_{\vv{a}\to\nat}$, просто вземете контекстът $\C$ просто да бъде $\C\df\rho~-$. Тогава е ясно, че $\C[\tau_1]$ и $\C[\tau_2]$ са завършени програми и следователно, щом $\tau_1 \leq_{ctx} \tau_2 : \vv{a}$, то е изпълнено, че:
  \[(\forall \vv{n})[\ \underbrace{\C[\tau_1]}_{\rho\tau_1} \Downarrow_{\nat} \vv{n} \implies \underbrace{\C[\tau_2]}_{\rho\tau_2} \Downarrow_{\nat} \vv{n}\ ].\]
  За $(\Leftarrow)$, нека разгледаме произволен контекст $\C[-]$,
  за който $\C[\tau_1] \opsem{}{nat} \vv{n}$. Трябва да докажем, че
  $\C[\tau_2] \opsem{}{nat} \vv{n}$.

  Нека $\vv{z}$ е променлива, която не се среща в $\C[-]$.
  Да разгледаме терма $\rho' \df \C[-/\vv{z}]$, т.е. в контекста $\C$ заменяме $-$ с нова променлива $\vv{z}$. Ясно е, че $\type{z}{a} \vdash \rho' : \nat$.
  Разгледайте затворения терм $\rho \df \lamb{z}{a}{\rho'}$ от тип $\vv{a} \to \nat$.
  Тогава, понеже $\tau_i$ са затворени термове, то
  $\C[\tau_i] \equiv \rho'\subst{z}{\tau_i}$.
  
  Да напомним, че от правилата на операционната семантика имаме, че
  \begin{prooftree}
    \AxiomC{$\rho$ е стойност}
    \LeftLabel{\footnotesize{(val)}}
    \UnaryInfC{$\rho \opsemGen{}{\vv{a}\to\nat} \rho$}
    \AxiomC{$\rho'\subst{z}{\tau_1} \opsem{}{nat} \vv{n}$}
    \RightLabel{\footnotesize{(cbn)}}
    \BinaryInfC{$\rho\tau_1 \opsem{}{nat} \vv{n}$}
  \end{prooftree}
  Това означава, че имаме и $\rho \tau_2 \opsem{}{nat} \vv{n}$.
  Тогава пак според правилата на операционната семантика е ясно, че със сигурност имаме и
  $\rho'\subst{z}{\tau_2} \opsem{}{nat} \vv{n}$.
  Така доказахме, че $\tau_1 \leq_{ctx} \tau_2 : \vv{a}$.
\end{proof}

Сега започва да става интересно, защото ще свържем логическата релация $\triangleleft_{\vv{a}}$ с релацията $\leq_{ctx}$.

\begin{proposition}\label{pr:pcf:context:relation}
  Нека $\tau_1$ и $\tau_2$ са произволни затворени термове от тип $\vv{a}$ и $d$ е елемент на областта на Скот $\val{\vv{a}}$. Тогава
  \[(d \triangleleft_{\vv{a}} \tau_1\ \&\ \tau_1 \leq_{ctx} \tau_2 : \vv{a}) \implies d \triangleleft_{\vv{a}} \tau_2.\]
\end{proposition}
\begin{proof}
  Доказателството следва дефиницията на логическата релация $\triangleleft_{\vv{a}}$, което означава, че
  ще направим индукция по построението на типовете $\vv{a}$.

  Нека $\vv{a} = \nat$.
  Да приемем, че $d \neq \bot$, защото е ясно, че $\bot \triangleleft_{\nat} \tau_2$.
  Понеже $d \triangleleft_{\nat} \tau_1$, то от дефиницията на $\triangleleft_{\nat}$ веднага следва, че $\tau_1 \opsem{}{nat} \vv{d}$.
  Понеже $\tau_1 \leq_{ctx} \tau_2 : \nat$, то според \ref{pr:pcf:context:simple:base} на \Prop{pcf:context:simple} имаме, че $\tau_2 \Downarrow_{\nat} \vv{d}$. Заключаваме, че $d \triangleleft_{\nat} \tau_2$.
  
  Нека $\vv{a} = \arr{b}{c}$ и да приемем, че имаме $d \triangleleft_{\arr{b}{c}} \tau_1$ и $\tau_1 \leq_{ctx} \tau_2 : \arr{b}{c}$.
  Трябва да докажем, че $d \triangleleft_{\vv{b}\to\vv{c}} \tau_2$, което според дефиницията на логическата релация $\triangleleft_{\vv{b}\to\vv{c}}$ означава,
  че за произволни $u \in \val{\vv{b}}$ и $\rho \in \text{PCF}_{\vv{b}}$, за които $u \triangleleft_{\vv{b}} \rho$, трябва да докажем, че $d(u) \triangleleft_{\vv{c}} \tau_2\rho$.
  Да разгледаме произволен такъв елемент $u$ и терм $\rho$. 
  От $d \triangleleft_{\vv{b} \to \vv{c}} \tau_1$ имаме, че $d(u) \triangleleft_{\vv{c}} \tau_1 \rho$.
  От \ref{pr:pcf:context:simple:step} на \Prop{pcf:context:simple} имаме, че щом $\tau_1 \leq_{ctx} \tau_2 : \vv{b} \to \vv{c}$, то $\tau_1 \rho \leq_{ctx} \tau_2 \rho : \vv{c}$.
  Сега от \IndHyp заключаваме, че $d(u) \triangleleft_{\vv{c}} \tau_2 \rho$.
\end{proof}

\begin{important}
  \begin{proposition}\label{pr:pcf:context:relation-characterization}
    За произволни затворени термове $\tau_1$ и $\tau_2$ от тип $\vv{a}$,
    \[\tau_1 \leq_{ctx} \tau_2 : \vv{a} \iff \val{\tau_1} \triangleleft_{\vv{a}} \tau_2.\]
  \end{proposition}    
\end{important}
\begin{proof}
  $(\Rightarrow)$ Нека $\tau_1 \leq_{ctx} \tau_2 : \vv{a}$.
  От \Prop{pcf:adequacy:implication} имаме, че
  \[\val{\tau_1} \triangleleft_{\vv{a}} \tau_1.\]
  Тогава от \Prop{pcf:context:relation} директно следва, че $\val{\tau_1} \triangleleft_{\vv{a}} \tau_2$.

  $(\Leftarrow)$ Нека сега $\val{\tau_1} \triangleleft_{\vv{a}} \tau_2$.
  Тук на помощ ни идва характеризацията на $\leq_{ctx}$ от \Prop{pcf:context:terms}.
  Да разгледаме произволен терм $\rho \in \text{PCF}_{\vv{a} \to \nat}$.
  Трябва да докажем, че:
  \[(\forall \vv{n})[\ \rho\tau_1 \opsem{}{nat} \vv{n} \implies \rho\tau_2 \opsem{}{nat} \vv{n}\ ].\]
  Според \Prop{pcf:adequacy:implication} имаме, че $\val{\rho} \triangleleft_{\arr{a}{nat}} \rho$.
  Тогава от дефиницията на релацията $\triangleleft_{\arr{a}{nat}}$ следва, че
  за произволен елемент $e$ от областта на Скот $\val{\vv{a}}$ и произволен затворен терм $\mu:\vv{a}$,
  ако $e \triangleleft_{\vv{a}} \mu$, то $\val{\rho}(e) \triangleleft_{\nat} \rho\mu$.
  Нека да положим $e \df \val{\tau_1}$ и $\mu \df \tau_2$. Тогава получаваме, че:
  \[\underbrace{\val{\rho\tau_1}}_{\val{\rho}(\val{\tau_1})} \triangleleft_{\nat} \rho\tau_2.\]
  Сега за произволна стойност $\vv{n}$ имаме, че:
  \begin{align*}
    \rho\tau_1 \Downarrow_{\nat} \vv{n} & \implies \val{\rho\tau_1} = n & \comment\text{\hyperref[th:pcf:soundness]{Теорема за коректност}}\\
                                            & \implies \rho\tau_2 \Downarrow_{\nat} \vv{n}. & \comment\text{от }\val{\rho\tau_1} \triangleleft_{\nat} \rho\tau_2
  \end{align*}
  Заключаваме, че $\tau_1 \leq_{ctx} \tau_2 : \vv{a}$.
\end{proof}


\begin{framed}
  \begin{proposition}\label{pr:pcf:context:extensionality}
    За произволни затворени термове $\tau_1$ и $\tau_2$ е изпълнено, че:
    \begin{enumerate}[(1)]
    \item
      \label{pr:pcf:context:extensionality:base}
      $\tau_1 \leq_{ctx} \tau_2 : \nat \iff (\forall \vv{n})[\tau_1 \Downarrow_{\nat} \vv{n} \implies \tau_2 \Downarrow_{\nat} \vv{n}]$;
    \item
      \label{pr:pcf:context:extensionality:step}
      $\tau_1 \leq_{ctx} \tau_2 : \arr{a}{b} \iff (\forall \rho \in \text{PCF}_{\vv{a}})[\ \tau_1\rho \leq_{ctx} \tau_2 \rho : \vv{b}\ ]$.
    \end{enumerate}
  \end{proposition}  
\end{framed}
\begin{proof}
  \begin{enumerate}[(1)]
  \item
    Посоката $(\Rightarrow)$ следва директно от \ref{pr:pcf:context:simple:base} на \Prop{pcf:context:simple}.
    За посоката $(\Leftarrow)$, понеже
    \begin{align*}
      \val{\tau_1} = \val{\vv{n}} & \implies \tau_1 \Downarrow_{\nat} \vv{n} & \comment\text{\hyperref[th:pcf:adequacy]{Теорема за адекватност}}\\
                                  & \implies \tau_2 \Downarrow_{\nat} \vv{n}, & \comment\text{от условието}
    \end{align*}
    то получаваме, че $\val{\tau_1} \triangleleft_{\nat} \tau_2$.
    Тогава от \Prop{pcf:context:relation-characterization} получаваме, че $\tau_1 \leq_{ctx} \tau_2 : \nat$.
  \item
    Посоката $(\Rightarrow)$ е на практика \ref{pr:pcf:context:simple:step} на \Prop{pcf:context:simple}.
    За посоката $(\Leftarrow)$, според \Prop{pcf:context:relation-characterization},
    достатъчно е да докажем, че $\val{\tau_1} \triangleleft_{\vv{a}\to\vv{b}} \tau_2$.
    Според дефиницията на логическата релация $\triangleleft_{\vv{a}\to\vv{b}}$, това означава да
    докажем че за произволни $u \in \val{\vv{a}}$ и $\rho \in \text{PCF}_{\vv{a}}$, за които $u \triangleleft_{\vv{a}} \rho$, то е изпълнено, че $\val{\tau_1}(u) \triangleleft_{\vv{b}} \tau_2\rho$.
    Да разгледаме произволни такъв елемент $u$ и терм $\rho$.
    Тъй като от \Cor{pcf:fundamental} знаем, че $\val{\tau_1} \triangleleft_{\vv{a}\to\vv{b}} \tau_1$, то
    $\val{\tau_1}(u) \triangleleft_{\vv{b}} \tau_1\rho$.
    Но понеже $\tau_1\rho \leq_{ctx} \tau_2 \rho : \vv{b}$, то от \Prop{pcf:context:relation} следва, че
    $\val{\tau_1}(u) \triangleleft_{\vv{b}} \tau_2 \rho$.
  \end{enumerate}
\end{proof}

\begin{problem}
  \label{prob:pcf:context:application}
  Докажете, че винаги можем да направим следния извод:
  \begin{prooftree}
    \AxiomC{$\tau_1 \leq_{ctx} \tau_2 : \arr{a}{b}$}
    \AxiomC{$\rho_1 \leq_{ctx} \rho_2 : \vv{a}$}
    \BinaryInfC{$\tau_1\rho_1 \leq_{ctx} \tau_2\rho_2 : \vv{b}$}
  \end{prooftree}
\end{problem}
\begin{hint}
  От $\tau_1 \leq_{ctx} \tau_2 : \arr{a}{b}$ имаме, че $\val{\tau_1} \triangleleft_{\arr{a}{b}} \tau_2$.
  Тогава за произволен елемент $u \in \val{\vv{a}}$ и произволен $\mu \in \PCF_{\vv{a}}$,
  за които $u \triangleleft_{\vv{a}} \mu$ е изпълнено, че $\val{\tau_1}(u) \triangleleft_{\vv{b}} \tau_2\mu$.
  
  От $\rho_1 \leq_{ctx} \rho_2 : \vv{a}$ имаме, че $\val{\rho_1} \triangleleft_{\vv{b}} \rho_2$.
  Нека положим $u \df \val{\rho_1}$ и $\mu \df \rho_2$.
  Тогава получаваме, че $\val{\tau_1}(\val{\rho_1}) \triangleleft_{\vv{b}} \tau_2 \rho_2$.
  Сега заключаваме, че $\tau_1\rho_1 \leq_{ctx} \tau_2\rho_2 : \vv{b}$.
\end{hint}

Естествено е да се запитаме дали можем да разширим \ref{pr:pcf:context:extensionality:base} на \Prop{pcf:context:extensionality} за по-сложни от $\nat$ типове $\vv{a}$, т.е. възможно ли е
\[\tau_1 \leq_{ctx} \tau_2 : \vv{a} \iff (\forall \vv{v})[\tau_1 \Downarrow_{\vv{a}} \vv{v} \implies \tau_2 \Downarrow_{\vv{a}} \vv{v}]?\]
Би било странно, ако можем, защото това би обезсмилило разглеждането на релацията $\leq_{ctx}$.
Първо в \Prop{context:op-left-right} ще видим, че винаги имаме импликацията $(\Leftarrow)$, но по-късно в \Prop{context:op-right-left} ще видим, че дори за типа $\vv{a} = \nat\to\nat$ нямаме импликация $(\Rightarrow)$.

\begin{proposition}\label{pr:context:op-left-right}
  Докажете, че за всеки два затворени терма $\tau_1, \tau_2$ от тип $\vv{a}$,
  \[(\forall \vv{v})[\tau_1 \Downarrow_{\vv{a}} \vv{v} \implies \tau_2 \Downarrow_{\vv{a}} \vv{v} ] \implies \tau_1 \leq_{ctx} \tau_2 : \vv{a}.\]
\end{proposition}
\begin{proof}
  Имаме всичко необходимо за да докажем това твърдение:
  \begin{prooftree}
    \AxiomC{от условието}
    \UnaryInfC{$(\forall \vv{v})[\tau_1 \Downarrow_{\vv{a}} \vv{v} \implies \tau_2 \Downarrow_{\vv{a}} \vv{v}]$.}
    \AxiomC{\Cor{pcf:fundamental}}
    \UnaryInfC{$\val{\tau_1} \triangleleft_{\vv{a}} \tau_1$}
    \RightLabel{\footnotesize{(\Prop{pcf:adequacy:implication})}}
    \BinaryInfC{$\val{\tau_1} \triangleleft_{\vv{a}} \tau_2$}
    \RightLabel{\footnotesize{(\Prop{pcf:context:relation-characterization})}}
    \UnaryInfC{$\tau_1 \leq_{ctx} \tau_2 : \vv{a}$}
  \end{prooftree}
\end{proof}

\begin{proposition}\label{pr:context:den-left-right}
  За произволни затворени термове $\tau_1$ и $\tau_2$ от тип $\vv{a}$,
  \[\val{\tau_1} \sqsubseteq \val{\tau_2} \implies \tau_1 \leq_{ctx} \tau_2 : \vv{a}.\]
\end{proposition}  
\begin{proof}
  Според \Prop{pcf:context:terms}, достатъчно е да докажем, че за произволен терм $\rho \in \text{PCF}_{\vv{a}\to\nat}$ имаме импликацията:
  \[(\forall \vv{n})[\ \rho\tau_1 \Downarrow_{\nat} \vv{n} \implies \rho\tau_2 \Downarrow_{\nat} \vv{n}\ ].\]
  Но понеже $\val{\tau_1} \sqsubseteq \val{\tau_2}$, това е лесно:
  \begin{align*}
    \rho\tau_1 \Downarrow_{\nat} \vv{n} & \implies \val{\rho\tau_1} = \val{\vv{n}} & \comment\text{\hyperref[th:pcf:soundness]{Теорема за коректност}}\\
                                            & \implies \val{\rho}(\val{\tau_1}) = \val{\vv{n}} \\% & \comment\text{\hyperref[lem:pcf:substitution]{Лема за замяната}}\\
                                            & \implies \val{\rho}(\val{\tau_2}) = \val{\vv{n}} & \comment\text{монотонност на }\val{\rho}\\
                                            & \implies \val{\rho\tau_2} = \val{\vv{n}}\\ % & \comment\text{\hyperref[lem:pcf:substitution]{Лема за замяната}}\\
                                            & \implies \rho\tau_2 \Downarrow_{\nat} \vv{n}. & \comment\text{\hyperref[th:pcf:adequacy]{Теорема за адекватност}}
  \end{align*}
\end{proof}

Можем да обобщим всичко, което сме направили до момента като формулираме следната теорема.

\begin{framed}
  \begin{theorem}\label{th:pcf:context:connection}
    За всички затворени термове $\tau_1$ и $\tau_2$ от тип $\vv{a}$ е изпълнено, че:
    \begin{enumerate}[(1)]
    \item
      \label{pcf:context:connection:operational}
      $(\forall \vv{v})[\tau_1 \Downarrow_{\vv{a}} \vv{v} \iff \tau_2 \Downarrow_{\vv{a}} \vv{v} ] \implies \tau_1 \cong_{ctx} \tau_2 : \vv{a}$;
    \item
      \label{pcf:context:connection:denotational}
      $\val{\tau_1} = \val{\tau_2} \implies \tau_1 \cong_{ctx} \tau_2 : \vv{a}$.
    \end{enumerate}
  \end{theorem}
\end{framed}

В частния случай на $\vv{a} = \nat$ имаме и обратните импликации на \Th{pcf:context:connection}.

\begin{framed}
    \begin{corollary}\label{cr:pcf:context:connection}
    За всички затворени термове $\tau_1$ и $\tau_2$ от тип $\nat$ е изпълнено, че:
    \begin{enumerate}[(1)]
    \item
      \label{cr:pcf:context:connection:operational}
      $(\forall \vv{n})[\tau_1 \opsem{}{nat} \vv{n} \iff \tau_2 \opsem{}{nat} \vv{n} ] \iff \tau_1 \cong_{ctx} \tau_2 : \nat$;
    \item
      \label{cr:pcf:context:connection:denotational}
      $\val{\tau_1} = \val{\tau_2} \iff \tau_1 \cong_{ctx} \tau_2 : \nat$.
    \end{enumerate}
  \end{corollary}
\end{framed}
\begin{proof}
  (1) представлява точно \ref{pr:pcf:context:extensionality:base} на \Prop{pcf:context:extensionality}.
  За обратната посока на (2), нека $\tau_1 \leq_{ctx} \tau_2 : \nat$. Трябва да докажем, че $\val{\tau_1} \sqsubseteq \val{\tau_2}$.
  Нека $\val{\tau_1} = n \neq \bot$, защото ако $\val{\tau_1} = \bot$, то е ясно, че $\val{\tau_1} \sqsubseteq \val{\tau_2}$.
  Имаме импликациите:
  \begin{align*}
    \val{\tau_1} = n & \implies \tau_1 \opsem{}{nat} \vv{n} & \comment\text{\hyperref[th:pcf:adequacy]{Теорема за адекватност}}\\
                     & \implies \tau_2 \opsem{}{nat} \vv{n} & \comment\text{\ref{pr:pcf:context:extensionality:base} на \Prop{pcf:context:extensionality}}\\
                     & \implies \val{\tau_2} = n. & \comment\text{\hyperref[th:pcf:soundness]{Теорема за коректност}}
  \end{align*} 
\end{proof}

Нека сега да видим, че нямаме обратните импликации на \Cor{pcf:context:connection} за по-високи типове от $\nat$ като разгледаме термовете $\Omega_{\vv{a}}$ и $\Omega'_{\vv{a}}$, които сме срещали и преди
и сме разглеждали техните свойства в \Problem{pcf:context:omega}.

\begin{framed}
\begin{proposition}\label{pr:context:op-right-left}
  $\Omega'_{\nat} \cong_{ctx} \Omega_{\nat\to\nat} : \arr{nat}{nat}$.
\end{proposition}  
\end{framed}
\marginpar{Да напомним, че
  \begin{align*}
    \Omega_{\vv{a}} &  \df \fix(\lamb{x}{a}{\vv{x}})\\
    \Omega'_{\vv{a}} & \df \lamb{x}{a}{\Omega_{\vv{a}}}.
  \end{align*}}
\begin{proof}
  % Първо да видим защо $\Omega'_{\nat} \leq_{ctx} \Omega_{\nat\to\nat} : \arr{nat}{nat}$.
  % Според \ref{pr:pcf:context:extensionality:step} на \Prop{pcf:context:extensionality}, трябва да докажем, че за
  % произволен затворен терм $\rho$ от тип $\nat$ е изпълнено, че
  % \[\Omega'_{\nat}\rho \leq_{ctx} \Omega_{\nat\to\nat}\rho : \nat,\]
  % което според \ref{pr:pcf:context:extensionality:base} на \Prop{pcf:context:extensionality} означава да докажем, че
  % \[(\forall \vv{n})[\ \Omega'_{\nat}\rho \opsem{}{nat} \vv{n} \implies \Omega_{\nat\to\nat} \rho \opsem{}{nat} \vv{n}\ ].\]
  % Да отбележим, че $\Omega'_{\nat} \rho \not\opsem{}{nat}$, защото $\Omega_{\nat}$ е затворен терм и $\Omega_{\nat}\subst{x}{\rho} \equiv \Omega_{\nat}$ и според правилата на операционната семантика имаме следното изчисление:
  % \begin{prooftree}
  %   \AxiomC{$\Omega'_{\nat}$ е стойност}
  %   \LeftLabel{\footnotesize{(val)}}
  %   \UnaryInfC{$\Omega'_{\nat} \opsem{0}{nat} \Omega'_{\nat}$}
  %   \AxiomC{\ref{pcf:omega:operational} на \Problem{pcf:context:omega}}
  %   \UnaryInfC{$\Omega_{\nat} \not\opsem{}{nat}$}
  %   \UnaryInfC{$\Omega_{\nat}\subst{x}{\rho} \not\opsem{}{nat}$}
  %   \RightLabel{\footnotesize{(cbn)}}    
  %   \BinaryInfC{$\Omega'_{\nat} \rho \not\opsem{}{nat}$}
  % \end{prooftree}
  % Тогава заключаваме, че за всеки затворен терм $\rho$ от тип $\nat$, е изпълнено
  % \[(\forall \vv{n})[\ \Omega'_{\nat}\rho \Downarrow_{\nat} \vv{n} \implies \Omega_{\nat\to\nat} \rho \Downarrow_{\nat} \vv{n}\ ],\]
  % защото лявата част на импликацията никога не е вярна.

  % Нека сега да видим защо $\Omega_{\nat\to\nat} \leq_{ctx} \Omega'_{\nat} : \nat \to \nat$. Разсъждаваме по сходен начин.
  % Ще докажем, че за произволен затворен терм $\rho$ от тип $\nat$ е изпълнено, че:
  % \[(\forall \vv{n})[\ \Omega_{\nat\to\nat}\rho \opsem{}{nat} \vv{n} \implies \Omega'_{\nat} \rho \opsem{}{nat} \vv{n}\ ].\]
  % Според правилата на операционната семантиката получаваме следното:
  % \begin{prooftree}
  %   \AxiomC{\ref{pcf:omega:operational} на \Problem{pcf:context:omega}}
  %   \UnaryInfC{$\Omega_{\nat\to\nat} \not\opsemGen{}{\nat\to\nat}$}
  %   \AxiomC{$...$}
  %   \RightLabel{\footnotesize{(cbn)}}
  %   \BinaryInfC{$\Omega_{\nat\to\nat}\rho \not\opsemGen{}{\nat\to\nat}$}
  % \end{prooftree}
  % Отново заключаваме, че за всеки затворен терм $\rho$ от тип $\nat$, е изпълнено
  % \[(\forall \vv{n})[\ \Omega_{\nat\to\nat}\rho \Downarrow_{\nat} \vv{n} \implies \Omega'_{\nat} \rho \Downarrow_{\nat} \vv{n}\ ],\]
  % защото лявата част на импликацията никога не е вярна.
  Понеже ние вече знаем, че $\val{\Omega'_{\nat}} = \val{\Omega_{\arr{nat}{nat}}}$, то
  според \ref{pcf:context:connection:denotational} на \Th{pcf:context:connection} следва, че
  $\Omega'_\nat \eqCtx \Omega_{\arr{nat}{nat}}$.
\end{proof}

Знаем, че $\Omega_{\nat\to\nat} \not\opsemGen{}{\nat\to\nat}$, но $\Omega'_{\nat} \opsemGen{}{\nat\to\nat} \Omega'_{\nat}$.
Това означава, че според \Prop{context:op-right-left}, термовете $\Omega_{\nat\to\nat}$ и $\Omega'_{\nat}$ ни дават пример кога нямаме обратната импликация в (1) на \Th{pcf:context:connection}.
% Обърнете внимание, че $\val{\Omega_{\nat\to\nat}} = \val{\Omega'_{\nat}}$.
Следователно, трябва да продължим да търсим термове $\tau_1$ и $\tau_2$ от някакъв тип $\vv{a}$, за които $\val{\tau_1} \neq \val{\tau_2}$
и $\tau_1 \cong_{ctx} \tau_2 : \vv{a}$.

\newpage

\begin{problem}
  \marginpar{Кеймбридж 2010 г.}
  Докажете, че е изпълнена импликацията:
  \[\tau_1 \cong_{ctx} \tau_2 : \nat\to\nat \implies \val{\tau_1} = \val{\tau_2}.\]  
\end{problem}
\begin{proof}
  За да проверим това е достатъчно да проверим, че ако $\tau_1 \leq_{ctx} \tau_2 : \nat\to\nat$, то $\val{\tau_1} \sqsubseteq \val{\tau_2}$.
  И така, да приемем, че имаме $\tau_1 \leq_{ctx} \tau_2 : \nat\to\nat$. Според \ref{pr:pcf:context:extensionality:step} на \Prop{pcf:context:extensionality}, това означава, че имаме
  за всеки затворен терм $\rho$ от тип $\nat$, $\tau_1 \rho \leq_{ctx} \tau_2 \rho : \nat$.
  Сега тук използваме \ref{pcf:context:connection:operational}, откъдето следва, че
  \begin{equation}
    \label{eq:context:1}
    \val{\tau_1\rho} \sqsubseteq \val{\tau_2\rho}.
  \end{equation}
  % Тук сега използваме \ref{pr:pcf:context:extensionality:base} на \Prop{pcf:context:extensionality} откъде получаваме, че
  % \begin{equation}
  %   \label{eq:context:1}
  %   (\forall \vv{k})[\tau_1 \rho \opsem{}{nat} \vv{k} \implies \tau_2 \rho \opsem{}{nat} \vv{k}].
  % \end{equation}
  \marginpar{Тук важното е това, че всеки елемент на $\Nat_\bot$ е определим с терм в езика \PCF.}
  Нека разгледаме произволен елемент $n \in \Nat_\bot$. 
  \begin{itemize}
  \item
    Ако $n \neq \bot$, то нека положим $\rho \df \vv{n}$. 
  \item
    Ако $n = \bot$, то нека положим $\rho \df \Omega_{\nat}$.
  \end{itemize}
  Ясно е, че сме избрали затворения терм $\rho$ от тип $\nat$ така, че $\val{\rho} = n$.
  Достатъчно е да проследим следните импликации
  \begin{align*}
    \val{\tau_1}(n) & = \val{\tau_1}(\val{\rho})\\
                    & = \val{\tau_1\rho}\\
                    & \sqsubseteq \val{\tau_2\rho} & \comment\text{от (\ref{eq:context:1})}\\
                    & = \val{\tau_2}(\val{\rho})\\
                    & = \val{\tau_2}(n).
  \end{align*}
  % \begin{align*}
  %   \val{\tau_1}(n) = k & \implies \val{\tau_1}(\val{\rho}) = k & \comment\val{\rho} = n\\
  %                       & \implies \val{\tau_1 \rho} = k \\
  %                       & \implies \tau_1 \rho \opsem{}{nat} \vv{k} & \comment\text{\hyperref[th:pcf:adequacy]{Теорема за адекватност}}\\
  %                       & \implies\tau_2 \rho \opsem{}{nat} \vv{k} & \comment\text{от (\ref{eq:context:1})}\\
  %                       & \implies \val{\tau_2 \rho} = k & \comment\text{\hyperref[th:pcf:soundness]{Теорема за коректност}}\\
  %                       & \implies \val{\tau_2}(\val{\rho}) = k\\
  %                       & \implies \val{\tau_2}(n) = k, & \comment\val{\rho} = n
  % \end{align*}
  за да се убедим, че $\val{\tau_1} \sqsubseteq \val{\tau_2}$.
\end{proof}

\begin{problem}
  \label{prob:context:chain:implication}
  Нека $\vv{a} = \nat \to \nat \to \cdots \to \nat$.
  Докажете, че 
  \[\tau_1 \cong_{ctx} \tau_2 : \vv{a} \implies \val{\tau_1} = \val{\tau_2}.\]  
\end{problem}

Това означава, че ако искаме да намерим затворени термове $\tau_1$ и $\tau_2$ от тип $\vv{a}$, за които $\val{\tau_1} \neq \val{\tau_2}$, то $\tau_1 \cong_{ctx} \tau_2 : \vv{a}$, то трябва да разгледаме по-сложни типове $\vv{a}$.

\begin{problem}
  Докажете или опровергайте твърденията:
  \begin{enumerate}[(1)]
  \item
    $\lamb{x}{nat}{\vv{x}} \cong_{ctx} \lamb{x}{nat}{\vv{x + 0}} : \arr{nat}{nat}$;
  \item
    $\lamb{x}{nat}{\vv{x}} \cong_{ctx} \lamb{x}{nat}{\vv{(x - 1) + 1}} : \arr{nat}{nat}$;
  \end{enumerate}

\end{problem}

\begin{problem}
  \marginpar{Задача в Кеймбридж 2020 г. \cite{cambridge-website}}
  Докажете или опровергайте твърдението, че за всеки два типа $\vv{a}$ и $\vv{b}$
  и всеки два затворени терма $\tau_1$ от тип $\arr{a}{b}$ и $\tau_2$ от тип $\arr{b}{a}$
  е изпълнено следното:  
  \[\texttt{fix}(\lamb{y}{b}{\tau_1(\tau_2(\vv{y}))}) \cong_{ctx} \tau_1(\texttt{fix}(\lamb{x}{a}{\tau_2(\tau_1(\vv{x}))})) : \vv{b}.\]  
\end{problem}
\begin{hint}
  Използвайте \Problem{domains:lfp:compositon}.
\end{hint}

\begin{problem}
  \marginpar{Задача в Кеймбридж 2015 г. \cite{cambridge-website}}
  % Докажете или опровергайте, че винаги можем да направим следния извод:
  % \begin{prooftree}
  %   \AxiomC{$\tau_1 \eqCtx \tau_2 : \arr{a}{b}$}
  %   \AxiomC{$\mu_1 \eqCtx \mu_2 : \arr{b}{c}$}
  %   \BinaryInfC{$\lamb{x}{a}{\mu_1(\tau_1 \vv{x})} \eqCtx \lamb{x}{a}{\mu_2(\tau_2 \vv{x})} : \arr{a}{c}$}
  % \end{prooftree}

  
  Нека $\tau_1$ и $\tau_2$ са затворени термове от тип $\arr{a}{b}$ и
  нека $\mu_1$ и $\mu_2$ са затворени термове от тип $\arr{b}{c}$, за които:
  \begin{align*}
    & \tau_1 \eqCtx \tau_2 : \arr{a}{b}\\
    & \mu_1 \eqCtx \mu_2 : \arr{b}{c}.
  \end{align*}
  Докажете или опровергайте, че тогава имаме еквивалентността:
  \begin{equation}
    \label{eq:pcf:context:2}
    \lamb{x}{a}{\mu_1(\tau_1 \vv{x})} \eqCtx \lamb{x}{a}{\mu_2(\tau_2 \vv{x})} : \arr{a}{c}.
  \end{equation}
\end{problem}
\begin{hint}
  За да докажем, че еквивалентността (\ref{eq:pcf:context:2}) е изпълнена, според \ref{pr:pcf:context:extensionality:step} на \Prop{pcf:context:extensionality} е достатъчно да докажем, че за произволен затворен терм $\rho$ от тип $\vv{a}$ е изпълнено, че:
  \[(\lamb{x}{a}{\mu_1(\tau_1 \vv{x})})\rho \eqCtx (\lamb{x}{a}{\mu_2(\tau_2 \vv{x})})\rho : \vv{c}.\]
  Сега, ще използваме равенствата за $i = 1,2$:
  \[\val{(\lamb{x}{a}{\mu_i(\tau_i \vv{x})})\rho}= \val{\mu_i}(\val{\tau_i}(\val{\rho})) = \val{\mu_i(\tau_i \rho)},\]
  откъдето според \ref{pcf:context:connection:denotational} на \Th{pcf:context:connection} следва, че
  \[(\lamb{x}{a}{\mu_i(\tau_i \vv{x})})\rho \eqCtx \mu_i(\tau_i \rho).\]
  Оттук следва, че е достатъчно да докажем, че
  \begin{equation}
    \label{eq:pcf:context:3}
    \mu_1(\tau_1 \rho) \eqCtx \mu_2(\tau_2 \rho) : \vv{c}.
  \end{equation}
  Но това е лесно да се види, защото, щом $\tau_1 \eqCtx \tau_2 : \arr{a}{b}$,
  то според \ref{pr:pcf:context:extensionality:step} на \Prop{pcf:context:extensionality} имаме, че
  $\tau_1\rho \eqCtx \tau_2\rho : \vv{b}$,
  Сега прилагаме \Problem{pcf:context:application} и оттам веднага получаваме (\ref{eq:pcf:context:3}).
\end{hint}


%%% Local Variables:
%%% mode: latex
%%% TeX-master: "../sep"
%%% End:
