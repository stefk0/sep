
\chapter{Езикът \PCF}\label{ch:pcf}

\marginpar{\PCF - programming language for computable functions. Plotkin’s language \PCF is often called the \emph{E. coli} of programming languages, the  subject  of  countless  studies  of  language  concept.}

\marginpar{Тази глава се оповава основно на \cite[Глава 19]{practical-foundations} и \cite{cambridge-den-sem}.}



\newcommand{\nat}{\vv{nat}}
\newcommand{\type}[2]{\vv{#1}:\vv{#2}}
\newcommand{\lamb}[3]{\lambda~\type{#1}{#2}~.~{#3}}


\section{Синтаксис}

\newcommand{\rename}[2]{\{\vv{#1}/{#2}\}}

\newcommand{\var}{\texttt{var}}
\newcommand{\fv}{\texttt{fv}}

\begin{itemize}
\item\index{тип}
  Типове
  \[\vv{a} ::= \vv{nat}\ |\ \vv{a} \to \vv{a}.\]
  Когато пишем 
  $\vv{a} \to \vv{b} \to \vv{c}$, то имаме предвид, че
  $\vv{a} \to (\vv{b} \to \vv{c})$.
\item\index{израз}
  \marginpar{Можем да си мислим за изразите като дървета.}
  Изрази
  \begin{align*}
    \tau ::=\ & \vv{n}\ |\ \vv{x}\ |\ \tau_1 + \tau_2\ |\ \tau_1 - \tau_2\ |\  \tau_1\ \vv{==}\ \tau_2\ |\ \ifelse{\tau_1}{\tau_2}{\tau_3}\ |\\
              & \tau_1\tau_2\ |\ \lamb{x}{a}{\tau_1}\ |\ \fix(\tau_1).
  \end{align*}
  Ще означаваме съвкупоността от всички изрази с $\mathcal{E}$, а съвкупността от всички променливи с $\mathcal{V}$.
  Да обърнем внимание, че не всички изрази са ,,смислени''. Например,
  $\lamb{x}{nat}{\vv{xx}}$ не е ясно какво означава.
  След малко ще дефинираме типизираща релация, с чиято помощ ще изхвърлим безслислените изрази.
  
  Понеже тук вече имаме свободни и свързани променливи, трябва да сме внимателни, когато правим замяна на един израз с друг. Да разгледаме функцията $\fv:\mathcal{E} \to \mathcal{P}(\mathcal{V})$, дефинирана със структурна индукция по построението на термовете по следния начин:
  
  \begin{itemize}
  \item
    $\fv(\vv{n}) = \emptyset$;
  \item
    $\fv(\vv{x}) = \{\vv{x}\}$;
  \item
    $\fv(\tau_1 + \tau_2) = \fv(\tau_1\ \vv{==}\ \tau_2) = \fv(\tau_1\tau_2) = \fv(\tau_1) \cup \fv(\tau_2)$;
  \item
    $\fv(\ifelse{\tau_1}{\tau_2}{\tau_3}) = \fv(\tau_1) \cup \fv(\tau_2) \cup \fv(\tau_3)$;
  \item
    $\fv(\lamb{x}{a}{\tau}) = \fv(\tau) \setminus \{x\}$.
  \item
    $\fv(\fix(\tau)) = \fv(\tau)$;
  \end{itemize}
  
  Ще казваме, че един израз $\tau$ е {\bf затворен}, ако $\fv(\tau) = \emptyset$.
  В противен случай, ще казваме, че изразът е {\bf отворен}.
\item
  \marginpar{Стойносттите понякога се наричат и канонични форми \cite{winskel}.}
  Ще казваме, че един израз е {\bf стойност}, ако той е затворен терм, съставен по следния начин:
  \[\vv{v} ::= \vv{n}\ |\ \lamb{x}{a}{\mu}.\]
  Интуицията тук е, че стойностите са затворени термове, в които не е възможно да се правят повече опростявания.
  Например, $\vv{5 + 6}$ не е стойност, защото може да се опрости до $\vv{11}$,
  но $\lamb{x}{nat}{\vv{5+6}}$ е стойност.
\end{itemize}

С $\tau\rename{x}{\rho}$ ще означаваме изразът получен от $\tau$, в който всяко \emph{свободно} срещане на променливата $\vv{x}$
е заменена с израза $\rho$. Можем да дадем формална дефиниция с индукция по построението на изразите:
\marginpar{Операцията $\rename{x}{\rho}$ заменя в дървото за израза $\tau$, всяко листо означено с $x$, което не участва в дърво с корен $\lambda x:a$, с дървото за израза $\rho$.}
\begin{itemize}
\item
  Ако $\tau \equiv \vv{n}$, то $\tau\rename{x}{\rho} \equiv \vv{n}$.
\item
  Ако $\tau \equiv \vv{x}$, то $\tau\rename{x}{\rho} \equiv \rho$.
\item
  Ако $\tau \equiv \vv{y}$, където $\vv{y} \not\equiv \vv{x}$, то $\tau\rename{x}{\rho} \equiv \vv{y}$.
\item
  Ако $\tau \equiv \tau_1 + \tau_2$, то $\tau\rename{x}{y} \equiv \tau_1\rename{x}{\rho} + \tau_2\rename{x}{\rho}$.
\item
  Ако $\tau \equiv \tau_1\ \vv{==}\ \tau_2$, то $\tau\rename{x}{\rho} \equiv \tau_1\rename{x}{\rho}\ \vv{==}\ \tau_2\rename{x}{\rho}$.
\item
  Ако $\tau \equiv \ifelse{\tau_1}{\tau_2}{\tau_3}$, то
  \[\tau\rename{x}{\rho} \equiv \ifelse{\tau_1\rename{x}{\rho}}{\tau_2\rename{x}{\rho}}{\tau_3\rename{x}{\rho}}.\]
\item
  Ако $\tau \equiv \lamb{x}{a}{\tau'}$, то $\tau\rename{x}{\rho} \equiv \tau$;
\item
  Ако $\tau \equiv \lamb{y}{a}{\tau'}$ и $\vv{y} \not\equiv \vv{x}$, то
  $\tau\rename{x}{\rho} \equiv \lamb{y}{a}{\tau'\rename{x}{\rho}}$.
\end{itemize}

Нека $\tau \equiv \lamb{x}{a}{\vv{x+y}}$. Обърнете внимание, че $\tau\rename{y}{\vv{x}} \equiv \lamb{x}{a}{\vv{x+x}}$,
т.е. тук получаваме израз, който ,,смислово'' е доста различен от първоначалния израз $\tau$.
Проблемът се състои в това, че при замяната на $\vv{x}$ с $\rho$, някоя свободна променлива на $\rho$ може да попадне под обхвата на някоя свързана
променлива на $\tau$.

\index{$\alpha$-еквивалентност}
\marginpar{Това е подходът на Хаскел Къри за дефиниране на замяна на променлива с израз \cite[стр. 578]{barendregt-handbook}.}
Сега ще дефинираме бинарна релация между изрази, която ще наричаме $\alpha$-ек\-ви\-ва\-лент\-ност.
Интуитивно, всеки два $\alpha$-еквивалентни израза трябва да бъдат смислово неотличими.
Това е най-малката релация между изрази, която ще означаваме с $\alphaEq$, за която са изпълнени свойствата:
\marginpar{
  В тази дефиниция единствено последният случай е интересен.
  Например,
  \[(\lamb{x}{a}{\vv{x+z}})\vv{x} \equiv_\alpha (\lamb{y}{a}{\vv{y+z}})\vv{x},\]
  но \[(\lamb{x}{a}{\vv{x+y}})\vv{x} \not\equiv_\alpha (\lamb{y}{a}{\vv{x+y}})\vv{x}.\]}
\begin{itemize}
\item
  $\vv{x} \alphaEq \vv{x}$;
\item
  $\vv{n} \alphaEq \vv{n}$;
\item
  ако $\tau_1 \alphaEq \rho_1$ и $\tau_2 \equiv_\alpha \rho_2$, то имаме, че
  \begin{align*}
    & \tau_1 + \tau_2 \alphaEq \rho_1 + \rho_2,\\
    & \tau_1\ \vv{==}\ \tau_2 \alphaEq \rho_1\ \vv{==}\ \rho_2,\\
    & \tau_1 \tau_2 \alphaEq \rho_1 \rho_2;
  \end{align*}
\item
  ако $\tau_1 \alphaEq \rho_1$, $\tau_2 \equiv_\alpha \rho_2$ и $\tau_3 \equiv_\alpha \rho_3$, то
  \[\ifelse{\tau_1}{\tau_2}{\tau_3} \alphaEq \ifelse{\rho_1}{\rho_2}{\rho_3};\]
\item
  ако $\tau \alphaEq \rho$, то $\fix(\tau) \equiv_\alpha \fix(\rho)$;
\item  
  ако $\tau\rename{x}{\vv{z}} \alphaEq \rho\rename{y}{\vv{z}}$, където $\vv{z} \not\in \var(\tau) \cup \var(\rho)$, то
  \[\lamb{x}{a}{\tau} \alphaEq \lamb{y}{a}{\rho}.\]
\end{itemize}

\marginpar{
Например,
\[\lamb{x}{a}{\vv{x+y}} \alphaEq \lamb{z}{a}{\vv{z+y}},\]
но
\[\lamb{x}{a}{\vv{x+y}} \not\alphaEq \lamb{y}{a}{\vv{y+y}},\]
защото
\[(\vv{x+y})\rename{x}{\vv{u}} \not\alphaEq (\vv{y+y})\rename{y}{\vv{u}}.\]}
Също така,
\[\lamb{x}{nat}{\vv{x + }\fix(\lamb{x}{nat}{\vv{x}})} \alphaEq \lamb{y}{nat}{\vv{y + }\fix(\lamb{z}{nat}{\vv{z}})}.\]

\index{терм}
\begin{framed}
\begin{definition}
  PCF терм е клас на еквивалентност от PCF изрази относно релацията $\alpha$-еквивалентност.
\end{definition}
\end{framed}

\marginpar{Това да се премести в нова глава за ламбда смятане.}
Де Бройн има по-програмистки подход за дефинирането на PCF термовете.
Всеки PCF терм има единствено представяне като израз, в който свързаните променливи са заменени с индекси.

{\bf Пример ......}


Сега искаме $\tau\subst{x}{\rho}$ да означава изразът получен от израза $\tau$, в който всяко \emph{свободно} срещане на променливата $\vv{x}$
е заменена с израза $\rho$. Тук имаме потенциален проблем. Искаме да направим тази замяна по такъв начин, че свободни променливи на $\rho$
да не попаднат под обхвата на свързани променливи от $\tau$. За да направим това, операцията $\subst{x}{\rho}$
трябва да работи не върху отделни изрази, а върху термове.
\index{субституция}
Можем да дадем формална дефиниция с индукция по построението на термовете:
\begin{itemize}
\item
  Ако $\tau \equiv \vv{n}$, то $\tau\subst{x}{\rho} \equiv \vv{n}$.
\item
  Ако $\tau \equiv \vv{x}$, то $\tau\subst{x}{\rho} \equiv \rho$.
\item
  Ако $\tau \equiv \vv{y}$ и $\vv{y} \not\equiv \vv{x}$, то $\tau\subst{x}{\rho} \equiv \vv{y}$.
\item
  Ако $\tau \equiv \tau_1 + \tau_2$, то
  \[\tau\subst{x}{y} \equiv \tau_1\subst{x}{\rho} + \tau_2\subst{x}{\rho}.\]
\item
  Ако $\tau \equiv \tau_1\ \vv{==}\ \tau_2$, то
  \[\tau\subst{x}{\rho} \equiv \tau_1\subst{x}{\rho}\ \vv{==}\ \tau_2\subst{x}{\rho}.\]
\item
  Ако $\tau \equiv \ifelse{\tau_1}{\tau_2}{\tau_3}$, то
  \[\tau\subst{x}{\rho} \equiv \ifelse{\tau_1\subst{x}{\rho}}{\tau_2\subst{x}{\rho}}{\tau_3\subst{x}{\rho}}.\]
\item
  Ако $\tau \equiv \lamb{y}{a}{\tau'}$, то
  \[\tau\subst{x}{\rho} \equiv \lamb{z}{a}{(\tau'\subst{y}{\vv{z}}\subst{x}{\rho})},\]
  където $\vv{z} \not\in \fv(\tau') \cup \fv(\rho) \cup \{\vv{x}\}$.
\end{itemize}

В тази дефиниция отново единствено последният случай е интересен.
Обърнете внимание, че според него заместването на $\vv{x}$ с $\rho$ дава като резултат безкрайно много
изрази, всички от които са $\alpha$-еквивалентни, т.е. ако работим на ниво термове операцията е добре дефинирана.

Това означава, че няма значение дали ще говорим за $\lamb{x}{a}{\vv{x+y}}$ или за $\lamb{y}{a}{\vv{y+z}}$.
Тези два израза описват един и същи терм и тези два израза са $\alpha$-еквивалентни.


%%% Local Variables:
%%% mode: latex
%%% TeX-master: "../sep"
%%% End:


\section{Добре типизирани термове}\index{тип}

Типовите контексти представляват крайни редици от двойки от вида $\type{x}{a}$, т.е.
\[\Gamma ::= \emptyset\ |\ \Gamma,\type{x}{a}.\]
Например,
\[\Gamma = \vv{x}:\vv{nat},\ \vv{y}:\vv{nat} \to \vv{nat},\ \vv{z}:\vv{nat}.\]
\marginpar{Обикновено ще означаваме типовите контексти с главните гръцки $\Gamma, \Delta, \dots$.}
На един типов контекст $\Gamma$ може да се гледа и като на \emph{крайна} функция приемаща като аргумент променлива и връщаща тип.
Например, може понякога да пишем $\Gamma(\vv{x}) = \vv{nat}$, когато искаме да кажем, че променливата $\vv{x}$ има тип $\vv{nat}$ в типовия контекст $\Gamma$.
\marginpar{$\Gamma$ също се нарича и type environment.}
Ние искаме да работим само с коректно типизирани термове.
Например, не е ясно какво означава терма $\vv{1 + }\lamb{x}{nat}{\vv{x}}$,
защото трябва да съберем число и функция - два обекта от различен тип.
Ако в изразите имаме свободни променливи, то дали един израз е коректно типизиран ще зависи от типовия контекст.
Например, термът $\lamb{x}{nat}{\vv{z+x}}$ е добре типизиран в контекста $\Gamma = \vv{z}:\vv{nat}$, но не е добре типизиран в контекста
$\Delta = \vv{z}:\vv{nat}\to\vv{nat}$.
Сега ще дефинираме релация $\Gamma \vdash \tau : \vv{a}$, която ще ни казва, че термът $\tau$, относно типовият контекст $\Gamma$,
е добре типизиран и има тип $\vv{a}$.

\marginpar{Обърнете внимание, че можем да правим доказателства свойства на типизиращата релация с индукция по простроението на термовете.}

\begin{figure}[H]
  \centering
\begin{prooftree}
  \AxiomC{}
  \RightLabel{\scriptsize{(const)}}
  \UnaryInfC{$\Gamma \vdash \vv{n} : \vv{nat}$}
\end{prooftree}

\begin{prooftree}
  \AxiomC{$\vv{x} \in \texttt{dom}(\Gamma)$}
  \AxiomC{$\Gamma(\vv{x}) = \vv{a}$}
  \RightLabel{\scriptsize{(var)}}
  \BinaryInfC{$\Gamma \vdash \vv{x} : \vv{a}$}
\end{prooftree}

\begin{prooftree}
  \AxiomC{$\Gamma \vdash \tau_1:\vv{nat}$}
  \AxiomC{$\Gamma \vdash \tau_2:\vv{nat}$}
  \RightLabel{\scriptsize{(plus)}}
  \BinaryInfC{$\Gamma \vdash \tau_1 + \tau_2 : \vv{nat}$}
\end{prooftree}

\begin{prooftree}
  \AxiomC{$\Gamma \vdash \tau_1:\vv{nat}$}
  \AxiomC{$\Gamma \vdash \tau_2:\vv{nat}$}
  \RightLabel{\scriptsize{(eq)}}
  \BinaryInfC{$\Gamma \vdash \tau_1\ \vv{==}\ \tau_2 : \vv{nat}$}
\end{prooftree}

\begin{prooftree}
  \AxiomC{$\Gamma \vdash \tau_1:\vv{nat}$}
  \AxiomC{$\Gamma \vdash \tau_2:\vv{a}$}
  \AxiomC{$\Gamma \vdash \tau_3:\vv{a}$}
  \RightLabel{\scriptsize{(if)}}
  \TrinaryInfC{$\Gamma \vdash \ifelse{\tau_1}{\tau_2}{\tau_3} : \vv{a}$}
\end{prooftree}

\begin{prooftree}
  \AxiomC{$\Gamma \vdash \tau_1:\vv{a}\to\vv{b}$}
  \AxiomC{$\Gamma \vdash \tau_2:\vv{a}$}
  \RightLabel{\scriptsize{(app)}}
  \BinaryInfC{$\Gamma \vdash \tau_1\tau_2 : \vv{b}$}
\end{prooftree}

\begin{prooftree}
  \AxiomC{$\Gamma \vdash \tau:\vv{a}\to\vv{a}$}
  \RightLabel{\scriptsize{(fix)}}
  \UnaryInfC{$\Gamma \vdash \fix(\tau) : \vv{a}$}
\end{prooftree}

\begin{prooftree}
  \AxiomC{$\vv{x} \not\in\vv{dom}(\Gamma)$}
  \AxiomC{$\Gamma, \type{x}{a} \vdash \tau:\vv{b}$}
  \RightLabel{\scriptsize{(lambda)}}
  \BinaryInfC{$\Gamma \vdash \lambda \type{x}{a}\ .\ \tau : \vv{a} \to \vv{b}$}
\end{prooftree}
  
  \caption{Релация за типизиране на термовете от езика \PCF}
  \label{fig:pcf:types:relation}
\end{figure}


Ако имаме затворен израз $\tau$, то ще пишем $\tau : \vv{a}$ вместо $\emptyset \vdash \tau : \vv{a}$.
Да положим
\[\text{PCF}_{\vv{a}} \df \{\tau \text{ е затворен терм}\mid \emptyset \vdash \tau : \vv{a}\}.\]

\begin{example}
  ????
\end{example}



\begin{proposition}
  Ако $\Gamma \vdash \tau : \vv{a}$, то $\fv(\tau) \subseteq \vv{dom}(\Gamma)$.
\end{proposition}

\begin{proposition}
  Ако $\Gamma \vdash \tau : \vv{a}$ и $\Gamma \vdash \tau : \vv{b}$, то $\vv{a} = \vv{b}$.
\end{proposition}

\begin{corollary}
  Всеки затворен терм има най-много един тип.
\end{corollary}

\begin{problem}
  \marginpar{\cite[стр. 104]{types-programming-languages}}
  Докажете или опровергайте дали е възможно да съществува типов контекст $\Gamma$ и тип $\vv{a}$, такива че
  $\Gamma \vdash \type{xx}{a}$.
\end{problem}

%%% Local Variables:
%%% mode: latex
%%% TeX-master: "../sep"
%%% End:


\section{Операционна семантика}\label{pcf:sect:operational-cbn}
\index{операционна семантика}

За затворен терм $\tau : \vv{a}$ и стойност $\vv{v} : \vv{a}$, дефинираме релацията $\tau \opsem{\ell}{a} \vv{v}$ по
следния начин.
\marginpar{Да напомним, че с $\vv{v}$ означаваме стойности, които биват или константи или затворени термове от вида $\lamb{x}{a}{\mu}$.

  Обърнете внимание, че заради случите (fix) и (cbn) не можем да доказваме свойства на операционната семантика с индукция по построението на термовете. Можем да направим това с индукция по броя на стъпки в изчислението.}

\begin{figure}[H]
  \centering
  
\begin{prooftree}
  \AxiomC{$\type{v}{a}$}
  \RightLabel{\scriptsize{(val)}}
  \UnaryInfC{$\vv{v} \opsem{0}{a} \vv{v}$}
\end{prooftree}

\begin{prooftree}
  \AxiomC{$\tau_1 \opsem{\ell_1}{nat} \vv{n}_1$}
  \AxiomC{$\tau_2 \opsem{\ell_2}{nat} \vv{n}_2$}
  \AxiomC{$n = \eq(n_1,n_2)$}
  \RightLabel{\scriptsize{(eq)}}
  \TrinaryInfC{$\tau_1\ \vv{==}\ \tau_2 \opsem{\ell_1+\ell_2+1}{nat} \vv{n}$}
\end{prooftree}

\begin{prooftree}
  \AxiomC{$\tau_1 \opsem{\ell_1}{nat} \vv{n}_1$}
  \AxiomC{$\tau_2 \opsem{\ell_2}{nat} \vv{n}_2$}
  \AxiomC{$n = \plus(n_1,n_2)$}
  \RightLabel{\scriptsize{(plus)}}
  \TrinaryInfC{$\tau_1\ \vv{+}\ \tau_2 \opsem{\ell_1+\ell_2+1}{nat} \vv{n}$}
\end{prooftree}

\begin{prooftree}
  \AxiomC{$\tau_1 \opsem{\ell_1}{nat} \vv{n}_1$}
  \AxiomC{$\tau_2 \opsem{\ell_2}{nat} \vv{n}_2$}
  \AxiomC{$n = \minus(n_1,n_2)$}
  \RightLabel{\scriptsize{(minus)}}
  \TrinaryInfC{$\tau_1\ \vv{-}\ \tau_2 \opsem{\ell_1+\ell_2+1}{nat} \vv{n}$}
\end{prooftree}

\begin{prooftree}
  \AxiomC{$\tau_1 \opsem{\ell_1}{nat} \vv{0}$}
  \AxiomC{$\tau_3 \opsem{\ell_2}{a} \vv{v}$}
  \RightLabel{\scriptsize{(if$_0$)}}
  \BinaryInfC{$\ifelse{\tau_1}{\tau_2}{\tau_3} \opsem{\ell_1+\ell_2+1}{a} \vv{v}$}
\end{prooftree}

\begin{prooftree}
  \AxiomC{$\tau_1 \opsem{\ell_1}{nat} \vv{n}$}
  \AxiomC{$\tau_2 \opsem{\ell_2}{a} \vv{v}$}
  \AxiomC{$\vv{n} \not\equiv \vv{0}$}
  \RightLabel{\scriptsize{(if$^+$)}}
  \TrinaryInfC{$\ifelse{\tau_1}{\tau_2}{\tau_3} \opsem{\ell_1+\ell_2+1}{a} \vv{v}$}
\end{prooftree}

\begin{prooftree}
  \AxiomC{$\tau_1 \opsemGen{\ell_1}{\vv{a}\to\vv{b}} \lamb{x}{a}{\tau'_1}$}
  \AxiomC{$\tau'_1\subst{x}{\tau_2} \opsem{\ell_2}{b} \vv{v}$}
  \RightLabel{\scriptsize{(cbn)}}
  \BinaryInfC{$\tau_1 \tau_2 \opsem{\ell_1+\ell_2+1}{b} \vv{v} $}
\end{prooftree}

\begin{prooftree}
  \AxiomC{$\tau\ \fix(\tau) \opsem{\ell}{a} \vv{v}$}
  \RightLabel{\scriptsize{(fix)}}
  \UnaryInfC{$\fix(\tau) \opsem{\ell+1}{a} \vv{v} $}
\end{prooftree}
\caption{Правила на операционната семантика за езика \PCF}
\end{figure}



\begin{itemize}
\item 
  Ще пишем $\tau \opsem{}{a} \vv{v}$, ако съществува $\ell$, за което $\tau \opsem{\ell}{a} \vv{v}$.  
\item
  Също така, ще пишем $\tau \not\opsem{}{a}$, ако не съществува стойност $\vv{v}$, за която $\tau \opsem{}{a} \vv{v}$.  
\end{itemize}

\begin{lemma}
  За произволен затворен терм $\tau$ и стойности $\vv{v}$ и $\vv{u}$,
  \[\tau \opsem{}{a} \vv{v}\ \&\ \tau \opsem{}{a} \vv{u}\ \implies\ \vv{v} \equiv \vv{u}.\]
\end{lemma}







%%% Local Variables:
%%% mode: latex
%%% TeX-master: "../sep"
%%% End:

\newpage
\begin{example}
  Нека $\vv{a} = \vv{nat} \to (\vv{nat} \to \vv{nat})$ и 
  \[\tau \equiv \fix\vv{(}\underbrace{\lamb{f}{a}{\overbrace{\lamb{x}{nat}{\lamb{y}{nat}{\ifelse{\vv{y == 0}}{\vv{0}}{\vv{x + (f x (y-1))} }}}}^{\tau'_0}}}_{\tau_0}\vv{)}.\]
  Лесно се вижда, че
  \[\emptyset \vdash \tau:\vv{nat} \to (\vv{nat} \to \vv{nat}).\]

  Също така за всяко $n$ и $k$, ако $m = n * k$, то
  \[\tau\ \vv{n}\ \vv{k} \Downarrow_{\vv{nat}} \vv{m}.\]
  Нека сега
  \[\rho \equiv \lamb{g}{a}{\fix(\lambda \vv{f} : \vv{nat}\to\vv{nat}\ .\ \lamb{x}{nat}{\ifelse{\vv{x == 0}}{\vv{1}}{\vv{g x f(x-1)}}}}).\]
  Лесно се вижда, че
  \[\emptyset \vdash \rho : \vv{a} \to (\vv{nat} \to \vv{nat}).\]
  Също така, за всяко $n$, ако $k = n!$, то
  \[ \rho\ \tau\ \vv{n} \opsem{}{nat} \vv{k}.\]
\end{example}

Нека да положим
\[\Gamma \df \vv{f} : \vv{a}, \vv{x} : \vv{nat}, \vv{y} : \vv{nat}.\]


\begin{landscape}
  \begin{framed}
  \begin{figure}[H]
    \centering
    % \begin{subfigure}[b]{1\textwidth}
      \begin{prooftree}
        \AxiomC{$\Gamma(\vv{y}) = \vv{nat}$}
        \UnaryInfC{$\Gamma \vdash \vv{y} : \vv{nat}$}
        \AxiomC{}
        \UnaryInfC{$\Gamma \vdash \vv{0} : \vv{nat}$}
        \BinaryInfC{$\Gamma \vdash \vv{y==0} : \vv{nat}$}
        \AxiomC{}
        \UnaryInfC{$\Gamma \vdash \vv{0} : \vv{nat}$}
        \AxiomC{$\Gamma(\vv{x}) = \vv{nat}$}
        \UnaryInfC{$\Gamma \vdash \vv{x}:\vv{nat}$}
        \AxiomC{$\Gamma(\vv{f}) = \vv{a}$}
        \UnaryInfC{$\Gamma \vdash \vv{f} : \vv{a}$}
        \AxiomC{$\Gamma(\vv{x}) = \vv{nat}$}
        \UnaryInfC{$\Gamma \vdash \vv{x} : \vv{nat}$}
        \BinaryInfC{$\Gamma \vdash \vv{f x} : \vv{nat} \to \vv{nat}$}
        \AxiomC{$\Gamma(\vv{y}) = \vv{nat}$}
        \UnaryInfC{$\Gamma \vdash \vv{y} : \vv{nat}$}
        \AxiomC{}
        \UnaryInfC{$\Gamma \vdash \vv{1} : \vv{nat}$}
        \BinaryInfC{$\Gamma \vdash \vv{y-1} : \vv{nat}$}
        \BinaryInfC{$\Gamma \vdash \vv{f x (y-1)} : \vv{nat}$}
        \BinaryInfC{$\Gamma \vdash \vv{x + f x (y-1)} : \vv{nat}$}
        \TrinaryInfC{$\vv{f} : \vv{a}, \vv{x} : \vv{nat}, \vv{y} : \vv{nat} \vdash \ifelse{\vv{y == 0}}{\vv{0}}{\vv{x + (f x (y-1))}} : \vv{nat}$}
        \UnaryInfC{$\vv{f} : \vv{a}, \vv{x} : \vv{nat} \vdash \lamb{y}{nat}{\ifelse{\vv{y == 0}}{\vv{0}}{\vv{x + (f x (y-1))}}} : \vv{nat} \to \vv{nat}$}
        \UnaryInfC{$\vv{f} : \vv{a} \vdash \tau'_0 : \vv{a}$}
        \UnaryInfC{$\emptyset \vdash \tau_0 : \vv{a} \to \vv{a}$}
        \UnaryInfC{$\emptyset \vdash \fix(\tau_0):\vv{a}$}
      \end{prooftree}
      \caption{Формален извод според правилата на типизиращата релацията, който показва, че $\emptyset \vdash \tau : \vv{a}$.}
    % \end{subfigure}
  \end{figure}
\end{framed}

След като видяхме, че $\tau$ е терм от тип $\vv{a}$, нека да видим
колко стъпки ще са ни нужни, според правилата на операционната семантика, за да проверим, че
$\tau\ \vv{3}\ \vv{2} \opsem{}{nat} \vv{6}$.

Първо да обърнем внимание, че $\tau_0$ е стойност, т.е. $\tau_0 \opsemGen{0}{\vv{a}\to\vv{a}} \tau_0$.
Макар и $\tau'_0$ да има свободна променлива $\vv{f}$, понеже термът $\fix(\tau_0)$ е затворен с тип $\vv{a}$, то термът
\[\tau'_0\subst{f}{\fix(\tau_0)} \equiv \lamb{x}{nat}{\lamb{y}{nat}{\ifelse{\vv{y == 0}}{\vv{0}}{\vv{ x + (}\fix(\tau_0)\vv{ x (y-1))}}}}\]
е стойност и следователно,
\[\tau'_0\subst{f}{\fix(\tau_0)} \opsem{0}{a} \tau'_0\subst{f}{\fix(\tau_0)}.\]

\begin{framed}
  \begin{figure}[H]
    \begin{prooftree}
      \AxiomC{$\tau_0$ е стойност}
      \UnaryInfC{$\tau_0 \opsemGen{0}{\vv{a}\to\vv{a}} \lamb{f}{a}{\tau'_0}$}
      \AxiomC{$\tau'_0\subst{f}{\fix(\tau_0)}$ е стойност}
      \UnaryInfC{$\tau'_0\subst{f}{\fix(\tau_0)} \opsem{0}{a}  \lamb{x}{nat}{\tau_2}$}
      \BinaryInfC{$\tau_0\fix(\tau_0) \opsem{1}{a} \lamb{x}{nat}{\tau_2}$}
      \UnaryInfC{$\fix(\tau_0) \opsem{2}{a} \lamb{x}{nat}{\tau_2}$}
      \AxiomC{$\tau_2\substConst{x}{3}$ е стойност}
      \UnaryInfC{$\tau_2\substConst{x}{3} \opsemGen{0}{\vv{nat}\to \vv{nat}} \lamb{y}{nat}{\tau_1}$}
      \BinaryInfC{$\fix(\tau_0)\ \vv{3} \opsemGen{3}{\vv{nat}\to\vv{nat}} \lamb{y}{nat}{\tau_1}$}
      \AxiomC{(продължава по-долу)}
      \UnaryInfC{$\tau_1[\vv{y}/\vv{2}] \opsem{t}{nat} \vv{6}$}
      \RightLabel{(app)}
      \BinaryInfC{$\fix(\tau_0)\ \vv{3}\ \vv{2} \opsem{t+4}{nat} \vv{6}$}
    \end{prooftree}
    \caption{Първа част от изчислението на $\tau\ \vv{3}\ \vv{2}$ според правилата на операционната семантика.}
  \end{figure}
\end{framed}

Нека за улеснение да положим
\begin{align*}
  \tau_2 & \equiv \lamb{y}{nat}{\ifelse{\vv{y == 0}}{\vv{0}}{\vv{ x + (}\fix(\tau_0)\vv{ x (y-1))}}}\\
  \tau_1 & \equiv \ifelse{\vv{y == 0}}{\vv{0}}{\vv{ 3 + (}\fix(\tau_0)\vv{ 3 (y-1))}}.
\end{align*}

Термът $\tau_2$ не е стойност, защото има свободна променлива $\vv{x}$, но вече термът $\tau_2\subst{x}{3}$ е стойност. Лесно се съобразява, че
\begin{align*}
  & \tau'_0\subst{f}{\fix(\tau_0)} \equiv \lamb{x}{nat}{\tau_2}\\
  & \tau_2 \substConst{x}{3} \equiv \lamb{y}{nat}{\tau_1}.
\end{align*}



% Оттук съобразяваме, че
%   \[\tau_2 \equiv \lamb{y}{nat}{\ifelse{\vv{y == 0}}{\vv{0}}{\vv{ x + (}\fix(\tau_0)\vv{ x (y-1))}}}.\]
%   Това означава, че
%   \[\tau_1 \equiv \ifelse{\vv{y == 0}}{\vv{0}}{\vv{ 3 + (}\fix(\tau_0)\vv{ 3 (y-1))}}\]
%   и
%   \[\tau_1\substConst{y}{2} \equiv \ifelse{\vv{2 == 0}}{\vv{0}}{\vv{ 3 + (}\fix(\tau_0)\vv{ 3 (2-1))}}\]
  Сега продължаваме десния клон на изчислението:
  \begin{framed}
    \begin{figure}[H]
      \begin{prooftree}
        \AxiomC{$\vv{2}$ е стойност}
        \UnaryInfC{$\vv{2} \opsem{0}{nat} \vv{2}$}
        \AxiomC{$\vv{0}$ е стойност}
        \UnaryInfC{$\vv{0} \opsem{0}{nat} \vv{0}$}
        \BinaryInfC{$\vv{2 == 0} \opsem{1}{nat} \vv{0}$}
        \AxiomC{$\vv{3}$ е стойност}
        \UnaryInfC{$\vv{3} \opsem{0}{nat} \vv{3}$}
        \AxiomC{(Повтаря се както по-горе)}
        \UnaryInfC{$\fix(\tau_0)\ \vv{3} \opsemGen{3}{\vv{nat}\to\vv{nat}} \lamb{y}{nat}{\tau_1}$}
        \AxiomC{(продължава по-долу)}
        \UnaryInfC{$\tau_1\substConst{y}{2-1} \opsem{k}{nat} \vv{3}$}
        \BinaryInfC{$\fix(\tau_0)\vv{ 3 (2-1) }\opsem{k+4}{nat} \vv{3}$}
        \BinaryInfC{$\vv{ 3 + (}\fix(\tau_0)\vv{ 3 (2-1))} \opsem{k+5}{nat} \vv{6}$}
        \BinaryInfC{$\underbrace{\ifelse{\vv{2 == 0}}{\vv{0}}{\vv{ 3 + (}\fix(\tau_0)\vv{ 3 (2-1))}}}_{\tau_1\substConst{y}{2}} \opsem{k+6}{nat} \vv{6}$}
      \end{prooftree}
      \caption{Втора част от изчислението, която показва, че $\tau_1\substConst{y}{2} \opsem{k+6}{nat} \vv{6}$.}
    \end{figure}
  \end{framed}
  Отново продължаваме с десния клон на изчислението, като тук вече имаме, че:
  \[\tau_1\substConst{y}{2-1} \equiv \ifelse{\vv{2-1 == 0}}{\vv{0}}{\vv{ 3 + (}\fix(\tau_0)\vv{ 3 (2-1-1))}}\]

  \begin{framed}
    \begin{figure}[H]
      \begin{prooftree}
        \AxiomC{$\vv{2}$ е стойност}
        \UnaryInfC{$\vv{2} \opsem{0}{nat} \vv{2}$}
        \AxiomC{$\vv{1}$ е стойност}
        \UnaryInfC{$\vv{1} \opsem{0}{nat} \vv{1}$}
        \BinaryInfC{$\vv{2-1} \opsem{1}{nat} \vv{1}$}
        \AxiomC{$\vv{0}$ е стойност}
        \UnaryInfC{$\vv{0} \opsem{0}{nat} \vv{0}$}
        \BinaryInfC{$\vv{2-1 == 0} \opsem{2}{nat} \vv{0}$}
        \AxiomC{$\vv{3}$ е стойност}
        \UnaryInfC{$\vv{3} \opsem{0}{nat} \vv{3}$}
        \AxiomC{(Повтаря се както по-горе)}
        \UnaryInfC{$\fix(\tau_0)\ \vv{3} \opsemGen{3}{\vv{nat}\to\vv{nat}} \lamb{y}{nat}{\tau_1}$}
        % \AxiomC{(продължава в (\ref{fig:operational-cbn-example:last-part}))}
        \AxiomC{(продължава по-долу)}
        \UnaryInfC{$\tau_1\substConst{y}{2-1-1} \opsem{\ell}{nat} \vv{0}$}
        \BinaryInfC{$\fix(\tau_0)\vv{ 3 (2-1-1) }\opsem{\ell+4}{nat} \vv{0}$}
        \BinaryInfC{$\vv{ 3 + (}\fix(\tau_0)\vv{ 3 (2-1-1))} \opsem{\ell+5}{nat} \vv{3}$}
        \BinaryInfC{$\underbrace{\ifelse{\vv{2-1 == 0}}{\vv{0}}{\vv{ 3 + (}\fix(\tau_0)\vv{ 3 (2-1-1))}}}_{\tau_1\substConst{y}{2-1}} \opsem{\ell+6}{nat} \vv{3}$}
      \end{prooftree}
      \caption{Трета част от изчислението, която показва, че $\tau_1\substConst{y}{2-1} \opsem{\ell+6}{nat} \vv{3}$ }
    \end{figure}
  \end{framed}
  Сега вече имаме, че
  \[\tau_1\substConst{y}{2-1-1} \equiv \ifelse{\vv{2-1-1 == 0}}{\vv{0}}{\vv{ 3 + (}\fix(\tau_0)\vv{ 3 (2-1-1-1))}}.\]

  Завършваме изчислението така:
  \begin{framed}
    \begin{figure}[H]
      \label{fig:operational-cbn-example:last-part}
      \begin{prooftree}
        \AxiomC{$\vv{2}$ е стойност}
        \UnaryInfC{$\vv{2} \opsem{0}{nat} \vv{2}$}
        \AxiomC{$\vv{1}$ е стойност}
        \UnaryInfC{$\vv{1} \opsem{0}{nat} \vv{1}$}
        \BinaryInfC{$\vv{2-1} \opsem{1}{nat} \vv{1} $}
        \AxiomC{$\vv{1}$ е стойност}
        \UnaryInfC{$\vv{1} \opsem{0}{nat} \vv{1}$}
        \BinaryInfC{$\vv{2-1-1} \opsem{2}{nat} \vv{0}$}
        \AxiomC{$\vv{0}$ е стойност}
        \UnaryInfC{$\vv{0} \opsem{0}{nat} \vv{0}$}
        \BinaryInfC{$\vv{2-1-1 == 0} \opsem{3}{nat} \vv{1}$}
        \AxiomC{$\vv{0}$ е стойност}
        \UnaryInfC{$\vv{0} \opsem{0}{nat} \vv{0}$}
        \BinaryInfC{$\underbrace{\ifelse{\vv{2-1-1 == 0}}{\vv{0}}{\vv{ 3 + (}\fix(\tau_0)\vv{ 3 (2-1-1-1))}}}_{\tau_1\substConst{y}{2-1-1}} \opsem{4}{nat} \vv{0}$}
      \end{prooftree}
      \caption{Последна част от изчислението, което показва, че $\tau_1\substConst{y}{2-1-1} \opsem{4}{nat} \vv{0}$.}
    \end{figure}
  \end{framed}  
\end{landscape}


Можем да напишем директно горния пример и на хаскел:
\begin{haskellcode}
ghci> fix f = f (fix f)
ghci> times = fix(\f x y -> if y == 0 then 0 else x + (f x (y-1)))
ghci> times 2 3
6
ghci> fct = \g -> fix(\f x -> if x == 0 then 1 else g x (f (x-1)))
ghci> fact = fct times
ghci> fact 5
120
\end{haskellcode}


\newpage
\section{Денотационна семантика с предаване на параметрите по име}

Семантиката на всеки тип ще бъде област на Скот както следва:
\begin{align*}
  & \val{\vv{nat}} \df \Nat_\bot\\
  & \val{\vv{a} \to \vv{b}} \df \Cont{\val{\vv{a}}}{\val{\vv{b}}}.
\end{align*}
\marginpar{Да напомним, че \[\emptyset_\bot = (\{\bot\},\sqsubseteq,\bot).\]}
\marginpar{От Раздел~\ref{subsect:domains:product} знаем, че $\val{\Gamma}$ е област на Скот.}
За един типов контекст $\Gamma$, дефинираме $\val{\Gamma}$ по следния начин:
\begin{itemize}
\item
  Ако $\Gamma = \emptyset$, то $\val{\Gamma} = \emptyset_\bot$;
\item
  Ако $\Gamma = \Gamma', \vv{x:a}$, то $\val{\Gamma} = \val{\Gamma'} \times \val{\vv{a}}$.
\end{itemize}
Например, ако $\Gamma = \vv{x}_1 : \vv{a}_1,\ \vv{x}_2 : \vv{a}_2,\ \vv{x}_3 : \vv{a}_3$, то
\[\val{\Gamma} = (\val{\vv{a}_1} \times \val{\vv{a}_2})\times \val{\vv{a}_3}.\]

Сега трябва да дефинираме семантика на термовете.
За всеки терм, за който $\fv(\tau) \subseteq \texttt{dom}(\Gamma)$ и
за произволни $\overline{u} \in \val{\Gamma}$, дефинираме неговата стойност $\val{\tau}_\Gamma(\overline{u})$ по следния начин:
\begin{itemize}
\item
  Нека $\tau \equiv \vv{n}$. Тогава
  \[\val{\vv{n}}_\Gamma(\overline{u}) \df n.\]
\item
  Нека $\tau \equiv \vv{x}_i$. Тогава
  \[\val{\vv{x}_i}_\Gamma(\overline{u}) \df u_i.\]
\item
  \marginpar{За $\texttt{plus}$ вижте Раздел~\ref{subsect:rec:term-value}.}
  Нека $\tau \equiv \tau_1 + \tau_2$. Тогава
  \[\val{\tau_1 + \tau_2}_\Gamma(\overline{u}) \df \texttt{plus}(\val{\tau_1}_\Gamma(\overline{u}), \val{\tau_2}_\Gamma(\overline{u})).\]
\item
  \marginpar{За $\texttt{eq}$ вижте Раздел~\ref{subsect:rec:term-value}.}
  Нека $\tau \equiv \tau_1\ \vv{==}\ \tau_2$. Тогава
  \[\val{\tau_1\ \vv{==}\ \tau_2}_\Gamma(\overline{u}) \df \texttt{eq}(\val{\tau_1}_\Gamma(\overline{u}), \val{\tau_2}_\Gamma(\overline{u})).\]
\item
  \marginpar{За $\texttt{if}$ вижте \Def{if}.}
  Нека $\tau \equiv \ifelse{\tau_1}{\tau_2}{\tau_3}$. Тогава
  \[\val{\ifelse{\tau_1}{\tau_2}{\tau_3}}_\Gamma(\overline{u}) \df \texttt{if}(\val{\tau_1}_\Gamma(\overline{u}),
  \val{\tau_2}_\Gamma(\overline{u}), \val{\tau_3}_\Gamma(\overline{u})).\]
\item
  \marginpar{За $\texttt{eval}$ вижте \Def{eval}.}
  Нека $\tau \equiv \tau_1 \tau_2$. Тогава
  \[\val{\tau_1 \tau_2}_\Gamma(\overline{u}) \df \texttt{eval}(\val{\tau_1}_\Gamma(\overline{u}), \val{\tau_2}_\Gamma(\overline{u})).\]
\item
  \marginpar{За $\lfp$ вижте Раздел~\ref{sect:lfp}.}
  Нека $\tau \equiv \fix(\tau')$. Тогава 
  \[\val{\fix(\tau')}_\Gamma(\overline{u}) \df \lfp(\val{\tau'}_\Gamma(\overline{u})).\]
\item
  \marginpar{За $\curry$ вижте \Def{curry}.}
  Нека $\tau \equiv \lamb{y}{b}{\tau'}$, като $\vv{y} \not \in \texttt{dom}(\Gamma)$.
  Нека $\Gamma' \df \Gamma, \type{y}{b}$. Тогава
  \[\val{\lamb{y}{b}{\tau'}}_\Gamma(\overline{u}) \df \curry(\val{\tau'}_{\Gamma'})(\overline{u}).\]
\end{itemize}

\begin{remark}
  За $\Gamma = \emptyset$, ще пишем $\val{\tau}$ вместо $\val{\tau}_\emptyset$.
\end{remark}

Не е ясно дали винаги горните дефиниции имат смисъл.
Сега ще докажем, че винаги, когато един терм е добре типизиран, то горната дефиниция има смисъл.

\begin{framed}
  \begin{lemma}
    Ако $\Gamma \vdash \tau : \vv{a}$, то $\val{\tau}_\Gamma \in \Cont{\val{\Gamma}}{\val{\vv{a}}}$.
  \end{lemma}  
\end{framed}
\begin{proof}
  Доказателството протича с индукция по построението на термовете
  като съществено използваме \Prop{composition} според което, ако $f \in \Cont{\A}{\B}$ и $g \in \Cont{\B}{\C}$, то
  $g \circ f \in \Cont{\A}{\C}$.
  \marginpar{Изображението $f \times g$ е дефинирано в \Prop{cartesian-continuous}.}
  \begin{itemize}
  \item
    Нека $\tau \equiv \vv{n}$. Щом $\Gamma \vdash \tau : \vv{a}$, то
    по правилата за типизиране следва, че $\vv{a} = \vv{nat}$.
    Сега лесно се съобразява, че изображението $\val{\vv{n}}_\Gamma \in \Cont{\val{\Gamma}}{\val{\vv{nat}}}$, където
    $\val{\vv{n}}_\Gamma(\overline{u}) = n$.
    Това е така, защото за всяка верига $\chain{\overline{u}}{i}$ от елементи на $\val{\Gamma}$,
    \[\val{\vv{n}}_\Gamma(\bigsqcup_i\overline{u}_i) = n = \bigsqcup_i\{ \val{\vv{n}}(\overline{u}_i)\}.\]
  \item
    Нека $\tau \equiv \vv{x}_i$. Щом $\Gamma \vdash \tau : \vv{a}$, то
    по правилата за типизиране следва, че $\vv{a} = \vv{a}_i$.
    Сега лесно се съобразява, че изображението $\val{\vv{n}}_\Gamma \in \Cont{\val{\Gamma}}{\val{\vv{a}_i}}$, където
    $\val{\vv{n}}_\Gamma(\overline{u}) = u_i$.
    \marginpar{$\overline{u}_k = (u_{1,k},\dots,u_{n,k})$.}
    Това е така, защото за всяка верига $\chain{\overline{u}}{n}$ от елементи на $\val{\Gamma}$,
    \[\val{\vv{x}_i}_\Gamma(\bigsqcup_n\overline{u}_n) = \bigsqcup_n u_{i,n} = \bigsqcup_i\{ \val{\vv{x}_i}(\overline{u}_n)\}.\]
  \item
    Нека $\tau \equiv \tau_1 + \tau_2$. Щом $\Gamma \vdash \tau : \vv{a}$, то
    по правилата за типизиране следва, че $\vv{a} = \vv{nat}$, а също и $\Gamma \vdash \tau_1 : \vv{nat}$ и $\Gamma \vdash \tau_2
    : \vv{nat}$.
    От И.П. имаме, че
    \begin{align*}
      & \val{\tau_1}_\Gamma \in \Cont{\val{\Gamma}}{\val{\vv{nat}}};\\
        & \val{\tau_2}_\Gamma \in \Cont{\val{\Gamma}}{\val{\vv{nat}}}.
    \end{align*}
    Това означава, че $(\val{\tau_1} \times \val{\tau_2}) \in \Cont{\val{\Gamma}}{\val{\vv{nat}} \times \val{\vv{nat}}}$.
    Тогава имаме следното равенство
    \marginpar{Използваме, че композиция на непрекъснати изображения е непрекъснато изображение.}
    \[\val{\tau_1 + \tau_2}_\Gamma = \texttt{plus} \circ (\val{\tau_1} \times \val{\tau_2}) \in \Cont{\val{\Gamma}}{\val{\vv{a}}},\]
    защото за произволни $\overline{u} \in \val{\Gamma}$,
    \begin{align*}
      (\texttt{plus} \circ (\val{\tau_1} \times \val{\tau_2}))(\overline{u}) & = \texttt{plus}((\val{\tau_1} \times \val{\tau_2})(\overline{u}))\\ 
                                                                             & = \texttt{plus}(\val{\tau_1}_\Gamma(\overline{u}), \val{\tau_2}_\Gamma(\overline{u}))\\
                                                                             & \df \val{\tau}_\Gamma(\overline{u}).
    \end{align*}
  \item
    Нека $\tau \equiv \tau_1\ \vv{==}\ \tau_2$. Съобразете сами, че 
    \[\val{\tau_1\ \vv{==}\ \tau_2}_\Gamma = \texttt{eq} \circ (\val{\tau_1}_\Gamma \times \val{\tau_2}_\Gamma) \in \Cont{\val{\Gamma}}{\val{\vv{a}}}.\]
  \item
    Нека $\tau \equiv \ifelse{\tau_1}{\tau_2}{\tau_3}$. Съобразете сами, че 
    \[\val{\ifelse{\tau_1}{\tau_2}{\tau_3}}_\Gamma = \texttt{if} \circ (\val{\tau_1}_\Gamma \times \val{\tau_2}_\Gamma \times \val{\tau_3}_\Gamma)  \in \Cont{\val{\Gamma}}{\val{\vv{a}}}.\]
  \item
    Нека $\tau \equiv \tau_1 \tau_2$.
    Щом $\Gamma \vdash \tau_1 \tau_2 : \vv{a}$, то от правилата за типизиране следва, че
    \begin{align*}
      & \Gamma \vdash \tau_1 : \vv{b} \to \vv{a}\\
      & \Gamma \vdash \tau_2 : \vv{b}.
    \end{align*}
    От И.П. за $\tau_1$ и $\tau_2$ знаем, че
    \begin{align*}
      & \val{\tau_1}_\Gamma \in \Cont{\val{\Gamma}}{\Cont{\val{\vv{b}}}{\val{\vv{a}}}} \\
      & \val{\tau_2}_\Gamma \in \Cont{\val{\Gamma}}{\val{\vv{b}}}
    \end{align*}
    Оттук получаваме, че за произволни $\overline{u} \in \val{\Gamma}$,
    \begin{align*}
      & \val{\tau_1}_\Gamma(\overline{u}) \in \Cont{\val{\vv{b}}}{\val{\vv{a}}} \\
      & \val{\tau_2}_\Gamma(\overline{u}) \in \val{\vv{b}}.
    \end{align*}
    Тогава 
    \[\val{\tau_1 \tau_2}_\Gamma = \texttt{eval} \circ (\val{\tau_1}_\Gamma \times \val{\tau_2}_\Gamma) \in \Cont{\val{\Gamma}}{\val{\vv{a}}},\]
    защото за произволни $\overline{u} \in \val{\Gamma}$,
    \begin{align*}
      (\texttt{eval} \circ \val{\tau_1}_\Gamma \times \val{\tau_2}_\Gamma)(\overline{u}) & = \texttt{eval}((\val{\tau_1}_\Gamma \times \val{\tau_2}_\Gamma)(\overline{u}))\\
                                                                                         & = \texttt{eval}(\val{\tau_1}_\Gamma(\overline{u}), \val{\tau_2}_\Gamma(\overline{u}))\\
                                                                                         & \df \val{\tau_1\tau_2}_\Gamma(\overline{u}).
    \end{align*}
    
  \item
    Нека сега $\tau \equiv \fix(\tau')$.
    Понеже $\Gamma \vdash \fix(\tau') : \vv{a}$, то от правилата за типизиране имаме, че
    $\Gamma \vdash \tau' : \vv{a} \to \vv{a}$.
    От И.П. знаем, че
    \[\val{\tau'}_\Gamma \in \Cont{\val{\Gamma}}{\Cont{\val{\vv{a}}}{\val{\vv{a}}}}.\]
    Това означава, че за произволни $\overline{u} \in \val{\Gamma}$,
    \[\val{\tau'}_\Gamma(\overline{u}) \in \Cont{\val{\vv{a}}}{\val{\vv{a}}}.\]
    Следователно
    $\val{\tau'}_\Gamma(\overline{u})$ е изображение, което според \Th{knaster-tarski}
    притежава най-малка неподвижна точка.
    \marginpar{Непрекъснатото изображението $Y$ е дефинирано \Th{Y}.}
    Тогава
    \[\val{\texttt{fix}(\tau')}_\Gamma = Y \circ \val{\tau'}_\Gamma \in \Cont{\val{\Gamma}}{\val{\vv{a}}},\]
    защото за произволни $\overline{u} \in \val{\Gamma}$,
    \begin{align*}
      (Y \circ \val{\tau'}_\Gamma)(\overline{u}) & = Y(\val{\tau'}_\Gamma(\overline{u}))\\
                                                 & = \lfp(\val{\tau'}_\Gamma(\overline{u}))\\
                                                 & \df \val{\fix(\tau')}_\Gamma(\ov{u}).
    \end{align*}
  \item
    Нека $\tau \equiv \lamb{y}{b}{\tau'}$, като $\vv{y} \not \in \texttt{dom}(\Gamma)$.
    Щом $\Gamma \vdash \lamb{y}{b}{\tau'} : \vv{a}$, то от правилата за типизиране следва, че $\vv{a} = \vv{b} \to \vv{c}$
    и 
    \[\Gamma, \type{y}{b} \vdash \tau' : \vv{c}.\]
    
    Нека $\Gamma' = \Gamma, \vv{y}:\vv{b}$. Тогава $\val{\Gamma'} = \val{\Gamma} \times \val{\vv{b}}$, а от И.П. имаме, че
    \[\val{\tau'}_{\Gamma'} \in \Cont{\val{\Gamma} \times \val{\vv{b}}}{\val{\vv{c}}}.\]
    Тогава от \Prop{curry} следва, че
    \[\val{\lamb{y}{b}{\tau}}_\Gamma \df \curry(\val{\tau'}_{\Gamma'}) \in \Cont{\val{\Gamma}}{\Cont{\val{\vv{b}}}{\val{\vv{c}}}}.\]
  \end{itemize}
\end{proof}

\begin{remark}
  В случая $\Gamma = \emptyset$, формално погледнато,
  $\val{\tau}_\emptyset \in \Cont{\emptyset_\bot}{\A}$, за някоя област на Скот $\A$.
  Но ние знаем, че $\Cont{\emptyset_\bot}{\A} \cong \A$.
  Следователно, можем да считаме, че $\val{\tau} \in \A$.
  В противен случай, трябва винаги да пишем $\val{\tau}(\bot)$ вместо $\val{\tau}$.
\end{remark}


\begin{proposition}
  \marginpar{Ясно е, че това твърдение се обобщава за произволна пермутация на индекстите $1,\dots,n$ \cite[стр. 106]{types-programming-languages}.}
  Нека имаме следните типови контексти:
  \begin{align*}
    &\Gamma = \vv{x}_1:\vv{a}_1, \dots, \vv{x}_i:\vv{a}_i, \dots, \vv{x}_j:\vv{a}_j, \dots, \vv{x}_n:\vv{a}_n;\\
    &\Delta = \vv{x}_1:\vv{a}_1, \dots, \vv{x}_j:\vv{a}_j, \dots, \vv{x}_i:\vv{a}_i, \dots, \vv{x}_n:\vv{a}_n,
  \end{align*}
  т.е. $\Delta$ се получава от $\Gamma$ като разменим местата на $i$-тата и $j$-тата двойка.
  Тогава за всеки терм $\tau$, такъв че $\Gamma \vdash \tau : \vv{a}$, е изпълено, че $\Delta \vdash \tau : \vv{a}$ и за всеки $(u_1,\dots,u_n) \in \val{\Gamma}$,
  \[\val{\tau}_\Gamma(u_1,\dots,u_i,\dots,u_j,\dots,u_n) = \val{\tau}_\Delta(u_1,\dots,u_j,\dots,u_i,\dots,u_n).\]
\end{proposition}
\begin{hint}
  Индукция по построението на терма $\tau$.
\end{hint}


%%% Local Variables:
%%% mode: latex
%%% TeX-master: "../sep"
%%% End:

\newpage
\begin{example}
  Нека $\vv{a} = \vv{nat} \to (\vv{nat} \to \vv{nat})$ и 
  \[\tau \equiv \fix\vv{(}\underbrace{\lamb{f}{a}{\overbrace{\lamb{x}{nat}{\lamb{y}{nat}{\ifelse{\vv{y == 0}}{\vv{0}}{\vv{x + (f x (y-1))} }}}}^{\tau'_0}}}_{\tau_0}\vv{)}.\]

  Знаем, че $\val{\tau} \in \Cont{\Nat_\bot}{\Cont{\Nat_\bot}{\Nat_\bot}}$, където
  \[\val{\tau} \df \lfp(\val{\tau_0}).\]

  Ясно е, че $\val{\tau_0} \in \Cont{\val{\vv{a}}}{\val{\vv{a}}}$ и
  $\val{\tau_0} = \curry(\val{\tau'_0}_{\vv{f:a}}) = \val{\tau'_0}_{\vv{f:a}}$ и
  сега пък ако положим
  \begin{align*}
    % & \tau_0 \equiv \lamb{f}{a}{\lamb{x}{nat}{\lamb{y}{nat}{\ifelse{\vv{y == 0}}{\vv{0}}{\vv{x + (f x (y-1))} }}}},\\
    % & \tau'_0 \equiv \lamb{x}{nat}{\lamb{y}{nat}{\ifelse{\vv{y == 0}}{\vv{0}}{\vv{x + (f x (y-1))}}}},\\
    & \tau''_0 \equiv \lamb{y}{nat}{\ifelse{\vv{y == 0}}{\vv{0}}{\vv{x + (f x (y-1))}}},\\
    & \tau'''_0 \equiv \ifelse{\vv{y == 0}}{\vv{0}}{\vv{x + (f x (y-1))}}.
  \end{align*}

  то ще получим, че
  \[\val{\tau'_0}_{\vv{f:a}} = \curry(\val{\tau''_0}_{\vv{f:a,x:nat}}),\]
  и тогава
  \[\val{\tau'_0}_{\vv{f:a}}(\varphi)(m) = \val{\tau''_0}_{\vv{f:a,x:nat}}(\varphi,m).\]
  Сега вече получаваме, че
  \[\val{\tau''_0}_{\vv{f:a,x:nat}} = \curry(\val{\tau'''_0}_{\vv{f:a,x:nat,y:nat}}),\]
  т.е.
  \[\val{\tau''_0}_{\vv{f:a,x:nat}}(\varphi,m)(n) = \val{\tau'''_0}_{\vv{f:a,x:nat,y:nat}}(\varphi,m,n).\]

  Обединявайки всичко получаваме, че:
  \begin{align*}
    \val{\tau_0}(\varphi)(m)(n) & = \val{\tau'_0}_{\vv{f:a}}(\varphi)(m)(n) \\
                                & = \val{\tau''_0}_{\vv{f:a,x:nat}}(\varphi,m)(n)\\
                                & = \val{\tau'''_0}_{\vv{f:a,x:nat,y:nat}}(\varphi,m,n)\\
                                & = \val{\ifelse{\vv{y==0}}{\vv{0}}{\vv{x + (f x (y-1))}}}(\varphi,m,n)\\
                                & = \texttt{if}(\val{\vv{y==0}}(\varphi,m,n), \val{\vv{0}}(\varphi,m,n),\val{\vv{x + (f x (y-1))}}(\varphi,m,n)).
                                % & = \texttt{if}(\eq(n,0),0,\plus(m, \texttt{eval}(\texttt{eval}(\varphi, m),n-1))).
  \end{align*}

  Накрая получаваме, че
  \[\val{\tau_0}(\varphi)(m)(n) = \begin{cases}
      0, & \text{ако }n = 0\\
      \plus(m, \varphi(m)(n-1)), & \text{ако } n > 0\\
      \bot, & \text{ако }n = \bot.
    \end{cases}
  \]
  
                                
  Сега вече знаем как по теоремата на Клини да докажем, че
  \[\lfp(\val{\tau_0})(m)(n) =
    \begin{cases}
      m*n,  & \text{ако }m,n\in\Nat\\
      \bot, & \text{иначе}
    \end{cases}
\]
                                
  
\end{example}

\begin{framed}
\begin{lemma}[Лема за замяната]\label{lem:pcf:substitution}
  Нека $\Gamma$ е типов контекст, $\tau$ и $\rho$ са термове, $\vv{x} \not\in \texttt{dom}(\Gamma)$,
  \begin{align*}
    & \Gamma \vdash \rho : \vv{a}\\
    & \Gamma, \type{x}{a} \vdash \tau : \vv{b}.
  \end{align*}
  Тогава
  \begin{enumerate}[1)]
  \item
    $\Gamma \vdash \tau\subst{x}{\rho} : \vv{b}$;
  \item
    за всяко $\overline{u} \in \val{\Gamma}$,
    \[\val{\tau\subst{x}{\rho}}_\Gamma(\overline{u}) = \val{\tau}_{\Gamma'}(\overline{u},\val{\rho}_\Gamma(\overline{u})),\]
    където $\Gamma' = \Gamma, \type{x}{a}$.  
  \end{enumerate}
\end{lemma}
\end{framed}
\marginpar{Защо да не взема $\rho$ да бъде затворен терм ?}
\begin{proof}
  Индукция по построението на термовете.
  \begin{itemize}
  \item
    Нека $\tau \equiv \vv{x}_i$, където $\vv{x}_i \not\equiv \vv{x}$.
  \item
    Нека $\tau \equiv \vv{x}$.
  \item
    Нека $\tau \equiv \vv{n}$.
  \item
    Нека $\tau \equiv \ifelse{\tau_1}{\tau_2}{\tau_3}$.
  \item
    Нека $\tau \equiv \tau_1 + \tau_2$.
  \item
    Нека $\tau \equiv \tau_1\ \vv{==}\ \tau_2$.
  \item
    Нека $\tau \equiv \tau_1 \tau_2$.
    Тук първата част е лесна. Понеже имаме, че
    \begin{prooftree}
      \AxiomC{$\Gamma, \type{x}{a} \vdash \tau_1: \vv{c} \to \vv{b}$}
      \AxiomC{$\Gamma, \type{x}{a} \vdash \tau_2: \vv{c}$}
      \BinaryInfC{$\Gamma, \type{x}{a} \vdash \tau_1 \tau_2 : \vv{b}$}
    \end{prooftree}
    то можем да приложим И.П. за да получим, че
    \begin{prooftree}
      \AxiomC{$\Gamma \vdash \rho : \vv{a}$}
      \AxiomC{$\Gamma, \type{x}{a} \vdash \tau_1: \vv{c} \to \vv{b}$}
      \LeftLabel{\scriptsize{(И.П.)}}
      \BinaryInfC{$\Gamma \vdash \tau_1\subst{\vv{x}}{\rho} : \vv{b}$}
      \AxiomC{$\Gamma \vdash \rho : \vv{a}$}
      \AxiomC{$\Gamma, \type{x}{a} \vdash \tau_2: \vv{c} \to \vv{b}$}
      \RightLabel{\scriptsize{(И.П.)}}
      \BinaryInfC{$\Gamma \vdash \tau_2\subst{x}{\rho} : \vv{c}$}
      \RightLabel{\scriptsize{(app)}}
      \BinaryInfC{$\Gamma \vdash \tau_1\subst{x}{\rho}(\tau_2\subst{x}{\rho}) : \vv{c}$}
      \RightLabel{\scriptsize{(правила на замяна)}}
      \UnaryInfC{$\Gamma \vdash \tau\subst{x}{\rho} : \vv{c}$}
    \end{prooftree}
    Втората част също е лесна.
    \begin{align*}
      \val{\tau_1\tau_2}_{\Gamma'}(\ov{u},\val{\rho}_\Gamma(\ov{u})) & \df \texttt{eval}(\val{\tau_1}_{\Gamma'}(\ov{u},\val{\rho}_\Gamma(\ov{u})), \val{\tau_2}_{\Gamma'}(\ov{u},\val{\rho}_\Gamma(\ov{u}))) & \comment\text{\Def{eval}}\\
                                                                   & = \texttt{eval}(\val{\tau_1\subst{x}{\rho}}_\Gamma(\ov{u}), \val{\tau_2\subst{x}{\rho}}_\Gamma(\ov{u})) & \comment\text{И.П.}\\
                                                                   & = \val{\tau_1\subst{x}{\rho}(\tau_2\subst{x}{\rho})}_\Gamma(\ov{u})\\
                                                                   & = \val{\tau\subst{x}{\rho}}_\Gamma(\ov{u})
    \end{align*}
  \item
    Нека $\tau \equiv \fix(\tau')$.
    Първо трябва да докажем, че $\Gamma \vdash \tau[\vv{x}/\rho] : \vv{b}$.
    От правилата за типизиране е ясно, че
    \begin{prooftree}
      \AxiomC{$\Gamma, \type{x}{a} \vdash \tau':\vv{b}\to\vv{b}$}
      \RightLabel{\scriptsize{(fix)}}
      \UnaryInfC{$\Gamma, \type{x}{a} \vdash \fix(\tau') : \vv{b}$}
    \end{prooftree}
    Сега можем да приложим И.П. за терма $\tau'$. Получаваме, че
    \begin{prooftree}
      \AxiomC{$\Gamma \vdash \rho: \vv{a}$}
      \AxiomC{$\Gamma, \type{x}{a} \vdash \tau':\vv{b}\to\vv{b}$}
      \RightLabel{\scriptsize{(И.П.)}}
      \BinaryInfC{$\Gamma \vdash \tau'\subst{x}{\rho} : \vv{b} \to \vv{b}$}
      \RightLabel{\scriptsize{(fix)}}
      \UnaryInfC{$\Gamma \vdash \fix(\tau'\subst{x}{\rho}) : \vv{b}$}
    \end{prooftree}
    Понеже $\fix(\tau'\subst{x}{\rho}) \equiv \fix(\tau')\subst{x}{\rho}$, то заключаваме, че
    \[\Gamma \vdash \tau[\vv{x}/\rho] : \vv{b}.\]
    
    Сега трябва да проверим защо $\val{\fix(\tau\subst{x}{\rho})}_{\Gamma}(\ov{u}) = \val{\fix(\tau)}_{\Gamma'})(\ov{u},\val{\rho}_\Gamma)$.
    Получаваме следното:
    \begin{align*}
      \val{\fix(\tau\subst{x}{\rho})}_{\Gamma}(\ov{u}) & = \lfp(\val{\tau\subst{x}{\rho}}_\Gamma(\ov{u})) & \comment\text{от деф.}\\
                                                       & = \lfp(\val{\tau}_{\Gamma'}(\ov{u},\val{\rho}_\Gamma(\ov{u}))) & \comment\text{от И.П. за }\tau\\
                                                       & = \val{\fix(\tau)}_{\Gamma'}(\ov{u},\val{\rho}_\Gamma(\ov{u})).
    \end{align*}
    
    
  \item
    \marginpar{Тук е важно, че $\val{\Delta} = \val{\Gamma} \times \val{\vv{a}_n}$}

    Нека $\tau \equiv \lamb{y}{c}{\tau'}$, където $\vv{y} \not\in \vv{dom}(\Gamma) \cup \{\vv{x}\}$.
    Първо трябва да докажем, че $\Gamma \vdash \tau[\vv{x}/\rho] : \vv{b}$.
    
    От правилата за типизиране е ясно, че щом $\Gamma, \type{x}{a} \vdash \tau : \vv{b}$, то
    $\vv{b} = \vv{c} \to \vv{d}$ за някой тип $\vv{d}$ и
    \begin{prooftree}
      \AxiomC{$\vv{y} \not\in\vv{dom}(\Gamma)\cup\{\vv{x}\}$}
      \AxiomC{$\Gamma, \type{x}{a}, \type{y}{c} \vdash \tau':\vv{d}$}
      \RightLabel{\scriptsize{(lambda)}}
      \BinaryInfC{$\Gamma, \type{x}{a} \vdash \lamb{y}{c}{\tau'} : \vv{c}\to\vv{d}$}
    \end{prooftree}
    Това означава, че можем да използваме И.П. за терма $\tau'$ и така получаваме, че
    \begin{prooftree}
      \AxiomC{$\vv{y} \not\in \vv{dom}(\Gamma)$}
      \AxiomC{$\Gamma \vdash \rho : \vv{a}$}
      \UnaryInfC{$\Gamma,\type{y}{c} \vdash \rho : \vv{a}$}
      \AxiomC{$\Gamma, \type{y}{c}, \type{x}{a} \vdash \tau':\vv{d}$}
      \RightLabel{\scriptsize{(И.П.)}}
      \BinaryInfC{$\Gamma, \type{y}{c} \vdash \tau'\subst{x}{\rho} : \vv{d}$}
      \RightLabel{\scriptsize{(lambda)}}
      \BinaryInfC{$\Gamma \vdash \lamb{y}{c}{\tau'\subst{x}{\rho}}:\vv{c}\to\vv{d}$}
    \end{prooftree}
    Накрая, понеже $\tau\subst{x}{\rho} \equiv \lamb{y}{c}{\tau'\subst{x}{\rho}}$, то
    заключаваме, че
    \[\Gamma \vdash \tau\subst{x}{\rho}:\vv{b}.\]

    Нека $\Delta \df \Gamma,\type{y}{c}$.
    Понеже имаме, че $\Delta \vdash \tau'\subst{x}{\rho} : \vv{d}$,
    то можем да приложим И.П. за $\tau'$ и така получаваме, че за всяко $\overline{u},v \in \val{\Delta}$,
    \begin{align*}
      \curry(\val{\tau'\subst{x}{\rho}}_\Delta)(\ov{u})(v) & \df \val{\tau'\subst{x}{\rho}}_\Delta(\overline{u},v) & \comment\text{\Def{curry}}\\
                                                                 & \stackrel{\text{И.П.}}{=} \val{\tau'}_{\Delta'}(\overline{u},v,\val{\rho}_\Delta(\ov{u},v)) & \comment \Delta' \df \Gamma, \type{y}{c}, \type{x}{a}\\
                                                                 & = \val{\tau'}_{\Delta'}(\ov{u},v,\val{\rho}_\Gamma(\ov{u})) & \comment \fv(\rho) \subseteq \vv{dom}(\Gamma)\\
                                                                 & = \val{\tau'}_{\Delta''}(\ov{u},\val{\rho}_\Gamma(\ov{u}),v) & \comment \Delta'' \df \Gamma, \type{x}{a}, \type{y}{c}\\
                                                                 & = \curry(\val{\tau'}_{\Delta''})(\ov{u},\val{\rho}_\Gamma(\ov{u}))(v) \\
                                                                 & = \val{\lamb{y}{c}{\tau'}}_{\Gamma'}(\ov{u},\val{\rho}_\Gamma(\ov{u}))(v). & \comment \Gamma' \df \Gamma, \type{x}{a}
    \end{align*}    
    Така получихме, че
    \begin{align*}
      \val{\lamb{y}{c}{\tau'\subst{x}{\rho}}}_\Gamma(\ov{u}) & \df \curry(\val{\tau'\subst{x}{\rho}}_\Delta)(\ov{u})\\
                                                                   & = \val{\lamb{y}{c}{\tau'}}_{\Gamma'}(\ov{u},\val{\rho}_\Gamma(\ov{u})).
    \end{align*}
    
  \end{itemize}
\end{proof}



%%% Local Variables:
%%% mode: latex
%%% TeX-master: "../sep"
%%% End:

Добре е още тук да разгледаме термове, които ще ни бъдат полезни по-нататък, когато искаме да изучим свойствата на операционната и денотационната семантики на \PCF.

\begin{framed}
  \begin{definition}
    За произволен тип $\vv{a}$, да означим затворените термове
    \begin{align*}
      & \Omega_{\vv{a}} \df \fix(\lamb{x}{a}{\vv{x}})\\
      & \Omega'_{\vv{a}} \df \lamb{x}{a}{\fix(\lamb{x}{a}{\vv{x}})}.
    \end{align*}
  \end{definition}
\end{framed}

\begin{problem}
  \label{prob:pcf:context:omega}
  \marginpar{\todo Студентите би трябвало да могат да докажат твърденията в \Problem{pcf:context:omega} сами!}
  Докажете, че за всеки тип $\vv{a}$ са изпълнени следните свойства:
  \begin{enumerate}[(1)]
  \item
    \label{pcf:omega:type}
    $\emptyset \vdash \Omega_{\vv{a}} : \vv{a}$ и $\emptyset \vdash \Omega'_{\vv{a}} : \vv{a} \to \vv{a}$;
  \item
    \label{pcf:omega:operational}
    $\Omega_{\vv{a}} \not\Downarrow_{\vv{a}}$ и $\Omega'_{\vv{a}} \Downarrow^0_{\vv{a}\to\vv{a}} \Omega'_{\vv{a}}$;
  \item
    \label{pcf:omega:denotational}
    $\val{\Omega_{\vv{a}}} = \bot^{\val{\vv{a}}}$ и $\val{\Omega'_{\vv{a}}} = \val{\Omega_{\vv{a}\to\vv{a}}} = \bot^{\val{\vv{a}\to\vv{a}}}$.    
  \end{enumerate}
\end{problem}
\begin{hint}
  Доказателството на Свойство~\ref{pcf:omega:type} представлява едно просто упражнение, което за пълнота на изложението ще направим.

  \begin{figure}[H]
    \begin{subfigure}{0.5\textwidth}
      \begin{prooftree}
        \AxiomC{$\type{x}{a} \vdash \type{x}{a}$}
        \UnaryInfC{$\emptyset \vdash \lamb{x}{a}{\vv{x}} : \vv{a}\to\vv{a}$}
        \UnaryInfC{$\emptyset \vdash \underbrace{\fix(\lamb{x}{a}{\vv{x}})}_{\Omega_{\vv{a}}} : \vv{a}$}
      \end{prooftree}
      % \caption{}
      % \label{fig:pcf:context:omega}
    \end{subfigure}
    ~
    \begin{subfigure}{0.5\textwidth}
      \begin{prooftree}
        % \AxiomC{от (\ref{fig:pcf:context:omega})}
        \AxiomC{вече доказано}
        \UnaryInfC{$\emptyset \vdash \Omega_{\vv{a}} : \vv{a}$}
        \UnaryInfC{$\type{x}{a} \vdash \Omega_{\vv{a}} : \vv{a}$}
        \UnaryInfC{$\underbrace{\emptyset \vdash \lamb{x}{a}{\Omega_{\vv{a}}}}_{\Omega'_{\vv{a}}} : \vv{a} \to \vv{a}$}
      \end{prooftree}
      % \caption{}
    \end{subfigure}
  \end{figure}
  
  \marginpar{Втората част на Свойство~\ref{pcf:omega:operational} не заслужава внимание, защото $\Omega'_{\vv{a}}$ е стойност и следователно по дефиниция $\Omega'_{\vv{a}} \opsemGen{0}{\vv{a}\to\vv{a}} \Omega'_{\vv{a}}$.}
  За първата част на Свойство~\ref{pcf:omega:operational}, да допуснем, че $\Omega_{\vv{a}} \Downarrow_{\vv{a}} \vv{v}$, за някоя стойност $\type{v}{a}$, и нека фиксираме $\ell$
  да бъде {\em най-малкият} брой стъпки, за които $\Omega_{\vv{a}} \Downarrow^{\ell}_{\vv{a}} \vv{v}$.
  Но тогава
  \begin{prooftree}
    \AxiomC{$\lamb{x}{a}{\vv{x}}$ е стойност}
    \UnaryInfC{$\lamb{x}{a}{\vv{x}} \Downarrow^0_{\vv{a}\to\vv{a}} \lamb{x}{a}{\vv{x}}$}
    \AxiomC{$\Omega_{\vv{a}} \Downarrow^{\ell-2}_{\vv{a}} \vv{v}$}
    \UnaryInfC{$\vv{x}\subst{x}{\fix(\lamb{x}{a}{\vv{x}})} \Downarrow^{\ell-2}_{\vv{a}} \vv{v}$}
    \RightLabel{\scriptsize{(cbn)}}
    \BinaryInfC{$(\lamb{x}{a}{\vv{x}})\fix(\lamb{x}{a}{\vv{x}}) \Downarrow^{\ell-1}_{\vv{a}} \vv{v}$}
    \RightLabel{\scriptsize{(fix)}}
    \UnaryInfC{$\underbrace{\fix(\lamb{x}{a}{\vv{x}})}_{\Omega_{\vv{a}}} \Downarrow^\ell_{\vv{a}} \vv{v}$}
  \end{prooftree}
  Получихме, че $\Omega_{\vv{a}} \Downarrow^{\ell-2}_{\vv{a}} \vv{v}$, което е противоречие с минималността на $\ell$.

  Сега да разгледаме първата част на Свойство~\ref{pcf:omega:denotational}. За произволен тип $\vv{a}$, да означим с $\texttt{id}_{\val{\vv{a}}}$ функцията идентитет за областта на Скот $\val{\vv{a}}$, т.е.
  $\texttt{id}_{\val{\vv{a}}}(x) = x$ за всяко $x \in \val{\vv{a}}$. Имаме, че:
  \marginpar{Да напомним, че $f^0 \df id$ и $f^{n+1} \df f \circ f^n$.
    В нашия случай, $f = id$ и следователно $id^n = id$ за всяко $n$.}
  \begin{align*}
    \val{\Omega_{\vv{a}}} & = \val{\fix(\lamb{x}{a}{\vv{x}})}\\
                          & = \lfp(\val{\lamb{x}{a}{\vv{x}}})\\
                          & = \lfp(\texttt{id}_{\val{\vv{a}}}) & \comment\val{\lamb{x}{a}{\vv{x}}} = \texttt{id}_{\val{\vv{a}}}\\
                          & = \bigsqcup_n\texttt{id}^n_{\val{\vv{a}}}(\bot^{\val{\vv{a}}}) & \comment\text{\hyperref[th:knaster-tarski]{Теоремата на Клини}}\\
                          & = \bigsqcup_n \bot^{\val{\vv{a}}} & \comment \texttt{id}^n(\bot^{\val{\vv{a}}}) = \bot^{\val{\vv{a}}}\\
                          & = \bot^{\val{\vv{a}}}.
  \end{align*}
  Сега преминаваме към втората част на Свойство~\ref{pcf:omega:denotational}. За произволен елемент $u \in \val{\vv{a}}$,
  \marginpar{Да напомним, че за всяко $u$, $\bot^{\val{\vv{a}\to\vv{a}}}(u) \df \bot^{\val{\vv{a}}}$.}
  \begin{align*}
    \val{\Omega'_a}(u) & = \val{\lamb{x}{a}{\Omega_{\vv{a}}}}(u)\\
                       & = \curry(\val{\Omega_{\vv{a}}}_{\type{x}{a}})(u)\\
                       & = \val{\Omega_{\vv{a}}}_{\type{x}{a}}(u)\\
                       & = \val{\Omega_{\vv{a}}}\\
                       & = \bot^{\val{\vv{a}}}.
  \end{align*}

  Заключаваме, че $\val{\Omega'_a} = \bot^{\val{\vv{a}\to\vv{a}}}$.
\end{hint}

\newpage
\section{Коректност}

\marginpar{Да напомним, че когато термът $\tau$ е затворен, то ще пишем $\val{\tau}$ вместо $\val{\tau}_\emptyset(\bot)$.}
Понеже вече имаме дефинирани операционна и денотационна семантика на термовете,
следващата стъпка е да разгледаме каква е връзката между тях.
В този раздел ще докажем едната (по-лесната) посока.
\marginpar{На англ. {\em soundness}.}
\begin{framed}
  \begin{theorem}[Теорема за коректност]\label{th:pcf:soundness}
    За всеки затворен терм $\tau : \vv{b}$ и стойност $\type{v}{b}$, е изпълнена импликацията:
    \[\tau \opsem{}{b} \vv{v}\ \implies\ \val{\tau} = \val{\vv{v}} \in \val{\vv{b}}.\]
  \end{theorem}  
\end{framed}
\begin{proof}
  Индукция по дължината $\ell$ на извода $\opsem{\ell}{b}$ за всеки тип $\vv{b}$.
  Нека $\ell = 0$. Имаме два случая, защото имаме два вида стойности.
  \begin{itemize}
  \item
    Нека $\tau \equiv \vv{n}$.
    Ясно е, че $\vv{b} = \vv{nat}$ и от правилата на операционната семантика имаме, че:
    \begin{prooftree}
      \AxiomC{}
      \RightLabel{\scriptsize{(val)}}
      \UnaryInfC{$\tau \opsem{0}{nat} \vv{n}$}
    \end{prooftree}
    От дефиницията на семантика на терм, директно получаваме, че
    $\val{\tau} = n = \val{\vv{n}}$.    
  \item
    Нека $\tau \equiv \lamb{x}{c}{\tau'}$. Тогава $\vv{b} = \vv{a}\to\vv{c}$ и от правилата на операционната семантика имаме, че:
    \begin{prooftree}
      \AxiomC{}
      \RightLabel{\scriptsize{(val)}}
      \UnaryInfC{$\tau \opsemGen{0}{\vv{a}\to\vv{c}} \tau$}
    \end{prooftree}
    Ясно е, че $\val{\tau} = \val{\tau} \in \val{b}$.
  \end{itemize}
  Така доказахме, че
  \[\tau \opsem{0}{b} \vv{v}\ \implies\ \val{\tau} = \val{\vv{v}} \in \val{\vv{b}}.\]
  Нека сега $\ell > 0$ и да приемем, че имаме следното индукционно предположение:
  \[\tau \opsem{<\ell}{b} \vv{v}\ \implies\ \val{\tau} = \val{\vv{v}} \in \val{\vv{b}}.\]
  Ще докажем, че
  \[\tau \opsem{\ell}{b} \vv{v}\ \implies\ \val{\tau} = \val{\vv{v}} \in \val{\vv{b}}.\]
  \begin{itemize}
  \item
    Нека $\tau \equiv \tau_1 + \tau_2$. Тогава от правилата на операционната семантика имаме, че:
    \begin{prooftree}
      \AxiomC{$\tau_1 \opsem{\ell_1}{nat} \vv{n}_1$}
      \AxiomC{$\tau_2 \opsem{\ell_2}{nat} \vv{n}_2$}
      \LeftLabel{\scriptsize{($\ell=\ell_1+\ell_2+1$)}}
      \RightLabel{\scriptsize{(plus)}}
      \BinaryInfC{$\tau_1 + \tau_2 \opsem{\ell_1+\ell_2+1}{nat} \vv{n},$}
    \end{prooftree}
    където $n = n_1 + n_2$. От \IndHyp получаваме, че
    \begin{align*}
      & \val{\tau_1} = \val{\vv{n}_1} = n_1\\
      & \val{\tau_2} = \val{\vv{n}_2} = n_2.
    \end{align*}
    Тогава
    \begin{align*}
      \val{\tau_1 + \tau_2} & = \plus(\val{\tau_1}, \val{\tau_2}) & \comment\text{от деф.}\\
                            & = n_1 + n_2 & \comment\text{\IndHyp}\\
                            & = n.
    \end{align*}
  \item
    Случаите $\tau \equiv \tau_1 - \tau_2$ и $\tau \equiv \tau_1\ \vv{==}\ \tau_2$ са аналогични. Оставяме ги на читателя.
  \item
    Нека $\tau \equiv \ifelse{\tau_1}{\tau_2}{\tau_3}$. Тогава от правилата на операционната семантика имаме, че:
    \begin{prooftree}
      \AxiomC{$\tau_1 \opsem{\ell_1}{nat} \vv{n}_1$}
      \AxiomC{$\tau_2 \opsem{\ell_2}{a} \vv{v}_2$}
      \AxiomC{$\vv{n}_1 \not\equiv \vv{0}$}
      \LeftLabel{\scriptsize{($\ell=\ell_1+\ell_2+1$)}}
      \RightLabel{\scriptsize{(if$^+$)}}
      \TrinaryInfC{$\ifelse{\tau_1}{\tau_2}{\tau_3} \opsem{\ell_1+\ell_2+1}{a} \vv{v}_2,$}
    \end{prooftree}
    Тогава от \IndHyp получаваме, че:
    \begin{align*}
      & \val{\tau_1} = n_1\\
      & \val{\tau_2} = \val{\vv{v}_2}.
    \end{align*}
    Тогава
    \begin{align*}
      \val{\ifelse{\tau_1}{\tau_2}{\tau_3}} & = \texttt{if}(\val{\tau_1}, \val{\tau_2}, \val{\tau_3})\\
                                            & = \texttt{if}(n_1,\val{\tau_2}, \val{\tau_3}) & \comment\text{от \IndHyp}\\
                                            & = \val{\tau_2} & \comment\text{от деф. на }\texttt{if}\\
                                            & = \val{\vv{v}_2}. & \comment\text{от \IndHyp}
    \end{align*}
    
    Случаят, когато $\vv{n}_1 \equiv \vv{0}$ е аналогичен.
  \item
    Нека $\tau \equiv \tau_1 \tau_2$. Тогава от правилата на операционната семантика имаме, че:
    \begin{prooftree}
      \AxiomC{$\tau_1 \opsemGen{\ell_1}{\vv{a}\to\vv{b}} \lamb{x}{a}{\tau'_1}$}
      \AxiomC{$\tau'_1[x/\tau_2] \opsem{\ell_2}{b} \vv{v}$}
      \LeftLabel{\scriptsize{($\ell=\ell_1+\ell_2+1$)}}
      \RightLabel{\scriptsize{(cbn)}}
      \BinaryInfC{$\tau_1 \tau_2 \opsem{\ell_1+\ell_2+1}{b} \vv{v} $}
    \end{prooftree}
    Тогава от \IndHyp получаваме, че:    
    \begin{align*}
      & \val{\tau_1} = \val{\lamb{x}{a}{\tau'_1}} \in \Cont{\val{\vv{a}}}{\val{\vv{b}}}\\
      & \val{\tau'_1\subst{x}{\tau_2}} = \val{\vv{v}} \in \val{\vv{b}}.
    \end{align*}
    Тогава
    \begin{align*}
      \val{\tau_1\tau_2} & = \texttt{eval}(\val{\tau_1},\val{\tau_2}) & \comment\text{от деф.}\\ 
                         & = \val{\tau_1}(\val{\tau_2}) & \comment \val{\tau_1} \in \Cont{\val{\vv{a}}}{\val{\vv{b}}}\\
                         & = \val{\lamb{x}{a}{\tau'_1}}(\val{\tau_2}) & \comment\text{\IndHyp}\\
                         & = \val{\tau'_1}_{\type{x}{a}}(\val{\tau_2})\\
                         & = \val{\tau'_1\subst{x}{\tau_2}} & \comment\text{от \hyperref[lem:pcf:substitution]{Лема за замяната}}\\
                         & = \val{\vv{v}} & \comment\text{\IndHyp}
    \end{align*}
  \item
    Нека $\tau \equiv \fix(\tau')$. Тогава от правилата на операционната семантика имаме, че:
    \begin{prooftree}
      \AxiomC{$\tau'\ \fix(\tau') \opsem{\ell-1}{a} \vv{v}$}
      \RightLabel{\scriptsize{(fix)}}
      \UnaryInfC{$\fix(\tau') \opsem{\ell}{a} \vv{v} $}
    \end{prooftree}
    Тогава от \IndHyp имаме, че:
    \[\val{\tau'\ \fix(\tau')} = \val{\vv{v}}.\]
    Тогава
    \begin{align*}
      \val{\tau} & = \lfp(\val{\tau'})\\
                 & = \val{\tau'}(\lfp(\val{\tau'}))\\
                 & = \val{\tau'}(\val{\fix(\tau')})\\
                 & = \val{\tau'\fix(\tau')}\\
                 & = \val{\vv{v}}. & \comment\text{\IndHyp}
    \end{align*}
  \end{itemize}
\end{proof}

\hyperref[th:pcf:soundness]{Теоремата за коректност}\ частично потвърждава нашата интуиция, че за типа $\vv{nat}$
можем да си мислим за $\bot^{\val{\vv{nat}}}$ като за изчисление, което никога не завършва.

\marginpar{Другата посока ще я получим след малко.}

\begin{corollary}
  Нека $\tau$ е затворен терм от тип $\vv{nat}$. Тогава
  \[\val{\tau} = \bot^{\val{\vv{nat}}}\ \implies\ \tau \not\opsem{}{nat}.\]
\end{corollary}

За жалост, тази наша интуиция се ,,губи'', когато се интересуваме от термове от по-висок от $\nat$ тип.
\marginpar{За дискусия по този въпрос вижте \cite[стр. 213]{models-of-computation}.}

Да разгледаме един пример. Нека 
\[\tau \equiv \lamb{y}{nat}{\fix(\lamb{x}{nat}{\vv{x}})}.\]
Лесно се съобразява, че $\tau : \nat\to\nat$.
За произволен елемент $a \in \Nat_\bot$ е изпъленено следното:
\begin{align*}
  \val{\tau}(a) & = \val{\lamb{y}{nat}{\fix(\lamb{x}{nat}{\vv{x}})}}(a)\\
                & = \curry(\val{\fix(\lamb{x}{nat}{\vv{x}})}_{\type{y}{nat}})(a)\\
                & = \val{\fix(\lamb{x}{nat}{\vv{x}})}_{\type{y}{nat}}(a)\\
                & = \lfp(\val{\lamb{x}{nat}{\vv{x}}}_{\type{y}{nat}}(a))\\
                & = \lfp(\underbrace{\curry(\val{x}_{\type{y}{nat},\type{x}{nat}})(a)}_{\texttt{id}\text{ за }\Nat_\bot})\\
                & = \bot^{\val{\nat}}
\end{align*}
С други думи, получаваме, че
\[\val{\tau} = \bot^{\val{\nat\to\nat}}.\]
От друга страна, обаче, $\tau$ представлява стойност. Следователно,
\[\tau \opsemGen{0}{\nat\to\nat} \tau.\]


%%% Local Variables:
%%% mode: latex
%%% TeX-master: "../sep"
%%% End:

\newpage
\section{Адекватност}
\marginpar{Adequacy ???}
Нашата цел в този раздел е да докажем следната теорема.
\begin{framed}
  \begin{theorem}[Теорема за адекватност]
    За всеки затворен терм $\tau : \vv{nat}$ е изпълнена импликацията
    \[\val{\tau} = n \neq \bot^{\val{\nat}} \implies \tau \Downarrow_{\vv{nat}} \vv{n}.\]
  \end{theorem}
\end{framed}
\marginpar{Тук $n$ е число, а $\vv{n}$ е константа.}

Оказва се, че доказателството на тази теорема не е леко.
Ще започнем като дефинираме за всеки тип $\vv{a}$ релацията 
$\triangleleft_{\vv{a}} \subseteq \val{\vv{a}} \times \vv{PCF}_{\vv{a}}$
с индукция по построението на типовете.

\begin{itemize}
\item
  \marginpar{Съобразете, че теоремата за адекватност на практика гласи, че $\val{\tau} \triangleleft_{\vv{nat}} \tau$.}
  Нека $\vv{a} = \vv{nat}$. Тогава 
  \marginpar{Обикновено $\triangleleft_{\vv{a}}$ се нарича \emph{логическа релация}.
    В \cite[стр. 210]{models-of-computation} е обяснено защо имаме нужда от тези релации за да докажем теоремата за адекватност. В \cite[стр. 134]{gunter} е представен синтактичен подход към решаването на този проблем.}
  \[n \triangleleft_{\vv{nat}} \tau \dff ( n\neq\bot^{\val{\vv{nat}}} \implies \tau \Downarrow_{\vv{nat}} \vv{n}).\]
\item
  Нека $\vv{a} = \vv{b} \to \vv{c}$. Тогава 
  \[f \triangleleft_{\vv{b}\to\vv{c}} \tau \dff (\forall e\in \val{\vv{b}})(\forall \mu \in \vv{PCF}_{\vv{b}})[\ e \triangleleft_{\vv{b}} \mu \implies f(e) \triangleleft_{\vv{c}} \tau(\mu)\ ].\]
\item
  Нека $\Gamma = \vv{x}_1:\vv{a}_1, \dots, \vv{x}_n:\vv{a}_n$. Тогава 
  \[(u_1,\dots,u_n) \triangleleft_\Gamma (\tau_1,\dots,\tau_n) \dff u_1 \triangleleft_{\vv{a}_1} \tau_1\ \&\ \cdots\ \&\ u_n \triangleleft_{\vv{a}_n} \tau_n.\]
\end{itemize}

\begin{example}
  Да проверим внимателно защо е изпълнено, че:
  \[\texttt{id}_{\val{\nat}} \triangleleft_{\vv{nat}\to\vv{nat}} \lamb{x}{nat}{\vv{x + 0}}.\]
  Според дефиницията трябва да проверим импликацията
  \[e \triangleleft_{\nat} \mu \implies \texttt{id}_{\val{\nat}}(e) \triangleleft_{\nat} \tau(\mu),\]
  за произволен елемент $e \in \Nat_\bot$ и произволен затворен терм $\mu : \nat$.
  \marginpar{Аналогично можем да видим, че $\texttt{id}_{\val{\nat}} \triangleleft_{\nat} \lamb{x}{nat}{x}$.}
  \begin{itemize}
  \item
    Ако $e = \bot$, то от дефиницията на $\triangleleft_{\nat}$ ведната следва, че за произволен затворен терм $\mu : \nat$, то
    $\bot \triangleleft_{\nat} \mu$. Понеже $\texttt{id}_{\val{\nat}}(\bot) = \bot$, то отново от дефиницията веднага следва, че
    $\bot \triangleleft_{\nat} \tau(\mu)$, за произволен затворен терм $\mu : \nat$.
  \item
    Нека $e = n\in\Nat$ и да разгледаме затворен терм $\mu : \nat$, за който $n \triangleleft_{\nat} \mu$.
    Според дефиницията на $\triangleleft_{\nat}$, това означава, че $\mu \opsem{}{nat} \vv{n}$.
    Сега да видим защо $\texttt{id}_{\val{\nat}}(n) = n \triangleleft_{\nat} \tau(\mu)$ или с други думи,
    трябва да проверим, че $\tau(\mu) \opsem{}{nat} \vv{n}$. Тук се позоваваме на правилата от операционната семантика:
    \begin{prooftree}
      \AxiomC{$\tau$ е стойност}
      \LeftLabel{\scriptsize{(val)}}
      \UnaryInfC{$\tau \opsemGen{}{\nat\to\nat} \lamb{x}{nat}{\vv{x + 0}}$}
      \AxiomC{$n \triangleleft_{\nat} \mu$}
      \UnaryInfC{$\mu \opsem{}{nat} \vv{n}$}
      \AxiomC{$\vv{0}$ е стойност}
      \RightLabel{\scriptsize{(val)}}
      \UnaryInfC{$\vv{0} \opsem{}{nat} \vv{0}$}
      \RightLabel{\scriptsize{(plus)}}
      \BinaryInfC{$\vv{(x+0)}\subst{x}{\mu} \opsem{}{nat} \vv{n}$}
      \RightLabel{\scriptsize{(app)}}
      \BinaryInfC{$\tau(\mu) \opsem{}{nat} \vv{n}$}
    \end{prooftree}
  \end{itemize}

\end{example}

\begin{problem}
  Нека положим $\tau \equiv \lamb{x}{nat}{\lamb{y}{nat}{x-y}}$.
  Проверете, че $f \triangleleft_{\nat\to\nat\to\nat} \tau$, където:
  \begin{itemize}
  \item
    $f = \curry(\minus)$;
  \item
    $f(a)(b) =
    \begin{cases}
      a-b, & \text{ако }a \geq b\\
      \bot, & \text{иначе}
    \end{cases}$;
  \item
    $f(a)(b) = \bot$ за произволни $a,b\in\Nat_\bot$.
  \end{itemize}
\end{problem}


Нека първо да разгледаме някои основни свойства на релацията $\triangleleft_{\vv{a}}$.
Тук доказателствата протичат с индукция по построението на типовете.
\marginpar{\cite[стр. 197]{gunter}}

\begin{proposition}\label{pr:pcf:adequacy:bottom}
  За всеки тип $\vv{a}$ и всеки затворен терм $\tau : \vv{a}$ е изпълнено, че $\bot^{\val{\vv{a}}} \triangleleft_{\vv{a}} \tau$.
\end{proposition}
\begin{proof}
  Индукция по построението на типовете $\vv{a}$.
  Първо, нека $\vv{a} = \vv{nat}$. По тривиални съображения имаме, че за произволен терм $\tau:\vv{a}$ е изпълнено, че $\bot^{\val{\vv{nat}}} \triangleleft_{\vv{nat}} \tau$.
  
  Второ, нека $\vv{a} = \vv{b} \to \vv{c}$ и да фиксираме произволен терм $\tau : \vv{b} \to \vv{c}$.
  Тук имаме, че $\bot^{\val{\vv{a}}} \in \Cont{\val{\vv{b}}}{\val{\vv{c}}}$ е изображение,
  за което $\bot^{\val{\vv{a}}}(e) =  \bot^{\val{\vv{c}}}$ за всеки елемент $e \in \val{\vv{b}}$.
  Нека $e \triangleleft_{\vv{b}} \mu$, където $\mu : \vv{b}$.
  Щом $\tau : \vv{b}\to\vv{c}$, от правилата за типизиране е ясно, че $\tau(\mu) : \vv{c}$.
  Сега от \IndHyp за типа $\vv{c}$ е ясно, че $\bot^{\val{\vv{a}}}(e) = \bot^{\val{\vv{c}}} \triangleleft_{\vv{c}} \tau(\mu)$.
\end{proof}


\begin{proposition}\label{pr:pcf:adequacy:chain}
  Нека за произволен тип $\vv{a}$ и произволен терм $\tau:\vv{a}$ да разгледаме множеството $D \df \{d \in \val{\vv{a}} \mid d \triangleleft_{\vv{a}} \tau\}$.
  Тогава ако $\chain{d}{i}$ е верига от елементи на $D$, то $\bigsqcup_i d_i$ също принадлежи на $D$.
\end{proposition}
\begin{proof}
  Индукция по построението на типовете $\vv{a}$.
  Първо, нека $\vv{a} = \vv{nat}$.
  \marginpar{Да напомним, че $\val{\vv{nat}} = \Nat_\bot$. Ясно е, че всяка верига от елементи на $\Nat_\bot$ се стабилизира.}
  Нека $\chain{d}{i}$ е верига от елементи на $\Nat_\bot$ и за всеки индекс $i$, $d_i \triangleleft_{\vv{nat}} \tau$.
  Ако за всяко $i$, $d_i = \bot$, то $\bigsqcup_i d_i = \bot^{\val{\vv{nat}}}$ и следователно $\bigsqcup_i d_i
  \triangleleft_{\vv{nat}} \tau$.
  Ако съществува индекс $i_0$, за който $d_{i_0} = n \neq \bot$, то е ясно, че за всяко $i > i_0$, $d_i = n$.
  Оттук следва, че $\bigsqcup_i d_i = n = d_{i_0}$.
  Понеже $d_{i_0} \triangleleft_{\vv{nat}} \tau$, то директно следва, че $\bigsqcup_i d_i \triangleleft_{\vv{nat}} \tau$.

  Второ, нека $\vv{a} = \vv{b} \to \vv{c}$ и да фиксираме произволен терм $\tau : \vv{b} \to \vv{c}$.
  Нека $\chain{f}{i}$ е верига от елементи на $\Cont{\val{\vv{b}}}{\val{\vv{c}}}$,
  за които е изпълнено, че $f_i \triangleleft_{\vv{a}} \tau$. Трябва да докажем, че $\bigsqcup_i f_i \triangleleft_{\vv{a}} \tau$,
  т.е. за произволен елемент $e \in \val{\vv{b}}$ и произволен затворен терм $\mu : \vv{b}$, за който $e \triangleleft_{\vv{b}} \mu$, то
  $(\bigsqcup_if)(e) \triangleleft_{\vv{c}} \tau(\mu)$.
  Но ние знаем от \Lem{double-chain:lub}, че $(\bigsqcup_if)(e) = \bigsqcup_i\{f_i(e)\}$.
  Щом $f_i \triangleleft_{\vv{b}\to\vv{c}} \tau$, то за разглежданите $e$ и $\mu$ имаме, че $f_i(e) \triangleleft_{\vv{c}} \tau(\mu)$.
  Ние знаем, че ${(f_i(e))}^\infty_{i=0}$ е верига и от \IndHyp за типа $\vv{c}$ следва, че $\bigsqcup_i\{f_i(e)\} \triangleleft_{\vv{c}} \tau(\mu)$.
\end{proof}


\begin{proposition}\label{pr:pcf:adequacy:implication}
  Да разгледаме произволен тип $\vv{a}$ и произволен затворен терм $\tau : \vv{a}$.
  \marginpar{Да напомним, че с $\vv{v}$ означаваме термове стойности.}
  Тогава е изпълнено следното:
  \begin{prooftree}
    \AxiomC{$d \triangleleft_{\vv{a}} \tau$}
    \AxiomC{$(\forall \vv{v})[\tau \Downarrow_{\vv{a}} \vv{v} \implies \rho \Downarrow_{\vv{a}} \vv{v}]$}
    \BinaryInfC{$d \triangleleft_{\vv{a}} \rho$}
  \end{prooftree}
\end{proposition}
\begin{proof}
  Индукция по построението на типовете $\vv{a}$.
  Първо, нека $\vv{a} = \vv{nat}$. 
  Нека $d \triangleleft_{\vv{nat}} \tau$ и $(\forall \vv{v})[\tau \Downarrow_{\vv{nat}} \vv{v} \implies \rho \Downarrow_{\vv{nat}} \vv{v}]$.
  Понеже $\sqsubseteq$ е плоската наредба в $\Nat_\bot$, то имаме два случая.
  Ако $d = \bot^{\val{\vv{nat}}}$, то е ясно от \Prop{pcf:adequacy:bottom}, че $d \triangleleft_{\vv{nat}} \rho$.
  Нека сега $d \neq \bot^{\val{\vv{nat}}}$. 
  Понеже $d \triangleleft_{\vv{nat}} \tau$, то $\tau \Downarrow_{\vv{nat}} \vv{d}$.
  Оттук следва, че $\rho \Downarrow_{\vv{nat}} \vv{d}$. Заключаваме, че $d \triangleleft_{\vv{nat}} \rho$.

  Второ, нека $\vv{a} = \vv{b} \to \vv{c}$ и да фиксираме произволен терм $\tau : \vv{b} \to \vv{c}$.
  Нека
  \begin{align}
    & f \triangleleft_{\vv{b}\to\vv{c}} \tau \label{eq:pcf:adequacy:implication:f-tau}\\
    & (\forall \vv{v})[\tau \Downarrow_{\vv{b}\to\vv{c}} \vv{v} \implies \rho \Downarrow_{\vv{b}\to\vv{c}} \vv{v}] \label{eq:pcf:adequacy:implication:value}
  \end{align}
  % $f \triangleleft_{\vv{b}\to\vv{c}} \tau$ и $(\forall \vv{v})[\tau \Downarrow_{\vv{b}\to\vv{c}} \vv{v} \implies \rho \Downarrow_{\vv{b}\to\vv{c}} \vv{v}]$.
  Ще докажем, че $f \triangleleft_{\vv{b} \to \vv{c}} \rho$.
  За целта, да разгледаме произволен елемент $e \in \val{\vv{b}}$ и произволен затворен терм $\mu : \vv{b}$, за който $e \triangleleft_{\vv{b}} \mu$.
  Достатъчно е да докажем, че $f(e) \triangleleft_{\vv{c}} \rho(\mu)$.
  За момента от Свойство~(\ref{eq:pcf:adequacy:implication:f-tau}) знаем само, че $f(e) \triangleleft_{\vv{c}} \tau(\mu)$.
  Понеже имаме следното правило в операционната семантика:
  \marginpar{Имаме от (\ref{eq:pcf:adequacy:implication:value}), че 
    \[\tau \Downarrow_{\vv{b}\to\vv{c}} \vv{v} \implies \rho \Downarrow_{\vv{b}\to\vv{c}} \vv{v},\] а тук $\vv{v} \equiv \lamb{x}{b}{\tau'}$.}
  \begin{prooftree}
    \AxiomC{$\tau \Downarrow_{\vv{b}\to\vv{c}} \overbrace{\lamb{x}{b}{\tau'}}^{\vv{v}}$}
    \AxiomC{$\tau'\subst{x}{\mu} \Downarrow_{\vv{c}} \vv{v}'$}
    \RightLabel{\scriptsize{(app)}}
    \BinaryInfC{$\tau(\mu) \Downarrow_{\vv{c}} \vv{v}'$}
  \end{prooftree}
  то от Свойство~(\ref{eq:pcf:adequacy:implication:value}) получаваме, че
  \begin{prooftree}
    \AxiomC{$\rho \Downarrow_{\vv{b}\to\vv{c}} \overbrace{\lamb{x}{b}{\tau'}}^{\vv{v}}$}
    \AxiomC{$\tau'\subst{x}{\mu} \Downarrow_{\vv{c}} \vv{v}'$}
    \RightLabel{\scriptsize{(app)}}
    \BinaryInfC{$\rho(\mu) \Downarrow_{\vv{c}} \vv{v}'$}
  \end{prooftree}
  Оттук следва, че
  \begin{equation}
    \label{eq:pcf:adequacy:implication:final}
    (\forall \vv{v}')[\tau(\mu) \Downarrow_{\vv{c}} \vv{v}' \implies \rho(\mu) \Downarrow_{\vv{c}} \vv{v}'].
  \end{equation}
  Сега от \IndHyp за типа $\vv{c}$ директно следва, че щом $f(e) \triangleleft_{\vv{c}}\tau(\mu)$ и Свойство (\ref{eq:pcf:adequacy:implication:final}), то \[f(e) \triangleleft_{\vv{c}} \rho(\mu).\]
\end{proof}




% \begin{lemma}\label{lem:pcf:relation}
%   Нека $\tau : \vv{a}$. Тогава:
%   \begin{enumerate}[1)]
%   \item
%     $\bot^{\val{\vv{a}}} \triangleleft_{\vv{a}} \tau$;
%   \item
%     $D = \{d \in \val{\vv{a}} \mid d \triangleleft_{\vv{a}} \tau\}$ е непрекъснато свойство в областта на Скот $\val{\vv{a}}$, т.е.
%     за всяка верига $\chain{d}{i}$ от елементи на $D$ е изпълнено, че $\bigsqcup_i d_i \in D$.
%   \item
%     \marginpar{Не ми трябва $u \sqsubseteq d$.}
%     Ако $d \triangleleft_{\vv{a}} \tau$ и $(\forall \vv{v})[\tau \Downarrow_{\vv{a}} \vv{v} \implies \rho
%     \Downarrow_{\vv{a}} \vv{v}]$, то $d \triangleleft_{\vv{a}} \rho$.
%   \end{enumerate}
% \end{lemma}
% \begin{proof}
%   Индукция по построението на типовете $\vv{a}$.
%   Първо, нека $\vv{a} = \vv{nat}$ и да фиксираме произволен терм $\tau : \vv{nat}$.
%   \marginpar{Да напомним, че $\val{\vv{nat}} \df \Nat_\bot$.}
%   \begin{enumerate}[1)]
%   \item
%     По тривиални съображения имаме, че
%     \[\bot^{\val{\vv{nat}}} \triangleleft_{\vv{nat}} \tau.\]
%   \item
%     Нека $\chain{d}{i}$ е верига от елементи на $\Nat_\bot$ и за всяко $i$, $d_i \triangleleft_{\vv{nat}} \tau$.
%     Ако за всяко $i$, $d_i = \bot$, то $\bigsqcup_i d_i = \bot^{\val{\vv{nat}}}$ и следователно $\bigsqcup_i d_i
%     \triangleleft_{\vv{nat}} \tau$.
%     Ако съществува $i_0$, за което $d_{i_0} = n \neq \bot$, то е ясно, че за всяко $i > i_0$, $d_i = n$.
%     Оттук следва, че $\bigsqcup_i d_i = n = d_{i_0}$.
%     Понеже $d_{i_0} \triangleleft_{\vv{nat}} \tau$, то директно следва, че $\bigsqcup_i d_i \triangleleft_{\vv{nat}} \tau$.
%   \item
%     Нека $d \triangleleft_{\vv{nat}} \tau$ и $(\forall \vv{v})[\tau \Downarrow_{\vv{nat}} \vv{v} \implies \rho
%     \Downarrow_{\vv{nat}} \vv{v}]$. Понеже $\sqsubseteq$ е плоската наредба в $\Nat_\bot$, то имаме два случая.
%     Ако $d = \bot^{\val{\vv{nat}}}$, то е ясно от 1), че $d \triangleleft_{\vv{nat}} \rho$.
%     Нека сега $d \neq \bot^{\val{\vv{nat}}}$. 
%     Понеже $d \triangleleft_{\vv{nat}} \tau$, то $\tau \Downarrow_{\vv{nat}} \vv{d}$.
%     Оттук следва, че $\rho \Downarrow_{\vv{nat}} \vv{d}$. Заключаваме, че $d \triangleleft_{\vv{nat}} \rho$.    
%   \end{enumerate}
  
%   Второ, нека $\vv{a} = \vv{b} \to \vv{c}$ и да фиксираме произволен терм $\tau : \vv{b} \to \vv{c}$.
%   \marginpar{Да напомним, че \[\val{\vv{b} \to \vv{c}} \df \Cont{\val{\vv{b}}}{\val{\vv{c}}}.\]}
%   \begin{enumerate}[1)]
%   \item
%     Тук имаме, че $\bot^{\val{\vv{a}}} \in \Cont{\val{\vv{b}}}{\val{\vv{c}}}$ е изображение,
%     за което $\bot^{\val{\vv{a}}}(e) =  \bot^{\val{\vv{c}}}$ за всеки елемент $e \in \val{\vv{b}}$.
%     Нека $e \triangleleft_{\vv{b}} \mu$, където $\mu : \vv{b}$.
%     От правилата за типизиране е ясно, че $\tau(\mu) : \vv{c}$.
%     Сега от И.П. е ясно, че $\bot^{\val{\vv{a}}}(e) = \bot^{\val{\vv{c}}} \triangleleft_{\vv{c}} \tau(\mu)$.
%   \item
%     Нека $\chain{f}{i}$ е верига от елементи на $\Cont{\val{\vv{b}}}{\val{\vv{c}}}$,
%     за които е изпълнено, че $f_i \triangleleft_{\vv{a}} \tau$. Трябва да докажем, че $\bigsqcup_i f_i \triangleleft_{\vv{a}} \tau$,
%     т.е. за произволни $e \in \val{\vv{b}}$ и произволни $\mu : \vv{b}$, за които $e \triangleleft_{\vv{b}} \mu$, то
%     $(\bigsqcup_if)(e) \triangleleft_{\vv{c}} \tau(\mu)$.
%     Но ние знаем, че $(\bigsqcup_if)(e) = \bigsqcup_i\{f_i(e)\}$.
%     Щом $f_i \triangleleft_{\vv{b}\to\vv{c}} \tau$, то за разглежданите $e$ и $\mu$ имаме, че $f_i(e) \triangleleft_{\vv{c}} \tau(\mu)$.
%     Ние знаем, че ${(f_i(e))}^\infty_{i=0}$ е верига и от И.П. следва, че $\bigsqcup_i\{f_i(e)\} \triangleleft_{\vv{c}} \tau(\mu)$.
%   \item
%     Нека $f \triangleleft_{\vv{b}\to\vv{c}} \tau$ и $\tau \Downarrow_{\vv{b}\to\vv{c}} \vv{v} \implies \rho
%     \Downarrow_{\vv{b}\to\vv{c}} \vv{v}$. Ще докажем, че $f \triangleleft_{\vv{b} \to \vv{c}} \rho$.
%     За целта, нека $e \in \val{\vv{b}}$, $\mu : \vv{b}$ и $e \triangleleft_{\vv{b}} \mu$.
%     Ще докажем, че $f(e) \triangleleft_{\vv{c}} \rho(\mu)$.
%     За момента знаем само, че $f(e) \triangleleft_{\vv{c}} \tau(\mu)$.
%     Понеже имаме следното правило в операционната семантика:
%     \marginpar{Имаме по условие, че 
%       \[\tau \Downarrow_{\vv{b}\to\vv{c}} \vv{v} \implies \rho \Downarrow_{\vv{b}\to\vv{c}} \vv{v},\] а тук $\vv{v} \equiv \lamb{x}{b}{\tau'}$.}
%     \begin{prooftree}
%       \AxiomC{$\tau \Downarrow_{\vv{b}\to\vv{c}} \lamb{x}{b}{\tau'}$}
%       \AxiomC{$\tau'\subst{x}{\mu} \Downarrow_{\vv{c}} \vv{v}'$}
%       \RightLabel{\scriptsize{(app)}}
%       \BinaryInfC{$\tau(\mu) \Downarrow_{\vv{c}} \vv{v}'$}
%     \end{prooftree}
%     то получаваме, че
%     \begin{prooftree}
%       \AxiomC{$\rho \Downarrow_{\vv{b}\to\vv{c}} \lamb{x}{b}{\tau'}$}
%       \AxiomC{$\tau'\subst{x}{\mu} \Downarrow_{\vv{c}} \vv{v}'$}
%       \RightLabel{\scriptsize{(app)}}
%       \BinaryInfC{$\rho(\mu) \Downarrow_{\vv{c}} \vv{v}'$}
%     \end{prooftree}
%     Оттук следва, че
%     \[(\forall \vv{v}')[\tau(\mu) \Downarrow_{\vv{c}} \vv{v}' \implies \rho(\mu) \Downarrow_{\vv{c}} \vv{v}'].\]
%     Сега от И.П. директно следва, че $f(e) \triangleleft_{\vv{c}} \rho(\mu)$.
%   \end{enumerate}
% \end{proof}

За да докажем, че $\val{\tau} \triangleleft_{\vv{nat}} \tau$, то трябва да докажем, че за всеки тип $\vv{a}$ и всеки затворен терм $\tau$ от тип $\vv{a}$, че е изпълнено
$\val{\tau} \triangleleft_{\vv{a}} \tau$ с индукция по построението на термовете.
Тук обаче имаме проблем. Ако $\tau \equiv \lamb{y}{b}{\tau_1}$, то трябва да използваме индукционно предположение за $\tau_1$,
в който вече има свободна променлива $\vv{y}$. Поради тази причина, ние трябва да разгледаме едно по-общо твърдение, при което позволяваме в термовете да се срещат свободни променливи.

\begin{framed}
  \begin{lemma}[Фундаментално свойство на $\triangleleft_{\vv{a}}$]\label{lem:pcf:fundamental}
    Нека $\Gamma = \vv{x}_1:\vv{a}_1,\dots,\vv{x}_n:\vv{a}_n$. Тогава
    \begin{prooftree}
      \AxiomC{$\Gamma \vdash \tau : \vv{a}$}
      \AxiomC{$(u_1,\dots,u_n) \triangleleft_\Gamma (\mu_1,\dots,\mu_n)$}
      \BinaryInfC{$\val{\tau}_\Gamma(\ov{u}) \triangleleft_{\vv{a}} \tau[\ov{\vv{x}}/\ov{\mu}]$}
    \end{prooftree}
  \end{lemma}  
\end{framed}
\begin{proof}
  Индукция по построението на термовете.
  \begin{itemize}
  \item
    Нека $\tau \equiv \vv{n}$. Тук директно от дефиницията на релацията $\triangleleft_{\vv{nat}}$ имаме, че
    \[n \triangleleft_{\vv{nat}} \vv{n}.\]
  \item
    \marginpar{Понеже $\Gamma \vdash \tau : \vv{a}$, то няма как $\tau$ да е променлива, която да не е някоя измежду $\vv{x}_1,\dots,\vv{x}_n$.}
    Нека $\tau \equiv \vv{x}_i$. Този случай също е много лесен.
    Имаме, че $\tau[\ov{x}/\ov{\mu}] \equiv \mu_i$ и $\val{\tau}_\Gamma(\ov{u}) = u_i$.
    Тогава, понеже $(u_1,\dots,u_n) \triangleleft_\Gamma (\mu_1,\dots,\mu_n)$,
    то директно получаваме, че
    \[\val{\tau}_\Gamma(\ov{u}) \triangleleft_{\vv{a}} \tau[\ov{x}/\ov{\mu}].\]
  \item
    Нека $\tau \equiv \tau_1 + \tau_2$. Тогава $\vv{a} = \vv{nat}$. Имаме също, че
    \begin{align*}
      & \tau[\ov{x}/\ov{\mu}] \equiv \tau_1[\ov{x}/\ov{\mu}] + \tau_2[\ov{x}/\ov{\mu}];\\
      & \val{\tau}_\Gamma(\ov{u}) = \plus(\underbrace{\val{\tau_1}_\Gamma(\ov{u})}_{n_1},\underbrace{\val{\tau_2}_\Gamma(\ov{u})}_{n_2}) = n.
    \end{align*}
    Можем да приемем, че $n \neq \bot$, защото в противен случай този случай е тривиален заради дефиницията на $\triangleleft_{\vv{nat}}$.
    От \IndHyp имаме, че $\val{\tau_1}_\Gamma(\ov{u}) \triangleleft_{\vv{nat}} \tau_1[\ov{x}/\ov{\mu}]$
    и $\val{\tau_2}_\Gamma(\ov{u}) \triangleleft_{\vv{nat}} \tau_2[\ov{x}/\ov{\mu}]$.
    Това означава, че $\tau_1[\ov{x}/\ov{\mu}] \Downarrow_{\vv{nat}} \vv{n}_1$ и $\tau_2[\ov{x}/\ov{\mu}] \Downarrow_{\vv{nat}}
    \vv{n}_2$.
    От правилата на операционната семантика е ясно, че $\tau[\ov{x}/\ov{\mu}] \Downarrow_{\vv{nat}} \vv{n}$ и следователно
    \[\val{\tau}_\Gamma(\ov{u}) \triangleleft_{\vv{a}} \tau[\ov{\vv{x}}/\ov{\mu}].\]
  \item
    Нека $\tau \equiv \tau_1\ \vv{-}\ \tau_2$. 
  \item
    Нека $\tau \equiv \tau_1\ \vv{==}\ \tau_2$. 
  \item
    Нека $\tau \equiv \ifelse{\tau_1}{\tau_2}{\tau_3}$.
  \item
    Нека $\tau \equiv \tau_1\tau_2$. От правилата за типизиране имаме, че
    \begin{prooftree}
      \AxiomC{$\Gamma \vdash \tau_1 : \vv{b} \to \vv{a}$}
      \AxiomC{$\Gamma \vdash \tau_2 : \vv{b}$}
      \BinaryInfC{$\Gamma \vdash \tau_1\tau_2 : \vv{a}$}
    \end{prooftree}
    Да напомним, че
    \[\val{\tau_1\tau_2}_\Gamma(\ov{u}) \df \texttt{eval}(\val{\tau_1}_\Gamma(\ov{u}), \val{\tau_2}_\Gamma(\ov{u})).\]
    От \IndHyp имаме следното:
    \begin{align*}
      & \val{\tau_1}_\Gamma(\ov{u}) \triangleleft_{\vv{b}\to\vv{a}} \tau_1[\ov{x}/\ov{\mu}];\\
      & \val{\tau_2}_\Gamma(\ov{u}) \triangleleft_{\vv{b}} \tau_2[\ov{x}/\ov{\mu}].
    \end{align*}
    Тогава директно следва, че
    \[\texttt{eval}(\val{\tau_1}_\Gamma(\ov{u}), \val{\tau_2}_\Gamma(\ov{u})) \triangleleft_{\vv{a}} \tau_1[\ov{x}/\ov{\mu}](\tau_2[\ov{x}/\ov{\mu}])\]
  \item
    Нека $\tau = \lamb{y}{b}{\tau'}$. Тогава от правилата за типизиране следва, че $\vv{a} = \vv{b} \to \vv{c}$ и
    $\Gamma' \vdash \tau' : \vv{c}$, където $\Gamma' = \Gamma, \type{y}{b}$.
    Да напомним, че
    \[\val{\tau}_\Gamma(\ov{u}) \df \texttt{curry}(\val{\tau'}_{\Gamma'})(\ov{u}) \in \Cont{\val{\vv{b}}}{\val{\vv{c}}}.\]
    Да положим $f \df \val{\tau}_\Gamma(\ov{u})$. Тогава $f(e) = \val{\tau'}_{\Gamma'}(\ov{u},e)$.
    
    Трябва да докажем, че $f \triangleleft_{\vv{b} \to \vv{c}} \tau[\ov{x}/\ov{\mu}]$.
    Това означава, че за произволни $e \in \val{\vv{b}}$ и $\rho : \vv{b}$, за които $e \triangleleft_{\vv{b}} \rho$,
    трябва да докажем, че $f(e) \triangleleft_{\vv{c}} \tau[\ov{x}/\ov{\mu}](\rho)$.
    Имаме, че
    \begin{prooftree}
      \AxiomC{$\Gamma' \vdash \tau' : \vv{c}$}
      \AxiomC{$(u_1,\dots,u_n,e) \triangleleft_{\Gamma'} (\mu_1,\dots,\mu_n,\rho)$}
      \RightLabel{\scriptsize{\IndHyp}}
      \BinaryInfC{$\val{\tau'}(\ov{u},e) \triangleleft_{\vv{c}} \tau'[\ov{x}/\ov{\mu}][y/\rho]$}
      \UnaryInfC{$f(e) \triangleleft_{\vv{c}} \tau'[\ov{x}/\ov{\mu}][y/\rho]$}
    \end{prooftree}
    От правилата на операционната семантика имаме следното:
    \begin{prooftree}
      \AxiomC{$\emptyset \vdash \rho : \vv{b}$}
      \AxiomC{$\tau'[\ov{x}/\ov{\mu}][y/\rho] \Downarrow_{\vv{c}} \vv{v}$}
      \BinaryInfC{$(\lamb{y}{b}{\tau'[\ov{x}/\ov{\mu}]})(\rho) \Downarrow_{\vv{c}} \vv{v}$}
      \UnaryInfC{$\tau[\ov{x}/\ov{\mu}](\rho) \Downarrow_{\vv{c}} \vv{v}$}
    \end{prooftree}
    От \Prop{pcf:adequacy:implication} веднага заключаваме, че $f(e) \triangleleft_{\vv{c}} \tau[\ov{x}/\ov{\mu}](\rho)$.
  \item
    Нека $\tau \equiv \fix(\tau')$. Тогава от правилата за типизиране имаме, че $\Gamma \vdash \tau' : \vv{a} \to \vv{a}$.
    От \IndHyp, приложено за $\tau'$, имаме, че
    \[\val{\tau'}_\Gamma(\ov{u}) \triangleleft_{\vv{a}\to\vv{a}} \tau'[\ov{x}/\ov{\mu}].\]
    Нека за улеснение да положим $f \df \val{\tau'}_\Gamma(\ov{u}) \in \Cont{\val{\vv{a}}}{\val{\vv{a}}}$.
    Да напомним, че
    \[\val{\fix(\tau')}_\Gamma(\ov{u}) = \lfp(f)\]
    От \Prop{pcf:adequacy:chain} знаем, че за всяка верига $\chain{d}{i}$ от елементи на
    \[D \df \{d \in \val{\vv{a}} \mid d \triangleleft_{\vv{a}} \fix(\tau'[\ov{x}/\ov{\mu}])\}\]
    е изпълнено, че $\bigsqcup_i d_i \in D$. Целта ни е да докажем, че
    $\lfp(f) \in D$. Да напомним, че $\lfp(f) = \bigsqcup_n f^n(\bot^{\val{\vv{a}}})$.
    С индукция по $n$ ще докажем, че за всяко $n$, $f^n(\bot^{\val{\vv{a}}}) \in D$.


    % Ще направим това като приложим правилото на Скот.
    % Да напомним, че
    % \begin{prooftree}
    %   \AxiomC{$\bot^{\val{\vv{a}}} \in D$}
    %   \AxiomC{$d \in D \implies f(d) \in D$}
    %   \BinaryInfC{$\lfp(f) \in D$}
    % \end{prooftree}
    За $n = 0$, понеже от \Prop{pcf:adequacy:bottom} имаме, че $f^0(\bot^{\val{\vv{a}}}) = \bot^{\val{\vv{a}}} \in D$.
    Нека $n > 0$ и от \IndHyp имаме, че $f^{n-1}(\bot^{\val{\vv{a}}}) \in D$.
    % нека вземем произволен елемент $d \in D$. Ще докажем, че $f(d) \in D$.

    Понеже $f \triangleleft_{\vv{a}\to\vv{a}} \tau'[\ov{x}/\ov{\mu}]$, то
    за произволно $e \triangleleft_{\vv{a}} \rho$ е изпълнено, че
    $f(e) \triangleleft_{\vv{a}} \tau'[\ov{x}/\ov{\mu}](\rho)$.
    Нека изберем $\rho = \fix(\tau'[\ov{x}/\ov{\mu}])$ и $e = f^{n-1}(\bot^{\val{\vv{a}}})$.
    Щом $f^{n-1}(\bot^{\val{\vv{a}}}) \in D$, то $f^{n-1}(\bot^{\val{\vv{a}}}) \triangleleft_{\vv{a}} \rho$ и следователно
    \[f(f^{n-1}(\bot^{\val{\vv{a}}})) \triangleleft_{\vv{a}} \tau'[\ov{x}/\ov{\mu}](\underbrace{\fix(\tau'[\ov{x}/\ov{\mu}])}_{\rho}).\]
    От правилата на операционната семантика имаме, че:
    \begin{prooftree}
      \AxiomC{$\tau'[\ov{x}/\ov{\mu}](\fix(\tau'[\ov{x}/\ov{\mu}])) \Downarrow_{\vv{a}} \vv{v}$}
      \RightLabel{\scriptsize{(fix)}}
      \UnaryInfC{$\fix(\tau'[\ov{x}/\ov{\mu}]) \Downarrow_{\vv{a}} \vv{v}$}
    \end{prooftree}
    Тогава от \Prop{pcf:adequacy:implication} следва, че
    \[f(f^{n-1}(\bot^{\val{\vv{a}}})) \triangleleft_{\vv{a}} \fix(\tau'[\ov{x}/\ov{\mu}]).\]
    Така доказахме, че $f^n(\bot^{\val{\vv{a}}}) \in D$.
    Заключаваме, че $\lfp(f) \in D$, т.е.
    \[\val{\fix(\tau')}_\Gamma(\ov{u}) \triangleleft_{\vv{a}} \fix(\tau'[\ov{x}/\ov{\mu}])\]
  \end{itemize}
\end{proof}

\begin{corollary}\label{cr:pcf:fundamental}
  За всеки тип $\vv{a}$ и за всеки затворен терм $\tau$ е изпълнено свойството:
  \[\tau \in \text{PCF}_{\vv{a}} \implies \val{\tau} \triangleleft_{\vv{a}} \tau.\]
\end{corollary}

Така на практика доказахме теоремата за адекватност.

\begin{framed}
  \begin{theorem}[Теорема за адекватност]\label{th:pcf:adequacy}
    За всеки затворен терм $\tau : \vv{nat}$, 
    \[\val{\tau} = n \neq \bot \implies \tau \Downarrow_{\vv{nat}} \vv{n}.\]
  \end{theorem}
\end{framed}
\begin{proof}
  Да разгледаме произволен затворен терм $\tau : \vv{nat}$.
  Нека $\val{\tau} = n \neq \bot$.
  От \Cor{pcf:fundamental} имаме, че $\val{\tau} \triangleleft_{\vv{nat}} \tau$.
  Тогава от дефиницията на $\triangleleft_{\vv{nat}}$ получаваме, че $\tau \Downarrow_{\vv{nat}} \vv{n}$.
\end{proof}

Да разгледаме термовете
\begin{align*}
  & \tau \equiv \lamb{x}{nat}{\vv{x + 0}}\\
  & \rho \equiv \lamb{x}{nat}{\vv{x}}.
\end{align*}


Ясно е, че $\val{\tau} = \val{\rho}$, но според правилата на операционната семантика, понеже $\tau$ е стойност, то
$\tau \not\Downarrow_{\vv{nat}\to\vv{nat}} \rho$.
Оттук веднага е ясно, че няма как да имаме теорема за адекватност за по-високи типове от $\vv{nat}$.


\begin{corollary}
  За всеки $\tau \in \text{PCF}_{\nat}$ е изпълнена импликацията:
  \[\tau \not\Downarrow_{\nat}\ \implies\ \val{\tau} = \bot^{\val{\nat}}.\]  
\end{corollary}
\begin{proof}
  Да допуснем, че $\tau \not\Downarrow_{\nat}$, но $\val{\tau} = n \neq \bot$.
  Но тогава от \hyperref[th:pcf:adequacy]{теоремата за адекватност} следва, че $\tau \Downarrow_{\nat} \vv{n}$, което е противоречие.
\end{proof}

\begin{framed}
  \begin{corollary}
    Нека $\tau$ е затворен терм от тип $\nat$. Тогава 
    \[\val{\tau} = \bot^{\val{\nat}} \text{ точно тогава, когато } \tau \not\Downarrow_{\nat}.\]
  \end{corollary}
\end{framed}

%%% Local Variables:
%%% mode: latex
%%% TeX-master: "../sep"
%%% End:

% \newpage
% \section{Контексти}\label{pcf:sect:context}

\index{контекст}
\marginpar{\cite[Глава 6.1]{gunter}}
\marginpar{\cite[Глава 48]{practical-foundations}}
\marginpar{Интуитивно, контекстите са програмни фрагменти. Не са пълни програми, защото имат празни места означени с $-$.}
\[\C ::= -\ |\ \vv{n}\ |\ \vv{x}\ |\ \C + \C\ |\ \C\ \vv{==}\ \C\ |\ \ifelse{\C}{\C}{\C}\ |\ \C\C\ |\ \lamb{x}{a}{\C}\ |\ \fix(\C).\]
Контекстите са на практика изрази, като позволяваме да имат специална свободна променлива, която означаваме с $-$.
% За произволен израз $\tau$, с $\C\{\tau\}$ означаваме израза $\C\rename{-}{\tau}$, където считаме $-$ като свободна променлива на $\C$.
% Това означава, че заместваме всички срещания на $-$ с $\tau$ като правим заместването директно, т.е.
% не се интересуваме дали някоя свободна променлива на $\tau$ няма да попадне под обхвата на свързана променлива в $\C$.
% Например, ако $\C = \lamb{x}{a}{-}$, то $\C\rename{-}{\vv{x}} =
% \lamb{x}{a}{\vv{x}}$.
За краткост, вместо $\C\{-/\tau\}$ ще пишем $\C[\tau]$.

\begin{proposition}
  % \marginpar{Да напомним, че с $\equiv$ означаваме релацията $\alpha$-еквивалентност.}
  Ако $\tau \equiv_\alpha \tau'$, то $\C[\tau] \equiv_\alpha \C[\tau']$.
\end{proposition}
\begin{hint}
  Индукция по построението на контекстите.
\end{hint}

\marginpar{Някои наричат тази релация observational equivalence. Тук наричаме релацията contextual equivalence.}
\begin{definition}\label{df:context:equivalence}
  За затворени термове $\tau_1$ и $\tau_2$, дефинираме
  $\tau_1 \leq_{ctx} \tau_2 : \vv{a}$, ако
  \begin{enumerate}[1)]
  \item
    $\emptyset \vdash \tau_1 : \vv{a}$ и $\emptyset \vdash \tau_2 : \vv{a}$
  \item
    За \emph{всички} контексти $C[-]$, за които $\emptyset \vdash C[\tau_1] : \nat$ и $\emptyset \vdash C[\tau_2] : \nat$, то
    \[C[\tau_1] \Downarrow_{\nat} \vv{n} \implies C[\tau_2] \Downarrow_{\nat} \vv{n}.\]
  \end{enumerate}
  Ще пишем $\tau_1 \cong_{ctx} \tau_2 : \vv{a}$, ако
  $\tau_1 \leq_{ctx} \tau_2 : \vv{a}$ и $\tau_2 \leq_{ctx} \tau_1 : \vv{a}$.
\end{definition}


% \begin{itemize}
% \item 
  % Ще пишем $\tau_1 \cong_{ctx} \tau_2 : \vv{a}$, ако
  % $\tau_1 \leq_{ctx} \tau_2 : \vv{a}$ и $\Gamma \vdash \tau_2 \leq_{ctx} \tau_1 : \vv{a}$.
% \item
%   Ако $\Gamma = \emptyset$, то ще пишем $\tau_1 \cong_{ctx} \tau_2 : \vv{a}$ вместо $\emptyset \vdash \tau_1 \cong_{ctx} \tau_2 : \vv{a}$.  
% \end{itemize}

\marginpar{Two phrases of a programming language are contextually equivalent if any occurrences of the first phrase in a complete
  program can be replaced by the second phrase without affecting the observable results of executing the program.
  This kind of program equivalence is also known as operational, or observational equivalence.}

\begin{framed}
  Денотационната семантика $\val{.}$ се нарича {\bf напълно абстрактна}, ако денотационната и операционната наредба съвпадат, т.е. $\val{\tau_1} \sqsubseteq \val{\tau_2}$ точно тогава, когато
  $\tau_1 \leq_{ctx} \tau_2$.  Един от основните ранни резултати в изучаването на семантиката на езици за програмиране е, че нашата денотационна семантика не е напълно абстрактна.
\end{framed}

Практически, ако имаме два терма, които са контекстно еквивалентни, то можем да заменим единия с другия в произволна програма,
без да има видими разлики на ниво изпълнение на програмата.
% Това представлява опит да се формализира математически практиката за тестване на програми.
Проблемът е, че с този формализъм е трудно да се работи, защото в дефиницията имаме квантор
за всеобщност относно всички контексти (програмни фрагменти).

\begin{proposition}\label{pr:pcf:context:simple}
  За произволни затворени термове $\tau_1$ и $\tau_2$ е изпълнено, че:
  \begin{enumerate}[(1)]
  \item
    $\tau_1 \leq_{ctx} \tau_2 : \nat \implies (\forall \vv{n})[\tau_1\Downarrow_{\nat}\vv{n} \implies \tau_2 \Downarrow_{\nat} \vv{n}]$.
  \item
    $\tau_1 \leq_{ctx} \tau_2 : \vv{b}\to\vv{c} \implies (\forall \rho\in\text{PCF}_{\vv{b}})[\tau_1\rho \leq_{ctx} \tau_2 \rho : \vv{c}]$.
  \end{enumerate}
\end{proposition}
\begin{hint}
  \begin{enumerate}[(1)]
  \item
    Просто вземете контекст $\C \df -$. Ясно е, че $\C[\tau_i] \equiv \tau_i$.
  \item
    Да фиксираме произволен терм $\rho\in\text{PCF}_{\vv{b}}$.
    Да разгледаме произволен контекст $\C[-]$, за който $\C[\tau_1\rho]$ и $\C[\tau_2\rho]$
    са затворени термове от тип $\nat$.
    Разглеждаме контекста $\C' \df \C\rename{-}{(-\rho)}$,
    т.е. заменяме $-$ с $-\rho$ в контекста.
    Лесно се съобразява, че \[\C'[\tau_i] \equiv \C[\tau_i\rho].\]
    Понеже $\tau_1 \leq_{ctx} \tau_2 : \vv{b}\to\vv{c}$,
    то за контекста $\C'$ имаме, че
    \[\C'[\tau_1] \Downarrow_{\nat} \vv{n} \implies \C'[\tau_2] \Downarrow_{\nat} \vv{n}.\]
    Оттук веднага следва, че 
    \[\C[\tau_1\rho] \Downarrow_{\nat} \vv{n} \implies \C[\tau_2\rho] \Downarrow_{\nat} \vv{n}.\]
  \end{enumerate}
\end{hint}

\begin{proposition}\label{pr:pcf:context:terms}
  За произволни затворени термове $\tau_1$ и $\tau_2$ е изпълнено, че:
  \[\tau_1 \leq_{ctx} \tau_2 : \vv{a} \iff (\forall \rho \in \text{PCF}_{\vv{a}\to\nat})[\ \rho\tau_1 \Downarrow_{\nat} \vv{n} \implies \rho\tau_2 \Downarrow_{\nat} \vv{n}\ ].\]
\end{proposition}
\begin{hint}
  За $(\Rightarrow)$, при даден терм $\rho \in \text{PCF}_{\vv{a}\to\nat}$, просто вземете контекст $\C~=~\rho~-$.
  За $(\Leftarrow)$, нека имаме контекст $\C[-]$.
  Нека $\vv{z}$ е променлива, която не се среща в $\C[-]$.
  Разгледайте $\rho \df \lamb{z}{a}{\C\rename{-}{\vv{z}}}$.
  Тогава, понеже $\tau_i$ са затворени термове, то
  $\C\rename{-}{\tau_i} \equiv \rho\tau_i$.
\end{hint}

\begin{proposition}\label{pr:pcf:context:relation}
  За произволни затворени термове $\tau_1$ и $\tau_2$ от тип $\vv{a}$,
  \[(d \triangleleft_{\vv{a}} \tau_1\ \&\ \tau_1 \leq_{ctx} \tau_2 : \vv{a}) \implies d \triangleleft_{\vv{a}} \tau_2.\]
\end{proposition}
\begin{proof}
  Индукция по построението на типовете $\vv{a}$.
  Нека $\vv{a} = \nat$.
  Нека $d \neq \bot$, защото е ясно, че $\bot \triangleleft_{\nat} \tau_2$.
  Понеже $d \triangleleft_{\nat} \tau_1$, то $\tau_1 \Downarrow_{\nat} \vv{d}$.
  Понеже $\tau_1 \leq_{ctx} \tau_2 : \nat$, то $\tau_2 \Downarrow_{\nat} \vv{d}$.
  Заключаваме, че $d \triangleleft_{\nat} \tau_2$.
  
  Нека $\vv{a} = \vv{b} \to \vv{c}$.
  Нека $d \triangleleft_{\vv{b} \to \vv{c}} \tau_1$ и $\tau_1 \leq_{ctx} \tau_2 : \vv{b}\to\vv{c}$.
  Трябва да докажем, че $d \triangleleft_{\vv{b}\to\vv{c}} \tau_2$.
  Нека $u \in \val{\vv{b}}$ и $\rho \in \text{PCF}_{\vv{b}}$,
  за които $u \triangleleft_{\vv{b}} \rho$. Трябва да докажем, че $d(u) \triangleleft_{\vv{c}} \tau_2\rho$.
  От $d \triangleleft_{\vv{b} \to \vv{c}} \tau_1$ имаме, че $d(u) \triangleleft_{\vv{c}} \tau_1 \rho$.
  От (2) на \Prop{pcf:context:simple} имаме, че $\tau_1 \rho \leq_{ctx} \tau_2 \rho : \vv{c}$.
  От И.П. заключаваме, че $d(u) \triangleleft_{\vv{c}} \tau_2 \rho$.
\end{proof}

\begin{proposition}\label{pr:pcf:context:relation-characterization}
  За произволни затворени термове $\tau_1$ и $\tau_2$,
  \[\tau_1 \leq_{ctx} \tau_2 : \vv{a} \iff \val{\tau_1} \triangleleft_{\vv{a}} \tau_2.\]
\end{proposition}
\begin{proof}
  $(\Rightarrow)$ Нека $\tau_1 \leq_{ctx} \tau_2 : \vv{a}$.
  От \Prop{pcf:adequacy:implication} имаме, че
  \[\val{\tau_1} \triangleleft_{\vv{a}} \tau_1.\]
  Тогава от предишното твърдение директно следва, че $\val{\tau_1} \leq_{\vv{a}} \tau_2$.

  $(\Leftarrow)$ Нека сега $\val{\tau_1} \triangleleft_{\vv{a}} \tau_2$.
  Тук ще използваме характеризацията на $\leq_{ctx}$ от \Prop{pcf:context:terms}.
  Да разгледаме произволен терм $\rho \in \text{PCF}_{\vv{a} \to \nat}$.
  Отново \Prop{pcf:adequacy:implication} имаме, че $\val{\rho} \triangleleft_{\vv{a} \to \nat} \rho$.
  Тогава от дефиницията на релацията $\triangleleft_{\vv{a}\to\nat}$ следва, че:
  \[\val{\rho\tau_1} = \val{\rho}(\val{\tau_1}) \triangleleft_{\nat} \rho\tau_2.\]
  Тогава:
  \begin{align*}
    \rho\tau_1 \Downarrow_{\nat} \vv{n} & \implies \val{\rho\tau_1} = n & \comment\text{\hyperref[th:pcf:soundness]{Теорема за коректност}}\\
                                            & \implies \rho\tau_2 \Downarrow_{\nat} \vv{n}. & \comment\val{\rho\tau_1} \triangleleft_{\nat} \rho\tau_2
  \end{align*}
  Заключаваме, че $\tau_1 \leq_{ctx} \tau_2 : \vv{a}$.
\end{proof}

\begin{proposition}\label{pr:pcf:context:extensionality}
  За произволни затворени термове $\tau_1$ и $\tau_2$ е изпълнено, че:
  \begin{enumerate}[(1)]
  \item
    $\tau_1 \leq_{ctx} \tau_2 : \nat \iff (\forall \vv{n})[\tau_1 \Downarrow_{\nat} \vv{n} \implies \tau_2 \Downarrow_{\nat} \vv{n}]$;
  \item
    $\tau_1 \leq_{ctx} \tau_2 : \vv{a}\to\vv{b} \iff (\forall \rho \in \text{PCF}_{\vv{a}})[\ \tau_1\rho \leq_{ctx} \tau_2 \rho : \vv{b}\ ]$.
  \end{enumerate}
\end{proposition}
\begin{proof}
  \begin{enumerate}[(1)]
  \item
    Посоката $(\Rightarrow)$ следва директно от \Prop{pcf:context:simple}.
    За посоката $(\Leftarrow)$, понеже
    \begin{align*}
      \val{\tau_1} = \val{\vv{n}} & \implies \tau_1 \Downarrow_{\nat} \vv{n} & \comment\text{\hyperref[th:pcf:adequacy]{Теорема за адекватност}}\\
                                  & \implies \tau_2 \Downarrow_{\nat} \vv{n}, & \comment\text{от условието}
    \end{align*}
    то получаваме, че $\val{\tau_1} \triangleleft_{\nat} \tau_2$.
    Тогава от предишното твърдение директно имаме, че $\tau_1 \leq_{ctx} \tau_2 : \nat$.
  \item
    Посоката $(\Rightarrow)$ следва директно от дефиницията на $\leq_{ctx}$.
    За посоката $(\Leftarrow)$, според предишното твърдение,
    достатъчно е да докажем, че $\val{\tau_1} \triangleleft_{\vv{a}\to\vv{b}} \tau_2$.
    Нека $u \in \val{\vv{a}}$ и $\rho \in \text{PCF}_{\vv{a}}$, за които $u \triangleleft_{\vv{a}} \rho$.
    Ще докажем, че $\val{\tau_1}(u) \triangleleft_{\vv{b}} \tau_2\rho$.
    Понеже $\val{\tau_1} \triangleleft_{\vv{a}\to\vv{b}} \tau_1$, то
    $\val{\tau_1}(u) \triangleleft_{\vv{b}} \tau_1\rho$.
    Но понеже $\tau_1\rho \leq_{ctx} \tau_2 \rho : \vv{b}$, то от \Prop{pcf:context:relation} следва, че
    $\val{\tau_1}(u) \triangleleft_{\vv{b}} \tau_2 \rho$.
  \end{enumerate}
\end{proof}

Естсвено е да се запитаме дали можем да разширим (1) на \Prop{pcf:context:extensionality} за по-сложни от $\nat$ типове $\vv{a}$, т.е. възможно ли е
\[\tau_1 \leq_{ctx} \tau_2 : \vv{a} \iff (\forall \vv{v} : \vv{a})[\tau_1 \Downarrow_{\vv{a}} \vv{v} \implies \tau_2 \Downarrow_{\vv{a}} \vv{v}]?\]
Първо в \Prop{context:op-left-right} ще видим, че винаги имаме импликацията $(\Leftarrow)$, но по-късно в \Prop{context:op-right-left} ще видим, че дори за типа $\vv{a} = \nat\to\nat$ нямаме импликация $(\Rightarrow)$.

\begin{proposition}\label{pr:context:op-left-right}
  Докажете, че за всеки два затворени терма $\tau_1, \tau_2 \in \text{PCF}_{\vv{a}}$,
  \marginpar{$(\forall \vv{v}:\vv{a})$ означава за всяка стойност $\vv{v}$ от тип $\vv{a}$.}
  \[(\forall \vv{v}:\vv{a})[\tau_1 \Downarrow_{\vv{a}} \vv{v} \implies \tau_2 \Downarrow_{\vv{a}} \vv{v} ] \implies \tau_1 \leq_{ctx} \tau_2 : \vv{a}.\]
\end{proposition}
\begin{hint}
  Според \Prop{pcf:context:relation-characterization}, достатъчно е да докажем, че $\val{\tau_1} \triangleleft_{\vv{a}} \tau_2$.
  Но това е лесно.
  От условието имаме, че $(\forall \vv{v}:\vv{a})[\tau_1 \Downarrow_{\vv{a}} \vv{v} \implies \tau_2 \Downarrow_{\vv{a}} \vv{v}]$.
  Знаем, че $\val{\tau_1} \triangleleft_{\vv{a}} \tau_1$.
  Тогава \Prop{pcf:adequacy:implication} следва, че $\val{\tau_1} \triangleleft_{\vv{a}} \tau_2$.
\end{hint}

\begin{proposition}\label{pr:context:den-left-right}
  За произволни затворени термове $\tau_1, \tau_2$ от тип $\vv{a}$,
  \[\val{\tau_1} \sqsubseteq \val{\tau_2} \implies \tau_1 \leq_{ctx} \tau_2 : \vv{a}.\]
\end{proposition}  
\begin{proof}
  Според \Prop{pcf:context:terms}, достатъчно е да докажем, че за произволен терм $\rho \in \text{PCF}_{\vv{a}\to\nat}$,
  $\rho\tau_1 \Downarrow_{\nat} \vv{n} \implies \rho\tau_2 \Downarrow_{\nat} \vv{n}$.
  \begin{align*}
    \rho\tau_1 \Downarrow_{\nat} \vv{n} & \implies \val{\rho\tau_1} = \val{\vv{n}} & \comment\text{\hyperref[th:pcf:soundness]{Теорема за коректност}}\\
                                            & \implies \val{\rho}(\val{\tau_1}) = \val{\vv{n}} & \comment\text{\hyperref[lem:pcf:substitution]{Лема за замяната}}\\
                                            & \implies \val{\rho}(\val{\tau_2}) = \val{\vv{n}} & \comment\text{монотонност на }\val{\rho}\\
                                            & \implies \val{\rho\tau_2} = \val{\vv{n}} & \comment\text{\hyperref[lem:pcf:substitution]{Лема за замяната}}\\
                                            & \implies \rho\tau_2 \Downarrow_{\nat} \vv{n}. & \comment\text{\hyperref[th:pcf:adequacy]{Теорема за адекватност}}
  \end{align*}
\end{proof}

\begin{framed}
  \begin{theorem}\label{th:pcf:context:connection}
    За всички затворени термове $\tau_1$ и $\tau_2$ от тип $\vv{a}$ е изпълнено, че:
    \begin{enumerate}[(1)]
    \item 
      $(\forall \vv{v}:\vv{a})[\tau_1 \Downarrow_{\vv{a}} \vv{v} \iff \tau_2 \Downarrow_{\vv{a}} \vv{v} ] \implies \tau_1 \cong_{ctx} \tau_2 : \vv{a}$;
    \item
      $\val{\tau_1} = \val{\tau_2} \implies \tau_1 \cong_{ctx} \tau_2 : \vv{a}$.
    \end{enumerate}
  \end{theorem}
\end{framed}

От \Prop{pcf:context:extensionality} и от \hyperref[th:pcf:soundness]{теоремата за коректност} имаме и обратните импликации за типа $\nat$.
За съжаление, ще видим, че нямаме обратните импликации за по-високи типове от $\nat$.


\begin{proposition}\label{pr:context:op-right-left}
  $\Omega'_{\nat} \cong_{ctx} \Omega_{\nat\to\nat} : \nat \to \nat$.
\end{proposition}
\begin{proof}
  Ще използваме (2) на \Prop{pcf:context:extensionality}.
  Първо да разгледаме посоката $(\Rightarrow)$. За произволно $\rho:\nat$, ще докажем, че $\Omega'_{\nat}\rho \leq_{ctx} \Omega_{\nat\to\nat}\rho : \nat$,
  което според (1) на \Prop{pcf:context:extensionality} означава да докажем, че
  \[(\forall \vv{n})[\ \Omega'_{\nat}\rho \opsem{}{nat} \vv{n} \implies \Omega_{\nat\to\nat} \rho \opsem{}{nat} \vv{n}\ ].\]
  Да отбележим, че $\Omega'_{\nat} \rho \not\opsem{}{nat}$, защото
  \begin{prooftree}
    \AxiomC{$\Omega'_{\nat}$ е стойност}
    \UnaryInfC{$\Omega'_{\nat} \opsem{0}{nat} \Omega'_{\nat}$}
    \AxiomC{$\Omega_{\nat} \not\opsem{}{nat}$}
    \UnaryInfC{$\Omega_{\nat}\subst{x}{\rho} \not\opsem{}{nat}$}
    \BinaryInfC{$\Omega'_{\nat} \rho \not\opsem{}{nat}$}
  \end{prooftree}
  Тогава заключаваме, че за всяко такова $\rho$,
  \[(\forall \vv{n})[\ \Omega'_{\nat}\rho \Downarrow_{\nat} \vv{n} \implies \Omega_{\nat\to\nat} \rho \Downarrow_{\nat} \vv{n}\ ].\]
  Сега да разгледаме посоката $(\Leftarrow)$.
  Тук правим сходни разсъждения, защото понеже $\Omega_{\nat\to\nat} \not\Downarrow_{\nat\to\nat}$, то и $\Omega_{\nat\to\nat} \rho \not\opsem{}{nat}$.
\end{proof}


% \begin{remark}
Знаем, че $\Omega_{\nat\to\nat} \not\opsemGen{}{\nat\to\nat}$, но $\Omega'_{\nat} \opsemGen{}{\nat\to\nat} \Omega'_{\nat}$.
Това означава, че според горната задача термовете $\Omega_{\nat\to\nat}$ и $\Omega'_{\nat}$ ни дават пример кога нямаме обратната импликация в (1) на \Th{pcf:context:connection}.
Обърнете внимание, че $\val{\Omega_{\nat\to\nat}} = \val{\Omega'_{\nat}}$.
Следователно, трябва да продължим да търсим термове $\tau_1$ и $\tau_2$ от тип $\vv{a}$, за които $\val{\tau_1} \neq \val{\tau_2}$
и $\tau_1 \cong_{ctx} \tau_2 : \vv{a}$.
% \end{remark}


%%% Local Variables:
%%% mode: latex
%%% TeX-master: "../sep"
%%% End:

% \newpage
% \section{Езикът \texttt{PCF(bool)}}

\marginpar{Всъщност в \cite[Глава 4.1]{gunter} това е ,,истинската'' дефиниция на езика \PCF.}

\begin{itemize}
\item
  Типове
  \[\vv{a} ::= \bool\ |\ \nat\ |\ \vv{a}\to\vv{a}\]
\item
  Изрази
  \begin{align*}
    \tau ::=\ & \tru\ |\ \fls\ |\ \vv{n}\ |\ \vv{x}\ |\ \tau_1 + \tau_2\ |\ \tau_1 - \tau_2\ |\  \tau_1\ \vv{==}\ \tau_2\ |\\
              & \ifelse{\tau_1}{\tau_2}{\tau_3}\ |\ \tau_1\tau_2\ |\ \lamb{x}{a}{\tau_1}\ |\ \fix(\tau_1).
  \end{align*}
  % Термове
  % \begin{align*}
  %   \tau ::=\ & \vv{0}\ |\ \tru\ |\ \fls\ |\ \vv{x}\ |\\
  %            & \scc(\tau_1)\ |\ \prd(\tau_1)\ |\ \texttt{iszero}(\tau)\ |\ \ifelse{\tau_1}{\tau_2}{\tau_3}\ |\\
  %            &\tau_1\tau_2\ |\ \lamb{x}{a}{\tau_1}\ |\ \fix(\tau_1).
  % \end{align*}
\item
  Стойностите са затворени термове от следния вид:
  \[\vv{v} ::= \tru\ |\ \fls\ |\ \vv{n}\ |\ \lamb{x}{a}{\mu}\]
\end{itemize}

\subsection{Типизираща релация}

Релацията $\Gamma \vdash \tau : \vv{a}$ за езика \PCFBOOL е почти същата
както за езика \PCF. Имаме две нови правила:

\begin{figure}[H]
  \begin{subfigure}[b]{0.5\textwidth}
    \begin{prooftree}
      \AxiomC{}
      \RightLabel{\scriptsize{(true)}}
      \UnaryInfC{$\Gamma \vdash \tru : \bool$}
    \end{prooftree}
  \end{subfigure}
  ~
  \begin{subfigure}[b]{0.5\textwidth}
    \begin{prooftree}
      \AxiomC{}
      \RightLabel{\scriptsize{(false)}}
      \UnaryInfC{$\Gamma \vdash \fls : \bool$}
    \end{prooftree}
  \end{subfigure}
\end{figure}
Имаме и две променени правила:
\begin{prooftree}
  \AxiomC{$\Gamma \vdash \tau_1:\nat$}
  \AxiomC{$\Gamma \vdash \tau_2:\nat$}
  \RightLabel{\scriptsize{(eq)}}
  \BinaryInfC{$\Gamma \vdash \tau_1\ \vv{==}\ \tau_2 : \bool$}
\end{prooftree}
\begin{prooftree}
  \AxiomC{$\Gamma \vdash \tau_1:\bool$}
  \AxiomC{$\Gamma \vdash \tau_2:\vv{a}$}
  \AxiomC{$\Gamma \vdash \tau_3:\vv{a}$}
  \RightLabel{\scriptsize{(if)}}
  \TrinaryInfC{$\Gamma \vdash \ifelse{\tau_1}{\tau_2}{\tau_3} : \vv{a}$}
\end{prooftree}
Всички останали правила са същите.

% \end{subfigure}
% ~
% \begin{subfigure}[b]{0.5\textwidth}
% \begin{prooftree}
%   \AxiomC{$\Gamma \vdash \tau:\vv{a}\to\vv{a}$}
%   \RightLabel{\scriptsize{(fix)}}
%   \UnaryInfC{$\Gamma \vdash \fix(\tau) : \vv{a}$}
% \end{prooftree}
% \end{subfigure}

% \vspace{10pt}

% \begin{subfigure}[b]{0.5\textwidth}
% \begin{prooftree}
%   \AxiomC{$\Gamma \vdash \tau_1:\vv{a}\to\vv{b}$}
%   \AxiomC{$\Gamma \vdash \tau_2:\vv{a}$}
%   \RightLabel{\scriptsize{(app)}}
%   \BinaryInfC{$\Gamma \vdash \tau_1\tau_2 : \vv{b}$}
% \end{prooftree}
% \end{subfigure}
% ~
% \begin{subfigure}[b]{0.5\textwidth}
% \begin{prooftree}
%   \AxiomC{$\vv{x} \not\in\vv{dom}(\Gamma)$}
%   \AxiomC{$\Gamma, \type{x}{a} \vdash \tau:\vv{b}$}
%   \RightLabel{\scriptsize{(lambda)}}
%   \BinaryInfC{$\Gamma \vdash \lambda \type{x}{a}\ .\ \tau : \vv{a} \to \vv{b}$}
% \end{prooftree}
% \end{subfigure}

% \caption{Релация за типизиране на термовете от езика \texttt{PCF++}}
% \label{fig:pcf:extensions:relation}
% \end{figure}


\subsection{Операционна семантика}

% \marginpar{\cite[стр. 109]{gunter}}

Тук отново всичко е почти същото както преди със следните разлики:

\begin{figure}[H]
  % \begin{subfigure}[b]{0.5\textwidth}
  %   \begin{prooftree}
  %     \AxiomC{$\type{v}{a}$}
  %     \RightLabel{\scriptsize{(val)}}
  %     \UnaryInfC{$\vv{v} \Downarrow^0_{\vv{a}} \vv{v}$}
  %   \end{prooftree}    
  % \end{subfigure}
  % ~
  % \begin{subfigure}[b]{0.5\textwidth}
  %   \begin{prooftree}
  %     \AxiomC{$\tau \Downarrow^\ell_{\nat} \vv{0}$}
  %     \UnaryInfC{$\prd(\tau) \Downarrow^{\ell+1}_{\nat} \vv{0}$}
  %   \end{prooftree}
  % \end{subfigure}

  % \vspace{10pt}

  % \begin{subfigure}[b]{0.5\textwidth}
  %   \begin{prooftree}
  %     \AxiomC{$\tau \Downarrow^\ell_{\nat} \scc(\vv{v})$}
  %     \UnaryInfC{$\prd(\tau) \Downarrow^{\ell+1}_{\nat} \vv{v}$}
  %   \end{prooftree}
  % \end{subfigure}
  % ~
  % \begin{subfigure}[b]{0.5\textwidth}
  %   \begin{prooftree}
  %     \AxiomC{$\tau \Downarrow^\ell_{\nat} \vv{v}$}
  %     \UnaryInfC{$\scc(\tau) \Downarrow^{\ell+1}_{\nat} \scc(\vv{v})$}
  %   \end{prooftree}
  % \end{subfigure}

  % \vspace{10pt}

  % \begin{subfigure}[b]{0.5\textwidth}
  %   \begin{prooftree}
  %     \AxiomC{$\tau \Downarrow^\ell_{\nat} \vv{0}$}
  %     \UnaryInfC{$\iszero(\tau) \Downarrow^{\ell+1}_{\bool} \tru$}
  %   \end{prooftree}
  % \end{subfigure}
  % ~
  % \begin{subfigure}[b]{0.5\textwidth}
  %   \begin{prooftree}
  %     \AxiomC{$\tau \Downarrow^\ell_{\nat} \scc(\vv{v})$}
  %     \UnaryInfC{$\iszero(\tau) \Downarrow^{\ell+1}_{\bool} \fls$}
  %   \end{prooftree}
  % \end{subfigure}

  % \vspace{10pt}
  
  \begin{subfigure}[b]{0.5\textwidth}
    \begin{prooftree}
      \AxiomC{$\tau_1 \opsem{\ell_1}{nat} \vv{v}_1$}
      \AxiomC{$\tau_3 \opsem{\ell_2}{a} \vv{v}_2$}
      \AxiomC{$\vv{v}_1 \equiv \vv{v}_2$}
      % \RightLabel{\scriptsize{(if$_\fls$)}}
      \TrinaryInfC{$\tau_1\ \vv{==}\ \tau_2 \opsem{\ell_1+\ell_2+1}{bool} \tru$}
    \end{prooftree}
  \end{subfigure}
  ~
  \begin{subfigure}[b]{0.5\textwidth}
    \begin{prooftree}
      \AxiomC{$\tau_1 \opsem{\ell_1}{nat} \vv{v}_1$}
      \AxiomC{$\tau_3 \opsem{\ell_2}{a} \vv{v}_2$}
      \AxiomC{$\vv{v}_1 \not\equiv \vv{v}_2$}
      % \RightLabel{\scriptsize{(if$_\fls$)}}
      \TrinaryInfC{$\tau_1\ \vv{==}\ \tau_2 \opsem{\ell_1+\ell_2+1}{bool} \fls$}
    \end{prooftree}
  \end{subfigure}

  \vspace{10pt}
  
  \begin{subfigure}[b]{0.5\textwidth}
    \begin{prooftree}
      \AxiomC{$\tau_1 \opsem{\ell_1}{bool} \fls$}
      \AxiomC{$\tau_3 \opsem{\ell_2}{a} \vv{v}$}
      % \RightLabel{\scriptsize{(if$_\fls$)}}
      \BinaryInfC{$\ifelse{\tau_1}{\tau_2}{\tau_3} \opsem{\ell_1+\ell_2+1}{a} \vv{v}$}
    \end{prooftree}
  \end{subfigure}
  ~
  \begin{subfigure}[b]{0.5\textwidth}
    \begin{prooftree}
      \AxiomC{$\tau_1 \opsem{\ell_1}{bool} \tru$}
      \AxiomC{$\tau_2 \opsem{\ell_2}{a} \vv{v}$}
      % \RightLabel{\scriptsize{(if$_\tru$)}}
      \BinaryInfC{$\ifelse{\tau_1}{\tau_2}{\tau_3} \opsem{\ell_1+\ell_2+1}{a} \vv{v}$}
    \end{prooftree}
  \end{subfigure}

%   \vspace{10pt}

%   \begin{subfigure}[b]{0.5\textwidth}
%     \begin{prooftree}
%       \AxiomC{$\tau_1 \Downarrow^{\ell_1}_{\vv{a}\to\vv{b}} \lamb{x}{a}{\tau'_1}$}
%       \AxiomC{$\tau'_1[x/\tau_2] \Downarrow^{\ell_2}_{\vv{b}} \vv{v}$}
%       % \RightLabel{\scriptsize{(cbn)}}
%       \BinaryInfC{$\tau_1 \tau_2 \Downarrow^{\ell_1+\ell_2+1}_{\vv{b}} \vv{v} $}
%     \end{prooftree}
%   \end{subfigure}
%   ~
%   \begin{subfigure}[b]{0.5\textwidth}
%   \begin{prooftree}
%     \AxiomC{$\tau\ \fix(\tau) \Downarrow^{\ell}_{\vv{a}} \vv{v}$}
%     \RightLabel{\scriptsize{(fix)}}
%     \UnaryInfC{$\fix(\tau) \Downarrow^{\ell+1}_{\vv{a}} \vv{v} $}
%   \end{prooftree}
% \end{subfigure}
% \caption{Правила на операционната семантика за езика \PCFPP}
\end{figure}




% \begin{lemma}
%   Нека $\tau$ е затворен терм от тип $\vv{a}$.
%   Тогава ако $\tau \Downarrow_{\vv{a}} \vv{v}$ и $\tau \Downarrow_{\vv{a}} \vv{u}$, то
%   $\vv{v} \equiv_\alpha \vv{u}$.
% \end{lemma}


\subsection{Денотационна семантика}

\marginpar{В \cite[Глава 4.3]{gunter} се нарича \emph{standard fixed-point semantics} of PCF.}

Семантиката на всеки тип ще бъде област на Скот както следва:
% \begin{align*}
\[\val{\bool} \df \Bool = \{true, false\}_\bot.\]
%   & \val{\nat} \df \Nat_\bot\\
%   & \val{\vv{a} \to \vv{b}} \df \Cont{\val{\vv{a}}}{\val{\vv{b}}}.
% \end{align*}

\begin{itemize}
% \item
%   Нека $\tau \equiv \vv{0}$. Тогава
%   \[\val{\vv{0}}_\Gamma(\overline{u}) \df 0.\]
\item
  Нека $\tau \equiv \tru$. Тогава
  \[\val{\tru}_\Gamma(\overline{u}) \df true.\]
\item
  Нека $\tau \equiv \fls$. Тогава
  \[\val{\fls}_\Gamma(\overline{u}) \df false.\]
% \item
%   Нека $\tau \equiv \vv{x}_i$. Тогава
%   \[\val{\vv{x}_i}_\Gamma(\overline{u}) \df u_i.\]
% \item
%   Нека $\tau \equiv \scc(\tau_1)$. Тогава
%   \[\val{\scc(\tau_1)}_\Gamma(\ov{u}) \df
%   \begin{cases}
%     \val{\tau_1}_\Gamma(\ov{u}) + 1, & \text{ ако }\val{\tau_1}_\Gamma(\ov{u}) \neq \bot\\
%     \bot, & \text{ ако }\val{\tau_1}_\Gamma(\ov{u}) = \bot.
%   \end{cases}\]

% \item
%   Нека $\tau \equiv \prd(\tau_1)$. Тогава
%   \[\val{\prd(\tau_1)}_\Gamma(\ov{u}) \df
%   \begin{cases}
%     0, & \text{ ако }\val{\tau_1}(\ov{u}) = 0\\
%     \val{\tau_1}_\Gamma(\ov{u}) - 1, & \text{ ако }\val{\tau_1}_\Gamma(\ov{u}) \neq 0, \bot\\
%     \bot, & \text{ ако }\val{\tau_1}_\Gamma(\ov{u}) = \bot.
%   \end{cases}\]

% \item
%   Нека $\tau \equiv \iszero(\tau_1)$. Тогава
%   \[\val{\iszero(\tau_1)}_\Gamma(\ov{u}) \df
%   \begin{cases}
%     true, &  \text{ ако }\val{\tau_1}_\Gamma(\ov{u}) = 0\\
%     false, & \text{ ако }\val{\tau_1}_\Gamma(\ov{u}) \neq 0,\bot\\
%     \bot, &  \text{ ако }\val{\tau_1}_\Gamma(\ov{u}) = \bot.
%   \end{cases}\]


% \item
%   \marginpar{За $\texttt{eq}$ вижте Раздел~\ref{subsect:rec:term-value}.}
%   Нека $\tau \equiv \tau_1\ \vv{==}\ \tau_2$. Тогава
%   \[\val{\tau_1\ \vv{==}\ \tau_2}_\Gamma(\overline{u}) \df \texttt{eq}(\val{\tau_1}_\Gamma(\overline{u}), \val{\tau_2}_\Gamma(\overline{u})).\]

\item
  Нека $\tau \equiv \tau_1\ \vv{==}\ \tau_2$. Тогава
  \[\val{\tau_1\ \vv{==}\ \tau_2}_\Gamma(\overline{u}) \df
    \begin{cases}
      true, & \text{ ако }\val{\tau_1}_\Gamma(\ov{u}) = \val{\tau_1}_\Gamma(\ov{u}) \in \Nat\\
      false, & \text{ ако }\val{\tau_1}_\Gamma(\ov{u}) \neq \val{\tau_1}_\Gamma(\ov{u}) \in \Nat\\
      \bot, & \text{ ако } \val{\tau_1}_\Gamma(\ov{u}) = \bot \text{ или } \val{\tau_1}_\Gamma(\ov{u}) = \bot.
    \end{cases}\]
\item
  % \marginpar{За $\texttt{if}$ вижте \Def{if}.}
  Нека $\tau \equiv \ifelse{\tau_1}{\tau_2}{\tau_3}$. Тогава
  \[\val{\ifelse{\tau_1}{\tau_2}{\tau_3}}_\Gamma(\overline{u}) \df
    \begin{cases}
      \val{\tau_2}_\Gamma(\ov{u}), & \text{ ако }\val{\tau_1}_\Gamma(\ov{u}) = true\\
      \val{\tau_3}_\Gamma(\ov{u}), & \text{ ако }\val{\tau_1}_\Gamma(\ov{u}) = false\\
      \bot, & \text{ ако } \val{\tau_1}_\Gamma(\ov{u}) = \bot.
    \end{cases}\]
% \item
%   \marginpar{За $\texttt{eval}$ вижте \Def{eval}.}
%   Нека $\tau \equiv \tau_1 \tau_2$. Тогава
%   \[\val{\tau_1 \tau_2}_\Gamma(\overline{u}) \df \texttt{eval}(\val{\tau_1}_\Gamma(\overline{u}), \val{\tau_2}_\Gamma(\overline{u})).\]
% \item
%   \marginpar{За $\lfp$ вижте Раздел~\ref{sect:lfp}.}
%   Нека $\tau \equiv \fix(\tau')$. Тогава 
%   \[\val{\fix(\tau')}_\Gamma(\overline{u}) \df \lfp(\val{\tau'}_\Gamma(\overline{u})).\]
% \item
%   \marginpar{За $\curry$ вижте \Def{curry}.}
%   Нека $\tau \equiv \lamb{y}{b}{\tau'}$, като $\vv{y} \not \in \texttt{dom}(\Gamma)$.
%   Нека $\Gamma' \df \Gamma, \type{y}{b}$. Тогава
%   \[\val{\lamb{y}{b}{\tau'}}_\Gamma(\overline{u}) \df \curry(\val{\tau'}_{\Gamma'})(\overline{u}).\]
\end{itemize}

% \begin{problem}
%   \marginpar{Аналог на \cite[Лема 4.19]{gunter}.}
%   Докажете, че ако $\Gamma \vdash \tau : \vv{a}$, то $\val{\tau}_\Gamma \in \Cont{\val{\Gamma}}{\val{\vv{a}}}$.
% \end{problem}

% \begin{problem}
%   Нека $\Gamma$ е типов контекст, $\tau$ и $\rho$ са термове, $\vv{x} \not\in \texttt{dom}(\Gamma)$,
%   \begin{align*}
%     & \Gamma \vdash \rho : \vv{a}\\
%     & \Gamma, \type{x}{a} \vdash \tau : \vv{b}.
%   \end{align*}
%   Докажете, че тогава:
%   \begin{enumerate}[1)]
%   \item
%     $\Gamma \vdash \tau\subst{x}{\rho} : \vv{b}$;
%   \item
%     за всяко $\overline{u} \in \val{\Gamma}$,
%     \[\val{\tau\subst{x}{\rho}}_\Gamma(\overline{u}) = \val{\tau}_{\Gamma'}(\overline{u},\val{\rho}_\Gamma(\overline{u})),\]
%     където $\Gamma' = \Gamma, \type{x}{a}$.  
%   \end{enumerate}
% \end{problem}

\begin{theorem}[Теорема за коректност за езика $\texttt{PCF(bool)}$]
  % \marginpar{Аналог на \cite[Твърдение 4.23]{gunter}.}
  Докажете, че за всеки затворен терм $\tau$ от тип $\vv{a}$ и стойност $\vv{v}$, е изпълнена импликацията:
  \[\tau \Downarrow_{\vv{a}} \vv{v}\ \implies\ \val{\tau} = \val{\vv{v}} \in \val{\vv{b}}.\]
\end{theorem}

\begin{theorem}[Теорема за адекватност за езика $\texttt{PCF(bool)}$]
  Нека разгледаме тип $\vv{a} = \nat$ или $\vv{a} = \bool$.
  За всеки затворен терм $\tau$ от тип $\vv{a}$ е изпълнена импликацията
  \[\val{\tau} = v \neq \bot^{\val{\vv{a}}} \implies \tau \Downarrow_{\vv{a}} \vv{v}.\]
\end{theorem}


%%% Local Variables:
%%% mode: latex
%%% TeX-master: "../sep"
%%% End:

% \newpage
% \section{Пълна абстракция}\label{pcf:sect:full-abstraction}
\marginpar{Full abstraction на англ.}
\begin{definition}
  \marginpar{\cite[стр. 179]{gunter}}
  Денотационната семантика $\val{.}$ се нарича {\bf напълно абстрактна}, ако
  контекстната (операционната) и денотационната наредба съвпадат, т.е.
  за произволни термове $\tau_1,\tau_2$ от тип $\vv{a}$ е изпълнено, че
  \[\val{\tau_1} \sqsubseteq \val{\tau_2} \iff \tau_1 \leq_{ctx} \tau_2 : \vv{a}.\]
\end{definition}

\begin{framed}
  \begin{theorem}[Гордън Плоткин 1977]
    Денотационната семантика $\val{.}$ за езика PCF {\bf не е} напълно абстрактна.
  \end{theorem}
\end{framed}
Да напомним, че от \Th{pcf:context:connection} винаги имаме следното:
\[ \val{\tau_1} = \val{\tau_2} \implies \tau_1 \cong_{ctx} \tau_2 : \vv{a}.\]
Сега ще се захванем с търсенето на термове $\tau_1$ и $\tau_2$, за които
$\val{\tau_1} \neq \val{\tau_2}$ и $\tau_1 \cong \tau_2 : \vv{a}$.


\begin{problem}
  Да дефинираме функцията $sor:\Nat_\bot \to (\Nat_\bot \to \Nat_\bot)$ по следния начин:
  \marginpar{$sor$ идва от sequential or.}
  
  \begin{tabular}{|c|c|c|c|}
    \hline
    $sor$ & $\bot$ & $0$ & $y>0$ \\
    \hline
    $\bot$ & $\bot$ & $\bot$ & $\bot$\\
    \hline
    $0$ & $\bot$ & $0$ & $1$\\
    \hline
    $x>0$ & $1$ & $1$ & $1$\\
    \hline
  \end{tabular}
  
  Докажете, че $\texttt{sor}$ е определима в PCF.
\end{problem}
\begin{hint}
  Разгледайте затворения терм
  \[\tau \df \lamb{x}{nat}{\lamb{y}{nat}{\ifelse{\vv{x}}{\vv{1}}{\ifelse{\vv{y}}{\vv{1}}{\vv{0}}}}}\]
  Докажете, че $\val{\tau} = sor$.
\end{hint}


\begin{problem}

Да дефинираме изображението $por:\Nat_\bot\to(\Nat_\bot \to \Nat_\bot)$ по следния начин:

\begin{tabular}{|c|c|c|c|}
  \hline
  $por$ & $\bot$ & $0$ & $y>0$\\
  \hline
  $\bot$ & $\bot$ & $\bot$ & $1$\\
  \hline
  $0$ & $\bot$ & $0$ & $1$\\
  \hline
  $x>0$ & $1$ & $1$ & $1$\\
  \hline
\end{tabular}
\marginpar{$por$ идва от parallel or.}

  Докажете, че $por$ е непрекъснато изображение.
\end{problem}
\begin{hint}
  Достатъчно е да се съобрази, че $por$ е монотонно изображение.
\end{hint}

\begin{framed}
  \begin{lemma}[Гордън Плоткин 1977]
    Изображението $por$ не е определимо в PCF, т.е. не съществува затворен терм $\rho$,
    за който $\val{\rho} = por$.
  \end{lemma}
\end{framed}

\begin{example}
Да видим, че операторът ,,или'' в хаскел не е паралелен.
\begin{haskellcode}
ghci> True || undefined
True
ghci> undefined || True
*** Exception: Prelude.undefined
\end{haskellcode}
\end{example}

\begin{problem}\label{prob:pcf:full-abstraction:por}
  Да разгледаме $f \in \Cont{\Nat_\bot}{\Cont{\Nat_\bot}{\Nat_\bot}}$, за което имаме ограниченията:

  \begin{tabular}{|c|c|c|c|}
    \hline
    $f$ & $\bot$ & $0$ & $y>0$\\
    \hline
    $\bot$ & $?$ & $?$ & $1$\\
    \hline
    $0$ & $?$ & $0$ & $?$\\
    \hline
    $x>0$ & $1$ & $?$ & $?$\\
    \hline
  \end{tabular}

  Докажете, че $f = \texttt{por}$.
  
\end{problem}
\begin{hint}
  Използвайте монотонността на $f$.
\end{hint}

\begin{problem}\label{prob:pcf:full-abstraction:not-definable}
  Да разгледаме изображението $f \in \Cont{\Nat_\bot}{\Cont{\Nat_\bot}{\Nat_\bot}}$, за което
  \[f(0)(0) = 0\text{ и } f(1)(\bot) = f(\bot)(1) = 1.\]
  Докажете, че $f$ не е определимо в PCF.
\end{problem}
\begin{hint}
  Да допуснем, че $f$ е определима в PCF.
  Тогава $f = \val{\tau}$, за някой затворен терм $\tau : \nat \to \nat \to \nat$.

  За произволна променлива $\vv{z}$, да положим
  \[\rho_{\vv{z}} \df \ifelse{\vv{z == 0}}{\vv{0}}{\vv{1}}.\]
  Нека също положим
  \begin{align*}
    \tau' & \df \tau\rho_{\vv{x}};\\
    \tau'' & \df \tau'\rho_{\vv{y}}.
  \end{align*}
  Нека също така $\Gamma \df \type{x}{nat}$ и $\Delta = \type{y}{nat}$.
  Ясно, че $\val{\tau'}_\Gamma \in \Cont{\Nat_\bot}{\Cont{\Nat_\bot}{\Nat_\bot}}$ и
  \begin{align*} 
    \val{\tau'}_\Gamma(u)  & = \val{\tau\rho_{\vv{x}}}_\Gamma(u)\\
                       & = \texttt{eval}(\val{\tau}, \val{\rho_{\vv{x}}}_\Gamma(u))\\
                       & = \val{\tau}(\val{\rho_{\vv{x}}}_\Gamma(u))
  \end{align*}
  Нека $f' = \val{\tau'}_\Gamma$. Получаваме следното за $f'$.
  \[f'(u) = f(\val{\rho_{\vv{x}}}_\Gamma(u)) =
    \begin{cases}
      f(u), & \text{ако } u = \bot \text{ или } u = 0\\
      f(1), & \text{ако } u > 0.
    \end{cases}\]
  Аналогично, ясно е, че $\val{\tau''}_{\Gamma,\Delta} \in \Cont{\Nat_\bot\times\Nat_\bot}{\Nat_\bot}$ и
  \begin{align*} 
    \val{\tau''}_{\Gamma,\Delta}(u,v)  & = \val{\tau'\rho_{\vv{y}}}_{\Gamma,\Delta}(u,v)\\
                                  & = \texttt{eval}(\val{\tau'}_\Gamma(u), \val{\rho_{\vv{y}}}_\Delta(v))\\
                                  & = \val{\tau'}_\Gamma(u)(\val{\rho_{\vv{y}}}_\Delta(v))\\
                                  & = \val{\tau}(\val{\rho_{\vv{x}}}_\Gamma(u))(\val{\rho_{\vv{y}}}_\Delta(v)).
  \end{align*}
  Нека сега $f'' = \val{\tau''}_{\Gamma,\Delta}$. Тогава
  \begin{align*}
    f''(u,v) & = f'(u)(\val{\rho_{\vv{y}}}_\Delta(v))\\
             & = \begin{cases}
               f'(u)(v), & \text{ако } v = \bot\text{ или } v = 0\\
               f'(u)(1), & \text{ако } v > 0
             \end{cases}
  \end{align*}
  Така получаваме следната характеризация на $f''$:

  \begin{tabular}{|c|c|c|c|}
    \hline
    $f''$ & $\bot$ & $0$ & $y>0$ \\
    \hline
    $\bot$ & $?$ & $?$ & $1$ \\
    \hline
    $0$ & $?$ & $0$ & $?$ \\
    \hline
    $x>0$ & $1$ & $?$ & $?$\\
    \hline
  \end{tabular}

  Нека сега $\rho \df \lamb{x}{nat}{\lamb{y}{nat}{\tau''}}$.
  Тогава за $g = \val{\rho}$ имаме, че
  \[g(x)(y) = f''(x,y).\]
  От \Problem{pcf:full-abstraction:por} получаваме, че $g = por$.
  Достигаме до противоречие, защото $por$ не е определимо изображение.
\end{hint}

В следващите твърдения ще използваме типовете
\begin{align*}
  & \vv{a} \df \nat \to (\nat \to \nat)\\
  & \vv{b} \df (\nat \to (\nat \to \nat))\to\nat.
\end{align*}
За $n = 0,1$, нека дефинираме затворените термове

\begin{lstlisting}
  $\tau_n \equiv \lambda \vv{f:a}$.if (f 1 $\Omega_\nat$) == 1 then
              if (f $\Omega_\nat$ 1) == 1 then
                if (f 0 0) == 0 then n
                  else $\Omega_\nat$
                else $\Omega_\nat$
              else $\Omega_\nat$
\end{lstlisting}

Лесно се съобразява, че $\tau_0$ и $\tau_1$ са добре типизирани термове от тип $\vv{b}$.

\begin{problem}
  Докажете, че 
  \[\val{\tau_0} \neq \val{\tau_1}.\]
\end{problem}
\begin{hint}
  Докажете, че за $n = 0,1$ е изпълнено, че
  \[\val{\tau_n}(por) = n.\]  
\end{hint}

\begin{proposition}
  $\tau_1 \cong_{ctx} \tau_2 : \vv{b}$.
\end{proposition}
\begin{proof}
  Понеже $\vv{b} = \vv{a} \to \nat$, от \Prop{pcf:context:extensionality} следва, че е достатъчно да докажем, че
  за всеки затворен терм $\rho:\vv{a}$ е изпълнено, че
  \[\tau_1\rho \Downarrow_{\nat} \vv{n} \iff \tau_2\rho \Downarrow_{\nat} \vv{n}.\]
  Да видим какво означава $\tau_i \rho \Downarrow_{\nat}$ за $i = 0,1$.
  Това означава, че трябва да са изпълнени и трите свойства:
  \begin{itemize}
  \item
    $\rho\ \vv{1}\ \Omega_{\nat} \Downarrow_{\nat} \vv{1}$;% , за някое $\vv{k} \not\equiv \vv{0}$;
  \item
    $\rho\ \Omega_{\nat}\ \vv{1} \Downarrow_{\nat} \vv{1}$;% , за някое $\vv{m} \not\equiv \vv{0}$;
  \item
    $\rho\ \vv{0}\ \vv{0} \Downarrow_{\nat} \vv{0}$.
  \end{itemize}
  Понеже $\val{\Omega_{\nat}} = \bot$, от \hyperref[th:pcf:soundness]{теоремата за коректност} получаваме, че трябва да са изпълнени следните три свойства:
  \begin{itemize}
  \item
    $\val{\rho}(1)(\bot) = 1$;
  \item
    $\val{\rho}(\bot)(1) = 1$;
  \item
    $\val{\rho}(0)(0) = 0$.
  \end{itemize}
  Но тогава $\val{\rho} = por$, което е противоречие с \Problem{pcf:full-abstraction:not-definable}.
\end{proof}

Доказателството на следващата теорема излиза извън обхата на този курс.
\index{Плоткин}
\begin{framed}
  \begin{theorem}[Плоткин 1977]
    Денотационната семантика $\val{.}$ за езика PCF+\texttt{por} е напълно абстрактна.
  \end{theorem}
\end{framed}
\marginpar{\cite[стр. 188]{gunter}}

% \index{Плоткин}
% \index{Милнър}
% \begin{theorem}[Милнър,Плоткин]
%   A continuous, order-extensional model of PCF is fully abstract if and only if for every type $\sigma$, $\val{\sigma}$ is a domain whose finite elements are definable.
% \end{theorem}

%%% Local Variables:
%%% mode: latex
%%% TeX-master: "../sep"
%%% End:



%%% Local Variables:
%%% mode: latex
%%% TeX-master: "../sep"
%%% End:
