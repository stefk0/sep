
\chapter{Езикът \PCF}\label{ch:pcf}

\marginpar{\PCF - programming language for computable functions. Plotkin’s language \PCF is often called the \emph{E. coli} of programming languages, the  subject  of  countless  studies  of  language  concept.

  Тази глава се оповава основно на \cite[Глава 19]{practical-foundations} и \cite{cambridge-den-sem}.}



\newcommand{\nat}{\vv{nat}}
\newcommand{\type}[2]{\vv{#1}:\vv{#2}}
\newcommand{\lamb}[3]{\lambda~\type{#1}{#2}~.~{#3}}
\newcommand{\AdequacyTheorem}{\hyperref[th:pcf:adequacy]{Теорема за адекватност}}
\newcommand{\SoundnessTheorem}{\hyperref[th:pcf:soundness]{Теорема за коректност}}


\section{Синтаксис}

\newcommand{\fix}{\texttt{fix}}
\newcommand{\fv}{\texttt{fv}}

\begin{itemize}
\item
  Типове
  \[\vv{a} ::= \vv{nat}\ |\ \vv{a} \to \vv{a}.\]
\item
  Изрази
  \[\tau ::= \vv{n}\ |\ \vv{x}\ |\ \tau + \tau\ |\ \tau\ \vv{==}\ \tau\ |\ \ifelse{\tau}{\tau}{\tau}\ |\ \tau\tau\ |\ \lambda \vv{x:a}.\tau\ |\ \fix(\tau).\]
  Ще казваме, че един израз $\tau$ е {\bf затворен}, ако $\fv(\tau) = \emptyset$.
  В противен случай, ще казваме, че изразът е {\bf отворен}.
\item
  \marginpar{или в канонична форма?}
  Ще казваме, че един израз е {\bf стойност}, ако той е съставен по следния начин:
  \[\vv{v} ::= \vv{n}\ |\ \lambda\type{x}{a}\ .\ \vv{v}.\]
\item
  Контексти
  \[\Gamma ::= \emptyset\ |\ \Gamma,\type{x}{a}.\]
\end{itemize}

Да обърнем внимание, че ние няма да правим разлика между два израза, които са $\alpha$-еквивалентни, т.е.
ще считаме, че $\lambda x:a. (x+1)$ е същият израз като $\lambda y:a. (y+1)$.

Понеже тук вече имам свободни и свързани променливи, трябва да дефинираме точно какво означава това.
\[\fv:\mathcal{E} \to \mathcal{V}\]
Ще дефинираме функцията $\texttt{fv}$ със структурна индукция по построението на термовете.

\begin{itemize}
\item
  $\fv(\vv{n}) = \emptyset$;
\item
  $\fv(\vv{x}) = \{\vv{x}\}$;
\item
  $\fv(\tau_1 + \tau_2) = \fv(\tau_1\ \vv{==}\ \tau_2) = \fv(\tau_1\tau_2) = \fv(\tau_1) \cup \fv(\tau_2)$;
\item
  $\fv(\ifelse{\tau_1}{\tau_2}{\tau_3}) = \fv(\tau_1) \cup \fv(\tau_2) \cup \fv(\tau_3)$;
\item
  $\fv(\lambda \type{x}{a}\ .\ \tau) = \fv(\tau) \setminus \{x\}$.
\item
  $\fv(\fix(\tau)) = \fv(\tau)$;
\end{itemize}



%%% Local Variables:
%%% mode: latex
%%% TeX-master: "../sep"
%%% End:


\section{Добре типизирани термове}\index{тип}

Типовите контексти представляват крайни редици от двойки от вида $\type{x}{a}$, т.е.
\[\Gamma ::= \emptyset\ |\ \Gamma,\type{x}{a}.\]
Обикновено ще означаваме типовите контексти с главните гръцки $\Gamma, \Delta, \dots$.
На един типов контекст $\Gamma$ може да се гледа и като на \emph{крайна} функция приемаща като аргумент променлива и връщаща тип.
\marginpar{$\Gamma$ също се нарича и type environment.}
Ние искаме да работим само с коректно типизирани термове.
Например, не е ясно какво означава терма $\vv{1 + }\lamb{x}{nat}{\vv{x}}$,
защото трябва да съберем число и функция - два обекта от различен тип.
Ако в изразите имаме свободни променливи, то дали един израз е коректно типизиран ще зависи от типовия контекст.
Сега ще дефинираме релация $\Gamma \vdash \tau : \vv{a}$, която ще ни казва, че термът $\tau$, относно типовият контекст $\Gamma$,
е добре типизиран и има тип $\vv{a}$.

\begin{prooftree}
  \AxiomC{}
  \RightLabel{\scriptsize{(const)}}
  \UnaryInfC{$\Gamma \vdash \vv{n} : \vv{nat}$}
\end{prooftree}

\begin{prooftree}
  \AxiomC{$\vv{x} \in \texttt{dom}(\Gamma)$}
  \AxiomC{$\Gamma(\vv{x}) = \vv{a}$}
  \RightLabel{\scriptsize{(var)}}
  \BinaryInfC{$\Gamma \vdash \vv{x} : \vv{a}$}
\end{prooftree}

\begin{prooftree}
  \AxiomC{$\Gamma \vdash \tau_1:\vv{nat}$}
  \AxiomC{$\Gamma \vdash \tau_2:\vv{nat}$}
  \RightLabel{\scriptsize{(plus)}}
  \BinaryInfC{$\Gamma \vdash \tau_1 + \tau_2 : \vv{nat}$}
\end{prooftree}

\begin{prooftree}
  \AxiomC{$\Gamma \vdash \tau_1:\vv{nat}$}
  \AxiomC{$\Gamma \vdash \tau_2:\vv{nat}$}
  \RightLabel{\scriptsize{(eq)}}
  \BinaryInfC{$\Gamma \vdash \tau_1\ \vv{==}\ \tau_2 : \vv{nat}$}
\end{prooftree}

\begin{prooftree}
  \AxiomC{$\Gamma \vdash \tau_1:\vv{nat}$}
  \AxiomC{$\Gamma \vdash \tau_2:\vv{a}$}
  \AxiomC{$\Gamma \vdash \tau_3:\vv{a}$}
  \RightLabel{\scriptsize{(if)}}
  \TrinaryInfC{$\Gamma \vdash \ifelse{\tau_1}{\tau_2}{\tau_3} : \vv{a}$}
\end{prooftree}

\begin{prooftree}
  \AxiomC{$\Gamma \vdash \tau_1:\vv{a}\to\vv{b}$}
  \AxiomC{$\Gamma \vdash \tau_2:\vv{a}$}
  \RightLabel{\scriptsize{(app)}}
  \BinaryInfC{$\Gamma \vdash \tau_1\tau_2 : \vv{b}$}
\end{prooftree}

\begin{prooftree}
  \AxiomC{$\Gamma \vdash \tau:\vv{a}\to\vv{a}$}
  \RightLabel{\scriptsize{(fix)}}
  \UnaryInfC{$\Gamma \vdash \fix(\tau) : \vv{a}$}
\end{prooftree}

\begin{prooftree}
  \AxiomC{$\vv{x} \not\in\vv{dom}(\Gamma)$}
  \AxiomC{$\Gamma, \type{x}{a} \vdash \tau:\vv{b}$}
  \RightLabel{\scriptsize{(lambda)}}
  \BinaryInfC{$\Gamma \vdash \lambda \type{x}{a}\ .\ \tau : \vv{a} \to \vv{b}$}
\end{prooftree}

Ако имаме затворен израз $\tau$, то ще пишем $\tau : \vv{a}$ вместо $\emptyset \vdash \tau : \vv{a}$.
Да положим
\[\text{PCF}_{\vv{a}} \dff \{\tau \text{ е затворен терм}\mid \emptyset \vdash \tau : \vv{a}\}.\]

\begin{proposition}
  Ако $\Gamma \vdash \tau : \vv{a}$, то $\fv(\tau) \subseteq \vv{dom}(\Gamma)$.
\end{proposition}

\begin{proposition}
  Ако $\Gamma \vdash \tau : \vv{a}$ и $\Gamma \vdash \tau : \vv{b}$, то $\vv{a} = \vv{b}$.
\end{proposition}

\begin{corollary}
  Всеки затворен терм има най-много един тип.
\end{corollary}

\begin{problem}
  \marginpar{\cite[стр. 104]{types-programming-languages}}
  Докажете или опровергайте дали е възможно да съществува типов контекст $\Gamma$ и тип $\vv{a}$, такива че
  \[\Gamma \vdash \type{xx}{a}.\]
\end{problem}



%%% Local Variables:
%%% mode: latex
%%% TeX-master: "../sep"
%%% End:


\section{Операционна семантика с предаване на параметрите по име}

Дефинираме релацията $\Downarrow^\ell_{\vv{a}}$ върху затворените изрази и стойностите.

\begin{prooftree}
  \AxiomC{$\type{v}{a}$}
  \RightLabel{\scriptsize{(val)}}
  \UnaryInfC{$\vv{v} \Downarrow^0_{\vv{a}} \vv{v}$}
\end{prooftree}

\begin{prooftree}
  \AxiomC{$\tau_1 \Downarrow^{\ell_1}_{\vv{nat}} \vv{n}_1$}
  \AxiomC{$\tau_2 \Downarrow^{\ell_2}_{\vv{nat}} \vv{n}_2$}
  \AxiomC{$n = \texttt{eq}(n_1,n_2)$}
  \RightLabel{\scriptsize{(eq)}}
  \TrinaryInfC{$\tau_1\ \vv{==}\ \tau_2 \Downarrow^{\ell_1+\ell_2+1}_{\vv{nat}} \vv{n}$}
\end{prooftree}

\begin{prooftree}
  \AxiomC{$\tau_1 \Downarrow^{\ell_1}_{\vv{nat}} \vv{n}_1$}
  \AxiomC{$\tau_2 \Downarrow^{\ell_2}_{\vv{nat}} \vv{n}_2$}
  \AxiomC{$n = n_1 + n_2$}
  \RightLabel{\scriptsize{(plus)}}
  \TrinaryInfC{$\tau_1\ \vv{+}\ \tau_2 \Downarrow^{\ell_1+\ell_2+1}_{\vv{nat}} \vv{n}$}
\end{prooftree}


\begin{prooftree}
  \AxiomC{$\tau_1 \Downarrow^{\ell_1}_{\vv{nat}} \vv{0}$}
  \AxiomC{$\tau_3 \Downarrow^{\ell_2}_{\vv{a}} \vv{v}$}
  \RightLabel{\scriptsize{(if$_0$)}}
  \BinaryInfC{$\ifelse{\tau_1}{\tau_2}{\tau_3} \Downarrow^{\ell_1+\ell_2+1}_{\vv{a}} \vv{v}$}
\end{prooftree}

\begin{prooftree}
  \AxiomC{$\tau_1 \Downarrow^{\ell_1}_{\vv{nat}} \vv{n}$}
  \AxiomC{$\tau_2 \Downarrow^{\ell_2}_{\vv{a}} \vv{v}$}
  \AxiomC{$\vv{n} \not\equiv \vv{0}$}
  \RightLabel{\scriptsize{(if$^+$)}}
  \TrinaryInfC{$\ifelse{\tau_1}{\tau_2}{\tau_3} \Downarrow^{\ell_1+\ell_2+1}_{\vv{a}} \vv{v}$}
\end{prooftree}

\begin{prooftree}
  \AxiomC{$\tau_1 \Downarrow^{\ell_1}_{\vv{a}\to\vv{b}} \lamb{x}{a}{\tau'_1}$}
  \AxiomC{$\tau'_1[x/\tau_2] \Downarrow^{\ell_2}_{\vv{b}} \vv{v}$}
  \RightLabel{\scriptsize{(cbn)}}
  \BinaryInfC{$\tau_1 \tau_2 \Downarrow^{\ell_1+\ell_2+1}_{\vv{b}} \vv{v} $}
\end{prooftree}

\begin{prooftree}
  \AxiomC{$\tau\ \fix(\tau) \Downarrow^{\ell}_{\vv{a}} \vv{v}$}
  \RightLabel{\scriptsize{(fix)}}
  \UnaryInfC{$\fix(\tau) \Downarrow^{\ell+1}_{\vv{a}} \vv{v} $}
\end{prooftree}

Ще пишем $\tau \Downarrow_{\vv{a}} \vv{v}$, ако съществува $\ell$, за което $\tau \Downarrow^\ell_{\vv{a}} \vv{v}$.


\begin{lemma}
  За произволен затворен терм $\tau$ и стойности $\vv{v}$ и $\vv{u}$,
  ако $\tau \Downarrow_{\vv{a}} \vv{v}$ и $\tau \Downarrow_{\vv{a}} \vv{u}$, то $\vv{v} \equiv \vv{u}$.
\end{lemma}

\begin{example}
  \[\Omega_{\vv{a}} \equiv \fix(\lamb{x}{a}{\vv{x}}).\]
  Лесно се вижда, че:
  \begin{itemize}
  \item
    $\emptyset \vdash \Omega_{\vv{a}} : \vv{a}$.
  \item
    За всяка стойност $\vv{v}$,
    $\Omega_{\vv{a}} \not\Downarrow_{\vv{a}} \vv{v}$.
  \end{itemize}
\end{example}
\begin{proof}
  Да допуснем, че $\Omega_{\vv{a}} \Downarrow_{\vv{a}} \vv{v}$ и нека фиксираме $\ell$
  да бъде най-малкия брой преобразования, за които
  \[\Omega_{\vv{a}} \Downarrow^{\ell}_{\vv{a}} \vv{v}.\]
\end{proof}


\begin{example}
  Нека $\vv{a} = \vv{nat} \to (\vv{nat} \to \vv{nat})$ и 
  \[\tau \equiv \fix\vv{(}\lamb{f}{a}{\lamb{x}{nat}{\lamb{y}{nat}{\ifelse{\vv{y == 0}}{\vv{0}}{\vv{x + (f x (y-1))} }}}}\vv{)}.\]
  Лесно се вижда, че
  \[\tau:\vv{nat} \to (\vv{nat} \to \vv{nat}).\]
  Също така за всяко $n$ и $k$, ако $m = n + k$, то
  \[\tau\ \vv{n}\ \vv{k} \Downarrow_{\vv{nat}} \vv{m}.\]
  Нека сега
  \[\rho \equiv \lamb{g}{a}{\fix(\lambda \vv{f} : \vv{nat}\to\vv{nat}\ .\ \lamb{x}{nat}{\ifelse{\vv{x == 0}}{\vv{1}}{\vv{g x f(x-1)}}}}).\]
  Лесно се вижда, че
  \[ \rho : \vv{a} \to (\vv{nat} \to \vv{nat}).\]
  Също така, за всяко $n$, ако $k = n!$, то
  \[ \rho\ \tau\ \vv{n} \Downarrow_{\vv{nat}} \vv{k}.\]
\end{example}

\begin{haskellcode}
> fix f = f (fix f)
> times = fix(\f x y -> if y == 0 then 0 else x + (f x (y-1)))
> times 2 3
6
> fct = \g -> fix(\f x -> if x == 0 then 1 else g x (f (x-1)))
> fact = fct times
> fact 5
120
\end{haskellcode}

%%% Local Variables:
%%% mode: latex
%%% TeX-master: "../sep"
%%% End:

\newpage
\begin{example}
  Нека $\vv{a} = \vv{nat} \to (\vv{nat} \to \vv{nat})$ и 
  \[\tau \equiv \fix\vv{(}\underbrace{\lamb{f}{a}{\overbrace{\lamb{x}{nat}{\lamb{y}{nat}{\ifelse{\vv{y == 0}}{\vv{0}}{\vv{x + (f x (y-1))} }}}}^{\tau'_0}}}_{\tau_0}\vv{)}.\]
  Както се вижда от \Figure{operational-cbn-example:typing}, имаме 
  \[\emptyset \vdash \tau:\vv{nat} \to (\vv{nat} \to \vv{nat}).\]

  По-долу ще видим, че
  \[\tau\ \vv{2}\ \vv{3} \Downarrow_{\vv{nat}} \vv{6}.\]
  
  По-общо, би трябвало да е ясно, че можем да докажем, че за всеки две естествени числа $n$ и $k$, ако $m = n * k$, то
  \[\tau\ \vv{n}\ \vv{k} \Downarrow_{\vv{nat}} \vv{m}.\]
  Нека сега
  \[\rho \equiv \lamb{g}{a}{\fix(\lambda \vv{f} : \vv{nat}\to\vv{nat}\ .\ \lamb{x}{nat}{\ifelse{\vv{x == 0}}{\vv{1}}{\vv{g x f(x-1)}}}}).\]
  Лесно се вижда, че
  \[\emptyset \vdash \rho : \vv{a} \to (\vv{nat} \to \vv{nat}).\]
  Също така, за всяко $n$, ако $k = n!$, то
  \[ \rho\ \tau\ \vv{n} \opsem{}{nat} \vv{k}.\]
  Можем да напишем директно горния пример и на хаскел:
\begin{haskellcode}
ghci> fix f = f (fix f)
ghci> times = fix(\f x y -> if y == 0 then 0 else x + (f x (y-1)))
ghci> times 2 3
6
ghci> rho = \g -> fix(\f x -> if x == 0 then 1 else g x (f (x-1)))
ghci> fact = rho times
ghci> fact 5
120
\end{haskellcode}
\end{example}
% Нека да положим
% \[\Gamma \df \vv{f} : \vv{a}, \vv{x} : \vv{nat}, \vv{y} : \vv{nat}.\]

\def\extraVskip{4pt}

\begin{landscape}
  \begin{framed}
    \begin{figure}[H]
      \centering
    % \begin{subfigure}[b]{1\textwidth}
      \begin{prooftree}
        \AxiomC{$\Gamma(\vv{y}) = \vv{nat}$}
        \UnaryInfC{$\Gamma \vdash \vv{y} : \vv{nat}$}
        \AxiomC{}
        \UnaryInfC{$\Gamma \vdash \vv{0} : \vv{nat}$}
        \BinaryInfC{$\Gamma \vdash \vv{y==0} : \vv{nat}$}
        \AxiomC{}
        \UnaryInfC{$\Gamma \vdash \vv{0} : \vv{nat}$}
        \AxiomC{$\Gamma(\vv{x}) = \vv{nat}$}
        \UnaryInfC{$\Gamma \vdash \vv{x}:\vv{nat}$}
        \AxiomC{$\Gamma(\vv{f}) = \vv{a}$}
        \UnaryInfC{$\Gamma \vdash \vv{f} : \vv{a}$}
        \AxiomC{$\Gamma(\vv{x}) = \vv{nat}$}
        \UnaryInfC{$\Gamma \vdash \vv{x} : \vv{nat}$}
        \BinaryInfC{$\Gamma \vdash \vv{f x} : \vv{nat} \to \vv{nat}$}
        \AxiomC{$\Gamma(\vv{y}) = \vv{nat}$}
        \UnaryInfC{$\Gamma \vdash \vv{y} : \vv{nat}$}
        \AxiomC{}
        \UnaryInfC{$\Gamma \vdash \vv{1} : \vv{nat}$}
        \BinaryInfC{$\Gamma \vdash \vv{y-1} : \vv{nat}$}
        \BinaryInfC{$\Gamma \vdash \vv{f x (y-1)} : \vv{nat}$}
        \BinaryInfC{$\Gamma \vdash \vv{x + f x (y-1)} : \vv{nat}$}
        \TrinaryInfC{$\underbrace{\vv{f} : \vv{a}, \vv{x} : \vv{nat}, \vv{y} : \vv{nat}}_{\Gamma} \vdash \ifelse{\vv{y == 0}}{\vv{0}}{\vv{x + (f x (y-1))}} : \vv{nat}$}
        \UnaryInfC{$\vv{f} : \vv{a}, \vv{x} : \vv{nat} \vdash \lamb{y}{nat}{\ifelse{\vv{y == 0}}{\vv{0}}{\vv{x + (f x (y-1))}}} : \vv{nat} \to \vv{nat}$}
        \UnaryInfC{$\vv{f} : \vv{a} \vdash \tau'_0 : \vv{a}$}
        \UnaryInfC{$\emptyset \vdash \tau_0 : \vv{a} \to \vv{a}$}
        \UnaryInfC{$\emptyset \vdash \fix(\tau_0):\vv{a}$}
      \end{prooftree}
      \caption{Формален извод според правилата на типизиращата релацията, който показва, че $\emptyset \vdash \tau : \vv{a}$.}
      \label{fig:operational-cbn-example:typing}
    % \end{subfigure}
  \end{figure}
\end{framed}

След като видяхме, че $\tau$ е терм от тип $\vv{a}$, нека да видим
колко стъпки ще са ни нужни, според правилата на операционната семантика, за да проверим, че
$\tau\ \vv{3}\ \vv{2} \opsem{}{nat} \vv{6}$.

Първо да обърнем внимание, че $\tau_0$ е стойност, т.е. $\tau_0 \opsemGen{0}{\vv{a}\to\vv{a}} \tau_0$.
Макар и $\tau'_0$ да има свободна променлива $\vv{f}$, понеже термът $\fix(\tau_0)$ е затворен с тип $\vv{a}$, то термът
\[\tau'_0\subst{f}{\fix(\tau_0)} \equiv \lamb{x}{nat}{\lamb{y}{nat}{\ifelse{\vv{y == 0}}{\vv{0}}{\vv{ x + (}\fix(\tau_0)\vv{ x (y-1))}}}}\]
е стойност и следователно,
\[\tau'_0\subst{f}{\fix(\tau_0)} \opsem{0}{a} \tau'_0\subst{f}{\fix(\tau_0)}.\]

\begin{framed}
  \begin{figure}[H]
    \begin{prooftree}
      \AxiomC{$\tau_0$ е стойност}
      \LeftLabel{\scriptsize{(val)}}
      \UnaryInfC{$\tau_0 \opsemGen{0}{\vv{a}\to\vv{a}} \lamb{f}{a}{\tau'_0}$}
      \AxiomC{$\tau'_0\subst{f}{\fix(\tau_0)}$ е стойност}
      \RightLabel{\scriptsize{(val)}}
      \UnaryInfC{$\tau'_0\subst{f}{\fix(\tau_0)} \opsem{0}{a}  \lamb{x}{nat}{\tau_2}$}
      \LeftLabel{\scriptsize{(app)}}
      \BinaryInfC{$\tau_0\fix(\tau_0) \opsem{1}{a} \lamb{x}{nat}{\tau_2}$}
      \LeftLabel{\scriptsize{(fix)}}
      \UnaryInfC{$\fix(\tau_0) \opsem{2}{a} \lamb{x}{nat}{\tau_2}$}
      \AxiomC{$\tau_2\substConst{x}{3}$ е стойност}
      \LeftLabel{\scriptsize{(val)}}
      \UnaryInfC{$\tau_2\substConst{x}{3} \opsemGen{0}{\vv{nat}\to \vv{nat}} \lamb{y}{nat}{\tau_1}$}
      \LeftLabel{\scriptsize{(app)}}
      \BinaryInfC{$\fix(\tau_0)\ \vv{3} \opsemGen{3}{\vv{nat}\to\vv{nat}} \lamb{y}{nat}{\tau_1}$}
      \AxiomC{\Figure{operational-cbn-example:second-part}}
      \UnaryInfC{$\tau_1[\vv{y}/\vv{2}] \opsem{16}{nat} \vv{6}$}
      \LeftLabel{\scriptsize{(app)}}
      \BinaryInfC{$\underbrace{\fix(\tau_0)}_{\tau}\ \vv{3}\ \vv{2} \opsem{20}{nat} \vv{6}$}
    \end{prooftree}
    \caption{Първа част от изчислението на $\tau\ \vv{3}\ \vv{2}$ според правилата на операционната семантика.}
    \label{fig:operational-cbn-example:first-part}
  \end{figure}
\end{framed}

Нека за улеснение да положим
\begin{align*}
  \tau_2 & \equiv \lamb{y}{nat}{\ifelse{\vv{y == 0}}{\vv{0}}{\vv{ x + (}\fix(\tau_0)\vv{ x (y-1))}}}\\
  \tau_1 & \equiv \ifelse{\vv{y == 0}}{\vv{0}}{\vv{ 3 + (}\fix(\tau_0)\vv{ 3 (y-1))}}.
\end{align*}

Термът $\tau_2$ не е стойност, защото има свободна променлива $\vv{x}$, но вече термът $\tau_2\subst{x}{3}$ е стойност. Лесно се съобразява, че
\begin{align*}
  & \tau'_0\subst{f}{\fix(\tau_0)} \equiv \lamb{x}{nat}{\tau_2}\\
  & \tau_2 \substConst{x}{3} \equiv \lamb{y}{nat}{\tau_1}.
\end{align*}

Сега във \Figure{operational-cbn-example:second-part} продължаваме десния клон на изчислението:
  \begin{framed}
    \begin{figure}[H]
      \begin{prooftree}
        \AxiomC{$\vv{2}$ е стойност}
        \LeftLabel{\scriptsize{(val)}}
        \UnaryInfC{$\vv{2} \opsem{0}{nat} \vv{2}$}
        \AxiomC{$\vv{0}$ е стойност}
        \RightLabel{\scriptsize{(val)}}
        \UnaryInfC{$\vv{0} \opsem{0}{nat} \vv{0}$}
        \LeftLabel{\scriptsize{(eq)}}
        \BinaryInfC{$\vv{2 == 0} \opsem{1}{nat} \vv{0}$}
        \AxiomC{$\vv{3}$ е стойност}
        \UnaryInfC{$\vv{3} \opsem{0}{nat} \vv{3}$}
        \AxiomC{(Повтаря се както по-горе)}
        \UnaryInfC{$\fix(\tau_0)\ \vv{3} \opsemGen{3}{\vv{nat}\to\vv{nat}} \lamb{y}{nat}{\tau_1}$}
        \AxiomC{\Figure{operational-cbn-example:third-part}}
        \UnaryInfC{$\tau_1\substConst{y}{2-1} \opsem{10}{nat} \vv{3}$}
        \RightLabel{\scriptsize{(app)}}
        \BinaryInfC{$\fix(\tau_0)\vv{ 3 (2-1) }\opsem{14}{nat} \vv{3}$}
        \RightLabel{\scriptsize{(plus)}}
        \BinaryInfC{$\vv{ 3 + (}\fix(\tau_0)\vv{ 3 (2-1))} \opsem{15}{nat} \vv{6}$}
        \LeftLabel{\scriptsize{(if$_0$)}}
        \BinaryInfC{$\underbrace{\ifelse{\vv{2 == 0}}{\vv{0}}{\vv{ 3 + (}\fix(\tau_0)\vv{ 3 (2-1))}}}_{\tau_1\substConst{y}{2}} \opsem{16}{nat} \vv{6}$}
      \end{prooftree}
      \caption{Втора част от изчислението, която показва, че $\tau_1\substConst{y}{2} \opsem{16}{nat} \vv{6}$.}
      \label{fig:operational-cbn-example:second-part}
    \end{figure}
  \end{framed}

  Във \Figure{operational-cbn-example:third-part} продължаваме с десния клон на изчислението, като тук вече имаме, че:
  \[\tau_1\substConst{y}{2-1} \equiv \ifelse{\vv{2-1 == 0}}{\vv{0}}{\vv{3 + (}\fix(\tau_0)\vv{ 3 (2-1-1))}}\]
  Най-накрая, във \Figure{operational-cbn-example:last-part} завършваме изчислението като използваме, че
  \[\tau_1\substConst{y}{2-1-1} \equiv \ifelse{\vv{2-1-1 == 0}}{\vv{0}}{\vv{3 + (}\fix(\tau_0)\vv{ 3 (2-1-1-1))}}.\]
  
  \def\defaultHypSeparation{\hskip 10pt}
  \def\proofSkipAmount{\vskip 3pt}
  \begin{framed}
%    {\footnotesize
    \begin{figure}[H]
      \begin{prooftree}
        \AxiomC{$\vv{2}$ е стойност}
        \UnaryInfC{$\vv{2} \opsem{0}{nat} \vv{2}$}
        \AxiomC{$\vv{1}$ е стойност}
        \UnaryInfC{$\vv{1} \opsem{0}{nat} \vv{1}$}
        \LeftLabel{\scriptsize{(minus)}}
        \BinaryInfC{$\vv{2-1} \opsem{1}{nat} \vv{1}$}
        \AxiomC{$\vv{0}$ е стойност}
        \UnaryInfC{$\vv{0} \opsem{0}{nat} \vv{0}$}
        \LeftLabel{\scriptsize{(eq)}}
        \BinaryInfC{$\vv{2-1 == 0} \opsem{2}{nat} \vv{0}$}
        \AxiomC{$\vv{3}$ е стойност}
        \UnaryInfC{$\vv{3} \opsem{0}{nat} \vv{3}$}
        \AxiomC{(Повтаря се както по-горе)}
        \UnaryInfC{$\fix(\tau_0)\ \vv{3} \opsemGen{3}{\vv{nat}\to\vv{nat}} \lamb{y}{nat}{\tau_1}$}
        \AxiomC{\Figure{operational-cbn-example:last-part}}
        \UnaryInfC{$\tau_1\substConst{y}{2-1-1} \opsem{4}{nat} \vv{0}$}
        \RightLabel{\scriptsize{(app)}}
        \BinaryInfC{$\fix(\tau_0)\vv{ 3 (2-1-1) }\opsem{8}{nat} \vv{0}$}
        \RightLabel{\scriptsize{(plus)}}
        \BinaryInfC{$\vv{ 3 + (}\fix(\tau_0)\vv{ 3 (2-1-1))} \opsem{9}{nat} \vv{3}$}
        \LeftLabel{\scriptsize{(if$_0$)}}
        \BinaryInfC{$\underbrace{\ifelse{\vv{2-1 == 0}}{\vv{0}}{\vv{ 3 + (}\fix(\tau_0)\vv{ 3 (2-1-1))}}}_{\tau_1\substConst{y}{2-1}} \opsem{10}{nat} \vv{3}$}
      \end{prooftree}
      \caption{Трета част от изчислението, която показва, че $\tau_1\substConst{y}{2-1} \opsem{10}{nat} \vv{3}$ }
      \label{fig:operational-cbn-example:third-part}
    \end{figure}
 %   }
  \end{framed}

  \begin{framed}
    \begin{figure}[H]
      \begin{prooftree}
        \AxiomC{$\vv{2}$ е стойност}
        \UnaryInfC{$\vv{2} \opsem{0}{nat} \vv{2}$}
        \AxiomC{$\vv{1}$ е стойност}
        \UnaryInfC{$\vv{1} \opsem{0}{nat} \vv{1}$}
        \LeftLabel{\scriptsize{(minus)}}
        \BinaryInfC{$\vv{2-1} \opsem{1}{nat} \vv{1} $}
        \AxiomC{$\vv{1}$ е стойност}
        \UnaryInfC{$\vv{1} \opsem{0}{nat} \vv{1}$}
        \LeftLabel{\scriptsize{(minus)}}
        \BinaryInfC{$\vv{2-1-1} \opsem{2}{nat} \vv{0}$}
        \AxiomC{$\vv{0}$ е стойност}
        \UnaryInfC{$\vv{0} \opsem{0}{nat} \vv{0}$}
        \LeftLabel{\scriptsize(eq)}
        \BinaryInfC{$\vv{2-1-1 == 0} \opsem{3}{nat} \vv{1}$}
        \AxiomC{$\vv{0}$ е стойност}
        \UnaryInfC{$\vv{0} \opsem{0}{nat} \vv{0}$}
        \LeftLabel{\scriptsize{(if$^+$)}}
        \BinaryInfC{$\underbrace{\ifelse{\vv{2-1-1 == 0}}{\vv{0}}{\vv{ 3 + (}\fix(\tau_0)\vv{ 3 (2-1-1-1))}}}_{\tau_1\substConst{y}{2-1-1}} \opsem{4}{nat} \vv{0}$}
      \end{prooftree}
      \caption{Последна част от изчислението, което показва, че $\tau_1\substConst{y}{2-1-1} \opsem{4}{nat} \vv{0}$.}
      \label{fig:operational-cbn-example:last-part}
    \end{figure}
  \end{framed}  
\end{landscape}


\newpage
\section{Денотационна семантика с предаване на параметрите по име}

Семантиката на всеки тип ще бъде област на Скот както следва:
\begin{align*}
  & \val{\vv{nat}} \df \Nat_\bot\\
  & \val{\vv{a} \to \vv{b}} \df \Cont{\val{\vv{a}}}{\val{\vv{b}}}.
\end{align*}
\marginpar{Да напомним, че \[\emptyset_\bot = (\{\bot\},\sqsubseteq,\bot).\]}
\marginpar{От Раздел~\ref{subsect:domains:product} знаем, че $\val{\Gamma}$ е област на Скот.}
За един типов контекст $\Gamma$, дефинираме $\val{\Gamma}$ по следния начин:
\begin{itemize}
\item
  Ако $\Gamma = \emptyset$, то $\val{\Gamma} = \emptyset_\bot$;
\item
  Ако $\Gamma = \Gamma', \vv{x:a}$, то $\val{\Gamma} = \val{\Gamma'} \times \val{\vv{a}}$.
\end{itemize}
Например, ако $\Gamma = \vv{x}_1 : \vv{a}_1,\ \vv{x}_2 : \vv{a}_2,\ \vv{x}_3 : \vv{a}_3$, то
\[\val{\Gamma} = (\val{\vv{a}_1} \times \val{\vv{a}_2})\times \val{\vv{a}_3}.\]

Сега трябва да дефинираме семантика на термовете.
За всеки терм, за който $\fv(\tau) \subseteq \texttt{dom}(\Gamma)$ и
за произволни $\overline{u} \in \val{\Gamma}$, дефинираме неговата стойност $\val{\tau}_\Gamma(\overline{u})$ по следния начин:
\begin{itemize}
\item
  Нека $\tau \equiv \vv{n}$. Тогава
  \[\val{\vv{n}}_\Gamma(\overline{u}) \df n.\]
\item
  Нека $\tau \equiv \vv{x}_i$. Тогава
  \[\val{\vv{x}_i}_\Gamma(\overline{u}) \df u_i.\]
\item
  \marginpar{За $\texttt{plus}$ вижте Раздел~\ref{subsect:rec:term-value}.}
  Нека $\tau \equiv \tau_1 + \tau_2$. Тогава
  \[\val{\tau_1 + \tau_2}_\Gamma(\overline{u}) \df \texttt{plus}(\val{\tau_1}_\Gamma(\overline{u}), \val{\tau_2}_\Gamma(\overline{u})).\]
\item
  \marginpar{За $\texttt{eq}$ вижте Раздел~\ref{subsect:rec:term-value}.}
  Нека $\tau \equiv \tau_1\ \vv{==}\ \tau_2$. Тогава
  \[\val{\tau_1\ \vv{==}\ \tau_2}_\Gamma(\overline{u}) \df \texttt{eq}(\val{\tau_1}_\Gamma(\overline{u}), \val{\tau_2}_\Gamma(\overline{u})).\]
\item
  \marginpar{За $\texttt{if}$ вижте \Def{if}.}
  Нека $\tau \equiv \ifelse{\tau_1}{\tau_2}{\tau_3}$. Тогава
  \[\val{\ifelse{\tau_1}{\tau_2}{\tau_3}}_\Gamma(\overline{u}) \df \texttt{if}(\val{\tau_1}_\Gamma(\overline{u}),
  \val{\tau_2}_\Gamma(\overline{u}), \val{\tau_3}_\Gamma(\overline{u})).\]
\item
  \marginpar{За $\texttt{eval}$ вижте \Def{eval}.}
  Нека $\tau \equiv \tau_1 \tau_2$. Тогава
  \[\val{\tau_1 \tau_2}_\Gamma(\overline{u}) \df \texttt{eval}(\val{\tau_1}_\Gamma(\overline{u}), \val{\tau_2}_\Gamma(\overline{u})).\]
\item
  \marginpar{За $\lfp$ вижте Раздел~\ref{sect:lfp}.}
  Нека $\tau \equiv \fix(\tau')$. Тогава 
  \[\val{\fix(\tau')}_\Gamma(\overline{u}) \df \lfp(\val{\tau'}_\Gamma(\overline{u})).\]
\item
  \marginpar{За $\curry$ вижте \Def{curry}.}
  Нека $\tau \equiv \lamb{y}{b}{\tau'}$, като $\vv{y} \not \in \texttt{dom}(\Gamma)$.
  Нека $\Gamma' \df \Gamma, \type{y}{b}$. Тогава
  \[\val{\lamb{y}{b}{\tau'}}_\Gamma(\overline{u}) \df \curry(\val{\tau'}_{\Gamma'})(\overline{u}).\]
\end{itemize}

\begin{remark}
  За $\Gamma = \emptyset$, ще пишем $\val{\tau}$ вместо $\val{\tau}_\emptyset$.
\end{remark}

Не е ясно дали винаги горните дефиниции имат смисъл.
Сега ще докажем, че винаги, когато един терм е добре типизиран, то горната дефиниция има смисъл.

\begin{framed}
  \begin{lemma}
    Ако $\Gamma \vdash \tau : \vv{a}$, то $\val{\tau}_\Gamma \in \Cont{\val{\Gamma}}{\val{\vv{a}}}$.
  \end{lemma}  
\end{framed}
\begin{proof}
  Доказателството протича с индукция по построението на термовете
  като съществено използваме \Prop{composition} според което, ако $f \in \Cont{\A}{\B}$ и $g \in \Cont{\B}{\C}$, то
  $g \circ f \in \Cont{\A}{\C}$.
  \marginpar{Изображението $f \times g$ е дефинирано в \Prop{cartesian-continuous}.}
  \begin{itemize}
  \item
    Нека $\tau \equiv \vv{n}$. Щом $\Gamma \vdash \tau : \vv{a}$, то
    по правилата за типизиране следва, че $\vv{a} = \vv{nat}$.
    Сега лесно се съобразява, че изображението $\val{\vv{n}}_\Gamma \in \Cont{\val{\Gamma}}{\val{\vv{nat}}}$, където
    $\val{\vv{n}}_\Gamma(\overline{u}) = n$.
    Това е така, защото за всяка верига $\chain{\overline{u}}{i}$ от елементи на $\val{\Gamma}$,
    \[\val{\vv{n}}_\Gamma(\bigsqcup_i\overline{u}_i) = n = \bigsqcup_i\{ \val{\vv{n}}(\overline{u}_i)\}.\]
  \item
    Нека $\tau \equiv \vv{x}_i$. Щом $\Gamma \vdash \tau : \vv{a}$, то
    по правилата за типизиране следва, че $\vv{a} = \vv{a}_i$.
    Сега лесно се съобразява, че изображението $\val{\vv{n}}_\Gamma \in \Cont{\val{\Gamma}}{\val{\vv{a}_i}}$, където
    $\val{\vv{n}}_\Gamma(\overline{u}) = u_i$.
    \marginpar{$\overline{u}_k = (u_{1,k},\dots,u_{n,k})$.}
    Това е така, защото за всяка верига $\chain{\overline{u}}{n}$ от елементи на $\val{\Gamma}$,
    \[\val{\vv{x}_i}_\Gamma(\bigsqcup_n\overline{u}_n) = \bigsqcup_n u_{i,n} = \bigsqcup_i\{ \val{\vv{x}_i}(\overline{u}_n)\}.\]
  \item
    Нека $\tau \equiv \tau_1 + \tau_2$. Щом $\Gamma \vdash \tau : \vv{a}$, то
    по правилата за типизиране следва, че $\vv{a} = \vv{nat}$, а също и $\Gamma \vdash \tau_1 : \vv{nat}$ и $\Gamma \vdash \tau_2
    : \vv{nat}$.
    От И.П. имаме, че
    \begin{align*}
      & \val{\tau_1}_\Gamma \in \Cont{\val{\Gamma}}{\val{\vv{nat}}};\\
        & \val{\tau_2}_\Gamma \in \Cont{\val{\Gamma}}{\val{\vv{nat}}}.
    \end{align*}
    Това означава, че $(\val{\tau_1} \times \val{\tau_2}) \in \Cont{\val{\Gamma}}{\val{\vv{nat}} \times \val{\vv{nat}}}$.
    Тогава имаме следното равенство
    \marginpar{Използваме, че композиция на непрекъснати изображения е непрекъснато изображение.}
    \[\val{\tau_1 + \tau_2}_\Gamma = \texttt{plus} \circ (\val{\tau_1} \times \val{\tau_2}) \in \Cont{\val{\Gamma}}{\val{\vv{a}}},\]
    защото за произволни $\overline{u} \in \val{\Gamma}$,
    \begin{align*}
      (\texttt{plus} \circ (\val{\tau_1} \times \val{\tau_2}))(\overline{u}) & = \texttt{plus}((\val{\tau_1} \times \val{\tau_2})(\overline{u}))\\ 
                                                                             & = \texttt{plus}(\val{\tau_1}_\Gamma(\overline{u}), \val{\tau_2}_\Gamma(\overline{u}))\\
                                                                             & \df \val{\tau}_\Gamma(\overline{u}).
    \end{align*}
  \item
    Нека $\tau \equiv \tau_1\ \vv{==}\ \tau_2$. Съобразете сами, че 
    \[\val{\tau_1\ \vv{==}\ \tau_2}_\Gamma = \texttt{eq} \circ (\val{\tau_1}_\Gamma \times \val{\tau_2}_\Gamma) \in \Cont{\val{\Gamma}}{\val{\vv{a}}}.\]
  \item
    Нека $\tau \equiv \ifelse{\tau_1}{\tau_2}{\tau_3}$. Съобразете сами, че 
    \[\val{\ifelse{\tau_1}{\tau_2}{\tau_3}}_\Gamma = \texttt{if} \circ (\val{\tau_1}_\Gamma \times \val{\tau_2}_\Gamma \times \val{\tau_3}_\Gamma)  \in \Cont{\val{\Gamma}}{\val{\vv{a}}}.\]
  \item
    Нека $\tau \equiv \tau_1 \tau_2$.
    Щом $\Gamma \vdash \tau_1 \tau_2 : \vv{a}$, то от правилата за типизиране следва, че
    \begin{align*}
      & \Gamma \vdash \tau_1 : \vv{b} \to \vv{a}\\
      & \Gamma \vdash \tau_2 : \vv{b}.
    \end{align*}
    От И.П. за $\tau_1$ и $\tau_2$ знаем, че
    \begin{align*}
      & \val{\tau_1}_\Gamma \in \Cont{\val{\Gamma}}{\Cont{\val{\vv{b}}}{\val{\vv{a}}}} \\
      & \val{\tau_2}_\Gamma \in \Cont{\val{\Gamma}}{\val{\vv{b}}}
    \end{align*}
    Оттук получаваме, че за произволни $\overline{u} \in \val{\Gamma}$,
    \begin{align*}
      & \val{\tau_1}_\Gamma(\overline{u}) \in \Cont{\val{\vv{b}}}{\val{\vv{a}}} \\
      & \val{\tau_2}_\Gamma(\overline{u}) \in \val{\vv{b}}.
    \end{align*}
    Тогава 
    \[\val{\tau_1 \tau_2}_\Gamma = \texttt{eval} \circ (\val{\tau_1}_\Gamma \times \val{\tau_2}_\Gamma) \in \Cont{\val{\Gamma}}{\val{\vv{a}}},\]
    защото за произволни $\overline{u} \in \val{\Gamma}$,
    \begin{align*}
      (\texttt{eval} \circ \val{\tau_1}_\Gamma \times \val{\tau_2}_\Gamma)(\overline{u}) & = \texttt{eval}((\val{\tau_1}_\Gamma \times \val{\tau_2}_\Gamma)(\overline{u}))\\
                                                                                         & = \texttt{eval}(\val{\tau_1}_\Gamma(\overline{u}), \val{\tau_2}_\Gamma(\overline{u}))\\
                                                                                         & \df \val{\tau_1\tau_2}_\Gamma(\overline{u}).
    \end{align*}
    
  \item
    Нека сега $\tau \equiv \fix(\tau')$.
    Понеже $\Gamma \vdash \fix(\tau') : \vv{a}$, то от правилата за типизиране имаме, че
    $\Gamma \vdash \tau' : \vv{a} \to \vv{a}$.
    От И.П. знаем, че
    \[\val{\tau'}_\Gamma \in \Cont{\val{\Gamma}}{\Cont{\val{\vv{a}}}{\val{\vv{a}}}}.\]
    Това означава, че за произволни $\overline{u} \in \val{\Gamma}$,
    \[\val{\tau'}_\Gamma(\overline{u}) \in \Cont{\val{\vv{a}}}{\val{\vv{a}}}.\]
    Следователно
    $\val{\tau'}_\Gamma(\overline{u})$ е изображение, което според \Th{knaster-tarski}
    притежава най-малка неподвижна точка.
    \marginpar{Непрекъснатото изображението $Y$ е дефинирано \Th{Y}.}
    Тогава
    \[\val{\texttt{fix}(\tau')}_\Gamma = Y \circ \val{\tau'}_\Gamma \in \Cont{\val{\Gamma}}{\val{\vv{a}}},\]
    защото за произволни $\overline{u} \in \val{\Gamma}$,
    \begin{align*}
      (Y \circ \val{\tau'}_\Gamma)(\overline{u}) & = Y(\val{\tau'}_\Gamma(\overline{u}))\\
                                                 & = \lfp(\val{\tau'}_\Gamma(\overline{u}))\\
                                                 & \df \val{\fix(\tau')}_\Gamma(\ov{u}).
    \end{align*}
  \item
    Нека $\tau \equiv \lamb{y}{b}{\tau'}$, като $\vv{y} \not \in \texttt{dom}(\Gamma)$.
    Щом $\Gamma \vdash \lamb{y}{b}{\tau'} : \vv{a}$, то от правилата за типизиране следва, че $\vv{a} = \vv{b} \to \vv{c}$
    и 
    \[\Gamma, \type{y}{b} \vdash \tau' : \vv{c}.\]
    
    Нека $\Gamma' = \Gamma, \vv{y}:\vv{b}$. Тогава $\val{\Gamma'} = \val{\Gamma} \times \val{\vv{b}}$, а от И.П. имаме, че
    \[\val{\tau'}_{\Gamma'} \in \Cont{\val{\Gamma} \times \val{\vv{b}}}{\val{\vv{c}}}.\]
    Тогава от \Prop{curry} следва, че
    \[\val{\lamb{y}{b}{\tau}}_\Gamma \df \curry(\val{\tau'}_{\Gamma'}) \in \Cont{\val{\Gamma}}{\Cont{\val{\vv{b}}}{\val{\vv{c}}}}.\]
  \end{itemize}
\end{proof}

\begin{remark}
  В случая $\Gamma = \emptyset$, формално погледнато,
  $\val{\tau}_\emptyset \in \Cont{\emptyset_\bot}{\A}$, за някоя област на Скот $\A$.
  Но ние знаем, че $\Cont{\emptyset_\bot}{\A} \cong \A$.
  Следователно, можем да считаме, че $\val{\tau} \in \A$.
  В противен случай, трябва винаги да пишем $\val{\tau}(\bot)$ вместо $\val{\tau}$.
\end{remark}


\begin{proposition}
  \marginpar{Ясно е, че това твърдение се обобщава за произволна пермутация на индекстите $1,\dots,n$ \cite[стр. 106]{types-programming-languages}.}
  Нека имаме следните типови контексти:
  \begin{align*}
    &\Gamma = \vv{x}_1:\vv{a}_1, \dots, \vv{x}_i:\vv{a}_i, \dots, \vv{x}_j:\vv{a}_j, \dots, \vv{x}_n:\vv{a}_n;\\
    &\Delta = \vv{x}_1:\vv{a}_1, \dots, \vv{x}_j:\vv{a}_j, \dots, \vv{x}_i:\vv{a}_i, \dots, \vv{x}_n:\vv{a}_n,
  \end{align*}
  т.е. $\Delta$ се получава от $\Gamma$ като разменим местата на $i$-тата и $j$-тата двойка.
  Тогава за всеки терм $\tau$, такъв че $\Gamma \vdash \tau : \vv{a}$, е изпълено, че $\Delta \vdash \tau : \vv{a}$ и за всеки $(u_1,\dots,u_n) \in \val{\Gamma}$,
  \[\val{\tau}_\Gamma(u_1,\dots,u_i,\dots,u_j,\dots,u_n) = \val{\tau}_\Delta(u_1,\dots,u_j,\dots,u_i,\dots,u_n).\]
\end{proposition}
\begin{hint}
  Индукция по построението на терма $\tau$.
\end{hint}


%%% Local Variables:
%%% mode: latex
%%% TeX-master: "../sep"
%%% End:

\newpage
\begin{example}
  Нека $\vv{a} = \vv{nat} \to (\vv{nat} \to \vv{nat})$ и 
  \[\tau \equiv \fix\vv{(}\underbrace{\lamb{f}{a}{\overbrace{\lamb{x}{nat}{\lamb{y}{nat}{\ifelse{\vv{y == 0}}{\vv{0}}{\vv{x + (f x (y-1))} }}}}^{\tau'_0}}}_{\tau_0}\vv{)}.\]

  Знаем, че $\val{\tau} \in \Cont{\Nat_\bot}{\Cont{\Nat_\bot}{\Nat_\bot}}$, където
  \[\val{\tau} \df \lfp(\val{\tau_0}).\]

  Ясно е, че $\val{\tau_0} \in \Cont{\val{\vv{a}}}{\val{\vv{a}}}$ и
  $\val{\tau_0} = \curry(\val{\tau'_0}_{\vv{f:a}}) = \val{\tau'_0}_{\vv{f:a}}$ и
  сега пък ако положим
  \begin{align*}
    % & \tau_0 \equiv \lamb{f}{a}{\lamb{x}{nat}{\lamb{y}{nat}{\ifelse{\vv{y == 0}}{\vv{0}}{\vv{x + (f x (y-1))} }}}},\\
    % & \tau'_0 \equiv \lamb{x}{nat}{\lamb{y}{nat}{\ifelse{\vv{y == 0}}{\vv{0}}{\vv{x + (f x (y-1))}}}},\\
    & \tau''_0 \equiv \lamb{y}{nat}{\ifelse{\vv{y == 0}}{\vv{0}}{\vv{x + (f x (y-1))}}},\\
    & \tau'''_0 \equiv \ifelse{\vv{y == 0}}{\vv{0}}{\vv{x + (f x (y-1))}}.
  \end{align*}

  то ще получим, че
  \[\val{\tau'_0}_{\vv{f:a}} = \curry(\val{\tau''_0}_{\vv{f:a,x:nat}}),\]
  и тогава
  \[\val{\tau'_0}_{\vv{f:a}}(\varphi)(m) = \val{\tau''_0}_{\vv{f:a,x:nat}}(\varphi,m).\]
  Сега вече получаваме, че
  \[\val{\tau''_0}_{\vv{f:a,x:nat}} = \curry(\val{\tau'''_0}_{\vv{f:a,x:nat,y:nat}}),\]
  т.е.
  \[\val{\tau''_0}_{\vv{f:a,x:nat}}(\varphi,m)(n) = \val{\tau'''_0}_{\vv{f:a,x:nat,y:nat}}(\varphi,m,n).\]

  Обединявайки всичко получаваме, че:
  \begin{align*}
    \val{\tau_0}(\varphi)(m)(n) & = \val{\tau'_0}_{\vv{f:a}}(\varphi)(m)(n) \\
                                & = \val{\tau''_0}_{\vv{f:a,x:nat}}(\varphi,m)(n)\\
                                & = \val{\tau'''_0}_{\vv{f:a,x:nat,y:nat}}(\varphi,m,n)\\
                                & = \val{\ifelse{\vv{y==0}}{\vv{0}}{\vv{x + (f x (y-1))}}}(\varphi,m,n)\\
                                & = \texttt{if}(\val{\vv{y==0}}(\varphi,m,n), \val{\vv{0}}(\varphi,m,n),\val{\vv{x + (f x (y-1))}}(\varphi,m,n)).
                                % & = \texttt{if}(\eq(n,0),0,\plus(m, \texttt{eval}(\texttt{eval}(\varphi, m),n-1))).
  \end{align*}

  Накрая получаваме, че
  \[\val{\tau_0}(\varphi)(m)(n) = \begin{cases}
      0, & \text{ако }n = 0\\
      \plus(m, \varphi(m)(n-1)), & \text{ако } n > 0\\
      \bot, & \text{ако }n = \bot.
    \end{cases}
  \]
  
                                
  Сега вече знаем как по теоремата на Клини да докажем, че
  \[\lfp(\val{\tau_0})(m)(n) =
    \begin{cases}
      m*n,  & \text{ако }m,n\in\Nat\\
      \bot, & \text{иначе}
    \end{cases}
\]
                                
  
\end{example}

\begin{framed}
\begin{lemma}[Лема за замяната]\label{lem:pcf:substitution}
  Нека $\Gamma$ е типов контекст, $\tau$ и $\rho$ са термове, като $\vv{x} \not\in \Dom(\Gamma)$,
  \begin{align*}
    & \Gamma \vdash \rho : \vv{a}\\
    & \Gamma, \type{x}{a} \vdash \tau : \vv{b}.
  \end{align*}
  Тогава
  \begin{enumerate}[1)]
  \item
    $\Gamma \vdash \tau\subst{x}{\rho} : \vv{b}$;
  \item
    за всяко $\overline{u} \in \val{\Gamma}$,
    \[\val{\tau\subst{x}{\rho}}_\Gamma(\overline{u}) = \val{\tau}_{\Gamma,\type{x}{a}}(\overline{u},\val{\rho}_\Gamma(\overline{u})).\]
  \end{enumerate}
\end{lemma}
\end{framed}
\marginpar{Защо да не взема $\rho$ да бъде затворен терм ?}
\begin{proof}
  Индукция по построението на термовете.
  Започваме с базовият случай, който може да се разбие на три подслучая.
  \begin{itemize}
  \item
    Нека $\tau \equiv \vv{n}$.
    Тогава е ясно, че $\vv{b} = \nat$ и $\vv{n}\subst{x}{\rho} \equiv \vv{n}$.
    Оттук веднага получаваме, че
    \[\Gamma \vdash \tau\subst{x}{\rho} : \vv{b}.\]
    Също така,
    \begin{align*}
      \val{\tau\subst{x}{\rho}}_\Gamma(\ov{u}) & = \val{\vv{n}\subst{x}{\rho}}_\Gamma(\ov{u})\\
                                               & = \val{\vv{n}}_\Gamma(\ov{u})\\
                                               & = n \\
                                               & = \val{\vv{n}}_{\Gamma,\type{x}{a}}(\ov{u},v) & \comment\text{за каквото и да е }v\\
                                               & = \val{\vv{n}}_{\Gamma,\type{x}{a}}(\ov{u},\val{\rho}_\Gamma(\ov{u}))\\
                                               & = \val{\tau}_{\Gamma,\type{x}{a}}(\ov{u},\val{\rho}_\Gamma(\ov{u})).
    \end{align*}
  \item
    Нека $\tau \equiv \vv{x}_i$, където $\vv{x}_i \not\equiv \vv{x}$.
    Тогава, щом $\Gamma,\type{x}{a} \vdash \vv{x}_i:\vv{a}_i$, то $\vv{b} = \vv{a}_i$.
    Тук е ясно, че $\vv{x}_i\subst{x}{\rho} \equiv \vv{x}_i$ и тогава веднага получаваме, че
    \[\Gamma \vdash \tau\subst{x}{\rho} : \vv{b}.\]
    Освен това,
    \begin{align*}
      \val{\tau \subst{x}{\rho}}_\Gamma(\ov{u}) & = \val{\vv{x}_i\subst{x}{\rho}}_\Gamma(\overline{u})\\
                                                & = \val{\vv{x}_i}_\Gamma(\overline{u})\\
                                                & = u_i\\
                                                & = \val{\vv{x}_i}_{\Gamma,\type{x}{a}}(\overline{u},v) & \comment\text{за каквото и да е }v\\
                                                & = \val{\vv{x}_i}_{\Gamma,\type{x}{a}}(\overline{u},\val{\rho}_\Gamma(\ov{u}))\\
                                                & = \val{\tau}_{\Gamma,\type{x}{a}}(\overline{u},\val{\rho}_\Gamma(\ov{u})).
    \end{align*}
  \item
    Нека $\tau \equiv \vv{x}$. Тогава, щом $\Gamma, \type{x}{a} \vdash \type{x}{b}$, то $\vv{b} = \vv{a}$.
    Тук е ясно, че $\vv{x}\subst{x}{\rho} \equiv \rho$ и тогава веднага получаваме, че
    \[\Gamma \vdash \tau\subst{x}{\rho} : \vv{b}.\]
    Освен това,
    \begin{align*}
      \val{\tau\subst{x}{\rho}}_\Gamma(\ov{u}) & = \val{\vv{x}\subst{x}{\rho}}_\Gamma(\overline{u})\\
                                               & = \val{\rho}_\Gamma(\overline{u})\\
                                               & = \val{\vv{x}}_{\Gamma,\type{x}{a}}(\overline{u},\val{\rho}_\Gamma(\ov{u}))\\
                                               & = \val{\tau}_{\Gamma,\type{x}{a}}(\overline{u},\val{\rho}_\Gamma(\ov{u})).
    \end{align*}
  \end{itemize}

  Сега преминаваме към индукционната стъпка.
  \begin{itemize}
  \item
    Нека $\tau \equiv \tau_1 + \tau_2$.
    За първата част, щом $\Gamma,\type{x}{a} \vdash \tau_1+\tau_2 : \vv{b}$, то
    е ясно, че $\vv{b} = \vv{nat}$. Имаме, че
    \begin{prooftree}
      \AxiomC{$\Gamma, \type{x}{a} \vdash \tau_1 : \vv{nat}$}
      \AxiomC{$\Gamma, \type{x}{a} \vdash \tau_2 : \vv{nat}$}
      \RightLabel{\scriptsize{(plus)}}
      \BinaryInfC{$\Gamma, \type{x}{a} \vdash \tau_1 + \tau_2 : \vv{nat}$}
    \end{prooftree}
    Сега можем да приложим \IndHyp и получаваме следния извод:
    \begin{prooftree}
      \AxiomC{\scriptsize{(от условието)}}
      \UnaryInfC{$\Gamma \vdash \rho : \vv{a}$}
      \AxiomC{$\Gamma, \type{x}{a} \vdash \tau_1 : \vv{nat}$}
      \LeftLabel{\scriptsize{\IndHyp}}
      \BinaryInfC{$\Gamma \vdash \tau_1\subst{x}{\rho} : \vv{nat}$}
      \AxiomC{\scriptsize{(от условието)}}
      \UnaryInfC{$\Gamma \vdash \rho : \vv{a}$}
      \AxiomC{$\Gamma, \type{x}{a} \vdash \tau_2 : \vv{nat}$}
      \RightLabel{\scriptsize{\IndHyp}}
      \BinaryInfC{$\Gamma \vdash \tau_2\subst{x}{\rho} : \vv{nat}$}
      \RightLabel{\scriptsize{(plus)}}
      \BinaryInfC{$\Gamma \vdash \tau_1\subst{x}{\rho} + \tau_2 \subst{x}{\rho} : \vv{nat}$}
      \RightLabel{\scriptsize{(правила за замяна)}}
      \UnaryInfC{$\Gamma \vdash \tau\subst{x}{\rho} : \vv{nat}$}
    \end{prooftree}
    За втората част,
    \begin{align*}
      \val{\tau}_{\Gamma,\type{x}{a}}(\ov{u},\val{\rho}_\Gamma(\ov{u})) & = \val{\tau_1+\tau_2}_{\Gamma,\type{x}{a}}(\ov{u},\val{\rho}_\Gamma(\ov{u}))\\
                                                                        & = \plus(\val{\tau_1}_{\Gamma,\type{x}{a}}(\ov{u},\val{\rho}_\Gamma(\ov{u})),\val{\tau_1}_{\Gamma,\type{x}{a}}(\ov{u},\val{\rho}_\Gamma(\ov{u})))\\
                                                                        & = \plus(\val{\tau_1\subst{x}{\rho}}_\Gamma(\ov{u}),\val{\tau_2\subst{x}{\rho}}_\Gamma(\ov{u})) & \comment\text{\IndHyp за $\tau_1$ и $\tau_2$}\\
                                                                        & = \val{\tau_1\subst{x}{\rho}+\tau_2\subst{x}{\rho}}_\Gamma(\ov{u})\\
                                                                        & = \val{\tau\subst{x}{\rho}}_\Gamma(\ov{u}).
    \end{align*}
  \item
    Нека $\tau \equiv \tau_1 - \tau_2$.
  \item
    Нека $\tau \equiv \tau_1\ \vv{==}\ \tau_2$.
  \item
    Нека $\tau \equiv \ifelse{\tau_1}{\tau_2}{\tau_3}$.
  \item
    Нека $\tau \equiv \tau_1 \tau_2$.
    Тук първата част е лесна. Понеже имаме, че
    \begin{prooftree}
      \AxiomC{$\Gamma, \type{x}{a} \vdash \tau_1: \vv{c} \to \vv{b}$}
      \AxiomC{$\Gamma, \type{x}{a} \vdash \tau_2: \vv{c}$}
      \RightLabel{\scriptsize{(app)}}
      \BinaryInfC{$\Gamma, \type{x}{a} \vdash \tau_1 \tau_2 : \vv{b}$}
    \end{prooftree}
    то можем да приложим \IndHyp за да получим, че
    \begin{prooftree}
      \AxiomC{\scriptsize{от условието}}
      \UnaryInfC{$\Gamma \vdash \rho : \vv{a}$}
      \AxiomC{$\Gamma, \type{x}{a} \vdash \tau_1: \vv{c} \to \vv{b}$}
      \LeftLabel{\scriptsize{\IndHyp}}
      \BinaryInfC{$\Gamma \vdash \tau_1\subst{\vv{x}}{\rho} : \vv{b}$}
      \AxiomC{\scriptsize{от условието}}
      \UnaryInfC{$\Gamma \vdash \rho : \vv{a}$}
      \AxiomC{$\Gamma, \type{x}{a} \vdash \tau_2: \vv{c} \to \vv{b}$}
      \RightLabel{\scriptsize{\IndHyp}}
      \BinaryInfC{$\Gamma \vdash \tau_2\subst{x}{\rho} : \vv{c}$}
      \RightLabel{\scriptsize{(app)}}
      \BinaryInfC{$\Gamma \vdash \tau_1\subst{x}{\rho}(\tau_2\subst{x}{\rho}) : \vv{c}$}
      \RightLabel{\scriptsize{(правила на замяна)}}
      \UnaryInfC{$\Gamma \vdash \tau\subst{x}{\rho} : \vv{c}$}
    \end{prooftree}
    Втората част също е лесна.
    \begin{align*}
      \val{\tau}_{\Gamma,\type{x}{a}}(\ov{u},\val{\rho}_\Gamma(\ov{u})) & = \val{\tau_1\tau_2}_{\Gamma,\type{x}{a}}(\ov{u},\val{\rho}_\Gamma(\ov{u})) \\
                                                                        & \df \texttt{eval}(\val{\tau_1}_{\Gamma,\type{x}{a}}(\ov{u},\val{\rho}_\Gamma(\ov{u})), \val{\tau_2}_{\Gamma,\type{x}{a}}(\ov{u},\val{\rho}_\Gamma(\ov{u}))) & \comment\text{\Def{eval}}\\
                                                                        & = \texttt{eval}(\val{\tau_1\subst{x}{\rho}}_\Gamma(\ov{u}), \val{\tau_2\subst{x}{\rho}}_\Gamma(\ov{u})) & \comment\text{\IndHyp за $\tau_1$ и $\tau_2$}\\
                                                                        & = \val{\tau_1\subst{x}{\rho}(\tau_2\subst{x}{\rho})}_\Gamma(\ov{u})\\
                                                                        & = \val{\tau\subst{x}{\rho}}_\Gamma(\ov{u}).
    \end{align*}
    
  \item
    Нека $\tau \equiv \fix(\tau')$.
    Първо трябва да докажем, че $\Gamma \vdash \tau[\vv{x}/\rho] : \vv{b}$.
    От правилата за типизиране е ясно, че имаме следния извод:
    \begin{prooftree}
      \AxiomC{$\Gamma, \type{x}{a} \vdash \tau':\vv{b}\to\vv{b}$}
      \RightLabel{\scriptsize{(fix)}}
      \UnaryInfC{$\Gamma, \type{x}{a} \vdash \fix(\tau') : \vv{b}$}
    \end{prooftree}
    Сега можем да приложим \IndHyp за терма $\tau'$. Получаваме, че
    \begin{prooftree}
      \AxiomC{\scriptsize{от условието}}
      \UnaryInfC{$\Gamma \vdash \rho: \vv{a}$}
      \AxiomC{$\Gamma, \type{x}{a} \vdash \tau':\vv{b}\to\vv{b}$}
      \RightLabel{\scriptsize{\IndHyp}}
      \BinaryInfC{$\Gamma \vdash \tau'\subst{x}{\rho} : \vv{b} \to \vv{b}$}
      \RightLabel{\scriptsize{(fix)}}
      \UnaryInfC{$\Gamma \vdash \fix(\tau'\subst{x}{\rho}) : \vv{b}$}
      \RightLabel{\scriptsize{(правила за замяна)}}
      \UnaryInfC{$\Gamma \vdash \fix(\tau')\subst{x}{\rho} : \vv{b}$}
    \end{prooftree}

    За втората част, получаваме следната верига от равенства:
    \begin{align*}
      \val{\tau\subst{x}{\rho}}_{\Gamma}(\ov{u}) & = \val{\fix(\tau'\subst{x}{\rho})}_{\Gamma}(\ov{u})\\
                                                 & = \lfp(\val{\tau'\subst{x}{\rho}}_\Gamma(\ov{u})) & \comment\text{от деф.}\\
                                                 & = \lfp(\val{\tau'}_{\Gamma,\type{x}{a}}(\ov{u},\val{\rho}_\Gamma(\ov{u}))) & \comment\text{от \IndHyp за }\tau'\\
                                                 & = \val{\fix(\tau')}_{\Gamma,\type{x}{a}}(\ov{u},\val{\rho}_\Gamma(\ov{u}))\\
                                                 & = \val{\tau}_{\Gamma,\type{x}{a}}(\ov{u},\val{\rho}_\Gamma(\ov{u})).
    \end{align*}
    
  \item
    \marginpar{Тук е важно, че \[\val{\Delta} = \val{\Gamma} \times \val{\vv{a}_n}.\]
      % За да бъде всичко максимално изчистено от формална гледна точка, трябва да докажем и следното:
      % \begin{prooftree}
        
      % \end{prooftree}
      
    }

    Нека $\tau \equiv \lamb{y}{c}{\tau'}$, където $\vv{y} \not\in \Dom(\Gamma) \cup \{\vv{x}\}$.
    Първо трябва да докажем, че $\Gamma \vdash \tau[\vv{x}/\rho] : \vv{b}$.
    
    От правилата за типизиране е ясно, че щом $\Gamma, \type{x}{a} \vdash \tau : \vv{b}$, то
    типът $\vv{b}$ е такъв, че $\vv{b} = \vv{c} \to \vv{d}$, за някой тип $\vv{d}$, и имаме извода:
    \begin{prooftree}
      \AxiomC{$\vv{y} \not\in\Dom(\Gamma)\cup\{\vv{x}\}$}
      \AxiomC{$\Gamma, \type{x}{a}, \type{y}{c} \vdash \tau':\vv{d}$}
      \RightLabel{\scriptsize{(lambda)}}
      \BinaryInfC{$\Gamma, \type{x}{a} \vdash \lamb{y}{c}{\tau'} : \vv{c}\to\vv{d}$}
    \end{prooftree}
    Това означава, че можем да използваме \IndHyp за терма $\tau'$ и така получаваме, че
    \begin{prooftree}
      \AxiomC{$\vv{y} \not\in \Dom(\Gamma)$}
      \AxiomC{$\vv{y} \not\in \Dom(\Gamma)$}
      \AxiomC{\scriptsize{от условието}}
      \UnaryInfC{$\Gamma \vdash \rho : \vv{a}$}
      \BinaryInfC{$\Gamma,\type{y}{c} \vdash \rho : \vv{a}$}
      \AxiomC{$\Gamma, \type{x}{a}, \type{y}{c} \vdash \tau':\vv{d}$}
      \UnaryInfC{$\Gamma, \type{y}{c}, \type{x}{a} \vdash \tau':\vv{d}$}
      \RightLabel{\scriptsize{\IndHyp}}
      \BinaryInfC{$\Gamma, \type{y}{c} \vdash \tau'\subst{x}{\rho} : \vv{d}$}
      \RightLabel{\scriptsize{(lambda)}}
      \BinaryInfC{$\Gamma \vdash \lamb{y}{c}{\tau'\subst{x}{\rho}}:\vv{c}\to\vv{d}$}
      \RightLabel{\scriptsize{(правила за замяна)}}
      \UnaryInfC{$\Gamma \vdash (\lamb{y}{c}{\tau'})\subst{x}{\rho}:\vv{c}\to\vv{d}$}
    \end{prooftree}

    За втората част, понеже имаме, че $\Gamma,\type{y}{c} \vdash \tau'\subst{x}{\rho} : \vv{d}$,
    то можем да приложим \IndHyp за $\tau'$ и така получаваме, че за всяко $\overline{u},v \in \val{\Gamma,\type{y}{c}}$,
%    \setlength{\jot}{10pt}
    \begin{align*}
      \val{\tau}_{\Gamma,\type{y}{c}}(\ov{u})(v) & = \curry(\val{\tau'\subst{x}{\rho}}_{\Gamma,\type{y}{c}})(\ov{u})(v)\\
                                                   & \df \val{\tau'\subst{x}{\rho}}_{\Gamma,\type{y}{c}}(\overline{u},v) & \comment\text{\Def{curry}}\\
                                                   & = \val{\tau'}_{\Gamma,\type{y}{c}, \type{x}{a}}(\overline{u},v,\val{\rho}_{\Gamma,\type{y}{c}}(\ov{u},v)) & \comment\text{\IndHyp}\\
                                                   & = \val{\tau'}_{\Gamma,\type{y}{c}, \type{x}{a}}(\ov{u},v,\val{\rho}_\Gamma(\ov{u})) & \comment \fv(\rho) \subseteq \Dom(\Gamma)\\
                                                   & = \val{\tau'}_{\Gamma,\type{x}{a},\type{y}{c}}(\ov{u},\val{\rho}_\Gamma(\ov{u}),v) \\
                                                   & = \curry(\val{\tau'}_{\Gamma,\type{x}{a},\type{y}{c}})(\ov{u},\val{\rho}_\Gamma(\ov{u}))(v) \\
                                                   & = \val{\lamb{y}{c}{\tau'}}_{\Gamma,\type{x}{a}}(\ov{u},\val{\rho}_\Gamma(\ov{u}))(v)\\
                                                   & = \val{\tau}_{\Gamma,\type{x}{a}}(\ov{u},\val{\rho}_\Gamma(\ov{u}))(v).\\
    \end{align*}
    Така получихме, че
    \[\val{\tau}_{\Gamma}(\ov{u}) = \val{\tau}_{\Gamma,\type{x}{a}}(\ov{u},\val{\rho}_\Gamma(\ov{u})).\]
  \end{itemize}
\end{proof}



%%% Local Variables:
%%% mode: latex
%%% TeX-master: "../sep"
%%% End:

Добре е още тук да разгледаме термове, които ще ни бъдат полезни по-нататък, когато искаме да изучим свойствата на операционната и денотационната семантики на \PCF.

\begin{framed}
  \begin{definition}
    За произволен тип $\vv{a}$, да означим затворените термове
    \begin{align*}
      & \Omega_{\vv{a}} \df \fix(\lamb{x}{a}{\vv{x}})\\
      & \Omega'_{\vv{a}} \df \lamb{x}{a}{\fix(\lamb{x}{a}{\vv{x}})}.
    \end{align*}
  \end{definition}
\end{framed}

\begin{problem}
  \label{prob:pcf:context:omega}
  \marginpar{\todo Студентите би трябвало да могат да докажат твърденията в \Problem{pcf:context:omega} сами!}
  Докажете, че за всеки тип $\vv{a}$ са изпълнени следните свойства:
  \begin{enumerate}[(1)]
  \item
    \label{pcf:omega:type}
    $\emptyset \vdash \Omega_{\vv{a}} : \vv{a}$ и $\emptyset \vdash \Omega'_{\vv{a}} : \vv{a} \to \vv{a}$;
  \item
    \label{pcf:omega:operational}
    $\Omega_{\vv{a}} \not\Downarrow_{\vv{a}}$ и $\Omega'_{\vv{a}} \Downarrow^0_{\vv{a}\to\vv{a}} \Omega'_{\vv{a}}$;
  \item
    \label{pcf:omega:denotational}
    $\val{\Omega_{\vv{a}}} = \bot^{\val{\vv{a}}}$ и $\val{\Omega'_{\vv{a}}} = \val{\Omega_{\vv{a}\to\vv{a}}} = \bot^{\val{\vv{a}\to\vv{a}}}$.    
  \end{enumerate}
\end{problem}
\begin{hint}
  Доказателството на Свойство~\ref{pcf:omega:type} представлява едно просто упражнение, което за пълнота на изложението ще направим.

  \begin{figure}[H]
    \begin{subfigure}{0.5\textwidth}
      \begin{prooftree}
        \AxiomC{$\type{x}{a} \vdash \type{x}{a}$}
        \UnaryInfC{$\emptyset \vdash \lamb{x}{a}{\vv{x}} : \vv{a}\to\vv{a}$}
        \UnaryInfC{$\emptyset \vdash \underbrace{\fix(\lamb{x}{a}{\vv{x}})}_{\Omega_{\vv{a}}} : \vv{a}$}
      \end{prooftree}
      % \caption{}
      % \label{fig:pcf:context:omega}
    \end{subfigure}
    ~
    \begin{subfigure}{0.5\textwidth}
      \begin{prooftree}
        % \AxiomC{от (\ref{fig:pcf:context:omega})}
        \AxiomC{вече доказано}
        \UnaryInfC{$\emptyset \vdash \Omega_{\vv{a}} : \vv{a}$}
        \UnaryInfC{$\type{x}{a} \vdash \Omega_{\vv{a}} : \vv{a}$}
        \UnaryInfC{$\underbrace{\emptyset \vdash \lamb{x}{a}{\Omega_{\vv{a}}}}_{\Omega'_{\vv{a}}} : \vv{a} \to \vv{a}$}
      \end{prooftree}
      % \caption{}
    \end{subfigure}
  \end{figure}
  
  \marginpar{Втората част на Свойство~\ref{pcf:omega:operational} не заслужава внимание, защото $\Omega'_{\vv{a}}$ е стойност и следователно по дефиниция $\Omega'_{\vv{a}} \opsemGen{0}{\vv{a}\to\vv{a}} \Omega'_{\vv{a}}$.}
  За първата част на Свойство~\ref{pcf:omega:operational}, да допуснем, че $\Omega_{\vv{a}} \Downarrow_{\vv{a}} \vv{v}$, за някоя стойност $\type{v}{a}$, и нека фиксираме $\ell$
  да бъде {\em най-малкият} брой стъпки, за които $\Omega_{\vv{a}} \Downarrow^{\ell}_{\vv{a}} \vv{v}$.
  Но тогава
  \begin{prooftree}
    \AxiomC{$\lamb{x}{a}{\vv{x}}$ е стойност}
    \UnaryInfC{$\lamb{x}{a}{\vv{x}} \Downarrow^0_{\vv{a}\to\vv{a}} \lamb{x}{a}{\vv{x}}$}
    \AxiomC{$\Omega_{\vv{a}} \Downarrow^{\ell-2}_{\vv{a}} \vv{v}$}
    \UnaryInfC{$\vv{x}\subst{x}{\fix(\lamb{x}{a}{\vv{x}})} \Downarrow^{\ell-2}_{\vv{a}} \vv{v}$}
    \RightLabel{\scriptsize{(cbn)}}
    \BinaryInfC{$(\lamb{x}{a}{\vv{x}})\fix(\lamb{x}{a}{\vv{x}}) \Downarrow^{\ell-1}_{\vv{a}} \vv{v}$}
    \RightLabel{\scriptsize{(fix)}}
    \UnaryInfC{$\underbrace{\fix(\lamb{x}{a}{\vv{x}})}_{\Omega_{\vv{a}}} \Downarrow^\ell_{\vv{a}} \vv{v}$}
  \end{prooftree}
  Получихме, че $\Omega_{\vv{a}} \Downarrow^{\ell-2}_{\vv{a}} \vv{v}$, което е противоречие с минималността на $\ell$.

  Сега да разгледаме първата част на Свойство~\ref{pcf:omega:denotational}. За произволен тип $\vv{a}$, да означим с $\texttt{id}_{\val{\vv{a}}}$ функцията идентитет за областта на Скот $\val{\vv{a}}$, т.е.
  $\texttt{id}_{\val{\vv{a}}}(x) = x$ за всяко $x \in \val{\vv{a}}$. Имаме, че:
  \marginpar{Да напомним, че $f^0 \df id$ и $f^{n+1} \df f \circ f^n$.
    В нашия случай, $f = id$ и следователно $id^n = id$ за всяко $n$.}
  \begin{align*}
    \val{\Omega_{\vv{a}}} & = \val{\fix(\lamb{x}{a}{\vv{x}})}\\
                          & = \lfp(\val{\lamb{x}{a}{\vv{x}}})\\
                          & = \lfp(\texttt{id}_{\val{\vv{a}}}) & \comment\val{\lamb{x}{a}{\vv{x}}} = \texttt{id}_{\val{\vv{a}}}\\
                          & = \bigsqcup_n\texttt{id}^n_{\val{\vv{a}}}(\bot^{\val{\vv{a}}}) & \comment\text{\hyperref[th:knaster-tarski]{Теоремата на Клини}}\\
                          & = \bigsqcup_n \bot^{\val{\vv{a}}} & \comment \texttt{id}^n(\bot^{\val{\vv{a}}}) = \bot^{\val{\vv{a}}}\\
                          & = \bot^{\val{\vv{a}}}.
  \end{align*}
  Сега преминаваме към втората част на Свойство~\ref{pcf:omega:denotational}. За произволен елемент $u \in \val{\vv{a}}$,
  \marginpar{Да напомним, че за всяко $u$, $\bot^{\val{\vv{a}\to\vv{a}}}(u) \df \bot^{\val{\vv{a}}}$.}
  \begin{align*}
    \val{\Omega'_a}(u) & = \val{\lamb{x}{a}{\Omega_{\vv{a}}}}(u)\\
                       & = \curry(\val{\Omega_{\vv{a}}}_{\type{x}{a}})(u)\\
                       & = \val{\Omega_{\vv{a}}}_{\type{x}{a}}(u)\\
                       & = \val{\Omega_{\vv{a}}}\\
                       & = \bot^{\val{\vv{a}}}.
  \end{align*}

  Заключаваме, че $\val{\Omega'_a} = \bot^{\val{\vv{a}\to\vv{a}}}$.
\end{hint}

\newpage
\section{Коректност}

\marginpar{Да напомним, че когато термът $\tau$ е затворен, то ще пишем $\val{\tau}$ вместо $\val{\tau}_\emptyset(\bot)$.}
Понеже вече имаме дефинирани операционна и денотационна семантика на термовете,
следващата стъпка е да разгледаме каква е връзката между тях.
В този раздел ще докажем едната (по-лесната) посока.
\marginpar{На англ. {\em soundness}.}
\begin{framed}
  \begin{theorem}[Теорема за коректност]\label{th:pcf:soundness}
    За всеки затворен терм $\tau : \vv{b}$ и стойност $\type{v}{b}$, е изпълнена импликацията:
    \[\tau \opsem{}{b} \vv{v}\ \implies\ \val{\tau} = \val{\vv{v}} \in \val{\vv{b}}.\]
  \end{theorem}  
\end{framed}
\begin{proof}
  Индукция по дължината $\ell$ на извода $\opsem{\ell}{b}$ за всеки тип $\vv{b}$.
  Нека $\ell = 0$. Имаме два случая, защото имаме два вида стойности.
  \begin{itemize}
  \item
    Нека $\tau \equiv \vv{n}$.
    Ясно е, че $\vv{b} = \vv{nat}$ и от правилата на операционната семантика имаме, че:
    \begin{prooftree}
      \AxiomC{}
      \RightLabel{\scriptsize{(val)}}
      \UnaryInfC{$\tau \opsem{0}{nat} \vv{n}$}
    \end{prooftree}
    От дефиницията на семантика на терм, директно получаваме, че
    $\val{\tau} = n = \val{\vv{n}}$.    
  \item
    Нека $\tau \equiv \lamb{x}{c}{\tau'}$. Тогава $\vv{b} = \vv{a}\to\vv{c}$ и от правилата на операционната семантика имаме, че:
    \begin{prooftree}
      \AxiomC{}
      \RightLabel{\scriptsize{(val)}}
      \UnaryInfC{$\tau \opsemGen{0}{\vv{a}\to\vv{c}} \tau$}
    \end{prooftree}
    Ясно е, че $\val{\tau} = \val{\tau} \in \val{b}$.
  \end{itemize}
  Така доказахме, че
  \[\tau \opsem{0}{b} \vv{v}\ \implies\ \val{\tau} = \val{\vv{v}} \in \val{\vv{b}}.\]
  Нека сега $\ell > 0$ и да приемем, че имаме следното индукционно предположение:
  \[\tau \opsem{<\ell}{b} \vv{v}\ \implies\ \val{\tau} = \val{\vv{v}} \in \val{\vv{b}}.\]
  Ще докажем, че
  \[\tau \opsem{\ell}{b} \vv{v}\ \implies\ \val{\tau} = \val{\vv{v}} \in \val{\vv{b}}.\]
  \begin{itemize}
  \item
    Нека $\tau \equiv \tau_1 + \tau_2$. Тогава от правилата на операционната семантика имаме, че:
    \begin{prooftree}
      \AxiomC{$\tau_1 \opsem{\ell_1}{nat} \vv{n}_1$}
      \AxiomC{$\tau_2 \opsem{\ell_2}{nat} \vv{n}_2$}
      \LeftLabel{\scriptsize{($\ell=\ell_1+\ell_2+1$)}}
      \RightLabel{\scriptsize{(plus)}}
      \BinaryInfC{$\tau_1 + \tau_2 \opsem{\ell_1+\ell_2+1}{nat} \vv{n},$}
    \end{prooftree}
    където $n = n_1 + n_2$. От \IndHyp получаваме, че
    \begin{align*}
      & \val{\tau_1} = \val{\vv{n}_1} = n_1\\
      & \val{\tau_2} = \val{\vv{n}_2} = n_2.
    \end{align*}
    Тогава
    \begin{align*}
      \val{\tau_1 + \tau_2} & = \plus(\val{\tau_1}, \val{\tau_2}) & \comment\text{от деф.}\\
                            & = n_1 + n_2 & \comment\text{\IndHyp}\\
                            & = n.
    \end{align*}
  \item
    Случаите $\tau \equiv \tau_1 - \tau_2$ и $\tau \equiv \tau_1\ \vv{==}\ \tau_2$ са аналогични. Оставяме ги на читателя.
  \item
    Нека $\tau \equiv \ifelse{\tau_1}{\tau_2}{\tau_3}$. Тогава от правилата на операционната семантика имаме, че:
    \begin{prooftree}
      \AxiomC{$\tau_1 \opsem{\ell_1}{nat} \vv{n}_1$}
      \AxiomC{$\tau_2 \opsem{\ell_2}{a} \vv{v}_2$}
      \AxiomC{$\vv{n}_1 \not\equiv \vv{0}$}
      \LeftLabel{\scriptsize{($\ell=\ell_1+\ell_2+1$)}}
      \RightLabel{\scriptsize{(if$^+$)}}
      \TrinaryInfC{$\ifelse{\tau_1}{\tau_2}{\tau_3} \opsem{\ell_1+\ell_2+1}{a} \vv{v}_2,$}
    \end{prooftree}
    Тогава от \IndHyp получаваме, че:
    \begin{align*}
      & \val{\tau_1} = n_1\\
      & \val{\tau_2} = \val{\vv{v}_2}.
    \end{align*}
    Тогава
    \begin{align*}
      \val{\ifelse{\tau_1}{\tau_2}{\tau_3}} & = \texttt{if}(\val{\tau_1}, \val{\tau_2}, \val{\tau_3})\\
                                            & = \texttt{if}(n_1,\val{\tau_2}, \val{\tau_3}) & \comment\text{от \IndHyp}\\
                                            & = \val{\tau_2} & \comment\text{от деф. на }\texttt{if}\\
                                            & = \val{\vv{v}_2}. & \comment\text{от \IndHyp}
    \end{align*}
    
    Случаят, когато $\vv{n}_1 \equiv \vv{0}$ е аналогичен.
  \item
    Нека $\tau \equiv \tau_1 \tau_2$. Тогава от правилата на операционната семантика имаме, че:
    \begin{prooftree}
      \AxiomC{$\tau_1 \opsemGen{\ell_1}{\vv{a}\to\vv{b}} \lamb{x}{a}{\tau'_1}$}
      \AxiomC{$\tau'_1[x/\tau_2] \opsem{\ell_2}{b} \vv{v}$}
      \LeftLabel{\scriptsize{($\ell=\ell_1+\ell_2+1$)}}
      \RightLabel{\scriptsize{(cbn)}}
      \BinaryInfC{$\tau_1 \tau_2 \opsem{\ell_1+\ell_2+1}{b} \vv{v} $}
    \end{prooftree}
    Тогава от \IndHyp получаваме, че:    
    \begin{align*}
      & \val{\tau_1} = \val{\lamb{x}{a}{\tau'_1}} \in \Cont{\val{\vv{a}}}{\val{\vv{b}}}\\
      & \val{\tau'_1\subst{x}{\tau_2}} = \val{\vv{v}} \in \val{\vv{b}}.
    \end{align*}
    Обядиняваме всичко и получаваме равенствата:
    \begin{align*}
      \val{\tau_1\tau_2} & = \texttt{eval}(\val{\tau_1},\val{\tau_2}) & \comment\text{от деф.}\\ 
                         & = \val{\tau_1}(\val{\tau_2}) & \comment \val{\tau_1} \in \Cont{\val{\vv{a}}}{\val{\vv{b}}}\\
                         & = \val{\lamb{x}{a}{\tau'_1}}(\val{\tau_2}) & \comment\text{\IndHyp}\\
                         & = \val{\tau'_1}_{\type{x}{a}}(\val{\tau_2})\\
                         & = \val{\tau'_1\subst{x}{\tau_2}} & \comment\text{от \hyperref[lem:pcf:substitution]{Лема за замяната}}\\
                         & = \val{\vv{v}} & \comment\text{\IndHyp}
    \end{align*}
  \item
    Нека $\tau \equiv \fix(\tau')$. Тогава от правилата на операционната семантика имаме, че:
    \begin{prooftree}
      \AxiomC{$\tau'\ \fix(\tau') \opsem{\ell-1}{a} \vv{v}$}
      \RightLabel{\scriptsize{(fix)}}
      \UnaryInfC{$\fix(\tau') \opsem{\ell}{a} \vv{v} $}
    \end{prooftree}
    Тогава от \IndHyp имаме, че:
    \[\val{\tau'\ \fix(\tau')} = \val{\vv{v}}.\]
    От правилата за типизиране знаем, че $\tau' : \vv{a}\to\vv{a}$.
    Сега остава да съобразим, че щом $\val{\tau'} \in \Cont{\val{\vv{a}}}{\val{\vv{a}}}$,
    то изображението $\val{\tau'}$ притежава най-малка неподвижна точка $\lfp(\val{\tau'})$.
    Знаем, че според дефиницията на неподвижна точка, $\val{\tau'}(\lfp(\val{\tau'}) = \lfp(\val{\tau'})$.
    Сега сме готови да завършим доказателството:
    \begin{align*}
      \val{\tau} & = \val{\fix(\tau')} \\
                 & = \lfp(\val{\tau'})& \comment\text{от деф.}\\
                 & = \val{\tau'}(\lfp(\val{\tau'})) & \comment\text{неподв. точка}\\
                 & = \val{\tau'}(\val{\fix(\tau')}) & \comment\text{от деф.}\\
                 & = \texttt{eval}(\val{\tau'}, \val{\fix(\tau')})\\
                 & = \val{\tau'\fix(\tau')} & \comment\text{от деф.}\\
                 & = \val{\vv{v}}. & \comment\text{\IndHyp}
    \end{align*}
  \end{itemize}
\end{proof}

\hyperref[th:pcf:soundness]{Теоремата за коректност}\ частично потвърждава нашата интуиция, че за типа $\vv{nat}$
можем да си мислим за $\bot^{\val{\vv{nat}}}$ като за изчисление, което никога не завършва.

\marginpar{Другата посока ще я получим след малко.}

\begin{framed}
  \begin{corollary}
    \label{cor:pcf:soundness}
    Нека $\tau$ е затворен терм от тип $\vv{nat}$. Тогава е изпълнена импликацията
    \[\val{\tau} = \bot^{\val{\vv{nat}}}\ \implies\ \tau \not\opsem{}{nat}.\]
  \end{corollary}
\end{framed}

За жалост, тази наша интуиция се ,,губи'', когато се интересуваме от термове от по-висок от $\nat$ тип.
\marginpar{За дискусия по този въпрос вижте \cite[стр. 213]{models-of-computation}.}
Нека просто да разгледаме терма $\Omega'_{\vv{a}}$ от тип $\vv{a}\to\vv{a}$.
Имаме, че $\val{\Omega'_{\vv{a}}} = \bot^{\val{\vv{a}\to\vv{a}}}$, но
$\Omega'_{\vv{a}}\opsemGen{}{\vv{a}\to\vv{a}} \Omega'_{\vv{a}}$.



% Да разгледаме един пример. Нека 
% \[\tau \equiv \lamb{y}{nat}{\fix(\lamb{x}{nat}{\vv{x}})}.\]
% Лесно се съобразява, че $\tau : \nat\to\nat$.
% За произволен елемент $a \in \Nat_\bot$ е изпъленено следното:
% \begin{equation*}
%   \setlength{\jot}{10pt}
%   \begin{split}
%     \val{\tau}(a) & = \val{\lamb{y}{nat}{\fix(\lamb{x}{nat}{\vv{x}})}}(a)\\
%     & = \curry(\val{\fix(\lamb{x}{nat}{\vv{x}})}_{\type{y}{nat}})(a)\\
%     & = \val{\fix(\lamb{x}{nat}{\vv{x}})}_{\type{y}{nat}}(a)\\
%     & = \lfp(\val{\lamb{x}{nat}{\vv{x}}}_{\type{y}{nat}}(a))\\
%     & = \lfp(\curry(\val{\vv{x}}_{\type{y}{nat},\type{x}{nat}})(a)).
%   \end{split}
% \end{equation*}
% Сега, ясно е, че
% \begin{align*}
%   & \val{\vv{x}}_{\type{y}{nat},\type{x}{nat}} \in \Cont{\Nat_\bot\times\Nat_\bot}{\Nat_\bot},\\
%   & \curry(\val{\vv{x}}_{\type{y}{nat},\type{x}{nat}}) \in \Cont{\Nat_\bot}{\Cont{\Nat_\bot}{\Nat_\bot}}.
% \end{align*}
% За произволни елементи $a,b\in\Nat_\bot$, имаме следното:
% \begin{align*}
%   \curry(\val{\vv{x}}_{\type{y}{nat},\type{x}{nat}})(a)(b) & = \val{\vv{x}}_{\type{y}{nat},\type{x}{nat}}(a,b)\\
%                                                            & = b.
% \end{align*}
% \marginpar{$\texttt{id}_{\val{\nat}}(b) = b$ за всяко $b \in \Nat_\bot$.}
% Следователно, ако означим с $\texttt{id}_{\val{\nat}}$ функцията идентитет върху $\Nat_\bot$, то за всяко $a \in \Nat_\bot$,
% \[\curry(\val{\vv{x}}_{\type{y}{nat},\type{x}{nat}})(a) = \texttt{id}_{\val{\nat}}.\]
% Сега завършваме горната верига от равенства така:
% \begin{equation*}
%   \setlength{\jot}{10pt}
%   \begin{split}
%     \val{\tau}(a) & = \lfp(\curry(\val{\vv{x}}_{\type{y}{nat},\type{x}{nat}})(a))\\
%     & = \lfp(\texttt{id}_{\val{\nat}})\\
%     & = \bot^{\val{\nat}}.
%   \end{split}
% \end{equation*}
\marginpar{Това означава, че ако искаме денотационната семантика да кореспондира по-точно с операционната семантика, ние трябва да въведем нов елемент за $\bot$ за области на Скот съответстващи на типове по-високи от $\nat$.}
% С други думи, получаваме, че
% \[\val{\tau} = \bot^{\val{\nat\to\nat}}.\]
% От друга страна, обаче, $\tau$ представлява стойност. Следователно,
% \[\tau \opsemGen{0}{\nat\to\nat} \tau.\]


%%% Local Variables:
%%% mode: latex
%%% TeX-master: "../sep"
%%% End:

\newpage
\section{Адекватност}
\marginpar{Adequacy ???}
Нашата цел в този раздел е да докажем следната теорема.
\begin{framed}
  \begin{theorem}[Теорема за адекватност]
    За всеки затворен терм $\tau : \vv{nat}$ е изпълнена импликацията
    \[\val{\tau} = n \neq \bot^{\val{\nat}} \implies \tau \Downarrow_{\vv{nat}} \vv{n}.\]
  \end{theorem}
\end{framed}
\marginpar{Тук $n$ е число, а $\vv{n}$ е константа.}

Оказва се, че доказателството на тази теорема не е леко.
Ще започнем като дефинираме за всеки тип $\vv{a}$ релацията 
$\triangleleft_{\vv{a}} \subseteq \val{\vv{a}} \times \vv{PCF}_{\vv{a}}$
с индукция по построението на типовете.

\begin{itemize}
\item
  \marginpar{Съобразете, че теоремата за адекватност на практика гласи, че $\val{\tau} \triangleleft_{\vv{nat}} \tau$.}
  Нека $\vv{a} = \vv{nat}$. Тогава 
  \marginpar{Обикновено $\triangleleft_{\vv{a}}$ се нарича \emph{логическа релация}.
    В \cite[стр. 210]{models-of-computation} е обяснено защо имаме нужда от тези релации за да докажем теоремата за адекватност. В \cite[стр. 134]{gunter} е представен синтактичен подход към решаването на този проблем.}
  \[n \triangleleft_{\vv{nat}} \tau \dff ( n\neq\bot^{\val{\vv{nat}}} \implies \tau \Downarrow_{\vv{nat}} \vv{n}).\]
\item
  Нека $\vv{a} = \vv{b} \to \vv{c}$. Тогава 
  \[f \triangleleft_{\vv{b}\to\vv{c}} \tau \dff (\forall e\in \val{\vv{b}})(\forall \mu \in \vv{PCF}_{\vv{b}})[\ e \triangleleft_{\vv{b}} \mu \implies f(e) \triangleleft_{\vv{c}} \tau(\mu)\ ].\]
\item
  Нека $\Gamma = \vv{x}_1:\vv{a}_1, \dots, \vv{x}_n:\vv{a}_n$. Тогава 
  \[(u_1,\dots,u_n) \triangleleft_\Gamma (\tau_1,\dots,\tau_n) \dff u_1 \triangleleft_{\vv{a}_1} \tau_1\ \&\ \cdots\ \&\ u_n \triangleleft_{\vv{a}_n} \tau_n.\]
\end{itemize}

\begin{example}
  Да проверим внимателно защо е изпълнено, че:
  \[\texttt{id}_{\val{\nat}} \triangleleft_{\vv{nat}\to\vv{nat}} \lamb{x}{nat}{\vv{x + 0}}.\]
  Според дефиницията трябва да проверим импликацията
  \[e \triangleleft_{\nat} \mu \implies \texttt{id}_{\val{\nat}}(e) \triangleleft_{\nat} \tau(\mu),\]
  за произволен елемент $e \in \Nat_\bot$ и произволен затворен терм $\mu : \nat$.
  \marginpar{Аналогично можем да видим, че $\texttt{id}_{\val{\nat}} \triangleleft_{\nat} \lamb{x}{nat}{x}$.}
  \begin{itemize}
  \item
    Ако $e = \bot$, то от дефиницията на $\triangleleft_{\nat}$ ведната следва, че за произволен затворен терм $\mu : \nat$, то
    $\bot \triangleleft_{\nat} \mu$. Понеже $\texttt{id}_{\val{\nat}}(\bot) = \bot$, то отново от дефиницията веднага следва, че
    $\bot \triangleleft_{\nat} \tau(\mu)$, за произволен затворен терм $\mu : \nat$.
  \item
    Нека $e = n\in\Nat$ и да разгледаме затворен терм $\mu : \nat$, за който $n \triangleleft_{\nat} \mu$.
    Според дефиницията на $\triangleleft_{\nat}$, това означава, че $\mu \opsem{}{nat} \vv{n}$.
    Сега да видим защо $\texttt{id}_{\val{\nat}}(n) = n \triangleleft_{\nat} \tau(\mu)$ или с други думи,
    трябва да проверим, че $\tau(\mu) \opsem{}{nat} \vv{n}$. Тук се позоваваме на правилата от операционната семантика:
    \begin{prooftree}
      \AxiomC{$\tau$ е стойност}
      \LeftLabel{\scriptsize{(val)}}
      \UnaryInfC{$\tau \opsemGen{}{\nat\to\nat} \lamb{x}{nat}{\vv{x + 0}}$}
      \AxiomC{$n \triangleleft_{\nat} \mu$}
      \UnaryInfC{$\mu \opsem{}{nat} \vv{n}$}
      \AxiomC{$\vv{0}$ е стойност}
      \RightLabel{\scriptsize{(val)}}
      \UnaryInfC{$\vv{0} \opsem{}{nat} \vv{0}$}
      \RightLabel{\scriptsize{(plus)}}
      \BinaryInfC{$\vv{(x+0)}\subst{x}{\mu} \opsem{}{nat} \vv{n}$}
      \RightLabel{\scriptsize{(app)}}
      \BinaryInfC{$\tau(\mu) \opsem{}{nat} \vv{n}$}
    \end{prooftree}
  \end{itemize}

\end{example}

\begin{problem}
  Нека положим $\tau \equiv \lamb{x}{nat}{\lamb{y}{nat}{x-y}}$.
  Проверете, че $f \triangleleft_{\nat\to\nat\to\nat} \tau$, където:
  \begin{itemize}
  \item
    $f = \curry(\minus)$;
  \item
    $f(a)(b) =
    \begin{cases}
      a-b, & \text{ако }a \geq b\\
      \bot, & \text{иначе}
    \end{cases}$;
  \item
    $f(a)(b) = \bot$ за произволни $a,b\in\Nat_\bot$.
  \end{itemize}
\end{problem}


Нека първо да разгледаме някои основни свойства на релацията $\triangleleft_{\vv{a}}$.
Тук доказателствата протичат с индукция по построението на типовете.
\marginpar{\cite[стр. 197]{gunter}}

\begin{proposition}\label{pr:pcf:adequacy:bottom}
  За всеки тип $\vv{a}$ и всеки затворен терм $\tau : \vv{a}$ е изпълнено, че $\bot^{\val{\vv{a}}} \triangleleft_{\vv{a}} \tau$.
\end{proposition}
\begin{proof}
  Индукция по построението на типовете $\vv{a}$.
  Първо, нека $\vv{a} = \vv{nat}$. По тривиални съображения имаме, че за произволен терм $\tau:\vv{a}$ е изпълнено, че $\bot^{\val{\vv{nat}}} \triangleleft_{\vv{nat}} \tau$.
  
  Второ, нека $\vv{a} = \vv{b} \to \vv{c}$ и да фиксираме произволен терм $\tau : \vv{b} \to \vv{c}$.
  Тук имаме, че $\bot^{\val{\vv{a}}} \in \Cont{\val{\vv{b}}}{\val{\vv{c}}}$ е изображение,
  за което $\bot^{\val{\vv{a}}}(e) =  \bot^{\val{\vv{c}}}$ за всеки елемент $e \in \val{\vv{b}}$.
  Нека $e \triangleleft_{\vv{b}} \mu$, където $\mu : \vv{b}$.
  Щом $\tau : \vv{b}\to\vv{c}$, от правилата за типизиране е ясно, че $\tau(\mu) : \vv{c}$.
  Сега от \IndHyp за типа $\vv{c}$ е ясно, че $\bot^{\val{\vv{a}}}(e) = \bot^{\val{\vv{c}}} \triangleleft_{\vv{c}} \tau(\mu)$.
\end{proof}


\begin{proposition}\label{pr:pcf:adequacy:chain}
  Нека за произволен тип $\vv{a}$ и произволен терм $\tau:\vv{a}$ да разгледаме множеството $D \df \{d \in \val{\vv{a}} \mid d \triangleleft_{\vv{a}} \tau\}$.
  Тогава ако $\chain{d}{i}$ е верига от елементи на $D$, то $\bigsqcup_i d_i$ също принадлежи на $D$.
\end{proposition}
\begin{proof}
  Индукция по построението на типовете $\vv{a}$.
  Първо, нека $\vv{a} = \vv{nat}$.
  \marginpar{Да напомним, че $\val{\vv{nat}} = \Nat_\bot$. Ясно е, че всяка верига от елементи на $\Nat_\bot$ се стабилизира.}
  Нека $\chain{d}{i}$ е верига от елементи на $\Nat_\bot$ и за всеки индекс $i$, $d_i \triangleleft_{\vv{nat}} \tau$.
  Ако за всяко $i$, $d_i = \bot$, то $\bigsqcup_i d_i = \bot^{\val{\vv{nat}}}$ и следователно $\bigsqcup_i d_i
  \triangleleft_{\vv{nat}} \tau$.
  Ако съществува индекс $i_0$, за който $d_{i_0} = n \neq \bot$, то е ясно, че за всяко $i > i_0$, $d_i = n$.
  Оттук следва, че $\bigsqcup_i d_i = n = d_{i_0}$.
  Понеже $d_{i_0} \triangleleft_{\vv{nat}} \tau$, то директно следва, че $\bigsqcup_i d_i \triangleleft_{\vv{nat}} \tau$.

  Второ, нека $\vv{a} = \vv{b} \to \vv{c}$ и да фиксираме произволен терм $\tau : \vv{b} \to \vv{c}$.
  Нека $\chain{f}{i}$ е верига от елементи на $\Cont{\val{\vv{b}}}{\val{\vv{c}}}$,
  за които е изпълнено, че $f_i \triangleleft_{\vv{a}} \tau$. Трябва да докажем, че $\bigsqcup_i f_i \triangleleft_{\vv{a}} \tau$,
  т.е. за произволен елемент $e \in \val{\vv{b}}$ и произволен затворен терм $\mu : \vv{b}$, за който $e \triangleleft_{\vv{b}} \mu$, то
  $(\bigsqcup_if)(e) \triangleleft_{\vv{c}} \tau(\mu)$.
  Но ние знаем от \Lem{double-chain:lub}, че $(\bigsqcup_if)(e) = \bigsqcup_i\{f_i(e)\}$.
  Щом $f_i \triangleleft_{\vv{b}\to\vv{c}} \tau$, то за разглежданите $e$ и $\mu$ имаме, че $f_i(e) \triangleleft_{\vv{c}} \tau(\mu)$.
  Ние знаем, че ${(f_i(e))}^\infty_{i=0}$ е верига и от \IndHyp за типа $\vv{c}$ следва, че $\bigsqcup_i\{f_i(e)\} \triangleleft_{\vv{c}} \tau(\mu)$.
\end{proof}


\begin{proposition}\label{pr:pcf:adequacy:implication}
  Да разгледаме произволен тип $\vv{a}$ и произволен затворен терм $\tau : \vv{a}$.
  \marginpar{Да напомним, че с $\vv{v}$ означаваме термове стойности.}
  Тогава е изпълнено следното:
  \begin{prooftree}
    \AxiomC{$d \triangleleft_{\vv{a}} \tau$}
    \AxiomC{$(\forall \vv{v})[\tau \Downarrow_{\vv{a}} \vv{v} \implies \rho \Downarrow_{\vv{a}} \vv{v}]$}
    \BinaryInfC{$d \triangleleft_{\vv{a}} \rho$}
  \end{prooftree}
\end{proposition}
\begin{proof}
  Индукция по построението на типовете $\vv{a}$.
  Първо, нека $\vv{a} = \vv{nat}$. 
  Нека $d \triangleleft_{\vv{nat}} \tau$ и $(\forall \vv{v})[\tau \Downarrow_{\vv{nat}} \vv{v} \implies \rho \Downarrow_{\vv{nat}} \vv{v}]$.
  Понеже $\sqsubseteq$ е плоската наредба в $\Nat_\bot$, то имаме два случая.
  Ако $d = \bot^{\val{\vv{nat}}}$, то е ясно от \Prop{pcf:adequacy:bottom}, че $d \triangleleft_{\vv{nat}} \rho$.
  Нека сега $d \neq \bot^{\val{\vv{nat}}}$. 
  Понеже $d \triangleleft_{\vv{nat}} \tau$, то $\tau \Downarrow_{\vv{nat}} \vv{d}$.
  Оттук следва, че $\rho \Downarrow_{\vv{nat}} \vv{d}$. Заключаваме, че $d \triangleleft_{\vv{nat}} \rho$.

  Второ, нека $\vv{a} = \vv{b} \to \vv{c}$ и да фиксираме произволен терм $\tau : \vv{b} \to \vv{c}$.
  Нека
  \begin{align}
    & f \triangleleft_{\vv{b}\to\vv{c}} \tau \label{eq:pcf:adequacy:implication:f-tau}\\
    & (\forall \vv{v})[\tau \Downarrow_{\vv{b}\to\vv{c}} \vv{v} \implies \rho \Downarrow_{\vv{b}\to\vv{c}} \vv{v}] \label{eq:pcf:adequacy:implication:value}
  \end{align}
  % $f \triangleleft_{\vv{b}\to\vv{c}} \tau$ и $(\forall \vv{v})[\tau \Downarrow_{\vv{b}\to\vv{c}} \vv{v} \implies \rho \Downarrow_{\vv{b}\to\vv{c}} \vv{v}]$.
  Ще докажем, че $f \triangleleft_{\vv{b} \to \vv{c}} \rho$.
  За целта, да разгледаме произволен елемент $e \in \val{\vv{b}}$ и произволен затворен терм $\mu : \vv{b}$, за който $e \triangleleft_{\vv{b}} \mu$.
  Достатъчно е да докажем, че $f(e) \triangleleft_{\vv{c}} \rho(\mu)$.
  За момента от Свойство~(\ref{eq:pcf:adequacy:implication:f-tau}) знаем само, че $f(e) \triangleleft_{\vv{c}} \tau(\mu)$.
  Понеже имаме следното правило в операционната семантика:
  \marginpar{Имаме от (\ref{eq:pcf:adequacy:implication:value}), че 
    \[\tau \Downarrow_{\vv{b}\to\vv{c}} \vv{v} \implies \rho \Downarrow_{\vv{b}\to\vv{c}} \vv{v},\] а тук $\vv{v} \equiv \lamb{x}{b}{\tau'}$.}
  \begin{prooftree}
    \AxiomC{$\tau \Downarrow_{\vv{b}\to\vv{c}} \overbrace{\lamb{x}{b}{\tau'}}^{\vv{v}}$}
    \AxiomC{$\tau'\subst{x}{\mu} \Downarrow_{\vv{c}} \vv{v}'$}
    \RightLabel{\scriptsize{(app)}}
    \BinaryInfC{$\tau(\mu) \Downarrow_{\vv{c}} \vv{v}'$}
  \end{prooftree}
  то от Свойство~(\ref{eq:pcf:adequacy:implication:value}) получаваме, че
  \begin{prooftree}
    \AxiomC{$\rho \Downarrow_{\vv{b}\to\vv{c}} \overbrace{\lamb{x}{b}{\tau'}}^{\vv{v}}$}
    \AxiomC{$\tau'\subst{x}{\mu} \Downarrow_{\vv{c}} \vv{v}'$}
    \RightLabel{\scriptsize{(app)}}
    \BinaryInfC{$\rho(\mu) \Downarrow_{\vv{c}} \vv{v}'$}
  \end{prooftree}
  Оттук следва, че
  \begin{equation}
    \label{eq:pcf:adequacy:implication:final}
    (\forall \vv{v}')[\tau(\mu) \Downarrow_{\vv{c}} \vv{v}' \implies \rho(\mu) \Downarrow_{\vv{c}} \vv{v}'].
  \end{equation}
  Сега от \IndHyp за типа $\vv{c}$ директно следва, че щом $f(e) \triangleleft_{\vv{c}}\tau(\mu)$ и Свойство (\ref{eq:pcf:adequacy:implication:final}), то \[f(e) \triangleleft_{\vv{c}} \rho(\mu).\]
\end{proof}




% \begin{lemma}\label{lem:pcf:relation}
%   Нека $\tau : \vv{a}$. Тогава:
%   \begin{enumerate}[1)]
%   \item
%     $\bot^{\val{\vv{a}}} \triangleleft_{\vv{a}} \tau$;
%   \item
%     $D = \{d \in \val{\vv{a}} \mid d \triangleleft_{\vv{a}} \tau\}$ е непрекъснато свойство в областта на Скот $\val{\vv{a}}$, т.е.
%     за всяка верига $\chain{d}{i}$ от елементи на $D$ е изпълнено, че $\bigsqcup_i d_i \in D$.
%   \item
%     \marginpar{Не ми трябва $u \sqsubseteq d$.}
%     Ако $d \triangleleft_{\vv{a}} \tau$ и $(\forall \vv{v})[\tau \Downarrow_{\vv{a}} \vv{v} \implies \rho
%     \Downarrow_{\vv{a}} \vv{v}]$, то $d \triangleleft_{\vv{a}} \rho$.
%   \end{enumerate}
% \end{lemma}
% \begin{proof}
%   Индукция по построението на типовете $\vv{a}$.
%   Първо, нека $\vv{a} = \vv{nat}$ и да фиксираме произволен терм $\tau : \vv{nat}$.
%   \marginpar{Да напомним, че $\val{\vv{nat}} \df \Nat_\bot$.}
%   \begin{enumerate}[1)]
%   \item
%     По тривиални съображения имаме, че
%     \[\bot^{\val{\vv{nat}}} \triangleleft_{\vv{nat}} \tau.\]
%   \item
%     Нека $\chain{d}{i}$ е верига от елементи на $\Nat_\bot$ и за всяко $i$, $d_i \triangleleft_{\vv{nat}} \tau$.
%     Ако за всяко $i$, $d_i = \bot$, то $\bigsqcup_i d_i = \bot^{\val{\vv{nat}}}$ и следователно $\bigsqcup_i d_i
%     \triangleleft_{\vv{nat}} \tau$.
%     Ако съществува $i_0$, за което $d_{i_0} = n \neq \bot$, то е ясно, че за всяко $i > i_0$, $d_i = n$.
%     Оттук следва, че $\bigsqcup_i d_i = n = d_{i_0}$.
%     Понеже $d_{i_0} \triangleleft_{\vv{nat}} \tau$, то директно следва, че $\bigsqcup_i d_i \triangleleft_{\vv{nat}} \tau$.
%   \item
%     Нека $d \triangleleft_{\vv{nat}} \tau$ и $(\forall \vv{v})[\tau \Downarrow_{\vv{nat}} \vv{v} \implies \rho
%     \Downarrow_{\vv{nat}} \vv{v}]$. Понеже $\sqsubseteq$ е плоската наредба в $\Nat_\bot$, то имаме два случая.
%     Ако $d = \bot^{\val{\vv{nat}}}$, то е ясно от 1), че $d \triangleleft_{\vv{nat}} \rho$.
%     Нека сега $d \neq \bot^{\val{\vv{nat}}}$. 
%     Понеже $d \triangleleft_{\vv{nat}} \tau$, то $\tau \Downarrow_{\vv{nat}} \vv{d}$.
%     Оттук следва, че $\rho \Downarrow_{\vv{nat}} \vv{d}$. Заключаваме, че $d \triangleleft_{\vv{nat}} \rho$.    
%   \end{enumerate}
  
%   Второ, нека $\vv{a} = \vv{b} \to \vv{c}$ и да фиксираме произволен терм $\tau : \vv{b} \to \vv{c}$.
%   \marginpar{Да напомним, че \[\val{\vv{b} \to \vv{c}} \df \Cont{\val{\vv{b}}}{\val{\vv{c}}}.\]}
%   \begin{enumerate}[1)]
%   \item
%     Тук имаме, че $\bot^{\val{\vv{a}}} \in \Cont{\val{\vv{b}}}{\val{\vv{c}}}$ е изображение,
%     за което $\bot^{\val{\vv{a}}}(e) =  \bot^{\val{\vv{c}}}$ за всеки елемент $e \in \val{\vv{b}}$.
%     Нека $e \triangleleft_{\vv{b}} \mu$, където $\mu : \vv{b}$.
%     От правилата за типизиране е ясно, че $\tau(\mu) : \vv{c}$.
%     Сега от И.П. е ясно, че $\bot^{\val{\vv{a}}}(e) = \bot^{\val{\vv{c}}} \triangleleft_{\vv{c}} \tau(\mu)$.
%   \item
%     Нека $\chain{f}{i}$ е верига от елементи на $\Cont{\val{\vv{b}}}{\val{\vv{c}}}$,
%     за които е изпълнено, че $f_i \triangleleft_{\vv{a}} \tau$. Трябва да докажем, че $\bigsqcup_i f_i \triangleleft_{\vv{a}} \tau$,
%     т.е. за произволни $e \in \val{\vv{b}}$ и произволни $\mu : \vv{b}$, за които $e \triangleleft_{\vv{b}} \mu$, то
%     $(\bigsqcup_if)(e) \triangleleft_{\vv{c}} \tau(\mu)$.
%     Но ние знаем, че $(\bigsqcup_if)(e) = \bigsqcup_i\{f_i(e)\}$.
%     Щом $f_i \triangleleft_{\vv{b}\to\vv{c}} \tau$, то за разглежданите $e$ и $\mu$ имаме, че $f_i(e) \triangleleft_{\vv{c}} \tau(\mu)$.
%     Ние знаем, че ${(f_i(e))}^\infty_{i=0}$ е верига и от И.П. следва, че $\bigsqcup_i\{f_i(e)\} \triangleleft_{\vv{c}} \tau(\mu)$.
%   \item
%     Нека $f \triangleleft_{\vv{b}\to\vv{c}} \tau$ и $\tau \Downarrow_{\vv{b}\to\vv{c}} \vv{v} \implies \rho
%     \Downarrow_{\vv{b}\to\vv{c}} \vv{v}$. Ще докажем, че $f \triangleleft_{\vv{b} \to \vv{c}} \rho$.
%     За целта, нека $e \in \val{\vv{b}}$, $\mu : \vv{b}$ и $e \triangleleft_{\vv{b}} \mu$.
%     Ще докажем, че $f(e) \triangleleft_{\vv{c}} \rho(\mu)$.
%     За момента знаем само, че $f(e) \triangleleft_{\vv{c}} \tau(\mu)$.
%     Понеже имаме следното правило в операционната семантика:
%     \marginpar{Имаме по условие, че 
%       \[\tau \Downarrow_{\vv{b}\to\vv{c}} \vv{v} \implies \rho \Downarrow_{\vv{b}\to\vv{c}} \vv{v},\] а тук $\vv{v} \equiv \lamb{x}{b}{\tau'}$.}
%     \begin{prooftree}
%       \AxiomC{$\tau \Downarrow_{\vv{b}\to\vv{c}} \lamb{x}{b}{\tau'}$}
%       \AxiomC{$\tau'\subst{x}{\mu} \Downarrow_{\vv{c}} \vv{v}'$}
%       \RightLabel{\scriptsize{(app)}}
%       \BinaryInfC{$\tau(\mu) \Downarrow_{\vv{c}} \vv{v}'$}
%     \end{prooftree}
%     то получаваме, че
%     \begin{prooftree}
%       \AxiomC{$\rho \Downarrow_{\vv{b}\to\vv{c}} \lamb{x}{b}{\tau'}$}
%       \AxiomC{$\tau'\subst{x}{\mu} \Downarrow_{\vv{c}} \vv{v}'$}
%       \RightLabel{\scriptsize{(app)}}
%       \BinaryInfC{$\rho(\mu) \Downarrow_{\vv{c}} \vv{v}'$}
%     \end{prooftree}
%     Оттук следва, че
%     \[(\forall \vv{v}')[\tau(\mu) \Downarrow_{\vv{c}} \vv{v}' \implies \rho(\mu) \Downarrow_{\vv{c}} \vv{v}'].\]
%     Сега от И.П. директно следва, че $f(e) \triangleleft_{\vv{c}} \rho(\mu)$.
%   \end{enumerate}
% \end{proof}

За да докажем, че $\val{\tau} \triangleleft_{\vv{nat}} \tau$, то трябва да докажем, че за всеки тип $\vv{a}$ и всеки затворен терм $\tau$ от тип $\vv{a}$, че е изпълнено
$\val{\tau} \triangleleft_{\vv{a}} \tau$ с индукция по построението на термовете.
Тук обаче имаме проблем. Ако $\tau \equiv \lamb{y}{b}{\tau_1}$, то трябва да използваме индукционно предположение за $\tau_1$,
в който вече има свободна променлива $\vv{y}$. Поради тази причина, ние трябва да разгледаме едно по-общо твърдение, при което позволяваме в термовете да се срещат свободни променливи.

\begin{framed}
  \begin{lemma}[Фундаментално свойство на $\triangleleft_{\vv{a}}$]\label{lem:pcf:fundamental}
    Нека $\Gamma = \vv{x}_1:\vv{a}_1,\dots,\vv{x}_n:\vv{a}_n$. Тогава
    \begin{prooftree}
      \AxiomC{$\Gamma \vdash \tau : \vv{a}$}
      \AxiomC{$(u_1,\dots,u_n) \triangleleft_\Gamma (\mu_1,\dots,\mu_n)$}
      \BinaryInfC{$\val{\tau}_\Gamma(\ov{u}) \triangleleft_{\vv{a}} \tau[\ov{\vv{x}}/\ov{\mu}]$}
    \end{prooftree}
  \end{lemma}  
\end{framed}
\begin{proof}
  Индукция по построението на термовете.
  \begin{itemize}
  \item
    Нека $\tau \equiv \vv{n}$. Тук директно от дефиницията на релацията $\triangleleft_{\vv{nat}}$ имаме, че
    \[n \triangleleft_{\vv{nat}} \vv{n}.\]
  \item
    \marginpar{Понеже $\Gamma \vdash \tau : \vv{a}$, то няма как $\tau$ да е променлива, която да не е някоя измежду $\vv{x}_1,\dots,\vv{x}_n$.}
    Нека $\tau \equiv \vv{x}_i$. Този случай също е много лесен.
    Имаме, че $\tau[\ov{x}/\ov{\mu}] \equiv \mu_i$ и $\val{\tau}_\Gamma(\ov{u}) = u_i$.
    Тогава, понеже $(u_1,\dots,u_n) \triangleleft_\Gamma (\mu_1,\dots,\mu_n)$,
    то директно получаваме, че
    \[\val{\tau}_\Gamma(\ov{u}) \triangleleft_{\vv{a}} \tau[\ov{x}/\ov{\mu}].\]
  \item
    Нека $\tau \equiv \tau_1 + \tau_2$. Тогава от правилата за типизиране имаме, че $\vv{a} = \vv{nat}$. Имаме също, че
    \begin{align*}
      & \tau[\ov{x}/\ov{\mu}] \equiv \tau_1[\ov{x}/\ov{\mu}] + \tau_2[\ov{x}/\ov{\mu}];\\
      & \val{\tau}_\Gamma(\ov{u}) = \plus(\underbrace{\val{\tau_1}_\Gamma(\ov{u})}_{n_1},\underbrace{\val{\tau_2}_\Gamma(\ov{u})}_{n_2}) = n.
    \end{align*}
    Можем да приемем, че $n \neq \bot$, защото иначе този случай е тривиален заради дефиницията на $\triangleleft_{\vv{nat}}$.
    От \IndHyp имаме, че $\val{\tau_1}_\Gamma(\ov{u}) \triangleleft_{\vv{nat}} \tau_1[\ov{x}/\ov{\mu}]$
    и $\val{\tau_2}_\Gamma(\ov{u}) \triangleleft_{\vv{nat}} \tau_2[\ov{x}/\ov{\mu}]$.
    Това означава, че $\tau_1[\ov{x}/\ov{\mu}] \Downarrow_{\vv{nat}} \vv{n}_1$ и $\tau_2[\ov{x}/\ov{\mu}] \Downarrow_{\vv{nat}}
    \vv{n}_2$.
    От правилата на операционната семантика е ясно, че $\tau[\ov{x}/\ov{\mu}] \Downarrow_{\vv{nat}} \vv{n}$ и следователно
    \[\val{\tau}_\Gamma(\ov{u}) \triangleleft_{\nat} \tau[\ov{\vv{x}}/\ov{\mu}].\]
  \item
    Нека $\tau \equiv \tau_1\ \vv{-}\ \tau_2$. 
  \item
    Нека $\tau \equiv \tau_1\ \vv{==}\ \tau_2$. 
  \item
    Нека $\tau \equiv \ifelse{\tau_1}{\tau_2}{\tau_3}$.
    От правилата за типизиране имаме, че $\Gamma \vdash \tau_1 : \nat$ и
    $\Gamma \vdash \tau_2 : \vv{a}$ и $\Gamma \vdash \tau_3 : \vv{a}$.
    Знаем също, че
    \begin{align*}
      & \tau[\ov{x}/\ov{\mu}] \equiv \ifelse{\tau_1[\ov{x}/\ov{\mu}]}{\tau_2[\ov{x}/\ov{\mu}]}{\tau_3[\ov{x}/\ov{\mu}]}\\
      & \val{\tau}_\Gamma(\ov{u}) = \texttt{if}(\val{\tau_1}_\Gamma(\ov{u}),\val{\tau_2}_\Gamma(\ov{u}),\val{\tau_3}_\Gamma(\ov{u})).
    \end{align*}

    Случаят, когато $\val{\tau_1}_\Gamma(\ov{u}) = \bot^{\val{\nat}}$ е очевиден и ще го пропуснем.
    \marginpar{Случаят, когато $\val{\tau_1}_\Gamma(\ov{u}) > 0$ е сходен и ще го оставим за трудолюбивия читател.}
    Нека да разгледаме случая, когато $\val{\tau_1}_\Gamma(\ov{u}) = 0$.
    Тогава $\val{\tau}_\Gamma(\ov{u}) = \val{\tau_3}_\Gamma(\ov{u})$.
    От \IndHyp за $\tau_1$ получаваме, че
    \[\val{\tau_1}_\Gamma(\ov{u}) \triangleleft_{\nat} \tau_1[\ov{x}/\ov{\mu}].\]
    От дефиницията на $\triangleleft_{\nat}$ следва, че щом $\val{\tau_1}_\Gamma(\ov{u}) = 0$, то $\tau_1[\ov{x}/\ov{\mu}] \Downarrow_{\nat} \vv{0}$.
    От \IndHyp за $\tau_3$ получаваме, че
    \[\underbrace{\val{\tau_3}_\Gamma(\ov{u})}_{\val{\tau}_\Gamma(\ov{u})} \triangleleft_{\vv{a}} \tau_3[\ov{x}/\ov{\mu}].\]
    Сега от правилата на операционната семантика имаме, че
    \begin{prooftree}
      \AxiomC{$\tau_1[\ov{x}/\ov{\mu}] \opsem{}{nat} \vv{0}$}
      \AxiomC{$\tau_3[\ov{x}/\ov{\mu}] \opsem{}{a} \vv{v}$}
      \RightLabel{\scriptsize{(if$_0$})}
      \BinaryInfC{$\underbrace{\ifelse{\tau_1[\ov{x}/\ov{\mu}]}{\tau_2[\ov{x}/\ov{\mu}]}{\tau_3[\ov{x}/\ov{\mu}]}}_{\tau[\ov{x}/\ov{\mu}]} \opsem{}{a} \vv{v}$}
    \end{prooftree}
    Да напишем на чисто важните неща, които имаме до момента.
    \begin{itemize}
    \item
      $\val{\tau}_\Gamma(\ov{u}) \triangleleft_{\vv{a}} \tau_3[\ov{x}/\ov{\mu}]$;
    \item
      $(\forall \vv{v})[\tau_3[\ov{x}/\ov{\mu}] \opsem{}{a} \vv{v} \implies \tau[\ov{x}/\ov{\mu}] \opsem{}{a} \vv{v}]$.
    \end{itemize}
    Сега директно прилагаме \Prop{pcf:adequacy:implication} и получаваме, че $\val{\tau}_\Gamma(\ov{u}) \triangleleft_{\vv{a}} \tau[\ov{x}/\ov{\mu}]$.
  \item
    Нека $\tau \equiv \tau_1\tau_2$. От правилата за типизиране имаме, че
    \begin{prooftree}
      \AxiomC{$\Gamma \vdash \tau_1 : \vv{b} \to \vv{a}$}
      \AxiomC{$\Gamma \vdash \tau_2 : \vv{b}$}
      \BinaryInfC{$\Gamma \vdash \tau_1\tau_2 : \vv{a}$}
    \end{prooftree}
    Да напомним, че
    \[\val{\tau_1\tau_2}_\Gamma(\ov{u}) \df \texttt{eval}(\val{\tau_1}_\Gamma(\ov{u}), \val{\tau_2}_\Gamma(\ov{u})).\]
    Понеже $\tau$ е съставен от $\tau_1$ и $\tau_2$, от \IndHyp имаме следното:
    \begin{align*}
      & \val{\tau_1}_\Gamma(\ov{u}) \triangleleft_{\vv{b}\to\vv{a}} \tau_1[\ov{x}/\ov{\mu}];\\
      & \val{\tau_2}_\Gamma(\ov{u}) \triangleleft_{\vv{b}} \tau_2[\ov{x}/\ov{\mu}].
    \end{align*}
    Щом $\val{\tau_1}_\Gamma(\ov{u}) \triangleleft_{\vv{b}\to\vv{a}} \tau_1[\ov{x}/\ov{\mu}]$, то
    от дефиницията на релацията $\triangleleft_{\vv{b}\to\vv{a}}$ следва, че за произволни $e \triangleleft_{\vv{b}} \rho$ имаме, че
    $\texttt{eval}(\val{\tau_1}_\Gamma(\ov{u}),e) \triangleleft_{\vv{a}} \tau_1[\ov{x}/\ov{\mu}](\rho)$. Нека сега да вземем $e \df \val{\tau_2}_\Gamma(\ov{u})$ и $\rho \df  \tau_2[\ov{x}/\ov{\mu}]$.
    Така получаваме  
    \[\underbrace{\texttt{eval}(\val{\tau_1}_\Gamma(\ov{u}), \val{\tau_2}_\Gamma(\ov{u}))}_{\val{\tau}_\Gamma(\ov{u})} \triangleleft_{\vv{a}} \underbrace{\tau_1[\ov{x}/\ov{\mu}](\tau_2[\ov{x}/\ov{\mu}])}_{\tau[\ov{x}/\ov{\mu}]}.\]
  \item
    Нека $\tau = \lamb{y}{b}{\tau'}$. Тогава от правилата за типизиране следва, че $\vv{a} = \vv{b} \to \vv{c}$, за някои типове $\vv{b}$ и $\vv{c}$, и
    $\Gamma,\type{y}{b} \vdash \tau' : \vv{c}$.
    Да напомним, че
    \[\val{\tau}_\Gamma(\ov{u}) \df \texttt{curry}(\val{\tau'}_{\Gamma,\type{y}{b}})(\ov{u}) \in \Cont{\val{\vv{b}}}{\val{\vv{c}}}.\]
    Да положим $f \df \val{\tau}_\Gamma(\ov{u})$. Тогава лесно се съобразява, че:
    \begin{align*}
      f(e) & = \val{\tau}_\Gamma(\ov{u})(e) \\
           & = \texttt{curry}(\val{\tau'}_{\Gamma,\type{y}{b}})(\ov{u})(e)\\
           & = \val{\tau'}_{\Gamma,\type{y}{b}}(\ov{u},e).
    \end{align*}
    Трябва да докажем, че $f \triangleleft_{\vv{b} \to \vv{c}} \tau[\ov{x}/\ov{\mu}]$.
    Това означава, че за произволни $e \in \val{\vv{b}}$ и $\rho : \vv{b}$, за които $e \triangleleft_{\vv{b}} \rho$,
    трябва да докажем, че $f(e) \triangleleft_{\vv{c}} \tau[\ov{x}/\ov{\mu}](\rho)$.

    Имаме, че
    \begin{prooftree}
      \AxiomC{$\Gamma,\type{y}{b} \vdash \tau' : \vv{c}$}
      \AxiomC{$(u_1,\dots,u_n,e) \triangleleft_{\Gamma,\type{y}{b}} (\mu_1,\dots,\mu_n,\rho)$}
      \RightLabel{\scriptsize{\IndHyp}}
      \BinaryInfC{$\underbrace{\val{\tau'}_{\Gamma,\type{y}{b}}(\ov{u},e)}_{f(e)} \triangleleft_{\vv{c}} \tau'[\ov{x}/\ov{\mu}][y/\rho]$}
    \end{prooftree}
    От правилата на операционната семантика имаме следното:
    \begin{prooftree}
      \AxiomC{$\lamb{y}{b}{\tau'[\ov{x}/\ov{\mu}]}$ е стойност}
      \UnaryInfC{$\lamb{y}{b}{\tau'[\ov{x}/\ov{\mu}]} \Downarrow_{\vv{b}\to\vv{c}} \lamb{y}{b}{\tau'[\ov{x}/\ov{\mu}]} $}
      \AxiomC{$\tau'[\ov{x}/\ov{\mu}][y/\rho] \Downarrow_{\vv{c}} \vv{v}$}
      \RightLabel{\scriptsize{(cbn)}}
      \BinaryInfC{$(\underbrace{\lamb{y}{b}{\tau'[\ov{x}/\ov{\mu}]}}_{\tau[\ov{x}/\ov{\mu}]})(\rho) \Downarrow_{\vv{c}} \vv{v}$}
    \end{prooftree}
    Да обобщим какво имаме до момента:
    \begin{itemize}
    \item
      $f(e) \triangleleft_{\vv{c}} \tau'[\ov{x}/\ov{\mu}][y/\rho]$;
    \item
      $(\forall \vv{v})[\ \tau'[\ov{x}/\ov{\mu}][y/\rho] \opsem{}{c} \vv{v} \implies \tau[\ov{x}/\ov{\mu}] \opsem{}{c} \vv{v}\ ]$.
    \end{itemize}
    От \Prop{pcf:adequacy:implication} веднага заключаваме, че $f(e) \triangleleft_{\vv{c}} \tau[\ov{x}/\ov{\mu}](\rho)$, което трябваше да докажем.
  \item
    Нека $\tau \equiv \fix(\tau')$.
    Тук трябва да докажем, че $\val{\fix(\tau')}(\ov{u}) \triangleleft_{\vv{a}} \tau[\ov{x}/\ov{\mu}]$.
    От правилата за типизиране имаме, че $\Gamma \vdash \tau' : \vv{a} \to \vv{a}$.
    От \IndHyp, приложено за $\tau'$, имаме, че
    \[\val{\tau'}_\Gamma(\ov{u}) \triangleleft_{\vv{a}\to\vv{a}} \tau'[\ov{x}/\ov{\mu}].\]
    Нека за улеснение да положим $f \df \val{\tau'}_\Gamma(\ov{u}) \in \Cont{\val{\vv{a}}}{\val{\vv{a}}}$.
    Да напомним, че
    \[\val{\fix(\tau')}_\Gamma(\ov{u}) = \lfp(f).\]
    От \Prop{pcf:adequacy:chain} знаем, че за всяка верига $\chain{d}{i}$ от елементи на
    \[D \df \{d \in \val{\vv{a}} \mid d \triangleleft_{\vv{a}} \fix(\tau'[\ov{x}/\ov{\mu}])\}\]
    е изпълнено, че $\bigsqcup_i d_i \in D$. Целта ни е да докажем, че
    $\lfp(f) \in D$. Да напомним, че $\lfp(f) = \bigsqcup_n f^n(\bot^{\val{\vv{a}}})$.
    С индукция по $n$ ще докажем, че за всяко $n$, $f^n(\bot^{\val{\vv{a}}}) \in D$.

    За $n = 0$ е ясно, понеже от \Prop{pcf:adequacy:bottom} имаме, че $f^0(\bot^{\val{\vv{a}}}) = \bot^{\val{\vv{a}}} \in D$.

    Нека сега $n > 0$. Ще използваме, че от \IndHyp имаме, че $f^{n-1}(\bot^{\val{\vv{a}}}) \in D$.
    Понеже $f \triangleleft_{\vv{a}\to\vv{a}} \tau'[\ov{x}/\ov{\mu}]$, то
    за произволно $e \triangleleft_{\vv{a}} \rho$ е изпълнено, че
    $f(e) \triangleleft_{\vv{a}} \tau'[\ov{x}/\ov{\mu}](\rho)$.
    Нека изберем $\rho \df \fix(\tau'[\ov{x}/\ov{\mu}])$ и $e \df f^{n-1}(\bot^{\val{\vv{a}}})$.
    Щом $f^{n-1}(\bot^{\val{\vv{a}}}) \in D$, то $f^{n-1}(\bot^{\val{\vv{a}}}) \triangleleft_{\vv{a}} \rho$ и следователно
    \[f(f^{n-1}(\bot^{\val{\vv{a}}})) \triangleleft_{\vv{a}} \tau'[\ov{x}/\ov{\mu}](\underbrace{\fix(\tau'[\ov{x}/\ov{\mu}])}_{\rho}).\]
    От правилата на операционната семантика имаме, че:
    \begin{prooftree}
      \AxiomC{$\tau'[\ov{x}/\ov{\mu}](\fix(\tau'[\ov{x}/\ov{\mu}])) \Downarrow_{\vv{a}} \vv{v}$}
      \RightLabel{\scriptsize{(fix)}}
      \UnaryInfC{$\fix(\tau'[\ov{x}/\ov{\mu}]) \Downarrow_{\vv{a}} \vv{v}$}
    \end{prooftree}
    Тогава от \Prop{pcf:adequacy:implication} следва, че
    \[f(f^{n-1}(\bot^{\val{\vv{a}}})) \triangleleft_{\vv{a}} \fix(\tau'[\ov{x}/\ov{\mu}]).\]
    Така доказахме, че $f^n(\bot^{\val{\vv{a}}}) \in D$.
    Заключаваме, че $\lfp(f) \in D$, т.е.
    \[\val{\fix(\tau')}_\Gamma(\ov{u}) \triangleleft_{\vv{a}} \fix(\tau'[\ov{x}/\ov{\mu}])\]
  \end{itemize}
\end{proof}

\begin{corollary}\label{cr:pcf:fundamental}
  За всеки тип $\vv{a}$ и за всеки затворен терм $\tau$ е изпълнено свойството:
  \[\tau \in \text{PCF}_{\vv{a}} \implies \val{\tau} \triangleleft_{\vv{a}} \tau.\]
\end{corollary}

Така на практика доказахме теоремата за адекватност.

\begin{framed}
  \begin{theorem}[Теорема за адекватност]\label{th:pcf:adequacy}
    За всеки затворен терм $\tau : \vv{nat}$, 
    \[\val{\tau} = n \neq \bot^{\val{\nat}} \implies \tau \Downarrow_{\vv{nat}} \vv{n}.\]
  \end{theorem}
\end{framed}
\begin{proof}
  Да разгледаме произволен затворен терм $\tau : \vv{nat}$.
  Нека $\val{\tau} = n \neq \bot$.
  От \Cor{pcf:fundamental} имаме, че $\val{\tau} \triangleleft_{\vv{nat}} \tau$.
  Тогава от дефиницията на $\triangleleft_{\vv{nat}}$ получаваме, че $\tau \Downarrow_{\vv{nat}} \vv{n}$.
\end{proof}

\begin{framed}
  \begin{corollary}
    За всеки затворен терм $\tau : \vv{nat}$, 
    \[\val{\tau} = n \neq \bot^{\val{\nat}} \text{ точно тогава, когато } \tau \Downarrow_{\vv{nat}} \vv{n}.\]
  \end{corollary}
\end{framed}
\begin{proof}
  Посоката $(\Rightarrow)$ е \hyperref[th:pcf:adequacy]{теоремата за адекватност}.
  Посоката $(\Leftarrow)$ е \hyperref[th:pcf:soundness]{теоремата за коректност}.
\end{proof}


Да разгледаме термовете
\begin{align*}
  & \tau \equiv \lamb{x}{nat}{\vv{x + 0}}\\
  & \rho \equiv \lamb{x}{nat}{\vv{x}}.
\end{align*}


Ясно е, че $\val{\tau} = \val{\rho}$, но според правилата на операционната семантика, понеже $\tau$ е стойност, то
$\tau \not\Downarrow_{\vv{nat}\to\vv{nat}} \rho$.
Оттук веднага е ясно, че няма как да имаме теорема за адекватност за по-високи типове от $\vv{nat}$.


% \begin{corollary}
%   За всеки $\tau \in \text{PCF}_{\nat}$ е изпълнена импликацията:
%   \[\tau \not\Downarrow_{\nat}\ \implies\ \val{\tau} = \bot^{\val{\nat}}.\]  
% \end{corollary}
% \begin{proof}

% \end{proof}

Следващото следствие ни казва, че за типа $\nat$, $\bot$ означава изчисление, което никога не завършва.

\begin{framed}
  \begin{corollary}
    Нека $\tau$ е затворен терм от тип $\nat$. Тогава 
    \[\val{\tau} = \bot^{\val{\nat}} \text{ точно тогава, когато } \tau \not\opsem{}{nat}.\]
  \end{corollary}
\end{framed}
% \begin{proof}
%   Първо, да допуснем, че $\val{\tau} = \bot^{\val{\nat}}$, но $\tau \opsem{}{nat} \vv{n}$, за някоя константа $\vv{n}$. Тогава от \hyperref[th:pcf:soundness]{теоремата за коректност} получаваме, че $\val{\tau} = n \neq \bot^{\val{\nat}}$, което е противоречие.
  
%   Второ, да допуснем, че $\tau \not\Downarrow_{\nat}$, но $\val{\tau} = n \neq \bot$.
%   Но тогава от \hyperref[th:pcf:adequacy]{теоремата за адекватност} следва, че $\tau \Downarrow_{\nat} \vv{n}$, което е противоречие.
% \end{proof}

%%% Local Variables:
%%% mode: latex
%%% TeX-master: "../sep"
%%% End:

\newpage
\section{Контекстна еквивалентност}\label{pcf:sect:context}

\index{контекст}
\marginpar{\cite[Глава 6.1]{gunter}}
\marginpar{\cite[Глава 48]{practical-foundations}}
\marginpar{Интуитивно, контекстите са програмни фрагменти. Не са пълни програми, защото имат празни места означени с $-$.}
\[\C ::= -\ |\ \vv{n}\ |\ \vv{x}\ |\ \C - \C\ |\ \C + \C\ |\ \C\ \vv{==}\ \C\ |\ \ifelse{\C}{\C}{\C}\ |\ \C\C\ |\ \lamb{x}{a}{\C}\ |\ \fix(\C).\]
Контекстите са на практика изрази, като позволяваме да имат специална свободна променлива, която означаваме с $-$.
% За произволен израз $\tau$, с $\C\{\tau\}$ означаваме израза $\C\rename{-}{\tau}$, където считаме $-$ като свободна променлива на $\C$.
% Това означава, че заместваме всички срещания на $-$ с $\tau$ като правим заместването директно, т.е.
% не се интересуваме дали някоя свободна променлива на $\tau$ няма да попадне под обхвата на свързана променлива в $\C$.
% Например, ако $\C = \lamb{x}{a}{-}$, то $\C\rename{-}{\vv{x}} =
% \lamb{x}{a}{\vv{x}}$.
За краткост, вместо $\C[-/\tau]$ ще пишем $\C[\tau]$.

% \begin{proposition}
%   % \marginpar{Да напомним, че с $\equiv$ означаваме релацията $\alpha$-еквивалентност.}
%   Ако $\tau \equiv_\alpha \tau'$, то $\C[\tau] \equiv_\alpha \C[\tau']$.
% \end{proposition}
% \begin{hint}
%   Индукция по построението на контекстите.
% \end{hint}

Ще казваме, че {\bf завършена програма} е затворен терм от тип $\nat$.


\marginpar{Някои наричат тази релация observational equivalence. Тук наричаме релацията contextual equivalence.}
\begin{framed}
  \begin{definition}\label{df:context:equivalence}
    За затворени термове $\tau_1$ и $\tau_2$, дефинираме
    $\tau_1 \leq_{ctx} \tau_2 : \vv{a}$, ако
    \begin{enumerate}[1)]
    \item
      $\emptyset \vdash \tau_1 : \vv{a}$ и $\emptyset \vdash \tau_2 : \vv{a}$
    \item
      За \emph{всички} контексти $\C[-]$, за които $\emptyset \vdash \C[\tau_1] : \nat$ и $\emptyset \vdash \C[\tau_2] : \nat$, то
      \[(\forall \vv{n})[\ \C[\tau_1] \Downarrow_{\nat} \vv{n} \implies \C[\tau_2] \Downarrow_{\nat} \vv{n}\ ].\]
    \end{enumerate}
    Ще пишем $\tau_1 \cong_{ctx} \tau_2 : \vv{a}$, ако
    $\tau_1 \leq_{ctx} \tau_2 : \vv{a}$ и $\tau_2 \leq_{ctx} \tau_1 : \vv{a}$.
  \end{definition}
\end{framed}

Възможно е по-общо да се дефинира релацията $\leq_{ctx}$ не само за затворени, но и за произволни добре типизирани термове. За нашите цели е достатъчно да се ограничим само до затворени термове.

% \begin{itemize}
% \item 
  % Ще пишем $\tau_1 \cong_{ctx} \tau_2 : \vv{a}$, ако
  % $\tau_1 \leq_{ctx} \tau_2 : \vv{a}$ и $\Gamma \vdash \tau_2 \leq_{ctx} \tau_1 : \vv{a}$.
% \item
%   Ако $\Gamma = \emptyset$, то ще пишем $\tau_1 \cong_{ctx} \tau_2 : \vv{a}$ вместо $\emptyset \vdash \tau_1 \cong_{ctx} \tau_2 : \vv{a}$.  
% \end{itemize}

\marginpar{Two phrases of a programming language are contextually equivalent if any occurrences of the first phrase in a complete
  program can be replaced by the second phrase without affecting the observable results of executing the program.
  This kind of program equivalence is also known as operational, or observational equivalence \cite[стр. 44]{cambridge-den-sem}.}

\begin{framed}
  Денотационната семантика $\val{.}$ се нарича {\bf напълно абстрактна}, ако денотационната и операционната наредба съвпадат, т.е. за всеки два затворени терма $\tau_1$ и $\tau_2$ от тип $\vv{a}$ е изпълнено, че:
  \[\val{\tau_1} \sqsubseteq \val{\tau_2}\text{ точно тогава, когато }\tau_1 \leq_{ctx} \tau_2 : \vv{a}.\]
\end{framed}

Един от основните ранни резултати в изучаването на семантиката на езици за програмиране е, че нашата денотационна семантика не е напълно абстрактна.
Практически, ако имаме два терма, които са контекстно еквивалентни, то можем да заменим единия с другия в произволна програма,
без да има видими разлики на ниво изпълнение на програмата.
% Това представлява опит да се формализира математически практиката за тестване на програми.
Проблемът е, че с този формализъм е трудно да се работи, защото в дефиницията имаме квантор
за всеобщност относно всички контексти (програмни фрагменти).

\begin{proposition}\label{pr:pcf:context:simple}
  За произволни добре типизирани затворени термове $\tau_1$ и $\tau_2$ е изпълнено, че:
  \begin{enumerate}[(1)]
  \item
    \label{pr:pcf:context:simple:base}
    $\tau_1 \leq_{ctx} \tau_2 : \nat \implies (\forall \vv{n})[\tau_1\Downarrow_{\nat}\vv{n} \implies \tau_2 \Downarrow_{\nat} \vv{n}]$.
  \item
    \label{pr:pcf:context:simple:step}
    $\tau_1 \leq_{ctx} \tau_2 : \arr{b}{c} \implies (\forall \rho\in\text{PCF}_{\vv{b}})[\tau_1\rho \leq_{ctx} \tau_2 \rho : \vv{c}]$.
  \end{enumerate}
\end{proposition}
\begin{hint}
  \begin{enumerate}[(1)]
  \item
    Просто вземете контекст $\C \df -$. Ясно е, че $\C[\tau_i] \equiv \tau_i$.
  \item
    Да фиксираме произволен терм $\rho\in\text{PCF}_{\vv{b}}$.
    Да разгледаме произволен контекст $\C[-]$, за който $\C[\tau_1\rho]$ и $\C[\tau_2\rho]$
    са затворени термове от тип $\nat$.
    Разглеждаме контекста $\C' \df \C[-\rho]$,
    т.е. заменяме $-$ с $-\rho$ в контекста $\C$.
    Лесно се съобразява, че \[\C'[\tau_i] \equiv \C[\tau_i\rho].\]
    Понеже $\tau_1 \leq_{ctx} \tau_2 : \vv{b}\to\vv{c}$,
    то за специалния контекст $\C'$ имаме, че
    \[(\forall \vv{n})[\ \C'[\tau_1] \Downarrow_{\nat} \vv{n} \implies \C'[\tau_2] \Downarrow_{\nat} \vv{n}\ ].\]
    Оттук веднага следва, че за първоначалния контекст $\C$ имаме, че
    \[(\forall \vv{n})[\ \C[\tau_1\rho] \Downarrow_{\nat} \vv{n} \implies \C[\tau_2\rho] \Downarrow_{\nat} \vv{n}\ ].\]
  \end{enumerate}
\end{hint}

\begin{proposition}\label{pr:pcf:context:terms}
  За произволни затворени термове $\tau_1$ и $\tau_2$ е изпълнено, че:
  \[\tau_1 \leq_{ctx} \tau_2 : \vv{a} \iff (\forall \rho \in \text{PCF}_{\vv{a}\to\nat})(\forall \vv{n})[\ \rho\tau_1 \Downarrow_{\nat} \vv{n} \implies \rho\tau_2 \Downarrow_{\nat} \vv{n}\ ].\]
\end{proposition}
\begin{proof}
  За $(\Rightarrow)$, при даден терм $\rho \in \text{PCF}_{\vv{a}\to\nat}$, просто вземете контекстът $\C$ просто да бъде $\C\df\rho~-$. Тогава е ясно, че $\C[\tau_1]$ и $\C[\tau_2]$ са завършени програми и следователно, щом $\tau_1 \leq_{ctx} \tau_2 : \vv{a}$, то е изпълнено, че:
  \[(\forall \vv{n})[\ \underbrace{\C[\tau_1]}_{\rho\tau_1} \Downarrow_{\nat} \vv{n} \implies \underbrace{\C[\tau_2]}_{\rho\tau_2} \Downarrow_{\nat} \vv{n}\ ].\]
  За $(\Leftarrow)$, нека разгледаме произволен контекст $\C[-]$,
  за който $\C[\tau_1] \opsem{}{nat} \vv{n}$. Трябва да докажем, че
  $\C[\tau_2] \opsem{}{nat} \vv{n}$.

  Нека $\vv{z}$ е променлива, която не се среща в $\C[-]$.
  Да разгледаме терма $\rho' \df \C[-/\vv{z}]$, т.е. в контекста $\C$ заменяме $-$ с нова променлива $\vv{z}$. Ясно е, че $\type{z}{a} \vdash \rho' : \nat$.
  Разгледайте затворения терм $\rho \df \lamb{z}{a}{\rho'}$ от тип $\vv{a} \to \nat$.
  Тогава, понеже $\tau_i$ са затворени термове, то
  $\C[\tau_i] \equiv \rho'\subst{z}{\tau_i}$.
  
  Да напомним, че от правилата на операционната семантика имаме, че
  \begin{prooftree}
    \AxiomC{$\rho$ е стойност}
    \LeftLabel{\footnotesize{(val)}}
    \UnaryInfC{$\rho \opsemGen{}{\vv{a}\to\nat} \rho$}
    \AxiomC{$\rho'\subst{z}{\tau_1} \opsem{}{nat} \vv{n}$}
    \RightLabel{\footnotesize{(cbn)}}
    \BinaryInfC{$\rho\tau_1 \opsem{}{nat} \vv{n}$}
  \end{prooftree}
  Това означава, че имаме и $\rho \tau_2 \opsem{}{nat} \vv{n}$.
  Тогава пак според правилата на операционната семантика е ясно, че със сигурност имаме и
  $\rho'\subst{z}{\tau_2} \opsem{}{nat} \vv{n}$.
  Така доказахме, че $\tau_1 \leq_{ctx} \tau_2 : \vv{a}$.
\end{proof}

Сега започва да става интересно, защото ще свържем логическата релация $\triangleleft_{\vv{a}}$ с релацията $\leq_{ctx}$.

\begin{proposition}\label{pr:pcf:context:relation}
  Нека $\tau_1$ и $\tau_2$ са произволни затворени термове от тип $\vv{a}$ и $d$ е елемент на областта на Скот $\val{\vv{a}}$. Тогава
  \[(d \triangleleft_{\vv{a}} \tau_1\ \&\ \tau_1 \leq_{ctx} \tau_2 : \vv{a}) \implies d \triangleleft_{\vv{a}} \tau_2.\]
\end{proposition}
\begin{proof}
  Доказателството следва дефиницията на логическата релация $\triangleleft_{\vv{a}}$, което означава, че
  ще направим индукция по построението на типовете $\vv{a}$.

  Нека $\vv{a} = \nat$.
  Да приемем, че $d \neq \bot$, защото е ясно, че $\bot \triangleleft_{\nat} \tau_2$.
  Понеже $d \triangleleft_{\nat} \tau_1$, то от дефиницията на $\triangleleft_{\nat}$ веднага следва, че $\tau_1 \opsem{}{nat} \vv{d}$.
  Понеже $\tau_1 \leq_{ctx} \tau_2 : \nat$, то според \ref{pr:pcf:context:simple:base} на \Prop{pcf:context:simple} имаме, че $\tau_2 \Downarrow_{\nat} \vv{d}$. Заключаваме, че $d \triangleleft_{\nat} \tau_2$.
  
  Нека $\vv{a} = \arr{b}{c}$ и да приемем, че имаме $d \triangleleft_{\arr{b}{c}} \tau_1$ и $\tau_1 \leq_{ctx} \tau_2 : \arr{b}{c}$.
  Трябва да докажем, че $d \triangleleft_{\vv{b}\to\vv{c}} \tau_2$, което според дефиницията на логическата релация $\triangleleft_{\vv{b}\to\vv{c}}$ означава,
  че за произволни $u \in \val{\vv{b}}$ и $\rho \in \text{PCF}_{\vv{b}}$, за които $u \triangleleft_{\vv{b}} \rho$, трябва да докажем, че $d(u) \triangleleft_{\vv{c}} \tau_2\rho$.
  Да разгледаме произволен такъв елемент $u$ и терм $\rho$. 
  От $d \triangleleft_{\vv{b} \to \vv{c}} \tau_1$ имаме, че $d(u) \triangleleft_{\vv{c}} \tau_1 \rho$.
  От \ref{pr:pcf:context:simple:step} на \Prop{pcf:context:simple} имаме, че щом $\tau_1 \leq_{ctx} \tau_2 : \vv{b} \to \vv{c}$, то $\tau_1 \rho \leq_{ctx} \tau_2 \rho : \vv{c}$.
  Сега от \IndHyp заключаваме, че $d(u) \triangleleft_{\vv{c}} \tau_2 \rho$.
\end{proof}

\begin{important}
  \begin{proposition}\label{pr:pcf:context:relation-characterization}
    За произволни затворени термове $\tau_1$ и $\tau_2$ от тип $\vv{a}$,
    \[\tau_1 \leq_{ctx} \tau_2 : \vv{a} \iff \val{\tau_1} \triangleleft_{\vv{a}} \tau_2.\]
  \end{proposition}    
\end{important}
\begin{proof}
  $(\Rightarrow)$ Нека $\tau_1 \leq_{ctx} \tau_2 : \vv{a}$.
  От \Prop{pcf:adequacy:implication} имаме, че
  \[\val{\tau_1} \triangleleft_{\vv{a}} \tau_1.\]
  Тогава от \Prop{pcf:context:relation} директно следва, че $\val{\tau_1} \triangleleft_{\vv{a}} \tau_2$.

  $(\Leftarrow)$ Нека сега $\val{\tau_1} \triangleleft_{\vv{a}} \tau_2$.
  Тук на помощ ни идва характеризацията на $\leq_{ctx}$ от \Prop{pcf:context:terms}.
  Да разгледаме произволен терм $\rho \in \text{PCF}_{\vv{a} \to \nat}$.
  Трябва да докажем, че:
  \[(\forall \vv{n})[\ \rho\tau_1 \opsem{}{nat} \vv{n} \implies \rho\tau_2 \opsem{}{nat} \vv{n}\ ].\]
  Според \Prop{pcf:adequacy:implication} имаме, че $\val{\rho} \triangleleft_{\arr{a}{nat}} \rho$.
  Тогава от дефиницията на релацията $\triangleleft_{\arr{a}{nat}}$ следва, че
  за произволен елемент $e$ от областта на Скот $\val{\vv{a}}$ и произволен затворен терм $\mu:\vv{a}$,
  ако $e \triangleleft_{\vv{a}} \mu$, то $\val{\rho}(e) \triangleleft_{\nat} \rho\mu$.
  Нека да положим $e \df \val{\tau_1}$ и $\mu \df \tau_2$. Тогава получаваме, че:
  \[\underbrace{\val{\rho\tau_1}}_{\val{\rho}(\val{\tau_1})} \triangleleft_{\nat} \rho\tau_2.\]
  Сега за произволна стойност $\vv{n}$ имаме, че:
  \begin{align*}
    \rho\tau_1 \Downarrow_{\nat} \vv{n} & \implies \val{\rho\tau_1} = n & \comment\text{\hyperref[th:pcf:soundness]{Теорема за коректност}}\\
                                            & \implies \rho\tau_2 \Downarrow_{\nat} \vv{n}. & \comment\text{от }\val{\rho\tau_1} \triangleleft_{\nat} \rho\tau_2
  \end{align*}
  Заключаваме, че $\tau_1 \leq_{ctx} \tau_2 : \vv{a}$.
\end{proof}


\begin{framed}
  \begin{proposition}\label{pr:pcf:context:extensionality}
    За произволни затворени термове $\tau_1$ и $\tau_2$ е изпълнено, че:
    \begin{enumerate}[(1)]
    \item
      \label{pr:pcf:context:extensionality:base}
      $\tau_1 \leq_{ctx} \tau_2 : \nat \iff (\forall \vv{n})[\tau_1 \Downarrow_{\nat} \vv{n} \implies \tau_2 \Downarrow_{\nat} \vv{n}]$;
    \item
      \label{pr:pcf:context:extensionality:step}
      $\tau_1 \leq_{ctx} \tau_2 : \arr{a}{b} \iff (\forall \rho \in \text{PCF}_{\vv{a}})[\ \tau_1\rho \leq_{ctx} \tau_2 \rho : \vv{b}\ ]$.
    \end{enumerate}
  \end{proposition}  
\end{framed}
\begin{proof}
  \begin{enumerate}[(1)]
  \item
    Посоката $(\Rightarrow)$ следва директно от \ref{pr:pcf:context:simple:base} на \Prop{pcf:context:simple}.
    За посоката $(\Leftarrow)$, понеже
    \begin{align*}
      \val{\tau_1} = \val{\vv{n}} & \implies \tau_1 \Downarrow_{\nat} \vv{n} & \comment\text{\hyperref[th:pcf:adequacy]{Теорема за адекватност}}\\
                                  & \implies \tau_2 \Downarrow_{\nat} \vv{n}, & \comment\text{от условието}
    \end{align*}
    то получаваме, че $\val{\tau_1} \triangleleft_{\nat} \tau_2$.
    Тогава от \Prop{pcf:context:relation-characterization} получаваме, че $\tau_1 \leq_{ctx} \tau_2 : \nat$.
  \item
    Посоката $(\Rightarrow)$ е на практика \ref{pr:pcf:context:simple:step} на \Prop{pcf:context:simple}.
    За посоката $(\Leftarrow)$, според \Prop{pcf:context:relation-characterization},
    достатъчно е да докажем, че $\val{\tau_1} \triangleleft_{\vv{a}\to\vv{b}} \tau_2$.
    Според дефиницията на логическата релация $\triangleleft_{\vv{a}\to\vv{b}}$, това означава да
    докажем че за произволни $u \in \val{\vv{a}}$ и $\rho \in \text{PCF}_{\vv{a}}$, за които $u \triangleleft_{\vv{a}} \rho$, то е изпълнено, че $\val{\tau_1}(u) \triangleleft_{\vv{b}} \tau_2\rho$.
    Да разгледаме произволни такъв елемент $u$ и терм $\rho$.
    Тъй като от \Cor{pcf:fundamental} знаем, че $\val{\tau_1} \triangleleft_{\vv{a}\to\vv{b}} \tau_1$, то
    $\val{\tau_1}(u) \triangleleft_{\vv{b}} \tau_1\rho$.
    Но понеже $\tau_1\rho \leq_{ctx} \tau_2 \rho : \vv{b}$, то от \Prop{pcf:context:relation} следва, че
    $\val{\tau_1}(u) \triangleleft_{\vv{b}} \tau_2 \rho$.
  \end{enumerate}
\end{proof}

\begin{problem}
  \label{prob:pcf:context:application}
  Докажете, че винаги можем да направим следния извод:
  \begin{prooftree}
    \AxiomC{$\tau_1 \leq_{ctx} \tau_2 : \arr{a}{b}$}
    \AxiomC{$\rho_1 \leq_{ctx} \rho_2 : \vv{a}$}
    \BinaryInfC{$\tau_1\rho_1 \leq_{ctx} \tau_2\rho_2 : \vv{b}$}
  \end{prooftree}
\end{problem}
\begin{hint}
  От $\tau_1 \leq_{ctx} \tau_2 : \arr{a}{b}$ имаме, че $\val{\tau_1} \triangleleft_{\arr{a}{b}} \tau_2$.
  Тогава за произволен елемент $u \in \val{\vv{a}}$ и произволен $\mu \in \PCF_{\vv{a}}$,
  за които $u \triangleleft_{\vv{a}} \mu$ е изпълнено, че $\val{\tau_1}(u) \triangleleft_{\vv{b}} \tau_2\mu$.
  
  От $\rho_1 \leq_{ctx} \rho_2 : \vv{a}$ имаме, че $\val{\rho_1} \triangleleft_{\vv{b}} \rho_2$.
  Нека положим $u \df \val{\rho_1}$ и $\mu \df \rho_2$.
  Тогава получаваме, че $\val{\tau_1}(\val{\rho_1}) \triangleleft_{\vv{b}} \tau_2 \rho_2$.
  Сега заключаваме, че $\tau_1\rho_1 \leq_{ctx} \tau_2\rho_2 : \vv{b}$.
\end{hint}

Естествено е да се запитаме дали можем да разширим \ref{pr:pcf:context:extensionality:base} на \Prop{pcf:context:extensionality} за по-сложни от $\nat$ типове $\vv{a}$, т.е. възможно ли е
\[\tau_1 \leq_{ctx} \tau_2 : \vv{a} \iff (\forall \vv{v})[\tau_1 \Downarrow_{\vv{a}} \vv{v} \implies \tau_2 \Downarrow_{\vv{a}} \vv{v}]?\]
Би било странно, ако можем, защото това би обезсмилило разглеждането на релацията $\leq_{ctx}$.
Първо в \Prop{context:op-left-right} ще видим, че винаги имаме импликацията $(\Leftarrow)$, но по-късно в \Prop{context:op-right-left} ще видим, че дори за типа $\vv{a} = \nat\to\nat$ нямаме импликация $(\Rightarrow)$.

\begin{proposition}\label{pr:context:op-left-right}
  Докажете, че за всеки два затворени терма $\tau_1, \tau_2$ от тип $\vv{a}$,
  \[(\forall \vv{v})[\tau_1 \Downarrow_{\vv{a}} \vv{v} \implies \tau_2 \Downarrow_{\vv{a}} \vv{v} ] \implies \tau_1 \leq_{ctx} \tau_2 : \vv{a}.\]
\end{proposition}
\begin{proof}
  Имаме всичко необходимо за да докажем това твърдение:
  \begin{prooftree}
    \AxiomC{от условието}
    \UnaryInfC{$(\forall \vv{v})[\tau_1 \Downarrow_{\vv{a}} \vv{v} \implies \tau_2 \Downarrow_{\vv{a}} \vv{v}]$.}
    \AxiomC{\Cor{pcf:fundamental}}
    \UnaryInfC{$\val{\tau_1} \triangleleft_{\vv{a}} \tau_1$}
    \RightLabel{\footnotesize{(\Prop{pcf:adequacy:implication})}}
    \BinaryInfC{$\val{\tau_1} \triangleleft_{\vv{a}} \tau_2$}
    \RightLabel{\footnotesize{(\Prop{pcf:context:relation-characterization})}}
    \UnaryInfC{$\tau_1 \leq_{ctx} \tau_2 : \vv{a}$}
  \end{prooftree}
\end{proof}

\begin{proposition}\label{pr:context:den-left-right}
  За произволни затворени термове $\tau_1$ и $\tau_2$ от тип $\vv{a}$,
  \[\val{\tau_1} \sqsubseteq \val{\tau_2} \implies \tau_1 \leq_{ctx} \tau_2 : \vv{a}.\]
\end{proposition}  
\begin{proof}
  Според \Prop{pcf:context:terms}, достатъчно е да докажем, че за произволен терм $\rho \in \text{PCF}_{\vv{a}\to\nat}$ имаме импликацията:
  \[(\forall \vv{n})[\ \rho\tau_1 \Downarrow_{\nat} \vv{n} \implies \rho\tau_2 \Downarrow_{\nat} \vv{n}\ ].\]
  Но понеже $\val{\tau_1} \sqsubseteq \val{\tau_2}$, това е лесно:
  \begin{align*}
    \rho\tau_1 \Downarrow_{\nat} \vv{n} & \implies \val{\rho\tau_1} = \val{\vv{n}} & \comment\text{\hyperref[th:pcf:soundness]{Теорема за коректност}}\\
                                            & \implies \val{\rho}(\val{\tau_1}) = \val{\vv{n}} \\% & \comment\text{\hyperref[lem:pcf:substitution]{Лема за замяната}}\\
                                            & \implies \val{\rho}(\val{\tau_2}) = \val{\vv{n}} & \comment\text{монотонност на }\val{\rho}\\
                                            & \implies \val{\rho\tau_2} = \val{\vv{n}}\\ % & \comment\text{\hyperref[lem:pcf:substitution]{Лема за замяната}}\\
                                            & \implies \rho\tau_2 \Downarrow_{\nat} \vv{n}. & \comment\text{\hyperref[th:pcf:adequacy]{Теорема за адекватност}}
  \end{align*}
\end{proof}

Можем да обобщим всичко, което сме направили до момента като формулираме следната теорема.

\begin{framed}
  \begin{theorem}\label{th:pcf:context:connection}
    За всички затворени термове $\tau_1$ и $\tau_2$ от тип $\vv{a}$ е изпълнено, че:
    \begin{enumerate}[(1)]
    \item
      \label{pcf:context:connection:operational}
      $(\forall \vv{v})[\tau_1 \Downarrow_{\vv{a}} \vv{v} \iff \tau_2 \Downarrow_{\vv{a}} \vv{v} ] \implies \tau_1 \cong_{ctx} \tau_2 : \vv{a}$;
    \item
      \label{pcf:context:connection:denotational}
      $\val{\tau_1} = \val{\tau_2} \implies \tau_1 \cong_{ctx} \tau_2 : \vv{a}$.
    \end{enumerate}
  \end{theorem}
\end{framed}

В частния случай на $\vv{a} = \nat$ имаме и обратните импликации на \Th{pcf:context:connection}.

\begin{framed}
    \begin{corollary}\label{cr:pcf:context:connection}
    За всички затворени термове $\tau_1$ и $\tau_2$ от тип $\nat$ е изпълнено, че:
    \begin{enumerate}[(1)]
    \item
      \label{cr:pcf:context:connection:operational}
      $(\forall \vv{n})[\tau_1 \opsem{}{nat} \vv{n} \iff \tau_2 \opsem{}{nat} \vv{n} ] \iff \tau_1 \cong_{ctx} \tau_2 : \nat$;
    \item
      \label{cr:pcf:context:connection:denotational}
      $\val{\tau_1} = \val{\tau_2} \iff \tau_1 \cong_{ctx} \tau_2 : \nat$.
    \end{enumerate}
  \end{corollary}
\end{framed}
\begin{proof}
  (1) представлява точно \ref{pr:pcf:context:extensionality:base} на \Prop{pcf:context:extensionality}.
  За обратната посока на (2), нека $\tau_1 \leq_{ctx} \tau_2 : \nat$. Трябва да докажем, че $\val{\tau_1} \sqsubseteq \val{\tau_2}$.
  Нека $\val{\tau_1} = n \neq \bot$, защото ако $\val{\tau_1} = \bot$, то е ясно, че $\val{\tau_1} \sqsubseteq \val{\tau_2}$.
  Имаме импликациите:
  \begin{align*}
    \val{\tau_1} = n & \implies \tau_1 \opsem{}{nat} \vv{n} & \comment\text{\hyperref[th:pcf:adequacy]{Теорема за адекватност}}\\
                     & \implies \tau_2 \opsem{}{nat} \vv{n} & \comment\text{\ref{pr:pcf:context:extensionality:base} на \Prop{pcf:context:extensionality}}\\
                     & \implies \val{\tau_2} = n. & \comment\text{\hyperref[th:pcf:soundness]{Теорема за коректност}}
  \end{align*} 
\end{proof}

Нека сега да видим, че нямаме обратните импликации на \Cor{pcf:context:connection} за по-високи типове от $\nat$ като разгледаме термовете $\Omega_{\vv{a}}$ и $\Omega'_{\vv{a}}$, които сме срещали и преди
и сме разглеждали техните свойства в \Problem{pcf:context:omega}.

\begin{framed}
\begin{proposition}\label{pr:context:op-right-left}
  $\Omega'_{\nat} \cong_{ctx} \Omega_{\nat\to\nat} : \arr{nat}{nat}$.
\end{proposition}  
\end{framed}
\marginpar{Да напомним, че
  \begin{align*}
    \Omega_{\vv{a}} &  \df \fix(\lamb{x}{a}{\vv{x}})\\
    \Omega'_{\vv{a}} & \df \lamb{x}{a}{\Omega_{\vv{a}}}.
  \end{align*}}
\begin{proof}
  % Първо да видим защо $\Omega'_{\nat} \leq_{ctx} \Omega_{\nat\to\nat} : \arr{nat}{nat}$.
  % Според \ref{pr:pcf:context:extensionality:step} на \Prop{pcf:context:extensionality}, трябва да докажем, че за
  % произволен затворен терм $\rho$ от тип $\nat$ е изпълнено, че
  % \[\Omega'_{\nat}\rho \leq_{ctx} \Omega_{\nat\to\nat}\rho : \nat,\]
  % което според \ref{pr:pcf:context:extensionality:base} на \Prop{pcf:context:extensionality} означава да докажем, че
  % \[(\forall \vv{n})[\ \Omega'_{\nat}\rho \opsem{}{nat} \vv{n} \implies \Omega_{\nat\to\nat} \rho \opsem{}{nat} \vv{n}\ ].\]
  % Да отбележим, че $\Omega'_{\nat} \rho \not\opsem{}{nat}$, защото $\Omega_{\nat}$ е затворен терм и $\Omega_{\nat}\subst{x}{\rho} \equiv \Omega_{\nat}$ и според правилата на операционната семантика имаме следното изчисление:
  % \begin{prooftree}
  %   \AxiomC{$\Omega'_{\nat}$ е стойност}
  %   \LeftLabel{\footnotesize{(val)}}
  %   \UnaryInfC{$\Omega'_{\nat} \opsem{0}{nat} \Omega'_{\nat}$}
  %   \AxiomC{\ref{pcf:omega:operational} на \Problem{pcf:context:omega}}
  %   \UnaryInfC{$\Omega_{\nat} \not\opsem{}{nat}$}
  %   \UnaryInfC{$\Omega_{\nat}\subst{x}{\rho} \not\opsem{}{nat}$}
  %   \RightLabel{\footnotesize{(cbn)}}    
  %   \BinaryInfC{$\Omega'_{\nat} \rho \not\opsem{}{nat}$}
  % \end{prooftree}
  % Тогава заключаваме, че за всеки затворен терм $\rho$ от тип $\nat$, е изпълнено
  % \[(\forall \vv{n})[\ \Omega'_{\nat}\rho \Downarrow_{\nat} \vv{n} \implies \Omega_{\nat\to\nat} \rho \Downarrow_{\nat} \vv{n}\ ],\]
  % защото лявата част на импликацията никога не е вярна.

  % Нека сега да видим защо $\Omega_{\nat\to\nat} \leq_{ctx} \Omega'_{\nat} : \nat \to \nat$. Разсъждаваме по сходен начин.
  % Ще докажем, че за произволен затворен терм $\rho$ от тип $\nat$ е изпълнено, че:
  % \[(\forall \vv{n})[\ \Omega_{\nat\to\nat}\rho \opsem{}{nat} \vv{n} \implies \Omega'_{\nat} \rho \opsem{}{nat} \vv{n}\ ].\]
  % Според правилата на операционната семантиката получаваме следното:
  % \begin{prooftree}
  %   \AxiomC{\ref{pcf:omega:operational} на \Problem{pcf:context:omega}}
  %   \UnaryInfC{$\Omega_{\nat\to\nat} \not\opsemGen{}{\nat\to\nat}$}
  %   \AxiomC{$...$}
  %   \RightLabel{\footnotesize{(cbn)}}
  %   \BinaryInfC{$\Omega_{\nat\to\nat}\rho \not\opsemGen{}{\nat\to\nat}$}
  % \end{prooftree}
  % Отново заключаваме, че за всеки затворен терм $\rho$ от тип $\nat$, е изпълнено
  % \[(\forall \vv{n})[\ \Omega_{\nat\to\nat}\rho \Downarrow_{\nat} \vv{n} \implies \Omega'_{\nat} \rho \Downarrow_{\nat} \vv{n}\ ],\]
  % защото лявата част на импликацията никога не е вярна.
  Понеже ние вече знаем, че $\val{\Omega'_{\nat}} = \val{\Omega_{\arr{nat}{nat}}}$, то
  според \ref{pcf:context:connection:denotational} на \Th{pcf:context:connection} следва, че
  $\Omega'_\nat \eqCtx \Omega_{\arr{nat}{nat}}$.
\end{proof}

Знаем, че $\Omega_{\nat\to\nat} \not\opsemGen{}{\nat\to\nat}$, но $\Omega'_{\nat} \opsemGen{}{\nat\to\nat} \Omega'_{\nat}$.
Това означава, че според \Prop{context:op-right-left}, термовете $\Omega_{\nat\to\nat}$ и $\Omega'_{\nat}$ ни дават пример кога нямаме обратната импликация в (1) на \Th{pcf:context:connection}.
% Обърнете внимание, че $\val{\Omega_{\nat\to\nat}} = \val{\Omega'_{\nat}}$.
Следователно, трябва да продължим да търсим термове $\tau_1$ и $\tau_2$ от някакъв тип $\vv{a}$, за които $\val{\tau_1} \neq \val{\tau_2}$
и $\tau_1 \cong_{ctx} \tau_2 : \vv{a}$.

\newpage

\begin{problem}
  \marginpar{Кеймбридж 2010 г.}
  Докажете, че е изпълнена импликацията:
  \[\tau_1 \cong_{ctx} \tau_2 : \nat\to\nat \implies \val{\tau_1} = \val{\tau_2}.\]  
\end{problem}
\begin{proof}
  За да проверим това е достатъчно да проверим, че ако $\tau_1 \leq_{ctx} \tau_2 : \nat\to\nat$, то $\val{\tau_1} \sqsubseteq \val{\tau_2}$.
  И така, да приемем, че имаме $\tau_1 \leq_{ctx} \tau_2 : \nat\to\nat$. Според \ref{pr:pcf:context:extensionality:step} на \Prop{pcf:context:extensionality}, това означава, че имаме
  за всеки затворен терм $\rho$ от тип $\nat$, $\tau_1 \rho \leq_{ctx} \tau_2 \rho : \nat$.
  Сега тук използваме \ref{pcf:context:connection:operational}, откъдето следва, че
  \begin{equation}
    \label{eq:context:1}
    \val{\tau_1\rho} \sqsubseteq \val{\tau_2\rho}.
  \end{equation}
  % Тук сега използваме \ref{pr:pcf:context:extensionality:base} на \Prop{pcf:context:extensionality} откъде получаваме, че
  % \begin{equation}
  %   \label{eq:context:1}
  %   (\forall \vv{k})[\tau_1 \rho \opsem{}{nat} \vv{k} \implies \tau_2 \rho \opsem{}{nat} \vv{k}].
  % \end{equation}
  \marginpar{Тук важното е това, че всеки елемент на $\Nat_\bot$ е определим с терм в езика \PCF.}
  Нека разгледаме произволен елемент $n \in \Nat_\bot$. 
  \begin{itemize}
  \item
    Ако $n \neq \bot$, то нека положим $\rho \df \vv{n}$. 
  \item
    Ако $n = \bot$, то нека положим $\rho \df \Omega_{\nat}$.
  \end{itemize}
  Ясно е, че сме избрали затворения терм $\rho$ от тип $\nat$ така, че $\val{\rho} = n$.
  Достатъчно е да проследим следните импликации
  \begin{align*}
    \val{\tau_1}(n) & = \val{\tau_1}(\val{\rho})\\
                    & = \val{\tau_1\rho}\\
                    & \sqsubseteq \val{\tau_2\rho} & \comment\text{от (\ref{eq:context:1})}\\
                    & = \val{\tau_2}(\val{\rho})\\
                    & = \val{\tau_2}(n).
  \end{align*}
  % \begin{align*}
  %   \val{\tau_1}(n) = k & \implies \val{\tau_1}(\val{\rho}) = k & \comment\val{\rho} = n\\
  %                       & \implies \val{\tau_1 \rho} = k \\
  %                       & \implies \tau_1 \rho \opsem{}{nat} \vv{k} & \comment\text{\hyperref[th:pcf:adequacy]{Теорема за адекватност}}\\
  %                       & \implies\tau_2 \rho \opsem{}{nat} \vv{k} & \comment\text{от (\ref{eq:context:1})}\\
  %                       & \implies \val{\tau_2 \rho} = k & \comment\text{\hyperref[th:pcf:soundness]{Теорема за коректност}}\\
  %                       & \implies \val{\tau_2}(\val{\rho}) = k\\
  %                       & \implies \val{\tau_2}(n) = k, & \comment\val{\rho} = n
  % \end{align*}
  за да се убедим, че $\val{\tau_1} \sqsubseteq \val{\tau_2}$.
\end{proof}

\begin{problem}
  \label{prob:context:chain:implication}
  Нека $\vv{a} = \nat \to \nat \to \cdots \to \nat$.
  Докажете, че 
  \[\tau_1 \cong_{ctx} \tau_2 : \vv{a} \implies \val{\tau_1} = \val{\tau_2}.\]  
\end{problem}

Това означава, че ако искаме да намерим затворени термове $\tau_1$ и $\tau_2$ от тип $\vv{a}$, за които $\val{\tau_1} \neq \val{\tau_2}$, то $\tau_1 \cong_{ctx} \tau_2 : \vv{a}$, то трябва да разгледаме по-сложни типове $\vv{a}$.

\begin{problem}
  Докажете или опровергайте твърденията:
  \begin{enumerate}[(1)]
  \item
    $\lamb{x}{nat}{\vv{x}} \cong_{ctx} \lamb{x}{nat}{\vv{x + 0}} : \arr{nat}{nat}$;
  \item
    $\lamb{x}{nat}{\vv{x}} \cong_{ctx} \lamb{x}{nat}{\vv{(x - 1) + 1}} : \arr{nat}{nat}$;
  \end{enumerate}

\end{problem}

\begin{problem}
  \marginpar{Задача в Кеймбридж 2020 г. \cite{cambridge-website}}
  Докажете или опровергайте твърдението, че за всеки два типа $\vv{a}$ и $\vv{b}$
  и всеки два затворени терма $\tau_1$ от тип $\arr{a}{b}$ и $\tau_2$ от тип $\arr{b}{a}$
  е изпълнено следното:  
  \[\texttt{fix}(\lamb{y}{b}{\tau_1(\tau_2(\vv{y}))}) \cong_{ctx} \tau_1(\texttt{fix}(\lamb{x}{a}{\tau_2(\tau_1(\vv{x}))})) : \vv{b}.\]  
\end{problem}
\begin{hint}
  Използвайте \Problem{domains:lfp:compositon}.
\end{hint}

\begin{problem}
  \marginpar{Задача в Кеймбридж 2015 г. \cite{cambridge-website}}
  % Докажете или опровергайте, че винаги можем да направим следния извод:
  % \begin{prooftree}
  %   \AxiomC{$\tau_1 \eqCtx \tau_2 : \arr{a}{b}$}
  %   \AxiomC{$\mu_1 \eqCtx \mu_2 : \arr{b}{c}$}
  %   \BinaryInfC{$\lamb{x}{a}{\mu_1(\tau_1 \vv{x})} \eqCtx \lamb{x}{a}{\mu_2(\tau_2 \vv{x})} : \arr{a}{c}$}
  % \end{prooftree}

  
  Нека $\tau_1$ и $\tau_2$ са затворени термове от тип $\arr{a}{b}$ и
  нека $\mu_1$ и $\mu_2$ са затворени термове от тип $\arr{b}{c}$, за които:
  \begin{align*}
    & \tau_1 \eqCtx \tau_2 : \arr{a}{b}\\
    & \mu_1 \eqCtx \mu_2 : \arr{b}{c}.
  \end{align*}
  Докажете или опровергайте, че тогава имаме еквивалентността:
  \begin{equation}
    \label{eq:pcf:context:2}
    \lamb{x}{a}{\mu_1(\tau_1 \vv{x})} \eqCtx \lamb{x}{a}{\mu_2(\tau_2 \vv{x})} : \arr{a}{c}.
  \end{equation}
\end{problem}
\begin{hint}
  За да докажем, че еквивалентността (\ref{eq:pcf:context:2}) е изпълнена, според \ref{pr:pcf:context:extensionality:step} на \Prop{pcf:context:extensionality} е достатъчно да докажем, че за произволен затворен терм $\rho$ от тип $\vv{a}$ е изпълнено, че:
  \[(\lamb{x}{a}{\mu_1(\tau_1 \vv{x})})\rho \eqCtx (\lamb{x}{a}{\mu_2(\tau_2 \vv{x})})\rho : \vv{c}.\]
  Сега, ще използваме равенствата за $i = 1,2$:
  \[\val{(\lamb{x}{a}{\mu_i(\tau_i \vv{x})})\rho}= \val{\mu_i}(\val{\tau_i}(\val{\rho})) = \val{\mu_i(\tau_i \rho)},\]
  откъдето според \ref{pcf:context:connection:denotational} на \Th{pcf:context:connection} следва, че
  \[(\lamb{x}{a}{\mu_i(\tau_i \vv{x})})\rho \eqCtx \mu_i(\tau_i \rho).\]
  Оттук следва, че е достатъчно да докажем, че
  \begin{equation}
    \label{eq:pcf:context:3}
    \mu_1(\tau_1 \rho) \eqCtx \mu_2(\tau_2 \rho) : \vv{c}.
  \end{equation}
  Но това е лесно да се види, защото, щом $\tau_1 \eqCtx \tau_2 : \arr{a}{b}$,
  то според \ref{pr:pcf:context:extensionality:step} на \Prop{pcf:context:extensionality} имаме, че
  $\tau_1\rho \eqCtx \tau_2\rho : \vv{b}$,
  Сега прилагаме \Problem{pcf:context:application} и оттам веднага получаваме (\ref{eq:pcf:context:3}).
\end{hint}


%%% Local Variables:
%%% mode: latex
%%% TeX-master: "../sep"
%%% End:

\newpage
\section{Езикът \texttt{PCF(bool)}}

\marginpar{Всъщност в \cite[Глава 4.1]{gunter} това е ,,истинската'' дефиниция на езика \PCF.}

\begin{itemize}
\item
  Типове
  \[\vv{a} ::= \bool\ |\ \nat\ |\ \vv{a}\to\vv{a}\]
\item
  Изрази
  \begin{align*}
    \tau ::=\ & \tru\ |\ \fls\ |\ \vv{n}\ |\ \vv{x}\ |\ \tau_1 + \tau_2\ |\ \tau_1 - \tau_2\ |\  \tau_1\ \vv{==}\ \tau_2\ |\\
              & \ifelse{\tau_1}{\tau_2}{\tau_3}\ |\ \tau_1\tau_2\ |\ \lamb{x}{a}{\tau_1}\ |\ \fix(\tau_1).
  \end{align*}
  % Термове
  % \begin{align*}
  %   \tau ::=\ & \vv{0}\ |\ \tru\ |\ \fls\ |\ \vv{x}\ |\\
  %            & \scc(\tau_1)\ |\ \prd(\tau_1)\ |\ \texttt{iszero}(\tau)\ |\ \ifelse{\tau_1}{\tau_2}{\tau_3}\ |\\
  %            &\tau_1\tau_2\ |\ \lamb{x}{a}{\tau_1}\ |\ \fix(\tau_1).
  % \end{align*}
\item
  Стойностите са затворени термове от следния вид:
  \[\vv{v} ::= \tru\ |\ \fls\ |\ \vv{n}\ |\ \lamb{x}{a}{\mu}\]
\end{itemize}

\subsection{Типизираща релация}

Релацията $\Gamma \vdash \tau : \vv{a}$ за езика \PCFBOOL е почти същата
както за езика \PCF. Имаме две нови правила:

\begin{figure}[H]
  \begin{subfigure}[b]{0.5\textwidth}
    \begin{prooftree}
      \AxiomC{}
      \RightLabel{\scriptsize{(true)}}
      \UnaryInfC{$\Gamma \vdash \tru : \bool$}
    \end{prooftree}
  \end{subfigure}
  ~
  \begin{subfigure}[b]{0.5\textwidth}
    \begin{prooftree}
      \AxiomC{}
      \RightLabel{\scriptsize{(false)}}
      \UnaryInfC{$\Gamma \vdash \fls : \bool$}
    \end{prooftree}
  \end{subfigure}
\end{figure}
Имаме и две променени правила:
\begin{prooftree}
  \AxiomC{$\Gamma \vdash \tau_1:\nat$}
  \AxiomC{$\Gamma \vdash \tau_2:\nat$}
  \RightLabel{\scriptsize{(eq)}}
  \BinaryInfC{$\Gamma \vdash \tau_1\ \vv{==}\ \tau_2 : \bool$}
\end{prooftree}
\begin{prooftree}
  \AxiomC{$\Gamma \vdash \tau_1:\bool$}
  \AxiomC{$\Gamma \vdash \tau_2:\vv{a}$}
  \AxiomC{$\Gamma \vdash \tau_3:\vv{a}$}
  \RightLabel{\scriptsize{(if)}}
  \TrinaryInfC{$\Gamma \vdash \ifelse{\tau_1}{\tau_2}{\tau_3} : \vv{a}$}
\end{prooftree}
Всички останали правила са същите.

% \end{subfigure}
% ~
% \begin{subfigure}[b]{0.5\textwidth}
% \begin{prooftree}
%   \AxiomC{$\Gamma \vdash \tau:\vv{a}\to\vv{a}$}
%   \RightLabel{\scriptsize{(fix)}}
%   \UnaryInfC{$\Gamma \vdash \fix(\tau) : \vv{a}$}
% \end{prooftree}
% \end{subfigure}

% \vspace{10pt}

% \begin{subfigure}[b]{0.5\textwidth}
% \begin{prooftree}
%   \AxiomC{$\Gamma \vdash \tau_1:\vv{a}\to\vv{b}$}
%   \AxiomC{$\Gamma \vdash \tau_2:\vv{a}$}
%   \RightLabel{\scriptsize{(app)}}
%   \BinaryInfC{$\Gamma \vdash \tau_1\tau_2 : \vv{b}$}
% \end{prooftree}
% \end{subfigure}
% ~
% \begin{subfigure}[b]{0.5\textwidth}
% \begin{prooftree}
%   \AxiomC{$\vv{x} \not\in\vv{dom}(\Gamma)$}
%   \AxiomC{$\Gamma, \type{x}{a} \vdash \tau:\vv{b}$}
%   \RightLabel{\scriptsize{(lambda)}}
%   \BinaryInfC{$\Gamma \vdash \lambda \type{x}{a}\ .\ \tau : \vv{a} \to \vv{b}$}
% \end{prooftree}
% \end{subfigure}

% \caption{Релация за типизиране на термовете от езика \texttt{PCF++}}
% \label{fig:pcf:extensions:relation}
% \end{figure}


\subsection{Операционна семантика}

% \marginpar{\cite[стр. 109]{gunter}}

Тук отново всичко е почти същото както преди със следните разлики:

\begin{figure}[H]
  % \begin{subfigure}[b]{0.5\textwidth}
  %   \begin{prooftree}
  %     \AxiomC{$\type{v}{a}$}
  %     \RightLabel{\scriptsize{(val)}}
  %     \UnaryInfC{$\vv{v} \Downarrow^0_{\vv{a}} \vv{v}$}
  %   \end{prooftree}    
  % \end{subfigure}
  % ~
  % \begin{subfigure}[b]{0.5\textwidth}
  %   \begin{prooftree}
  %     \AxiomC{$\tau \Downarrow^\ell_{\nat} \vv{0}$}
  %     \UnaryInfC{$\prd(\tau) \Downarrow^{\ell+1}_{\nat} \vv{0}$}
  %   \end{prooftree}
  % \end{subfigure}

  % \vspace{10pt}

  % \begin{subfigure}[b]{0.5\textwidth}
  %   \begin{prooftree}
  %     \AxiomC{$\tau \Downarrow^\ell_{\nat} \scc(\vv{v})$}
  %     \UnaryInfC{$\prd(\tau) \Downarrow^{\ell+1}_{\nat} \vv{v}$}
  %   \end{prooftree}
  % \end{subfigure}
  % ~
  % \begin{subfigure}[b]{0.5\textwidth}
  %   \begin{prooftree}
  %     \AxiomC{$\tau \Downarrow^\ell_{\nat} \vv{v}$}
  %     \UnaryInfC{$\scc(\tau) \Downarrow^{\ell+1}_{\nat} \scc(\vv{v})$}
  %   \end{prooftree}
  % \end{subfigure}

  % \vspace{10pt}

  % \begin{subfigure}[b]{0.5\textwidth}
  %   \begin{prooftree}
  %     \AxiomC{$\tau \Downarrow^\ell_{\nat} \vv{0}$}
  %     \UnaryInfC{$\iszero(\tau) \Downarrow^{\ell+1}_{\bool} \tru$}
  %   \end{prooftree}
  % \end{subfigure}
  % ~
  % \begin{subfigure}[b]{0.5\textwidth}
  %   \begin{prooftree}
  %     \AxiomC{$\tau \Downarrow^\ell_{\nat} \scc(\vv{v})$}
  %     \UnaryInfC{$\iszero(\tau) \Downarrow^{\ell+1}_{\bool} \fls$}
  %   \end{prooftree}
  % \end{subfigure}

  % \vspace{10pt}
  
  \begin{subfigure}[b]{0.5\textwidth}
    \begin{prooftree}
      \AxiomC{$\tau_1 \opsem{\ell_1}{nat} \vv{v}_1$}
      \AxiomC{$\tau_3 \opsem{\ell_2}{a} \vv{v}_2$}
      \AxiomC{$\vv{v}_1 \equiv \vv{v}_2$}
      % \RightLabel{\scriptsize{(if$_\fls$)}}
      \TrinaryInfC{$\tau_1\ \vv{==}\ \tau_2 \opsem{\ell_1+\ell_2+1}{bool} \tru$}
    \end{prooftree}
  \end{subfigure}
  ~
  \begin{subfigure}[b]{0.5\textwidth}
    \begin{prooftree}
      \AxiomC{$\tau_1 \opsem{\ell_1}{nat} \vv{v}_1$}
      \AxiomC{$\tau_3 \opsem{\ell_2}{a} \vv{v}_2$}
      \AxiomC{$\vv{v}_1 \not\equiv \vv{v}_2$}
      % \RightLabel{\scriptsize{(if$_\fls$)}}
      \TrinaryInfC{$\tau_1\ \vv{==}\ \tau_2 \opsem{\ell_1+\ell_2+1}{bool} \fls$}
    \end{prooftree}
  \end{subfigure}

  \vspace{10pt}
  
  \begin{subfigure}[b]{0.5\textwidth}
    \begin{prooftree}
      \AxiomC{$\tau_1 \opsem{\ell_1}{bool} \fls$}
      \AxiomC{$\tau_3 \opsem{\ell_2}{a} \vv{v}$}
      % \RightLabel{\scriptsize{(if$_\fls$)}}
      \BinaryInfC{$\ifelse{\tau_1}{\tau_2}{\tau_3} \opsem{\ell_1+\ell_2+1}{a} \vv{v}$}
    \end{prooftree}
  \end{subfigure}
  ~
  \begin{subfigure}[b]{0.5\textwidth}
    \begin{prooftree}
      \AxiomC{$\tau_1 \opsem{\ell_1}{bool} \tru$}
      \AxiomC{$\tau_2 \opsem{\ell_2}{a} \vv{v}$}
      % \RightLabel{\scriptsize{(if$_\tru$)}}
      \BinaryInfC{$\ifelse{\tau_1}{\tau_2}{\tau_3} \opsem{\ell_1+\ell_2+1}{a} \vv{v}$}
    \end{prooftree}
  \end{subfigure}

%   \vspace{10pt}

%   \begin{subfigure}[b]{0.5\textwidth}
%     \begin{prooftree}
%       \AxiomC{$\tau_1 \Downarrow^{\ell_1}_{\vv{a}\to\vv{b}} \lamb{x}{a}{\tau'_1}$}
%       \AxiomC{$\tau'_1[x/\tau_2] \Downarrow^{\ell_2}_{\vv{b}} \vv{v}$}
%       % \RightLabel{\scriptsize{(cbn)}}
%       \BinaryInfC{$\tau_1 \tau_2 \Downarrow^{\ell_1+\ell_2+1}_{\vv{b}} \vv{v} $}
%     \end{prooftree}
%   \end{subfigure}
%   ~
%   \begin{subfigure}[b]{0.5\textwidth}
%   \begin{prooftree}
%     \AxiomC{$\tau\ \fix(\tau) \Downarrow^{\ell}_{\vv{a}} \vv{v}$}
%     \RightLabel{\scriptsize{(fix)}}
%     \UnaryInfC{$\fix(\tau) \Downarrow^{\ell+1}_{\vv{a}} \vv{v} $}
%   \end{prooftree}
% \end{subfigure}
% \caption{Правила на операционната семантика за езика \PCFPP}
\end{figure}




% \begin{lemma}
%   Нека $\tau$ е затворен терм от тип $\vv{a}$.
%   Тогава ако $\tau \Downarrow_{\vv{a}} \vv{v}$ и $\tau \Downarrow_{\vv{a}} \vv{u}$, то
%   $\vv{v} \equiv_\alpha \vv{u}$.
% \end{lemma}


\subsection{Денотационна семантика}

\marginpar{В \cite[Глава 4.3]{gunter} се нарича \emph{standard fixed-point semantics} of PCF.}

Семантиката на всеки тип ще бъде област на Скот както следва:
% \begin{align*}
\[\val{\bool} \df \Bool = \{true, false\}_\bot.\]
%   & \val{\nat} \df \Nat_\bot\\
%   & \val{\vv{a} \to \vv{b}} \df \Cont{\val{\vv{a}}}{\val{\vv{b}}}.
% \end{align*}

\begin{itemize}
% \item
%   Нека $\tau \equiv \vv{0}$. Тогава
%   \[\val{\vv{0}}_\Gamma(\overline{u}) \df 0.\]
\item
  Нека $\tau \equiv \tru$. Тогава
  \[\val{\tru}_\Gamma(\overline{u}) \df true.\]
\item
  Нека $\tau \equiv \fls$. Тогава
  \[\val{\fls}_\Gamma(\overline{u}) \df false.\]
% \item
%   Нека $\tau \equiv \vv{x}_i$. Тогава
%   \[\val{\vv{x}_i}_\Gamma(\overline{u}) \df u_i.\]
% \item
%   Нека $\tau \equiv \scc(\tau_1)$. Тогава
%   \[\val{\scc(\tau_1)}_\Gamma(\ov{u}) \df
%   \begin{cases}
%     \val{\tau_1}_\Gamma(\ov{u}) + 1, & \text{ ако }\val{\tau_1}_\Gamma(\ov{u}) \neq \bot\\
%     \bot, & \text{ ако }\val{\tau_1}_\Gamma(\ov{u}) = \bot.
%   \end{cases}\]

% \item
%   Нека $\tau \equiv \prd(\tau_1)$. Тогава
%   \[\val{\prd(\tau_1)}_\Gamma(\ov{u}) \df
%   \begin{cases}
%     0, & \text{ ако }\val{\tau_1}(\ov{u}) = 0\\
%     \val{\tau_1}_\Gamma(\ov{u}) - 1, & \text{ ако }\val{\tau_1}_\Gamma(\ov{u}) \neq 0, \bot\\
%     \bot, & \text{ ако }\val{\tau_1}_\Gamma(\ov{u}) = \bot.
%   \end{cases}\]

% \item
%   Нека $\tau \equiv \iszero(\tau_1)$. Тогава
%   \[\val{\iszero(\tau_1)}_\Gamma(\ov{u}) \df
%   \begin{cases}
%     true, &  \text{ ако }\val{\tau_1}_\Gamma(\ov{u}) = 0\\
%     false, & \text{ ако }\val{\tau_1}_\Gamma(\ov{u}) \neq 0,\bot\\
%     \bot, &  \text{ ако }\val{\tau_1}_\Gamma(\ov{u}) = \bot.
%   \end{cases}\]


% \item
%   \marginpar{За $\texttt{eq}$ вижте Раздел~\ref{subsect:rec:term-value}.}
%   Нека $\tau \equiv \tau_1\ \vv{==}\ \tau_2$. Тогава
%   \[\val{\tau_1\ \vv{==}\ \tau_2}_\Gamma(\overline{u}) \df \texttt{eq}(\val{\tau_1}_\Gamma(\overline{u}), \val{\tau_2}_\Gamma(\overline{u})).\]

\item
  Нека $\tau \equiv \tau_1\ \vv{==}\ \tau_2$. Тогава
  \[\val{\tau_1\ \vv{==}\ \tau_2}_\Gamma(\overline{u}) \df
    \begin{cases}
      true, & \text{ ако }\val{\tau_1}_\Gamma(\ov{u}) = \val{\tau_1}_\Gamma(\ov{u}) \in \Nat\\
      false, & \text{ ако }\val{\tau_1}_\Gamma(\ov{u}) \neq \val{\tau_1}_\Gamma(\ov{u}) \in \Nat\\
      \bot, & \text{ ако } \val{\tau_1}_\Gamma(\ov{u}) = \bot \text{ или } \val{\tau_1}_\Gamma(\ov{u}) = \bot.
    \end{cases}\]
\item
  % \marginpar{За $\texttt{if}$ вижте \Def{if}.}
  Нека $\tau \equiv \ifelse{\tau_1}{\tau_2}{\tau_3}$. Тогава
  \[\val{\ifelse{\tau_1}{\tau_2}{\tau_3}}_\Gamma(\overline{u}) \df
    \begin{cases}
      \val{\tau_2}_\Gamma(\ov{u}), & \text{ ако }\val{\tau_1}_\Gamma(\ov{u}) = true\\
      \val{\tau_3}_\Gamma(\ov{u}), & \text{ ако }\val{\tau_1}_\Gamma(\ov{u}) = false\\
      \bot, & \text{ ако } \val{\tau_1}_\Gamma(\ov{u}) = \bot.
    \end{cases}\]
% \item
%   \marginpar{За $\texttt{eval}$ вижте \Def{eval}.}
%   Нека $\tau \equiv \tau_1 \tau_2$. Тогава
%   \[\val{\tau_1 \tau_2}_\Gamma(\overline{u}) \df \texttt{eval}(\val{\tau_1}_\Gamma(\overline{u}), \val{\tau_2}_\Gamma(\overline{u})).\]
% \item
%   \marginpar{За $\lfp$ вижте Раздел~\ref{sect:lfp}.}
%   Нека $\tau \equiv \fix(\tau')$. Тогава 
%   \[\val{\fix(\tau')}_\Gamma(\overline{u}) \df \lfp(\val{\tau'}_\Gamma(\overline{u})).\]
% \item
%   \marginpar{За $\curry$ вижте \Def{curry}.}
%   Нека $\tau \equiv \lamb{y}{b}{\tau'}$, като $\vv{y} \not \in \texttt{dom}(\Gamma)$.
%   Нека $\Gamma' \df \Gamma, \type{y}{b}$. Тогава
%   \[\val{\lamb{y}{b}{\tau'}}_\Gamma(\overline{u}) \df \curry(\val{\tau'}_{\Gamma'})(\overline{u}).\]
\end{itemize}

% \begin{problem}
%   \marginpar{Аналог на \cite[Лема 4.19]{gunter}.}
%   Докажете, че ако $\Gamma \vdash \tau : \vv{a}$, то $\val{\tau}_\Gamma \in \Cont{\val{\Gamma}}{\val{\vv{a}}}$.
% \end{problem}

% \begin{problem}
%   Нека $\Gamma$ е типов контекст, $\tau$ и $\rho$ са термове, $\vv{x} \not\in \texttt{dom}(\Gamma)$,
%   \begin{align*}
%     & \Gamma \vdash \rho : \vv{a}\\
%     & \Gamma, \type{x}{a} \vdash \tau : \vv{b}.
%   \end{align*}
%   Докажете, че тогава:
%   \begin{enumerate}[1)]
%   \item
%     $\Gamma \vdash \tau\subst{x}{\rho} : \vv{b}$;
%   \item
%     за всяко $\overline{u} \in \val{\Gamma}$,
%     \[\val{\tau\subst{x}{\rho}}_\Gamma(\overline{u}) = \val{\tau}_{\Gamma'}(\overline{u},\val{\rho}_\Gamma(\overline{u})),\]
%     където $\Gamma' = \Gamma, \type{x}{a}$.  
%   \end{enumerate}
% \end{problem}

\begin{theorem}[Теорема за коректност за езика $\texttt{PCF(bool)}$]
  % \marginpar{Аналог на \cite[Твърдение 4.23]{gunter}.}
  Докажете, че за всеки затворен терм $\tau$ от тип $\vv{a}$ и стойност $\vv{v}$, е изпълнена импликацията:
  \[\tau \Downarrow_{\vv{a}} \vv{v}\ \implies\ \val{\tau} = \val{\vv{v}} \in \val{\vv{b}}.\]
\end{theorem}

\begin{theorem}[Теорема за адекватност за езика $\texttt{PCF(bool)}$]
  Нека разгледаме тип $\vv{a} = \nat$ или $\vv{a} = \bool$.
  За всеки затворен терм $\tau$ от тип $\vv{a}$ е изпълнена импликацията
  \[\val{\tau} = v \neq \bot^{\val{\vv{a}}} \implies \tau \Downarrow_{\vv{a}} \vv{v}.\]
\end{theorem}


%%% Local Variables:
%%% mode: latex
%%% TeX-master: "../sep"
%%% End:

\newpage
\section{Пълна абстракция}\label{pcf:sect:full-abstraction}

\begin{definition}
  \marginpar{\cite[стр. 179]{gunter}}
  Денотационната семантика $\val{.}$ се нарича {\bf напълно абстрактна}, ако
  контекстната (операционната) и денотационната наредба съвпадат, т.е.
  за произволни термове $\tau_1,\tau_2$ от тип $\vv{a}$ е изпълнено, че
  \[\val{\tau_1} \sqsubseteq \val{\tau_2} \iff \tau_1 \leq_{ctx} \tau_2 : \vv{a}.\]
\end{definition}

\begin{framed}
  \begin{theorem}[Гордън Плоткин 1977]
    Денотационната семантика $\val{.}$ за езика PCF {\bf не е} напълно абстрактна.
  \end{theorem}
\end{framed}
Да напомним, че винаги имаме следното:
\[ \val{\tau_1} = \val{\tau_2} \implies \tau_1 \cong_{ctx} \tau_2 : \vv{a}.\]
Това означава, че трябва да термове $\tau_1$ и $\tau_2$, за които
$\val{\tau_1} \neq \val{\tau_2}$ и $\tau_1 \cong \tau_2 : \vv{a}$.

Да дефинираме функцията $sor:\Nat_\bot \to (\Nat_\bot \to \Nat_\bot)$ по следния начин:
\marginpar{$sor$ идва от sequential or.}

\begin{tabular}{|c|c|c|c|}
  \hline
  $sor$ & $\bot$ & $0$ & $y>0$ \\
  \hline
  $\bot$ & $\bot$ & $\bot$ & $\bot$\\
  \hline
  $0$ & $\bot$ & $0$ & $1$\\
  \hline
  $x>0$ & $1$ & $1$ & $1$\\
  \hline
\end{tabular}

\begin{problem}
  Докажете, че $sor$ е определима в PCF.
\end{problem}
\begin{hint}
  Разгледайте затворения терм
  \[\tau \dff \lamb{x}{nat}{\lamb{y}{nat}{\ifelse{\vv{x}}{\vv{1}}{\ifelse{\vv{y}}{\vv{1}}{\vv{0}}}}}\]
  Докажете, че $\val{\tau} = sor$.
\end{hint}

Сега дефинираме функция $por:\Nat_\bot\to(\Nat_\bot \to \Nat_\bot)$ по следния начин:

\begin{tabular}{|c|c|c|c|}
  \hline
  $por$ & $\bot$ & $0$ & $y>0$\\
  \hline
  $\bot$ & $\bot$ & $\bot$ & $1$\\
  \hline
  $0$ & $\bot$ & $0$ & $1$\\
  \hline
  $x>0$ & $1$ & $1$ & $1$\\
  \hline
\end{tabular}
\marginpar{$por$ идва от parallel or.}

\begin{problem}
  Докажете, че $por$ е непрекъснато изображение.
\end{problem}
\begin{hint}
  Достатъчно е да се съобрази, че $por$ е монотонно изображение.
\end{hint}

\begin{framed}
  \begin{lemma}[Гордън Плоткин 1977]
    Изображението $por$ не е определимо в PCF, т.е. не съществува затворен терм $\rho$,
    за който $\val{\rho} = por$.
  \end{lemma}
\end{framed}

\begin{problem}\label{prob:pcf:full-abstraction:por}
  Да разгледаме $f \in \Cont{\Nat_\bot}{\Cont{\Nat_\bot}{\Nat_\bot}}$, за което имаме ограниченията:

  \begin{tabular}{|c|c|c|c|}
    \hline
    $f$ & $\bot$ & $0$ & $y>0$\\
    \hline
    $\bot$ & $?$ & $?$ & $1$\\
    \hline
    $0$ & $?$ & $0$ & $?$\\
    \hline
    $x>0$ & $1$ & $?$ & $?$\\
    \hline
  \end{tabular}

  Докажете, че $f = por$.
  
\end{problem}
\begin{hint}
  Използвайте монотонността на $f$.
\end{hint}

\begin{problem}\label{prob:pcf:full-abstraction:not-definble}
  Да разгледаме изображението $f \in \Cont{\Nat_\bot}{\Cont{\Nat_\bot}{\Nat_\bot}}$, за което
  \[f(0)(0) = 0\text{ и } f(1)(0) = f(0)(1) = 1.\]
  Докажете, че $f$ не е определима в PCF.
\end{problem}
\begin{hint}
  Да допуснем, че $f$ е определима в PCF.
  Тогава $f = \val{\tau}$, за някой затворен терм $\tau : \nat \to \nat \to \nat$.

  За произволна променлива $\vv{z}$, да положим
  \[\rho_{\vv{z}} \dff \ifelse{\vv{z == 0}}{\vv{0}}{\vv{1}}.\]
  Нека също положим
  \begin{align*}
    \tau' & \dff \tau\rho_{\vv{x}};\\
    \tau'' & \dff \tau'\rho_{\vv{y}}.
  \end{align*}
  Нека също така $\Gamma \dff \type{x}{nat}$, $\Delta = \type{y}{nat}$.
  Тогава е ясно, че $\val{\tau'}_\Gamma \in \Cont{\Nat_\bot}{\Cont{\Nat_\bot}{\Nat_\bot}}$ и
  \begin{align*} 
    \val{\tau'}_\Gamma  & = \val{\tau\rho_{\vv{x}}}_\Gamma\\
                       & = \texttt{eval}(\val{\tau}, \val{\rho_{\vv{x}}}_\Gamma)\\
                       & = \val{\tau}(\val{\rho_{\vv{x}}}_\Gamma)
  \end{align*}
  
  Нека $f' = \val{\tau'}_\Gamma$. Получаваме следното за $f'$.
  \[f'(u) = f(\val{\rho_{\vv{x}}}_\Gamma(u)) =
    \begin{cases}
      f(u), & \text{ако } u = \bot \text{ или } u = 0\\
      f(1), & \text{ако } u > 0.
    \end{cases}\]
  
  Аналогично, ясно е, че $\val{\tau''}_{\Delta,\Gamma} \in \Cont{\Nat_\bot\times\Nat_\bot}{\Nat_\bot}$ и
  \begin{align*} 
    \val{\tau''}_{\Gamma,\Delta}  & = \val{\tau'\rho_{\vv{y}}}_{\Gamma,\Delta}\\
                                  & = \texttt{eval}(\val{\tau'}_\Gamma, \val{\rho_{\vv{y}}}_\Delta)\\
                                  & = \val{\tau'}_\Gamma(\val{\rho_{\vv{y}}}_\Delta)\\
                                  & = \val{\tau}(\val{\rho_{\vv{x}}}_\Gamma)(\val{\rho_{\vv{y}}}_\Delta).
  \end{align*}
  Тогава
  \[\val{\tau''}_{\Gamma,\Delta}(u,v) =  \val{\tau}(\val{\rho_{\vv{x}}}_\Gamma(u))(\val{\rho_{\vv{y}}}_\Delta(v)).\]
  Нека сега $f'' = \val{\tau''}_{\Gamma,\Delta}$. Тогава
  \begin{align*}
    f''(u,v) & = f'(u)(\val{\rho_{\vv{y}}}_\Delta(v))\\
             & = \begin{cases}
               f'(u)(v), & \text{ако } v = \bot\text{ или } v = 0\\
               f'(u)(1), & \text{ако } v > 0
             \end{cases}.
  \end{align*}
  Така получаваме следната характеризация на $f''$:

  \begin{tabular}{|c|c|c|c|}
    \hline
    $f''$ & $\bot$ & $0$ & $y>0$ \\
    \hline
    $\bot$ & $?$ & $?$ & $1$ \\
    \hline
    $0$ & $?$ & $0$ & $?$ \\
    \hline
    $x>0$ & $1$ & $?$ & $?$\\
    \hline
  \end{tabular}

  Нека сега $\rho \dff \lamb{x}{nat}{\lamb{y}{nat}{\tau''}}$.
  Тогава за $g = \val{\rho}$ имаме, че
  \[g(x)(y) = f''(x,y).\]
  От \Prob{pcf:full-abstraction:por} получаваме, че $g = por$.
  Достигаме до противоречие, защото $por$ не е определимо изображение.
\end{hint}

В следващите твърдения ще използваме типовете
\begin{align*}
  & \vv{a} \dff \vv{nat} \to (\vv{nat} \to \vv{nat})\\
  & \vv{b} \dff (\vv{nat} \to (\vv{nat} \to \vv{nat}))\to\vv{nat}.
\end{align*}

За $n = 0,1$, нека дефинираме затворените термове
\begin{align*}
  \tau_n \dff \lamb{f}{a}{\ifelse{(\vv{f}\ \vv{1}\ \Omega_{\vv{nat}}) \vv{ == 1}}{\ifelse{(\vv{f}\ \Omega_{\vv{nat}}\ \vv{1}) \vv{ == 1}}{\ifelse{(\vv{f\ 0\ 0}) \vv{ == 1}}{\Omega_{\vv{nat}}}{\vv{n}}}{\Omega_{\vv{nat}}}}{\Omega_{\vv{nat}}}}.
\end{align*}
Лесно се съобразява, че $\tau_0$ и $\tau_1$ са добре типизирани термове от тип $\vv{b}$.

\begin{problem}
  Докажете, че 
  \[\val{\tau_0} \neq \val{\tau_1}.\]
\end{problem}
\begin{hint}
  Докажете, че за $n = 0,1$ е изпълнено, че
  \[\val{\tau_n}(por) = n.\]  
\end{hint}

\begin{proposition}
  $\tau_1 \cong_{ctx} \tau_2 : \vv{b}$.
\end{proposition}
\begin{proof}
  Понеже $\vv{b} = \vv{a} \to \vv{nat}$, от \Prop{pcf:context:extensionality} следва, че е достатъчно да докажем, че
  за всеки затворен терм $\rho:\vv{a}$ е изпълнено, че
  \[\tau_1\rho \Downarrow_{\vv{nat}} \vv{n} \iff \tau_2\rho \Downarrow_{\vv{nat}} \vv{n}.\]
  Да видим какво означава по принцип $\tau_i \rho \Downarrow_{\vv{nat}}$ за $i = 0,1$.
  Това означава, че трябва да са изпълнени и трите свойства:
  \begin{itemize}
  \item
    $\rho\ \vv{1}\ \Omega_{\vv{nat}} \Downarrow_{\vv{nat}} \vv{1}$;% , за някое $\vv{k} \not\equiv \vv{0}$;
  \item
    $\rho\ \Omega_{\vv{nat}}\ \vv{1} \Downarrow_{\vv{nat}} \vv{1}$;% , за някое $\vv{m} \not\equiv \vv{0}$;
  \item
    $\rho\ \vv{0}\ \vv{0} \Downarrow_{\vv{nat}} \vv{0}$.
  \end{itemize}
  Понеже $\val{\Omega_{\vv{nat}}} = \bot$, от \hyperref[th:pcf:soundness]{теоремата за коректност} получаваме, че са трябва да са изпълнени следните три свойства:
  \begin{itemize}
  \item
    $\val{\rho}(1)(\bot) = 1$;
  \item
    $\val{\rho}(\bot)(1) = 1$;
  \item
    $\val{\rho}(0)(0) = 0$.
  \end{itemize}
  Но тогава $\val{\rho} = por$, което е противоречие с \Prop{pcf:full-abstraction:not-definble}.
\end{proof}

Доказателството на следващата теорема излиза извън обхата на този курс.
\begin{framed}
  \begin{theorem}[Плоткин 1977]
    Денотационната семантика $\val{.}$ за езика PCF+\texttt{por} е напълно абстрактна.
  \end{theorem}
\end{framed}
\marginpar{\cite[стр. 188]{gunter}}

\begin{theorem}[Милнър,Плоткин]
  A continuous, order-extensional model of PCF is fully abstract if and only if for every type $\sigma$, $\val{\sigma}$ is a domain whose finite elements are definable.
\end{theorem}

%%% Local Variables:
%%% mode: latex
%%% TeX-master: "../sep"
%%% End:



%%% Local Variables:
%%% mode: latex
%%% TeX-master: "../sep"
%%% End:
