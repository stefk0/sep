\section{Операционна семантика}\label{pcf:sect:operational-cbn}
\index{операционна семантика}

За затворен терм $\tau : \vv{a}$ и стойност $\vv{v} : \vv{a}$, дефинираме релацията $\tau \opsem{\ell}{a} \vv{v}$ по
следния начин.
\marginpar{Да напомним, че с $\vv{v}$ означаваме стойности, които биват или константи или затворени термове от вида $\lamb{x}{a}{\mu}$.

  Обърнете внимание, че заради случите (fix) и (cbn) не можем да доказваме свойства на операционната семантика с индукция по построението на термовете. Можем да направим това с индукция по броя на стъпки в изчислението.}

\begin{figure}[H]
  \centering
  
\begin{prooftree}
  \AxiomC{$\type{v}{a}$}
  \RightLabel{\scriptsize{(val)}}
  \UnaryInfC{$\vv{v} \opsem{0}{a} \vv{v}$}
\end{prooftree}

\begin{prooftree}
  \AxiomC{$\tau_1 \opsem{\ell_1}{nat} \vv{n}_1$}
  \AxiomC{$\tau_2 \opsem{\ell_2}{nat} \vv{n}_2$}
  \AxiomC{$n = \eq(n_1,n_2)$}
  \RightLabel{\scriptsize{(eq)}}
  \TrinaryInfC{$\tau_1\ \vv{==}\ \tau_2 \opsem{\ell_1+\ell_2+1}{nat} \vv{n}$}
\end{prooftree}

\begin{prooftree}
  \AxiomC{$\tau_1 \opsem{\ell_1}{nat} \vv{n}_1$}
  \AxiomC{$\tau_2 \opsem{\ell_2}{nat} \vv{n}_2$}
  \AxiomC{$n = \plus(n_1,n_2)$}
  \RightLabel{\scriptsize{(plus)}}
  \TrinaryInfC{$\tau_1\ \vv{+}\ \tau_2 \opsem{\ell_1+\ell_2+1}{nat} \vv{n}$}
\end{prooftree}

\begin{prooftree}
  \AxiomC{$\tau_1 \opsem{\ell_1}{nat} \vv{n}_1$}
  \AxiomC{$\tau_2 \opsem{\ell_2}{nat} \vv{n}_2$}
  \AxiomC{$n = \minus(n_1,n_2)$}
  \RightLabel{\scriptsize{(minus)}}
  \TrinaryInfC{$\tau_1\ \vv{-}\ \tau_2 \opsem{\ell_1+\ell_2+1}{nat} \vv{n}$}
\end{prooftree}

\begin{prooftree}
  \AxiomC{$\tau_1 \opsem{\ell_1}{nat} \vv{0}$}
  \AxiomC{$\tau_3 \opsem{\ell_2}{a} \vv{v}$}
  \RightLabel{\scriptsize{(if$_0$)}}
  \BinaryInfC{$\ifelse{\tau_1}{\tau_2}{\tau_3} \opsem{\ell_1+\ell_2+1}{a} \vv{v}$}
\end{prooftree}

\begin{prooftree}
  \AxiomC{$\tau_1 \opsem{\ell_1}{nat} \vv{n}$}
  \AxiomC{$\tau_2 \opsem{\ell_2}{a} \vv{v}$}
  \AxiomC{$\vv{n} \not\equiv \vv{0}$}
  \RightLabel{\scriptsize{(if$^+$)}}
  \TrinaryInfC{$\ifelse{\tau_1}{\tau_2}{\tau_3} \opsem{\ell_1+\ell_2+1}{a} \vv{v}$}
\end{prooftree}

\begin{prooftree}
  \AxiomC{$\tau_1 \opsemGen{\ell_1}{\vv{a}\to\vv{b}} \lamb{x}{a}{\tau'_1}$}
  \AxiomC{$\tau'_1\subst{x}{\tau_2} \opsem{\ell_2}{b} \vv{v}$}
  \RightLabel{\scriptsize{(cbn)}}
  \BinaryInfC{$\tau_1 \tau_2 \opsem{\ell_1+\ell_2+1}{b} \vv{v} $}
\end{prooftree}

\begin{prooftree}
  \AxiomC{$\tau\ \fix(\tau) \opsem{\ell}{a} \vv{v}$}
  \RightLabel{\scriptsize{(fix)}}
  \UnaryInfC{$\fix(\tau) \opsem{\ell+1}{a} \vv{v} $}
\end{prooftree}
\caption{Правила на операционната семантика за езика \PCF}
\end{figure}



\begin{itemize}
\item 
  Ще пишем $\tau \opsem{}{a} \vv{v}$, ако съществува $\ell$, за което $\tau \opsem{\ell}{a} \vv{v}$.  
\item
  Също така, ще пишем $\tau \not\opsem{}{a}$, ако не съществува стойност $\vv{v}$, за която $\tau \opsem{}{a} \vv{v}$.  
\end{itemize}

\begin{lemma}
  За произволен затворен терм $\tau$ и стойности $\vv{v}$ и $\vv{u}$,
  \[\tau \opsem{}{a} \vv{v}\ \&\ \tau \opsem{}{a} \vv{u}\ \implies\ \vv{v} \equiv \vv{u}.\]
\end{lemma}







%%% Local Variables:
%%% mode: latex
%%% TeX-master: "../sep"
%%% End:
