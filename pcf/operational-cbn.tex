\section{Операционна семантика с предаване на параметрите по име}\label{pcf:sect:operational-cbn}

Дефинираме релацията $\Downarrow^\ell_{\vv{a}}$ върху затворените изрази и стойностите.
\marginpar{Да напомним, че с $\vv{v}$ означаваме стойности, които биват или константи или затворени термове от вида $\lamb{x}{a}{\mu}$.}
\begin{prooftree}
  \AxiomC{$\type{v}{a}$}
  \RightLabel{\scriptsize{(val)}}
  \UnaryInfC{$\vv{v} \Downarrow^0_{\vv{a}} \vv{v}$}
\end{prooftree}

\begin{prooftree}
  \AxiomC{$\tau_1 \Downarrow^{\ell_1}_{\vv{nat}} \vv{n}_1$}
  \AxiomC{$\tau_2 \Downarrow^{\ell_2}_{\vv{nat}} \vv{n}_2$}
  \AxiomC{$n = \texttt{eq}(n_1,n_2)$}
  \RightLabel{\scriptsize{(eq)}}
  \TrinaryInfC{$\tau_1\ \vv{==}\ \tau_2 \Downarrow^{\ell_1+\ell_2+1}_{\vv{nat}} \vv{n}$}
\end{prooftree}

\begin{prooftree}
  \AxiomC{$\tau_1 \Downarrow^{\ell_1}_{\vv{nat}} \vv{n}_1$}
  \AxiomC{$\tau_2 \Downarrow^{\ell_2}_{\vv{nat}} \vv{n}_2$}
  \AxiomC{$n = n_1 + n_2$}
  \RightLabel{\scriptsize{(plus)}}
  \TrinaryInfC{$\tau_1\ \vv{+}\ \tau_2 \Downarrow^{\ell_1+\ell_2+1}_{\vv{nat}} \vv{n}$}
\end{prooftree}


\begin{prooftree}
  \AxiomC{$\tau_1 \Downarrow^{\ell_1}_{\vv{nat}} \vv{0}$}
  \AxiomC{$\tau_3 \Downarrow^{\ell_2}_{\vv{a}} \vv{v}$}
  \RightLabel{\scriptsize{(if$_0$)}}
  \BinaryInfC{$\ifelse{\tau_1}{\tau_2}{\tau_3} \Downarrow^{\ell_1+\ell_2+1}_{\vv{a}} \vv{v}$}
\end{prooftree}

\begin{prooftree}
  \AxiomC{$\tau_1 \Downarrow^{\ell_1}_{\vv{nat}} \vv{n}$}
  \AxiomC{$\tau_2 \Downarrow^{\ell_2}_{\vv{a}} \vv{v}$}
  \AxiomC{$\vv{n} \not\equiv \vv{0}$}
  \RightLabel{\scriptsize{(if$^+$)}}
  \TrinaryInfC{$\ifelse{\tau_1}{\tau_2}{\tau_3} \Downarrow^{\ell_1+\ell_2+1}_{\vv{a}} \vv{v}$}
\end{prooftree}

\begin{prooftree}
  \AxiomC{$\tau_1 \Downarrow^{\ell_1}_{\vv{a}\to\vv{b}} \lamb{x}{a}{\tau'_1}$}
  \AxiomC{$\tau'_1[x/\tau_2] \Downarrow^{\ell_2}_{\vv{b}} \vv{v}$}
  \RightLabel{\scriptsize{(cbn)}}
  \BinaryInfC{$\tau_1 \tau_2 \Downarrow^{\ell_1+\ell_2+1}_{\vv{b}} \vv{v} $}
\end{prooftree}

\begin{prooftree}
  \AxiomC{$\tau\ \fix(\tau) \Downarrow^{\ell}_{\vv{a}} \vv{v}$}
  \RightLabel{\scriptsize{(fix)}}
  \UnaryInfC{$\fix(\tau) \Downarrow^{\ell+1}_{\vv{a}} \vv{v} $}
\end{prooftree}

\begin{itemize}
\item 
  Ще пишем $\tau \Downarrow_{\vv{a}} \vv{v}$, ако съществува $\ell$, за което $\tau \Downarrow^\ell_{\vv{a}} \vv{v}$.  
\item
  Също така, ще пишем $\tau \not\Downarrow_{\vv{a}}$, ако не съществува стойност $\vv{v}$, за която $\tau \Downarrow_{\vv{a}} \vv{v}$.  
\end{itemize}

\begin{lemma}
  За произволен затворен терм $\tau$ и стойности $\vv{v}$ и $\vv{u}$,
  \[\tau \Downarrow_{\vv{a}} \vv{v}\ \&\ \tau \Downarrow_{\vv{a}} \vv{u}\ \implies\ \vv{v} \equiv \vv{u}.\]
\end{lemma}


\begin{example}
  Нека $\vv{a} = \vv{nat} \to (\vv{nat} \to \vv{nat})$ и 
  \[\tau \equiv \fix\vv{(}\lamb{f}{a}{\lamb{x}{nat}{\lamb{y}{nat}{\ifelse{\vv{y == 0}}{\vv{0}}{\vv{x + (f x (y-1))} }}}}\vv{)}.\]
  Лесно се вижда, че
  \[\tau:\vv{nat} \to (\vv{nat} \to \vv{nat}).\]
  Също така за всяко $n$ и $k$, ако $m = n + k$, то
  \[\tau\ \vv{n}\ \vv{k} \Downarrow_{\vv{nat}} \vv{m}.\]
  Нека сега
  \[\rho \equiv \lamb{g}{a}{\fix(\lambda \vv{f} : \vv{nat}\to\vv{nat}\ .\ \lamb{x}{nat}{\ifelse{\vv{x == 0}}{\vv{1}}{\vv{g x f(x-1)}}}}).\]
  Лесно се вижда, че
  \[ \rho : \vv{a} \to (\vv{nat} \to \vv{nat}).\]
  Също така, за всяко $n$, ако $k = n!$, то
  \[ \rho\ \tau\ \vv{n} \Downarrow_{\vv{nat}} \vv{k}.\]
\end{example}


Можем да напишем директно горния пример и на хаскел:
\begin{haskellcode}
> fix f = f (fix f)
> times = fix(\f x y -> if y == 0 then 0 else x + (f x (y-1)))
> times 2 3
6
> fct = \g -> fix(\f x -> if x == 0 then 1 else g x (f (x-1)))
> fact = fct times
> fact 5
120
\end{haskellcode}

%%% Local Variables:
%%% mode: latex
%%% TeX-master: "../sep"
%%% End:
