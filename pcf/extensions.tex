\section{Езикът \texttt{PCF(bool)}}

\marginpar{Всъщност в \cite[Глава 4.1]{gunter} това е ,,истинската'' дефиниция на езика \PCF.}

\begin{itemize}
\item
  Типове
  \[\vv{a} ::= \bool\ |\ \nat\ |\ \vv{a}\to\vv{a}\]
\item
  Изрази
  \begin{align*}
    \tau ::=\ & \tru\ |\ \fls\ |\ \vv{n}\ |\ \vv{x}\ |\ \tau_1 + \tau_2\ |\ \tau_1 - \tau_2\ |\  \tau_1\ \vv{==}\ \tau_2\ |\\
              & \ifelse{\tau_1}{\tau_2}{\tau_3}\ |\ \tau_1\tau_2\ |\ \lamb{x}{a}{\tau_1}\ |\ \fix(\tau_1).
  \end{align*}
  % Термове
  % \begin{align*}
  %   \tau ::=\ & \vv{0}\ |\ \tru\ |\ \fls\ |\ \vv{x}\ |\\
  %            & \scc(\tau_1)\ |\ \prd(\tau_1)\ |\ \texttt{iszero}(\tau)\ |\ \ifelse{\tau_1}{\tau_2}{\tau_3}\ |\\
  %            &\tau_1\tau_2\ |\ \lamb{x}{a}{\tau_1}\ |\ \fix(\tau_1).
  % \end{align*}
\item
  Стойностите са затворени термове от следния вид:
  \[\vv{v} ::= \tru\ |\ \fls\ |\ \vv{n}\ |\ \lamb{x}{a}{\mu}\]
\end{itemize}

\subsection{Типизираща релация}

Релацията $\Gamma \vdash \tau : \vv{a}$ за езика \PCFBOOL е почти същата
както за езика \PCF. Имаме две нови правила:

\begin{figure}[H]
  \begin{subfigure}[b]{0.5\textwidth}
    \begin{prooftree}
      \AxiomC{}
      \RightLabel{\scriptsize{(true)}}
      \UnaryInfC{$\Gamma \vdash \tru : \bool$}
    \end{prooftree}
  \end{subfigure}
  ~
  \begin{subfigure}[b]{0.5\textwidth}
    \begin{prooftree}
      \AxiomC{}
      \RightLabel{\scriptsize{(false)}}
      \UnaryInfC{$\Gamma \vdash \fls : \bool$}
    \end{prooftree}
  \end{subfigure}
\end{figure}
Имаме и две променени правила:
\begin{prooftree}
  \AxiomC{$\Gamma \vdash \tau_1:\nat$}
  \AxiomC{$\Gamma \vdash \tau_2:\nat$}
  \RightLabel{\scriptsize{(eq)}}
  \BinaryInfC{$\Gamma \vdash \tau_1\ \vv{==}\ \tau_2 : \bool$}
\end{prooftree}
\begin{prooftree}
  \AxiomC{$\Gamma \vdash \tau_1:\bool$}
  \AxiomC{$\Gamma \vdash \tau_2:\vv{a}$}
  \AxiomC{$\Gamma \vdash \tau_3:\vv{a}$}
  \RightLabel{\scriptsize{(if)}}
  \TrinaryInfC{$\Gamma \vdash \ifelse{\tau_1}{\tau_2}{\tau_3} : \vv{a}$}
\end{prooftree}
Всички останали правила са същите.

% \end{subfigure}
% ~
% \begin{subfigure}[b]{0.5\textwidth}
% \begin{prooftree}
%   \AxiomC{$\Gamma \vdash \tau:\vv{a}\to\vv{a}$}
%   \RightLabel{\scriptsize{(fix)}}
%   \UnaryInfC{$\Gamma \vdash \fix(\tau) : \vv{a}$}
% \end{prooftree}
% \end{subfigure}

% \vspace{10pt}

% \begin{subfigure}[b]{0.5\textwidth}
% \begin{prooftree}
%   \AxiomC{$\Gamma \vdash \tau_1:\vv{a}\to\vv{b}$}
%   \AxiomC{$\Gamma \vdash \tau_2:\vv{a}$}
%   \RightLabel{\scriptsize{(app)}}
%   \BinaryInfC{$\Gamma \vdash \tau_1\tau_2 : \vv{b}$}
% \end{prooftree}
% \end{subfigure}
% ~
% \begin{subfigure}[b]{0.5\textwidth}
% \begin{prooftree}
%   \AxiomC{$\vv{x} \not\in\vv{dom}(\Gamma)$}
%   \AxiomC{$\Gamma, \type{x}{a} \vdash \tau:\vv{b}$}
%   \RightLabel{\scriptsize{(lambda)}}
%   \BinaryInfC{$\Gamma \vdash \lambda \type{x}{a}\ .\ \tau : \vv{a} \to \vv{b}$}
% \end{prooftree}
% \end{subfigure}

% \caption{Релация за типизиране на термовете от езика \texttt{PCF++}}
% \label{fig:pcf:extensions:relation}
% \end{figure}


\subsection{Операционна семантика}

% \marginpar{\cite[стр. 109]{gunter}}

Тук отново всичко е почти същото както преди със следните разлики:

\begin{figure}[H]
  % \begin{subfigure}[b]{0.5\textwidth}
  %   \begin{prooftree}
  %     \AxiomC{$\type{v}{a}$}
  %     \RightLabel{\scriptsize{(val)}}
  %     \UnaryInfC{$\vv{v} \Downarrow^0_{\vv{a}} \vv{v}$}
  %   \end{prooftree}    
  % \end{subfigure}
  % ~
  % \begin{subfigure}[b]{0.5\textwidth}
  %   \begin{prooftree}
  %     \AxiomC{$\tau \Downarrow^\ell_{\nat} \vv{0}$}
  %     \UnaryInfC{$\prd(\tau) \Downarrow^{\ell+1}_{\nat} \vv{0}$}
  %   \end{prooftree}
  % \end{subfigure}

  % \vspace{10pt}

  % \begin{subfigure}[b]{0.5\textwidth}
  %   \begin{prooftree}
  %     \AxiomC{$\tau \Downarrow^\ell_{\nat} \scc(\vv{v})$}
  %     \UnaryInfC{$\prd(\tau) \Downarrow^{\ell+1}_{\nat} \vv{v}$}
  %   \end{prooftree}
  % \end{subfigure}
  % ~
  % \begin{subfigure}[b]{0.5\textwidth}
  %   \begin{prooftree}
  %     \AxiomC{$\tau \Downarrow^\ell_{\nat} \vv{v}$}
  %     \UnaryInfC{$\scc(\tau) \Downarrow^{\ell+1}_{\nat} \scc(\vv{v})$}
  %   \end{prooftree}
  % \end{subfigure}

  % \vspace{10pt}

  % \begin{subfigure}[b]{0.5\textwidth}
  %   \begin{prooftree}
  %     \AxiomC{$\tau \Downarrow^\ell_{\nat} \vv{0}$}
  %     \UnaryInfC{$\iszero(\tau) \Downarrow^{\ell+1}_{\bool} \tru$}
  %   \end{prooftree}
  % \end{subfigure}
  % ~
  % \begin{subfigure}[b]{0.5\textwidth}
  %   \begin{prooftree}
  %     \AxiomC{$\tau \Downarrow^\ell_{\nat} \scc(\vv{v})$}
  %     \UnaryInfC{$\iszero(\tau) \Downarrow^{\ell+1}_{\bool} \fls$}
  %   \end{prooftree}
  % \end{subfigure}

  % \vspace{10pt}
  
  \begin{subfigure}[b]{0.5\textwidth}
    \begin{prooftree}
      \AxiomC{$\tau_1 \opsem{\ell_1}{nat} \vv{v}_1$}
      \AxiomC{$\tau_3 \opsem{\ell_2}{a} \vv{v}_2$}
      \AxiomC{$\vv{v}_1 \equiv \vv{v}_2$}
      % \RightLabel{\scriptsize{(if$_\fls$)}}
      \TrinaryInfC{$\tau_1\ \vv{==}\ \tau_2 \opsem{\ell_1+\ell_2+1}{bool} \tru$}
    \end{prooftree}
  \end{subfigure}
  ~
  \begin{subfigure}[b]{0.5\textwidth}
    \begin{prooftree}
      \AxiomC{$\tau_1 \opsem{\ell_1}{nat} \vv{v}_1$}
      \AxiomC{$\tau_3 \opsem{\ell_2}{a} \vv{v}_2$}
      \AxiomC{$\vv{v}_1 \not\equiv \vv{v}_2$}
      % \RightLabel{\scriptsize{(if$_\fls$)}}
      \TrinaryInfC{$\tau_1\ \vv{==}\ \tau_2 \opsem{\ell_1+\ell_2+1}{bool} \fls$}
    \end{prooftree}
  \end{subfigure}

  \vspace{10pt}
  
  \begin{subfigure}[b]{0.5\textwidth}
    \begin{prooftree}
      \AxiomC{$\tau_1 \opsem{\ell_1}{bool} \fls$}
      \AxiomC{$\tau_3 \opsem{\ell_2}{a} \vv{v}$}
      % \RightLabel{\scriptsize{(if$_\fls$)}}
      \BinaryInfC{$\ifelse{\tau_1}{\tau_2}{\tau_3} \opsem{\ell_1+\ell_2+1}{a} \vv{v}$}
    \end{prooftree}
  \end{subfigure}
  ~
  \begin{subfigure}[b]{0.5\textwidth}
    \begin{prooftree}
      \AxiomC{$\tau_1 \opsem{\ell_1}{bool} \tru$}
      \AxiomC{$\tau_2 \opsem{\ell_2}{a} \vv{v}$}
      % \RightLabel{\scriptsize{(if$_\tru$)}}
      \BinaryInfC{$\ifelse{\tau_1}{\tau_2}{\tau_3} \opsem{\ell_1+\ell_2+1}{a} \vv{v}$}
    \end{prooftree}
  \end{subfigure}

%   \vspace{10pt}

%   \begin{subfigure}[b]{0.5\textwidth}
%     \begin{prooftree}
%       \AxiomC{$\tau_1 \Downarrow^{\ell_1}_{\vv{a}\to\vv{b}} \lamb{x}{a}{\tau'_1}$}
%       \AxiomC{$\tau'_1[x/\tau_2] \Downarrow^{\ell_2}_{\vv{b}} \vv{v}$}
%       % \RightLabel{\scriptsize{(cbn)}}
%       \BinaryInfC{$\tau_1 \tau_2 \Downarrow^{\ell_1+\ell_2+1}_{\vv{b}} \vv{v} $}
%     \end{prooftree}
%   \end{subfigure}
%   ~
%   \begin{subfigure}[b]{0.5\textwidth}
%   \begin{prooftree}
%     \AxiomC{$\tau\ \fix(\tau) \Downarrow^{\ell}_{\vv{a}} \vv{v}$}
%     \RightLabel{\scriptsize{(fix)}}
%     \UnaryInfC{$\fix(\tau) \Downarrow^{\ell+1}_{\vv{a}} \vv{v} $}
%   \end{prooftree}
% \end{subfigure}
% \caption{Правила на операционната семантика за езика \PCFPP}
\end{figure}




% \begin{lemma}
%   Нека $\tau$ е затворен терм от тип $\vv{a}$.
%   Тогава ако $\tau \Downarrow_{\vv{a}} \vv{v}$ и $\tau \Downarrow_{\vv{a}} \vv{u}$, то
%   $\vv{v} \equiv_\alpha \vv{u}$.
% \end{lemma}


\subsection{Денотационна семантика}

\marginpar{В \cite[Глава 4.3]{gunter} се нарича \emph{standard fixed-point semantics} of PCF.}

Семантиката на всеки тип ще бъде област на Скот както следва:
% \begin{align*}
\[\val{\bool} \df \Bool = \{true, false\}_\bot.\]
%   & \val{\nat} \df \Nat_\bot\\
%   & \val{\vv{a} \to \vv{b}} \df \Cont{\val{\vv{a}}}{\val{\vv{b}}}.
% \end{align*}

\begin{itemize}
% \item
%   Нека $\tau \equiv \vv{0}$. Тогава
%   \[\val{\vv{0}}_\Gamma(\overline{u}) \df 0.\]
\item
  Нека $\tau \equiv \tru$. Тогава
  \[\val{\tru}_\Gamma(\overline{u}) \df true.\]
\item
  Нека $\tau \equiv \fls$. Тогава
  \[\val{\fls}_\Gamma(\overline{u}) \df false.\]
% \item
%   Нека $\tau \equiv \vv{x}_i$. Тогава
%   \[\val{\vv{x}_i}_\Gamma(\overline{u}) \df u_i.\]
% \item
%   Нека $\tau \equiv \scc(\tau_1)$. Тогава
%   \[\val{\scc(\tau_1)}_\Gamma(\ov{u}) \df
%   \begin{cases}
%     \val{\tau_1}_\Gamma(\ov{u}) + 1, & \text{ ако }\val{\tau_1}_\Gamma(\ov{u}) \neq \bot\\
%     \bot, & \text{ ако }\val{\tau_1}_\Gamma(\ov{u}) = \bot.
%   \end{cases}\]

% \item
%   Нека $\tau \equiv \prd(\tau_1)$. Тогава
%   \[\val{\prd(\tau_1)}_\Gamma(\ov{u}) \df
%   \begin{cases}
%     0, & \text{ ако }\val{\tau_1}(\ov{u}) = 0\\
%     \val{\tau_1}_\Gamma(\ov{u}) - 1, & \text{ ако }\val{\tau_1}_\Gamma(\ov{u}) \neq 0, \bot\\
%     \bot, & \text{ ако }\val{\tau_1}_\Gamma(\ov{u}) = \bot.
%   \end{cases}\]

% \item
%   Нека $\tau \equiv \iszero(\tau_1)$. Тогава
%   \[\val{\iszero(\tau_1)}_\Gamma(\ov{u}) \df
%   \begin{cases}
%     true, &  \text{ ако }\val{\tau_1}_\Gamma(\ov{u}) = 0\\
%     false, & \text{ ако }\val{\tau_1}_\Gamma(\ov{u}) \neq 0,\bot\\
%     \bot, &  \text{ ако }\val{\tau_1}_\Gamma(\ov{u}) = \bot.
%   \end{cases}\]


% \item
%   \marginpar{За $\texttt{eq}$ вижте Раздел~\ref{subsect:rec:term-value}.}
%   Нека $\tau \equiv \tau_1\ \vv{==}\ \tau_2$. Тогава
%   \[\val{\tau_1\ \vv{==}\ \tau_2}_\Gamma(\overline{u}) \df \texttt{eq}(\val{\tau_1}_\Gamma(\overline{u}), \val{\tau_2}_\Gamma(\overline{u})).\]

\item
  Нека $\tau \equiv \tau_1\ \vv{==}\ \tau_2$. Тогава
  \[\val{\tau_1\ \vv{==}\ \tau_2}_\Gamma(\overline{u}) \df
    \begin{cases}
      true, & \text{ ако }\val{\tau_1}_\Gamma(\ov{u}) = \val{\tau_1}_\Gamma(\ov{u}) \in \Nat\\
      false, & \text{ ако }\val{\tau_1}_\Gamma(\ov{u}) \neq \val{\tau_1}_\Gamma(\ov{u}) \in \Nat\\
      \bot, & \text{ ако } \val{\tau_1}_\Gamma(\ov{u}) = \bot \text{ или } \val{\tau_1}_\Gamma(\ov{u}) = \bot.
    \end{cases}\]
\item
  % \marginpar{За $\texttt{if}$ вижте \Def{if}.}
  Нека $\tau \equiv \ifelse{\tau_1}{\tau_2}{\tau_3}$. Тогава
  \[\val{\ifelse{\tau_1}{\tau_2}{\tau_3}}_\Gamma(\overline{u}) \df
    \begin{cases}
      \val{\tau_2}_\Gamma(\ov{u}), & \text{ ако }\val{\tau_1}_\Gamma(\ov{u}) = true\\
      \val{\tau_3}_\Gamma(\ov{u}), & \text{ ако }\val{\tau_1}_\Gamma(\ov{u}) = false\\
      \bot, & \text{ ако } \val{\tau_1}_\Gamma(\ov{u}) = \bot.
    \end{cases}\]
% \item
%   \marginpar{За $\texttt{eval}$ вижте \Def{eval}.}
%   Нека $\tau \equiv \tau_1 \tau_2$. Тогава
%   \[\val{\tau_1 \tau_2}_\Gamma(\overline{u}) \df \texttt{eval}(\val{\tau_1}_\Gamma(\overline{u}), \val{\tau_2}_\Gamma(\overline{u})).\]
% \item
%   \marginpar{За $\lfp$ вижте Раздел~\ref{sect:lfp}.}
%   Нека $\tau \equiv \fix(\tau')$. Тогава 
%   \[\val{\fix(\tau')}_\Gamma(\overline{u}) \df \lfp(\val{\tau'}_\Gamma(\overline{u})).\]
% \item
%   \marginpar{За $\curry$ вижте \Def{curry}.}
%   Нека $\tau \equiv \lamb{y}{b}{\tau'}$, като $\vv{y} \not \in \texttt{dom}(\Gamma)$.
%   Нека $\Gamma' \df \Gamma, \type{y}{b}$. Тогава
%   \[\val{\lamb{y}{b}{\tau'}}_\Gamma(\overline{u}) \df \curry(\val{\tau'}_{\Gamma'})(\overline{u}).\]
\end{itemize}

% \begin{problem}
%   \marginpar{Аналог на \cite[Лема 4.19]{gunter}.}
%   Докажете, че ако $\Gamma \vdash \tau : \vv{a}$, то $\val{\tau}_\Gamma \in \Cont{\val{\Gamma}}{\val{\vv{a}}}$.
% \end{problem}

% \begin{problem}
%   Нека $\Gamma$ е типов контекст, $\tau$ и $\rho$ са термове, $\vv{x} \not\in \texttt{dom}(\Gamma)$,
%   \begin{align*}
%     & \Gamma \vdash \rho : \vv{a}\\
%     & \Gamma, \type{x}{a} \vdash \tau : \vv{b}.
%   \end{align*}
%   Докажете, че тогава:
%   \begin{enumerate}[1)]
%   \item
%     $\Gamma \vdash \tau\subst{x}{\rho} : \vv{b}$;
%   \item
%     за всяко $\overline{u} \in \val{\Gamma}$,
%     \[\val{\tau\subst{x}{\rho}}_\Gamma(\overline{u}) = \val{\tau}_{\Gamma'}(\overline{u},\val{\rho}_\Gamma(\overline{u})),\]
%     където $\Gamma' = \Gamma, \type{x}{a}$.  
%   \end{enumerate}
% \end{problem}

\begin{theorem}[Теорема за коректност за езика $\texttt{PCF(bool)}$]
  % \marginpar{Аналог на \cite[Твърдение 4.23]{gunter}.}
  Докажете, че за всеки затворен терм $\tau$ от тип $\vv{a}$ и стойност $\vv{v}$, е изпълнена импликацията:
  \[\tau \Downarrow_{\vv{a}} \vv{v}\ \implies\ \val{\tau} = \val{\vv{v}} \in \val{\vv{b}}.\]
\end{theorem}

\begin{theorem}[Теорема за адекватност за езика $\texttt{PCF(bool)}$]
  Нека разгледаме тип $\vv{a} = \nat$ или $\vv{a} = \bool$.
  За всеки затворен терм $\tau$ от тип $\vv{a}$ е изпълнена импликацията
  \[\val{\tau} = v \neq \bot^{\val{\vv{a}}} \implies \tau \Downarrow_{\vv{a}} \vv{v}.\]
\end{theorem}


%%% Local Variables:
%%% mode: latex
%%% TeX-master: "../sep"
%%% End:
