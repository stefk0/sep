\section{Денотационна семантика}

Започваме със семантика на типовете.

\begin{align*}
  & \val{\vv{nat}} \dff \Nat_\bot\\
  & \val{\vv{a} \to \vv{b}} \dff \Cont{\val{\vv{a}}}{\val{\vv{b}}}.
\end{align*}
За един типов контекст $\Gamma = \vv{x}_1\ :\ \vv{a}_1,\dots,\vv{x}_n\ :\ \vv{a}_n$, дефинираме
\[\val{\Gamma} = \val{\vv{a}_1}\times\cdots\times\val{\vv{a}_n}.\]
Сега трябва да дефинираме семантика на термовете.
Искаме да дефинираме за всеки терм, за който $\Gamma \vdash \tau : \vv{a}$,
изборажение $\val{\tau}_\Gamma$, така че
\[\val{\tau}_\Gamma : \Cont{\val{\Gamma}}{\val{\vv{a}}}.\]

\begin{itemize}
\item
  $\val{\vv{n}}_\Gamma(\overline{u}) = n$;
\item
  $\val{\vv{x}_i}_\Gamma(\overline{u}) = u_i$;
\item
  $\val{\tau_1 + \tau_2}_\Gamma(\overline{u}) = \texttt{plus}(\val{\tau_1}(\overline{u}), \val{\tau_2}(\overline{u}))$;
\item
  $\val{\Gamma \vdash \tau_1\ \vv{==}\ \tau_2}\rho = \texttt{eq}(\val{\tau_1}\rho, \val{\tau_2}\rho)$;
\item
  $\val{\Gamma \vdash \ifelse{\tau_1}{\tau_2}{\tau_3}}\rho = \texttt{if}(\val{\tau_1}\rho, \val{\tau_2}\rho, \val{\tau_3}\rho)$;
\item
  Щом $\Gamma \vdash \tau_1 \tau_2 : \vv{a}$, то от правилата за типизиране следва, че
  \begin{align*}
    & \Gamma \vdash \tau_1 : \vv{b} \to \vv{a}\\
    & \Gamma \vdash \tau_2 : \vv{b}.
  \end{align*}
  От И.П. за $\tau_1$ и $\tau_2$ знаем, че
  \begin{align*}
    & \val{\tau_1}_\Gamma \in \Cont{\val{\Gamma}}{\Cont{\val{\vv{b}}}{\val{\vv{a}}}} \\
    & \val{\tau_2}_\Gamma \in \Cont{\val{\Gamma}}{\val{\vv{b}}}
  \end{align*}
  Оттук получаваме, че за произволни $\overline{u} \in \val{\Gamma}$,
  \begin{align*}
    & \val{\tau_1}_\Gamma \in \Cont{\val{\vv{b}}}{\val{\vv{a}}} \\
    & \val{\tau_2}_\Gamma \in \val{\vv{b}}.
  \end{align*}
  Сега дефинираме
  \[\val{\tau_1 \tau_2}_\Gamma(\overline{u}) \dff \texttt{eval}(\val{\tau_1}_\Gamma(\overline{u}), \val{\tau_2}_\Gamma(\overline{u})).\]
\item
  Нека $\tau \equiv \lamb{y}{b}{\tau'}$, като $\vv{y} \not \in \texttt{dom}(\Gamma)$.
  Щом $\Gamma \vdash \lamb{y}{b}{\tau'} : \vv{a}$, то от правилата за типизиране следва, че $\vv{a} = \vv{b} \to \vv{c}$
  и 
  \[\Gamma, \type{y}{b} \vdash \tau' : \vv{c}.\]

  Нека $\Gamma' = \Gamma, \vv{y}:\vv{b}$. Тогава $\val{\Gamma'} = \val{\Gamma} \times \val{\vv{b}}$, а от И.П. имаме, че
  \[\val{\tau'}_{\Gamma'} \in \Cont{\val{\Gamma} \times \val{\vv{b}}}{\val{\vv{c}}},\]
  и следователно според ....
  \[\curry(\val{\tau'}_{\Gamma'}) \in \Cont{\val{\Gamma}}{\Cont{\val{\vv{b}}}{\val{\vv{c}}}}\]
  Дефинираме 
  \[\val{\lamb{y}{b}{\tau}}_\Gamma \dff \curry(\val{\tau'}_{\Gamma'}),\]
  което означава, че за всяко $\overline{u} \in \val{\Gamma}$,
  \[\val{\lamb{y}{b}{\tau}}_\Gamma(\overline{u}) = \curry(\val{\tau'}_{\Gamma'})(\overline{u}) \in \Cont{\val{\vv{b}}}{\val{\vv{c}}}.\]
\item
  Нека сега $\tau \equiv \fix(\tau')$.
  Понеже $\Gamma \vdash \fix(\tau') : \vv{a}$, то $\Gamma \vdash \tau' : \vv{a} \to \vv{a}$.
  От И.П. знаем, че
  \[\val{\tau'}_\Gamma \in \Cont{\val{\Gamma}}{\Cont{\val{\vv{a}}}{\val{\vv{a}}}}.\]
  Това означава, че за произволни $\overline{u} \in \val{\Gamma}$,
  \[\val{\tau'}_\Gamma(\overline{u}) \in \Cont{\val{\vv{a}}}{\val{\vv{a}}}.\]
  Това означава, че
  $\val{\tau'}_\Gamma(\overline{u})$ е изображение, което според Теорема ...
  притежава най-малка неподвижна точка.
  Дефинираме
  \[\val{\texttt{fix}(\tau')}_\Gamma(\overline{u}) \dff \lfp(\val{\tau'}_\Gamma(\overline{u})).\]
\end{itemize}


%%% Local Variables:
%%% mode: latex
%%% TeX-master: "../sep"
%%% End:
