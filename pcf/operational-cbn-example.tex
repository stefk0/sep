\begin{example}
  Нека $\vv{a} = \vv{nat} \to (\vv{nat} \to \vv{nat})$ и 
  \[\tau \equiv \fix\vv{(}\underbrace{\lamb{f}{a}{\overbrace{\lamb{x}{nat}{\lamb{y}{nat}{\ifelse{\vv{y == 0}}{\vv{0}}{\vv{x + (f x (y-1))} }}}}^{\tau'_0}}}_{\tau_0}\vv{)}.\]
  Лесно се вижда, че
  \[\emptyset \vdash \tau:\vv{nat} \to (\vv{nat} \to \vv{nat}).\]

  Също така за всяко $n$ и $k$, ако $m = n * k$, то
  \[\tau\ \vv{n}\ \vv{k} \Downarrow_{\vv{nat}} \vv{m}.\]
  Нека сега
  \[\rho \equiv \lamb{g}{a}{\fix(\lambda \vv{f} : \vv{nat}\to\vv{nat}\ .\ \lamb{x}{nat}{\ifelse{\vv{x == 0}}{\vv{1}}{\vv{g x f(x-1)}}}}).\]
  Лесно се вижда, че
  \[\emptyset \vdash \rho : \vv{a} \to (\vv{nat} \to \vv{nat}).\]
  Също така, за всяко $n$, ако $k = n!$, то
  \[ \rho\ \tau\ \vv{n} \opsem{}{nat} \vv{k}.\]
\end{example}

Нека да положим
\[\Gamma \df \vv{f} : \vv{a}, \vv{x} : \vv{nat}, \vv{y} : \vv{nat}.\]


\begin{landscape}
  \begin{framed}
  \begin{figure}[H]
    \centering
    % \begin{subfigure}[b]{1\textwidth}
      \begin{prooftree}
        \AxiomC{$\Gamma(\vv{y}) = \vv{nat}$}
        \UnaryInfC{$\Gamma \vdash \vv{y} : \vv{nat}$}
        \AxiomC{}
        \UnaryInfC{$\Gamma \vdash \vv{0} : \vv{nat}$}
        \BinaryInfC{$\Gamma \vdash \vv{y==0} : \vv{nat}$}
        \AxiomC{}
        \UnaryInfC{$\Gamma \vdash \vv{0} : \vv{nat}$}
        \AxiomC{$\Gamma(\vv{x}) = \vv{nat}$}
        \UnaryInfC{$\Gamma \vdash \vv{x}:\vv{nat}$}
        \AxiomC{$\Gamma(\vv{f}) = \vv{a}$}
        \UnaryInfC{$\Gamma \vdash \vv{f} : \vv{a}$}
        \AxiomC{$\Gamma(\vv{x}) = \vv{nat}$}
        \UnaryInfC{$\Gamma \vdash \vv{x} : \vv{nat}$}
        \BinaryInfC{$\Gamma \vdash \vv{f x} : \vv{nat} \to \vv{nat}$}
        \AxiomC{$\Gamma(\vv{y}) = \vv{nat}$}
        \UnaryInfC{$\Gamma \vdash \vv{y} : \vv{nat}$}
        \AxiomC{}
        \UnaryInfC{$\Gamma \vdash \vv{1} : \vv{nat}$}
        \BinaryInfC{$\Gamma \vdash \vv{y-1} : \vv{nat}$}
        \BinaryInfC{$\Gamma \vdash \vv{f x (y-1)} : \vv{nat}$}
        \BinaryInfC{$\Gamma \vdash \vv{x + f x (y-1)} : \vv{nat}$}
        \TrinaryInfC{$\vv{f} : \vv{a}, \vv{x} : \vv{nat}, \vv{y} : \vv{nat} \vdash \ifelse{\vv{y == 0}}{\vv{0}}{\vv{x + (f x (y-1))}} : \vv{nat}$}
        \UnaryInfC{$\vv{f} : \vv{a}, \vv{x} : \vv{nat} \vdash \lamb{y}{nat}{\ifelse{\vv{y == 0}}{\vv{0}}{\vv{x + (f x (y-1))}}} : \vv{nat} \to \vv{nat}$}
        \UnaryInfC{$\vv{f} : \vv{a} \vdash \tau'_0 : \vv{a}$}
        \UnaryInfC{$\emptyset \vdash \tau_0 : \vv{a} \to \vv{a}$}
        \UnaryInfC{$\emptyset \vdash \fix(\tau_0):\vv{a}$}
      \end{prooftree}
      \caption{Формален извод според правилата на типизиращата релацията, който показва, че $\emptyset \vdash \tau : \vv{a}$.}
    % \end{subfigure}
  \end{figure}
\end{framed}

След като видяхме, че $\tau$ е терм от тип $\vv{a}$, нека да видим
колко стъпки ще са ни нужни, според правилата на операционната семантика, за да проверим, че
$\tau\ \vv{3}\ \vv{2} \opsem{}{nat} \vv{6}$.

Първо да обърнем внимание, че $\tau_0$ е стойност, т.е. $\tau_0 \opsemGen{0}{\vv{a}\to\vv{a}} \tau_0$.
Макар и $\tau'_0$ да има свободна променлива $\vv{f}$, понеже термът $\fix(\tau_0)$ е затворен с тип $\vv{a}$, то термът
\[\tau'_0\subst{f}{\fix(\tau_0)} \equiv \lamb{x}{nat}{\lamb{y}{nat}{\ifelse{\vv{y == 0}}{\vv{0}}{\vv{ x + (}\fix(\tau_0)\vv{ x (y-1))}}}}\]
е стойност и следователно,
\[\tau'_0\subst{f}{\fix(\tau_0)} \opsem{0}{a} \tau'_0\subst{f}{\fix(\tau_0)}.\]

\begin{framed}
  \begin{figure}[H]
    \begin{prooftree}
      \AxiomC{$\tau_0$ е стойност}
      \UnaryInfC{$\tau_0 \opsemGen{0}{\vv{a}\to\vv{a}} \lamb{f}{a}{\tau'_0}$}
      \AxiomC{$\tau'_0\subst{f}{\fix(\tau_0)}$ е стойност}
      \UnaryInfC{$\tau'_0\subst{f}{\fix(\tau_0)} \opsem{0}{a}  \lamb{x}{nat}{\tau_2}$}
      \BinaryInfC{$\tau_0\fix(\tau_0) \opsem{1}{a} \lamb{x}{nat}{\tau_2}$}
      \UnaryInfC{$\fix(\tau_0) \opsem{2}{a} \lamb{x}{nat}{\tau_2}$}
      \AxiomC{$\tau_2\substConst{x}{3}$ е стойност}
      \UnaryInfC{$\tau_2\substConst{x}{3} \opsemGen{0}{\vv{nat}\to \vv{nat}} \lamb{y}{nat}{\tau_1}$}
      \BinaryInfC{$\fix(\tau_0)\ \vv{3} \opsemGen{3}{\vv{nat}\to\vv{nat}} \lamb{y}{nat}{\tau_1}$}
      \AxiomC{(продължава по-долу)}
      \UnaryInfC{$\tau_1[\vv{y}/\vv{2}] \opsem{t}{nat} \vv{6}$}
      \RightLabel{(app)}
      \BinaryInfC{$\fix(\tau_0)\ \vv{3}\ \vv{2} \opsem{t+4}{nat} \vv{6}$}
    \end{prooftree}
    \caption{Първа част от изчислението на $\tau\ \vv{3}\ \vv{2}$ според правилата на операционната семантика.}
  \end{figure}
\end{framed}

Нека за улеснение да положим
\begin{align*}
  \tau_2 & \equiv \lamb{y}{nat}{\ifelse{\vv{y == 0}}{\vv{0}}{\vv{ x + (}\fix(\tau_0)\vv{ x (y-1))}}}\\
  \tau_1 & \equiv \ifelse{\vv{y == 0}}{\vv{0}}{\vv{ 3 + (}\fix(\tau_0)\vv{ 3 (y-1))}}.
\end{align*}

Термът $\tau_2$ не е стойност, защото има свободна променлива $\vv{x}$, но вече термът $\tau_2\subst{x}{3}$ е стойност. Лесно се съобразява, че
\begin{align*}
  & \tau'_0\subst{f}{\fix(\tau_0)} \equiv \lamb{x}{nat}{\tau_2}\\
  & \tau_2 \substConst{x}{3} \equiv \lamb{y}{nat}{\tau_1}.
\end{align*}



% Оттук съобразяваме, че
%   \[\tau_2 \equiv \lamb{y}{nat}{\ifelse{\vv{y == 0}}{\vv{0}}{\vv{ x + (}\fix(\tau_0)\vv{ x (y-1))}}}.\]
%   Това означава, че
%   \[\tau_1 \equiv \ifelse{\vv{y == 0}}{\vv{0}}{\vv{ 3 + (}\fix(\tau_0)\vv{ 3 (y-1))}}\]
%   и
%   \[\tau_1\substConst{y}{2} \equiv \ifelse{\vv{2 == 0}}{\vv{0}}{\vv{ 3 + (}\fix(\tau_0)\vv{ 3 (2-1))}}\]
  Сега продължаваме десния клон на изчислението:
  \begin{framed}
    \begin{figure}[H]
      \begin{prooftree}
        \AxiomC{$\vv{2}$ е стойност}
        \UnaryInfC{$\vv{2} \opsem{0}{nat} \vv{2}$}
        \AxiomC{$\vv{0}$ е стойност}
        \UnaryInfC{$\vv{0} \opsem{0}{nat} \vv{0}$}
        \BinaryInfC{$\vv{2 == 0} \opsem{1}{nat} \vv{0}$}
        \AxiomC{$\vv{3}$ е стойност}
        \UnaryInfC{$\vv{3} \opsem{0}{nat} \vv{3}$}
        \AxiomC{(Повтаря се както по-горе)}
        \UnaryInfC{$\fix(\tau_0)\ \vv{3} \opsemGen{3}{\vv{nat}\to\vv{nat}} \lamb{y}{nat}{\tau_1}$}
        \AxiomC{(продължава по-долу)}
        \UnaryInfC{$\tau_1\substConst{y}{2-1} \opsem{k}{nat} \vv{3}$}
        \BinaryInfC{$\fix(\tau_0)\vv{ 3 (2-1) }\opsem{k+4}{nat} \vv{3}$}
        \BinaryInfC{$\vv{ 3 + (}\fix(\tau_0)\vv{ 3 (2-1))} \opsem{k+5}{nat} \vv{6}$}
        \BinaryInfC{$\underbrace{\ifelse{\vv{2 == 0}}{\vv{0}}{\vv{ 3 + (}\fix(\tau_0)\vv{ 3 (2-1))}}}_{\tau_1\substConst{y}{2}} \opsem{k+6}{nat} \vv{6}$}
      \end{prooftree}
      \caption{Втора част от изчислението, която показва, че $\tau_1\substConst{y}{2} \opsem{k+6}{nat} \vv{6}$.}
    \end{figure}
  \end{framed}
  Отново продължаваме с десния клон на изчислението, като тук вече имаме, че:
  \[\tau_1\substConst{y}{2-1} \equiv \ifelse{\vv{2-1 == 0}}{\vv{0}}{\vv{ 3 + (}\fix(\tau_0)\vv{ 3 (2-1-1))}}\]

  \begin{framed}
    \begin{figure}[H]
      \begin{prooftree}
        \AxiomC{$\vv{2}$ е стойност}
        \UnaryInfC{$\vv{2} \opsem{0}{nat} \vv{2}$}
        \AxiomC{$\vv{1}$ е стойност}
        \UnaryInfC{$\vv{1} \opsem{0}{nat} \vv{1}$}
        \BinaryInfC{$\vv{2-1} \opsem{1}{nat} \vv{1}$}
        \AxiomC{$\vv{0}$ е стойност}
        \UnaryInfC{$\vv{0} \opsem{0}{nat} \vv{0}$}
        \BinaryInfC{$\vv{2-1 == 0} \opsem{2}{nat} \vv{0}$}
        \AxiomC{$\vv{3}$ е стойност}
        \UnaryInfC{$\vv{3} \opsem{0}{nat} \vv{3}$}
        \AxiomC{(Повтаря се както по-горе)}
        \UnaryInfC{$\fix(\tau_0)\ \vv{3} \opsemGen{3}{\vv{nat}\to\vv{nat}} \lamb{y}{nat}{\tau_1}$}
        % \AxiomC{(продължава в (\ref{fig:operational-cbn-example:last-part}))}
        \AxiomC{(продължава по-долу)}
        \UnaryInfC{$\tau_1\substConst{y}{2-1-1} \opsem{\ell}{nat} \vv{0}$}
        \BinaryInfC{$\fix(\tau_0)\vv{ 3 (2-1-1) }\opsem{\ell+4}{nat} \vv{0}$}
        \BinaryInfC{$\vv{ 3 + (}\fix(\tau_0)\vv{ 3 (2-1-1))} \opsem{\ell+5}{nat} \vv{3}$}
        \BinaryInfC{$\underbrace{\ifelse{\vv{2-1 == 0}}{\vv{0}}{\vv{ 3 + (}\fix(\tau_0)\vv{ 3 (2-1-1))}}}_{\tau_1\substConst{y}{2-1}} \opsem{\ell+6}{nat} \vv{3}$}
      \end{prooftree}
      \caption{Трета част от изчислението, която показва, че $\tau_1\substConst{y}{2-1} \opsem{\ell+6}{nat} \vv{3}$ }
    \end{figure}
  \end{framed}
  Сега вече имаме, че
  \[\tau_1\substConst{y}{2-1-1} \equiv \ifelse{\vv{2-1-1 == 0}}{\vv{0}}{\vv{ 3 + (}\fix(\tau_0)\vv{ 3 (2-1-1-1))}}.\]

  Завършваме изчислението така:
  \begin{framed}
    \begin{figure}[H]
      \label{fig:operational-cbn-example:last-part}
      \begin{prooftree}
        \AxiomC{$\vv{2}$ е стойност}
        \UnaryInfC{$\vv{2} \opsem{0}{nat} \vv{2}$}
        \AxiomC{$\vv{1}$ е стойност}
        \UnaryInfC{$\vv{1} \opsem{0}{nat} \vv{1}$}
        \BinaryInfC{$\vv{2-1} \opsem{1}{nat} \vv{1} $}
        \AxiomC{$\vv{1}$ е стойност}
        \UnaryInfC{$\vv{1} \opsem{0}{nat} \vv{1}$}
        \BinaryInfC{$\vv{2-1-1} \opsem{2}{nat} \vv{0}$}
        \AxiomC{$\vv{0}$ е стойност}
        \UnaryInfC{$\vv{0} \opsem{0}{nat} \vv{0}$}
        \BinaryInfC{$\vv{2-1-1 == 0} \opsem{3}{nat} \vv{1}$}
        \AxiomC{$\vv{0}$ е стойност}
        \UnaryInfC{$\vv{0} \opsem{0}{nat} \vv{0}$}
        \BinaryInfC{$\underbrace{\ifelse{\vv{2-1-1 == 0}}{\vv{0}}{\vv{ 3 + (}\fix(\tau_0)\vv{ 3 (2-1-1-1))}}}_{\tau_1\substConst{y}{2-1-1}} \opsem{4}{nat} \vv{0}$}
      \end{prooftree}
      \caption{Последна част от изчислението, което показва, че $\tau_1\substConst{y}{2-1-1} \opsem{4}{nat} \vv{0}$.}
    \end{figure}
  \end{framed}  
\end{landscape}


Можем да напишем директно горния пример и на хаскел:
\begin{haskellcode}
ghci> fix f = f (fix f)
ghci> times = fix(\f x y -> if y == 0 then 0 else x + (f x (y-1)))
ghci> times 2 3
6
ghci> fct = \g -> fix(\f x -> if x == 0 then 1 else g x (f (x-1)))
ghci> fact = fct times
ghci> fact 5
120
\end{haskellcode}

