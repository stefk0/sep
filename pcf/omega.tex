Добре е още тук да разгледаме термове, които ще ни бъдат полезни по-нататък, когато искаме да изучим свойствата на операционната и денотационната семантики на \PCF.

\begin{framed}
  \begin{definition}
    За произволен тип $\vv{a}$, да означим затворените термове
    \begin{align*}
      & \Omega_{\vv{a}} \df \fix(\lamb{x}{a}{\vv{x}})\\
      & \Omega'_{\vv{a}} \df \lamb{x}{a}{\fix(\lamb{x}{a}{\vv{x}})}.
    \end{align*}
  \end{definition}
\end{framed}

\begin{problem}
  \label{prob:pcf:context:omega}
  \marginpar{\todo Студентите би трябвало да могат да докажат твърденията в \Problem{pcf:context:omega} сами!}
  Докажете, че за всеки тип $\vv{a}$ са изпълнени следните свойства:
  \begin{enumerate}[(1)]
  \item
    \label{pcf:omega:type}
    $\emptyset \vdash \Omega_{\vv{a}} : \vv{a}$ и $\emptyset \vdash \Omega'_{\vv{a}} : \vv{a} \to \vv{a}$;
  \item
    \label{pcf:omega:operational}
    $\Omega_{\vv{a}} \not\Downarrow_{\vv{a}}$ и $\Omega'_{\vv{a}} \Downarrow^0_{\vv{a}\to\vv{a}} \Omega'_{\vv{a}}$;
  \item
    \label{pcf:omega:denotational}
    $\val{\Omega_{\vv{a}}} = \bot^{\val{\vv{a}}}$ и $\val{\Omega'_{\vv{a}}} = \val{\Omega_{\vv{a}\to\vv{a}}} = \bot^{\val{\vv{a}\to\vv{a}}}$.    
  \end{enumerate}
\end{problem}
\begin{hint}
  Доказателството на Свойство~\ref{pcf:omega:type} представлява едно просто упражнение, което за пълнота на изложението ще направим.

  \begin{figure}[H]
    \begin{subfigure}{0.5\textwidth}
      \begin{prooftree}
        \AxiomC{$\type{x}{a} \vdash \type{x}{a}$}
        \UnaryInfC{$\emptyset \vdash \lamb{x}{a}{\vv{x}} : \vv{a}\to\vv{a}$}
        \UnaryInfC{$\emptyset \vdash \underbrace{\fix(\lamb{x}{a}{\vv{x}})}_{\Omega_{\vv{a}}} : \vv{a}$}
      \end{prooftree}
      % \caption{}
      % \label{fig:pcf:context:omega}
    \end{subfigure}
    ~
    \begin{subfigure}{0.5\textwidth}
      \begin{prooftree}
        % \AxiomC{от (\ref{fig:pcf:context:omega})}
        \AxiomC{вече доказано}
        \UnaryInfC{$\emptyset \vdash \Omega_{\vv{a}} : \vv{a}$}
        \UnaryInfC{$\type{x}{a} \vdash \Omega_{\vv{a}} : \vv{a}$}
        \UnaryInfC{$\underbrace{\emptyset \vdash \lamb{x}{a}{\Omega_{\vv{a}}}}_{\Omega'_{\vv{a}}} : \vv{a} \to \vv{a}$}
      \end{prooftree}
      % \caption{}
    \end{subfigure}
  \end{figure}
  
  \marginpar{Втората част на Свойство~\ref{pcf:omega:operational} не заслужава внимание, защото $\Omega'_{\vv{a}}$ е стойност и следователно по дефиниция $\Omega'_{\vv{a}} \opsemGen{0}{\vv{a}\to\vv{a}} \Omega'_{\vv{a}}$.}
  За първата част на Свойство~\ref{pcf:omega:operational}, да допуснем, че $\Omega_{\vv{a}} \Downarrow_{\vv{a}} \vv{v}$, за някоя стойност $\type{v}{a}$, и нека фиксираме $\ell$
  да бъде {\em най-малкият} брой стъпки, за които $\Omega_{\vv{a}} \Downarrow^{\ell}_{\vv{a}} \vv{v}$.
  Но тогава
  \begin{prooftree}
    \AxiomC{$\lamb{x}{a}{\vv{x}}$ е стойност}
    \UnaryInfC{$\lamb{x}{a}{\vv{x}} \Downarrow^0_{\vv{a}\to\vv{a}} \lamb{x}{a}{\vv{x}}$}
    \AxiomC{$\Omega_{\vv{a}} \Downarrow^{\ell-2}_{\vv{a}} \vv{v}$}
    \UnaryInfC{$\vv{x}\subst{x}{\fix(\lamb{x}{a}{\vv{x}})} \Downarrow^{\ell-2}_{\vv{a}} \vv{v}$}
    \RightLabel{\scriptsize{(cbn)}}
    \BinaryInfC{$(\lamb{x}{a}{\vv{x}})\fix(\lamb{x}{a}{\vv{x}}) \Downarrow^{\ell-1}_{\vv{a}} \vv{v}$}
    \RightLabel{\scriptsize{(fix)}}
    \UnaryInfC{$\underbrace{\fix(\lamb{x}{a}{\vv{x}})}_{\Omega_{\vv{a}}} \Downarrow^\ell_{\vv{a}} \vv{v}$}
  \end{prooftree}
  Получихме, че $\Omega_{\vv{a}} \Downarrow^{\ell-2}_{\vv{a}} \vv{v}$, което е противоречие с минималността на $\ell$.

  Сега да разгледаме първата част на Свойство~\ref{pcf:omega:denotational}. За произволен тип $\vv{a}$, да означим с $\texttt{id}_{\val{\vv{a}}}$ функцията идентитет за областта на Скот $\val{\vv{a}}$, т.е.
  $\texttt{id}_{\val{\vv{a}}}(x) = x$ за всяко $x \in \val{\vv{a}}$. Имаме, че:
  \marginpar{Да напомним, че $f^0 \df id$ и $f^{n+1} \df f \circ f^n$.
    В нашия случай, $f = id$ и следователно $id^n = id$ за всяко $n$.}
  \begin{align*}
    \val{\Omega_{\vv{a}}} & = \val{\fix(\lamb{x}{a}{\vv{x}})}\\
                          & = \lfp(\val{\lamb{x}{a}{\vv{x}}})\\
                          & = \lfp(\texttt{id}_{\val{\vv{a}}}) & \comment\val{\lamb{x}{a}{\vv{x}}} = \texttt{id}_{\val{\vv{a}}}\\
                          & = \bigsqcup_n\texttt{id}^n_{\val{\vv{a}}}(\bot^{\val{\vv{a}}}) & \comment\text{\hyperref[th:knaster-tarski]{Теоремата на Клини}}\\
                          & = \bigsqcup_n \bot^{\val{\vv{a}}} & \comment \texttt{id}^n(\bot^{\val{\vv{a}}}) = \bot^{\val{\vv{a}}}\\
                          & = \bot^{\val{\vv{a}}}.
  \end{align*}
  Сега преминаваме към втората част на Свойство~\ref{pcf:omega:denotational}. За произволен елемент $u \in \val{\vv{a}}$,
  \marginpar{Да напомним, че за всяко $u$, $\bot^{\val{\vv{a}\to\vv{a}}}(u) \df \bot^{\val{\vv{a}}}$.}
  \begin{align*}
    \val{\Omega'_a}(u) & = \val{\lamb{x}{a}{\Omega_{\vv{a}}}}(u)\\
                       & = \curry(\val{\Omega_{\vv{a}}}_{\type{x}{a}})(u)\\
                       & = \val{\Omega_{\vv{a}}}_{\type{x}{a}}(u)\\
                       & = \val{\Omega_{\vv{a}}}\\
                       & = \bot^{\val{\vv{a}}}.
  \end{align*}

  Заключаваме, че $\val{\Omega'_a} = \bot^{\val{\vv{a}\to\vv{a}}}$.
\end{hint}
