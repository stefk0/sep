\begin{example}
  Нека $\vv{a} = \vv{nat} \to (\vv{nat} \to \vv{nat})$ и 
  \[\tau \equiv \fix\vv{(}\underbrace{\lamb{f}{a}{\overbrace{\lamb{x}{nat}{\lamb{y}{nat}{\ifelse{\vv{y == 0}}{\vv{0}}{\vv{x + (f x (y-1))} }}}}^{\tau'_0}}}_{\tau_0}\vv{)}.\]

  Знаем, че $\val{\tau} \in \Cont{\Nat_\bot}{\Cont{\Nat_\bot}{\Nat_\bot}}$, където
  \[\val{\tau} \df \lfp(\val{\tau_0}).\]

  Ясно е, че $\val{\tau_0} \in \Cont{\val{\vv{a}}}{\val{\vv{a}}}$ и
  $\val{\tau_0} = \curry(\val{\tau'_0}_{\vv{f:a}}) = \val{\tau'_0}_{\vv{f:a}}$ и
  сега пък ако положим
  \begin{align*}
    % & \tau_0 \equiv \lamb{f}{a}{\lamb{x}{nat}{\lamb{y}{nat}{\ifelse{\vv{y == 0}}{\vv{0}}{\vv{x + (f x (y-1))} }}}},\\
    % & \tau'_0 \equiv \lamb{x}{nat}{\lamb{y}{nat}{\ifelse{\vv{y == 0}}{\vv{0}}{\vv{x + (f x (y-1))}}}},\\
    & \tau''_0 \equiv \lamb{y}{nat}{\ifelse{\vv{y == 0}}{\vv{0}}{\vv{x + (f x (y-1))}}},\\
    & \tau'''_0 \equiv \ifelse{\vv{y == 0}}{\vv{0}}{\vv{x + (f x (y-1))}}.
  \end{align*}

  то ще получим, че
  \[\val{\tau'_0}_{\vv{f:a}} = \curry(\val{\tau''_0}_{\vv{f:a,x:nat}}),\]
  и тогава
  \[\val{\tau'_0}_{\vv{f:a}}(\varphi)(m) = \val{\tau''_0}_{\vv{f:a,x:nat}}(\varphi,m).\]
  Сега вече получаваме, че
  \[\val{\tau''_0}_{\vv{f:a,x:nat}} = \curry(\val{\tau'''_0}_{\vv{f:a,x:nat,y:nat}}),\]
  т.е.
  \[\val{\tau''_0}_{\vv{f:a,x:nat}}(\varphi,m)(n) = \val{\tau'''_0}_{\vv{f:a,x:nat,y:nat}}(\varphi,m,n).\]
  Обединявайки всичко получаваме, че:
  \begin{align*}
      \val{\tau_0}(\varphi)(m)(n) & = \val{\tau'_0}_{\vv{f:a}}(\varphi)(m)(n) \\
                                  & = \val{\tau''_0}_{\vv{f:a,x:nat}}(\varphi,m)(n)\\
                                  & = \val{\tau'''_0}_{\vv{f:a,x:nat,y:nat}}(\varphi,m,n)\\
                                  & = \val{\ifelse{\vv{y==0}}{\vv{0}}{\vv{x + (f x (y-1))}}}(\varphi,m,n)\\
                                  & = \texttt{if}(\val{\vv{y==0}}(\varphi,m,n), \val{\vv{0}}(\varphi,m,n),\val{\vv{x + (f x (y-1))}}(\varphi,m,n)).
  \end{align*}

  Накрая получаваме, че
  \[\val{\tau_0}(\varphi)(m)(n) = \begin{cases}
      0, & \text{ако }n = 0\\
      \plus(m, \varphi(m)(n-1)), & \text{ако } n > 0\\
      \bot, & \text{ако }n = \bot.
    \end{cases}
  \]
  
                                
  Сега вече знаем как по теоремата на Клини да докажем, че
  \[\lfp(\val{\tau_0})(m)(n) =
    \begin{cases}
      m*n,  & \text{ако }m,n\in\Nat\\
      \bot, & \text{иначе}
    \end{cases}
\]
                                
  
\end{example}
