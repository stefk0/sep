\section{Добре типизирани термове}\index{тип}
Всяка \emph{крайна} функция от вида $\Gamma : \mathcal{V} \to \mathcal{T}$
ще наричаме {\bf типов контекст}.
\marginpar{$\Gamma$ също се нарича и type environment.}
Ние искаме да работим само с коректно типизирани термове.
Например, не е ясно какво означава терма $\vv{1 + }\lamb{x}{nat}{\vv{x}}$.
Ако в изразите имаме свободни променливи, то дали един израз е коректно типизиран ще зависи от типовия контекст.

\begin{prooftree}
  \AxiomC{}
  \RightLabel{\scriptsize{(const)}}
  \UnaryInfC{$\Gamma \vdash \vv{n} : \vv{nat}$}
\end{prooftree}

\begin{prooftree}
  \AxiomC{$\vv{x} \in \texttt{dom}(\Gamma)$}
  \AxiomC{$\Gamma(\vv{x}) = \vv{a}$}
  \RightLabel{\scriptsize{(var)}}
  \BinaryInfC{$\Gamma \vdash \vv{x} : \vv{a}$}
\end{prooftree}

\begin{prooftree}
  \AxiomC{$\Gamma \vdash \tau_1:\vv{nat}$}
  \AxiomC{$\Gamma \vdash \tau_2:\vv{nat}$}
  \RightLabel{\scriptsize{(plus)}}
  \BinaryInfC{$\Gamma \vdash \tau_1 + \tau_2 : \vv{nat}$}
\end{prooftree}

\begin{prooftree}
  \AxiomC{$\Gamma \vdash \tau_1:\vv{nat}$}
  \AxiomC{$\Gamma \vdash \tau_2:\vv{nat}$}
  \RightLabel{\scriptsize{(eq)}}
  \BinaryInfC{$\Gamma \vdash \tau_1\ \vv{==}\ \tau_2 : \vv{nat}$}
\end{prooftree}

\begin{prooftree}
  \AxiomC{$\Gamma \vdash \tau_1:\vv{nat}$}
  \AxiomC{$\Gamma \vdash \tau_2:\vv{a}$}
  \AxiomC{$\Gamma \vdash \tau_3:\vv{a}$}
  \RightLabel{\scriptsize{(if)}}
  \TrinaryInfC{$\Gamma \vdash \ifelse{\tau_1}{\tau_2}{\tau_3} : \vv{a}$}
\end{prooftree}

\begin{prooftree}
  \AxiomC{$\Gamma \vdash \tau_1:\vv{a}\to\vv{b}$}
  \AxiomC{$\Gamma \vdash \tau_2:\vv{a}$}
  \RightLabel{\scriptsize{(app)}}
  \BinaryInfC{$\Gamma \vdash \tau_1\tau_2 : \vv{b}$}
\end{prooftree}

\begin{prooftree}
  \AxiomC{$\Gamma \vdash \tau:\vv{a}\to\vv{a}$}
  \RightLabel{\scriptsize{(fix)}}
  \UnaryInfC{$\Gamma \vdash \fix(\tau) : \vv{a}$}
\end{prooftree}

\begin{prooftree}
  \AxiomC{$\vv{x} \not\in\vv{dom}(\Gamma)$}
  \AxiomC{$\Gamma, \type{x}{a} \vdash \tau:\vv{b}$}
  \RightLabel{\scriptsize{(lambda)}}
  \BinaryInfC{$\Gamma \vdash \lambda \type{x}{a}\ .\ \tau : \vv{a} \to \vv{b}$}
\end{prooftree}

Ако имаме затворен израз $\tau$, то ще пишем $\tau : \vv{a}$ вместо $\emptyset \vdash \tau : \vv{a}$.

Да положим
\[PCF_{\vv{a}} \dff \{\tau \text{ е затворен терм}\mid \emptyset \vdash \tau : \vv{a}\}.\]


\begin{proposition}
  Ако $\Gamma \vdash \tau : \vv{a}$, то $\fv(\tau) \subseteq \vv{dom}(\Gamma)$.
\end{proposition}

\begin{proposition}
  Ако $\Gamma \vdash \tau : \vv{a}$ и $\Gamma \vdash \tau : \vv{b}$, то $\vv{a} = \vv{b}$.
\end{proposition}

\begin{corollary}
  Всеки затворен терм има най-много един тип.
\end{corollary}

% \begin{proposition}
%   \marginpar{Къде се използва това твърдение?}
%   За всеки контекст $\Gamma$, произволни термове $\tau$, $\rho$, променлива $\vv{x}$, $\vv{x} \not \in \vv{dom}(\Gamma)$ и типове $\vv{a}$ и $\vv{b}$, е изпълнено, че:
%   \begin{prooftree}
%     \AxiomC{$\Gamma \vdash \rho : \vv{a}$}
%     \AxiomC{$\Gamma, \type{x}{a} \vdash \tau : \vv{b}$}
%     \BinaryInfC{$\Gamma \vdash \tau[\vv{x}/\rho] : \vv{b}$}
%   \end{prooftree}
% \end{proposition}
% \begin{proof}
%   Индукция по построението на термовете.
%   \begin{itemize}
%   \item
%     Нека $\tau \equiv \fix(\tau')$ и нека
%     \begin{align*}
%       & \Gamma \vdash \rho : \vv{a},\\
%       & \Gamma, \type{x}{a} \vdash \tau : \vv{b}.
%     \end{align*}
%     Трябва да докажем, че $\Gamma \vdash \tau[\vv{x}/\rho] : \vv{b}$.
%     От правилата за типизиране е ясно, че
%     \begin{prooftree}
%       \AxiomC{$\Gamma, \type{x}{a} \vdash \tau':\vv{b}\to\vv{b}$}
%       \RightLabel{\scriptsize{(fix)}}
%       \UnaryInfC{$\Gamma, \type{x}{a} \vdash \fix(\tau') : \vv{b}$}
%     \end{prooftree}
%     Сега можем да приложим И.П. за терма $\tau'$. Получаваме, че
%     \begin{prooftree}
%       \AxiomC{$\Gamma \vdash \rho: \vv{a}$}
%       \AxiomC{$\Gamma, \type{x}{a} \vdash \tau':\vv{b}\to\vv{b}$}
%       \RightLabel{\scriptsize{(И.П.)}}
%       \BinaryInfC{$\Gamma \vdash \tau'\subst{x}{\rho} : \vv{b} \to \vv{b}$}
%       \RightLabel{\scriptsize{(fix)}}
%       \UnaryInfC{$\Gamma \vdash \fix(\tau'\subst{x}{\rho}) : \vv{b}$}
%     \end{prooftree}
%     Понеже $\fix(\tau'\subst{x}{\rho}) \equiv \fix(\tau')\subst{x}{\rho}$, то заключаваме, че
%     \[\Gamma \vdash \tau[\vv{x}/\rho] : \vv{b}.\]
%   \item
%     Нека $\tau \equiv \lamb{y}{c}{\tau'}$, където $\vv{y} \not\in \fv(\rho) \cup \{\vv{x}\}$
%     и нека
%     \begin{align*}
%       & \Gamma \vdash \rho : \vv{a},\\
%       & \Gamma, \type{x}{a} \vdash \tau : \vv{b}.
%     \end{align*}
%     Трябва да докажем, че $\Gamma \vdash \tau[\vv{x}/\rho] : \vv{b}$.
    
%     От правилата за типизиране е ясно, че щом $\Gamma, \type{x}{a} \vdash \tau : \vv{b}$, то
%     $\vv{b} = \vv{c} \to \vv{d}$ за някой тип $\vv{d}$ и
%     \begin{prooftree}
%       \AxiomC{$\vv{y} \not\in\vv{dom}(\Gamma)\cup\{\vv{x}\}$}
%       \AxiomC{$\Gamma, \type{x}{a}, \type{y}{c} \vdash \tau':\vv{d}$}
%       \RightLabel{\scriptsize{(lambda)}}
%       \BinaryInfC{$\Gamma, \type{x}{a} \vdash \lamb{y}{c}{\tau'} : \vv{c}\to\vv{d}$}
%     \end{prooftree}
%     Това означава, че можем да използваме И.П. за терма $\tau'$ и така получаваме, че
%     \begin{prooftree}
%       \AxiomC{$\vv{y} \not\in \vv{dom}(\Gamma)$}
%       \AxiomC{$\Gamma \vdash \rho : \vv{a}$}
%       \UnaryInfC{$\Gamma,\type{y}{c} \vdash \rho : \vv{a}$}
%       \AxiomC{$\Gamma, \type{y}{c}, \type{x}{a} \vdash \tau':\vv{d}$}
%       \RightLabel{\scriptsize{(И.П.)}}
%       \BinaryInfC{$\Gamma, \type{y}{c} \vdash \tau'\subst{x}{\rho} : \vv{d}$}
%       \RightLabel{\scriptsize{(lambda)}}
%       \BinaryInfC{$\Gamma \vdash \lamb{y}{c}{\tau'\subst{x}{\rho}}:\vv{c}\to\vv{d}$}
%     \end{prooftree}
%     Накрая, понеже $\tau\subst{x}{\rho} \equiv \lamb{y}{c}{\tau'\subst{x}{\rho}}$, то
%     заключаваме, че
%     \[\Gamma \vdash \tau\subst{x}{\rho}:\vv{b}.\]
%   \end{itemize}
% \end{proof}

\begin{problem}
  \marginpar{\cite[стр. 104]{types-programming-languages}}
  Докажете или опровергайте дали е възможно да съществува контекст $\Gamma$ и тип $\vv{a}$, такива че
  \[\Gamma \vdash \type{xx}{a}.\]
\end{problem}



%%% Local Variables:
%%% mode: latex
%%% TeX-master: "../sep"
%%% End:
