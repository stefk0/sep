\section{Пълна абстракция}\label{pcf:sect:full-abstraction}
\marginpar{Full abstraction на англ., \cite{milner-fully-abstract}.}
\begin{definition}
  \marginpar{\cite[стр. 179]{gunter}}
  Денотационната семантика $\val{.}$ се нарича {\bf напълно абстрактна}, ако
  контекстната (операционната) и денотационната наредба съвпадат, т.е.
  за произволни термове $\tau_1,\tau_2$ от тип $\vv{a}$ е изпълнено, че
  \[\val{\tau_1} \sqsubseteq \val{\tau_2} \iff \tau_1 \leq_{ctx} \tau_2 : \vv{a}.\]
\end{definition}

\begin{framed}
  \begin{theorem}[Гордън Плоткин 1977]
    Денотационната семантика $\val{.}$ за \PCF++ {\bf не е} напълно абстрактна.
  \end{theorem}
\end{framed}
Да напомним, че от \Th{pcf:context:connection} винаги имаме следното:
\[ \val{\tau_1} = \val{\tau_2} \implies \tau_1 \cong_{ctx} \tau_2 : \vv{a}.\]
Сега ще се захванем с търсенето на термове $\tau_1$ и $\tau_2$, за които
$\val{\tau_1} \neq \val{\tau_2}$ и $\tau_1 \cong \tau_2 : \vv{a}$.

\begin{problem}
  Да дефинираме функцията $sor:\Bool \to (\Bool \to \Bool)$ по следния начин:
  \marginpar{$sor$ идва от sequential or.}

  \vspace{10pt}
  
  \begin{tabular}{|c|c|c|c|}
    \hline
    $sor$ & $\bot$ & $false$ & $true$ \\
    \hline
    $\bot$ & $\bot$ & $\bot$ & $\bot$\\
    \hline
    $false$ & $\bot$ & $false$ & $true$\\
    \hline
    $true$ & $true$ & $true$ & $true$\\
    \hline
  \end{tabular}

  \vspace{10pt}
  
  Докажете, че $\texttt{sor}$ е определима в \PCFPP.
\end{problem}
\begin{hint}
  Разгледайте затворения терм
  
  
  \[\tau \df \lamb{x}{bool}{\lamb{y}{bool}{\ifelse{\vv{x}}{\tru}{{\ifelse{\vv{y}}{\tru}{\fls}}}}}\]
  Докажете, че $\val{\tau} = sor$.
\end{hint}



% \begin{problem}
%   Да дефинираме функцията $sor:\Nat_\bot \to (\Nat_\bot \to \Nat_\bot)$ по следния начин:
%   \marginpar{$sor$ идва от sequential or.}
  
%   \begin{tabular}{|c|c|c|c|}
%     \hline
%     $sor$ & $\bot$ & $0$ & $y>0$ \\
%     \hline
%     $\bot$ & $\bot$ & $\bot$ & $\bot$\\
%     \hline
%     $0$ & $\bot$ & $0$ & $1$\\
%     \hline
%     $x>0$ & $1$ & $1$ & $1$\\
%     \hline
%   \end{tabular}
  
%   Докажете, че $\texttt{sor}$ е определима в PCF.
% \end{problem}
% \begin{hint}
%   Разгледайте затворения терм
%   \[\tau \df \lamb{x}{nat}{\lamb{y}{nat}{\ifelse{\vv{x}}{\vv{1}}{\ifelse{\vv{y}}{\vv{1}}{\vv{0}}}}}\]
%   Докажете, че $\val{\tau} = sor$.
% \end{hint}

\begin{problem}
  Да дефинираме изображението $por:\Bool \to (\Bool \to \Bool)$ по следния начин:

  \vspace{10pt}
  
  \begin{tabular}{|c|c|c|c|}
  \hline
  $por$ & $\bot$ & $false$ & $true$\\
  \hline
  $\bot$ & $\bot$ & $\bot$ & $true$\\
  \hline
  $false$ & $\bot$ & $false$ & $true$\\
  \hline
  $true$ & $true$ & $true$ & $true$\\
  \hline
\end{tabular}
\marginpar{$por$ идва от parallel or.}

\vspace{10pt}

  Докажете, че $por$ е непрекъснато изображение.
\end{problem}
\begin{hint}
  Достатъчно е да се съобрази, че $por$ е монотонно изображение.
\end{hint}



% \begin{problem}

% Да дефинираме изображението $por:\Nat_\bot\to(\Nat_\bot \to \Nat_\bot)$ по следния начин:

% \begin{tabular}{|c|c|c|c|}
%   \hline
%   $por$ & $\bot$ & $0$ & $y>0$\\
%   \hline
%   $\bot$ & $\bot$ & $\bot$ & $1$\\
%   \hline
%   $0$ & $\bot$ & $0$ & $1$\\
%   \hline
%   $x>0$ & $1$ & $1$ & $1$\\
%   \hline
% \end{tabular}
% \marginpar{$por$ идва от parallel or.}

%   Докажете, че $por$ е непрекъснато изображение.
% \end{problem}
% \begin{hint}
%   Достатъчно е да се съобрази, че $por$ е монотонно изображение.
% \end{hint}

\begin{framed}
  \begin{lemma}[Гордън Плоткин 1977]
    Изображението $por$ не е определимо в \PCFPP, т.е. не съществува затворен терм $\tau$ на езика \PCFPP,
    за който $\val{\tau} = por$.
  \end{lemma}
\end{framed}

\begin{example}
Да видим, че операторът ,,или'' в хаскел не е паралелен.
\begin{haskellcode}
ghci> True || undefined
True
ghci> undefined || True
*** Exception: Prelude.undefined
\end{haskellcode}
\end{example}

\begin{problem}\label{prob:pcf:full-abstraction:por}
  Да разгледаме $f \in \Cont{\Bool}{\Cont{\Bool}{\Bool}}$, за което имаме ограниченията:

  \vspace{10pt}
  
  \begin{tabular}{|c|c|c|c|}
    \hline
    $f$ & $\bot$ & $false$ & $true$\\
    \hline
    $\bot$ & $?$ & $?$ & $true$\\
    \hline
    $false$ & $?$ & $false$ & $?$\\
    \hline
    $true$ & $true$ & $?$ & $?$\\
    \hline
  \end{tabular}

  \vspace{10pt}
  
  Докажете, че $f = \texttt{por}$.
  Заключете, че $f$ не е определимо в \PCFPP.
  
\end{problem}
\begin{hint}
  Използвайте монотонността на $f$.
\end{hint}


В следващите твърдения ще използваме типовете
\begin{align*}
  & \vv{a} \df \bool \to (\bool \to \bool)\\
  & \vv{b} \df (\bool \to (\bool \to \bool))\to\bool.
\end{align*}
За $n = 0,1$, нека дефинираме затворените термове

\marginpar{Да напомним, че \[\Omega_{\vv{a}} \df \fix(\lamb{x}{a}{\vv{x}}).\]}

\begin{lstlisting}
  $\tau_n \equiv \lambda \vv{f:a}$.if (f true $\Omega_\bool$) then
              if (f $\Omega_\bool$ true) then
                if (f false false) then $\Omega_\bool$
                  else $\vv{b}_n$
                else $\Omega_\bool$
              else $\Omega_\bool$
\end{lstlisting}

Тук $\vv{b}_0 = \tru$, $\vv{b}_1 = \fls$.

\begin{problem}
  Докажете, че $\emptyset \vdash \tau_0:\vv{b}$ и $\emptyset \vdash \tau_1:\vv{b}$.
\end{problem}

% \begin{problem}
%   Докажете, че 
%   \[\val{\tau_0} \neq \val{\tau_1}.\]
% \end{problem}
% \begin{hint}
  % Докажете, че за $n = 0,1$ е изпълнено, че
  % \[\val{\tau_n}(por) = b_n.\]  
% \end{hint}

Следващото твърдение показва защо денотационната семантика $\val{.}$ на езика \PCFPP не е напълно абстрактна.
\begin{framed}
\begin{proposition}
  $\tau_1 \cong_{ctx} \tau_2 : \vv{b}$, но $\val{\tau_0} \neq \val{\tau_1}$.
\end{proposition}  
\end{framed}
\begin{proof}
  Първо докажете, че за $n = 0,1$ е изпълнено, че
  \[\val{\tau_n}(por) = b_n.\]  
  
  Второ, понеже $\vv{b} = \vv{a} \to \bool$, от \Prop{pcf:context:extensionality} следва, че е достатъчно да докажем, че
  за всеки затворен терм $\rho:\vv{a}$ е изпълнено, че
  \[(\forall \vv{v})[\tau_1\rho \Downarrow_{\bool} \vv{v} \iff \tau_2\rho \Downarrow_{\bool} \vv{v}].\]
  Да видим какво означава $\tau_i \rho \Downarrow_{\bool}$ за $i = 0,1$.
  Това означава, че трябва да са изпълнени и трите свойства:
  \begin{itemize}
  \item
    $\rho\ \tru\ \Omega_{\bool} \Downarrow_{\bool} \tru$;% , за някое $\vv{k} \not\equiv \vv{0}$;
  \item
    $\rho\ \Omega_{\bool}\ \tru \Downarrow_{\bool} \tru$;% , за някое $\vv{m} \not\equiv \vv{0}$;
  \item
    $\rho\ \fls\ \fls \Downarrow_{\bool} \fls$.
  \end{itemize}
  Понеже $\val{\Omega_{\bool}} = \bot$, от \hyperref[th:pcf:soundness]{теоремата за коректност} получаваме, че трябва да са изпълнени следните три свойства:
  \begin{itemize}
  \item
    $\val{\rho}(true)(\bot) = true$;
  \item
    $\val{\rho}(\bot)(true) = true$;
  \item
    $\val{\rho}(false)(false) = false$.
  \end{itemize}
  Но тогава $\val{\rho} = por$, което е противоречие с \Problem{pcf:full-abstraction:por}.
  Заключаваме, че не съществува затворен терм $\rho$ на езика \PCFPP, за който $\tau_i\rho\Downarrow_\bool$ за $i = 0,1$. Това означава, че $\tau_1 \cong_{ctx} \tau_2 : \vv{b}$ по тривиални причини.
\end{proof}

Доказателството на следващата теорема излиза извън обхата на този курс.


\subsection{Езикът \PCFPOR}

Дефинираме езика \PCFPOR като разширение на \PCFPP по следния начин:

\begin{itemize}
\item
  Типовете на \PCFPOR са типовете на \PCFPP
\item
  Термовете на \PCFPOR са термовете на \PCFPP с едно ново правило:
  \[\tau ::= \dots\ |\ \por(\tau_1,\tau_2).\]
\item
  Имаме следните нови правила в операционната семантика:

  \begin{figure}[H]
    \begin{subfigure}[b]{0.5\textwidth}
      \begin{prooftree}
        \AxiomC{$\tau_1 \Downarrow_\bool \tru$}
        \UnaryInfC{$\por(\tau_1,\tau_2) \Downarrow_\bool \tru$}
      \end{prooftree}
    \end{subfigure}
    ~
    \begin{subfigure}[b]{0.5\textwidth}
      \begin{prooftree}
        \AxiomC{$\tau_2 \Downarrow_\bool \tru$}
        \UnaryInfC{$\por(\tau_1,\tau_2) \Downarrow_\bool \tru$}
      \end{prooftree}
    \end{subfigure}
  \end{figure}

  \begin{prooftree}
    \AxiomC{$\tau_1 \Downarrow_\bool \fls$}
    \AxiomC{$\tau_2 \Downarrow_\bool \fls$}
    \BinaryInfC{$\por(\tau_1,\tau_2) \Downarrow_\bool \fls$}
  \end{prooftree}
  
\item
  Денотационна семантика:
  \[\val{\por(\tau_1,\tau_2)}_\Gamma(\ov{u}) \df por(\val{\tau_1}_\Gamma(\ov{u}), \val{\tau_2}(\ov{u})).\]
\end{itemize}


\begin{problem}
  Нека $\tau$ е затворен терм на езика \PCFPOR. Докажете, че ако
  $\tau \Downarrow \vv{v}$ и $\tau \Downarrow \vv{v}'$, то $\vv{v} \equiv_\alpha \vv{v}'$.
\end{problem}


\index{Плоткин}
\begin{framed}
  \begin{theorem}[Плоткин 1977]
    Денотационната семантика $\val{.}$ за \PCFPP(\texttt{por}) е напълно абстрактна.
  \end{theorem}
\end{framed}
\marginpar{\cite[стр. 188]{gunter}}

% \index{Плоткин}
% \index{Милнър}
% \begin{theorem}[Милнър,Плоткин]
%   A continuous, order-extensional model of PCF is fully abstract if and only if for every type $\sigma$, $\val{\sigma}$ is a domain whose finite elements are definable.
% \end{theorem}


%%% Local Variables:
%%% mode: latex
%%% TeX-master: "../sep"
%%% End:
