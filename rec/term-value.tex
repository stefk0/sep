\subsection{Термални оператори}\label{subsect:rec:term-value}

% Нека първо да дефинираме следните {\em изображения}
% \begin{align*}
%   & \texttt{plus} : \Nat^2_\bot \to \Nat_\bot\text{, където}\\
%   & \texttt{plus}(a,b) \df
%     \begin{cases}
%       a+b, & \text{ако }a,b \in \Nat\\
%       \bot, & \text{ако }\bot \in \{a,b\}
%     \end{cases}\\
%   & \texttt{eq} : \Nat^2_\bot \to \Nat_\bot\text{, където}\\
%   & \texttt{eq}(a,b) \df
%     \begin{cases}
%       1, & \text{ако }a = b\ \&\ a,b \in \Nat\\
%       0, & \text{ако }a \neq b\ \&\ a,b \in \Nat\\
%       \bot, & \text{ако }\bot \in \{a,b\}
%     \end{cases}
% \end{align*}

% \marginpar{Спестяваме си труда от въвеждането на булевия тип променливи}
% \marginpar{Озн. $\Nat^+ \df \Nat \setminus \{0\}$}

% \begin{problem}\label{prob:rec:if:continuous}
%   Докажете, че изображенията $\texttt{plus}$ и $\texttt{eq}$ са непрекъснати.
% \end{problem}
% \begin{hint}
%   Понеже $\Mon{\Nat^n_\bot}{\Nat_\bot} = \Cont{\Nat^n_\bot}{\Nat_\bot}$,
%   достатъчно е да докажете, че изображенията са монотонни.
% \end{hint}

\index{денотационна семантика!по име}
\marginpar{\cite[стр. 155]{winskel}}
\index{терм!стойност}
\index{оператор!термален}
За всеки терм $\tau[\vv{x}_1,\dots,\vv{x}_{n},\vv{f}_1,\dots,\vv{f}_k]$,
ще разгледаме изображението със сигнатура
\[\val{\tau}:\Cont{\Nat^{m_1}_\bot}{\Nat_\bot}\times\cdots\times\Cont{\Nat^{m_k}_\bot}{\Nat_\bot} \to \Mapping{\Nat^{n}_\bot}{\Nat_\bot},\]
което ще дефинираме с индукция по построението на термовете.
Изображенията от вида $\val{\tau}$ ще наричаме {\bf термални оператори}.

\begin{itemize}
\item
  ако $\tau \equiv \vv{c}$, за някоя константа $\vv{c}$, то 
  \[\val{\vv{c}}(\ov{\varphi})(\ov{a}) \df c.\]
\item
  ако $\tau \equiv \vv{x}_i$, за някоя обектова променлива $\vv{x}_i$, то 
  \[\val{\vv{x}_i}(\ov{\varphi})(\ov{a}) \df a_i.\]
\item
  ако $\tau \equiv \tau_1 + \tau_2$, то
  \[\val{\tau_1 + \tau_2}(\ov{\varphi})(\ov{a}) \df \plus(\val{\tau_1}(\ov{\varphi})(\ov{a}), \val{\tau_2}(\ov{\varphi})(\ov{a})).\]
\item
  ако $\tau \equiv \tau_1 - \tau_2$, то
  \[\val{\tau_1 - \tau_2}(\ov{\varphi})(\ov{a}) \df \minus(\val{\tau_1}(\ov{\varphi})(\ov{a}), \val{\tau_2}(\ov{\varphi})(\ov{a})).\]

\item
  \marginpar{От \Problem{basic-operations:continuous} знаем, че изображенията $\plus$, $\minus$, $\eq$ са непрекъснати.}
  ако $\tau \equiv \tau_1\ \vv{==}\ \tau_2$, то
  \[\val{\tau_1\ \vv{==}\ \tau_2}(\ov{\varphi})(\ov{a}) \df \eq(\val{\tau_1}(\ov{\varphi})(\ov{a}), \val{\tau_2}(\ov{\varphi})(\ov{a})).\]

\item
  \marginpar{От \Problem{basic-operations:if:continuous} знаем, че $\texttt{if}$ е непрекъснато изображение.}
  ако $\tau \equiv\ \ifelse{\tau_1}{\tau_2}{\tau_3}$, то
  \[\val{\ifelse{\tau_1}{\tau_2}{\tau_3}}(\ov{\varphi})(\ov{a}) \df \texttt{if}_{\Nat_\bot}(\val{\tau_1}(\ov{\varphi})(\ov{a}), \val{\tau_2}(\ov{\varphi})(\ov{a}), \val{\tau_3}(\ov{\varphi})(\ov{a})).\]

\item
  ако $\tau \equiv \vv{f}_i(\tau_1,\dots,\tau_{m_i})$, то
  \marginpar{Термовете $\tau_j$ може да имат различен брой променливи. Ако се наложи, разширяваме ги с фиктивни променливи}
  \[\val{\vv{f}_i(\tau_1,\dots,\tau_{m_i})}(\ov{\varphi})(\ov{a}) \df \varphi_i(\val{\tau_1}(\ov{\varphi})(\ov{a}),\dots,\val{\tau_{m_i}}(\ov{\varphi})(\ov{a})).\]

\end{itemize}



% \begin{example}
%   Да разгледаме следния терм:
%   \[\tau[\vv{x},\vv{y},\vv{z}] \dfff \ifelse{\vv{x}\ \vv{==}\ 5}{\vv{y}}{\vv{z}}.\]
%   Да видим каква е негова стойност при произволни стойности $a,b,c \in \Nat_\bot$:
%   \begin{align*}
%     \val{\tau}(a,b,c) & =
%                         \begin{cases}
%                           \val{\vv{y}}(a,b,c), & \text{ако }\val{x \ \vv{==}\ 5}(a,b,c) \in \Nat^+\\
%                           \val{\vv{z}}(a,b,c), & \text{ако }\val{x \ \vv{==}\  5}(a,b,c) = 0\\
%                           \bot,                & \text{ако }\val{x\ \vv{==}\ 5}(a,b,c) = \bot
%                         \end{cases}\\
%                       & =
%                         \begin{cases}
%                           b,    & \text{ако }a = 5\\
%                           c,    & \text{ако }a \in \Nat\ \&\ a \neq 5\\
%                           \bot, & \text{ако }a = \bot.
%                         \end{cases}
%   \end{align*}
% \end{example}
% Нека сега да видим следния пример на хаскел.
% \begin{haskellcode}
% ghci> let f(x) = if x == undefined then 0 else 1
% ghci> f(2)
% *** Exception: Prelude.undefined
% \end{haskellcode}
% Това означава, че функцията на хаскел \eq е точна, т.е. не можем да сравняваме с $\bot$, което съответства на нашата
% денотационна семантика, т.е. функцията $\eq$.

\begin{framed}
  \begin{lemma}[Лема за замяната]
    \label{lem:rec:substitution}
    Да разгледаме терма $\tau[\vv{x}_1,\dots,\vv{x}_n,\vv{f}_1,\dots,\vv{f}_{k}]$ и {\em функционалните} термове 
    $\mu_1[\vv{f}_1,\dots,\vv{f}_k],\dots, \mu_n[\vv{f}_1,\dots,\vv{f}_k]$.
    Тогава
    \[\val{\tau[\bar{\vv{x}}/\bar{\mu}]}(\bar{\varphi}) = \val{\tau}(\ov{\varphi})(\val{\mu_1}(\bar{\varphi}),\dots,\val{\mu_n}(\bar{\varphi}))\]
    за произволни $\varphi_i \in \Cont{\Nat^{m_i}_\bot}{\Nat_\bot}$, за $i = 1, \dots, k$.
  \end{lemma}
\end{framed}
\begin{proof}
  Доказателството се провежда с {\em индукция по построението на терма $\tau$.}
  \marginpar{Понеже нямаме обектови променливи в $\mu_1,\dots,\mu_n$, то е очевидно как правим замяната.
    Тук следваме \cite[стр. 149]{winskel}. Това твърдение е без доказателство в \cite[стр. 188]{ditchev-soskov}.}
  \begin{itemize}
  \item
    \marginpar{$\vv{c}$ е константа, докато $c \in \Nat_\bot$}
    Нека да започнем с най-лесния случай.
    Нека $\tau \equiv \vv{c}$, за някоя константа.
    Тогава $\tau[\ov{\vv{x}}/\ov{\mu}] \equiv \vv{c}$.
    Това означава, че 
    \[\val{\tau[\ov{\vv{x}}/\ov{\mu}]}(\ov{\varphi}) = \val{\vv{c}}(\ov{\varphi}) \df c.\]
    От друга страна, имаме също, че 
    \[\val{\vv{c}}(\ov{\varphi})(\val{\mu_1}(\ov{\varphi}),\dots,\val{\mu_n}(\ov{\varphi})) \df c.\]
  \item
    Нека $\tau \equiv \vv{x}_i$. Тогава $\vv{x}_i[\varsx/\bar{\mu}] \equiv \mu_i$
    и следователно 
    \[\val{\vv{x}_i[\varsx/\bar{\mu}]}(\bar{\varphi}) = \val{\mu_i}(\bar{\varphi}).\]
    От друга страна, по дефиниция на стойност на терм, 
    \[\val{\vv{x}_i}(\ov{\varphi})(\val{\mu_1}(\bar{\varphi}),\dots,\val{\mu_n}(\bar{\varphi})) = \val{\mu_i}(\bar{\varphi}).\]
  \item
    Нека $\tau \equiv \tau_1 + \tau_2$.
    От \IndHyp имаме, че за $j = 1,2$ е изпъленено следното:
    \[\val{\tau_j[\varsx/\ov{\mu}]}(\ov{\varphi}) = \val{\tau_j}(\ov{\varphi})(\underbrace{\val{\mu_1}(\ov{\varphi})}_{b_1},\dots,\underbrace{\val{\mu_n}(\ov{\varphi})}_{b_n}).\]
    Тогава
    \begin{align*}
      \val{\tau[\varsx/\ov{\mu}]}(\ov{\varphi}) & = \val{\tau_1[\varsx/\ov{\mu}] + \tau_2[\varsx/\ov{\mu}]}(\ov{\varphi})\\
                                                & = \plus(\val{\tau_1[\varsx/\ov{\mu}]}(\ov{\varphi}), \val{\tau_2[\varsx/\ov{\mu}]}(\ov{\varphi})) & \comment\text{от деф.}\\
                                                & = \plus(\val{\tau_1}(\ov{\varphi})(b_1,\dots,b_n),\val{\tau_2}(\ov{\varphi})(b_1,\dots,b_n)) & \comment\text{от \IndHyp}\\
                                                & = \val{\tau}(\ov{\varphi})(b_1,\dots,b_n) & \comment\text{от деф.}\\
                                                & = \val{\tau}(\ov{\varphi})(\val{\mu_1}(\bar{\varphi}),\dots,\val{\mu_n}(\ov{\varphi})). & \comment{b_j \df \val{\mu_j}(\ov{\varphi})}
    \end{align*}
  \item
    \marginpar{\writedown Докажете сами останалите два случая. Те не крият изненади.}
    Нека $\tau \equiv \tau_1\ -\  \tau_2$.
  \item
    % \marginpar{\writedown Докажете сами останалите два случая. Те не крият изненади.}
    Нека $\tau \equiv \tau_1\ \vv{==}\  \tau_2$.
  \item
    Нека $\tau \equiv \ifelse{\tau_0}{\tau_1}{\tau_2}$.
  \item 
    Нека $\tau \equiv \vv{f}_i(\tau_1,\dots,\tau_{m_i})$.
    Имаме, че 
    \begin{equation}
      \label{eq:1}
      \tau[\varsx/\bar{\mu}] \equiv \vv{f}_i(\tau_1[\varsx/\bar{\mu}],\dots,\tau_{m_i}[\varsx/\bar{\mu}]).
    \end{equation}
    Нека за улеснение да означим $b_i \df \val{\mu_i}(\bar{\varphi})$, за $i = 1,\dots,n$.
    Прилагаме \IndHyp за термовете $\tau_1,\dots,\tau_{m_i}$ и получаваме за $j = 1, \dots, m_i$,
    \begin{align*}
      \val{\tau_j[\varsx /\bar{\mu}]}(\bar{\varphi}) & = \val{\tau_j}(\ov{\varphi})(\underbrace{\val{\mu_1}(\bar{\varphi})}_{b_1},\dots,\underbrace{\val{\mu_{n}}(\bar{\varphi})}_{b_{n}}) & \comment\text{от \IndHyp}\\
      & = \val{\tau_j}(\ov{\varphi})(b_1,\dots,b_{n}) & \comment{b_i \df \val{\mu_i}(\bar{\varphi})}.
    \end{align*}
    Следователно,
    \begin{align*}
      \val{\tau[\varsx/\ov{\mu}]}(\ov{\varphi}) & = \val{\vv{f}_i(\tau_1[\varsx/\ov{\mu}],\dots,\tau_{m_i}[\varsx/\ov{\mu}])}(\ov{\varphi}) & \comment{\text{от (\ref{eq:1})}}\\
                                                & = \varphi_i(\val{\tau_1[\varsx/\ov{\mu}]}(\ov{\varphi}),\dots,\val{\tau_{m_i}[\varsx/\ov{\mu}]}(\ov{\varphi})) & \comment\text{от деф.}\\
                                                & = \varphi_i(\val{\tau_1}(\ov{\varphi})(\ov{b}),\dots,\val{\tau_{m_i}}(\ov{\varphi})(\ov{b})) & \comment\text{от \IndHyp} \\
                                                & = \val{\vv{f}_i(\tau_1,\dots,\tau_{m_i})}(\ov{\varphi})(\ov{b}) & \comment\text{от деф.}\\
                                                & = \val{\tau}(\ov{\varphi})(\ov{b}) & \comment{\tau \equiv \vv{f}_i(\tau_1,\dots,\tau_{m_i})}\\
                                                & = \val{\tau}(\ov{\varphi})(\val{\mu_1}(\ov{\varphi}),\dots,\val{\mu_n}(\ov{\varphi})) & \comment{b_i \df \val{\mu_i}(\ov{\varphi})}.
    \end{align*}
  \end{itemize}
\end{proof}

% \begin{remark}
%   \marginpar{Тук $\ov{c}$ са естествени числа.}
%   В частния случай, когато функционалните термове $\mu_1,\dots, \mu_n$ са константите $\vv{c}_1, \dots, \vv{c}_n$, получаваме, че 
%   \[\val{\tau[\ov{\vv{x}}/\ov{\vv{c}}]}(\ov{\varphi}) = \val{\tau}(\ov{\varphi})(\ov{c}).\]  
% \end{remark}

%%% Local Variables:
%%% mode: latex
%%% TeX-master: "../sep"
%%% End:
