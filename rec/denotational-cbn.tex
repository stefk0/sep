\subsection{Предаване на параметрите по име}

Нека е дадена една рекурсивна програма $\vv{P}[\varsx,\varsf]$, където:
\marginpar{Можем да си мислим, че $\vv{f}_1$ е нещо като \texttt{main} функция за програмата $\vv{P}$.}
\begin{align*}
  \vv{P} = & 
             \begin{cases}
               & \vv{f}_1(\vv{x}_1,\dots,\vv{x}_{m_1}) = \tau_1[\vv{x}_1,\dots,\vv{x}_{m_1},\vv{f}_1,\dots,\vv{f}_k]\\
               & \vv{f}_2(\vv{x}_1,\dots,\vv{x}_{m_2}) = \tau_2[\vv{x}_1,\dots,\vv{x}_{m_2},\vv{f}_1,\dots,\vv{f}_k]\\
               & \vdots\\
               & \vv{f}_k(\vv{x}_1,\dots,\vv{x}_{m_k}) = \tau_k[\vv{x}_1,\dots,\vv{x}_{m_k},\vv{f}_1,\dots,\vv{f}_k]
             \end{cases}
\end{align*}
Нека $\ov{\gamma} \in \DomOpCBN$
е {\em най-малкото решение} на системата
\marginpar{В тази система неизвестните са $\varphi_1,\dots,\varphi_k$.}
\begin{align*}
  & \val{\tau_1}(\varphi_1,\dots,\varphi_k) = \varphi_1\\
  & \ \vdots \\
  & \val{\tau_k}(\varphi_1,\dots,\varphi_k) = \varphi_k.
\end{align*}
От \hyperref[th:knaster-tarski]{Теоремата на Клини} знаем, че такова най-малко решение съществува.

\index{денотационна семантика!по име}
\begin{framed}
  За дадената рекурсивна програма $\vv{P}[\varsx,\varsf]$, 
  определяме {\bf денотационната семантика с предаване на параметрите по име} 
  като изображението $\D_N\val{\vv{P}} \in \Cont{\Nat^{m_1}_\bot}{\Nat_\bot}$, където:
  \[\D_N\val{\vv{P}}(a_1,\dots,a_{m_1}) \df
    \begin{cases}
      \gamma_1(a_1,\dots,a_{m_1}), & \text{ако }\bot\not\in\{a_1,\dots,a_{m_1}\}\\
      \bot, & \text{ако }\bot\in\{a_1,\dots,a_{m_1}\}.
    \end{cases}\]
\end{framed}


%%% Local Variables:
%%% mode: latex
%%% TeX-master: "../sep"
%%% End:
