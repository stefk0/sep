\subsection{Теорема за еквивалентност}
\marginpar{Винаги с $\mu$ ще означаваме функционални термове; с $\tau$ - произволни термове}

\begin{proposition}\label{pr:rec:op-name-inclusion1}
  \marginpar{В \cite[стр. 192]{ditchev-soskov}, \cite[стр. 157]{winskel} доказателството е друго.}
  % \marginpar{Сравнете с \Prop{rec:op-value-inclusion1}}
  Да разгледаме една програма \vv{P}.
  Нека $\mu[\vv{f}_1,\dots,\vv{f}_n]$ е {\em функционален} терм. Тогава
  \[\eval{\mu} \sqsubseteq \val{\mu}(\bar{\gamma}),\]
  където $\bar{\gamma} = (\gamma_1,\dots,\gamma_n)$ е 
  \marginpar{Тук използваме наготово \Prop{product-continuous}.}
  някоя неподвижна точка на непрекъснатия оператор
  \[\Gamma \df \val{\tau_1}\times \cdots \times \val{\tau_n},\]
  който съответства на програмата \vv{P}.
\end{proposition}
\begin{proof}
  \marginpar{За това твърдение не е необходимо $\bar{\gamma}$ да бъде най-малката неподвижна точка на $\Gamma$, а просто решение.
    Само за другата посока ще е важно да е най-малкото решение.}
  Ако от терма $\mu$ {\em няма извод} до константа, то
  по дефиниция $\eval{\mu} = \bot$ и в този случай е очевидно, че
  \[\eval{\mu} \sqsubseteq \val{\mu}(\ov{\gamma}).\]
  
  Интересният случай е когато от терма $\mu$ {\em има извод} до константа.
  Тогава ще докажем, че за произволен функционален терм $\mu$ и константа $\vv{a}$,
  \begin{equation}
    \label{eq:16}
    \mu\Downarrow_P \vv{a}\ \implies \val{\mu}(\ov{\gamma}) = a.
  \end{equation}
  Доказателството на (\ref{eq:16}) ще проведем с индукция по дължината $\ell$ на извода $\mu\Downarrow^\ell_P a$.

  \marginpar{Константата $\vv{a}$ има стойност числото $a$.}
  Нека изводът има дължина $0$. Понеже $\mu$ е функционален терм, според правилата на операционната семантика, единствената възможност е $\mu \equiv \vv{a}$.
  Тогава е очевидно, че $\val{\mu}(\bar{\gamma}) = a$. Ясно е, че в този случай,
  \[\vv{a} \Downarrow^0_P \vv{a}\ \implies\ \val{\vv{a}}(\ov{\gamma}) = a.\]
  Нака имаме следното индукционно предположение:
  \begin{equation}
    \label{eq:ind-hyp}
    \mu\Downarrow^{<\ell}_P \vv{a}\ \implies \val{\mu}(\ov{\gamma}) = a.
  \end{equation}
  Ще докажем, че
  \[\mu\Downarrow^{\ell}_P \vv{a}\ \implies \val{\mu}(\ov{\gamma}) = a.\]
  Ще разгледаме всички случаи в зависимост от вида на функционалния терм $\mu$.
  \begin{itemize}
  \item
    Да разгледаме случая, когато $\mu \equiv \mu_1 + \mu_2$.
    Нека $\mu_1 + \mu_2 \Downarrow^\ell_P \vv{a}$.
    Според правилата за извод в операционната семантика по име, имаме следната ситуация:
    \begin{prooftree}
      \AxiomC{$\vdots$}
      % \LeftLabel{\scriptsize{(Извод с дълж. $\ell_1$)}}
      \UnaryInfC{$\mu_1 \Downarrow^{\ell_1}_P \vv{a}_1$}
      \AxiomC{$\vdots$}
      % \RightLabel{\scriptsize{(Извод с дълж. $\ell_2$)}}
      \UnaryInfC{$\mu_2 \Downarrow^{\ell_2}_P \vv{a}_2$}
      \RightLabel{\scriptsize{правило (plus)}}
      \BinaryInfC{$\mu_1 + \mu_2 \Downarrow^{\ell_1+\ell_2+1}_P \vv{a}$}
    \end{prooftree}

    където $a = \texttt{plus}(a_1,a_2)$.
    Ясно е, че изводите на $\mu_1\Downarrow^{<\ell}_P \vv{a}_1$ и $\mu_2 \Downarrow^{<\ell}_P \vv{a}_2$.
    Следователно можем да приложим {\bf И.П.} за $\mu_1$ и $\mu_2$, откъдето получаваме, че
    \begin{align*}
      & \mu_1 \Downarrow^{<\ell}_P \vv{a}_1\ \implies \val{\mu_1}(\ov{\gamma}) = \vv{a}_1\\
      & \mu_2 \Downarrow^{<\ell}_P \vv{a}_2\ \implies \val{\mu_2}(\ov{\gamma}) = \vv{a}_2.
    \end{align*}
    Тогава получаваме, че ако $\mu_1 + \mu_2 \Downarrow^\ell_P \vv{a}$, то
    \begin{align*}
      \val{\mu_1 + \mu_2}(\ov{\gamma}) & \df \texttt{plus}(\val{\mu_1}(\ov{\gamma}), \val{\mu_2}(\ov{\gamma}))\\
                                       & = \texttt{plus}(a_1,a_2)\\
                                       & = a.
    \end{align*}
  \item
    Случаят, когато $\mu \equiv \mu_1\ \vv{==}\ \mu_2$ е лесен.
  \item
    Случаят, когато $\mu \equiv \ifelse{\mu_0}{\mu_1}{\mu_2}$, е лесен.
  \item
    Нека имаме функционалния терм $\mu \equiv \vv{f}_i(\mu_1,\dots,\mu_{m_i})$ и $\mu \Downarrow^\ell_P \vv{a}$.
    Според правилата за извод в операционната семантика по име, имаме следната ситуация:
    \begin{prooftree}
      \AxiomC{$\vdots$}
      % \RightLabel{\scriptsize{Извод с дълж. $(\ell-1)$}}
      \UnaryInfC{$\tau_i[\vv{x}_1/\mu_1,\dots,\vv{x}_{m_i}/\mu_{m_i}] \Downarrow^{\ell-1}_P \vv{a}$}
      \RightLabel{\scriptsize{правило (cbn)}}
      \UnaryInfC{$\vv{f}_i(\mu_1,\dots,\mu_{m_i}) \Downarrow^{\ell}_P \vv{a}$}
    \end{prooftree}
    \marginpar{Индукционното предположение е цялата импликация. Според случая, който разглеждаме и според правилата на операционната семантика, лявата страна на импликацията е изпълнена. Следователно, и дясната страна на импликацията е изпълнена.}
    От {\bf И.П.} имаме, че  
    \[\tau_i[\varsx/\ov{\mu}] \Downarrow^{<\ell}_P \vv{a}\ \implies\ \val{\tau_i[\varsx/\ov{\mu}]}(\ov{\gamma}) = a.\]
    Понеже знаем, че $\tau_i[\varsx/\ov{\mu}] \Downarrow^{<\ell}_P \vv{a}$, то $\val{\tau_i[\varsx/\ov{\mu}]}(\ov{\gamma}) = a$.
    \marginpar{Озн. $\ov{\gamma} = (\gamma_1,\dots,\gamma_n)$. Единствено в този случай се използва, че $\ov{\gamma}$ е решение на системата от оператори, като не е задължително да е най-малкото решение.}
    Лесно се съобразява, че:
    \begin{align*}
      \val{\tau_i[\varsx/\ov{\mu}]}(\ov{\gamma}) & = \val{\tau_i}(\ov{\gamma})(\val{\mu_1}(\ov{\gamma}),\dots,\val{\mu_{m_i}}(\ov{\gamma})) & \comment{\text{\hyperref[lem:rec:substitution]{Лема за замяната}}}\\
                                                 & = \gamma_i(\val{\mu_1}(\ov{\gamma}),\dots, \val{\mu_{m_i}}(\ov{\gamma})) & \comment{\gamma_i = \val{\tau_i}(\ov{\gamma})}\\
                                                 & = \val{\vv{f}_i(\mu_1,\dots,\mu_{m_i})}(\ov{\gamma}). & \comment{\text{деф. на стойност на терм}}\\
    \end{align*}
    Обединявайки всичко, което знаем, получаваме:
    \begin{align*}
      \vv{f}_i(\mu_1,\dots,\mu_{m_i}) \Downarrow^\ell_P \vv{a} & \implies \tau_i[\varsx/\ov{\mu}] \Downarrow^{<\ell}_P \vv{a} & \comment{\text{правило (cbn)}}\\
                                                & \implies \val{\tau_i[\varsx/\ov{\mu}]}(\ov{\gamma}) = a & \comment{\text{{\bf И.П.}}}\\
                                                & \implies  \val{\vv{f}_i(\mu_1,\dots,\mu_{m_i})}(\ov{\gamma}) = a.
    \end{align*}
  \end{itemize}
  С това доказателството на (\ref{eq:16}) е завършено.
\end{proof}

\begin{framed}
  \begin{corollary}[Теорема за коректност]
    \label{cr:on-in-dn}
    За всяка рекурсивна програма $\vv{P}$ на езика {\bf REC} имаме, че 
    \[\O\val{\vv{P}} \sqsubseteq \D\val{\vv{P}}.\]
  \end{corollary}
\end{framed}
\begin{proof}
  Нека $\ov{\gamma}$ е най-малкото решение на системата от непрекъснати оператори, която съответства на програмата $\vv{P}$.
  Тогава за произволни $a_1,\dots,a_{m_1} \in \Nat$,
  \begin{align*}
    \O\val{\vv{P}}(a_1,\dots,a_{m_1}) & = \eval{\vv{f}_1(\vv{a}_1,\dots,\vv{a}_{m_1})} & \comment\text{от деф.}\\
                                      & \sqsubseteq \val{\vv{f}_1(\vv{a}_1,\dots,\vv{a}_{m_1})}(\ov{\gamma}) & \comment{\text{\Prop{rec:op-name-inclusion1}}} \\
                                      & = \gamma_1(\val{\vv{a}_1}(\ov{\gamma}),\dots,\val{\vv{a}_{m_1}}(\ov{\gamma})) & \comment{\text{стойност на терм}}\\
                                      & = \gamma_1(a_1,\dots,a_{m_1}) & \comment \val{\vv{a}_i}(\ov{\gamma}) = a_i\\
                                      & = \D\val{\vv{P}}(a_1,\dots,a_{m_1}). & \comment\text{от деф.}
  \end{align*}
\end{proof}

Така получихме едната посока на теоремата за еквивалентност.
Сега преминаваме към доказателството на другата посока.

Нека сега $\ov{\gamma} = (\gamma_1,\dots,\gamma_k)$ е най-малкото решение на системата от оператори, която съответства на програма \vv{P}
за денотационната семантика по име.
Да напомним, че това означава, че $\ov{\gamma} = \bigsqcup_r \ov{\gamma}_r$, 
където $\ov{\gamma}_0 = (\bot^{(m_1)},\dots,\bot^{(m_k)})$ и за всяко $r$,
\begin{align*}
  & \gamma^i_{r+1} = \val{\tau_i}(\ov{\gamma}_r)\\
  & \ov{\gamma}_r = ( \gamma^1_r, \dots, \gamma^k_r).
\end{align*}



\begin{proposition}
  \label{pr:op-name-inclusion2}
  \marginpar{\writedown На пръв поглед не е ясно защо трябва да докажем толкова сложно формулирано твърдение.Съобразете защо не можем да докажем по-простото твърдение, че за всеки функционален терм $\mu$ и всяко $r$,
  $\val{\mu}(\ov{\gamma}_r) \sqsubseteq \eval{\mu}$.}
  Да разгледаме една програма \vv{P} в езика $\REC$, като
  $\ov{\gamma} = \bigsqcup_r\ov{\gamma}_r$ е най-малкото решение на системата от оператори за \vv{P}.
  Тогава за произволен терм $\tau[\vv{x}_1,\dots,\vv{x}_n,\vv{f}_1,\dots,\vv{f}_k]$ и
  произволни функционални термове
  \[\mu_1[\vv{f}_1,\dots,\vv{f}_k],\dots,\mu_n[\vv{f}_1\dots,\vv{f}_k],\]
  и всяко $r$, е изпълнено, че:
  \[\val{\tau}(\ov{\gamma}_r)(\eval{\mu_1},\dots,\eval{\mu_n}) \sqsubseteq \eval{\tau[\varsx/\ov{\mu}]}.\]
\end{proposition}
\begin{proof}
  За произволно естествено число $r$, нека твърдението $\texttt{Include}(r)$ да гласи следното: ,,за произволен терм $\tau[\vv{x}_1,\dots,\vv{x}_n,\vv{f}_1,\dots,\vv{f}_k]$ и
  произволни функционални термове $\mu_1[\vv{f}_1,\dots,\vv{f}_k],\dots,\mu_n[\vv{f}_1\dots,\vv{f}_k]$,
  е изпълнено, че:
  \[\val{\tau}(\ov{\gamma}_r)(\eval{\mu_1},\dots,\eval{\mu_n}) \sqsubseteq \eval{\tau[\varsx/\ov{\mu}]}.\text{''}\]
  Ще докажем с индукция по $r$, че $\texttt{Include}(r)$ е изпълнено за всяко $r$.

  \begin{itemize}
  \item 
    Първо ще докажем $\texttt{Include}(0)$.
    Това ще направим с индукция по построението на терма $\tau$.
    Да разгледаме произволни функционални термове $\mu_i$ и за улеснение да положим $a_i \df \eval{\mu_i}$.
    \begin{itemize}
    \item
      Нека $\tau \equiv \vv{c}$. Тогава:
      \begin{align*}
        \val{\vv{c}}(\ov{\gamma}_0)(\ov{a}) & = c & \comment\text{стойност на терма}\\
                                            & = \eval{\vv{c}} & \comment{\text{правило (1)}}\\
                                            & = \eval{\vv{c}[\varsx/\ov{\mu}]}.
      \end{align*}
    \item
      Нека $\tau \equiv \vv{x}_i$. Тогава:
      \begin{align*}
        \val{\vv{x}_i}(\ov{\gamma}_0)(\ov{a}) & = a_i & \comment\text{стойност на терма}\\
                                              & = \eval{\mu_i} & \comment{a_i \df \eval{\mu_i}}\\
                                              & = \eval{\vv{x}_i[\varsx/\ov{\mu}]}.                    
      \end{align*}
    \item
      Нека $\tau \equiv \tau_1 + \tau_2$.
      В този случай, $\tau$ е съставен от по-простите термове $\tau_1$ и $\tau_2$.
      От {\bf И.П.} за $\tau_1$ и $\tau_2$ следва, че за $i = 1,2$ е изпълнено следното:
      \begin{equation}
        \label{eq:15}
        \val{\tau_i}(\ov{\gamma}_0)(\ov{a}) \sqsubseteq \eval{\tau_i[\varsx/\ov{\mu}]}.
      \end{equation}
      Да напомним, че изображението $\texttt{plus}$ е непрекъснато, откъдето следва, че също така е монотонно. 
      Тогава:
      \begin{align*}
        \val{\tau_1 + \tau_2}(\ov{\gamma}_0)(\ov{a}) & = \texttt{plus}(\val{\tau_1}(\ov{\gamma}_0)(\ov{a}), \val{\tau_2}(\ov{\gamma}_0)(\ov{a})) & \comment\text{стойност на терма}\\
                                                    & \sqsubseteq \texttt{plus}(\eval{\tau_1[\varsx/\ov{\mu}]}, \eval{\tau_2[\varsx/\ov{\mu}]}) & \comment{\text{от (\ref{eq:15})}}\\
                                                    & = \eval{\tau[\varsx/\ov{\mu}]} & \comment{\text{правило }(\texttt{plus})}
      \end{align*}
    \item
      \marginpar{\writedown Тези два случая са за домашно!}
      Нека $\tau \equiv \tau_1\ \vv{==}\ \tau_2$.
    \item
      Нека $\tau \equiv \ifelse{\tau_0}{\tau_1}{\tau_2}$.
    \item
      Нека $\tau \equiv \vv{f}_i(\rho_1,\dots,\rho_{m_i})$. Тогава
      \begin{align*}
        \val{\tau}(\ov{\gamma}_0)(\ov{a}) & = \gamma^i_0(\val{\rho_1}(\ov{\gamma}_0)(\ov{a}), \dots,\val{\rho_{m_i}}(\ov{\gamma}_0)(\ov{a})) & \comment\text{стойност на терма}\\
                                          & = \bot & \comment{\gamma^i_0(\ov{x}) \df \bot}\\
                                          & \sqsubseteq \eval{\tau[\varsx/\ov{\mu}]}.
      \end{align*}
      Така доказахме, че $\texttt{Include}(0)$ е изпълнено.
    \end{itemize}
  \item
    Нека $r > 0$. Да приемем, че $\texttt{Include}(r-1)$ е изпълнено. Ще докажем $\texttt{Include}(r)$
    отново с индукция по построението на термовете.
    \marginpar{\writedown Разгледайте сами останалите случаи за $\tau$ и се убедете, че те наистина се доказват по същия начин както при $r = 0$.} 
    Единственият случай, който заслужава внимание е 
    \[\tau \equiv \vv{f}_i(\rho_1,\dots,\rho_{m_i}).\]
    Доказателствата на всички останали случаи за $\tau$ протичат по абсолютно същия начин както при $r = 0$.

    Понеже термът $\tau$ е построен с помощта на термовете $\rho_j$, за $j = 1, \dots, m_i$,
    можем да приложим {\bf И.П.} за тях и да получим, че 
    \begin{align*}
      \val{\rho_{j}}(\ov{\gamma}_r)(\underbrace{\eval{\mu_1}}_{a_1},\dots,\underbrace{\eval{\mu_n}}_{a_n}) & = \underbrace{\val{\rho_{j}}(\ov{\gamma}_r)(a_1,\dots, a_n)}_{b_j} \\
                                                                                                             & \sqsubseteq \underbrace{\eval{\rho_j[\varsx/\ov{\mu}]}}_{c_j}. & \comment{\text{от {\bf И.П.}}}
    \end{align*}

    \marginpar{Обърнете внимание, че $\rho'_j$ са функционални термове}
    Нека за наше улеснение да положим $\rho'_j \df \rho[\varsx/\ov{\mu}]$.
    Това означава, че до момента имаме следното:
    \[\val{\tau_i}(\ov{\varphi})(\underbrace{\val{\rho_1}(\ov{\gamma}_r)(\ov{a})}_{b_1}, \dots, \underbrace{\val{\rho_{m_i}}(\ov{\gamma}_r)(\ov{a})}_{b_{m_i}}) \sqsubseteq  \val{\tau_i}(\ov{\varphi})(\underbrace{\eval{\rho'_1}}_{c_1},\dots, \underbrace{\eval{\rho'_{m_i}}}_{c_{m_i}}),\]
    за произволни непрекъснати изображения $\ov{\varphi}$.
    
    Като обединим всичко от по-горе, получаваме следното:
    \begin{align*}
      \val{\tau}(\ov{\gamma}_r)(\ov{a}) & = \val{\vv{f}_i(\rho_1,\dots,\rho_{m_i})}(\ov{\gamma}_r)(\ov{a})\\
                                        & = \gamma^i_r(\underbrace{\val{\rho_1}(\ov{\gamma}_r)(\ov{a})}_{b_1}, \dots, \underbrace{\val{\rho_{m_i}}(\ov{\gamma}_r)(\ov{a})}_{b_{m_i}}) & \comment\text{стойност на терма}\\
                                        & = \gamma^i_r(b_1, \dots, b_{m_i})\\
                                        & \sqsubseteq \gamma^i_r(c_1, \dots, c_{m_i}) & \comment{\gamma^i_r\text{ е непр. и следователно мон.}}\\
                                        & = \val{\tau_i}(\ov{\gamma}_{r-1})(c_1,\dots,c_{m_i}) & \comment{\gamma^i_r \df \val{\tau_i}(\ov{\gamma}_{r-1})}\\
                                        & = \val{\tau_i}(\ov{\gamma}_{r-1})(\underbrace{\eval{\rho'_1}}_{c_1},\dots,\underbrace{\eval{\rho'_{m_i}}}_{c_{m_i}}) & \comment{c_i = \eval{\rho'_i}} \\
                                        & \sqsubseteq \eval{\tau_i[\vv{x}_1/\rho'_1,\dots,\vv{x}_{m_i}/\rho'_{m_i}]} & \comment{\text{от }\texttt{Include}(r-1)}\\
                                        & = \eval{\vv{f}_i(\rho'_1,\dots,\rho'_{m_i})} & \comment{\text{от правило (4)}}\\
                                        & = \eval{\vv{f}_i(\rho_1[\varsx/\ov{\mu}],\dots,\rho_{m_i}[\varsx/\ov{\mu}])} & \comment{\rho'_j \df \rho[\varsx/\ov{\mu}]}\\
                                        & = \eval{\vv{f}_i(\rho_1,\dots,\rho_{m_i})[\varsx/\ov{\mu}]} & \comment{\text{правила за замяна}}\\
                                        & = \eval{\tau[\varsx/\ov{\mu}]}.
    \end{align*}
    Заключаваме, че
    \[\val{\tau}(\ov{\gamma}_r)(\underbrace{\eval{\mu_1}}_{a_1},\dots,\underbrace{\eval{\mu_n}}_{a_n}) \sqsubseteq  \eval{\tau[\varsx/\ov{\mu}]}.\]
    Така доказахме $\texttt{Include}(r)$.
  \end{itemize}
  Най-накрая заключаваме, че $(\forall r)\texttt{Include}(r)$.
\end{proof}

\begin{corollary}\label{cr:rec:equivalence-cbn:inclusion2}
  Нека $\mu$ е функционален терм.
  Тогава $\eval{\mu}$ е горна граница на веригата $(\val{\mu}(\ov{\gamma}_r))^{\infty}_{r=0}$.
\end{corollary}

\begin{lemma}
  \marginpar{$f\times g$ е дефинирано в \Prop{cartesian-continuous}.}
  За всяка рекурсивна програма $\vv{P}$,
  произволен {\em функционален} терм $\mu$,
  \[\val{\mu}(\ov{\gamma}) \sqsubseteq \eval{\mu},\]
  където $\ov{\gamma} = \lfp(\Gamma)$, а $\Gamma = \val{\tau_1} \times \val{\tau_2} \times \cdots \times \val{\tau_k}$ е операторът, който съответства на програмата $\vv{P}$.
\end{lemma}
\begin{hint}
  
  \begin{align*}
    \val{\mu}(\ov{\gamma}) & = \val{\mu}(\bigsqcup_r\ov{\gamma}_r) & \comment{\ov{\gamma} = \bigsqcup_r \ov{\gamma}_r}\\
                           & = \bigsqcup_r \val{\mu}(\ov{\gamma}_r) & \comment{\text{от \Lem{rec:functional:term:continuous}}}\\
                           & \sqsubseteq \eval{\mu}. & \comment{\text{от \Cor{rec:equivalence-cbn:inclusion2}}}
  \end{align*}
\end{hint}

\begin{framed}
  \begin{theorem}
    За всяка рекурсивна програма $\vv{P}$ на езика {\bf REC} е изпълнено:
    \[\O\val{\vv{P}} = \D\val{\vv{P}}.\]
  \end{theorem}
\end{framed}
\begin{proof}
  \marginpar{Озн. $\ov{\gamma} = (\gamma_1,\dots,\gamma_n)$ е най-малкото решение на системата от оператори съответстващи на програмата \vv{P} }
  Ние вече знаем от \Cor{on-in-dn}, че 
  \[\O\val{\vv{P}} \sqsubseteq \D\val{\vv{P}}.\]
  Остава да докажем обратната посока, а именно 
  \[\D\val{\vv{P}} \sqsubseteq \O\val{\vv{P}}.\]
  За произволни числа $\ov{a}$, имаме следните връзки:
  \begin{align*}
    \D\val{\vv{P}}(\ov{a}) & = \gamma_1(\ov{a}) & \comment\text{деф. на денот. сем.}\\
                           & = \val{\tau_1}(\ov{\gamma})(\ov{a}) & \comment{\gamma_1 \df \val{\tau_1}(\ov{\gamma})}\\
                           & = \val{\tau_1[\varsx/\ov{\vv{a}}]}(\ov{\gamma}) & \comment{\text{\hyperref[lem:rec:substitution]{Лема за замяната}}}\\
                           & \sqsubseteq \eval{\tau_1[\varsx/\ov{\vv{a}}]} & \comment{\text{\Prop{op-name-inclusion2}}}\\
                           & = \eval{\vv{f}_1(\vv{a}_1,\dots,\vv{a}_{m_1})} & \comment{\text{правило (cbn) на опер. сем.}}\\
                           & = \O\val{\vv{P}}(\ov{a}). & \comment\text{деф. на опер. сем.}
  \end{align*}
\end{proof}


%%% Local Variables:
%%% mode: latex
%%% TeX-master: "../sep"
%%% End:
